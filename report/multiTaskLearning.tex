\chapter{Multi-Task Learning}
\label{multi-task_learning}
In this section we introduce an extension to the supervised machine learning framework called multi-task learning. We first cover the main ideas and motivation for multi-task learning. We then summarize different variations on statistical learning theory to gain an intuition of how and when multi-task learning works. Finally, we describe how to implement multi-task learning with neural networks.

\section{Multi-Task and Single-Task Learning}
\label{multiTaskAndSingleTaskLearning}

In our description of supervised machine learning so far we have assumed that the input for the learning system was an annotated dataset $\mathcal{D}$ in which all samples $(\vector{x}_i, \vector{y}_i) \in \mathcal{D}$ are drawn independently from the same distribution $P(\vector{x}, \vector{y})$. In the real world however, it's often possible to combine data from disparate sources if we relax this assumption: We may have access to a set $\data_M$ of $M$ datasets $\data_m \in \data_M$, drawn from a set $\mathcal{P}$ of $M$ different distributions $P_m(\vector{x}, \vector{y}) \in \mathcal{P}$. Since creating new labels for a machine learning task is often both cumbersome and costly, it would be desirable if re-using previously labeled data could reduce the need for data annotation
\\\\
In many cases, we are not interested in implementing a learning system that performs well on all $M$ learning tasks. We really only care about one \textbf{target} task defined by a distribution $P_t \in \mathcal{P}$ and $\data_t \in \data_M$ in which case we consider the other datasets $\data_A = \{\data_m \mid \data_m \in \data_M, m \neq t\}$ to be \textbf{auxiliary}. Since we are dealing with more than one probability distribution, it becomes useful to think of generalization error with respect to a particular distribution. Thus, we extend our notation for generalization error from $E$ to $E_m$ to mean:
$$
E_m(h) = \mathbb{E}_{(\vector{x},\vector{y}) \sim P_m(\vector{x},\vector{y})}[e(h(\vector{x}),\vector{y})]
$$
\noindent
We can speculate that if $\mathcal{D}_t$ and $\data_A$ are related somehow, and if the learning system is able to share what is learnt between the learning tasks, learning the tasks simultaneously may improve generalization for the target task relative to learning from $\data_t$ in isolation \citep{caruana1997}. To distinguish the two approaches, the traditional approach to supervised machine learning as described in section \ref{supervised_machine_learning} is called \textbf{single-task learning}, and the new approach in which the learning system uses all of $\data_M$ is called \textbf{multi-task learning}.
\\\\
In the following sections we introduce contributions from statistical learning theory that shed some light on when and how learning from $\data_M$ is beneficial.





\section{Bias Learning}
\label{bias_learning}
Selecting the hypothesis space $\mathcal{H}$, sometimes referred to as \textbf{biasing} the hypothesis space, is often the hardest problem in supervised machine learning \citep{baxter2000}. Vapnik-Chervonenkis analysis tells us that $\mathcal{H}$ must be large enough to contain a good solution to the learning problem of interest, yet small that the selected model can generalize from a small sample. This motivates developing techniques that can learn a good $\hypspace$ from the data.
\\\\
\citet{baxter2000} formalizes this idea by introducing a model of \textbf{bias learning}, in which the learning system is tasked with learning a hypothesis space $\mathcal{H}$ from a family of hypothesis spaces $\mathbb{H} = \{\mathcal{H}\}$. The system is supplied with $M$ datasets $\data_m$ each drawn from $M$ distributions $P_m$ over $\mathcal{X} \times \mathcal{Y}$. The goal of the system is then to first select a good hypothesis space $\mathcal{H} \in \mathbb{H}$, and then to select a vector $\vector{h}$ of $M$ hypothesis $h_m \in \mathcal{H}$. In his framework, the goal of the learning system is to minimize the multi-task generalization error defined as the average generalization error over the $M$ learning problems:
$$
E(\vector{h}) = \frac{1}{M}\sum\limits_{m = 1}^M E_m(h_m)
$$
Similarly, we can generalize the empirical single-task error to an average multi-task empirical error $\hat{E}(\vector{h}, \data_M)$: 
$$
\hat{E}(\vector{h}, \data_M) = \frac{1}{M}\sum\limits_{m = 1}^M\hat{E}(h_m, \data_m)
$$
The bias learning model of \citet{baxter2000} extends Vapnik-Chervonenkis analysis to the multi-task learning problem. To this end, he defines $\mathcal{H}(N, M)$ to be the set of all matrices of dichotomies, that can be formed from selecting $M$ hypothesis from $\mathcal{H}$ and applying them to the to the $N$ samples of the $M$ datasets in $\data_M$:
$$
\mathcal{H}(N, M) = \left\{ \begin{bmatrix}
	h_1(\vector{x}_{11}) & \cdots & h_1(\vector{x}_{1N}) \\
	\vdots & \ddots & \vdots \\
	h_M(\vector{x}_{M1}) & \cdots & h_M(\vector{x}_{MN})
\end{bmatrix}\, :\, h_1,\, \dots,\, h_M \in \mathcal{H} \right\}
$$

This allows him to define a concept of dichotomies on multi-task samples $\data_M$ for hypothesis space families, $\mathbb{H}(N, M)$:
$$
\mathbb{H}(N, M) = \bigcup\limits_{\mathcal{H} \in \mathbb{H}} \mathcal{H}(N, M)
$$
And extend the growth function $m$ to the multi-task setting:
$$
m(N, M, \mathbb{H}) = \max|\mathbb{H}(N, M)|
$$
With a binary label space, the maximum size of $\mathbb{H}(N, M)$ is $2^{NM}$. Baxter uses this to define the Vapnik-Chervonenkis dimension $d(M, \mathbb{H})$ of the hypothesis space family $\mathbb{H}$:
$$
d(M, \mathbb{H}) = \max\{N\, : \, m(N, M, \mathbb{H}) = 2^{NM}\}
$$
In words, the Vapnik-Chervonenkis dimension of the hypothesis space family $\mathbb{H}$, is the largest number of samples $N$ for which the family can generate all possible binary dichotomy matrices, when learning from $M$ datasets of size $N$.
\\\\
Using the same reasoning as is the basis of the original Vapnik-Chervonenkis bound, Baxter is able to show that in order for the average true error $E(\vector{h})$, to be within $\epsilon$ of the average empirical error $\hat{E}(\vector{h}, \data_M)$ with probability $1 - \delta$, it requires that the number of samples $N$ for each task is:
$$
N = O\left(\frac{1}{\epsilon^2}\left(d(M, \mathbb{H}) \log \frac{1}{\epsilon} + \frac{1}{M} \log \frac{1}{\delta}\right)\right)
$$
Ignoring the confidence parameters $\epsilon$ and $\delta$, we see that the number of examples $N$ depends inversely on the number of tasks $M$. This means that we can reduce the number of samples required to keep $E$ close to $\hat{E}$, if we can increase the number of learning tasks. This is an important result since it shows that multi-task bias learning can improve our confidence that $E_m$ is close to $\hat{E}(h_m, \data_m)$ at least on average. 

On the other hand, it's also a limited result in the sense that it doesn't tells anything about how $\hat{E}(h_m, \data_m)$ behaves in multi-task learning relative to single-task learning. In other words, it may be possible that bias learning leads to a hypothesis space $\mathcal{H}$ where $\hat{E}(\vector{h}, \data_M)$ is close to $E(\vector{h})$, but every $\hat{E}(h_m, \data_m)$ is much larger than would have been possible to achieve if the tasks had been learned separately.
\section{Representation Learning}
\label{representation_learning}
\subsection{Auto Encoders}
\subsection{Word Embeddings}
\section{Task Relatedness}
Our presentation of multi-task learning so far has been limited to the statistical properties of learning multiple tasks simultaneously without consideration as to how the tasks are related to one another. Intuitively, we expect that learning related tasks should yield better results than learning unrelated tasks.
\\\\
\citet{ben2003} attempts to quantify this intuition by extending the work of \citet{baxter2000} with a notion of task "relatedness". They focus on modeling similarity between the $M$ distributions $P_1$ to $P_M$ from which the $M$ datasets $D_m \in \data_M$ are drawn.

Specifically, they consider two learning tasks, defined by the probability distributions $P_1$ and $P_2$ on the input space $\mathcal{X}$, to be related if $P_1$ and $P_2$ are identical up to a transformation $f: \mathcal{X} \to \mathcal{X}$. To formalize this, they define a set of such transformations $\mathcal{F}$, and say that two learning tasks are $\mathcal{F}$-related if for some fixed distribution the data in each of these tasks are generated by applying some $f \in \mathcal{F}$ to that distribution.
\\\\
Formally, let $\mathcal{F}$ be a set of transformations $f: \mathcal{X} \to \mathcal{X}$, and $P_1$, $P_2$ be probability distributions over $\mathcal{X} \times \mathcal{Y}$ where $\mathcal{Y} = \{0,1\}$. $P_1$ and $P_2$ are $\mathcal{F}$-related if there exists some $f \in \mathcal{F}$ such that for any $T \subseteq \mathcal{X} \times \mathcal{Y}$, $T$ is $P_1$-measurable iff $f[T] = \{(f(\vector{x}), \vector{y}) \mid (\vector{x}, \vector{y}) \in T\}$ is $P_2$-measurable and $P_1(T) = P_2(f[T])$. Two samples are $\mathcal{F}$-related if they are sampled from $\mathcal{F}$-related distributions \citep{ben2003}.
\\\\
In the framework of \citet{ben2003} they assume that the learning system knows the set $\mathcal{F}$ but doesn't know which function $f \in \mathcal{F}$ relates the distributions the system is learning from. Therefore, the ease with which the learner can transfer information about the underlying distributions from one learning task to another depends on the size of $\mathcal{F}$. The larger this set is, the looser the notion of relatedness between the learning tasks.
\\\\
In order to let the learning system take advantage of the multiple datasets $\data_M$, \citet{ben2003} uses their notion of task relatedness to reduce the complexity of the hypothesis space $\mathcal{H}$ by first using all the data $\data_M$ to select a subspace of $\mathcal{H}$ which is likely to contain good solutions to the set of learning problems. After this initial biasing of $\hypspace$, $M$ functions $h_m$ are selected from this subspace for each learning problem. Specifically, from the hypothesis space $\mathcal{H}$, create a family of hypothesis spaces $H$ of sets of hypotheses $h \in \mathcal{H}$ that are equivalent up to transformations in $\mathcal{F}$, assuming that for each $f \in \mathcal{F}$ and $h \in \mathcal{H}$, we have $h \circ f \in \mathcal{H}$. 
\\\\
To formalize this, \citet{ben2003} define an equivalence relation $\sim \mathcal{F}$ on $\mathcal{H}$. This means that $h_1$ and $h_2$ are equivalent if there exists $f\ \in \mathcal{F}$ such that $h_2 = h_1 \circ f$. They use the notation $[h]_{\sim \mathcal{F}}$ to mean the equivalence class of $h$ under $\mathcal{F}$. 

\citet{ben2003} uses the notion of equivalence classes to partition $\mathcal{H}$ into the family $H$ of equivalence classes of $\mathcal{H}$ under $\mathcal{F}$, i.e $H = \mathcal{H} \setminus \sim \mathcal{F}$
\\\\
Note that if two learning tasks defined by the distributions $P_1$ and $P_2$ are $\mathcal{F}$-related, then there exists $f \in \mathcal{F}$ such that the generalization errors of any function $h \in \mathcal{H}$ on both tasks are equal. In other words, there exists $f \in \mathcal{F}$ such that:
$$
E_1(h) = E_2(h \circ f)
$$
This means that the equivalence classes of $\mathcal{H}$ perform equally well on the different tasks, when measured by:
$$
E_m(H) = \inf\limits_{h \in H}E(h)
$$

\citet{ben2003} uses this fact of equivalence classes to build on \citet{baxter2000} and shows that if the number of examples $N$ in each learning task satisfy:
$$
N \geq O\left(\frac{1}{\epsilon^2}\left(d(M, H) \log \frac{1}{\epsilon} + \frac{1}{M} \log \frac{1}{\delta}\right)\right)
$$
Then, with probability $1 - \delta$, for any $1 \leq i \leq M$:
$$
\left| E_i([h]_{\sim \mathcal{F}}) - \inf\limits_{h_1,\dots,h_M \in [h]_{\sim \mathcal{F}}} \frac{1}{M}\sum\limits_{m = 1}^M \hat{E}(h_m, \data_m)\right|  \leq \epsilon
$$
The main difference between this result and the one obtained by \citet{baxter2000} is that \citet{ben2003} bounds the distance between $E_m([h]_{\sim \mathcal{F}})$, i.e the generalization error of the equivalent functions $[h]_{\sim \mathcal{F}}$ for \emph{any} task $m$, and the functions that minimizes the training errors for each $\data_m$, whereas \citet{baxter2000} bounds the distance between the \emph{average} generalization error and training error.
\\\\
This is an important result because it gives credence to our intuition that learning related tasks improves the guarantees that can be made on the distance between training and generalization error over learning unrelated tasks. However, just as the bound provided by \citet{baxter2000}, this bound does not reveal anything about how learning from $\data_M$ might improve $\hat{E}(h_m, \data_m)$. Moreover, the range of domains where tasks are $\mathcal{F}$-related are limited. Specifically, the notion of $\mathcal{F}$-relatedness is limited to domains where two tasks are essentially two different views of the same data, for example video footage of the same scenery from two different perspectives. This may not be an appropriate assumption for natural language processing tasks.
\section{Deep Multi-Task Learning}
\label{deep_multi-task_learning}
Neural networks have the advantage of being easy to adapt from single-task learning to multi-task learning. The simplest way of turning two single task learning problems into a multi-task learning problem using neural networks is by hard weight sharing of a subset of the weights of the networks for the learning tasks and learning them simultaneously \citep{caruana1997}. As an example, consider figure \ref{no_weight_sharing} and \ref{weight_sharing}.
\\\\
\begin{figure}[h]
	\centering
	\begin{tikzpicture}[->, >=stealth, swap]
			\node [neuron] (x11) at (0,0) {};
			\node [neuron] (x12) at (1,0) {};
			\node [neuron] (x13) at (2,0) {};
			\node [neuron] (x14) at (3,0) {};
			\node [neuron] (x15) at (4,0) {};
			\node [neuron] (sigma11) at (1,1.5) {};
			\node [neuron] (sigma12) at (2,1.5) {};
			\node [neuron] (sigma13) at (3,1.5) {};
			\node [neuron] (out1) at (2,3) {};	
			\node [] (label1) at (2,4) {Task 1};	
			\draw (x11) edge (sigma11);
			\draw (x12) edge (sigma11);
			\draw (x13) edge (sigma11);
			
			\draw (x12) edge (sigma12);
			\draw (x13) edge (sigma12);
			\draw (x14) edge (sigma12);
			
			\draw (x13) edge (sigma13);
			\draw (x14) edge (sigma13);
			\draw (x15) edge (sigma13);
			
			\draw (sigma11) edge (out1);
			\draw (sigma12) edge (out1);
			\draw (sigma13) edge (out1);
			
			\node [neuron] (x21) at (6,0) {};
			\node [neuron] (x22) at (7,0) {};
			\node [neuron] (x23) at (8,0) {};
			\node [neuron] (x24) at (9,0) {};
			\node [neuron] (x25) at (10,0) {};
			\node [neuron] (sigma21) at (7,1.5) {};
			\node [neuron] (sigma22) at (8,1.5) {};
			\node [neuron] (sigma23) at (9,1.5) {};
			\node [neuron] (out2) at (8,3) {};	
			\node [] (label2) at (8,4) {Task 2};	
			\draw (x21) edge (sigma21);
			\draw (x22) edge (sigma21);
			\draw (x23) edge (sigma21);
			
			\draw (x22) edge (sigma22);
			\draw (x23) edge (sigma22);
			\draw (x24) edge (sigma22);
			
			\draw (x23) edge (sigma23);
			\draw (x24) edge (sigma23);
			\draw (x25) edge (sigma23);
			
			\draw (sigma21) edge (out2);
			\draw (sigma22) edge (out2);
			\draw (sigma23) edge (out2);
	\end{tikzpicture}
	\caption{Visual representation of single-task learning with neural networks. A set of neural network weights are learnt separately for Task 1 and Task 2.}
	\label{no_weight_sharing}
\end{figure}

\begin{figure}[h]
	\centering
	\begin{tikzpicture}[->, >=stealth, swap]
			\node [neuron] (x1) at (0,0) {};
			\node [neuron] (x2) at (1,0) {};
			\node [neuron] (x3) at (2,0) {};
			\node [neuron] (x4) at (3,0) {};
			\node [neuron] (x5) at (4,0) {};
			\node [neuron] (sigma1) at (1,1.5) {};
			\node [neuron] (sigma2) at (2,1.5) {};
			\node [neuron] (sigma3) at (3,1.5) {};
			\node [neuron] (out1) at (1.3,3) {};	
			\node [neuron] (out2) at (2.7,3) {};
			\node [] (label1) at (1.3,4) {Task 1};
			\node [] (label2) at (2.7,4) {Task 2};
			\draw (x1) edge (sigma1);
			\draw (x2) edge (sigma1);
			\draw (x3) edge (sigma1);
			
			\draw (x2) edge (sigma2);
			\draw (x3) edge (sigma2);
			\draw (x4) edge (sigma2);
			
			\draw (x3) edge (sigma3);
			\draw (x4) edge (sigma3);
			\draw (x5) edge (sigma3);
			
			\draw (sigma1) edge (out1);
			\draw (sigma2) edge (out1);
			\draw (sigma3) edge (out1);
			
			\draw (sigma1) edge (out2);
			\draw (sigma2) edge (out2);
			\draw (sigma3) edge (out2);
	\end{tikzpicture}
	\caption{Visual representation multi-task learning with neural networks. The weights of the hidden layer is shared between the two tasks.}
	\label{weight_sharing}
\end{figure}
\noindent
Multi-task learning techniques that are based on sharing neural network weights between tasks are collectively known as \textbf{deep multi-task learning} techniques. Deep multi-task learning is closely associated with the idea of representation learning presented in section \ref{representation_learning} and the more general framework of bias learning presented in section \ref{bias_learning}. The network represents a shared representation $f(\vector{x})$ represented by $S$ shared layers, often the first layers, and hypotheses $h_1 = (g_1 \circ f)(\vector{x})$ to $h_M = (g_M \circ f)(\vector{x})$ for each learning task, where $g_m$ is $L - S$ neural network layers specific to each task.
\\\\
The exact circumstances under which deep multi-task learning leads to lower overall generalization error compared to deep single-task learning are not yet theoretically well understood. \citet{caruana1997} lists 3 suggestions for how multi-task learning can reduce generalization error:
\begin{description}
	\item [Statistical Data Amplification] The effective number of training examples available to a deep multi-task learning system is increased due to the examples in the auxiliary data. The extensions to the Vapnik-Chervonenkis bound seen in the preceding sections gives us confidence that this reduces the risk that generalization error is far away from training error.
	\item [Eavesdropping] If a hidden layer feature is useful to both Task 1 and Task 2, but much easier to learn when learning Task 2, sharing the hidden layer between the two tasks is likely to reduce generalization error for Task 1.
	\item [Representation Bias] If Task 1 and Task 2 share a common minimum in weight-space, learning the tasks with weight sharing biases the learning system to choose the shared minimum. This is effectively a form of regularization that forces the learning system to search for a good hypothesis in a hypothesis space that is restricted to hypotheses that are useful for more than one task.
\end{description}
\noindent
\citet{baxter2000} applies his bias learning framework to the case where the hypothesis space family $\mathbb{H}$ is constructed of neural networks where the first two layers are shared between tasks.

Specifically, a feature map $\phi_{\vector{w}}:\mathbb{R}^d\mapsto\mathbb{R}^{d^{(2)}}$ is a two layer neural network parameterized by $\vector{w}$ that maps an input vector $\vector{x} \in \mathbb{R}^d$ to feature vector $\phi_{\vector{w}}(\vector{x})$. Each feature $\phi_{\vector{w},j} \in \phi_{\vector{w}}$ is defined by:
$$
\phi_{\vector{w},j}(\vector{x}) = \sigma\left( w_{0j} + \sum\limits_{i=1}^{d^{(1)}} w_{ij}h_i(\vector{x}) \right)
$$
\noindent
$h_{i}$ is the output of unit $i$ in the first layer, $w_{ij}$ is the weight connecting unit $\phi_{\vector{w},j}$ and $i$ and $w_{0j}$ is the bias weight. For simplicity, \citet{baxter2000} considers only the binary threshold activation function:
$$
\sigma(a) = \begin{cases}
	+1 & \text{when } a \geq 0 \\
	-1 & \text{otherwise}
\end{cases}
$$
The output of each unit in the first layer $h_j$ is computed as
$$
h_j(\vector{x}) = \sigma\left( v_{0j} + \sum\limits_{i=1}^{d} v_{ij}x_{i} \right)
$$

where $v_{ij}$ is the weight connecting the input feature $x_i$ to unit $j$ in the first layer and $v_{0j}$ is a bias weight. The total number of weights $W$ in these two layers is thus $W = d^{(1)}(d^{(0)} + 1) + d^{(0)}(d + 1)$. The space of all such feature maps $\{\phi_{\vector{w}} \mid \vector{w} \in \mathbb{R}^W\}$ can be thought of as the representation space $\mathcal{F}$ in the representation learning framework of \citet{baxter1995}.
\\\\
\cite{baxter2000} defines a hypothesis space $\hypspace_{\vector{w}}$ as a set of binary decision functions on top the feature maps. Specifically:
$$
\hypspace_{\vector{w}} = \left\{ \sigma\left(  a_{0} + \sum\limits_{i=1}^{d^{(2)}} a_i\phi_{\vector{w},i}\right) \;\middle| \; a_{0},\dots,a_{d^{(2)}} \in \mathbb{R} \right\}
$$
where $a_i$ is the weight connecting feature $\phi_{\vector{w},i}$ and the output unit and $a_0$ is a bias weight. The set of all such hypothesis spaces can be considered a hypothesis space $\mathbb{H}$ family such that:
$$
\mathbb{H} = \{\hypspace_{\vector{w}} \mid \vector{w} \in \mathbb{R}^W\}
$$
Recall that the goal in bias learning is to select a vector $\vector{h}$ of M hypotheses $h_m$ that minimizes the average generalization error $E(\vector{h}) = \frac{1}{M}\sum_{m=1}^M E_m(h_m)$. With the limitations described above, \citet{baxter2000} is able to show that in order for the average empirical error $\hat{E}(\vector{h}, \data_M)$ to be within $\epsilon$ of the average generalization error $E(\vector{h})$ with probability $1 - \delta$, it suffices that the number of examples $N$ per task satisfies:
$$
N \geq O\left( \frac{1}{\epsilon^2}\left( \frac{W}{M} + d^{(2)} + 1\right)\log \frac{1}{\epsilon} + \frac{1}{M}\log\frac{1}{\delta} \right)
$$
Ignoring the confidence parameters $\epsilon$ and $\delta$, the sample complexity bound tells us that learning complicated neural network representations where $d^{(1)}$ and $d^{(2)}$ and therefore also $W$ are large, is harder than learning simple representations in the sense that it requires more samples to succeed. The benefit gained by multi-task learning is that we can reduce $N$ by increasing $M$, an option we don't have in the single task learning setting. This means we can afford to learn more complicated representations in the hope that this can lead to lower training error for one or more tasks and still have high confidence that generalization is possible by increasing $M$.

\section{Summary}
In this section we have introduced important contributions to statistical learning theory that shed light on some of the possible benefits of multi-task learning. Specifically, we have seen how multi-task learning can be seen as a form of bias learning that automatically learns a hypothesis space from a family of hypothesis spaces $\mathbb{H}$. Vapnik-Chervonkis style analysis of this view of multi-task learning shows that the distance between generalization and training error depends inversely on the number of tasks $M$. This means that multi-task learning is a feasible approach of re-using annotated data.
\\\\
We have discussed representation learning as a specific instance of bias learning. We have cited a result that shows that representation learning leads to much the same sample complexity dynamics as bias learning.
\\\\
Moreover, we have discussed the impact of learning related vs. unrelated tasks. We have seen that Vapnik-Chervonenkis analysis confirms our intuition that learning related tasks can provide stronger guarantees on the potential benefits for generalization error. We have also discussed efforts towards finding characteristics of target and auxiliary tasks that are good predictors of gains in generalization performance for multi-task learning. We have argued that despite these efforts, the most reliable approach for determining if multi-task learning is appropriate for a particular problem is through trial and error.
\\\\
Finally, we have discussed how to adapt neural networks for multi-task learning using hard weight sharing. We have presented a Vapnik-Chervonkenkis style analysis of deep multi-task learning that shows that the difficulty of learning a good representations for a set of tasks using hard neural network weight sharing is governed by the amount of target and auxiliary data and the the number of weights in the shared layers.
\\\\
In the next section we explain how we have applied this understanding of deep multi-task learning to implement an experiment that tests it's effectiveness in the context of relation classification.