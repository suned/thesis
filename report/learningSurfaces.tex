\section{Learning Surfaces}
A learning curve is a common technique for visualizing the empirical sample complexity dynamics for a single task. It's generated by varying the sample size $N$ and recording either training or generalization error for each configuration of $N$. In the multi-task setting, the sample size may be varied in both the target and auxiliary direction and the learning curve becomes a learning surface.
\\\\
In this section we present learning surfaces for each of our multi-task learning experiments. The x-axes show the fraction of auxiliary data used for model training and the y-axes show the fraction of target data used for model training. The z-axes show the mean macro-F1 computed for the validation target data for the 25 random splits created for each combination of fractional target and auxiliary data.

\newpage
\pagestyle{empty}
\newgeometry{twoside, inner=4.8cm, outer=3.2cm, top=1.5cm, bottom=1.5cm}
\vspace*{2cm}
\begin{figure}[h!]
	\hspace*{-2cm}%% Creator: Matplotlib, PGF backend
%%
%% To include the figure in your LaTeX document, write
%%   \input{<filename>.pgf}
%%
%% Make sure the required packages are loaded in your preamble
%%   \usepackage{pgf}
%%
%% Figures using additional raster images can only be included by \input if
%% they are in the same directory as the main LaTeX file. For loading figures
%% from other directories you can use the `import` package
%%   \usepackage{import}
%% and then include the figures with
%%   \import{<path to file>}{<filename>.pgf}
%%
%% Matplotlib used the following preamble
%%   \usepackage{fontspec}
%%   \setmainfont{Palatino}
%%   \setsansfont{Lucida Grande}
%%   \setmonofont{Andale Mono}
%%
\begingroup%
\makeatletter%
\begin{pgfpicture}%
\pgfpathrectangle{\pgfpointorigin}{\pgfqpoint{6.253282in}{3.615372in}}%
\pgfusepath{use as bounding box, clip}%
\begin{pgfscope}%
\pgfsetbuttcap%
\pgfsetmiterjoin%
\definecolor{currentfill}{rgb}{1.000000,1.000000,1.000000}%
\pgfsetfillcolor{currentfill}%
\pgfsetlinewidth{0.000000pt}%
\definecolor{currentstroke}{rgb}{1.000000,1.000000,1.000000}%
\pgfsetstrokecolor{currentstroke}%
\pgfsetdash{}{0pt}%
\pgfpathmoveto{\pgfqpoint{0.000000in}{0.000000in}}%
\pgfpathlineto{\pgfqpoint{6.253282in}{0.000000in}}%
\pgfpathlineto{\pgfqpoint{6.253282in}{3.615372in}}%
\pgfpathlineto{\pgfqpoint{0.000000in}{3.615372in}}%
\pgfpathclose%
\pgfusepath{fill}%
\end{pgfscope}%
\begin{pgfscope}%
\pgfsetbuttcap%
\pgfsetmiterjoin%
\definecolor{currentfill}{rgb}{1.000000,1.000000,1.000000}%
\pgfsetfillcolor{currentfill}%
\pgfsetlinewidth{0.000000pt}%
\definecolor{currentstroke}{rgb}{0.000000,0.000000,0.000000}%
\pgfsetstrokecolor{currentstroke}%
\pgfsetstrokeopacity{0.000000}%
\pgfsetdash{}{0pt}%
\pgfpathmoveto{\pgfqpoint{0.017500in}{0.017500in}}%
\pgfpathlineto{\pgfqpoint{6.235782in}{0.017500in}}%
\pgfpathlineto{\pgfqpoint{6.235782in}{3.597872in}}%
\pgfpathlineto{\pgfqpoint{0.017500in}{3.597872in}}%
\pgfpathclose%
\pgfusepath{fill}%
\end{pgfscope}%
\begin{pgfscope}%
\pgfsetbuttcap%
\pgfsetmiterjoin%
\definecolor{currentfill}{rgb}{0.950000,0.950000,0.950000}%
\pgfsetfillcolor{currentfill}%
\pgfsetfillopacity{0.500000}%
\pgfsetlinewidth{0.200750pt}%
\definecolor{currentstroke}{rgb}{0.950000,0.950000,0.950000}%
\pgfsetstrokecolor{currentstroke}%
\pgfsetstrokeopacity{0.500000}%
\pgfsetdash{}{0pt}%
\pgfpathmoveto{\pgfqpoint{1.233525in}{0.981833in}}%
\pgfpathlineto{\pgfqpoint{2.713039in}{1.712806in}}%
\pgfpathlineto{\pgfqpoint{2.692063in}{3.119783in}}%
\pgfpathlineto{\pgfqpoint{1.144557in}{2.450885in}}%
\pgfusepath{stroke,fill}%
\end{pgfscope}%
\begin{pgfscope}%
\pgfsetbuttcap%
\pgfsetmiterjoin%
\definecolor{currentfill}{rgb}{0.900000,0.900000,0.900000}%
\pgfsetfillcolor{currentfill}%
\pgfsetfillopacity{0.500000}%
\pgfsetlinewidth{0.200750pt}%
\definecolor{currentstroke}{rgb}{0.900000,0.900000,0.900000}%
\pgfsetstrokecolor{currentstroke}%
\pgfsetstrokeopacity{0.500000}%
\pgfsetdash{}{0pt}%
\pgfpathmoveto{\pgfqpoint{2.713039in}{1.712806in}}%
\pgfpathlineto{\pgfqpoint{5.135278in}{1.301993in}}%
\pgfpathlineto{\pgfqpoint{5.219480in}{2.744306in}}%
\pgfpathlineto{\pgfqpoint{2.692063in}{3.119783in}}%
\pgfusepath{stroke,fill}%
\end{pgfscope}%
\begin{pgfscope}%
\pgfsetbuttcap%
\pgfsetmiterjoin%
\definecolor{currentfill}{rgb}{0.925000,0.925000,0.925000}%
\pgfsetfillcolor{currentfill}%
\pgfsetfillopacity{0.500000}%
\pgfsetlinewidth{0.200750pt}%
\definecolor{currentstroke}{rgb}{0.925000,0.925000,0.925000}%
\pgfsetstrokecolor{currentstroke}%
\pgfsetstrokeopacity{0.500000}%
\pgfsetdash{}{0pt}%
\pgfpathmoveto{\pgfqpoint{1.233525in}{0.981833in}}%
\pgfpathlineto{\pgfqpoint{3.760968in}{0.515142in}}%
\pgfpathlineto{\pgfqpoint{5.135278in}{1.301993in}}%
\pgfpathlineto{\pgfqpoint{2.713039in}{1.712806in}}%
\pgfusepath{stroke,fill}%
\end{pgfscope}%
\begin{pgfscope}%
\pgfsetrectcap%
\pgfsetroundjoin%
\pgfsetlinewidth{0.501875pt}%
\definecolor{currentstroke}{rgb}{0.000000,0.000000,0.000000}%
\pgfsetstrokecolor{currentstroke}%
\pgfsetdash{}{0pt}%
\pgfpathmoveto{\pgfqpoint{1.233525in}{0.981833in}}%
\pgfpathlineto{\pgfqpoint{3.760968in}{0.515142in}}%
\pgfusepath{stroke}%
\end{pgfscope}%
\begin{pgfscope}%
\pgftext[x=2.292998in,y=0.574477in,left,base,rotate=349.538214]{\rmfamily\fontsize{10.000000}{12.000000}\selectfont }%
\end{pgfscope}%
\begin{pgfscope}%
\pgftext[x=1.953595in,y=0.474946in,left,base,rotate=349.538214]{\rmfamily\fontsize{10.000000}{12.000000}\selectfont fraction of}%
\end{pgfscope}%
\begin{pgfscope}%
\pgftext[x=1.938170in,y=0.315593in,left,base,rotate=349.538214]{\rmfamily\fontsize{10.000000}{12.000000}\selectfont ACE 2005}%
\end{pgfscope}%
\begin{pgfscope}%
\pgfsetbuttcap%
\pgfsetroundjoin%
\pgfsetlinewidth{0.803000pt}%
\definecolor{currentstroke}{rgb}{0.690196,0.690196,0.690196}%
\pgfsetstrokecolor{currentstroke}%
\pgfsetdash{}{0pt}%
\pgfpathmoveto{\pgfqpoint{1.388733in}{0.953174in}}%
\pgfpathlineto{\pgfqpoint{2.862115in}{1.687522in}}%
\pgfpathlineto{\pgfqpoint{2.847388in}{3.096708in}}%
\pgfusepath{stroke}%
\end{pgfscope}%
\begin{pgfscope}%
\pgfsetbuttcap%
\pgfsetroundjoin%
\pgfsetlinewidth{0.803000pt}%
\definecolor{currentstroke}{rgb}{0.690196,0.690196,0.690196}%
\pgfsetstrokecolor{currentstroke}%
\pgfsetdash{}{0pt}%
\pgfpathmoveto{\pgfqpoint{1.818260in}{0.873862in}}%
\pgfpathlineto{\pgfqpoint{3.274447in}{1.617591in}}%
\pgfpathlineto{\pgfqpoint{3.277159in}{3.032860in}}%
\pgfusepath{stroke}%
\end{pgfscope}%
\begin{pgfscope}%
\pgfsetbuttcap%
\pgfsetroundjoin%
\pgfsetlinewidth{0.803000pt}%
\definecolor{currentstroke}{rgb}{0.690196,0.690196,0.690196}%
\pgfsetstrokecolor{currentstroke}%
\pgfsetdash{}{0pt}%
\pgfpathmoveto{\pgfqpoint{2.253467in}{0.793501in}}%
\pgfpathlineto{\pgfqpoint{3.691895in}{1.546791in}}%
\pgfpathlineto{\pgfqpoint{3.712491in}{2.968187in}}%
\pgfusepath{stroke}%
\end{pgfscope}%
\begin{pgfscope}%
\pgfsetbuttcap%
\pgfsetroundjoin%
\pgfsetlinewidth{0.803000pt}%
\definecolor{currentstroke}{rgb}{0.690196,0.690196,0.690196}%
\pgfsetstrokecolor{currentstroke}%
\pgfsetdash{}{0pt}%
\pgfpathmoveto{\pgfqpoint{2.694467in}{0.712071in}}%
\pgfpathlineto{\pgfqpoint{4.114556in}{1.475108in}}%
\pgfpathlineto{\pgfqpoint{4.153492in}{2.902671in}}%
\pgfusepath{stroke}%
\end{pgfscope}%
\begin{pgfscope}%
\pgfsetbuttcap%
\pgfsetroundjoin%
\pgfsetlinewidth{0.803000pt}%
\definecolor{currentstroke}{rgb}{0.690196,0.690196,0.690196}%
\pgfsetstrokecolor{currentstroke}%
\pgfsetdash{}{0pt}%
\pgfpathmoveto{\pgfqpoint{3.141375in}{0.629550in}}%
\pgfpathlineto{\pgfqpoint{4.542526in}{1.402524in}}%
\pgfpathlineto{\pgfqpoint{4.600274in}{2.836296in}}%
\pgfusepath{stroke}%
\end{pgfscope}%
\begin{pgfscope}%
\pgfsetbuttcap%
\pgfsetroundjoin%
\pgfsetlinewidth{0.803000pt}%
\definecolor{currentstroke}{rgb}{0.690196,0.690196,0.690196}%
\pgfsetstrokecolor{currentstroke}%
\pgfsetdash{}{0pt}%
\pgfpathmoveto{\pgfqpoint{3.594311in}{0.545915in}}%
\pgfpathlineto{\pgfqpoint{4.975908in}{1.329022in}}%
\pgfpathlineto{\pgfqpoint{5.052952in}{2.769046in}}%
\pgfusepath{stroke}%
\end{pgfscope}%
\begin{pgfscope}%
\pgfsetrectcap%
\pgfsetroundjoin%
\pgfsetlinewidth{0.803000pt}%
\definecolor{currentstroke}{rgb}{0.000000,0.000000,0.000000}%
\pgfsetstrokecolor{currentstroke}%
\pgfsetdash{}{0pt}%
\pgfpathmoveto{\pgfqpoint{1.401276in}{0.959425in}}%
\pgfpathlineto{\pgfqpoint{1.363606in}{0.940650in}}%
\pgfusepath{stroke}%
\end{pgfscope}%
\begin{pgfscope}%
\pgftext[x=1.295334in,y=0.815921in,,top]{\rmfamily\fontsize{8.000000}{9.600000}\selectfont 0.0}%
\end{pgfscope}%
\begin{pgfscope}%
\pgfsetrectcap%
\pgfsetroundjoin%
\pgfsetlinewidth{0.803000pt}%
\definecolor{currentstroke}{rgb}{0.000000,0.000000,0.000000}%
\pgfsetstrokecolor{currentstroke}%
\pgfsetdash{}{0pt}%
\pgfpathmoveto{\pgfqpoint{1.830663in}{0.880196in}}%
\pgfpathlineto{\pgfqpoint{1.793417in}{0.861173in}}%
\pgfusepath{stroke}%
\end{pgfscope}%
\begin{pgfscope}%
\pgftext[x=1.724832in,y=0.735598in,,top]{\rmfamily\fontsize{8.000000}{9.600000}\selectfont 0.2}%
\end{pgfscope}%
\begin{pgfscope}%
\pgfsetrectcap%
\pgfsetroundjoin%
\pgfsetlinewidth{0.803000pt}%
\definecolor{currentstroke}{rgb}{0.000000,0.000000,0.000000}%
\pgfsetstrokecolor{currentstroke}%
\pgfsetdash{}{0pt}%
\pgfpathmoveto{\pgfqpoint{2.265723in}{0.799919in}}%
\pgfpathlineto{\pgfqpoint{2.228917in}{0.780644in}}%
\pgfusepath{stroke}%
\end{pgfscope}%
\begin{pgfscope}%
\pgftext[x=2.160017in,y=0.654212in,,top]{\rmfamily\fontsize{8.000000}{9.600000}\selectfont 0.4}%
\end{pgfscope}%
\begin{pgfscope}%
\pgfsetrectcap%
\pgfsetroundjoin%
\pgfsetlinewidth{0.803000pt}%
\definecolor{currentstroke}{rgb}{0.000000,0.000000,0.000000}%
\pgfsetstrokecolor{currentstroke}%
\pgfsetdash{}{0pt}%
\pgfpathmoveto{\pgfqpoint{2.706571in}{0.718575in}}%
\pgfpathlineto{\pgfqpoint{2.670219in}{0.699042in}}%
\pgfusepath{stroke}%
\end{pgfscope}%
\begin{pgfscope}%
\pgftext[x=2.601002in,y=0.571741in,,top]{\rmfamily\fontsize{8.000000}{9.600000}\selectfont 0.6}%
\end{pgfscope}%
\begin{pgfscope}%
\pgfsetrectcap%
\pgfsetroundjoin%
\pgfsetlinewidth{0.803000pt}%
\definecolor{currentstroke}{rgb}{0.000000,0.000000,0.000000}%
\pgfsetstrokecolor{currentstroke}%
\pgfsetdash{}{0pt}%
\pgfpathmoveto{\pgfqpoint{3.153323in}{0.636141in}}%
\pgfpathlineto{\pgfqpoint{3.117440in}{0.616346in}}%
\pgfusepath{stroke}%
\end{pgfscope}%
\begin{pgfscope}%
\pgftext[x=3.047904in,y=0.488164in,,top]{\rmfamily\fontsize{8.000000}{9.600000}\selectfont 0.8}%
\end{pgfscope}%
\begin{pgfscope}%
\pgfsetrectcap%
\pgfsetroundjoin%
\pgfsetlinewidth{0.803000pt}%
\definecolor{currentstroke}{rgb}{0.000000,0.000000,0.000000}%
\pgfsetstrokecolor{currentstroke}%
\pgfsetdash{}{0pt}%
\pgfpathmoveto{\pgfqpoint{3.606097in}{0.552596in}}%
\pgfpathlineto{\pgfqpoint{3.570701in}{0.532532in}}%
\pgfusepath{stroke}%
\end{pgfscope}%
\begin{pgfscope}%
\pgftext[x=3.500842in,y=0.403458in,,top]{\rmfamily\fontsize{8.000000}{9.600000}\selectfont 1.0}%
\end{pgfscope}%
\begin{pgfscope}%
\pgfsetrectcap%
\pgfsetroundjoin%
\pgfsetlinewidth{0.501875pt}%
\definecolor{currentstroke}{rgb}{0.000000,0.000000,0.000000}%
\pgfsetstrokecolor{currentstroke}%
\pgfsetdash{}{0pt}%
\pgfpathmoveto{\pgfqpoint{5.135278in}{1.301993in}}%
\pgfpathlineto{\pgfqpoint{3.760968in}{0.515142in}}%
\pgfusepath{stroke}%
\end{pgfscope}%
\begin{pgfscope}%
\pgftext[x=4.748349in,y=0.779064in,left,base,rotate=29.792981]{\rmfamily\fontsize{10.000000}{12.000000}\selectfont }%
\end{pgfscope}%
\begin{pgfscope}%
\pgftext[x=4.553640in,y=0.483787in,left,base,rotate=29.792981]{\rmfamily\fontsize{10.000000}{12.000000}\selectfont fraction of}%
\end{pgfscope}%
\begin{pgfscope}%
\pgftext[x=4.542818in,y=0.293793in,left,base,rotate=29.792981]{\rmfamily\fontsize{10.000000}{12.000000}\selectfont SemEval data}%
\end{pgfscope}%
\begin{pgfscope}%
\pgfsetbuttcap%
\pgfsetroundjoin%
\pgfsetlinewidth{0.803000pt}%
\definecolor{currentstroke}{rgb}{0.690196,0.690196,0.690196}%
\pgfsetstrokecolor{currentstroke}%
\pgfsetdash{}{0pt}%
\pgfpathmoveto{\pgfqpoint{1.249232in}{2.496130in}}%
\pgfpathlineto{\pgfqpoint{1.333347in}{1.031151in}}%
\pgfpathlineto{\pgfqpoint{3.853895in}{0.568347in}}%
\pgfusepath{stroke}%
\end{pgfscope}%
\begin{pgfscope}%
\pgfsetbuttcap%
\pgfsetroundjoin%
\pgfsetlinewidth{0.803000pt}%
\definecolor{currentstroke}{rgb}{0.690196,0.690196,0.690196}%
\pgfsetstrokecolor{currentstroke}%
\pgfsetdash{}{0pt}%
\pgfpathmoveto{\pgfqpoint{1.531798in}{2.618267in}}%
\pgfpathlineto{\pgfqpoint{1.602996in}{1.164374in}}%
\pgfpathlineto{\pgfqpoint{4.104771in}{0.711984in}}%
\pgfusepath{stroke}%
\end{pgfscope}%
\begin{pgfscope}%
\pgfsetbuttcap%
\pgfsetroundjoin%
\pgfsetlinewidth{0.803000pt}%
\definecolor{currentstroke}{rgb}{0.690196,0.690196,0.690196}%
\pgfsetstrokecolor{currentstroke}%
\pgfsetdash{}{0pt}%
\pgfpathmoveto{\pgfqpoint{1.807886in}{2.737604in}}%
\pgfpathlineto{\pgfqpoint{1.866722in}{1.294672in}}%
\pgfpathlineto{\pgfqpoint{4.349930in}{0.852348in}}%
\pgfusepath{stroke}%
\end{pgfscope}%
\begin{pgfscope}%
\pgfsetbuttcap%
\pgfsetroundjoin%
\pgfsetlinewidth{0.803000pt}%
\definecolor{currentstroke}{rgb}{0.690196,0.690196,0.690196}%
\pgfsetstrokecolor{currentstroke}%
\pgfsetdash{}{0pt}%
\pgfpathmoveto{\pgfqpoint{2.077717in}{2.854237in}}%
\pgfpathlineto{\pgfqpoint{2.124720in}{1.422139in}}%
\pgfpathlineto{\pgfqpoint{4.589565in}{0.989549in}}%
\pgfusepath{stroke}%
\end{pgfscope}%
\begin{pgfscope}%
\pgfsetbuttcap%
\pgfsetroundjoin%
\pgfsetlinewidth{0.803000pt}%
\definecolor{currentstroke}{rgb}{0.690196,0.690196,0.690196}%
\pgfsetstrokecolor{currentstroke}%
\pgfsetdash{}{0pt}%
\pgfpathmoveto{\pgfqpoint{2.341501in}{2.968255in}}%
\pgfpathlineto{\pgfqpoint{2.377173in}{1.546866in}}%
\pgfpathlineto{\pgfqpoint{4.823859in}{1.123693in}}%
\pgfusepath{stroke}%
\end{pgfscope}%
\begin{pgfscope}%
\pgfsetbuttcap%
\pgfsetroundjoin%
\pgfsetlinewidth{0.803000pt}%
\definecolor{currentstroke}{rgb}{0.690196,0.690196,0.690196}%
\pgfsetstrokecolor{currentstroke}%
\pgfsetdash{}{0pt}%
\pgfpathmoveto{\pgfqpoint{2.599439in}{3.079747in}}%
\pgfpathlineto{\pgfqpoint{2.624258in}{1.668942in}}%
\pgfpathlineto{\pgfqpoint{5.052991in}{1.254881in}}%
\pgfusepath{stroke}%
\end{pgfscope}%
\begin{pgfscope}%
\pgfsetrectcap%
\pgfsetroundjoin%
\pgfsetlinewidth{0.803000pt}%
\definecolor{currentstroke}{rgb}{0.000000,0.000000,0.000000}%
\pgfsetstrokecolor{currentstroke}%
\pgfsetdash{}{0pt}%
\pgfpathmoveto{\pgfqpoint{3.832959in}{0.572191in}}%
\pgfpathlineto{\pgfqpoint{3.895805in}{0.560652in}}%
\pgfusepath{stroke}%
\end{pgfscope}%
\begin{pgfscope}%
\pgftext[x=4.015897in,y=0.448445in,,top]{\rmfamily\fontsize{8.000000}{9.600000}\selectfont 0.0}%
\end{pgfscope}%
\begin{pgfscope}%
\pgfsetrectcap%
\pgfsetroundjoin%
\pgfsetlinewidth{0.803000pt}%
\definecolor{currentstroke}{rgb}{0.000000,0.000000,0.000000}%
\pgfsetstrokecolor{currentstroke}%
\pgfsetdash{}{0pt}%
\pgfpathmoveto{\pgfqpoint{4.084000in}{0.715740in}}%
\pgfpathlineto{\pgfqpoint{4.146351in}{0.704465in}}%
\pgfusepath{stroke}%
\end{pgfscope}%
\begin{pgfscope}%
\pgftext[x=4.264814in,y=0.593411in,,top]{\rmfamily\fontsize{8.000000}{9.600000}\selectfont 0.2}%
\end{pgfscope}%
\begin{pgfscope}%
\pgfsetrectcap%
\pgfsetroundjoin%
\pgfsetlinewidth{0.803000pt}%
\definecolor{currentstroke}{rgb}{0.000000,0.000000,0.000000}%
\pgfsetstrokecolor{currentstroke}%
\pgfsetdash{}{0pt}%
\pgfpathmoveto{\pgfqpoint{4.329322in}{0.856019in}}%
\pgfpathlineto{\pgfqpoint{4.391184in}{0.845000in}}%
\pgfusepath{stroke}%
\end{pgfscope}%
\begin{pgfscope}%
\pgftext[x=4.508060in,y=0.735074in,,top]{\rmfamily\fontsize{8.000000}{9.600000}\selectfont 0.4}%
\end{pgfscope}%
\begin{pgfscope}%
\pgfsetrectcap%
\pgfsetroundjoin%
\pgfsetlinewidth{0.803000pt}%
\definecolor{currentstroke}{rgb}{0.000000,0.000000,0.000000}%
\pgfsetstrokecolor{currentstroke}%
\pgfsetdash{}{0pt}%
\pgfpathmoveto{\pgfqpoint{4.569117in}{0.993138in}}%
\pgfpathlineto{\pgfqpoint{4.630496in}{0.982365in}}%
\pgfusepath{stroke}%
\end{pgfscope}%
\begin{pgfscope}%
\pgftext[x=4.745827in,y=0.873546in,,top]{\rmfamily\fontsize{8.000000}{9.600000}\selectfont 0.6}%
\end{pgfscope}%
\begin{pgfscope}%
\pgfsetrectcap%
\pgfsetroundjoin%
\pgfsetlinewidth{0.803000pt}%
\definecolor{currentstroke}{rgb}{0.000000,0.000000,0.000000}%
\pgfsetstrokecolor{currentstroke}%
\pgfsetdash{}{0pt}%
\pgfpathmoveto{\pgfqpoint{4.803571in}{1.127202in}}%
\pgfpathlineto{\pgfqpoint{4.864473in}{1.116668in}}%
\pgfusepath{stroke}%
\end{pgfscope}%
\begin{pgfscope}%
\pgftext[x=4.978298in,y=1.008934in,,top]{\rmfamily\fontsize{8.000000}{9.600000}\selectfont 0.8}%
\end{pgfscope}%
\begin{pgfscope}%
\pgfsetrectcap%
\pgfsetroundjoin%
\pgfsetlinewidth{0.803000pt}%
\definecolor{currentstroke}{rgb}{0.000000,0.000000,0.000000}%
\pgfsetstrokecolor{currentstroke}%
\pgfsetdash{}{0pt}%
\pgfpathmoveto{\pgfqpoint{5.032859in}{1.258313in}}%
\pgfpathlineto{\pgfqpoint{5.093291in}{1.248010in}}%
\pgfusepath{stroke}%
\end{pgfscope}%
\begin{pgfscope}%
\pgftext[x=5.205648in,y=1.141340in,,top]{\rmfamily\fontsize{8.000000}{9.600000}\selectfont 1.0}%
\end{pgfscope}%
\begin{pgfscope}%
\pgfsetrectcap%
\pgfsetroundjoin%
\pgfsetlinewidth{0.501875pt}%
\definecolor{currentstroke}{rgb}{0.000000,0.000000,0.000000}%
\pgfsetstrokecolor{currentstroke}%
\pgfsetdash{}{0pt}%
\pgfpathmoveto{\pgfqpoint{5.135278in}{1.301993in}}%
\pgfpathlineto{\pgfqpoint{5.219480in}{2.744306in}}%
\pgfusepath{stroke}%
\end{pgfscope}%
\begin{pgfscope}%
\pgftext[x=5.653991in,y=2.051089in,left,base,rotate=86.658871]{\rmfamily\fontsize{10.000000}{12.000000}\selectfont }%
\end{pgfscope}%
\begin{pgfscope}%
\pgftext[x=5.798047in,y=1.781819in,left,base,rotate=86.658871]{\rmfamily\fontsize{10.000000}{12.000000}\selectfont mean F1}%
\end{pgfscope}%
\begin{pgfscope}%
\pgfsetbuttcap%
\pgfsetroundjoin%
\pgfsetlinewidth{0.803000pt}%
\definecolor{currentstroke}{rgb}{0.690196,0.690196,0.690196}%
\pgfsetstrokecolor{currentstroke}%
\pgfsetdash{}{0pt}%
\pgfpathmoveto{\pgfqpoint{5.142508in}{1.425849in}}%
\pgfpathlineto{\pgfqpoint{2.711236in}{1.833797in}}%
\pgfpathlineto{\pgfqpoint{1.225893in}{1.107847in}}%
\pgfusepath{stroke}%
\end{pgfscope}%
\begin{pgfscope}%
\pgfsetbuttcap%
\pgfsetroundjoin%
\pgfsetlinewidth{0.803000pt}%
\definecolor{currentstroke}{rgb}{0.690196,0.690196,0.690196}%
\pgfsetstrokecolor{currentstroke}%
\pgfsetdash{}{0pt}%
\pgfpathmoveto{\pgfqpoint{5.154587in}{1.632747in}}%
\pgfpathlineto{\pgfqpoint{2.708223in}{2.035837in}}%
\pgfpathlineto{\pgfqpoint{1.213141in}{1.318408in}}%
\pgfusepath{stroke}%
\end{pgfscope}%
\begin{pgfscope}%
\pgfsetbuttcap%
\pgfsetroundjoin%
\pgfsetlinewidth{0.803000pt}%
\definecolor{currentstroke}{rgb}{0.690196,0.690196,0.690196}%
\pgfsetstrokecolor{currentstroke}%
\pgfsetdash{}{0pt}%
\pgfpathmoveto{\pgfqpoint{5.166818in}{1.842249in}}%
\pgfpathlineto{\pgfqpoint{2.705174in}{2.240330in}}%
\pgfpathlineto{\pgfqpoint{1.200225in}{1.531693in}}%
\pgfusepath{stroke}%
\end{pgfscope}%
\begin{pgfscope}%
\pgfsetbuttcap%
\pgfsetroundjoin%
\pgfsetlinewidth{0.803000pt}%
\definecolor{currentstroke}{rgb}{0.690196,0.690196,0.690196}%
\pgfsetstrokecolor{currentstroke}%
\pgfsetdash{}{0pt}%
\pgfpathmoveto{\pgfqpoint{5.179203in}{2.054403in}}%
\pgfpathlineto{\pgfqpoint{2.702088in}{2.447320in}}%
\pgfpathlineto{\pgfqpoint{1.187139in}{1.747753in}}%
\pgfusepath{stroke}%
\end{pgfscope}%
\begin{pgfscope}%
\pgfsetbuttcap%
\pgfsetroundjoin%
\pgfsetlinewidth{0.803000pt}%
\definecolor{currentstroke}{rgb}{0.690196,0.690196,0.690196}%
\pgfsetstrokecolor{currentstroke}%
\pgfsetdash{}{0pt}%
\pgfpathmoveto{\pgfqpoint{5.191747in}{2.269261in}}%
\pgfpathlineto{\pgfqpoint{2.698965in}{2.656852in}}%
\pgfpathlineto{\pgfqpoint{1.173883in}{1.966645in}}%
\pgfusepath{stroke}%
\end{pgfscope}%
\begin{pgfscope}%
\pgfsetbuttcap%
\pgfsetroundjoin%
\pgfsetlinewidth{0.803000pt}%
\definecolor{currentstroke}{rgb}{0.690196,0.690196,0.690196}%
\pgfsetstrokecolor{currentstroke}%
\pgfsetdash{}{0pt}%
\pgfpathmoveto{\pgfqpoint{5.204451in}{2.486875in}}%
\pgfpathlineto{\pgfqpoint{2.695802in}{2.868975in}}%
\pgfpathlineto{\pgfqpoint{1.160452in}{2.188424in}}%
\pgfusepath{stroke}%
\end{pgfscope}%
\begin{pgfscope}%
\pgfsetbuttcap%
\pgfsetroundjoin%
\pgfsetlinewidth{0.803000pt}%
\definecolor{currentstroke}{rgb}{0.690196,0.690196,0.690196}%
\pgfsetstrokecolor{currentstroke}%
\pgfsetdash{}{0pt}%
\pgfpathmoveto{\pgfqpoint{5.217319in}{2.707298in}}%
\pgfpathlineto{\pgfqpoint{2.692600in}{3.083735in}}%
\pgfpathlineto{\pgfqpoint{1.146842in}{2.413147in}}%
\pgfusepath{stroke}%
\end{pgfscope}%
\begin{pgfscope}%
\pgfsetrectcap%
\pgfsetroundjoin%
\pgfsetlinewidth{0.803000pt}%
\definecolor{currentstroke}{rgb}{0.000000,0.000000,0.000000}%
\pgfsetstrokecolor{currentstroke}%
\pgfsetdash{}{0pt}%
\pgfpathmoveto{\pgfqpoint{5.122355in}{1.429231in}}%
\pgfpathlineto{\pgfqpoint{5.182850in}{1.419080in}}%
\pgfusepath{stroke}%
\end{pgfscope}%
\begin{pgfscope}%
\pgftext[x=5.380726in,y=1.445191in,,top]{\rmfamily\fontsize{8.000000}{9.600000}\selectfont 0.1}%
\end{pgfscope}%
\begin{pgfscope}%
\pgfsetrectcap%
\pgfsetroundjoin%
\pgfsetlinewidth{0.803000pt}%
\definecolor{currentstroke}{rgb}{0.000000,0.000000,0.000000}%
\pgfsetstrokecolor{currentstroke}%
\pgfsetdash{}{0pt}%
\pgfpathmoveto{\pgfqpoint{5.134304in}{1.636089in}}%
\pgfpathlineto{\pgfqpoint{5.195188in}{1.626057in}}%
\pgfusepath{stroke}%
\end{pgfscope}%
\begin{pgfscope}%
\pgftext[x=5.394272in,y=1.651862in,,top]{\rmfamily\fontsize{8.000000}{9.600000}\selectfont 0.2}%
\end{pgfscope}%
\begin{pgfscope}%
\pgfsetrectcap%
\pgfsetroundjoin%
\pgfsetlinewidth{0.803000pt}%
\definecolor{currentstroke}{rgb}{0.000000,0.000000,0.000000}%
\pgfsetstrokecolor{currentstroke}%
\pgfsetdash{}{0pt}%
\pgfpathmoveto{\pgfqpoint{5.146404in}{1.845550in}}%
\pgfpathlineto{\pgfqpoint{5.207681in}{1.835640in}}%
\pgfusepath{stroke}%
\end{pgfscope}%
\begin{pgfscope}%
\pgftext[x=5.407987in,y=1.861130in,,top]{\rmfamily\fontsize{8.000000}{9.600000}\selectfont 0.3}%
\end{pgfscope}%
\begin{pgfscope}%
\pgfsetrectcap%
\pgfsetroundjoin%
\pgfsetlinewidth{0.803000pt}%
\definecolor{currentstroke}{rgb}{0.000000,0.000000,0.000000}%
\pgfsetstrokecolor{currentstroke}%
\pgfsetdash{}{0pt}%
\pgfpathmoveto{\pgfqpoint{5.158657in}{2.057662in}}%
\pgfpathlineto{\pgfqpoint{5.220333in}{2.047879in}}%
\pgfusepath{stroke}%
\end{pgfscope}%
\begin{pgfscope}%
\pgftext[x=5.421876in,y=2.073044in,,top]{\rmfamily\fontsize{8.000000}{9.600000}\selectfont 0.4}%
\end{pgfscope}%
\begin{pgfscope}%
\pgfsetrectcap%
\pgfsetroundjoin%
\pgfsetlinewidth{0.803000pt}%
\definecolor{currentstroke}{rgb}{0.000000,0.000000,0.000000}%
\pgfsetstrokecolor{currentstroke}%
\pgfsetdash{}{0pt}%
\pgfpathmoveto{\pgfqpoint{5.171065in}{2.272477in}}%
\pgfpathlineto{\pgfqpoint{5.233146in}{2.262824in}}%
\pgfusepath{stroke}%
\end{pgfscope}%
\begin{pgfscope}%
\pgftext[x=5.435942in,y=2.287653in,,top]{\rmfamily\fontsize{8.000000}{9.600000}\selectfont 0.5}%
\end{pgfscope}%
\begin{pgfscope}%
\pgfsetrectcap%
\pgfsetroundjoin%
\pgfsetlinewidth{0.803000pt}%
\definecolor{currentstroke}{rgb}{0.000000,0.000000,0.000000}%
\pgfsetstrokecolor{currentstroke}%
\pgfsetdash{}{0pt}%
\pgfpathmoveto{\pgfqpoint{5.183633in}{2.490046in}}%
\pgfpathlineto{\pgfqpoint{5.246124in}{2.480528in}}%
\pgfusepath{stroke}%
\end{pgfscope}%
\begin{pgfscope}%
\pgftext[x=5.450188in,y=2.505011in,,top]{\rmfamily\fontsize{8.000000}{9.600000}\selectfont 0.6}%
\end{pgfscope}%
\begin{pgfscope}%
\pgfsetrectcap%
\pgfsetroundjoin%
\pgfsetlinewidth{0.803000pt}%
\definecolor{currentstroke}{rgb}{0.000000,0.000000,0.000000}%
\pgfsetstrokecolor{currentstroke}%
\pgfsetdash{}{0pt}%
\pgfpathmoveto{\pgfqpoint{5.196363in}{2.710422in}}%
\pgfpathlineto{\pgfqpoint{5.259269in}{2.701043in}}%
\pgfusepath{stroke}%
\end{pgfscope}%
\begin{pgfscope}%
\pgftext[x=5.464617in,y=2.725169in,,top]{\rmfamily\fontsize{8.000000}{9.600000}\selectfont 0.7}%
\end{pgfscope}%
\begin{pgfscope}%
\pgfpathrectangle{\pgfqpoint{0.017500in}{0.017500in}}{\pgfqpoint{6.218282in}{3.580372in}} %
\pgfusepath{clip}%
\pgfsetbuttcap%
\pgfsetroundjoin%
\definecolor{currentfill}{rgb}{0.895882,0.849906,0.823499}%
\pgfsetfillcolor{currentfill}%
\pgfsetlinewidth{0.000000pt}%
\definecolor{currentstroke}{rgb}{0.000000,0.000000,0.000000}%
\pgfsetstrokecolor{currentstroke}%
\pgfsetdash{}{0pt}%
\pgfpathmoveto{\pgfqpoint{2.755750in}{3.012758in}}%
\pgfpathlineto{\pgfqpoint{1.486672in}{1.030836in}}%
\pgfpathlineto{\pgfqpoint{2.755750in}{3.012758in}}%
\pgfpathclose%
\pgfusepath{fill}%
\end{pgfscope}%
\begin{pgfscope}%
\pgfpathrectangle{\pgfqpoint{0.017500in}{0.017500in}}{\pgfqpoint{6.218282in}{3.580372in}} %
\pgfusepath{clip}%
\pgfsetbuttcap%
\pgfsetroundjoin%
\definecolor{currentfill}{rgb}{0.895882,0.849906,0.823499}%
\pgfsetfillcolor{currentfill}%
\pgfsetlinewidth{0.000000pt}%
\definecolor{currentstroke}{rgb}{0.000000,0.000000,0.000000}%
\pgfsetstrokecolor{currentstroke}%
\pgfsetdash{}{0pt}%
\pgfpathmoveto{\pgfqpoint{2.755750in}{3.012758in}}%
\pgfpathlineto{\pgfqpoint{1.486672in}{1.030836in}}%
\pgfpathlineto{\pgfqpoint{2.755750in}{3.012758in}}%
\pgfpathclose%
\pgfusepath{fill}%
\end{pgfscope}%
\begin{pgfscope}%
\pgfpathrectangle{\pgfqpoint{0.017500in}{0.017500in}}{\pgfqpoint{6.218282in}{3.580372in}} %
\pgfusepath{clip}%
\pgfsetbuttcap%
\pgfsetroundjoin%
\definecolor{currentfill}{rgb}{0.895882,0.849906,0.823499}%
\pgfsetfillcolor{currentfill}%
\pgfsetlinewidth{0.000000pt}%
\definecolor{currentstroke}{rgb}{0.000000,0.000000,0.000000}%
\pgfsetstrokecolor{currentstroke}%
\pgfsetdash{}{0pt}%
\pgfpathmoveto{\pgfqpoint{2.755750in}{3.012758in}}%
\pgfpathlineto{\pgfqpoint{1.486672in}{1.030836in}}%
\pgfpathlineto{\pgfqpoint{2.755750in}{3.012758in}}%
\pgfpathclose%
\pgfusepath{fill}%
\end{pgfscope}%
\begin{pgfscope}%
\pgfpathrectangle{\pgfqpoint{0.017500in}{0.017500in}}{\pgfqpoint{6.218282in}{3.580372in}} %
\pgfusepath{clip}%
\pgfsetbuttcap%
\pgfsetroundjoin%
\definecolor{currentfill}{rgb}{0.895882,0.849906,0.823499}%
\pgfsetfillcolor{currentfill}%
\pgfsetlinewidth{0.000000pt}%
\definecolor{currentstroke}{rgb}{0.000000,0.000000,0.000000}%
\pgfsetstrokecolor{currentstroke}%
\pgfsetdash{}{0pt}%
\pgfpathmoveto{\pgfqpoint{2.755750in}{3.012758in}}%
\pgfpathlineto{\pgfqpoint{1.486672in}{1.030836in}}%
\pgfpathlineto{\pgfqpoint{2.755750in}{3.012758in}}%
\pgfpathclose%
\pgfusepath{fill}%
\end{pgfscope}%
\begin{pgfscope}%
\pgfpathrectangle{\pgfqpoint{0.017500in}{0.017500in}}{\pgfqpoint{6.218282in}{3.580372in}} %
\pgfusepath{clip}%
\pgfsetbuttcap%
\pgfsetroundjoin%
\definecolor{currentfill}{rgb}{0.895882,0.849906,0.823499}%
\pgfsetfillcolor{currentfill}%
\pgfsetlinewidth{0.000000pt}%
\definecolor{currentstroke}{rgb}{0.000000,0.000000,0.000000}%
\pgfsetstrokecolor{currentstroke}%
\pgfsetdash{}{0pt}%
\pgfpathmoveto{\pgfqpoint{2.755750in}{3.012758in}}%
\pgfpathlineto{\pgfqpoint{1.486672in}{1.030836in}}%
\pgfpathlineto{\pgfqpoint{2.755750in}{3.012758in}}%
\pgfpathclose%
\pgfusepath{fill}%
\end{pgfscope}%
\begin{pgfscope}%
\pgfpathrectangle{\pgfqpoint{0.017500in}{0.017500in}}{\pgfqpoint{6.218282in}{3.580372in}} %
\pgfusepath{clip}%
\pgfsetbuttcap%
\pgfsetroundjoin%
\definecolor{currentfill}{rgb}{0.895882,0.849906,0.823499}%
\pgfsetfillcolor{currentfill}%
\pgfsetlinewidth{0.000000pt}%
\definecolor{currentstroke}{rgb}{0.000000,0.000000,0.000000}%
\pgfsetstrokecolor{currentstroke}%
\pgfsetdash{}{0pt}%
\pgfpathmoveto{\pgfqpoint{2.755750in}{3.012758in}}%
\pgfpathlineto{\pgfqpoint{1.486672in}{1.030836in}}%
\pgfpathlineto{\pgfqpoint{2.755750in}{3.012758in}}%
\pgfpathclose%
\pgfusepath{fill}%
\end{pgfscope}%
\begin{pgfscope}%
\pgfpathrectangle{\pgfqpoint{0.017500in}{0.017500in}}{\pgfqpoint{6.218282in}{3.580372in}} %
\pgfusepath{clip}%
\pgfsetbuttcap%
\pgfsetroundjoin%
\definecolor{currentfill}{rgb}{0.895882,0.849906,0.823499}%
\pgfsetfillcolor{currentfill}%
\pgfsetlinewidth{0.000000pt}%
\definecolor{currentstroke}{rgb}{0.000000,0.000000,0.000000}%
\pgfsetstrokecolor{currentstroke}%
\pgfsetdash{}{0pt}%
\pgfpathmoveto{\pgfqpoint{2.755750in}{3.012758in}}%
\pgfpathlineto{\pgfqpoint{1.486672in}{1.030836in}}%
\pgfpathlineto{\pgfqpoint{2.755750in}{3.012758in}}%
\pgfpathclose%
\pgfusepath{fill}%
\end{pgfscope}%
\begin{pgfscope}%
\pgfpathrectangle{\pgfqpoint{0.017500in}{0.017500in}}{\pgfqpoint{6.218282in}{3.580372in}} %
\pgfusepath{clip}%
\pgfsetbuttcap%
\pgfsetroundjoin%
\definecolor{currentfill}{rgb}{0.895882,0.849906,0.823499}%
\pgfsetfillcolor{currentfill}%
\pgfsetlinewidth{0.000000pt}%
\definecolor{currentstroke}{rgb}{0.000000,0.000000,0.000000}%
\pgfsetstrokecolor{currentstroke}%
\pgfsetdash{}{0pt}%
\pgfpathmoveto{\pgfqpoint{2.755750in}{3.012758in}}%
\pgfpathlineto{\pgfqpoint{1.486672in}{1.030836in}}%
\pgfpathlineto{\pgfqpoint{2.755750in}{3.012758in}}%
\pgfpathclose%
\pgfusepath{fill}%
\end{pgfscope}%
\begin{pgfscope}%
\pgfpathrectangle{\pgfqpoint{0.017500in}{0.017500in}}{\pgfqpoint{6.218282in}{3.580372in}} %
\pgfusepath{clip}%
\pgfsetbuttcap%
\pgfsetroundjoin%
\definecolor{currentfill}{rgb}{0.895882,0.849906,0.823499}%
\pgfsetfillcolor{currentfill}%
\pgfsetlinewidth{0.000000pt}%
\definecolor{currentstroke}{rgb}{0.000000,0.000000,0.000000}%
\pgfsetstrokecolor{currentstroke}%
\pgfsetdash{}{0pt}%
\pgfpathmoveto{\pgfqpoint{2.755750in}{3.012758in}}%
\pgfpathlineto{\pgfqpoint{1.486672in}{1.030836in}}%
\pgfpathlineto{\pgfqpoint{2.755750in}{3.012758in}}%
\pgfpathclose%
\pgfusepath{fill}%
\end{pgfscope}%
\begin{pgfscope}%
\pgfpathrectangle{\pgfqpoint{0.017500in}{0.017500in}}{\pgfqpoint{6.218282in}{3.580372in}} %
\pgfusepath{clip}%
\pgfsetbuttcap%
\pgfsetroundjoin%
\definecolor{currentfill}{rgb}{0.895882,0.849906,0.823499}%
\pgfsetfillcolor{currentfill}%
\pgfsetlinewidth{0.000000pt}%
\definecolor{currentstroke}{rgb}{0.000000,0.000000,0.000000}%
\pgfsetstrokecolor{currentstroke}%
\pgfsetdash{}{0pt}%
\pgfpathmoveto{\pgfqpoint{2.755750in}{3.012758in}}%
\pgfpathlineto{\pgfqpoint{1.486672in}{1.030836in}}%
\pgfpathlineto{\pgfqpoint{2.755750in}{3.012758in}}%
\pgfpathclose%
\pgfusepath{fill}%
\end{pgfscope}%
\begin{pgfscope}%
\pgfpathrectangle{\pgfqpoint{0.017500in}{0.017500in}}{\pgfqpoint{6.218282in}{3.580372in}} %
\pgfusepath{clip}%
\pgfsetbuttcap%
\pgfsetroundjoin%
\definecolor{currentfill}{rgb}{0.895882,0.849906,0.823499}%
\pgfsetfillcolor{currentfill}%
\pgfsetlinewidth{0.000000pt}%
\definecolor{currentstroke}{rgb}{0.000000,0.000000,0.000000}%
\pgfsetstrokecolor{currentstroke}%
\pgfsetdash{}{0pt}%
\pgfpathmoveto{\pgfqpoint{2.755750in}{3.012758in}}%
\pgfpathlineto{\pgfqpoint{1.486672in}{1.030836in}}%
\pgfpathlineto{\pgfqpoint{2.755750in}{3.012758in}}%
\pgfpathclose%
\pgfusepath{fill}%
\end{pgfscope}%
\begin{pgfscope}%
\pgfpathrectangle{\pgfqpoint{0.017500in}{0.017500in}}{\pgfqpoint{6.218282in}{3.580372in}} %
\pgfusepath{clip}%
\pgfsetbuttcap%
\pgfsetroundjoin%
\definecolor{currentfill}{rgb}{0.895882,0.849906,0.823499}%
\pgfsetfillcolor{currentfill}%
\pgfsetlinewidth{0.000000pt}%
\definecolor{currentstroke}{rgb}{0.000000,0.000000,0.000000}%
\pgfsetstrokecolor{currentstroke}%
\pgfsetdash{}{0pt}%
\pgfpathmoveto{\pgfqpoint{2.755750in}{3.012758in}}%
\pgfpathlineto{\pgfqpoint{1.486672in}{1.030836in}}%
\pgfpathlineto{\pgfqpoint{2.755750in}{3.012758in}}%
\pgfpathclose%
\pgfusepath{fill}%
\end{pgfscope}%
\begin{pgfscope}%
\pgfpathrectangle{\pgfqpoint{0.017500in}{0.017500in}}{\pgfqpoint{6.218282in}{3.580372in}} %
\pgfusepath{clip}%
\pgfsetbuttcap%
\pgfsetroundjoin%
\definecolor{currentfill}{rgb}{0.895882,0.849906,0.823499}%
\pgfsetfillcolor{currentfill}%
\pgfsetlinewidth{0.000000pt}%
\definecolor{currentstroke}{rgb}{0.000000,0.000000,0.000000}%
\pgfsetstrokecolor{currentstroke}%
\pgfsetdash{}{0pt}%
\pgfpathmoveto{\pgfqpoint{2.755750in}{3.012758in}}%
\pgfpathlineto{\pgfqpoint{1.486672in}{1.030836in}}%
\pgfpathlineto{\pgfqpoint{2.755750in}{3.012758in}}%
\pgfpathclose%
\pgfusepath{fill}%
\end{pgfscope}%
\begin{pgfscope}%
\pgfpathrectangle{\pgfqpoint{0.017500in}{0.017500in}}{\pgfqpoint{6.218282in}{3.580372in}} %
\pgfusepath{clip}%
\pgfsetbuttcap%
\pgfsetroundjoin%
\definecolor{currentfill}{rgb}{0.895882,0.849906,0.823499}%
\pgfsetfillcolor{currentfill}%
\pgfsetlinewidth{0.000000pt}%
\definecolor{currentstroke}{rgb}{0.000000,0.000000,0.000000}%
\pgfsetstrokecolor{currentstroke}%
\pgfsetdash{}{0pt}%
\pgfpathmoveto{\pgfqpoint{2.755750in}{3.012758in}}%
\pgfpathlineto{\pgfqpoint{1.486672in}{1.030836in}}%
\pgfpathlineto{\pgfqpoint{2.755750in}{3.012758in}}%
\pgfpathclose%
\pgfusepath{fill}%
\end{pgfscope}%
\begin{pgfscope}%
\pgfpathrectangle{\pgfqpoint{0.017500in}{0.017500in}}{\pgfqpoint{6.218282in}{3.580372in}} %
\pgfusepath{clip}%
\pgfsetbuttcap%
\pgfsetroundjoin%
\definecolor{currentfill}{rgb}{0.895882,0.849906,0.823499}%
\pgfsetfillcolor{currentfill}%
\pgfsetlinewidth{0.000000pt}%
\definecolor{currentstroke}{rgb}{0.000000,0.000000,0.000000}%
\pgfsetstrokecolor{currentstroke}%
\pgfsetdash{}{0pt}%
\pgfpathmoveto{\pgfqpoint{2.755750in}{3.012758in}}%
\pgfpathlineto{\pgfqpoint{1.486672in}{1.030836in}}%
\pgfpathlineto{\pgfqpoint{2.755750in}{3.012758in}}%
\pgfpathclose%
\pgfusepath{fill}%
\end{pgfscope}%
\begin{pgfscope}%
\pgfpathrectangle{\pgfqpoint{0.017500in}{0.017500in}}{\pgfqpoint{6.218282in}{3.580372in}} %
\pgfusepath{clip}%
\pgfsetbuttcap%
\pgfsetroundjoin%
\definecolor{currentfill}{rgb}{0.895882,0.849906,0.823499}%
\pgfsetfillcolor{currentfill}%
\pgfsetlinewidth{0.000000pt}%
\definecolor{currentstroke}{rgb}{0.000000,0.000000,0.000000}%
\pgfsetstrokecolor{currentstroke}%
\pgfsetdash{}{0pt}%
\pgfpathmoveto{\pgfqpoint{2.755750in}{3.012758in}}%
\pgfpathlineto{\pgfqpoint{1.486672in}{1.030836in}}%
\pgfpathlineto{\pgfqpoint{2.755750in}{3.012758in}}%
\pgfpathclose%
\pgfusepath{fill}%
\end{pgfscope}%
\begin{pgfscope}%
\pgfpathrectangle{\pgfqpoint{0.017500in}{0.017500in}}{\pgfqpoint{6.218282in}{3.580372in}} %
\pgfusepath{clip}%
\pgfsetbuttcap%
\pgfsetroundjoin%
\definecolor{currentfill}{rgb}{0.895882,0.849906,0.823499}%
\pgfsetfillcolor{currentfill}%
\pgfsetlinewidth{0.000000pt}%
\definecolor{currentstroke}{rgb}{0.000000,0.000000,0.000000}%
\pgfsetstrokecolor{currentstroke}%
\pgfsetdash{}{0pt}%
\pgfpathmoveto{\pgfqpoint{2.755750in}{3.012758in}}%
\pgfpathlineto{\pgfqpoint{1.486672in}{1.030836in}}%
\pgfpathlineto{\pgfqpoint{2.755750in}{3.012758in}}%
\pgfpathclose%
\pgfusepath{fill}%
\end{pgfscope}%
\begin{pgfscope}%
\pgfpathrectangle{\pgfqpoint{0.017500in}{0.017500in}}{\pgfqpoint{6.218282in}{3.580372in}} %
\pgfusepath{clip}%
\pgfsetbuttcap%
\pgfsetroundjoin%
\definecolor{currentfill}{rgb}{0.895882,0.849906,0.823499}%
\pgfsetfillcolor{currentfill}%
\pgfsetlinewidth{0.000000pt}%
\definecolor{currentstroke}{rgb}{0.000000,0.000000,0.000000}%
\pgfsetstrokecolor{currentstroke}%
\pgfsetdash{}{0pt}%
\pgfpathmoveto{\pgfqpoint{2.755750in}{3.012758in}}%
\pgfpathlineto{\pgfqpoint{1.486672in}{1.030836in}}%
\pgfpathlineto{\pgfqpoint{2.755750in}{3.012758in}}%
\pgfpathclose%
\pgfusepath{fill}%
\end{pgfscope}%
\begin{pgfscope}%
\pgfpathrectangle{\pgfqpoint{0.017500in}{0.017500in}}{\pgfqpoint{6.218282in}{3.580372in}} %
\pgfusepath{clip}%
\pgfsetbuttcap%
\pgfsetroundjoin%
\definecolor{currentfill}{rgb}{0.895882,0.849906,0.823499}%
\pgfsetfillcolor{currentfill}%
\pgfsetlinewidth{0.000000pt}%
\definecolor{currentstroke}{rgb}{0.000000,0.000000,0.000000}%
\pgfsetstrokecolor{currentstroke}%
\pgfsetdash{}{0pt}%
\pgfpathmoveto{\pgfqpoint{2.755750in}{3.012758in}}%
\pgfpathlineto{\pgfqpoint{1.486672in}{1.030836in}}%
\pgfpathlineto{\pgfqpoint{2.755750in}{3.012758in}}%
\pgfpathclose%
\pgfusepath{fill}%
\end{pgfscope}%
\begin{pgfscope}%
\pgfpathrectangle{\pgfqpoint{0.017500in}{0.017500in}}{\pgfqpoint{6.218282in}{3.580372in}} %
\pgfusepath{clip}%
\pgfsetbuttcap%
\pgfsetroundjoin%
\definecolor{currentfill}{rgb}{0.895882,0.849906,0.823499}%
\pgfsetfillcolor{currentfill}%
\pgfsetlinewidth{0.000000pt}%
\definecolor{currentstroke}{rgb}{0.000000,0.000000,0.000000}%
\pgfsetstrokecolor{currentstroke}%
\pgfsetdash{}{0pt}%
\pgfpathmoveto{\pgfqpoint{2.755750in}{3.012758in}}%
\pgfpathlineto{\pgfqpoint{1.486672in}{1.030836in}}%
\pgfpathlineto{\pgfqpoint{2.755750in}{3.012758in}}%
\pgfpathclose%
\pgfusepath{fill}%
\end{pgfscope}%
\begin{pgfscope}%
\pgfpathrectangle{\pgfqpoint{0.017500in}{0.017500in}}{\pgfqpoint{6.218282in}{3.580372in}} %
\pgfusepath{clip}%
\pgfsetbuttcap%
\pgfsetroundjoin%
\definecolor{currentfill}{rgb}{0.895882,0.849906,0.823499}%
\pgfsetfillcolor{currentfill}%
\pgfsetlinewidth{0.000000pt}%
\definecolor{currentstroke}{rgb}{0.000000,0.000000,0.000000}%
\pgfsetstrokecolor{currentstroke}%
\pgfsetdash{}{0pt}%
\pgfpathmoveto{\pgfqpoint{2.755750in}{3.012758in}}%
\pgfpathlineto{\pgfqpoint{1.486672in}{1.030836in}}%
\pgfpathlineto{\pgfqpoint{2.755750in}{3.012758in}}%
\pgfpathclose%
\pgfusepath{fill}%
\end{pgfscope}%
\begin{pgfscope}%
\pgfpathrectangle{\pgfqpoint{0.017500in}{0.017500in}}{\pgfqpoint{6.218282in}{3.580372in}} %
\pgfusepath{clip}%
\pgfsetbuttcap%
\pgfsetroundjoin%
\definecolor{currentfill}{rgb}{0.895882,0.849906,0.823499}%
\pgfsetfillcolor{currentfill}%
\pgfsetlinewidth{0.000000pt}%
\definecolor{currentstroke}{rgb}{0.000000,0.000000,0.000000}%
\pgfsetstrokecolor{currentstroke}%
\pgfsetdash{}{0pt}%
\pgfpathmoveto{\pgfqpoint{2.755750in}{3.012758in}}%
\pgfpathlineto{\pgfqpoint{1.486672in}{1.030836in}}%
\pgfpathlineto{\pgfqpoint{2.755750in}{3.012758in}}%
\pgfpathclose%
\pgfusepath{fill}%
\end{pgfscope}%
\begin{pgfscope}%
\pgfpathrectangle{\pgfqpoint{0.017500in}{0.017500in}}{\pgfqpoint{6.218282in}{3.580372in}} %
\pgfusepath{clip}%
\pgfsetbuttcap%
\pgfsetroundjoin%
\definecolor{currentfill}{rgb}{0.895882,0.849906,0.823499}%
\pgfsetfillcolor{currentfill}%
\pgfsetlinewidth{0.000000pt}%
\definecolor{currentstroke}{rgb}{0.000000,0.000000,0.000000}%
\pgfsetstrokecolor{currentstroke}%
\pgfsetdash{}{0pt}%
\pgfpathmoveto{\pgfqpoint{2.755750in}{3.012758in}}%
\pgfpathlineto{\pgfqpoint{1.486672in}{1.030836in}}%
\pgfpathlineto{\pgfqpoint{2.755750in}{3.012758in}}%
\pgfpathclose%
\pgfusepath{fill}%
\end{pgfscope}%
\begin{pgfscope}%
\pgfpathrectangle{\pgfqpoint{0.017500in}{0.017500in}}{\pgfqpoint{6.218282in}{3.580372in}} %
\pgfusepath{clip}%
\pgfsetbuttcap%
\pgfsetroundjoin%
\definecolor{currentfill}{rgb}{0.895882,0.849906,0.823499}%
\pgfsetfillcolor{currentfill}%
\pgfsetlinewidth{0.000000pt}%
\definecolor{currentstroke}{rgb}{0.000000,0.000000,0.000000}%
\pgfsetstrokecolor{currentstroke}%
\pgfsetdash{}{0pt}%
\pgfpathmoveto{\pgfqpoint{2.755750in}{3.012758in}}%
\pgfpathlineto{\pgfqpoint{1.486672in}{1.030836in}}%
\pgfpathlineto{\pgfqpoint{2.755750in}{3.012758in}}%
\pgfpathclose%
\pgfusepath{fill}%
\end{pgfscope}%
\begin{pgfscope}%
\pgfpathrectangle{\pgfqpoint{0.017500in}{0.017500in}}{\pgfqpoint{6.218282in}{3.580372in}} %
\pgfusepath{clip}%
\pgfsetbuttcap%
\pgfsetroundjoin%
\definecolor{currentfill}{rgb}{0.895882,0.849906,0.823499}%
\pgfsetfillcolor{currentfill}%
\pgfsetlinewidth{0.000000pt}%
\definecolor{currentstroke}{rgb}{0.000000,0.000000,0.000000}%
\pgfsetstrokecolor{currentstroke}%
\pgfsetdash{}{0pt}%
\pgfpathmoveto{\pgfqpoint{2.755750in}{3.012758in}}%
\pgfpathlineto{\pgfqpoint{1.486672in}{1.030836in}}%
\pgfpathlineto{\pgfqpoint{2.755750in}{3.012758in}}%
\pgfpathclose%
\pgfusepath{fill}%
\end{pgfscope}%
\begin{pgfscope}%
\pgfpathrectangle{\pgfqpoint{0.017500in}{0.017500in}}{\pgfqpoint{6.218282in}{3.580372in}} %
\pgfusepath{clip}%
\pgfsetbuttcap%
\pgfsetroundjoin%
\definecolor{currentfill}{rgb}{0.723315,0.068898,0.162989}%
\pgfsetfillcolor{currentfill}%
\pgfsetlinewidth{0.000000pt}%
\definecolor{currentstroke}{rgb}{0.000000,0.000000,0.000000}%
\pgfsetstrokecolor{currentstroke}%
\pgfsetdash{}{0pt}%
\pgfpathmoveto{\pgfqpoint{2.499580in}{2.886064in}}%
\pgfpathlineto{\pgfqpoint{2.755750in}{3.012758in}}%
\pgfpathlineto{\pgfqpoint{2.499580in}{2.886064in}}%
\pgfpathclose%
\pgfusepath{fill}%
\end{pgfscope}%
\begin{pgfscope}%
\pgfpathrectangle{\pgfqpoint{0.017500in}{0.017500in}}{\pgfqpoint{6.218282in}{3.580372in}} %
\pgfusepath{clip}%
\pgfsetbuttcap%
\pgfsetroundjoin%
\definecolor{currentfill}{rgb}{0.723315,0.068898,0.162989}%
\pgfsetfillcolor{currentfill}%
\pgfsetlinewidth{0.000000pt}%
\definecolor{currentstroke}{rgb}{0.000000,0.000000,0.000000}%
\pgfsetstrokecolor{currentstroke}%
\pgfsetdash{}{0pt}%
\pgfpathmoveto{\pgfqpoint{2.499580in}{2.886064in}}%
\pgfpathlineto{\pgfqpoint{2.755750in}{3.012758in}}%
\pgfpathlineto{\pgfqpoint{2.499580in}{2.886064in}}%
\pgfpathclose%
\pgfusepath{fill}%
\end{pgfscope}%
\begin{pgfscope}%
\pgfpathrectangle{\pgfqpoint{0.017500in}{0.017500in}}{\pgfqpoint{6.218282in}{3.580372in}} %
\pgfusepath{clip}%
\pgfsetbuttcap%
\pgfsetroundjoin%
\definecolor{currentfill}{rgb}{0.723315,0.068898,0.162989}%
\pgfsetfillcolor{currentfill}%
\pgfsetlinewidth{0.000000pt}%
\definecolor{currentstroke}{rgb}{0.000000,0.000000,0.000000}%
\pgfsetstrokecolor{currentstroke}%
\pgfsetdash{}{0pt}%
\pgfpathmoveto{\pgfqpoint{2.499580in}{2.886064in}}%
\pgfpathlineto{\pgfqpoint{2.755750in}{3.012758in}}%
\pgfpathlineto{\pgfqpoint{2.499580in}{2.886064in}}%
\pgfpathclose%
\pgfusepath{fill}%
\end{pgfscope}%
\begin{pgfscope}%
\pgfpathrectangle{\pgfqpoint{0.017500in}{0.017500in}}{\pgfqpoint{6.218282in}{3.580372in}} %
\pgfusepath{clip}%
\pgfsetbuttcap%
\pgfsetroundjoin%
\definecolor{currentfill}{rgb}{0.723315,0.068898,0.162989}%
\pgfsetfillcolor{currentfill}%
\pgfsetlinewidth{0.000000pt}%
\definecolor{currentstroke}{rgb}{0.000000,0.000000,0.000000}%
\pgfsetstrokecolor{currentstroke}%
\pgfsetdash{}{0pt}%
\pgfpathmoveto{\pgfqpoint{2.499580in}{2.886064in}}%
\pgfpathlineto{\pgfqpoint{2.755750in}{3.012758in}}%
\pgfpathlineto{\pgfqpoint{2.499580in}{2.886064in}}%
\pgfpathclose%
\pgfusepath{fill}%
\end{pgfscope}%
\begin{pgfscope}%
\pgfpathrectangle{\pgfqpoint{0.017500in}{0.017500in}}{\pgfqpoint{6.218282in}{3.580372in}} %
\pgfusepath{clip}%
\pgfsetbuttcap%
\pgfsetroundjoin%
\definecolor{currentfill}{rgb}{0.723315,0.068898,0.162989}%
\pgfsetfillcolor{currentfill}%
\pgfsetlinewidth{0.000000pt}%
\definecolor{currentstroke}{rgb}{0.000000,0.000000,0.000000}%
\pgfsetstrokecolor{currentstroke}%
\pgfsetdash{}{0pt}%
\pgfpathmoveto{\pgfqpoint{2.499580in}{2.886064in}}%
\pgfpathlineto{\pgfqpoint{2.755750in}{3.012758in}}%
\pgfpathlineto{\pgfqpoint{2.499580in}{2.886064in}}%
\pgfpathclose%
\pgfusepath{fill}%
\end{pgfscope}%
\begin{pgfscope}%
\pgfpathrectangle{\pgfqpoint{0.017500in}{0.017500in}}{\pgfqpoint{6.218282in}{3.580372in}} %
\pgfusepath{clip}%
\pgfsetbuttcap%
\pgfsetroundjoin%
\definecolor{currentfill}{rgb}{0.723315,0.068898,0.162989}%
\pgfsetfillcolor{currentfill}%
\pgfsetlinewidth{0.000000pt}%
\definecolor{currentstroke}{rgb}{0.000000,0.000000,0.000000}%
\pgfsetstrokecolor{currentstroke}%
\pgfsetdash{}{0pt}%
\pgfpathmoveto{\pgfqpoint{2.499580in}{2.886064in}}%
\pgfpathlineto{\pgfqpoint{2.755750in}{3.012758in}}%
\pgfpathlineto{\pgfqpoint{2.499580in}{2.886064in}}%
\pgfpathclose%
\pgfusepath{fill}%
\end{pgfscope}%
\begin{pgfscope}%
\pgfpathrectangle{\pgfqpoint{0.017500in}{0.017500in}}{\pgfqpoint{6.218282in}{3.580372in}} %
\pgfusepath{clip}%
\pgfsetbuttcap%
\pgfsetroundjoin%
\definecolor{currentfill}{rgb}{0.723315,0.068898,0.162989}%
\pgfsetfillcolor{currentfill}%
\pgfsetlinewidth{0.000000pt}%
\definecolor{currentstroke}{rgb}{0.000000,0.000000,0.000000}%
\pgfsetstrokecolor{currentstroke}%
\pgfsetdash{}{0pt}%
\pgfpathmoveto{\pgfqpoint{2.499580in}{2.886064in}}%
\pgfpathlineto{\pgfqpoint{2.755750in}{3.012758in}}%
\pgfpathlineto{\pgfqpoint{2.499580in}{2.886064in}}%
\pgfpathclose%
\pgfusepath{fill}%
\end{pgfscope}%
\begin{pgfscope}%
\pgfpathrectangle{\pgfqpoint{0.017500in}{0.017500in}}{\pgfqpoint{6.218282in}{3.580372in}} %
\pgfusepath{clip}%
\pgfsetbuttcap%
\pgfsetroundjoin%
\definecolor{currentfill}{rgb}{0.723315,0.068898,0.162989}%
\pgfsetfillcolor{currentfill}%
\pgfsetlinewidth{0.000000pt}%
\definecolor{currentstroke}{rgb}{0.000000,0.000000,0.000000}%
\pgfsetstrokecolor{currentstroke}%
\pgfsetdash{}{0pt}%
\pgfpathmoveto{\pgfqpoint{2.499580in}{2.886064in}}%
\pgfpathlineto{\pgfqpoint{2.755750in}{3.012758in}}%
\pgfpathlineto{\pgfqpoint{2.499580in}{2.886064in}}%
\pgfpathclose%
\pgfusepath{fill}%
\end{pgfscope}%
\begin{pgfscope}%
\pgfpathrectangle{\pgfqpoint{0.017500in}{0.017500in}}{\pgfqpoint{6.218282in}{3.580372in}} %
\pgfusepath{clip}%
\pgfsetbuttcap%
\pgfsetroundjoin%
\definecolor{currentfill}{rgb}{0.723315,0.068898,0.162989}%
\pgfsetfillcolor{currentfill}%
\pgfsetlinewidth{0.000000pt}%
\definecolor{currentstroke}{rgb}{0.000000,0.000000,0.000000}%
\pgfsetstrokecolor{currentstroke}%
\pgfsetdash{}{0pt}%
\pgfpathmoveto{\pgfqpoint{2.499580in}{2.886064in}}%
\pgfpathlineto{\pgfqpoint{2.755750in}{3.012758in}}%
\pgfpathlineto{\pgfqpoint{2.499580in}{2.886064in}}%
\pgfpathclose%
\pgfusepath{fill}%
\end{pgfscope}%
\begin{pgfscope}%
\pgfpathrectangle{\pgfqpoint{0.017500in}{0.017500in}}{\pgfqpoint{6.218282in}{3.580372in}} %
\pgfusepath{clip}%
\pgfsetbuttcap%
\pgfsetroundjoin%
\definecolor{currentfill}{rgb}{0.723315,0.068898,0.162989}%
\pgfsetfillcolor{currentfill}%
\pgfsetlinewidth{0.000000pt}%
\definecolor{currentstroke}{rgb}{0.000000,0.000000,0.000000}%
\pgfsetstrokecolor{currentstroke}%
\pgfsetdash{}{0pt}%
\pgfpathmoveto{\pgfqpoint{2.499580in}{2.886064in}}%
\pgfpathlineto{\pgfqpoint{2.755750in}{3.012758in}}%
\pgfpathlineto{\pgfqpoint{2.499580in}{2.886064in}}%
\pgfpathclose%
\pgfusepath{fill}%
\end{pgfscope}%
\begin{pgfscope}%
\pgfpathrectangle{\pgfqpoint{0.017500in}{0.017500in}}{\pgfqpoint{6.218282in}{3.580372in}} %
\pgfusepath{clip}%
\pgfsetbuttcap%
\pgfsetroundjoin%
\definecolor{currentfill}{rgb}{0.723315,0.068898,0.162989}%
\pgfsetfillcolor{currentfill}%
\pgfsetlinewidth{0.000000pt}%
\definecolor{currentstroke}{rgb}{0.000000,0.000000,0.000000}%
\pgfsetstrokecolor{currentstroke}%
\pgfsetdash{}{0pt}%
\pgfpathmoveto{\pgfqpoint{2.499580in}{2.886064in}}%
\pgfpathlineto{\pgfqpoint{2.755750in}{3.012758in}}%
\pgfpathlineto{\pgfqpoint{2.499580in}{2.886064in}}%
\pgfpathclose%
\pgfusepath{fill}%
\end{pgfscope}%
\begin{pgfscope}%
\pgfpathrectangle{\pgfqpoint{0.017500in}{0.017500in}}{\pgfqpoint{6.218282in}{3.580372in}} %
\pgfusepath{clip}%
\pgfsetbuttcap%
\pgfsetroundjoin%
\definecolor{currentfill}{rgb}{0.723315,0.068898,0.162989}%
\pgfsetfillcolor{currentfill}%
\pgfsetlinewidth{0.000000pt}%
\definecolor{currentstroke}{rgb}{0.000000,0.000000,0.000000}%
\pgfsetstrokecolor{currentstroke}%
\pgfsetdash{}{0pt}%
\pgfpathmoveto{\pgfqpoint{2.499580in}{2.886064in}}%
\pgfpathlineto{\pgfqpoint{2.755750in}{3.012758in}}%
\pgfpathlineto{\pgfqpoint{2.499580in}{2.886064in}}%
\pgfpathclose%
\pgfusepath{fill}%
\end{pgfscope}%
\begin{pgfscope}%
\pgfpathrectangle{\pgfqpoint{0.017500in}{0.017500in}}{\pgfqpoint{6.218282in}{3.580372in}} %
\pgfusepath{clip}%
\pgfsetbuttcap%
\pgfsetroundjoin%
\definecolor{currentfill}{rgb}{0.723315,0.068898,0.162989}%
\pgfsetfillcolor{currentfill}%
\pgfsetlinewidth{0.000000pt}%
\definecolor{currentstroke}{rgb}{0.000000,0.000000,0.000000}%
\pgfsetstrokecolor{currentstroke}%
\pgfsetdash{}{0pt}%
\pgfpathmoveto{\pgfqpoint{2.499580in}{2.886064in}}%
\pgfpathlineto{\pgfqpoint{2.755750in}{3.012758in}}%
\pgfpathlineto{\pgfqpoint{2.499580in}{2.886064in}}%
\pgfpathclose%
\pgfusepath{fill}%
\end{pgfscope}%
\begin{pgfscope}%
\pgfpathrectangle{\pgfqpoint{0.017500in}{0.017500in}}{\pgfqpoint{6.218282in}{3.580372in}} %
\pgfusepath{clip}%
\pgfsetbuttcap%
\pgfsetroundjoin%
\definecolor{currentfill}{rgb}{0.723315,0.068898,0.162989}%
\pgfsetfillcolor{currentfill}%
\pgfsetlinewidth{0.000000pt}%
\definecolor{currentstroke}{rgb}{0.000000,0.000000,0.000000}%
\pgfsetstrokecolor{currentstroke}%
\pgfsetdash{}{0pt}%
\pgfpathmoveto{\pgfqpoint{2.499580in}{2.886064in}}%
\pgfpathlineto{\pgfqpoint{2.755750in}{3.012758in}}%
\pgfpathlineto{\pgfqpoint{2.499580in}{2.886064in}}%
\pgfpathclose%
\pgfusepath{fill}%
\end{pgfscope}%
\begin{pgfscope}%
\pgfpathrectangle{\pgfqpoint{0.017500in}{0.017500in}}{\pgfqpoint{6.218282in}{3.580372in}} %
\pgfusepath{clip}%
\pgfsetbuttcap%
\pgfsetroundjoin%
\definecolor{currentfill}{rgb}{0.723315,0.068898,0.162989}%
\pgfsetfillcolor{currentfill}%
\pgfsetlinewidth{0.000000pt}%
\definecolor{currentstroke}{rgb}{0.000000,0.000000,0.000000}%
\pgfsetstrokecolor{currentstroke}%
\pgfsetdash{}{0pt}%
\pgfpathmoveto{\pgfqpoint{2.499580in}{2.886064in}}%
\pgfpathlineto{\pgfqpoint{2.755750in}{3.012758in}}%
\pgfpathlineto{\pgfqpoint{2.499580in}{2.886064in}}%
\pgfpathclose%
\pgfusepath{fill}%
\end{pgfscope}%
\begin{pgfscope}%
\pgfpathrectangle{\pgfqpoint{0.017500in}{0.017500in}}{\pgfqpoint{6.218282in}{3.580372in}} %
\pgfusepath{clip}%
\pgfsetbuttcap%
\pgfsetroundjoin%
\definecolor{currentfill}{rgb}{0.723315,0.068898,0.162989}%
\pgfsetfillcolor{currentfill}%
\pgfsetlinewidth{0.000000pt}%
\definecolor{currentstroke}{rgb}{0.000000,0.000000,0.000000}%
\pgfsetstrokecolor{currentstroke}%
\pgfsetdash{}{0pt}%
\pgfpathmoveto{\pgfqpoint{2.499580in}{2.886064in}}%
\pgfpathlineto{\pgfqpoint{2.755750in}{3.012758in}}%
\pgfpathlineto{\pgfqpoint{2.499580in}{2.886064in}}%
\pgfpathclose%
\pgfusepath{fill}%
\end{pgfscope}%
\begin{pgfscope}%
\pgfpathrectangle{\pgfqpoint{0.017500in}{0.017500in}}{\pgfqpoint{6.218282in}{3.580372in}} %
\pgfusepath{clip}%
\pgfsetbuttcap%
\pgfsetroundjoin%
\definecolor{currentfill}{rgb}{0.723315,0.068898,0.162989}%
\pgfsetfillcolor{currentfill}%
\pgfsetlinewidth{0.000000pt}%
\definecolor{currentstroke}{rgb}{0.000000,0.000000,0.000000}%
\pgfsetstrokecolor{currentstroke}%
\pgfsetdash{}{0pt}%
\pgfpathmoveto{\pgfqpoint{2.499580in}{2.886064in}}%
\pgfpathlineto{\pgfqpoint{2.755750in}{3.012758in}}%
\pgfpathlineto{\pgfqpoint{2.499580in}{2.886064in}}%
\pgfpathclose%
\pgfusepath{fill}%
\end{pgfscope}%
\begin{pgfscope}%
\pgfpathrectangle{\pgfqpoint{0.017500in}{0.017500in}}{\pgfqpoint{6.218282in}{3.580372in}} %
\pgfusepath{clip}%
\pgfsetbuttcap%
\pgfsetroundjoin%
\definecolor{currentfill}{rgb}{0.723315,0.068898,0.162989}%
\pgfsetfillcolor{currentfill}%
\pgfsetlinewidth{0.000000pt}%
\definecolor{currentstroke}{rgb}{0.000000,0.000000,0.000000}%
\pgfsetstrokecolor{currentstroke}%
\pgfsetdash{}{0pt}%
\pgfpathmoveto{\pgfqpoint{2.499580in}{2.886064in}}%
\pgfpathlineto{\pgfqpoint{2.755750in}{3.012758in}}%
\pgfpathlineto{\pgfqpoint{2.499580in}{2.886064in}}%
\pgfpathclose%
\pgfusepath{fill}%
\end{pgfscope}%
\begin{pgfscope}%
\pgfpathrectangle{\pgfqpoint{0.017500in}{0.017500in}}{\pgfqpoint{6.218282in}{3.580372in}} %
\pgfusepath{clip}%
\pgfsetbuttcap%
\pgfsetroundjoin%
\definecolor{currentfill}{rgb}{0.723315,0.068898,0.162989}%
\pgfsetfillcolor{currentfill}%
\pgfsetlinewidth{0.000000pt}%
\definecolor{currentstroke}{rgb}{0.000000,0.000000,0.000000}%
\pgfsetstrokecolor{currentstroke}%
\pgfsetdash{}{0pt}%
\pgfpathmoveto{\pgfqpoint{2.499580in}{2.886064in}}%
\pgfpathlineto{\pgfqpoint{2.755750in}{3.012758in}}%
\pgfpathlineto{\pgfqpoint{2.499580in}{2.886064in}}%
\pgfpathclose%
\pgfusepath{fill}%
\end{pgfscope}%
\begin{pgfscope}%
\pgfpathrectangle{\pgfqpoint{0.017500in}{0.017500in}}{\pgfqpoint{6.218282in}{3.580372in}} %
\pgfusepath{clip}%
\pgfsetbuttcap%
\pgfsetroundjoin%
\definecolor{currentfill}{rgb}{0.723315,0.068898,0.162989}%
\pgfsetfillcolor{currentfill}%
\pgfsetlinewidth{0.000000pt}%
\definecolor{currentstroke}{rgb}{0.000000,0.000000,0.000000}%
\pgfsetstrokecolor{currentstroke}%
\pgfsetdash{}{0pt}%
\pgfpathmoveto{\pgfqpoint{2.499580in}{2.886064in}}%
\pgfpathlineto{\pgfqpoint{2.755750in}{3.012758in}}%
\pgfpathlineto{\pgfqpoint{2.499580in}{2.886064in}}%
\pgfpathclose%
\pgfusepath{fill}%
\end{pgfscope}%
\begin{pgfscope}%
\pgfpathrectangle{\pgfqpoint{0.017500in}{0.017500in}}{\pgfqpoint{6.218282in}{3.580372in}} %
\pgfusepath{clip}%
\pgfsetbuttcap%
\pgfsetroundjoin%
\definecolor{currentfill}{rgb}{0.723315,0.068898,0.162989}%
\pgfsetfillcolor{currentfill}%
\pgfsetlinewidth{0.000000pt}%
\definecolor{currentstroke}{rgb}{0.000000,0.000000,0.000000}%
\pgfsetstrokecolor{currentstroke}%
\pgfsetdash{}{0pt}%
\pgfpathmoveto{\pgfqpoint{2.499580in}{2.886064in}}%
\pgfpathlineto{\pgfqpoint{2.755750in}{3.012758in}}%
\pgfpathlineto{\pgfqpoint{2.499580in}{2.886064in}}%
\pgfpathclose%
\pgfusepath{fill}%
\end{pgfscope}%
\begin{pgfscope}%
\pgfpathrectangle{\pgfqpoint{0.017500in}{0.017500in}}{\pgfqpoint{6.218282in}{3.580372in}} %
\pgfusepath{clip}%
\pgfsetbuttcap%
\pgfsetroundjoin%
\definecolor{currentfill}{rgb}{0.723315,0.068898,0.162989}%
\pgfsetfillcolor{currentfill}%
\pgfsetlinewidth{0.000000pt}%
\definecolor{currentstroke}{rgb}{0.000000,0.000000,0.000000}%
\pgfsetstrokecolor{currentstroke}%
\pgfsetdash{}{0pt}%
\pgfpathmoveto{\pgfqpoint{2.499580in}{2.886064in}}%
\pgfpathlineto{\pgfqpoint{2.755750in}{3.012758in}}%
\pgfpathlineto{\pgfqpoint{2.499580in}{2.886064in}}%
\pgfpathclose%
\pgfusepath{fill}%
\end{pgfscope}%
\begin{pgfscope}%
\pgfpathrectangle{\pgfqpoint{0.017500in}{0.017500in}}{\pgfqpoint{6.218282in}{3.580372in}} %
\pgfusepath{clip}%
\pgfsetbuttcap%
\pgfsetroundjoin%
\definecolor{currentfill}{rgb}{0.723315,0.068898,0.162989}%
\pgfsetfillcolor{currentfill}%
\pgfsetlinewidth{0.000000pt}%
\definecolor{currentstroke}{rgb}{0.000000,0.000000,0.000000}%
\pgfsetstrokecolor{currentstroke}%
\pgfsetdash{}{0pt}%
\pgfpathmoveto{\pgfqpoint{2.499580in}{2.886064in}}%
\pgfpathlineto{\pgfqpoint{2.755750in}{3.012758in}}%
\pgfpathlineto{\pgfqpoint{2.499580in}{2.886064in}}%
\pgfpathclose%
\pgfusepath{fill}%
\end{pgfscope}%
\begin{pgfscope}%
\pgfpathrectangle{\pgfqpoint{0.017500in}{0.017500in}}{\pgfqpoint{6.218282in}{3.580372in}} %
\pgfusepath{clip}%
\pgfsetbuttcap%
\pgfsetroundjoin%
\definecolor{currentfill}{rgb}{0.723315,0.068898,0.162989}%
\pgfsetfillcolor{currentfill}%
\pgfsetlinewidth{0.000000pt}%
\definecolor{currentstroke}{rgb}{0.000000,0.000000,0.000000}%
\pgfsetstrokecolor{currentstroke}%
\pgfsetdash{}{0pt}%
\pgfpathmoveto{\pgfqpoint{2.499580in}{2.886064in}}%
\pgfpathlineto{\pgfqpoint{2.755750in}{3.012758in}}%
\pgfpathlineto{\pgfqpoint{2.499580in}{2.886064in}}%
\pgfpathclose%
\pgfusepath{fill}%
\end{pgfscope}%
\begin{pgfscope}%
\pgfpathrectangle{\pgfqpoint{0.017500in}{0.017500in}}{\pgfqpoint{6.218282in}{3.580372in}} %
\pgfusepath{clip}%
\pgfsetbuttcap%
\pgfsetroundjoin%
\definecolor{currentfill}{rgb}{0.723315,0.068898,0.162989}%
\pgfsetfillcolor{currentfill}%
\pgfsetlinewidth{0.000000pt}%
\definecolor{currentstroke}{rgb}{0.000000,0.000000,0.000000}%
\pgfsetstrokecolor{currentstroke}%
\pgfsetdash{}{0pt}%
\pgfpathmoveto{\pgfqpoint{2.499580in}{2.886064in}}%
\pgfpathlineto{\pgfqpoint{2.755750in}{3.012758in}}%
\pgfpathlineto{\pgfqpoint{2.499580in}{2.886064in}}%
\pgfpathclose%
\pgfusepath{fill}%
\end{pgfscope}%
\begin{pgfscope}%
\pgfpathrectangle{\pgfqpoint{0.017500in}{0.017500in}}{\pgfqpoint{6.218282in}{3.580372in}} %
\pgfusepath{clip}%
\pgfsetbuttcap%
\pgfsetroundjoin%
\definecolor{currentfill}{rgb}{0.723315,0.068898,0.162989}%
\pgfsetfillcolor{currentfill}%
\pgfsetlinewidth{0.000000pt}%
\definecolor{currentstroke}{rgb}{0.000000,0.000000,0.000000}%
\pgfsetstrokecolor{currentstroke}%
\pgfsetdash{}{0pt}%
\pgfpathmoveto{\pgfqpoint{2.499580in}{2.886064in}}%
\pgfpathlineto{\pgfqpoint{2.755750in}{3.012758in}}%
\pgfpathlineto{\pgfqpoint{2.499580in}{2.886064in}}%
\pgfpathclose%
\pgfusepath{fill}%
\end{pgfscope}%
\begin{pgfscope}%
\pgfpathrectangle{\pgfqpoint{0.017500in}{0.017500in}}{\pgfqpoint{6.218282in}{3.580372in}} %
\pgfusepath{clip}%
\pgfsetbuttcap%
\pgfsetroundjoin%
\definecolor{currentfill}{rgb}{0.723315,0.068898,0.162989}%
\pgfsetfillcolor{currentfill}%
\pgfsetlinewidth{0.000000pt}%
\definecolor{currentstroke}{rgb}{0.000000,0.000000,0.000000}%
\pgfsetstrokecolor{currentstroke}%
\pgfsetdash{}{0pt}%
\pgfpathmoveto{\pgfqpoint{2.499580in}{2.886064in}}%
\pgfpathlineto{\pgfqpoint{2.755750in}{3.012758in}}%
\pgfpathlineto{\pgfqpoint{2.499580in}{2.886064in}}%
\pgfpathclose%
\pgfusepath{fill}%
\end{pgfscope}%
\begin{pgfscope}%
\pgfpathrectangle{\pgfqpoint{0.017500in}{0.017500in}}{\pgfqpoint{6.218282in}{3.580372in}} %
\pgfusepath{clip}%
\pgfsetbuttcap%
\pgfsetroundjoin%
\definecolor{currentfill}{rgb}{0.723315,0.068898,0.162989}%
\pgfsetfillcolor{currentfill}%
\pgfsetlinewidth{0.000000pt}%
\definecolor{currentstroke}{rgb}{0.000000,0.000000,0.000000}%
\pgfsetstrokecolor{currentstroke}%
\pgfsetdash{}{0pt}%
\pgfpathmoveto{\pgfqpoint{2.499580in}{2.886064in}}%
\pgfpathlineto{\pgfqpoint{2.755750in}{3.012758in}}%
\pgfpathlineto{\pgfqpoint{2.499580in}{2.886064in}}%
\pgfpathclose%
\pgfusepath{fill}%
\end{pgfscope}%
\begin{pgfscope}%
\pgfpathrectangle{\pgfqpoint{0.017500in}{0.017500in}}{\pgfqpoint{6.218282in}{3.580372in}} %
\pgfusepath{clip}%
\pgfsetbuttcap%
\pgfsetroundjoin%
\definecolor{currentfill}{rgb}{0.723315,0.068898,0.162989}%
\pgfsetfillcolor{currentfill}%
\pgfsetlinewidth{0.000000pt}%
\definecolor{currentstroke}{rgb}{0.000000,0.000000,0.000000}%
\pgfsetstrokecolor{currentstroke}%
\pgfsetdash{}{0pt}%
\pgfpathmoveto{\pgfqpoint{2.499580in}{2.886064in}}%
\pgfpathlineto{\pgfqpoint{2.755750in}{3.012758in}}%
\pgfpathlineto{\pgfqpoint{2.499580in}{2.886064in}}%
\pgfpathclose%
\pgfusepath{fill}%
\end{pgfscope}%
\begin{pgfscope}%
\pgfpathrectangle{\pgfqpoint{0.017500in}{0.017500in}}{\pgfqpoint{6.218282in}{3.580372in}} %
\pgfusepath{clip}%
\pgfsetbuttcap%
\pgfsetroundjoin%
\definecolor{currentfill}{rgb}{0.723315,0.068898,0.162989}%
\pgfsetfillcolor{currentfill}%
\pgfsetlinewidth{0.000000pt}%
\definecolor{currentstroke}{rgb}{0.000000,0.000000,0.000000}%
\pgfsetstrokecolor{currentstroke}%
\pgfsetdash{}{0pt}%
\pgfpathmoveto{\pgfqpoint{2.499580in}{2.886064in}}%
\pgfpathlineto{\pgfqpoint{2.755750in}{3.012758in}}%
\pgfpathlineto{\pgfqpoint{2.499580in}{2.886064in}}%
\pgfpathclose%
\pgfusepath{fill}%
\end{pgfscope}%
\begin{pgfscope}%
\pgfpathrectangle{\pgfqpoint{0.017500in}{0.017500in}}{\pgfqpoint{6.218282in}{3.580372in}} %
\pgfusepath{clip}%
\pgfsetbuttcap%
\pgfsetroundjoin%
\definecolor{currentfill}{rgb}{0.800601,0.850358,0.930008}%
\pgfsetfillcolor{currentfill}%
\pgfsetlinewidth{0.000000pt}%
\definecolor{currentstroke}{rgb}{0.000000,0.000000,0.000000}%
\pgfsetstrokecolor{currentstroke}%
\pgfsetdash{}{0pt}%
\pgfpathmoveto{\pgfqpoint{2.755750in}{3.012758in}}%
\pgfpathlineto{\pgfqpoint{3.186126in}{2.961388in}}%
\pgfpathlineto{\pgfqpoint{1.915196in}{0.958373in}}%
\pgfpathlineto{\pgfqpoint{1.486672in}{1.030836in}}%
\pgfpathlineto{\pgfqpoint{2.755750in}{3.012758in}}%
\pgfpathclose%
\pgfusepath{fill}%
\end{pgfscope}%
\begin{pgfscope}%
\pgfpathrectangle{\pgfqpoint{0.017500in}{0.017500in}}{\pgfqpoint{6.218282in}{3.580372in}} %
\pgfusepath{clip}%
\pgfsetbuttcap%
\pgfsetroundjoin%
\definecolor{currentfill}{rgb}{0.800601,0.850358,0.930008}%
\pgfsetfillcolor{currentfill}%
\pgfsetlinewidth{0.000000pt}%
\definecolor{currentstroke}{rgb}{0.000000,0.000000,0.000000}%
\pgfsetstrokecolor{currentstroke}%
\pgfsetdash{}{0pt}%
\pgfpathmoveto{\pgfqpoint{2.755750in}{3.012758in}}%
\pgfpathlineto{\pgfqpoint{3.186126in}{2.961388in}}%
\pgfpathlineto{\pgfqpoint{1.915196in}{0.958373in}}%
\pgfpathlineto{\pgfqpoint{1.486672in}{1.030836in}}%
\pgfpathlineto{\pgfqpoint{2.755750in}{3.012758in}}%
\pgfpathclose%
\pgfusepath{fill}%
\end{pgfscope}%
\begin{pgfscope}%
\pgfpathrectangle{\pgfqpoint{0.017500in}{0.017500in}}{\pgfqpoint{6.218282in}{3.580372in}} %
\pgfusepath{clip}%
\pgfsetbuttcap%
\pgfsetroundjoin%
\definecolor{currentfill}{rgb}{0.800601,0.850358,0.930008}%
\pgfsetfillcolor{currentfill}%
\pgfsetlinewidth{0.000000pt}%
\definecolor{currentstroke}{rgb}{0.000000,0.000000,0.000000}%
\pgfsetstrokecolor{currentstroke}%
\pgfsetdash{}{0pt}%
\pgfpathmoveto{\pgfqpoint{2.755750in}{3.012758in}}%
\pgfpathlineto{\pgfqpoint{3.186126in}{2.961388in}}%
\pgfpathlineto{\pgfqpoint{1.915196in}{0.958373in}}%
\pgfpathlineto{\pgfqpoint{1.486672in}{1.030836in}}%
\pgfpathlineto{\pgfqpoint{2.755750in}{3.012758in}}%
\pgfpathclose%
\pgfusepath{fill}%
\end{pgfscope}%
\begin{pgfscope}%
\pgfpathrectangle{\pgfqpoint{0.017500in}{0.017500in}}{\pgfqpoint{6.218282in}{3.580372in}} %
\pgfusepath{clip}%
\pgfsetbuttcap%
\pgfsetroundjoin%
\definecolor{currentfill}{rgb}{0.800601,0.850358,0.930008}%
\pgfsetfillcolor{currentfill}%
\pgfsetlinewidth{0.000000pt}%
\definecolor{currentstroke}{rgb}{0.000000,0.000000,0.000000}%
\pgfsetstrokecolor{currentstroke}%
\pgfsetdash{}{0pt}%
\pgfpathmoveto{\pgfqpoint{2.755750in}{3.012758in}}%
\pgfpathlineto{\pgfqpoint{3.186126in}{2.961388in}}%
\pgfpathlineto{\pgfqpoint{1.915196in}{0.958373in}}%
\pgfpathlineto{\pgfqpoint{1.486672in}{1.030836in}}%
\pgfpathlineto{\pgfqpoint{2.755750in}{3.012758in}}%
\pgfpathclose%
\pgfusepath{fill}%
\end{pgfscope}%
\begin{pgfscope}%
\pgfpathrectangle{\pgfqpoint{0.017500in}{0.017500in}}{\pgfqpoint{6.218282in}{3.580372in}} %
\pgfusepath{clip}%
\pgfsetbuttcap%
\pgfsetroundjoin%
\definecolor{currentfill}{rgb}{0.800601,0.850358,0.930008}%
\pgfsetfillcolor{currentfill}%
\pgfsetlinewidth{0.000000pt}%
\definecolor{currentstroke}{rgb}{0.000000,0.000000,0.000000}%
\pgfsetstrokecolor{currentstroke}%
\pgfsetdash{}{0pt}%
\pgfpathmoveto{\pgfqpoint{2.755750in}{3.012758in}}%
\pgfpathlineto{\pgfqpoint{3.186126in}{2.961388in}}%
\pgfpathlineto{\pgfqpoint{1.915196in}{0.958373in}}%
\pgfpathlineto{\pgfqpoint{1.486672in}{1.030836in}}%
\pgfpathlineto{\pgfqpoint{2.755750in}{3.012758in}}%
\pgfpathclose%
\pgfusepath{fill}%
\end{pgfscope}%
\begin{pgfscope}%
\pgfpathrectangle{\pgfqpoint{0.017500in}{0.017500in}}{\pgfqpoint{6.218282in}{3.580372in}} %
\pgfusepath{clip}%
\pgfsetbuttcap%
\pgfsetroundjoin%
\definecolor{currentfill}{rgb}{0.717435,0.051118,0.158737}%
\pgfsetfillcolor{currentfill}%
\pgfsetlinewidth{0.000000pt}%
\definecolor{currentstroke}{rgb}{0.000000,0.000000,0.000000}%
\pgfsetstrokecolor{currentstroke}%
\pgfsetdash{}{0pt}%
\pgfpathmoveto{\pgfqpoint{2.499580in}{2.886064in}}%
\pgfpathlineto{\pgfqpoint{2.932971in}{2.820255in}}%
\pgfpathlineto{\pgfqpoint{3.186126in}{2.961388in}}%
\pgfpathlineto{\pgfqpoint{2.755750in}{3.012758in}}%
\pgfpathlineto{\pgfqpoint{2.499580in}{2.886064in}}%
\pgfpathclose%
\pgfusepath{fill}%
\end{pgfscope}%
\begin{pgfscope}%
\pgfpathrectangle{\pgfqpoint{0.017500in}{0.017500in}}{\pgfqpoint{6.218282in}{3.580372in}} %
\pgfusepath{clip}%
\pgfsetbuttcap%
\pgfsetroundjoin%
\definecolor{currentfill}{rgb}{0.717435,0.051118,0.158737}%
\pgfsetfillcolor{currentfill}%
\pgfsetlinewidth{0.000000pt}%
\definecolor{currentstroke}{rgb}{0.000000,0.000000,0.000000}%
\pgfsetstrokecolor{currentstroke}%
\pgfsetdash{}{0pt}%
\pgfpathmoveto{\pgfqpoint{2.499580in}{2.886064in}}%
\pgfpathlineto{\pgfqpoint{2.932971in}{2.820255in}}%
\pgfpathlineto{\pgfqpoint{3.186126in}{2.961388in}}%
\pgfpathlineto{\pgfqpoint{2.755750in}{3.012758in}}%
\pgfpathlineto{\pgfqpoint{2.499580in}{2.886064in}}%
\pgfpathclose%
\pgfusepath{fill}%
\end{pgfscope}%
\begin{pgfscope}%
\pgfpathrectangle{\pgfqpoint{0.017500in}{0.017500in}}{\pgfqpoint{6.218282in}{3.580372in}} %
\pgfusepath{clip}%
\pgfsetbuttcap%
\pgfsetroundjoin%
\definecolor{currentfill}{rgb}{0.717435,0.051118,0.158737}%
\pgfsetfillcolor{currentfill}%
\pgfsetlinewidth{0.000000pt}%
\definecolor{currentstroke}{rgb}{0.000000,0.000000,0.000000}%
\pgfsetstrokecolor{currentstroke}%
\pgfsetdash{}{0pt}%
\pgfpathmoveto{\pgfqpoint{2.499580in}{2.886064in}}%
\pgfpathlineto{\pgfqpoint{2.932971in}{2.820255in}}%
\pgfpathlineto{\pgfqpoint{3.186126in}{2.961388in}}%
\pgfpathlineto{\pgfqpoint{2.755750in}{3.012758in}}%
\pgfpathlineto{\pgfqpoint{2.499580in}{2.886064in}}%
\pgfpathclose%
\pgfusepath{fill}%
\end{pgfscope}%
\begin{pgfscope}%
\pgfpathrectangle{\pgfqpoint{0.017500in}{0.017500in}}{\pgfqpoint{6.218282in}{3.580372in}} %
\pgfusepath{clip}%
\pgfsetbuttcap%
\pgfsetroundjoin%
\definecolor{currentfill}{rgb}{0.717435,0.051118,0.158737}%
\pgfsetfillcolor{currentfill}%
\pgfsetlinewidth{0.000000pt}%
\definecolor{currentstroke}{rgb}{0.000000,0.000000,0.000000}%
\pgfsetstrokecolor{currentstroke}%
\pgfsetdash{}{0pt}%
\pgfpathmoveto{\pgfqpoint{2.499580in}{2.886064in}}%
\pgfpathlineto{\pgfqpoint{2.932971in}{2.820255in}}%
\pgfpathlineto{\pgfqpoint{3.186126in}{2.961388in}}%
\pgfpathlineto{\pgfqpoint{2.755750in}{3.012758in}}%
\pgfpathlineto{\pgfqpoint{2.499580in}{2.886064in}}%
\pgfpathclose%
\pgfusepath{fill}%
\end{pgfscope}%
\begin{pgfscope}%
\pgfpathrectangle{\pgfqpoint{0.017500in}{0.017500in}}{\pgfqpoint{6.218282in}{3.580372in}} %
\pgfusepath{clip}%
\pgfsetbuttcap%
\pgfsetroundjoin%
\definecolor{currentfill}{rgb}{0.717435,0.051118,0.158737}%
\pgfsetfillcolor{currentfill}%
\pgfsetlinewidth{0.000000pt}%
\definecolor{currentstroke}{rgb}{0.000000,0.000000,0.000000}%
\pgfsetstrokecolor{currentstroke}%
\pgfsetdash{}{0pt}%
\pgfpathmoveto{\pgfqpoint{2.499580in}{2.886064in}}%
\pgfpathlineto{\pgfqpoint{2.932971in}{2.820255in}}%
\pgfpathlineto{\pgfqpoint{3.186126in}{2.961388in}}%
\pgfpathlineto{\pgfqpoint{2.755750in}{3.012758in}}%
\pgfpathlineto{\pgfqpoint{2.499580in}{2.886064in}}%
\pgfpathclose%
\pgfusepath{fill}%
\end{pgfscope}%
\begin{pgfscope}%
\pgfpathrectangle{\pgfqpoint{0.017500in}{0.017500in}}{\pgfqpoint{6.218282in}{3.580372in}} %
\pgfusepath{clip}%
\pgfsetbuttcap%
\pgfsetroundjoin%
\definecolor{currentfill}{rgb}{0.717435,0.051118,0.158737}%
\pgfsetfillcolor{currentfill}%
\pgfsetlinewidth{0.000000pt}%
\definecolor{currentstroke}{rgb}{0.000000,0.000000,0.000000}%
\pgfsetstrokecolor{currentstroke}%
\pgfsetdash{}{0pt}%
\pgfpathmoveto{\pgfqpoint{2.499580in}{2.886064in}}%
\pgfpathlineto{\pgfqpoint{2.932971in}{2.820255in}}%
\pgfpathlineto{\pgfqpoint{3.186126in}{2.961388in}}%
\pgfpathlineto{\pgfqpoint{2.755750in}{3.012758in}}%
\pgfpathlineto{\pgfqpoint{2.499580in}{2.886064in}}%
\pgfpathclose%
\pgfusepath{fill}%
\end{pgfscope}%
\begin{pgfscope}%
\pgfpathrectangle{\pgfqpoint{0.017500in}{0.017500in}}{\pgfqpoint{6.218282in}{3.580372in}} %
\pgfusepath{clip}%
\pgfsetbuttcap%
\pgfsetroundjoin%
\definecolor{currentfill}{rgb}{0.902849,0.844796,0.811970}%
\pgfsetfillcolor{currentfill}%
\pgfsetlinewidth{0.000000pt}%
\definecolor{currentstroke}{rgb}{0.000000,0.000000,0.000000}%
\pgfsetstrokecolor{currentstroke}%
\pgfsetdash{}{0pt}%
\pgfpathmoveto{\pgfqpoint{3.186126in}{2.961388in}}%
\pgfpathlineto{\pgfqpoint{1.915196in}{0.958373in}}%
\pgfpathlineto{\pgfqpoint{3.186126in}{2.961388in}}%
\pgfpathclose%
\pgfusepath{fill}%
\end{pgfscope}%
\begin{pgfscope}%
\pgfpathrectangle{\pgfqpoint{0.017500in}{0.017500in}}{\pgfqpoint{6.218282in}{3.580372in}} %
\pgfusepath{clip}%
\pgfsetbuttcap%
\pgfsetroundjoin%
\definecolor{currentfill}{rgb}{0.902849,0.844796,0.811970}%
\pgfsetfillcolor{currentfill}%
\pgfsetlinewidth{0.000000pt}%
\definecolor{currentstroke}{rgb}{0.000000,0.000000,0.000000}%
\pgfsetstrokecolor{currentstroke}%
\pgfsetdash{}{0pt}%
\pgfpathmoveto{\pgfqpoint{3.186126in}{2.961388in}}%
\pgfpathlineto{\pgfqpoint{1.915196in}{0.958373in}}%
\pgfpathlineto{\pgfqpoint{3.186126in}{2.961388in}}%
\pgfpathclose%
\pgfusepath{fill}%
\end{pgfscope}%
\begin{pgfscope}%
\pgfpathrectangle{\pgfqpoint{0.017500in}{0.017500in}}{\pgfqpoint{6.218282in}{3.580372in}} %
\pgfusepath{clip}%
\pgfsetbuttcap%
\pgfsetroundjoin%
\definecolor{currentfill}{rgb}{0.902849,0.844796,0.811970}%
\pgfsetfillcolor{currentfill}%
\pgfsetlinewidth{0.000000pt}%
\definecolor{currentstroke}{rgb}{0.000000,0.000000,0.000000}%
\pgfsetstrokecolor{currentstroke}%
\pgfsetdash{}{0pt}%
\pgfpathmoveto{\pgfqpoint{3.186126in}{2.961388in}}%
\pgfpathlineto{\pgfqpoint{1.915196in}{0.958373in}}%
\pgfpathlineto{\pgfqpoint{3.186126in}{2.961388in}}%
\pgfpathclose%
\pgfusepath{fill}%
\end{pgfscope}%
\begin{pgfscope}%
\pgfpathrectangle{\pgfqpoint{0.017500in}{0.017500in}}{\pgfqpoint{6.218282in}{3.580372in}} %
\pgfusepath{clip}%
\pgfsetbuttcap%
\pgfsetroundjoin%
\definecolor{currentfill}{rgb}{0.902849,0.844796,0.811970}%
\pgfsetfillcolor{currentfill}%
\pgfsetlinewidth{0.000000pt}%
\definecolor{currentstroke}{rgb}{0.000000,0.000000,0.000000}%
\pgfsetstrokecolor{currentstroke}%
\pgfsetdash{}{0pt}%
\pgfpathmoveto{\pgfqpoint{3.186126in}{2.961388in}}%
\pgfpathlineto{\pgfqpoint{1.915196in}{0.958373in}}%
\pgfpathlineto{\pgfqpoint{3.186126in}{2.961388in}}%
\pgfpathclose%
\pgfusepath{fill}%
\end{pgfscope}%
\begin{pgfscope}%
\pgfpathrectangle{\pgfqpoint{0.017500in}{0.017500in}}{\pgfqpoint{6.218282in}{3.580372in}} %
\pgfusepath{clip}%
\pgfsetbuttcap%
\pgfsetroundjoin%
\definecolor{currentfill}{rgb}{0.902849,0.844796,0.811970}%
\pgfsetfillcolor{currentfill}%
\pgfsetlinewidth{0.000000pt}%
\definecolor{currentstroke}{rgb}{0.000000,0.000000,0.000000}%
\pgfsetstrokecolor{currentstroke}%
\pgfsetdash{}{0pt}%
\pgfpathmoveto{\pgfqpoint{3.186126in}{2.961388in}}%
\pgfpathlineto{\pgfqpoint{1.915196in}{0.958373in}}%
\pgfpathlineto{\pgfqpoint{3.186126in}{2.961388in}}%
\pgfpathclose%
\pgfusepath{fill}%
\end{pgfscope}%
\begin{pgfscope}%
\pgfpathrectangle{\pgfqpoint{0.017500in}{0.017500in}}{\pgfqpoint{6.218282in}{3.580372in}} %
\pgfusepath{clip}%
\pgfsetbuttcap%
\pgfsetroundjoin%
\definecolor{currentfill}{rgb}{0.902849,0.844796,0.811970}%
\pgfsetfillcolor{currentfill}%
\pgfsetlinewidth{0.000000pt}%
\definecolor{currentstroke}{rgb}{0.000000,0.000000,0.000000}%
\pgfsetstrokecolor{currentstroke}%
\pgfsetdash{}{0pt}%
\pgfpathmoveto{\pgfqpoint{3.186126in}{2.961388in}}%
\pgfpathlineto{\pgfqpoint{1.915196in}{0.958373in}}%
\pgfpathlineto{\pgfqpoint{3.186126in}{2.961388in}}%
\pgfpathclose%
\pgfusepath{fill}%
\end{pgfscope}%
\begin{pgfscope}%
\pgfpathrectangle{\pgfqpoint{0.017500in}{0.017500in}}{\pgfqpoint{6.218282in}{3.580372in}} %
\pgfusepath{clip}%
\pgfsetbuttcap%
\pgfsetroundjoin%
\definecolor{currentfill}{rgb}{0.902849,0.844796,0.811970}%
\pgfsetfillcolor{currentfill}%
\pgfsetlinewidth{0.000000pt}%
\definecolor{currentstroke}{rgb}{0.000000,0.000000,0.000000}%
\pgfsetstrokecolor{currentstroke}%
\pgfsetdash{}{0pt}%
\pgfpathmoveto{\pgfqpoint{3.186126in}{2.961388in}}%
\pgfpathlineto{\pgfqpoint{1.915196in}{0.958373in}}%
\pgfpathlineto{\pgfqpoint{3.186126in}{2.961388in}}%
\pgfpathclose%
\pgfusepath{fill}%
\end{pgfscope}%
\begin{pgfscope}%
\pgfpathrectangle{\pgfqpoint{0.017500in}{0.017500in}}{\pgfqpoint{6.218282in}{3.580372in}} %
\pgfusepath{clip}%
\pgfsetbuttcap%
\pgfsetroundjoin%
\definecolor{currentfill}{rgb}{0.902849,0.844796,0.811970}%
\pgfsetfillcolor{currentfill}%
\pgfsetlinewidth{0.000000pt}%
\definecolor{currentstroke}{rgb}{0.000000,0.000000,0.000000}%
\pgfsetstrokecolor{currentstroke}%
\pgfsetdash{}{0pt}%
\pgfpathmoveto{\pgfqpoint{3.186126in}{2.961388in}}%
\pgfpathlineto{\pgfqpoint{1.915196in}{0.958373in}}%
\pgfpathlineto{\pgfqpoint{3.186126in}{2.961388in}}%
\pgfpathclose%
\pgfusepath{fill}%
\end{pgfscope}%
\begin{pgfscope}%
\pgfpathrectangle{\pgfqpoint{0.017500in}{0.017500in}}{\pgfqpoint{6.218282in}{3.580372in}} %
\pgfusepath{clip}%
\pgfsetbuttcap%
\pgfsetroundjoin%
\definecolor{currentfill}{rgb}{0.902849,0.844796,0.811970}%
\pgfsetfillcolor{currentfill}%
\pgfsetlinewidth{0.000000pt}%
\definecolor{currentstroke}{rgb}{0.000000,0.000000,0.000000}%
\pgfsetstrokecolor{currentstroke}%
\pgfsetdash{}{0pt}%
\pgfpathmoveto{\pgfqpoint{3.186126in}{2.961388in}}%
\pgfpathlineto{\pgfqpoint{1.915196in}{0.958373in}}%
\pgfpathlineto{\pgfqpoint{3.186126in}{2.961388in}}%
\pgfpathclose%
\pgfusepath{fill}%
\end{pgfscope}%
\begin{pgfscope}%
\pgfpathrectangle{\pgfqpoint{0.017500in}{0.017500in}}{\pgfqpoint{6.218282in}{3.580372in}} %
\pgfusepath{clip}%
\pgfsetbuttcap%
\pgfsetroundjoin%
\definecolor{currentfill}{rgb}{0.902849,0.844796,0.811970}%
\pgfsetfillcolor{currentfill}%
\pgfsetlinewidth{0.000000pt}%
\definecolor{currentstroke}{rgb}{0.000000,0.000000,0.000000}%
\pgfsetstrokecolor{currentstroke}%
\pgfsetdash{}{0pt}%
\pgfpathmoveto{\pgfqpoint{3.186126in}{2.961388in}}%
\pgfpathlineto{\pgfqpoint{1.915196in}{0.958373in}}%
\pgfpathlineto{\pgfqpoint{3.186126in}{2.961388in}}%
\pgfpathclose%
\pgfusepath{fill}%
\end{pgfscope}%
\begin{pgfscope}%
\pgfpathrectangle{\pgfqpoint{0.017500in}{0.017500in}}{\pgfqpoint{6.218282in}{3.580372in}} %
\pgfusepath{clip}%
\pgfsetbuttcap%
\pgfsetroundjoin%
\definecolor{currentfill}{rgb}{0.902849,0.844796,0.811970}%
\pgfsetfillcolor{currentfill}%
\pgfsetlinewidth{0.000000pt}%
\definecolor{currentstroke}{rgb}{0.000000,0.000000,0.000000}%
\pgfsetstrokecolor{currentstroke}%
\pgfsetdash{}{0pt}%
\pgfpathmoveto{\pgfqpoint{3.186126in}{2.961388in}}%
\pgfpathlineto{\pgfqpoint{1.915196in}{0.958373in}}%
\pgfpathlineto{\pgfqpoint{3.186126in}{2.961388in}}%
\pgfpathclose%
\pgfusepath{fill}%
\end{pgfscope}%
\begin{pgfscope}%
\pgfpathrectangle{\pgfqpoint{0.017500in}{0.017500in}}{\pgfqpoint{6.218282in}{3.580372in}} %
\pgfusepath{clip}%
\pgfsetbuttcap%
\pgfsetroundjoin%
\definecolor{currentfill}{rgb}{0.902849,0.844796,0.811970}%
\pgfsetfillcolor{currentfill}%
\pgfsetlinewidth{0.000000pt}%
\definecolor{currentstroke}{rgb}{0.000000,0.000000,0.000000}%
\pgfsetstrokecolor{currentstroke}%
\pgfsetdash{}{0pt}%
\pgfpathmoveto{\pgfqpoint{3.186126in}{2.961388in}}%
\pgfpathlineto{\pgfqpoint{1.915196in}{0.958373in}}%
\pgfpathlineto{\pgfqpoint{3.186126in}{2.961388in}}%
\pgfpathclose%
\pgfusepath{fill}%
\end{pgfscope}%
\begin{pgfscope}%
\pgfpathrectangle{\pgfqpoint{0.017500in}{0.017500in}}{\pgfqpoint{6.218282in}{3.580372in}} %
\pgfusepath{clip}%
\pgfsetbuttcap%
\pgfsetroundjoin%
\definecolor{currentfill}{rgb}{0.902849,0.844796,0.811970}%
\pgfsetfillcolor{currentfill}%
\pgfsetlinewidth{0.000000pt}%
\definecolor{currentstroke}{rgb}{0.000000,0.000000,0.000000}%
\pgfsetstrokecolor{currentstroke}%
\pgfsetdash{}{0pt}%
\pgfpathmoveto{\pgfqpoint{3.186126in}{2.961388in}}%
\pgfpathlineto{\pgfqpoint{1.915196in}{0.958373in}}%
\pgfpathlineto{\pgfqpoint{3.186126in}{2.961388in}}%
\pgfpathclose%
\pgfusepath{fill}%
\end{pgfscope}%
\begin{pgfscope}%
\pgfpathrectangle{\pgfqpoint{0.017500in}{0.017500in}}{\pgfqpoint{6.218282in}{3.580372in}} %
\pgfusepath{clip}%
\pgfsetbuttcap%
\pgfsetroundjoin%
\definecolor{currentfill}{rgb}{0.902849,0.844796,0.811970}%
\pgfsetfillcolor{currentfill}%
\pgfsetlinewidth{0.000000pt}%
\definecolor{currentstroke}{rgb}{0.000000,0.000000,0.000000}%
\pgfsetstrokecolor{currentstroke}%
\pgfsetdash{}{0pt}%
\pgfpathmoveto{\pgfqpoint{3.186126in}{2.961388in}}%
\pgfpathlineto{\pgfqpoint{1.915196in}{0.958373in}}%
\pgfpathlineto{\pgfqpoint{3.186126in}{2.961388in}}%
\pgfpathclose%
\pgfusepath{fill}%
\end{pgfscope}%
\begin{pgfscope}%
\pgfpathrectangle{\pgfqpoint{0.017500in}{0.017500in}}{\pgfqpoint{6.218282in}{3.580372in}} %
\pgfusepath{clip}%
\pgfsetbuttcap%
\pgfsetroundjoin%
\definecolor{currentfill}{rgb}{0.902849,0.844796,0.811970}%
\pgfsetfillcolor{currentfill}%
\pgfsetlinewidth{0.000000pt}%
\definecolor{currentstroke}{rgb}{0.000000,0.000000,0.000000}%
\pgfsetstrokecolor{currentstroke}%
\pgfsetdash{}{0pt}%
\pgfpathmoveto{\pgfqpoint{3.186126in}{2.961388in}}%
\pgfpathlineto{\pgfqpoint{1.915196in}{0.958373in}}%
\pgfpathlineto{\pgfqpoint{3.186126in}{2.961388in}}%
\pgfpathclose%
\pgfusepath{fill}%
\end{pgfscope}%
\begin{pgfscope}%
\pgfpathrectangle{\pgfqpoint{0.017500in}{0.017500in}}{\pgfqpoint{6.218282in}{3.580372in}} %
\pgfusepath{clip}%
\pgfsetbuttcap%
\pgfsetroundjoin%
\definecolor{currentfill}{rgb}{0.902849,0.844796,0.811970}%
\pgfsetfillcolor{currentfill}%
\pgfsetlinewidth{0.000000pt}%
\definecolor{currentstroke}{rgb}{0.000000,0.000000,0.000000}%
\pgfsetstrokecolor{currentstroke}%
\pgfsetdash{}{0pt}%
\pgfpathmoveto{\pgfqpoint{3.186126in}{2.961388in}}%
\pgfpathlineto{\pgfqpoint{1.915196in}{0.958373in}}%
\pgfpathlineto{\pgfqpoint{3.186126in}{2.961388in}}%
\pgfpathclose%
\pgfusepath{fill}%
\end{pgfscope}%
\begin{pgfscope}%
\pgfpathrectangle{\pgfqpoint{0.017500in}{0.017500in}}{\pgfqpoint{6.218282in}{3.580372in}} %
\pgfusepath{clip}%
\pgfsetbuttcap%
\pgfsetroundjoin%
\definecolor{currentfill}{rgb}{0.902849,0.844796,0.811970}%
\pgfsetfillcolor{currentfill}%
\pgfsetlinewidth{0.000000pt}%
\definecolor{currentstroke}{rgb}{0.000000,0.000000,0.000000}%
\pgfsetstrokecolor{currentstroke}%
\pgfsetdash{}{0pt}%
\pgfpathmoveto{\pgfqpoint{3.186126in}{2.961388in}}%
\pgfpathlineto{\pgfqpoint{1.915196in}{0.958373in}}%
\pgfpathlineto{\pgfqpoint{3.186126in}{2.961388in}}%
\pgfpathclose%
\pgfusepath{fill}%
\end{pgfscope}%
\begin{pgfscope}%
\pgfpathrectangle{\pgfqpoint{0.017500in}{0.017500in}}{\pgfqpoint{6.218282in}{3.580372in}} %
\pgfusepath{clip}%
\pgfsetbuttcap%
\pgfsetroundjoin%
\definecolor{currentfill}{rgb}{0.902849,0.844796,0.811970}%
\pgfsetfillcolor{currentfill}%
\pgfsetlinewidth{0.000000pt}%
\definecolor{currentstroke}{rgb}{0.000000,0.000000,0.000000}%
\pgfsetstrokecolor{currentstroke}%
\pgfsetdash{}{0pt}%
\pgfpathmoveto{\pgfqpoint{3.186126in}{2.961388in}}%
\pgfpathlineto{\pgfqpoint{1.915196in}{0.958373in}}%
\pgfpathlineto{\pgfqpoint{3.186126in}{2.961388in}}%
\pgfpathclose%
\pgfusepath{fill}%
\end{pgfscope}%
\begin{pgfscope}%
\pgfpathrectangle{\pgfqpoint{0.017500in}{0.017500in}}{\pgfqpoint{6.218282in}{3.580372in}} %
\pgfusepath{clip}%
\pgfsetbuttcap%
\pgfsetroundjoin%
\definecolor{currentfill}{rgb}{0.902849,0.844796,0.811970}%
\pgfsetfillcolor{currentfill}%
\pgfsetlinewidth{0.000000pt}%
\definecolor{currentstroke}{rgb}{0.000000,0.000000,0.000000}%
\pgfsetstrokecolor{currentstroke}%
\pgfsetdash{}{0pt}%
\pgfpathmoveto{\pgfqpoint{3.186126in}{2.961388in}}%
\pgfpathlineto{\pgfqpoint{1.915196in}{0.958373in}}%
\pgfpathlineto{\pgfqpoint{3.186126in}{2.961388in}}%
\pgfpathclose%
\pgfusepath{fill}%
\end{pgfscope}%
\begin{pgfscope}%
\pgfpathrectangle{\pgfqpoint{0.017500in}{0.017500in}}{\pgfqpoint{6.218282in}{3.580372in}} %
\pgfusepath{clip}%
\pgfsetbuttcap%
\pgfsetroundjoin%
\definecolor{currentfill}{rgb}{0.902849,0.844796,0.811970}%
\pgfsetfillcolor{currentfill}%
\pgfsetlinewidth{0.000000pt}%
\definecolor{currentstroke}{rgb}{0.000000,0.000000,0.000000}%
\pgfsetstrokecolor{currentstroke}%
\pgfsetdash{}{0pt}%
\pgfpathmoveto{\pgfqpoint{3.186126in}{2.961388in}}%
\pgfpathlineto{\pgfqpoint{1.915196in}{0.958373in}}%
\pgfpathlineto{\pgfqpoint{3.186126in}{2.961388in}}%
\pgfpathclose%
\pgfusepath{fill}%
\end{pgfscope}%
\begin{pgfscope}%
\pgfpathrectangle{\pgfqpoint{0.017500in}{0.017500in}}{\pgfqpoint{6.218282in}{3.580372in}} %
\pgfusepath{clip}%
\pgfsetbuttcap%
\pgfsetroundjoin%
\definecolor{currentfill}{rgb}{0.902849,0.844796,0.811970}%
\pgfsetfillcolor{currentfill}%
\pgfsetlinewidth{0.000000pt}%
\definecolor{currentstroke}{rgb}{0.000000,0.000000,0.000000}%
\pgfsetstrokecolor{currentstroke}%
\pgfsetdash{}{0pt}%
\pgfpathmoveto{\pgfqpoint{3.186126in}{2.961388in}}%
\pgfpathlineto{\pgfqpoint{1.915196in}{0.958373in}}%
\pgfpathlineto{\pgfqpoint{3.186126in}{2.961388in}}%
\pgfpathclose%
\pgfusepath{fill}%
\end{pgfscope}%
\begin{pgfscope}%
\pgfpathrectangle{\pgfqpoint{0.017500in}{0.017500in}}{\pgfqpoint{6.218282in}{3.580372in}} %
\pgfusepath{clip}%
\pgfsetbuttcap%
\pgfsetroundjoin%
\definecolor{currentfill}{rgb}{0.902849,0.844796,0.811970}%
\pgfsetfillcolor{currentfill}%
\pgfsetlinewidth{0.000000pt}%
\definecolor{currentstroke}{rgb}{0.000000,0.000000,0.000000}%
\pgfsetstrokecolor{currentstroke}%
\pgfsetdash{}{0pt}%
\pgfpathmoveto{\pgfqpoint{3.186126in}{2.961388in}}%
\pgfpathlineto{\pgfqpoint{1.915196in}{0.958373in}}%
\pgfpathlineto{\pgfqpoint{3.186126in}{2.961388in}}%
\pgfpathclose%
\pgfusepath{fill}%
\end{pgfscope}%
\begin{pgfscope}%
\pgfpathrectangle{\pgfqpoint{0.017500in}{0.017500in}}{\pgfqpoint{6.218282in}{3.580372in}} %
\pgfusepath{clip}%
\pgfsetbuttcap%
\pgfsetroundjoin%
\definecolor{currentfill}{rgb}{0.902849,0.844796,0.811970}%
\pgfsetfillcolor{currentfill}%
\pgfsetlinewidth{0.000000pt}%
\definecolor{currentstroke}{rgb}{0.000000,0.000000,0.000000}%
\pgfsetstrokecolor{currentstroke}%
\pgfsetdash{}{0pt}%
\pgfpathmoveto{\pgfqpoint{3.186126in}{2.961388in}}%
\pgfpathlineto{\pgfqpoint{1.915196in}{0.958373in}}%
\pgfpathlineto{\pgfqpoint{3.186126in}{2.961388in}}%
\pgfpathclose%
\pgfusepath{fill}%
\end{pgfscope}%
\begin{pgfscope}%
\pgfpathrectangle{\pgfqpoint{0.017500in}{0.017500in}}{\pgfqpoint{6.218282in}{3.580372in}} %
\pgfusepath{clip}%
\pgfsetbuttcap%
\pgfsetroundjoin%
\definecolor{currentfill}{rgb}{0.902849,0.844796,0.811970}%
\pgfsetfillcolor{currentfill}%
\pgfsetlinewidth{0.000000pt}%
\definecolor{currentstroke}{rgb}{0.000000,0.000000,0.000000}%
\pgfsetstrokecolor{currentstroke}%
\pgfsetdash{}{0pt}%
\pgfpathmoveto{\pgfqpoint{3.186126in}{2.961388in}}%
\pgfpathlineto{\pgfqpoint{1.915196in}{0.958373in}}%
\pgfpathlineto{\pgfqpoint{3.186126in}{2.961388in}}%
\pgfpathclose%
\pgfusepath{fill}%
\end{pgfscope}%
\begin{pgfscope}%
\pgfpathrectangle{\pgfqpoint{0.017500in}{0.017500in}}{\pgfqpoint{6.218282in}{3.580372in}} %
\pgfusepath{clip}%
\pgfsetbuttcap%
\pgfsetroundjoin%
\definecolor{currentfill}{rgb}{0.902849,0.844796,0.811970}%
\pgfsetfillcolor{currentfill}%
\pgfsetlinewidth{0.000000pt}%
\definecolor{currentstroke}{rgb}{0.000000,0.000000,0.000000}%
\pgfsetstrokecolor{currentstroke}%
\pgfsetdash{}{0pt}%
\pgfpathmoveto{\pgfqpoint{3.186126in}{2.961388in}}%
\pgfpathlineto{\pgfqpoint{1.915196in}{0.958373in}}%
\pgfpathlineto{\pgfqpoint{3.186126in}{2.961388in}}%
\pgfpathclose%
\pgfusepath{fill}%
\end{pgfscope}%
\begin{pgfscope}%
\pgfpathrectangle{\pgfqpoint{0.017500in}{0.017500in}}{\pgfqpoint{6.218282in}{3.580372in}} %
\pgfusepath{clip}%
\pgfsetbuttcap%
\pgfsetroundjoin%
\definecolor{currentfill}{rgb}{0.717435,0.051118,0.158737}%
\pgfsetfillcolor{currentfill}%
\pgfsetlinewidth{0.000000pt}%
\definecolor{currentstroke}{rgb}{0.000000,0.000000,0.000000}%
\pgfsetstrokecolor{currentstroke}%
\pgfsetdash{}{0pt}%
\pgfpathmoveto{\pgfqpoint{2.932971in}{2.820255in}}%
\pgfpathlineto{\pgfqpoint{3.186126in}{2.961388in}}%
\pgfpathlineto{\pgfqpoint{2.932971in}{2.820255in}}%
\pgfpathclose%
\pgfusepath{fill}%
\end{pgfscope}%
\begin{pgfscope}%
\pgfpathrectangle{\pgfqpoint{0.017500in}{0.017500in}}{\pgfqpoint{6.218282in}{3.580372in}} %
\pgfusepath{clip}%
\pgfsetbuttcap%
\pgfsetroundjoin%
\definecolor{currentfill}{rgb}{0.717435,0.051118,0.158737}%
\pgfsetfillcolor{currentfill}%
\pgfsetlinewidth{0.000000pt}%
\definecolor{currentstroke}{rgb}{0.000000,0.000000,0.000000}%
\pgfsetstrokecolor{currentstroke}%
\pgfsetdash{}{0pt}%
\pgfpathmoveto{\pgfqpoint{2.932971in}{2.820255in}}%
\pgfpathlineto{\pgfqpoint{3.186126in}{2.961388in}}%
\pgfpathlineto{\pgfqpoint{2.932971in}{2.820255in}}%
\pgfpathclose%
\pgfusepath{fill}%
\end{pgfscope}%
\begin{pgfscope}%
\pgfpathrectangle{\pgfqpoint{0.017500in}{0.017500in}}{\pgfqpoint{6.218282in}{3.580372in}} %
\pgfusepath{clip}%
\pgfsetbuttcap%
\pgfsetroundjoin%
\definecolor{currentfill}{rgb}{0.717435,0.051118,0.158737}%
\pgfsetfillcolor{currentfill}%
\pgfsetlinewidth{0.000000pt}%
\definecolor{currentstroke}{rgb}{0.000000,0.000000,0.000000}%
\pgfsetstrokecolor{currentstroke}%
\pgfsetdash{}{0pt}%
\pgfpathmoveto{\pgfqpoint{2.932971in}{2.820255in}}%
\pgfpathlineto{\pgfqpoint{3.186126in}{2.961388in}}%
\pgfpathlineto{\pgfqpoint{2.932971in}{2.820255in}}%
\pgfpathclose%
\pgfusepath{fill}%
\end{pgfscope}%
\begin{pgfscope}%
\pgfpathrectangle{\pgfqpoint{0.017500in}{0.017500in}}{\pgfqpoint{6.218282in}{3.580372in}} %
\pgfusepath{clip}%
\pgfsetbuttcap%
\pgfsetroundjoin%
\definecolor{currentfill}{rgb}{0.717435,0.051118,0.158737}%
\pgfsetfillcolor{currentfill}%
\pgfsetlinewidth{0.000000pt}%
\definecolor{currentstroke}{rgb}{0.000000,0.000000,0.000000}%
\pgfsetstrokecolor{currentstroke}%
\pgfsetdash{}{0pt}%
\pgfpathmoveto{\pgfqpoint{2.932971in}{2.820255in}}%
\pgfpathlineto{\pgfqpoint{3.186126in}{2.961388in}}%
\pgfpathlineto{\pgfqpoint{2.932971in}{2.820255in}}%
\pgfpathclose%
\pgfusepath{fill}%
\end{pgfscope}%
\begin{pgfscope}%
\pgfpathrectangle{\pgfqpoint{0.017500in}{0.017500in}}{\pgfqpoint{6.218282in}{3.580372in}} %
\pgfusepath{clip}%
\pgfsetbuttcap%
\pgfsetroundjoin%
\definecolor{currentfill}{rgb}{0.717435,0.051118,0.158737}%
\pgfsetfillcolor{currentfill}%
\pgfsetlinewidth{0.000000pt}%
\definecolor{currentstroke}{rgb}{0.000000,0.000000,0.000000}%
\pgfsetstrokecolor{currentstroke}%
\pgfsetdash{}{0pt}%
\pgfpathmoveto{\pgfqpoint{2.932971in}{2.820255in}}%
\pgfpathlineto{\pgfqpoint{3.186126in}{2.961388in}}%
\pgfpathlineto{\pgfqpoint{2.932971in}{2.820255in}}%
\pgfpathclose%
\pgfusepath{fill}%
\end{pgfscope}%
\begin{pgfscope}%
\pgfpathrectangle{\pgfqpoint{0.017500in}{0.017500in}}{\pgfqpoint{6.218282in}{3.580372in}} %
\pgfusepath{clip}%
\pgfsetbuttcap%
\pgfsetroundjoin%
\definecolor{currentfill}{rgb}{0.717435,0.051118,0.158737}%
\pgfsetfillcolor{currentfill}%
\pgfsetlinewidth{0.000000pt}%
\definecolor{currentstroke}{rgb}{0.000000,0.000000,0.000000}%
\pgfsetstrokecolor{currentstroke}%
\pgfsetdash{}{0pt}%
\pgfpathmoveto{\pgfqpoint{2.932971in}{2.820255in}}%
\pgfpathlineto{\pgfqpoint{3.186126in}{2.961388in}}%
\pgfpathlineto{\pgfqpoint{2.932971in}{2.820255in}}%
\pgfpathclose%
\pgfusepath{fill}%
\end{pgfscope}%
\begin{pgfscope}%
\pgfpathrectangle{\pgfqpoint{0.017500in}{0.017500in}}{\pgfqpoint{6.218282in}{3.580372in}} %
\pgfusepath{clip}%
\pgfsetbuttcap%
\pgfsetroundjoin%
\definecolor{currentfill}{rgb}{0.717435,0.051118,0.158737}%
\pgfsetfillcolor{currentfill}%
\pgfsetlinewidth{0.000000pt}%
\definecolor{currentstroke}{rgb}{0.000000,0.000000,0.000000}%
\pgfsetstrokecolor{currentstroke}%
\pgfsetdash{}{0pt}%
\pgfpathmoveto{\pgfqpoint{2.932971in}{2.820255in}}%
\pgfpathlineto{\pgfqpoint{3.186126in}{2.961388in}}%
\pgfpathlineto{\pgfqpoint{2.932971in}{2.820255in}}%
\pgfpathclose%
\pgfusepath{fill}%
\end{pgfscope}%
\begin{pgfscope}%
\pgfpathrectangle{\pgfqpoint{0.017500in}{0.017500in}}{\pgfqpoint{6.218282in}{3.580372in}} %
\pgfusepath{clip}%
\pgfsetbuttcap%
\pgfsetroundjoin%
\definecolor{currentfill}{rgb}{0.717435,0.051118,0.158737}%
\pgfsetfillcolor{currentfill}%
\pgfsetlinewidth{0.000000pt}%
\definecolor{currentstroke}{rgb}{0.000000,0.000000,0.000000}%
\pgfsetstrokecolor{currentstroke}%
\pgfsetdash{}{0pt}%
\pgfpathmoveto{\pgfqpoint{2.932971in}{2.820255in}}%
\pgfpathlineto{\pgfqpoint{3.186126in}{2.961388in}}%
\pgfpathlineto{\pgfqpoint{2.932971in}{2.820255in}}%
\pgfpathclose%
\pgfusepath{fill}%
\end{pgfscope}%
\begin{pgfscope}%
\pgfpathrectangle{\pgfqpoint{0.017500in}{0.017500in}}{\pgfqpoint{6.218282in}{3.580372in}} %
\pgfusepath{clip}%
\pgfsetbuttcap%
\pgfsetroundjoin%
\definecolor{currentfill}{rgb}{0.717435,0.051118,0.158737}%
\pgfsetfillcolor{currentfill}%
\pgfsetlinewidth{0.000000pt}%
\definecolor{currentstroke}{rgb}{0.000000,0.000000,0.000000}%
\pgfsetstrokecolor{currentstroke}%
\pgfsetdash{}{0pt}%
\pgfpathmoveto{\pgfqpoint{2.932971in}{2.820255in}}%
\pgfpathlineto{\pgfqpoint{3.186126in}{2.961388in}}%
\pgfpathlineto{\pgfqpoint{2.932971in}{2.820255in}}%
\pgfpathclose%
\pgfusepath{fill}%
\end{pgfscope}%
\begin{pgfscope}%
\pgfpathrectangle{\pgfqpoint{0.017500in}{0.017500in}}{\pgfqpoint{6.218282in}{3.580372in}} %
\pgfusepath{clip}%
\pgfsetbuttcap%
\pgfsetroundjoin%
\definecolor{currentfill}{rgb}{0.717435,0.051118,0.158737}%
\pgfsetfillcolor{currentfill}%
\pgfsetlinewidth{0.000000pt}%
\definecolor{currentstroke}{rgb}{0.000000,0.000000,0.000000}%
\pgfsetstrokecolor{currentstroke}%
\pgfsetdash{}{0pt}%
\pgfpathmoveto{\pgfqpoint{2.932971in}{2.820255in}}%
\pgfpathlineto{\pgfqpoint{3.186126in}{2.961388in}}%
\pgfpathlineto{\pgfqpoint{2.932971in}{2.820255in}}%
\pgfpathclose%
\pgfusepath{fill}%
\end{pgfscope}%
\begin{pgfscope}%
\pgfpathrectangle{\pgfqpoint{0.017500in}{0.017500in}}{\pgfqpoint{6.218282in}{3.580372in}} %
\pgfusepath{clip}%
\pgfsetbuttcap%
\pgfsetroundjoin%
\definecolor{currentfill}{rgb}{0.717435,0.051118,0.158737}%
\pgfsetfillcolor{currentfill}%
\pgfsetlinewidth{0.000000pt}%
\definecolor{currentstroke}{rgb}{0.000000,0.000000,0.000000}%
\pgfsetstrokecolor{currentstroke}%
\pgfsetdash{}{0pt}%
\pgfpathmoveto{\pgfqpoint{2.932971in}{2.820255in}}%
\pgfpathlineto{\pgfqpoint{3.186126in}{2.961388in}}%
\pgfpathlineto{\pgfqpoint{2.932971in}{2.820255in}}%
\pgfpathclose%
\pgfusepath{fill}%
\end{pgfscope}%
\begin{pgfscope}%
\pgfpathrectangle{\pgfqpoint{0.017500in}{0.017500in}}{\pgfqpoint{6.218282in}{3.580372in}} %
\pgfusepath{clip}%
\pgfsetbuttcap%
\pgfsetroundjoin%
\definecolor{currentfill}{rgb}{0.717435,0.051118,0.158737}%
\pgfsetfillcolor{currentfill}%
\pgfsetlinewidth{0.000000pt}%
\definecolor{currentstroke}{rgb}{0.000000,0.000000,0.000000}%
\pgfsetstrokecolor{currentstroke}%
\pgfsetdash{}{0pt}%
\pgfpathmoveto{\pgfqpoint{2.932971in}{2.820255in}}%
\pgfpathlineto{\pgfqpoint{3.186126in}{2.961388in}}%
\pgfpathlineto{\pgfqpoint{2.932971in}{2.820255in}}%
\pgfpathclose%
\pgfusepath{fill}%
\end{pgfscope}%
\begin{pgfscope}%
\pgfpathrectangle{\pgfqpoint{0.017500in}{0.017500in}}{\pgfqpoint{6.218282in}{3.580372in}} %
\pgfusepath{clip}%
\pgfsetbuttcap%
\pgfsetroundjoin%
\definecolor{currentfill}{rgb}{0.717435,0.051118,0.158737}%
\pgfsetfillcolor{currentfill}%
\pgfsetlinewidth{0.000000pt}%
\definecolor{currentstroke}{rgb}{0.000000,0.000000,0.000000}%
\pgfsetstrokecolor{currentstroke}%
\pgfsetdash{}{0pt}%
\pgfpathmoveto{\pgfqpoint{2.932971in}{2.820255in}}%
\pgfpathlineto{\pgfqpoint{3.186126in}{2.961388in}}%
\pgfpathlineto{\pgfqpoint{2.932971in}{2.820255in}}%
\pgfpathclose%
\pgfusepath{fill}%
\end{pgfscope}%
\begin{pgfscope}%
\pgfpathrectangle{\pgfqpoint{0.017500in}{0.017500in}}{\pgfqpoint{6.218282in}{3.580372in}} %
\pgfusepath{clip}%
\pgfsetbuttcap%
\pgfsetroundjoin%
\definecolor{currentfill}{rgb}{0.717435,0.051118,0.158737}%
\pgfsetfillcolor{currentfill}%
\pgfsetlinewidth{0.000000pt}%
\definecolor{currentstroke}{rgb}{0.000000,0.000000,0.000000}%
\pgfsetstrokecolor{currentstroke}%
\pgfsetdash{}{0pt}%
\pgfpathmoveto{\pgfqpoint{2.932971in}{2.820255in}}%
\pgfpathlineto{\pgfqpoint{3.186126in}{2.961388in}}%
\pgfpathlineto{\pgfqpoint{2.932971in}{2.820255in}}%
\pgfpathclose%
\pgfusepath{fill}%
\end{pgfscope}%
\begin{pgfscope}%
\pgfpathrectangle{\pgfqpoint{0.017500in}{0.017500in}}{\pgfqpoint{6.218282in}{3.580372in}} %
\pgfusepath{clip}%
\pgfsetbuttcap%
\pgfsetroundjoin%
\definecolor{currentfill}{rgb}{0.717435,0.051118,0.158737}%
\pgfsetfillcolor{currentfill}%
\pgfsetlinewidth{0.000000pt}%
\definecolor{currentstroke}{rgb}{0.000000,0.000000,0.000000}%
\pgfsetstrokecolor{currentstroke}%
\pgfsetdash{}{0pt}%
\pgfpathmoveto{\pgfqpoint{2.932971in}{2.820255in}}%
\pgfpathlineto{\pgfqpoint{3.186126in}{2.961388in}}%
\pgfpathlineto{\pgfqpoint{2.932971in}{2.820255in}}%
\pgfpathclose%
\pgfusepath{fill}%
\end{pgfscope}%
\begin{pgfscope}%
\pgfpathrectangle{\pgfqpoint{0.017500in}{0.017500in}}{\pgfqpoint{6.218282in}{3.580372in}} %
\pgfusepath{clip}%
\pgfsetbuttcap%
\pgfsetroundjoin%
\definecolor{currentfill}{rgb}{0.717435,0.051118,0.158737}%
\pgfsetfillcolor{currentfill}%
\pgfsetlinewidth{0.000000pt}%
\definecolor{currentstroke}{rgb}{0.000000,0.000000,0.000000}%
\pgfsetstrokecolor{currentstroke}%
\pgfsetdash{}{0pt}%
\pgfpathmoveto{\pgfqpoint{2.932971in}{2.820255in}}%
\pgfpathlineto{\pgfqpoint{3.186126in}{2.961388in}}%
\pgfpathlineto{\pgfqpoint{2.932971in}{2.820255in}}%
\pgfpathclose%
\pgfusepath{fill}%
\end{pgfscope}%
\begin{pgfscope}%
\pgfpathrectangle{\pgfqpoint{0.017500in}{0.017500in}}{\pgfqpoint{6.218282in}{3.580372in}} %
\pgfusepath{clip}%
\pgfsetbuttcap%
\pgfsetroundjoin%
\definecolor{currentfill}{rgb}{0.717435,0.051118,0.158737}%
\pgfsetfillcolor{currentfill}%
\pgfsetlinewidth{0.000000pt}%
\definecolor{currentstroke}{rgb}{0.000000,0.000000,0.000000}%
\pgfsetstrokecolor{currentstroke}%
\pgfsetdash{}{0pt}%
\pgfpathmoveto{\pgfqpoint{2.932971in}{2.820255in}}%
\pgfpathlineto{\pgfqpoint{3.186126in}{2.961388in}}%
\pgfpathlineto{\pgfqpoint{2.932971in}{2.820255in}}%
\pgfpathclose%
\pgfusepath{fill}%
\end{pgfscope}%
\begin{pgfscope}%
\pgfpathrectangle{\pgfqpoint{0.017500in}{0.017500in}}{\pgfqpoint{6.218282in}{3.580372in}} %
\pgfusepath{clip}%
\pgfsetbuttcap%
\pgfsetroundjoin%
\definecolor{currentfill}{rgb}{0.717435,0.051118,0.158737}%
\pgfsetfillcolor{currentfill}%
\pgfsetlinewidth{0.000000pt}%
\definecolor{currentstroke}{rgb}{0.000000,0.000000,0.000000}%
\pgfsetstrokecolor{currentstroke}%
\pgfsetdash{}{0pt}%
\pgfpathmoveto{\pgfqpoint{2.932971in}{2.820255in}}%
\pgfpathlineto{\pgfqpoint{3.186126in}{2.961388in}}%
\pgfpathlineto{\pgfqpoint{2.932971in}{2.820255in}}%
\pgfpathclose%
\pgfusepath{fill}%
\end{pgfscope}%
\begin{pgfscope}%
\pgfpathrectangle{\pgfqpoint{0.017500in}{0.017500in}}{\pgfqpoint{6.218282in}{3.580372in}} %
\pgfusepath{clip}%
\pgfsetbuttcap%
\pgfsetroundjoin%
\definecolor{currentfill}{rgb}{0.717435,0.051118,0.158737}%
\pgfsetfillcolor{currentfill}%
\pgfsetlinewidth{0.000000pt}%
\definecolor{currentstroke}{rgb}{0.000000,0.000000,0.000000}%
\pgfsetstrokecolor{currentstroke}%
\pgfsetdash{}{0pt}%
\pgfpathmoveto{\pgfqpoint{2.932971in}{2.820255in}}%
\pgfpathlineto{\pgfqpoint{3.186126in}{2.961388in}}%
\pgfpathlineto{\pgfqpoint{2.932971in}{2.820255in}}%
\pgfpathclose%
\pgfusepath{fill}%
\end{pgfscope}%
\begin{pgfscope}%
\pgfpathrectangle{\pgfqpoint{0.017500in}{0.017500in}}{\pgfqpoint{6.218282in}{3.580372in}} %
\pgfusepath{clip}%
\pgfsetbuttcap%
\pgfsetroundjoin%
\definecolor{currentfill}{rgb}{0.717435,0.051118,0.158737}%
\pgfsetfillcolor{currentfill}%
\pgfsetlinewidth{0.000000pt}%
\definecolor{currentstroke}{rgb}{0.000000,0.000000,0.000000}%
\pgfsetstrokecolor{currentstroke}%
\pgfsetdash{}{0pt}%
\pgfpathmoveto{\pgfqpoint{2.932971in}{2.820255in}}%
\pgfpathlineto{\pgfqpoint{3.186126in}{2.961388in}}%
\pgfpathlineto{\pgfqpoint{2.932971in}{2.820255in}}%
\pgfpathclose%
\pgfusepath{fill}%
\end{pgfscope}%
\begin{pgfscope}%
\pgfpathrectangle{\pgfqpoint{0.017500in}{0.017500in}}{\pgfqpoint{6.218282in}{3.580372in}} %
\pgfusepath{clip}%
\pgfsetbuttcap%
\pgfsetroundjoin%
\definecolor{currentfill}{rgb}{0.717435,0.051118,0.158737}%
\pgfsetfillcolor{currentfill}%
\pgfsetlinewidth{0.000000pt}%
\definecolor{currentstroke}{rgb}{0.000000,0.000000,0.000000}%
\pgfsetstrokecolor{currentstroke}%
\pgfsetdash{}{0pt}%
\pgfpathmoveto{\pgfqpoint{2.932971in}{2.820255in}}%
\pgfpathlineto{\pgfqpoint{3.186126in}{2.961388in}}%
\pgfpathlineto{\pgfqpoint{2.932971in}{2.820255in}}%
\pgfpathclose%
\pgfusepath{fill}%
\end{pgfscope}%
\begin{pgfscope}%
\pgfpathrectangle{\pgfqpoint{0.017500in}{0.017500in}}{\pgfqpoint{6.218282in}{3.580372in}} %
\pgfusepath{clip}%
\pgfsetbuttcap%
\pgfsetroundjoin%
\definecolor{currentfill}{rgb}{0.717435,0.051118,0.158737}%
\pgfsetfillcolor{currentfill}%
\pgfsetlinewidth{0.000000pt}%
\definecolor{currentstroke}{rgb}{0.000000,0.000000,0.000000}%
\pgfsetstrokecolor{currentstroke}%
\pgfsetdash{}{0pt}%
\pgfpathmoveto{\pgfqpoint{2.932971in}{2.820255in}}%
\pgfpathlineto{\pgfqpoint{3.186126in}{2.961388in}}%
\pgfpathlineto{\pgfqpoint{2.932971in}{2.820255in}}%
\pgfpathclose%
\pgfusepath{fill}%
\end{pgfscope}%
\begin{pgfscope}%
\pgfpathrectangle{\pgfqpoint{0.017500in}{0.017500in}}{\pgfqpoint{6.218282in}{3.580372in}} %
\pgfusepath{clip}%
\pgfsetbuttcap%
\pgfsetroundjoin%
\definecolor{currentfill}{rgb}{0.717435,0.051118,0.158737}%
\pgfsetfillcolor{currentfill}%
\pgfsetlinewidth{0.000000pt}%
\definecolor{currentstroke}{rgb}{0.000000,0.000000,0.000000}%
\pgfsetstrokecolor{currentstroke}%
\pgfsetdash{}{0pt}%
\pgfpathmoveto{\pgfqpoint{2.932971in}{2.820255in}}%
\pgfpathlineto{\pgfqpoint{3.186126in}{2.961388in}}%
\pgfpathlineto{\pgfqpoint{2.932971in}{2.820255in}}%
\pgfpathclose%
\pgfusepath{fill}%
\end{pgfscope}%
\begin{pgfscope}%
\pgfpathrectangle{\pgfqpoint{0.017500in}{0.017500in}}{\pgfqpoint{6.218282in}{3.580372in}} %
\pgfusepath{clip}%
\pgfsetbuttcap%
\pgfsetroundjoin%
\definecolor{currentfill}{rgb}{0.717435,0.051118,0.158737}%
\pgfsetfillcolor{currentfill}%
\pgfsetlinewidth{0.000000pt}%
\definecolor{currentstroke}{rgb}{0.000000,0.000000,0.000000}%
\pgfsetstrokecolor{currentstroke}%
\pgfsetdash{}{0pt}%
\pgfpathmoveto{\pgfqpoint{2.932971in}{2.820255in}}%
\pgfpathlineto{\pgfqpoint{3.186126in}{2.961388in}}%
\pgfpathlineto{\pgfqpoint{2.932971in}{2.820255in}}%
\pgfpathclose%
\pgfusepath{fill}%
\end{pgfscope}%
\begin{pgfscope}%
\pgfpathrectangle{\pgfqpoint{0.017500in}{0.017500in}}{\pgfqpoint{6.218282in}{3.580372in}} %
\pgfusepath{clip}%
\pgfsetbuttcap%
\pgfsetroundjoin%
\definecolor{currentfill}{rgb}{0.717435,0.051118,0.158737}%
\pgfsetfillcolor{currentfill}%
\pgfsetlinewidth{0.000000pt}%
\definecolor{currentstroke}{rgb}{0.000000,0.000000,0.000000}%
\pgfsetstrokecolor{currentstroke}%
\pgfsetdash{}{0pt}%
\pgfpathmoveto{\pgfqpoint{2.932971in}{2.820255in}}%
\pgfpathlineto{\pgfqpoint{3.186126in}{2.961388in}}%
\pgfpathlineto{\pgfqpoint{2.932971in}{2.820255in}}%
\pgfpathclose%
\pgfusepath{fill}%
\end{pgfscope}%
\begin{pgfscope}%
\pgfpathrectangle{\pgfqpoint{0.017500in}{0.017500in}}{\pgfqpoint{6.218282in}{3.580372in}} %
\pgfusepath{clip}%
\pgfsetbuttcap%
\pgfsetroundjoin%
\definecolor{currentfill}{rgb}{0.717435,0.051118,0.158737}%
\pgfsetfillcolor{currentfill}%
\pgfsetlinewidth{0.000000pt}%
\definecolor{currentstroke}{rgb}{0.000000,0.000000,0.000000}%
\pgfsetstrokecolor{currentstroke}%
\pgfsetdash{}{0pt}%
\pgfpathmoveto{\pgfqpoint{2.932971in}{2.820255in}}%
\pgfpathlineto{\pgfqpoint{3.186126in}{2.961388in}}%
\pgfpathlineto{\pgfqpoint{2.932971in}{2.820255in}}%
\pgfpathclose%
\pgfusepath{fill}%
\end{pgfscope}%
\begin{pgfscope}%
\pgfpathrectangle{\pgfqpoint{0.017500in}{0.017500in}}{\pgfqpoint{6.218282in}{3.580372in}} %
\pgfusepath{clip}%
\pgfsetbuttcap%
\pgfsetroundjoin%
\definecolor{currentfill}{rgb}{0.717435,0.051118,0.158737}%
\pgfsetfillcolor{currentfill}%
\pgfsetlinewidth{0.000000pt}%
\definecolor{currentstroke}{rgb}{0.000000,0.000000,0.000000}%
\pgfsetstrokecolor{currentstroke}%
\pgfsetdash{}{0pt}%
\pgfpathmoveto{\pgfqpoint{2.932971in}{2.820255in}}%
\pgfpathlineto{\pgfqpoint{3.186126in}{2.961388in}}%
\pgfpathlineto{\pgfqpoint{2.932971in}{2.820255in}}%
\pgfpathclose%
\pgfusepath{fill}%
\end{pgfscope}%
\begin{pgfscope}%
\pgfpathrectangle{\pgfqpoint{0.017500in}{0.017500in}}{\pgfqpoint{6.218282in}{3.580372in}} %
\pgfusepath{clip}%
\pgfsetbuttcap%
\pgfsetroundjoin%
\definecolor{currentfill}{rgb}{0.717435,0.051118,0.158737}%
\pgfsetfillcolor{currentfill}%
\pgfsetlinewidth{0.000000pt}%
\definecolor{currentstroke}{rgb}{0.000000,0.000000,0.000000}%
\pgfsetstrokecolor{currentstroke}%
\pgfsetdash{}{0pt}%
\pgfpathmoveto{\pgfqpoint{2.932971in}{2.820255in}}%
\pgfpathlineto{\pgfqpoint{3.186126in}{2.961388in}}%
\pgfpathlineto{\pgfqpoint{2.932971in}{2.820255in}}%
\pgfpathclose%
\pgfusepath{fill}%
\end{pgfscope}%
\begin{pgfscope}%
\pgfpathrectangle{\pgfqpoint{0.017500in}{0.017500in}}{\pgfqpoint{6.218282in}{3.580372in}} %
\pgfusepath{clip}%
\pgfsetbuttcap%
\pgfsetroundjoin%
\definecolor{currentfill}{rgb}{0.717435,0.051118,0.158737}%
\pgfsetfillcolor{currentfill}%
\pgfsetlinewidth{0.000000pt}%
\definecolor{currentstroke}{rgb}{0.000000,0.000000,0.000000}%
\pgfsetstrokecolor{currentstroke}%
\pgfsetdash{}{0pt}%
\pgfpathmoveto{\pgfqpoint{2.932971in}{2.820255in}}%
\pgfpathlineto{\pgfqpoint{3.186126in}{2.961388in}}%
\pgfpathlineto{\pgfqpoint{2.932971in}{2.820255in}}%
\pgfpathclose%
\pgfusepath{fill}%
\end{pgfscope}%
\begin{pgfscope}%
\pgfpathrectangle{\pgfqpoint{0.017500in}{0.017500in}}{\pgfqpoint{6.218282in}{3.580372in}} %
\pgfusepath{clip}%
\pgfsetbuttcap%
\pgfsetroundjoin%
\definecolor{currentfill}{rgb}{0.717435,0.051118,0.158737}%
\pgfsetfillcolor{currentfill}%
\pgfsetlinewidth{0.000000pt}%
\definecolor{currentstroke}{rgb}{0.000000,0.000000,0.000000}%
\pgfsetstrokecolor{currentstroke}%
\pgfsetdash{}{0pt}%
\pgfpathmoveto{\pgfqpoint{2.932971in}{2.820255in}}%
\pgfpathlineto{\pgfqpoint{3.186126in}{2.961388in}}%
\pgfpathlineto{\pgfqpoint{2.932971in}{2.820255in}}%
\pgfpathclose%
\pgfusepath{fill}%
\end{pgfscope}%
\begin{pgfscope}%
\pgfpathrectangle{\pgfqpoint{0.017500in}{0.017500in}}{\pgfqpoint{6.218282in}{3.580372in}} %
\pgfusepath{clip}%
\pgfsetbuttcap%
\pgfsetroundjoin%
\definecolor{currentfill}{rgb}{0.804965,0.851666,0.926165}%
\pgfsetfillcolor{currentfill}%
\pgfsetlinewidth{0.000000pt}%
\definecolor{currentstroke}{rgb}{0.000000,0.000000,0.000000}%
\pgfsetstrokecolor{currentstroke}%
\pgfsetdash{}{0pt}%
\pgfpathmoveto{\pgfqpoint{3.186126in}{2.961388in}}%
\pgfpathlineto{\pgfqpoint{3.622269in}{2.896533in}}%
\pgfpathlineto{\pgfqpoint{2.349703in}{0.878379in}}%
\pgfpathlineto{\pgfqpoint{1.915196in}{0.958373in}}%
\pgfpathlineto{\pgfqpoint{3.186126in}{2.961388in}}%
\pgfpathclose%
\pgfusepath{fill}%
\end{pgfscope}%
\begin{pgfscope}%
\pgfpathrectangle{\pgfqpoint{0.017500in}{0.017500in}}{\pgfqpoint{6.218282in}{3.580372in}} %
\pgfusepath{clip}%
\pgfsetbuttcap%
\pgfsetroundjoin%
\definecolor{currentfill}{rgb}{0.804965,0.851666,0.926165}%
\pgfsetfillcolor{currentfill}%
\pgfsetlinewidth{0.000000pt}%
\definecolor{currentstroke}{rgb}{0.000000,0.000000,0.000000}%
\pgfsetstrokecolor{currentstroke}%
\pgfsetdash{}{0pt}%
\pgfpathmoveto{\pgfqpoint{3.186126in}{2.961388in}}%
\pgfpathlineto{\pgfqpoint{3.622269in}{2.896533in}}%
\pgfpathlineto{\pgfqpoint{2.349703in}{0.878379in}}%
\pgfpathlineto{\pgfqpoint{1.915196in}{0.958373in}}%
\pgfpathlineto{\pgfqpoint{3.186126in}{2.961388in}}%
\pgfpathclose%
\pgfusepath{fill}%
\end{pgfscope}%
\begin{pgfscope}%
\pgfpathrectangle{\pgfqpoint{0.017500in}{0.017500in}}{\pgfqpoint{6.218282in}{3.580372in}} %
\pgfusepath{clip}%
\pgfsetbuttcap%
\pgfsetroundjoin%
\definecolor{currentfill}{rgb}{0.804965,0.851666,0.926165}%
\pgfsetfillcolor{currentfill}%
\pgfsetlinewidth{0.000000pt}%
\definecolor{currentstroke}{rgb}{0.000000,0.000000,0.000000}%
\pgfsetstrokecolor{currentstroke}%
\pgfsetdash{}{0pt}%
\pgfpathmoveto{\pgfqpoint{3.186126in}{2.961388in}}%
\pgfpathlineto{\pgfqpoint{3.622269in}{2.896533in}}%
\pgfpathlineto{\pgfqpoint{2.349703in}{0.878379in}}%
\pgfpathlineto{\pgfqpoint{1.915196in}{0.958373in}}%
\pgfpathlineto{\pgfqpoint{3.186126in}{2.961388in}}%
\pgfpathclose%
\pgfusepath{fill}%
\end{pgfscope}%
\begin{pgfscope}%
\pgfpathrectangle{\pgfqpoint{0.017500in}{0.017500in}}{\pgfqpoint{6.218282in}{3.580372in}} %
\pgfusepath{clip}%
\pgfsetbuttcap%
\pgfsetroundjoin%
\definecolor{currentfill}{rgb}{0.804965,0.851666,0.926165}%
\pgfsetfillcolor{currentfill}%
\pgfsetlinewidth{0.000000pt}%
\definecolor{currentstroke}{rgb}{0.000000,0.000000,0.000000}%
\pgfsetstrokecolor{currentstroke}%
\pgfsetdash{}{0pt}%
\pgfpathmoveto{\pgfqpoint{3.186126in}{2.961388in}}%
\pgfpathlineto{\pgfqpoint{3.622269in}{2.896533in}}%
\pgfpathlineto{\pgfqpoint{2.349703in}{0.878379in}}%
\pgfpathlineto{\pgfqpoint{1.915196in}{0.958373in}}%
\pgfpathlineto{\pgfqpoint{3.186126in}{2.961388in}}%
\pgfpathclose%
\pgfusepath{fill}%
\end{pgfscope}%
\begin{pgfscope}%
\pgfpathrectangle{\pgfqpoint{0.017500in}{0.017500in}}{\pgfqpoint{6.218282in}{3.580372in}} %
\pgfusepath{clip}%
\pgfsetbuttcap%
\pgfsetroundjoin%
\definecolor{currentfill}{rgb}{0.804965,0.851666,0.926165}%
\pgfsetfillcolor{currentfill}%
\pgfsetlinewidth{0.000000pt}%
\definecolor{currentstroke}{rgb}{0.000000,0.000000,0.000000}%
\pgfsetstrokecolor{currentstroke}%
\pgfsetdash{}{0pt}%
\pgfpathmoveto{\pgfqpoint{3.186126in}{2.961388in}}%
\pgfpathlineto{\pgfqpoint{3.622269in}{2.896533in}}%
\pgfpathlineto{\pgfqpoint{2.349703in}{0.878379in}}%
\pgfpathlineto{\pgfqpoint{1.915196in}{0.958373in}}%
\pgfpathlineto{\pgfqpoint{3.186126in}{2.961388in}}%
\pgfpathclose%
\pgfusepath{fill}%
\end{pgfscope}%
\begin{pgfscope}%
\pgfpathrectangle{\pgfqpoint{0.017500in}{0.017500in}}{\pgfqpoint{6.218282in}{3.580372in}} %
\pgfusepath{clip}%
\pgfsetbuttcap%
\pgfsetroundjoin%
\definecolor{currentfill}{rgb}{0.711554,0.033337,0.154485}%
\pgfsetfillcolor{currentfill}%
\pgfsetlinewidth{0.000000pt}%
\definecolor{currentstroke}{rgb}{0.000000,0.000000,0.000000}%
\pgfsetstrokecolor{currentstroke}%
\pgfsetdash{}{0pt}%
\pgfpathmoveto{\pgfqpoint{2.932971in}{2.820255in}}%
\pgfpathlineto{\pgfqpoint{3.372089in}{2.760957in}}%
\pgfpathlineto{\pgfqpoint{3.622269in}{2.896533in}}%
\pgfpathlineto{\pgfqpoint{3.186126in}{2.961388in}}%
\pgfpathlineto{\pgfqpoint{2.932971in}{2.820255in}}%
\pgfpathclose%
\pgfusepath{fill}%
\end{pgfscope}%
\begin{pgfscope}%
\pgfpathrectangle{\pgfqpoint{0.017500in}{0.017500in}}{\pgfqpoint{6.218282in}{3.580372in}} %
\pgfusepath{clip}%
\pgfsetbuttcap%
\pgfsetroundjoin%
\definecolor{currentfill}{rgb}{0.711554,0.033337,0.154485}%
\pgfsetfillcolor{currentfill}%
\pgfsetlinewidth{0.000000pt}%
\definecolor{currentstroke}{rgb}{0.000000,0.000000,0.000000}%
\pgfsetstrokecolor{currentstroke}%
\pgfsetdash{}{0pt}%
\pgfpathmoveto{\pgfqpoint{2.932971in}{2.820255in}}%
\pgfpathlineto{\pgfqpoint{3.372089in}{2.760957in}}%
\pgfpathlineto{\pgfqpoint{3.622269in}{2.896533in}}%
\pgfpathlineto{\pgfqpoint{3.186126in}{2.961388in}}%
\pgfpathlineto{\pgfqpoint{2.932971in}{2.820255in}}%
\pgfpathclose%
\pgfusepath{fill}%
\end{pgfscope}%
\begin{pgfscope}%
\pgfpathrectangle{\pgfqpoint{0.017500in}{0.017500in}}{\pgfqpoint{6.218282in}{3.580372in}} %
\pgfusepath{clip}%
\pgfsetbuttcap%
\pgfsetroundjoin%
\definecolor{currentfill}{rgb}{0.711554,0.033337,0.154485}%
\pgfsetfillcolor{currentfill}%
\pgfsetlinewidth{0.000000pt}%
\definecolor{currentstroke}{rgb}{0.000000,0.000000,0.000000}%
\pgfsetstrokecolor{currentstroke}%
\pgfsetdash{}{0pt}%
\pgfpathmoveto{\pgfqpoint{2.932971in}{2.820255in}}%
\pgfpathlineto{\pgfqpoint{3.372089in}{2.760957in}}%
\pgfpathlineto{\pgfqpoint{3.622269in}{2.896533in}}%
\pgfpathlineto{\pgfqpoint{3.186126in}{2.961388in}}%
\pgfpathlineto{\pgfqpoint{2.932971in}{2.820255in}}%
\pgfpathclose%
\pgfusepath{fill}%
\end{pgfscope}%
\begin{pgfscope}%
\pgfpathrectangle{\pgfqpoint{0.017500in}{0.017500in}}{\pgfqpoint{6.218282in}{3.580372in}} %
\pgfusepath{clip}%
\pgfsetbuttcap%
\pgfsetroundjoin%
\definecolor{currentfill}{rgb}{0.711554,0.033337,0.154485}%
\pgfsetfillcolor{currentfill}%
\pgfsetlinewidth{0.000000pt}%
\definecolor{currentstroke}{rgb}{0.000000,0.000000,0.000000}%
\pgfsetstrokecolor{currentstroke}%
\pgfsetdash{}{0pt}%
\pgfpathmoveto{\pgfqpoint{2.932971in}{2.820255in}}%
\pgfpathlineto{\pgfqpoint{3.372089in}{2.760957in}}%
\pgfpathlineto{\pgfqpoint{3.622269in}{2.896533in}}%
\pgfpathlineto{\pgfqpoint{3.186126in}{2.961388in}}%
\pgfpathlineto{\pgfqpoint{2.932971in}{2.820255in}}%
\pgfpathclose%
\pgfusepath{fill}%
\end{pgfscope}%
\begin{pgfscope}%
\pgfpathrectangle{\pgfqpoint{0.017500in}{0.017500in}}{\pgfqpoint{6.218282in}{3.580372in}} %
\pgfusepath{clip}%
\pgfsetbuttcap%
\pgfsetroundjoin%
\definecolor{currentfill}{rgb}{0.711554,0.033337,0.154485}%
\pgfsetfillcolor{currentfill}%
\pgfsetlinewidth{0.000000pt}%
\definecolor{currentstroke}{rgb}{0.000000,0.000000,0.000000}%
\pgfsetstrokecolor{currentstroke}%
\pgfsetdash{}{0pt}%
\pgfpathmoveto{\pgfqpoint{2.932971in}{2.820255in}}%
\pgfpathlineto{\pgfqpoint{3.372089in}{2.760957in}}%
\pgfpathlineto{\pgfqpoint{3.622269in}{2.896533in}}%
\pgfpathlineto{\pgfqpoint{3.186126in}{2.961388in}}%
\pgfpathlineto{\pgfqpoint{2.932971in}{2.820255in}}%
\pgfpathclose%
\pgfusepath{fill}%
\end{pgfscope}%
\begin{pgfscope}%
\pgfpathrectangle{\pgfqpoint{0.017500in}{0.017500in}}{\pgfqpoint{6.218282in}{3.580372in}} %
\pgfusepath{clip}%
\pgfsetbuttcap%
\pgfsetroundjoin%
\definecolor{currentfill}{rgb}{0.711554,0.033337,0.154485}%
\pgfsetfillcolor{currentfill}%
\pgfsetlinewidth{0.000000pt}%
\definecolor{currentstroke}{rgb}{0.000000,0.000000,0.000000}%
\pgfsetstrokecolor{currentstroke}%
\pgfsetdash{}{0pt}%
\pgfpathmoveto{\pgfqpoint{2.932971in}{2.820255in}}%
\pgfpathlineto{\pgfqpoint{3.372089in}{2.760957in}}%
\pgfpathlineto{\pgfqpoint{3.622269in}{2.896533in}}%
\pgfpathlineto{\pgfqpoint{3.186126in}{2.961388in}}%
\pgfpathlineto{\pgfqpoint{2.932971in}{2.820255in}}%
\pgfpathclose%
\pgfusepath{fill}%
\end{pgfscope}%
\begin{pgfscope}%
\pgfpathrectangle{\pgfqpoint{0.017500in}{0.017500in}}{\pgfqpoint{6.218282in}{3.580372in}} %
\pgfusepath{clip}%
\pgfsetbuttcap%
\pgfsetroundjoin%
\definecolor{currentfill}{rgb}{0.758112,0.168122,0.188827}%
\pgfsetfillcolor{currentfill}%
\pgfsetlinewidth{0.000000pt}%
\definecolor{currentstroke}{rgb}{0.000000,0.000000,0.000000}%
\pgfsetstrokecolor{currentstroke}%
\pgfsetdash{}{0pt}%
\pgfpathmoveto{\pgfqpoint{2.238213in}{2.742590in}}%
\pgfpathlineto{\pgfqpoint{2.499580in}{2.886064in}}%
\pgfpathlineto{\pgfqpoint{2.238213in}{2.742590in}}%
\pgfpathclose%
\pgfusepath{fill}%
\end{pgfscope}%
\begin{pgfscope}%
\pgfpathrectangle{\pgfqpoint{0.017500in}{0.017500in}}{\pgfqpoint{6.218282in}{3.580372in}} %
\pgfusepath{clip}%
\pgfsetbuttcap%
\pgfsetroundjoin%
\definecolor{currentfill}{rgb}{0.758112,0.168122,0.188827}%
\pgfsetfillcolor{currentfill}%
\pgfsetlinewidth{0.000000pt}%
\definecolor{currentstroke}{rgb}{0.000000,0.000000,0.000000}%
\pgfsetstrokecolor{currentstroke}%
\pgfsetdash{}{0pt}%
\pgfpathmoveto{\pgfqpoint{2.238213in}{2.742590in}}%
\pgfpathlineto{\pgfqpoint{2.499580in}{2.886064in}}%
\pgfpathlineto{\pgfqpoint{2.238213in}{2.742590in}}%
\pgfpathclose%
\pgfusepath{fill}%
\end{pgfscope}%
\begin{pgfscope}%
\pgfpathrectangle{\pgfqpoint{0.017500in}{0.017500in}}{\pgfqpoint{6.218282in}{3.580372in}} %
\pgfusepath{clip}%
\pgfsetbuttcap%
\pgfsetroundjoin%
\definecolor{currentfill}{rgb}{0.758112,0.168122,0.188827}%
\pgfsetfillcolor{currentfill}%
\pgfsetlinewidth{0.000000pt}%
\definecolor{currentstroke}{rgb}{0.000000,0.000000,0.000000}%
\pgfsetstrokecolor{currentstroke}%
\pgfsetdash{}{0pt}%
\pgfpathmoveto{\pgfqpoint{2.238213in}{2.742590in}}%
\pgfpathlineto{\pgfqpoint{2.499580in}{2.886064in}}%
\pgfpathlineto{\pgfqpoint{2.238213in}{2.742590in}}%
\pgfpathclose%
\pgfusepath{fill}%
\end{pgfscope}%
\begin{pgfscope}%
\pgfpathrectangle{\pgfqpoint{0.017500in}{0.017500in}}{\pgfqpoint{6.218282in}{3.580372in}} %
\pgfusepath{clip}%
\pgfsetbuttcap%
\pgfsetroundjoin%
\definecolor{currentfill}{rgb}{0.758112,0.168122,0.188827}%
\pgfsetfillcolor{currentfill}%
\pgfsetlinewidth{0.000000pt}%
\definecolor{currentstroke}{rgb}{0.000000,0.000000,0.000000}%
\pgfsetstrokecolor{currentstroke}%
\pgfsetdash{}{0pt}%
\pgfpathmoveto{\pgfqpoint{2.238213in}{2.742590in}}%
\pgfpathlineto{\pgfqpoint{2.499580in}{2.886064in}}%
\pgfpathlineto{\pgfqpoint{2.238213in}{2.742590in}}%
\pgfpathclose%
\pgfusepath{fill}%
\end{pgfscope}%
\begin{pgfscope}%
\pgfpathrectangle{\pgfqpoint{0.017500in}{0.017500in}}{\pgfqpoint{6.218282in}{3.580372in}} %
\pgfusepath{clip}%
\pgfsetbuttcap%
\pgfsetroundjoin%
\definecolor{currentfill}{rgb}{0.758112,0.168122,0.188827}%
\pgfsetfillcolor{currentfill}%
\pgfsetlinewidth{0.000000pt}%
\definecolor{currentstroke}{rgb}{0.000000,0.000000,0.000000}%
\pgfsetstrokecolor{currentstroke}%
\pgfsetdash{}{0pt}%
\pgfpathmoveto{\pgfqpoint{2.238213in}{2.742590in}}%
\pgfpathlineto{\pgfqpoint{2.499580in}{2.886064in}}%
\pgfpathlineto{\pgfqpoint{2.238213in}{2.742590in}}%
\pgfpathclose%
\pgfusepath{fill}%
\end{pgfscope}%
\begin{pgfscope}%
\pgfpathrectangle{\pgfqpoint{0.017500in}{0.017500in}}{\pgfqpoint{6.218282in}{3.580372in}} %
\pgfusepath{clip}%
\pgfsetbuttcap%
\pgfsetroundjoin%
\definecolor{currentfill}{rgb}{0.758112,0.168122,0.188827}%
\pgfsetfillcolor{currentfill}%
\pgfsetlinewidth{0.000000pt}%
\definecolor{currentstroke}{rgb}{0.000000,0.000000,0.000000}%
\pgfsetstrokecolor{currentstroke}%
\pgfsetdash{}{0pt}%
\pgfpathmoveto{\pgfqpoint{2.238213in}{2.742590in}}%
\pgfpathlineto{\pgfqpoint{2.499580in}{2.886064in}}%
\pgfpathlineto{\pgfqpoint{2.238213in}{2.742590in}}%
\pgfpathclose%
\pgfusepath{fill}%
\end{pgfscope}%
\begin{pgfscope}%
\pgfpathrectangle{\pgfqpoint{0.017500in}{0.017500in}}{\pgfqpoint{6.218282in}{3.580372in}} %
\pgfusepath{clip}%
\pgfsetbuttcap%
\pgfsetroundjoin%
\definecolor{currentfill}{rgb}{0.758112,0.168122,0.188827}%
\pgfsetfillcolor{currentfill}%
\pgfsetlinewidth{0.000000pt}%
\definecolor{currentstroke}{rgb}{0.000000,0.000000,0.000000}%
\pgfsetstrokecolor{currentstroke}%
\pgfsetdash{}{0pt}%
\pgfpathmoveto{\pgfqpoint{2.238213in}{2.742590in}}%
\pgfpathlineto{\pgfqpoint{2.499580in}{2.886064in}}%
\pgfpathlineto{\pgfqpoint{2.238213in}{2.742590in}}%
\pgfpathclose%
\pgfusepath{fill}%
\end{pgfscope}%
\begin{pgfscope}%
\pgfpathrectangle{\pgfqpoint{0.017500in}{0.017500in}}{\pgfqpoint{6.218282in}{3.580372in}} %
\pgfusepath{clip}%
\pgfsetbuttcap%
\pgfsetroundjoin%
\definecolor{currentfill}{rgb}{0.758112,0.168122,0.188827}%
\pgfsetfillcolor{currentfill}%
\pgfsetlinewidth{0.000000pt}%
\definecolor{currentstroke}{rgb}{0.000000,0.000000,0.000000}%
\pgfsetstrokecolor{currentstroke}%
\pgfsetdash{}{0pt}%
\pgfpathmoveto{\pgfqpoint{2.238213in}{2.742590in}}%
\pgfpathlineto{\pgfqpoint{2.499580in}{2.886064in}}%
\pgfpathlineto{\pgfqpoint{2.238213in}{2.742590in}}%
\pgfpathclose%
\pgfusepath{fill}%
\end{pgfscope}%
\begin{pgfscope}%
\pgfpathrectangle{\pgfqpoint{0.017500in}{0.017500in}}{\pgfqpoint{6.218282in}{3.580372in}} %
\pgfusepath{clip}%
\pgfsetbuttcap%
\pgfsetroundjoin%
\definecolor{currentfill}{rgb}{0.758112,0.168122,0.188827}%
\pgfsetfillcolor{currentfill}%
\pgfsetlinewidth{0.000000pt}%
\definecolor{currentstroke}{rgb}{0.000000,0.000000,0.000000}%
\pgfsetstrokecolor{currentstroke}%
\pgfsetdash{}{0pt}%
\pgfpathmoveto{\pgfqpoint{2.238213in}{2.742590in}}%
\pgfpathlineto{\pgfqpoint{2.499580in}{2.886064in}}%
\pgfpathlineto{\pgfqpoint{2.238213in}{2.742590in}}%
\pgfpathclose%
\pgfusepath{fill}%
\end{pgfscope}%
\begin{pgfscope}%
\pgfpathrectangle{\pgfqpoint{0.017500in}{0.017500in}}{\pgfqpoint{6.218282in}{3.580372in}} %
\pgfusepath{clip}%
\pgfsetbuttcap%
\pgfsetroundjoin%
\definecolor{currentfill}{rgb}{0.758112,0.168122,0.188827}%
\pgfsetfillcolor{currentfill}%
\pgfsetlinewidth{0.000000pt}%
\definecolor{currentstroke}{rgb}{0.000000,0.000000,0.000000}%
\pgfsetstrokecolor{currentstroke}%
\pgfsetdash{}{0pt}%
\pgfpathmoveto{\pgfqpoint{2.238213in}{2.742590in}}%
\pgfpathlineto{\pgfqpoint{2.499580in}{2.886064in}}%
\pgfpathlineto{\pgfqpoint{2.238213in}{2.742590in}}%
\pgfpathclose%
\pgfusepath{fill}%
\end{pgfscope}%
\begin{pgfscope}%
\pgfpathrectangle{\pgfqpoint{0.017500in}{0.017500in}}{\pgfqpoint{6.218282in}{3.580372in}} %
\pgfusepath{clip}%
\pgfsetbuttcap%
\pgfsetroundjoin%
\definecolor{currentfill}{rgb}{0.758112,0.168122,0.188827}%
\pgfsetfillcolor{currentfill}%
\pgfsetlinewidth{0.000000pt}%
\definecolor{currentstroke}{rgb}{0.000000,0.000000,0.000000}%
\pgfsetstrokecolor{currentstroke}%
\pgfsetdash{}{0pt}%
\pgfpathmoveto{\pgfqpoint{2.238213in}{2.742590in}}%
\pgfpathlineto{\pgfqpoint{2.499580in}{2.886064in}}%
\pgfpathlineto{\pgfqpoint{2.238213in}{2.742590in}}%
\pgfpathclose%
\pgfusepath{fill}%
\end{pgfscope}%
\begin{pgfscope}%
\pgfpathrectangle{\pgfqpoint{0.017500in}{0.017500in}}{\pgfqpoint{6.218282in}{3.580372in}} %
\pgfusepath{clip}%
\pgfsetbuttcap%
\pgfsetroundjoin%
\definecolor{currentfill}{rgb}{0.758112,0.168122,0.188827}%
\pgfsetfillcolor{currentfill}%
\pgfsetlinewidth{0.000000pt}%
\definecolor{currentstroke}{rgb}{0.000000,0.000000,0.000000}%
\pgfsetstrokecolor{currentstroke}%
\pgfsetdash{}{0pt}%
\pgfpathmoveto{\pgfqpoint{2.238213in}{2.742590in}}%
\pgfpathlineto{\pgfqpoint{2.499580in}{2.886064in}}%
\pgfpathlineto{\pgfqpoint{2.238213in}{2.742590in}}%
\pgfpathclose%
\pgfusepath{fill}%
\end{pgfscope}%
\begin{pgfscope}%
\pgfpathrectangle{\pgfqpoint{0.017500in}{0.017500in}}{\pgfqpoint{6.218282in}{3.580372in}} %
\pgfusepath{clip}%
\pgfsetbuttcap%
\pgfsetroundjoin%
\definecolor{currentfill}{rgb}{0.758112,0.168122,0.188827}%
\pgfsetfillcolor{currentfill}%
\pgfsetlinewidth{0.000000pt}%
\definecolor{currentstroke}{rgb}{0.000000,0.000000,0.000000}%
\pgfsetstrokecolor{currentstroke}%
\pgfsetdash{}{0pt}%
\pgfpathmoveto{\pgfqpoint{2.238213in}{2.742590in}}%
\pgfpathlineto{\pgfqpoint{2.499580in}{2.886064in}}%
\pgfpathlineto{\pgfqpoint{2.238213in}{2.742590in}}%
\pgfpathclose%
\pgfusepath{fill}%
\end{pgfscope}%
\begin{pgfscope}%
\pgfpathrectangle{\pgfqpoint{0.017500in}{0.017500in}}{\pgfqpoint{6.218282in}{3.580372in}} %
\pgfusepath{clip}%
\pgfsetbuttcap%
\pgfsetroundjoin%
\definecolor{currentfill}{rgb}{0.758112,0.168122,0.188827}%
\pgfsetfillcolor{currentfill}%
\pgfsetlinewidth{0.000000pt}%
\definecolor{currentstroke}{rgb}{0.000000,0.000000,0.000000}%
\pgfsetstrokecolor{currentstroke}%
\pgfsetdash{}{0pt}%
\pgfpathmoveto{\pgfqpoint{2.238213in}{2.742590in}}%
\pgfpathlineto{\pgfqpoint{2.499580in}{2.886064in}}%
\pgfpathlineto{\pgfqpoint{2.238213in}{2.742590in}}%
\pgfpathclose%
\pgfusepath{fill}%
\end{pgfscope}%
\begin{pgfscope}%
\pgfpathrectangle{\pgfqpoint{0.017500in}{0.017500in}}{\pgfqpoint{6.218282in}{3.580372in}} %
\pgfusepath{clip}%
\pgfsetbuttcap%
\pgfsetroundjoin%
\definecolor{currentfill}{rgb}{0.758112,0.168122,0.188827}%
\pgfsetfillcolor{currentfill}%
\pgfsetlinewidth{0.000000pt}%
\definecolor{currentstroke}{rgb}{0.000000,0.000000,0.000000}%
\pgfsetstrokecolor{currentstroke}%
\pgfsetdash{}{0pt}%
\pgfpathmoveto{\pgfqpoint{2.238213in}{2.742590in}}%
\pgfpathlineto{\pgfqpoint{2.499580in}{2.886064in}}%
\pgfpathlineto{\pgfqpoint{2.238213in}{2.742590in}}%
\pgfpathclose%
\pgfusepath{fill}%
\end{pgfscope}%
\begin{pgfscope}%
\pgfpathrectangle{\pgfqpoint{0.017500in}{0.017500in}}{\pgfqpoint{6.218282in}{3.580372in}} %
\pgfusepath{clip}%
\pgfsetbuttcap%
\pgfsetroundjoin%
\definecolor{currentfill}{rgb}{0.758112,0.168122,0.188827}%
\pgfsetfillcolor{currentfill}%
\pgfsetlinewidth{0.000000pt}%
\definecolor{currentstroke}{rgb}{0.000000,0.000000,0.000000}%
\pgfsetstrokecolor{currentstroke}%
\pgfsetdash{}{0pt}%
\pgfpathmoveto{\pgfqpoint{2.238213in}{2.742590in}}%
\pgfpathlineto{\pgfqpoint{2.499580in}{2.886064in}}%
\pgfpathlineto{\pgfqpoint{2.238213in}{2.742590in}}%
\pgfpathclose%
\pgfusepath{fill}%
\end{pgfscope}%
\begin{pgfscope}%
\pgfpathrectangle{\pgfqpoint{0.017500in}{0.017500in}}{\pgfqpoint{6.218282in}{3.580372in}} %
\pgfusepath{clip}%
\pgfsetbuttcap%
\pgfsetroundjoin%
\definecolor{currentfill}{rgb}{0.758112,0.168122,0.188827}%
\pgfsetfillcolor{currentfill}%
\pgfsetlinewidth{0.000000pt}%
\definecolor{currentstroke}{rgb}{0.000000,0.000000,0.000000}%
\pgfsetstrokecolor{currentstroke}%
\pgfsetdash{}{0pt}%
\pgfpathmoveto{\pgfqpoint{2.238213in}{2.742590in}}%
\pgfpathlineto{\pgfqpoint{2.499580in}{2.886064in}}%
\pgfpathlineto{\pgfqpoint{2.238213in}{2.742590in}}%
\pgfpathclose%
\pgfusepath{fill}%
\end{pgfscope}%
\begin{pgfscope}%
\pgfpathrectangle{\pgfqpoint{0.017500in}{0.017500in}}{\pgfqpoint{6.218282in}{3.580372in}} %
\pgfusepath{clip}%
\pgfsetbuttcap%
\pgfsetroundjoin%
\definecolor{currentfill}{rgb}{0.758112,0.168122,0.188827}%
\pgfsetfillcolor{currentfill}%
\pgfsetlinewidth{0.000000pt}%
\definecolor{currentstroke}{rgb}{0.000000,0.000000,0.000000}%
\pgfsetstrokecolor{currentstroke}%
\pgfsetdash{}{0pt}%
\pgfpathmoveto{\pgfqpoint{2.238213in}{2.742590in}}%
\pgfpathlineto{\pgfqpoint{2.499580in}{2.886064in}}%
\pgfpathlineto{\pgfqpoint{2.238213in}{2.742590in}}%
\pgfpathclose%
\pgfusepath{fill}%
\end{pgfscope}%
\begin{pgfscope}%
\pgfpathrectangle{\pgfqpoint{0.017500in}{0.017500in}}{\pgfqpoint{6.218282in}{3.580372in}} %
\pgfusepath{clip}%
\pgfsetbuttcap%
\pgfsetroundjoin%
\definecolor{currentfill}{rgb}{0.758112,0.168122,0.188827}%
\pgfsetfillcolor{currentfill}%
\pgfsetlinewidth{0.000000pt}%
\definecolor{currentstroke}{rgb}{0.000000,0.000000,0.000000}%
\pgfsetstrokecolor{currentstroke}%
\pgfsetdash{}{0pt}%
\pgfpathmoveto{\pgfqpoint{2.238213in}{2.742590in}}%
\pgfpathlineto{\pgfqpoint{2.499580in}{2.886064in}}%
\pgfpathlineto{\pgfqpoint{2.238213in}{2.742590in}}%
\pgfpathclose%
\pgfusepath{fill}%
\end{pgfscope}%
\begin{pgfscope}%
\pgfpathrectangle{\pgfqpoint{0.017500in}{0.017500in}}{\pgfqpoint{6.218282in}{3.580372in}} %
\pgfusepath{clip}%
\pgfsetbuttcap%
\pgfsetroundjoin%
\definecolor{currentfill}{rgb}{0.758112,0.168122,0.188827}%
\pgfsetfillcolor{currentfill}%
\pgfsetlinewidth{0.000000pt}%
\definecolor{currentstroke}{rgb}{0.000000,0.000000,0.000000}%
\pgfsetstrokecolor{currentstroke}%
\pgfsetdash{}{0pt}%
\pgfpathmoveto{\pgfqpoint{2.238213in}{2.742590in}}%
\pgfpathlineto{\pgfqpoint{2.499580in}{2.886064in}}%
\pgfpathlineto{\pgfqpoint{2.238213in}{2.742590in}}%
\pgfpathclose%
\pgfusepath{fill}%
\end{pgfscope}%
\begin{pgfscope}%
\pgfpathrectangle{\pgfqpoint{0.017500in}{0.017500in}}{\pgfqpoint{6.218282in}{3.580372in}} %
\pgfusepath{clip}%
\pgfsetbuttcap%
\pgfsetroundjoin%
\definecolor{currentfill}{rgb}{0.758112,0.168122,0.188827}%
\pgfsetfillcolor{currentfill}%
\pgfsetlinewidth{0.000000pt}%
\definecolor{currentstroke}{rgb}{0.000000,0.000000,0.000000}%
\pgfsetstrokecolor{currentstroke}%
\pgfsetdash{}{0pt}%
\pgfpathmoveto{\pgfqpoint{2.238213in}{2.742590in}}%
\pgfpathlineto{\pgfqpoint{2.499580in}{2.886064in}}%
\pgfpathlineto{\pgfqpoint{2.238213in}{2.742590in}}%
\pgfpathclose%
\pgfusepath{fill}%
\end{pgfscope}%
\begin{pgfscope}%
\pgfpathrectangle{\pgfqpoint{0.017500in}{0.017500in}}{\pgfqpoint{6.218282in}{3.580372in}} %
\pgfusepath{clip}%
\pgfsetbuttcap%
\pgfsetroundjoin%
\definecolor{currentfill}{rgb}{0.758112,0.168122,0.188827}%
\pgfsetfillcolor{currentfill}%
\pgfsetlinewidth{0.000000pt}%
\definecolor{currentstroke}{rgb}{0.000000,0.000000,0.000000}%
\pgfsetstrokecolor{currentstroke}%
\pgfsetdash{}{0pt}%
\pgfpathmoveto{\pgfqpoint{2.238213in}{2.742590in}}%
\pgfpathlineto{\pgfqpoint{2.499580in}{2.886064in}}%
\pgfpathlineto{\pgfqpoint{2.238213in}{2.742590in}}%
\pgfpathclose%
\pgfusepath{fill}%
\end{pgfscope}%
\begin{pgfscope}%
\pgfpathrectangle{\pgfqpoint{0.017500in}{0.017500in}}{\pgfqpoint{6.218282in}{3.580372in}} %
\pgfusepath{clip}%
\pgfsetbuttcap%
\pgfsetroundjoin%
\definecolor{currentfill}{rgb}{0.758112,0.168122,0.188827}%
\pgfsetfillcolor{currentfill}%
\pgfsetlinewidth{0.000000pt}%
\definecolor{currentstroke}{rgb}{0.000000,0.000000,0.000000}%
\pgfsetstrokecolor{currentstroke}%
\pgfsetdash{}{0pt}%
\pgfpathmoveto{\pgfqpoint{2.238213in}{2.742590in}}%
\pgfpathlineto{\pgfqpoint{2.499580in}{2.886064in}}%
\pgfpathlineto{\pgfqpoint{2.238213in}{2.742590in}}%
\pgfpathclose%
\pgfusepath{fill}%
\end{pgfscope}%
\begin{pgfscope}%
\pgfpathrectangle{\pgfqpoint{0.017500in}{0.017500in}}{\pgfqpoint{6.218282in}{3.580372in}} %
\pgfusepath{clip}%
\pgfsetbuttcap%
\pgfsetroundjoin%
\definecolor{currentfill}{rgb}{0.758112,0.168122,0.188827}%
\pgfsetfillcolor{currentfill}%
\pgfsetlinewidth{0.000000pt}%
\definecolor{currentstroke}{rgb}{0.000000,0.000000,0.000000}%
\pgfsetstrokecolor{currentstroke}%
\pgfsetdash{}{0pt}%
\pgfpathmoveto{\pgfqpoint{2.238213in}{2.742590in}}%
\pgfpathlineto{\pgfqpoint{2.499580in}{2.886064in}}%
\pgfpathlineto{\pgfqpoint{2.238213in}{2.742590in}}%
\pgfpathclose%
\pgfusepath{fill}%
\end{pgfscope}%
\begin{pgfscope}%
\pgfpathrectangle{\pgfqpoint{0.017500in}{0.017500in}}{\pgfqpoint{6.218282in}{3.580372in}} %
\pgfusepath{clip}%
\pgfsetbuttcap%
\pgfsetroundjoin%
\definecolor{currentfill}{rgb}{0.758112,0.168122,0.188827}%
\pgfsetfillcolor{currentfill}%
\pgfsetlinewidth{0.000000pt}%
\definecolor{currentstroke}{rgb}{0.000000,0.000000,0.000000}%
\pgfsetstrokecolor{currentstroke}%
\pgfsetdash{}{0pt}%
\pgfpathmoveto{\pgfqpoint{2.238213in}{2.742590in}}%
\pgfpathlineto{\pgfqpoint{2.499580in}{2.886064in}}%
\pgfpathlineto{\pgfqpoint{2.238213in}{2.742590in}}%
\pgfpathclose%
\pgfusepath{fill}%
\end{pgfscope}%
\begin{pgfscope}%
\pgfpathrectangle{\pgfqpoint{0.017500in}{0.017500in}}{\pgfqpoint{6.218282in}{3.580372in}} %
\pgfusepath{clip}%
\pgfsetbuttcap%
\pgfsetroundjoin%
\definecolor{currentfill}{rgb}{0.758112,0.168122,0.188827}%
\pgfsetfillcolor{currentfill}%
\pgfsetlinewidth{0.000000pt}%
\definecolor{currentstroke}{rgb}{0.000000,0.000000,0.000000}%
\pgfsetstrokecolor{currentstroke}%
\pgfsetdash{}{0pt}%
\pgfpathmoveto{\pgfqpoint{2.238213in}{2.742590in}}%
\pgfpathlineto{\pgfqpoint{2.499580in}{2.886064in}}%
\pgfpathlineto{\pgfqpoint{2.238213in}{2.742590in}}%
\pgfpathclose%
\pgfusepath{fill}%
\end{pgfscope}%
\begin{pgfscope}%
\pgfpathrectangle{\pgfqpoint{0.017500in}{0.017500in}}{\pgfqpoint{6.218282in}{3.580372in}} %
\pgfusepath{clip}%
\pgfsetbuttcap%
\pgfsetroundjoin%
\definecolor{currentfill}{rgb}{0.758112,0.168122,0.188827}%
\pgfsetfillcolor{currentfill}%
\pgfsetlinewidth{0.000000pt}%
\definecolor{currentstroke}{rgb}{0.000000,0.000000,0.000000}%
\pgfsetstrokecolor{currentstroke}%
\pgfsetdash{}{0pt}%
\pgfpathmoveto{\pgfqpoint{2.238213in}{2.742590in}}%
\pgfpathlineto{\pgfqpoint{2.499580in}{2.886064in}}%
\pgfpathlineto{\pgfqpoint{2.238213in}{2.742590in}}%
\pgfpathclose%
\pgfusepath{fill}%
\end{pgfscope}%
\begin{pgfscope}%
\pgfpathrectangle{\pgfqpoint{0.017500in}{0.017500in}}{\pgfqpoint{6.218282in}{3.580372in}} %
\pgfusepath{clip}%
\pgfsetbuttcap%
\pgfsetroundjoin%
\definecolor{currentfill}{rgb}{0.758112,0.168122,0.188827}%
\pgfsetfillcolor{currentfill}%
\pgfsetlinewidth{0.000000pt}%
\definecolor{currentstroke}{rgb}{0.000000,0.000000,0.000000}%
\pgfsetstrokecolor{currentstroke}%
\pgfsetdash{}{0pt}%
\pgfpathmoveto{\pgfqpoint{2.238213in}{2.742590in}}%
\pgfpathlineto{\pgfqpoint{2.499580in}{2.886064in}}%
\pgfpathlineto{\pgfqpoint{2.238213in}{2.742590in}}%
\pgfpathclose%
\pgfusepath{fill}%
\end{pgfscope}%
\begin{pgfscope}%
\pgfpathrectangle{\pgfqpoint{0.017500in}{0.017500in}}{\pgfqpoint{6.218282in}{3.580372in}} %
\pgfusepath{clip}%
\pgfsetbuttcap%
\pgfsetroundjoin%
\definecolor{currentfill}{rgb}{0.758112,0.168122,0.188827}%
\pgfsetfillcolor{currentfill}%
\pgfsetlinewidth{0.000000pt}%
\definecolor{currentstroke}{rgb}{0.000000,0.000000,0.000000}%
\pgfsetstrokecolor{currentstroke}%
\pgfsetdash{}{0pt}%
\pgfpathmoveto{\pgfqpoint{2.238213in}{2.742590in}}%
\pgfpathlineto{\pgfqpoint{2.499580in}{2.886064in}}%
\pgfpathlineto{\pgfqpoint{2.238213in}{2.742590in}}%
\pgfpathclose%
\pgfusepath{fill}%
\end{pgfscope}%
\begin{pgfscope}%
\pgfpathrectangle{\pgfqpoint{0.017500in}{0.017500in}}{\pgfqpoint{6.218282in}{3.580372in}} %
\pgfusepath{clip}%
\pgfsetbuttcap%
\pgfsetroundjoin%
\definecolor{currentfill}{rgb}{0.758112,0.168122,0.188827}%
\pgfsetfillcolor{currentfill}%
\pgfsetlinewidth{0.000000pt}%
\definecolor{currentstroke}{rgb}{0.000000,0.000000,0.000000}%
\pgfsetstrokecolor{currentstroke}%
\pgfsetdash{}{0pt}%
\pgfpathmoveto{\pgfqpoint{2.238213in}{2.742590in}}%
\pgfpathlineto{\pgfqpoint{2.499580in}{2.886064in}}%
\pgfpathlineto{\pgfqpoint{2.238213in}{2.742590in}}%
\pgfpathclose%
\pgfusepath{fill}%
\end{pgfscope}%
\begin{pgfscope}%
\pgfpathrectangle{\pgfqpoint{0.017500in}{0.017500in}}{\pgfqpoint{6.218282in}{3.580372in}} %
\pgfusepath{clip}%
\pgfsetbuttcap%
\pgfsetroundjoin%
\definecolor{currentfill}{rgb}{0.902849,0.844796,0.811970}%
\pgfsetfillcolor{currentfill}%
\pgfsetlinewidth{0.000000pt}%
\definecolor{currentstroke}{rgb}{0.000000,0.000000,0.000000}%
\pgfsetstrokecolor{currentstroke}%
\pgfsetdash{}{0pt}%
\pgfpathmoveto{\pgfqpoint{3.622269in}{2.896533in}}%
\pgfpathlineto{\pgfqpoint{2.349703in}{0.878379in}}%
\pgfpathlineto{\pgfqpoint{3.622269in}{2.896533in}}%
\pgfpathclose%
\pgfusepath{fill}%
\end{pgfscope}%
\begin{pgfscope}%
\pgfpathrectangle{\pgfqpoint{0.017500in}{0.017500in}}{\pgfqpoint{6.218282in}{3.580372in}} %
\pgfusepath{clip}%
\pgfsetbuttcap%
\pgfsetroundjoin%
\definecolor{currentfill}{rgb}{0.902849,0.844796,0.811970}%
\pgfsetfillcolor{currentfill}%
\pgfsetlinewidth{0.000000pt}%
\definecolor{currentstroke}{rgb}{0.000000,0.000000,0.000000}%
\pgfsetstrokecolor{currentstroke}%
\pgfsetdash{}{0pt}%
\pgfpathmoveto{\pgfqpoint{3.622269in}{2.896533in}}%
\pgfpathlineto{\pgfqpoint{2.349703in}{0.878379in}}%
\pgfpathlineto{\pgfqpoint{3.622269in}{2.896533in}}%
\pgfpathclose%
\pgfusepath{fill}%
\end{pgfscope}%
\begin{pgfscope}%
\pgfpathrectangle{\pgfqpoint{0.017500in}{0.017500in}}{\pgfqpoint{6.218282in}{3.580372in}} %
\pgfusepath{clip}%
\pgfsetbuttcap%
\pgfsetroundjoin%
\definecolor{currentfill}{rgb}{0.902849,0.844796,0.811970}%
\pgfsetfillcolor{currentfill}%
\pgfsetlinewidth{0.000000pt}%
\definecolor{currentstroke}{rgb}{0.000000,0.000000,0.000000}%
\pgfsetstrokecolor{currentstroke}%
\pgfsetdash{}{0pt}%
\pgfpathmoveto{\pgfqpoint{3.622269in}{2.896533in}}%
\pgfpathlineto{\pgfqpoint{2.349703in}{0.878379in}}%
\pgfpathlineto{\pgfqpoint{3.622269in}{2.896533in}}%
\pgfpathclose%
\pgfusepath{fill}%
\end{pgfscope}%
\begin{pgfscope}%
\pgfpathrectangle{\pgfqpoint{0.017500in}{0.017500in}}{\pgfqpoint{6.218282in}{3.580372in}} %
\pgfusepath{clip}%
\pgfsetbuttcap%
\pgfsetroundjoin%
\definecolor{currentfill}{rgb}{0.902849,0.844796,0.811970}%
\pgfsetfillcolor{currentfill}%
\pgfsetlinewidth{0.000000pt}%
\definecolor{currentstroke}{rgb}{0.000000,0.000000,0.000000}%
\pgfsetstrokecolor{currentstroke}%
\pgfsetdash{}{0pt}%
\pgfpathmoveto{\pgfqpoint{3.622269in}{2.896533in}}%
\pgfpathlineto{\pgfqpoint{2.349703in}{0.878379in}}%
\pgfpathlineto{\pgfqpoint{3.622269in}{2.896533in}}%
\pgfpathclose%
\pgfusepath{fill}%
\end{pgfscope}%
\begin{pgfscope}%
\pgfpathrectangle{\pgfqpoint{0.017500in}{0.017500in}}{\pgfqpoint{6.218282in}{3.580372in}} %
\pgfusepath{clip}%
\pgfsetbuttcap%
\pgfsetroundjoin%
\definecolor{currentfill}{rgb}{0.902849,0.844796,0.811970}%
\pgfsetfillcolor{currentfill}%
\pgfsetlinewidth{0.000000pt}%
\definecolor{currentstroke}{rgb}{0.000000,0.000000,0.000000}%
\pgfsetstrokecolor{currentstroke}%
\pgfsetdash{}{0pt}%
\pgfpathmoveto{\pgfqpoint{3.622269in}{2.896533in}}%
\pgfpathlineto{\pgfqpoint{2.349703in}{0.878379in}}%
\pgfpathlineto{\pgfqpoint{3.622269in}{2.896533in}}%
\pgfpathclose%
\pgfusepath{fill}%
\end{pgfscope}%
\begin{pgfscope}%
\pgfpathrectangle{\pgfqpoint{0.017500in}{0.017500in}}{\pgfqpoint{6.218282in}{3.580372in}} %
\pgfusepath{clip}%
\pgfsetbuttcap%
\pgfsetroundjoin%
\definecolor{currentfill}{rgb}{0.902849,0.844796,0.811970}%
\pgfsetfillcolor{currentfill}%
\pgfsetlinewidth{0.000000pt}%
\definecolor{currentstroke}{rgb}{0.000000,0.000000,0.000000}%
\pgfsetstrokecolor{currentstroke}%
\pgfsetdash{}{0pt}%
\pgfpathmoveto{\pgfqpoint{3.622269in}{2.896533in}}%
\pgfpathlineto{\pgfqpoint{2.349703in}{0.878379in}}%
\pgfpathlineto{\pgfqpoint{3.622269in}{2.896533in}}%
\pgfpathclose%
\pgfusepath{fill}%
\end{pgfscope}%
\begin{pgfscope}%
\pgfpathrectangle{\pgfqpoint{0.017500in}{0.017500in}}{\pgfqpoint{6.218282in}{3.580372in}} %
\pgfusepath{clip}%
\pgfsetbuttcap%
\pgfsetroundjoin%
\definecolor{currentfill}{rgb}{0.902849,0.844796,0.811970}%
\pgfsetfillcolor{currentfill}%
\pgfsetlinewidth{0.000000pt}%
\definecolor{currentstroke}{rgb}{0.000000,0.000000,0.000000}%
\pgfsetstrokecolor{currentstroke}%
\pgfsetdash{}{0pt}%
\pgfpathmoveto{\pgfqpoint{3.622269in}{2.896533in}}%
\pgfpathlineto{\pgfqpoint{2.349703in}{0.878379in}}%
\pgfpathlineto{\pgfqpoint{3.622269in}{2.896533in}}%
\pgfpathclose%
\pgfusepath{fill}%
\end{pgfscope}%
\begin{pgfscope}%
\pgfpathrectangle{\pgfqpoint{0.017500in}{0.017500in}}{\pgfqpoint{6.218282in}{3.580372in}} %
\pgfusepath{clip}%
\pgfsetbuttcap%
\pgfsetroundjoin%
\definecolor{currentfill}{rgb}{0.902849,0.844796,0.811970}%
\pgfsetfillcolor{currentfill}%
\pgfsetlinewidth{0.000000pt}%
\definecolor{currentstroke}{rgb}{0.000000,0.000000,0.000000}%
\pgfsetstrokecolor{currentstroke}%
\pgfsetdash{}{0pt}%
\pgfpathmoveto{\pgfqpoint{3.622269in}{2.896533in}}%
\pgfpathlineto{\pgfqpoint{2.349703in}{0.878379in}}%
\pgfpathlineto{\pgfqpoint{3.622269in}{2.896533in}}%
\pgfpathclose%
\pgfusepath{fill}%
\end{pgfscope}%
\begin{pgfscope}%
\pgfpathrectangle{\pgfqpoint{0.017500in}{0.017500in}}{\pgfqpoint{6.218282in}{3.580372in}} %
\pgfusepath{clip}%
\pgfsetbuttcap%
\pgfsetroundjoin%
\definecolor{currentfill}{rgb}{0.902849,0.844796,0.811970}%
\pgfsetfillcolor{currentfill}%
\pgfsetlinewidth{0.000000pt}%
\definecolor{currentstroke}{rgb}{0.000000,0.000000,0.000000}%
\pgfsetstrokecolor{currentstroke}%
\pgfsetdash{}{0pt}%
\pgfpathmoveto{\pgfqpoint{3.622269in}{2.896533in}}%
\pgfpathlineto{\pgfqpoint{2.349703in}{0.878379in}}%
\pgfpathlineto{\pgfqpoint{3.622269in}{2.896533in}}%
\pgfpathclose%
\pgfusepath{fill}%
\end{pgfscope}%
\begin{pgfscope}%
\pgfpathrectangle{\pgfqpoint{0.017500in}{0.017500in}}{\pgfqpoint{6.218282in}{3.580372in}} %
\pgfusepath{clip}%
\pgfsetbuttcap%
\pgfsetroundjoin%
\definecolor{currentfill}{rgb}{0.902849,0.844796,0.811970}%
\pgfsetfillcolor{currentfill}%
\pgfsetlinewidth{0.000000pt}%
\definecolor{currentstroke}{rgb}{0.000000,0.000000,0.000000}%
\pgfsetstrokecolor{currentstroke}%
\pgfsetdash{}{0pt}%
\pgfpathmoveto{\pgfqpoint{3.622269in}{2.896533in}}%
\pgfpathlineto{\pgfqpoint{2.349703in}{0.878379in}}%
\pgfpathlineto{\pgfqpoint{3.622269in}{2.896533in}}%
\pgfpathclose%
\pgfusepath{fill}%
\end{pgfscope}%
\begin{pgfscope}%
\pgfpathrectangle{\pgfqpoint{0.017500in}{0.017500in}}{\pgfqpoint{6.218282in}{3.580372in}} %
\pgfusepath{clip}%
\pgfsetbuttcap%
\pgfsetroundjoin%
\definecolor{currentfill}{rgb}{0.902849,0.844796,0.811970}%
\pgfsetfillcolor{currentfill}%
\pgfsetlinewidth{0.000000pt}%
\definecolor{currentstroke}{rgb}{0.000000,0.000000,0.000000}%
\pgfsetstrokecolor{currentstroke}%
\pgfsetdash{}{0pt}%
\pgfpathmoveto{\pgfqpoint{3.622269in}{2.896533in}}%
\pgfpathlineto{\pgfqpoint{2.349703in}{0.878379in}}%
\pgfpathlineto{\pgfqpoint{3.622269in}{2.896533in}}%
\pgfpathclose%
\pgfusepath{fill}%
\end{pgfscope}%
\begin{pgfscope}%
\pgfpathrectangle{\pgfqpoint{0.017500in}{0.017500in}}{\pgfqpoint{6.218282in}{3.580372in}} %
\pgfusepath{clip}%
\pgfsetbuttcap%
\pgfsetroundjoin%
\definecolor{currentfill}{rgb}{0.902849,0.844796,0.811970}%
\pgfsetfillcolor{currentfill}%
\pgfsetlinewidth{0.000000pt}%
\definecolor{currentstroke}{rgb}{0.000000,0.000000,0.000000}%
\pgfsetstrokecolor{currentstroke}%
\pgfsetdash{}{0pt}%
\pgfpathmoveto{\pgfqpoint{3.622269in}{2.896533in}}%
\pgfpathlineto{\pgfqpoint{2.349703in}{0.878379in}}%
\pgfpathlineto{\pgfqpoint{3.622269in}{2.896533in}}%
\pgfpathclose%
\pgfusepath{fill}%
\end{pgfscope}%
\begin{pgfscope}%
\pgfpathrectangle{\pgfqpoint{0.017500in}{0.017500in}}{\pgfqpoint{6.218282in}{3.580372in}} %
\pgfusepath{clip}%
\pgfsetbuttcap%
\pgfsetroundjoin%
\definecolor{currentfill}{rgb}{0.902849,0.844796,0.811970}%
\pgfsetfillcolor{currentfill}%
\pgfsetlinewidth{0.000000pt}%
\definecolor{currentstroke}{rgb}{0.000000,0.000000,0.000000}%
\pgfsetstrokecolor{currentstroke}%
\pgfsetdash{}{0pt}%
\pgfpathmoveto{\pgfqpoint{3.622269in}{2.896533in}}%
\pgfpathlineto{\pgfqpoint{2.349703in}{0.878379in}}%
\pgfpathlineto{\pgfqpoint{3.622269in}{2.896533in}}%
\pgfpathclose%
\pgfusepath{fill}%
\end{pgfscope}%
\begin{pgfscope}%
\pgfpathrectangle{\pgfqpoint{0.017500in}{0.017500in}}{\pgfqpoint{6.218282in}{3.580372in}} %
\pgfusepath{clip}%
\pgfsetbuttcap%
\pgfsetroundjoin%
\definecolor{currentfill}{rgb}{0.902849,0.844796,0.811970}%
\pgfsetfillcolor{currentfill}%
\pgfsetlinewidth{0.000000pt}%
\definecolor{currentstroke}{rgb}{0.000000,0.000000,0.000000}%
\pgfsetstrokecolor{currentstroke}%
\pgfsetdash{}{0pt}%
\pgfpathmoveto{\pgfqpoint{3.622269in}{2.896533in}}%
\pgfpathlineto{\pgfqpoint{2.349703in}{0.878379in}}%
\pgfpathlineto{\pgfqpoint{3.622269in}{2.896533in}}%
\pgfpathclose%
\pgfusepath{fill}%
\end{pgfscope}%
\begin{pgfscope}%
\pgfpathrectangle{\pgfqpoint{0.017500in}{0.017500in}}{\pgfqpoint{6.218282in}{3.580372in}} %
\pgfusepath{clip}%
\pgfsetbuttcap%
\pgfsetroundjoin%
\definecolor{currentfill}{rgb}{0.902849,0.844796,0.811970}%
\pgfsetfillcolor{currentfill}%
\pgfsetlinewidth{0.000000pt}%
\definecolor{currentstroke}{rgb}{0.000000,0.000000,0.000000}%
\pgfsetstrokecolor{currentstroke}%
\pgfsetdash{}{0pt}%
\pgfpathmoveto{\pgfqpoint{3.622269in}{2.896533in}}%
\pgfpathlineto{\pgfqpoint{2.349703in}{0.878379in}}%
\pgfpathlineto{\pgfqpoint{3.622269in}{2.896533in}}%
\pgfpathclose%
\pgfusepath{fill}%
\end{pgfscope}%
\begin{pgfscope}%
\pgfpathrectangle{\pgfqpoint{0.017500in}{0.017500in}}{\pgfqpoint{6.218282in}{3.580372in}} %
\pgfusepath{clip}%
\pgfsetbuttcap%
\pgfsetroundjoin%
\definecolor{currentfill}{rgb}{0.902849,0.844796,0.811970}%
\pgfsetfillcolor{currentfill}%
\pgfsetlinewidth{0.000000pt}%
\definecolor{currentstroke}{rgb}{0.000000,0.000000,0.000000}%
\pgfsetstrokecolor{currentstroke}%
\pgfsetdash{}{0pt}%
\pgfpathmoveto{\pgfqpoint{3.622269in}{2.896533in}}%
\pgfpathlineto{\pgfqpoint{2.349703in}{0.878379in}}%
\pgfpathlineto{\pgfqpoint{3.622269in}{2.896533in}}%
\pgfpathclose%
\pgfusepath{fill}%
\end{pgfscope}%
\begin{pgfscope}%
\pgfpathrectangle{\pgfqpoint{0.017500in}{0.017500in}}{\pgfqpoint{6.218282in}{3.580372in}} %
\pgfusepath{clip}%
\pgfsetbuttcap%
\pgfsetroundjoin%
\definecolor{currentfill}{rgb}{0.902849,0.844796,0.811970}%
\pgfsetfillcolor{currentfill}%
\pgfsetlinewidth{0.000000pt}%
\definecolor{currentstroke}{rgb}{0.000000,0.000000,0.000000}%
\pgfsetstrokecolor{currentstroke}%
\pgfsetdash{}{0pt}%
\pgfpathmoveto{\pgfqpoint{3.622269in}{2.896533in}}%
\pgfpathlineto{\pgfqpoint{2.349703in}{0.878379in}}%
\pgfpathlineto{\pgfqpoint{3.622269in}{2.896533in}}%
\pgfpathclose%
\pgfusepath{fill}%
\end{pgfscope}%
\begin{pgfscope}%
\pgfpathrectangle{\pgfqpoint{0.017500in}{0.017500in}}{\pgfqpoint{6.218282in}{3.580372in}} %
\pgfusepath{clip}%
\pgfsetbuttcap%
\pgfsetroundjoin%
\definecolor{currentfill}{rgb}{0.902849,0.844796,0.811970}%
\pgfsetfillcolor{currentfill}%
\pgfsetlinewidth{0.000000pt}%
\definecolor{currentstroke}{rgb}{0.000000,0.000000,0.000000}%
\pgfsetstrokecolor{currentstroke}%
\pgfsetdash{}{0pt}%
\pgfpathmoveto{\pgfqpoint{3.622269in}{2.896533in}}%
\pgfpathlineto{\pgfqpoint{2.349703in}{0.878379in}}%
\pgfpathlineto{\pgfqpoint{3.622269in}{2.896533in}}%
\pgfpathclose%
\pgfusepath{fill}%
\end{pgfscope}%
\begin{pgfscope}%
\pgfpathrectangle{\pgfqpoint{0.017500in}{0.017500in}}{\pgfqpoint{6.218282in}{3.580372in}} %
\pgfusepath{clip}%
\pgfsetbuttcap%
\pgfsetroundjoin%
\definecolor{currentfill}{rgb}{0.902849,0.844796,0.811970}%
\pgfsetfillcolor{currentfill}%
\pgfsetlinewidth{0.000000pt}%
\definecolor{currentstroke}{rgb}{0.000000,0.000000,0.000000}%
\pgfsetstrokecolor{currentstroke}%
\pgfsetdash{}{0pt}%
\pgfpathmoveto{\pgfqpoint{3.622269in}{2.896533in}}%
\pgfpathlineto{\pgfqpoint{2.349703in}{0.878379in}}%
\pgfpathlineto{\pgfqpoint{3.622269in}{2.896533in}}%
\pgfpathclose%
\pgfusepath{fill}%
\end{pgfscope}%
\begin{pgfscope}%
\pgfpathrectangle{\pgfqpoint{0.017500in}{0.017500in}}{\pgfqpoint{6.218282in}{3.580372in}} %
\pgfusepath{clip}%
\pgfsetbuttcap%
\pgfsetroundjoin%
\definecolor{currentfill}{rgb}{0.902849,0.844796,0.811970}%
\pgfsetfillcolor{currentfill}%
\pgfsetlinewidth{0.000000pt}%
\definecolor{currentstroke}{rgb}{0.000000,0.000000,0.000000}%
\pgfsetstrokecolor{currentstroke}%
\pgfsetdash{}{0pt}%
\pgfpathmoveto{\pgfqpoint{3.622269in}{2.896533in}}%
\pgfpathlineto{\pgfqpoint{2.349703in}{0.878379in}}%
\pgfpathlineto{\pgfqpoint{3.622269in}{2.896533in}}%
\pgfpathclose%
\pgfusepath{fill}%
\end{pgfscope}%
\begin{pgfscope}%
\pgfpathrectangle{\pgfqpoint{0.017500in}{0.017500in}}{\pgfqpoint{6.218282in}{3.580372in}} %
\pgfusepath{clip}%
\pgfsetbuttcap%
\pgfsetroundjoin%
\definecolor{currentfill}{rgb}{0.902849,0.844796,0.811970}%
\pgfsetfillcolor{currentfill}%
\pgfsetlinewidth{0.000000pt}%
\definecolor{currentstroke}{rgb}{0.000000,0.000000,0.000000}%
\pgfsetstrokecolor{currentstroke}%
\pgfsetdash{}{0pt}%
\pgfpathmoveto{\pgfqpoint{3.622269in}{2.896533in}}%
\pgfpathlineto{\pgfqpoint{2.349703in}{0.878379in}}%
\pgfpathlineto{\pgfqpoint{3.622269in}{2.896533in}}%
\pgfpathclose%
\pgfusepath{fill}%
\end{pgfscope}%
\begin{pgfscope}%
\pgfpathrectangle{\pgfqpoint{0.017500in}{0.017500in}}{\pgfqpoint{6.218282in}{3.580372in}} %
\pgfusepath{clip}%
\pgfsetbuttcap%
\pgfsetroundjoin%
\definecolor{currentfill}{rgb}{0.902849,0.844796,0.811970}%
\pgfsetfillcolor{currentfill}%
\pgfsetlinewidth{0.000000pt}%
\definecolor{currentstroke}{rgb}{0.000000,0.000000,0.000000}%
\pgfsetstrokecolor{currentstroke}%
\pgfsetdash{}{0pt}%
\pgfpathmoveto{\pgfqpoint{3.622269in}{2.896533in}}%
\pgfpathlineto{\pgfqpoint{2.349703in}{0.878379in}}%
\pgfpathlineto{\pgfqpoint{3.622269in}{2.896533in}}%
\pgfpathclose%
\pgfusepath{fill}%
\end{pgfscope}%
\begin{pgfscope}%
\pgfpathrectangle{\pgfqpoint{0.017500in}{0.017500in}}{\pgfqpoint{6.218282in}{3.580372in}} %
\pgfusepath{clip}%
\pgfsetbuttcap%
\pgfsetroundjoin%
\definecolor{currentfill}{rgb}{0.902849,0.844796,0.811970}%
\pgfsetfillcolor{currentfill}%
\pgfsetlinewidth{0.000000pt}%
\definecolor{currentstroke}{rgb}{0.000000,0.000000,0.000000}%
\pgfsetstrokecolor{currentstroke}%
\pgfsetdash{}{0pt}%
\pgfpathmoveto{\pgfqpoint{3.622269in}{2.896533in}}%
\pgfpathlineto{\pgfqpoint{2.349703in}{0.878379in}}%
\pgfpathlineto{\pgfqpoint{3.622269in}{2.896533in}}%
\pgfpathclose%
\pgfusepath{fill}%
\end{pgfscope}%
\begin{pgfscope}%
\pgfpathrectangle{\pgfqpoint{0.017500in}{0.017500in}}{\pgfqpoint{6.218282in}{3.580372in}} %
\pgfusepath{clip}%
\pgfsetbuttcap%
\pgfsetroundjoin%
\definecolor{currentfill}{rgb}{0.902849,0.844796,0.811970}%
\pgfsetfillcolor{currentfill}%
\pgfsetlinewidth{0.000000pt}%
\definecolor{currentstroke}{rgb}{0.000000,0.000000,0.000000}%
\pgfsetstrokecolor{currentstroke}%
\pgfsetdash{}{0pt}%
\pgfpathmoveto{\pgfqpoint{3.622269in}{2.896533in}}%
\pgfpathlineto{\pgfqpoint{2.349703in}{0.878379in}}%
\pgfpathlineto{\pgfqpoint{3.622269in}{2.896533in}}%
\pgfpathclose%
\pgfusepath{fill}%
\end{pgfscope}%
\begin{pgfscope}%
\pgfpathrectangle{\pgfqpoint{0.017500in}{0.017500in}}{\pgfqpoint{6.218282in}{3.580372in}} %
\pgfusepath{clip}%
\pgfsetbuttcap%
\pgfsetroundjoin%
\definecolor{currentfill}{rgb}{0.902849,0.844796,0.811970}%
\pgfsetfillcolor{currentfill}%
\pgfsetlinewidth{0.000000pt}%
\definecolor{currentstroke}{rgb}{0.000000,0.000000,0.000000}%
\pgfsetstrokecolor{currentstroke}%
\pgfsetdash{}{0pt}%
\pgfpathmoveto{\pgfqpoint{3.622269in}{2.896533in}}%
\pgfpathlineto{\pgfqpoint{2.349703in}{0.878379in}}%
\pgfpathlineto{\pgfqpoint{3.622269in}{2.896533in}}%
\pgfpathclose%
\pgfusepath{fill}%
\end{pgfscope}%
\begin{pgfscope}%
\pgfpathrectangle{\pgfqpoint{0.017500in}{0.017500in}}{\pgfqpoint{6.218282in}{3.580372in}} %
\pgfusepath{clip}%
\pgfsetbuttcap%
\pgfsetroundjoin%
\definecolor{currentfill}{rgb}{0.752704,0.157576,0.184258}%
\pgfsetfillcolor{currentfill}%
\pgfsetlinewidth{0.000000pt}%
\definecolor{currentstroke}{rgb}{0.000000,0.000000,0.000000}%
\pgfsetstrokecolor{currentstroke}%
\pgfsetdash{}{0pt}%
\pgfpathmoveto{\pgfqpoint{2.238213in}{2.742590in}}%
\pgfpathlineto{\pgfqpoint{2.674281in}{2.686821in}}%
\pgfpathlineto{\pgfqpoint{2.932971in}{2.820255in}}%
\pgfpathlineto{\pgfqpoint{2.499580in}{2.886064in}}%
\pgfpathlineto{\pgfqpoint{2.238213in}{2.742590in}}%
\pgfpathclose%
\pgfusepath{fill}%
\end{pgfscope}%
\begin{pgfscope}%
\pgfpathrectangle{\pgfqpoint{0.017500in}{0.017500in}}{\pgfqpoint{6.218282in}{3.580372in}} %
\pgfusepath{clip}%
\pgfsetbuttcap%
\pgfsetroundjoin%
\definecolor{currentfill}{rgb}{0.752704,0.157576,0.184258}%
\pgfsetfillcolor{currentfill}%
\pgfsetlinewidth{0.000000pt}%
\definecolor{currentstroke}{rgb}{0.000000,0.000000,0.000000}%
\pgfsetstrokecolor{currentstroke}%
\pgfsetdash{}{0pt}%
\pgfpathmoveto{\pgfqpoint{2.238213in}{2.742590in}}%
\pgfpathlineto{\pgfqpoint{2.674281in}{2.686821in}}%
\pgfpathlineto{\pgfqpoint{2.932971in}{2.820255in}}%
\pgfpathlineto{\pgfqpoint{2.499580in}{2.886064in}}%
\pgfpathlineto{\pgfqpoint{2.238213in}{2.742590in}}%
\pgfpathclose%
\pgfusepath{fill}%
\end{pgfscope}%
\begin{pgfscope}%
\pgfpathrectangle{\pgfqpoint{0.017500in}{0.017500in}}{\pgfqpoint{6.218282in}{3.580372in}} %
\pgfusepath{clip}%
\pgfsetbuttcap%
\pgfsetroundjoin%
\definecolor{currentfill}{rgb}{0.752704,0.157576,0.184258}%
\pgfsetfillcolor{currentfill}%
\pgfsetlinewidth{0.000000pt}%
\definecolor{currentstroke}{rgb}{0.000000,0.000000,0.000000}%
\pgfsetstrokecolor{currentstroke}%
\pgfsetdash{}{0pt}%
\pgfpathmoveto{\pgfqpoint{2.238213in}{2.742590in}}%
\pgfpathlineto{\pgfqpoint{2.674281in}{2.686821in}}%
\pgfpathlineto{\pgfqpoint{2.932971in}{2.820255in}}%
\pgfpathlineto{\pgfqpoint{2.499580in}{2.886064in}}%
\pgfpathlineto{\pgfqpoint{2.238213in}{2.742590in}}%
\pgfpathclose%
\pgfusepath{fill}%
\end{pgfscope}%
\begin{pgfscope}%
\pgfpathrectangle{\pgfqpoint{0.017500in}{0.017500in}}{\pgfqpoint{6.218282in}{3.580372in}} %
\pgfusepath{clip}%
\pgfsetbuttcap%
\pgfsetroundjoin%
\definecolor{currentfill}{rgb}{0.752704,0.157576,0.184258}%
\pgfsetfillcolor{currentfill}%
\pgfsetlinewidth{0.000000pt}%
\definecolor{currentstroke}{rgb}{0.000000,0.000000,0.000000}%
\pgfsetstrokecolor{currentstroke}%
\pgfsetdash{}{0pt}%
\pgfpathmoveto{\pgfqpoint{2.238213in}{2.742590in}}%
\pgfpathlineto{\pgfqpoint{2.674281in}{2.686821in}}%
\pgfpathlineto{\pgfqpoint{2.932971in}{2.820255in}}%
\pgfpathlineto{\pgfqpoint{2.499580in}{2.886064in}}%
\pgfpathlineto{\pgfqpoint{2.238213in}{2.742590in}}%
\pgfpathclose%
\pgfusepath{fill}%
\end{pgfscope}%
\begin{pgfscope}%
\pgfpathrectangle{\pgfqpoint{0.017500in}{0.017500in}}{\pgfqpoint{6.218282in}{3.580372in}} %
\pgfusepath{clip}%
\pgfsetbuttcap%
\pgfsetroundjoin%
\definecolor{currentfill}{rgb}{0.752704,0.157576,0.184258}%
\pgfsetfillcolor{currentfill}%
\pgfsetlinewidth{0.000000pt}%
\definecolor{currentstroke}{rgb}{0.000000,0.000000,0.000000}%
\pgfsetstrokecolor{currentstroke}%
\pgfsetdash{}{0pt}%
\pgfpathmoveto{\pgfqpoint{2.238213in}{2.742590in}}%
\pgfpathlineto{\pgfqpoint{2.674281in}{2.686821in}}%
\pgfpathlineto{\pgfqpoint{2.932971in}{2.820255in}}%
\pgfpathlineto{\pgfqpoint{2.499580in}{2.886064in}}%
\pgfpathlineto{\pgfqpoint{2.238213in}{2.742590in}}%
\pgfpathclose%
\pgfusepath{fill}%
\end{pgfscope}%
\begin{pgfscope}%
\pgfpathrectangle{\pgfqpoint{0.017500in}{0.017500in}}{\pgfqpoint{6.218282in}{3.580372in}} %
\pgfusepath{clip}%
\pgfsetbuttcap%
\pgfsetroundjoin%
\definecolor{currentfill}{rgb}{0.752704,0.157576,0.184258}%
\pgfsetfillcolor{currentfill}%
\pgfsetlinewidth{0.000000pt}%
\definecolor{currentstroke}{rgb}{0.000000,0.000000,0.000000}%
\pgfsetstrokecolor{currentstroke}%
\pgfsetdash{}{0pt}%
\pgfpathmoveto{\pgfqpoint{2.238213in}{2.742590in}}%
\pgfpathlineto{\pgfqpoint{2.674281in}{2.686821in}}%
\pgfpathlineto{\pgfqpoint{2.932971in}{2.820255in}}%
\pgfpathlineto{\pgfqpoint{2.499580in}{2.886064in}}%
\pgfpathlineto{\pgfqpoint{2.238213in}{2.742590in}}%
\pgfpathclose%
\pgfusepath{fill}%
\end{pgfscope}%
\begin{pgfscope}%
\pgfpathrectangle{\pgfqpoint{0.017500in}{0.017500in}}{\pgfqpoint{6.218282in}{3.580372in}} %
\pgfusepath{clip}%
\pgfsetbuttcap%
\pgfsetroundjoin%
\definecolor{currentfill}{rgb}{0.711554,0.033337,0.154485}%
\pgfsetfillcolor{currentfill}%
\pgfsetlinewidth{0.000000pt}%
\definecolor{currentstroke}{rgb}{0.000000,0.000000,0.000000}%
\pgfsetstrokecolor{currentstroke}%
\pgfsetdash{}{0pt}%
\pgfpathmoveto{\pgfqpoint{3.372089in}{2.760957in}}%
\pgfpathlineto{\pgfqpoint{3.622269in}{2.896533in}}%
\pgfpathlineto{\pgfqpoint{3.372089in}{2.760957in}}%
\pgfpathclose%
\pgfusepath{fill}%
\end{pgfscope}%
\begin{pgfscope}%
\pgfpathrectangle{\pgfqpoint{0.017500in}{0.017500in}}{\pgfqpoint{6.218282in}{3.580372in}} %
\pgfusepath{clip}%
\pgfsetbuttcap%
\pgfsetroundjoin%
\definecolor{currentfill}{rgb}{0.711554,0.033337,0.154485}%
\pgfsetfillcolor{currentfill}%
\pgfsetlinewidth{0.000000pt}%
\definecolor{currentstroke}{rgb}{0.000000,0.000000,0.000000}%
\pgfsetstrokecolor{currentstroke}%
\pgfsetdash{}{0pt}%
\pgfpathmoveto{\pgfqpoint{3.372089in}{2.760957in}}%
\pgfpathlineto{\pgfqpoint{3.622269in}{2.896533in}}%
\pgfpathlineto{\pgfqpoint{3.372089in}{2.760957in}}%
\pgfpathclose%
\pgfusepath{fill}%
\end{pgfscope}%
\begin{pgfscope}%
\pgfpathrectangle{\pgfqpoint{0.017500in}{0.017500in}}{\pgfqpoint{6.218282in}{3.580372in}} %
\pgfusepath{clip}%
\pgfsetbuttcap%
\pgfsetroundjoin%
\definecolor{currentfill}{rgb}{0.711554,0.033337,0.154485}%
\pgfsetfillcolor{currentfill}%
\pgfsetlinewidth{0.000000pt}%
\definecolor{currentstroke}{rgb}{0.000000,0.000000,0.000000}%
\pgfsetstrokecolor{currentstroke}%
\pgfsetdash{}{0pt}%
\pgfpathmoveto{\pgfqpoint{3.372089in}{2.760957in}}%
\pgfpathlineto{\pgfqpoint{3.622269in}{2.896533in}}%
\pgfpathlineto{\pgfqpoint{3.372089in}{2.760957in}}%
\pgfpathclose%
\pgfusepath{fill}%
\end{pgfscope}%
\begin{pgfscope}%
\pgfpathrectangle{\pgfqpoint{0.017500in}{0.017500in}}{\pgfqpoint{6.218282in}{3.580372in}} %
\pgfusepath{clip}%
\pgfsetbuttcap%
\pgfsetroundjoin%
\definecolor{currentfill}{rgb}{0.711554,0.033337,0.154485}%
\pgfsetfillcolor{currentfill}%
\pgfsetlinewidth{0.000000pt}%
\definecolor{currentstroke}{rgb}{0.000000,0.000000,0.000000}%
\pgfsetstrokecolor{currentstroke}%
\pgfsetdash{}{0pt}%
\pgfpathmoveto{\pgfqpoint{3.372089in}{2.760957in}}%
\pgfpathlineto{\pgfqpoint{3.622269in}{2.896533in}}%
\pgfpathlineto{\pgfqpoint{3.372089in}{2.760957in}}%
\pgfpathclose%
\pgfusepath{fill}%
\end{pgfscope}%
\begin{pgfscope}%
\pgfpathrectangle{\pgfqpoint{0.017500in}{0.017500in}}{\pgfqpoint{6.218282in}{3.580372in}} %
\pgfusepath{clip}%
\pgfsetbuttcap%
\pgfsetroundjoin%
\definecolor{currentfill}{rgb}{0.711554,0.033337,0.154485}%
\pgfsetfillcolor{currentfill}%
\pgfsetlinewidth{0.000000pt}%
\definecolor{currentstroke}{rgb}{0.000000,0.000000,0.000000}%
\pgfsetstrokecolor{currentstroke}%
\pgfsetdash{}{0pt}%
\pgfpathmoveto{\pgfqpoint{3.372089in}{2.760957in}}%
\pgfpathlineto{\pgfqpoint{3.622269in}{2.896533in}}%
\pgfpathlineto{\pgfqpoint{3.372089in}{2.760957in}}%
\pgfpathclose%
\pgfusepath{fill}%
\end{pgfscope}%
\begin{pgfscope}%
\pgfpathrectangle{\pgfqpoint{0.017500in}{0.017500in}}{\pgfqpoint{6.218282in}{3.580372in}} %
\pgfusepath{clip}%
\pgfsetbuttcap%
\pgfsetroundjoin%
\definecolor{currentfill}{rgb}{0.711554,0.033337,0.154485}%
\pgfsetfillcolor{currentfill}%
\pgfsetlinewidth{0.000000pt}%
\definecolor{currentstroke}{rgb}{0.000000,0.000000,0.000000}%
\pgfsetstrokecolor{currentstroke}%
\pgfsetdash{}{0pt}%
\pgfpathmoveto{\pgfqpoint{3.372089in}{2.760957in}}%
\pgfpathlineto{\pgfqpoint{3.622269in}{2.896533in}}%
\pgfpathlineto{\pgfqpoint{3.372089in}{2.760957in}}%
\pgfpathclose%
\pgfusepath{fill}%
\end{pgfscope}%
\begin{pgfscope}%
\pgfpathrectangle{\pgfqpoint{0.017500in}{0.017500in}}{\pgfqpoint{6.218282in}{3.580372in}} %
\pgfusepath{clip}%
\pgfsetbuttcap%
\pgfsetroundjoin%
\definecolor{currentfill}{rgb}{0.711554,0.033337,0.154485}%
\pgfsetfillcolor{currentfill}%
\pgfsetlinewidth{0.000000pt}%
\definecolor{currentstroke}{rgb}{0.000000,0.000000,0.000000}%
\pgfsetstrokecolor{currentstroke}%
\pgfsetdash{}{0pt}%
\pgfpathmoveto{\pgfqpoint{3.372089in}{2.760957in}}%
\pgfpathlineto{\pgfqpoint{3.622269in}{2.896533in}}%
\pgfpathlineto{\pgfqpoint{3.372089in}{2.760957in}}%
\pgfpathclose%
\pgfusepath{fill}%
\end{pgfscope}%
\begin{pgfscope}%
\pgfpathrectangle{\pgfqpoint{0.017500in}{0.017500in}}{\pgfqpoint{6.218282in}{3.580372in}} %
\pgfusepath{clip}%
\pgfsetbuttcap%
\pgfsetroundjoin%
\definecolor{currentfill}{rgb}{0.711554,0.033337,0.154485}%
\pgfsetfillcolor{currentfill}%
\pgfsetlinewidth{0.000000pt}%
\definecolor{currentstroke}{rgb}{0.000000,0.000000,0.000000}%
\pgfsetstrokecolor{currentstroke}%
\pgfsetdash{}{0pt}%
\pgfpathmoveto{\pgfqpoint{3.372089in}{2.760957in}}%
\pgfpathlineto{\pgfqpoint{3.622269in}{2.896533in}}%
\pgfpathlineto{\pgfqpoint{3.372089in}{2.760957in}}%
\pgfpathclose%
\pgfusepath{fill}%
\end{pgfscope}%
\begin{pgfscope}%
\pgfpathrectangle{\pgfqpoint{0.017500in}{0.017500in}}{\pgfqpoint{6.218282in}{3.580372in}} %
\pgfusepath{clip}%
\pgfsetbuttcap%
\pgfsetroundjoin%
\definecolor{currentfill}{rgb}{0.711554,0.033337,0.154485}%
\pgfsetfillcolor{currentfill}%
\pgfsetlinewidth{0.000000pt}%
\definecolor{currentstroke}{rgb}{0.000000,0.000000,0.000000}%
\pgfsetstrokecolor{currentstroke}%
\pgfsetdash{}{0pt}%
\pgfpathmoveto{\pgfqpoint{3.372089in}{2.760957in}}%
\pgfpathlineto{\pgfqpoint{3.622269in}{2.896533in}}%
\pgfpathlineto{\pgfqpoint{3.372089in}{2.760957in}}%
\pgfpathclose%
\pgfusepath{fill}%
\end{pgfscope}%
\begin{pgfscope}%
\pgfpathrectangle{\pgfqpoint{0.017500in}{0.017500in}}{\pgfqpoint{6.218282in}{3.580372in}} %
\pgfusepath{clip}%
\pgfsetbuttcap%
\pgfsetroundjoin%
\definecolor{currentfill}{rgb}{0.711554,0.033337,0.154485}%
\pgfsetfillcolor{currentfill}%
\pgfsetlinewidth{0.000000pt}%
\definecolor{currentstroke}{rgb}{0.000000,0.000000,0.000000}%
\pgfsetstrokecolor{currentstroke}%
\pgfsetdash{}{0pt}%
\pgfpathmoveto{\pgfqpoint{3.372089in}{2.760957in}}%
\pgfpathlineto{\pgfqpoint{3.622269in}{2.896533in}}%
\pgfpathlineto{\pgfqpoint{3.372089in}{2.760957in}}%
\pgfpathclose%
\pgfusepath{fill}%
\end{pgfscope}%
\begin{pgfscope}%
\pgfpathrectangle{\pgfqpoint{0.017500in}{0.017500in}}{\pgfqpoint{6.218282in}{3.580372in}} %
\pgfusepath{clip}%
\pgfsetbuttcap%
\pgfsetroundjoin%
\definecolor{currentfill}{rgb}{0.711554,0.033337,0.154485}%
\pgfsetfillcolor{currentfill}%
\pgfsetlinewidth{0.000000pt}%
\definecolor{currentstroke}{rgb}{0.000000,0.000000,0.000000}%
\pgfsetstrokecolor{currentstroke}%
\pgfsetdash{}{0pt}%
\pgfpathmoveto{\pgfqpoint{3.372089in}{2.760957in}}%
\pgfpathlineto{\pgfqpoint{3.622269in}{2.896533in}}%
\pgfpathlineto{\pgfqpoint{3.372089in}{2.760957in}}%
\pgfpathclose%
\pgfusepath{fill}%
\end{pgfscope}%
\begin{pgfscope}%
\pgfpathrectangle{\pgfqpoint{0.017500in}{0.017500in}}{\pgfqpoint{6.218282in}{3.580372in}} %
\pgfusepath{clip}%
\pgfsetbuttcap%
\pgfsetroundjoin%
\definecolor{currentfill}{rgb}{0.711554,0.033337,0.154485}%
\pgfsetfillcolor{currentfill}%
\pgfsetlinewidth{0.000000pt}%
\definecolor{currentstroke}{rgb}{0.000000,0.000000,0.000000}%
\pgfsetstrokecolor{currentstroke}%
\pgfsetdash{}{0pt}%
\pgfpathmoveto{\pgfqpoint{3.372089in}{2.760957in}}%
\pgfpathlineto{\pgfqpoint{3.622269in}{2.896533in}}%
\pgfpathlineto{\pgfqpoint{3.372089in}{2.760957in}}%
\pgfpathclose%
\pgfusepath{fill}%
\end{pgfscope}%
\begin{pgfscope}%
\pgfpathrectangle{\pgfqpoint{0.017500in}{0.017500in}}{\pgfqpoint{6.218282in}{3.580372in}} %
\pgfusepath{clip}%
\pgfsetbuttcap%
\pgfsetroundjoin%
\definecolor{currentfill}{rgb}{0.711554,0.033337,0.154485}%
\pgfsetfillcolor{currentfill}%
\pgfsetlinewidth{0.000000pt}%
\definecolor{currentstroke}{rgb}{0.000000,0.000000,0.000000}%
\pgfsetstrokecolor{currentstroke}%
\pgfsetdash{}{0pt}%
\pgfpathmoveto{\pgfqpoint{3.372089in}{2.760957in}}%
\pgfpathlineto{\pgfqpoint{3.622269in}{2.896533in}}%
\pgfpathlineto{\pgfqpoint{3.372089in}{2.760957in}}%
\pgfpathclose%
\pgfusepath{fill}%
\end{pgfscope}%
\begin{pgfscope}%
\pgfpathrectangle{\pgfqpoint{0.017500in}{0.017500in}}{\pgfqpoint{6.218282in}{3.580372in}} %
\pgfusepath{clip}%
\pgfsetbuttcap%
\pgfsetroundjoin%
\definecolor{currentfill}{rgb}{0.711554,0.033337,0.154485}%
\pgfsetfillcolor{currentfill}%
\pgfsetlinewidth{0.000000pt}%
\definecolor{currentstroke}{rgb}{0.000000,0.000000,0.000000}%
\pgfsetstrokecolor{currentstroke}%
\pgfsetdash{}{0pt}%
\pgfpathmoveto{\pgfqpoint{3.372089in}{2.760957in}}%
\pgfpathlineto{\pgfqpoint{3.622269in}{2.896533in}}%
\pgfpathlineto{\pgfqpoint{3.372089in}{2.760957in}}%
\pgfpathclose%
\pgfusepath{fill}%
\end{pgfscope}%
\begin{pgfscope}%
\pgfpathrectangle{\pgfqpoint{0.017500in}{0.017500in}}{\pgfqpoint{6.218282in}{3.580372in}} %
\pgfusepath{clip}%
\pgfsetbuttcap%
\pgfsetroundjoin%
\definecolor{currentfill}{rgb}{0.711554,0.033337,0.154485}%
\pgfsetfillcolor{currentfill}%
\pgfsetlinewidth{0.000000pt}%
\definecolor{currentstroke}{rgb}{0.000000,0.000000,0.000000}%
\pgfsetstrokecolor{currentstroke}%
\pgfsetdash{}{0pt}%
\pgfpathmoveto{\pgfqpoint{3.372089in}{2.760957in}}%
\pgfpathlineto{\pgfqpoint{3.622269in}{2.896533in}}%
\pgfpathlineto{\pgfqpoint{3.372089in}{2.760957in}}%
\pgfpathclose%
\pgfusepath{fill}%
\end{pgfscope}%
\begin{pgfscope}%
\pgfpathrectangle{\pgfqpoint{0.017500in}{0.017500in}}{\pgfqpoint{6.218282in}{3.580372in}} %
\pgfusepath{clip}%
\pgfsetbuttcap%
\pgfsetroundjoin%
\definecolor{currentfill}{rgb}{0.711554,0.033337,0.154485}%
\pgfsetfillcolor{currentfill}%
\pgfsetlinewidth{0.000000pt}%
\definecolor{currentstroke}{rgb}{0.000000,0.000000,0.000000}%
\pgfsetstrokecolor{currentstroke}%
\pgfsetdash{}{0pt}%
\pgfpathmoveto{\pgfqpoint{3.372089in}{2.760957in}}%
\pgfpathlineto{\pgfqpoint{3.622269in}{2.896533in}}%
\pgfpathlineto{\pgfqpoint{3.372089in}{2.760957in}}%
\pgfpathclose%
\pgfusepath{fill}%
\end{pgfscope}%
\begin{pgfscope}%
\pgfpathrectangle{\pgfqpoint{0.017500in}{0.017500in}}{\pgfqpoint{6.218282in}{3.580372in}} %
\pgfusepath{clip}%
\pgfsetbuttcap%
\pgfsetroundjoin%
\definecolor{currentfill}{rgb}{0.711554,0.033337,0.154485}%
\pgfsetfillcolor{currentfill}%
\pgfsetlinewidth{0.000000pt}%
\definecolor{currentstroke}{rgb}{0.000000,0.000000,0.000000}%
\pgfsetstrokecolor{currentstroke}%
\pgfsetdash{}{0pt}%
\pgfpathmoveto{\pgfqpoint{3.372089in}{2.760957in}}%
\pgfpathlineto{\pgfqpoint{3.622269in}{2.896533in}}%
\pgfpathlineto{\pgfqpoint{3.372089in}{2.760957in}}%
\pgfpathclose%
\pgfusepath{fill}%
\end{pgfscope}%
\begin{pgfscope}%
\pgfpathrectangle{\pgfqpoint{0.017500in}{0.017500in}}{\pgfqpoint{6.218282in}{3.580372in}} %
\pgfusepath{clip}%
\pgfsetbuttcap%
\pgfsetroundjoin%
\definecolor{currentfill}{rgb}{0.711554,0.033337,0.154485}%
\pgfsetfillcolor{currentfill}%
\pgfsetlinewidth{0.000000pt}%
\definecolor{currentstroke}{rgb}{0.000000,0.000000,0.000000}%
\pgfsetstrokecolor{currentstroke}%
\pgfsetdash{}{0pt}%
\pgfpathmoveto{\pgfqpoint{3.372089in}{2.760957in}}%
\pgfpathlineto{\pgfqpoint{3.622269in}{2.896533in}}%
\pgfpathlineto{\pgfqpoint{3.372089in}{2.760957in}}%
\pgfpathclose%
\pgfusepath{fill}%
\end{pgfscope}%
\begin{pgfscope}%
\pgfpathrectangle{\pgfqpoint{0.017500in}{0.017500in}}{\pgfqpoint{6.218282in}{3.580372in}} %
\pgfusepath{clip}%
\pgfsetbuttcap%
\pgfsetroundjoin%
\definecolor{currentfill}{rgb}{0.711554,0.033337,0.154485}%
\pgfsetfillcolor{currentfill}%
\pgfsetlinewidth{0.000000pt}%
\definecolor{currentstroke}{rgb}{0.000000,0.000000,0.000000}%
\pgfsetstrokecolor{currentstroke}%
\pgfsetdash{}{0pt}%
\pgfpathmoveto{\pgfqpoint{3.372089in}{2.760957in}}%
\pgfpathlineto{\pgfqpoint{3.622269in}{2.896533in}}%
\pgfpathlineto{\pgfqpoint{3.372089in}{2.760957in}}%
\pgfpathclose%
\pgfusepath{fill}%
\end{pgfscope}%
\begin{pgfscope}%
\pgfpathrectangle{\pgfqpoint{0.017500in}{0.017500in}}{\pgfqpoint{6.218282in}{3.580372in}} %
\pgfusepath{clip}%
\pgfsetbuttcap%
\pgfsetroundjoin%
\definecolor{currentfill}{rgb}{0.711554,0.033337,0.154485}%
\pgfsetfillcolor{currentfill}%
\pgfsetlinewidth{0.000000pt}%
\definecolor{currentstroke}{rgb}{0.000000,0.000000,0.000000}%
\pgfsetstrokecolor{currentstroke}%
\pgfsetdash{}{0pt}%
\pgfpathmoveto{\pgfqpoint{3.372089in}{2.760957in}}%
\pgfpathlineto{\pgfqpoint{3.622269in}{2.896533in}}%
\pgfpathlineto{\pgfqpoint{3.372089in}{2.760957in}}%
\pgfpathclose%
\pgfusepath{fill}%
\end{pgfscope}%
\begin{pgfscope}%
\pgfpathrectangle{\pgfqpoint{0.017500in}{0.017500in}}{\pgfqpoint{6.218282in}{3.580372in}} %
\pgfusepath{clip}%
\pgfsetbuttcap%
\pgfsetroundjoin%
\definecolor{currentfill}{rgb}{0.711554,0.033337,0.154485}%
\pgfsetfillcolor{currentfill}%
\pgfsetlinewidth{0.000000pt}%
\definecolor{currentstroke}{rgb}{0.000000,0.000000,0.000000}%
\pgfsetstrokecolor{currentstroke}%
\pgfsetdash{}{0pt}%
\pgfpathmoveto{\pgfqpoint{3.372089in}{2.760957in}}%
\pgfpathlineto{\pgfqpoint{3.622269in}{2.896533in}}%
\pgfpathlineto{\pgfqpoint{3.372089in}{2.760957in}}%
\pgfpathclose%
\pgfusepath{fill}%
\end{pgfscope}%
\begin{pgfscope}%
\pgfpathrectangle{\pgfqpoint{0.017500in}{0.017500in}}{\pgfqpoint{6.218282in}{3.580372in}} %
\pgfusepath{clip}%
\pgfsetbuttcap%
\pgfsetroundjoin%
\definecolor{currentfill}{rgb}{0.711554,0.033337,0.154485}%
\pgfsetfillcolor{currentfill}%
\pgfsetlinewidth{0.000000pt}%
\definecolor{currentstroke}{rgb}{0.000000,0.000000,0.000000}%
\pgfsetstrokecolor{currentstroke}%
\pgfsetdash{}{0pt}%
\pgfpathmoveto{\pgfqpoint{3.372089in}{2.760957in}}%
\pgfpathlineto{\pgfqpoint{3.622269in}{2.896533in}}%
\pgfpathlineto{\pgfqpoint{3.372089in}{2.760957in}}%
\pgfpathclose%
\pgfusepath{fill}%
\end{pgfscope}%
\begin{pgfscope}%
\pgfpathrectangle{\pgfqpoint{0.017500in}{0.017500in}}{\pgfqpoint{6.218282in}{3.580372in}} %
\pgfusepath{clip}%
\pgfsetbuttcap%
\pgfsetroundjoin%
\definecolor{currentfill}{rgb}{0.711554,0.033337,0.154485}%
\pgfsetfillcolor{currentfill}%
\pgfsetlinewidth{0.000000pt}%
\definecolor{currentstroke}{rgb}{0.000000,0.000000,0.000000}%
\pgfsetstrokecolor{currentstroke}%
\pgfsetdash{}{0pt}%
\pgfpathmoveto{\pgfqpoint{3.372089in}{2.760957in}}%
\pgfpathlineto{\pgfqpoint{3.622269in}{2.896533in}}%
\pgfpathlineto{\pgfqpoint{3.372089in}{2.760957in}}%
\pgfpathclose%
\pgfusepath{fill}%
\end{pgfscope}%
\begin{pgfscope}%
\pgfpathrectangle{\pgfqpoint{0.017500in}{0.017500in}}{\pgfqpoint{6.218282in}{3.580372in}} %
\pgfusepath{clip}%
\pgfsetbuttcap%
\pgfsetroundjoin%
\definecolor{currentfill}{rgb}{0.711554,0.033337,0.154485}%
\pgfsetfillcolor{currentfill}%
\pgfsetlinewidth{0.000000pt}%
\definecolor{currentstroke}{rgb}{0.000000,0.000000,0.000000}%
\pgfsetstrokecolor{currentstroke}%
\pgfsetdash{}{0pt}%
\pgfpathmoveto{\pgfqpoint{3.372089in}{2.760957in}}%
\pgfpathlineto{\pgfqpoint{3.622269in}{2.896533in}}%
\pgfpathlineto{\pgfqpoint{3.372089in}{2.760957in}}%
\pgfpathclose%
\pgfusepath{fill}%
\end{pgfscope}%
\begin{pgfscope}%
\pgfpathrectangle{\pgfqpoint{0.017500in}{0.017500in}}{\pgfqpoint{6.218282in}{3.580372in}} %
\pgfusepath{clip}%
\pgfsetbuttcap%
\pgfsetroundjoin%
\definecolor{currentfill}{rgb}{0.711554,0.033337,0.154485}%
\pgfsetfillcolor{currentfill}%
\pgfsetlinewidth{0.000000pt}%
\definecolor{currentstroke}{rgb}{0.000000,0.000000,0.000000}%
\pgfsetstrokecolor{currentstroke}%
\pgfsetdash{}{0pt}%
\pgfpathmoveto{\pgfqpoint{3.372089in}{2.760957in}}%
\pgfpathlineto{\pgfqpoint{3.622269in}{2.896533in}}%
\pgfpathlineto{\pgfqpoint{3.372089in}{2.760957in}}%
\pgfpathclose%
\pgfusepath{fill}%
\end{pgfscope}%
\begin{pgfscope}%
\pgfpathrectangle{\pgfqpoint{0.017500in}{0.017500in}}{\pgfqpoint{6.218282in}{3.580372in}} %
\pgfusepath{clip}%
\pgfsetbuttcap%
\pgfsetroundjoin%
\definecolor{currentfill}{rgb}{0.711554,0.033337,0.154485}%
\pgfsetfillcolor{currentfill}%
\pgfsetlinewidth{0.000000pt}%
\definecolor{currentstroke}{rgb}{0.000000,0.000000,0.000000}%
\pgfsetstrokecolor{currentstroke}%
\pgfsetdash{}{0pt}%
\pgfpathmoveto{\pgfqpoint{3.372089in}{2.760957in}}%
\pgfpathlineto{\pgfqpoint{3.622269in}{2.896533in}}%
\pgfpathlineto{\pgfqpoint{3.372089in}{2.760957in}}%
\pgfpathclose%
\pgfusepath{fill}%
\end{pgfscope}%
\begin{pgfscope}%
\pgfpathrectangle{\pgfqpoint{0.017500in}{0.017500in}}{\pgfqpoint{6.218282in}{3.580372in}} %
\pgfusepath{clip}%
\pgfsetbuttcap%
\pgfsetroundjoin%
\definecolor{currentfill}{rgb}{0.711554,0.033337,0.154485}%
\pgfsetfillcolor{currentfill}%
\pgfsetlinewidth{0.000000pt}%
\definecolor{currentstroke}{rgb}{0.000000,0.000000,0.000000}%
\pgfsetstrokecolor{currentstroke}%
\pgfsetdash{}{0pt}%
\pgfpathmoveto{\pgfqpoint{3.372089in}{2.760957in}}%
\pgfpathlineto{\pgfqpoint{3.622269in}{2.896533in}}%
\pgfpathlineto{\pgfqpoint{3.372089in}{2.760957in}}%
\pgfpathclose%
\pgfusepath{fill}%
\end{pgfscope}%
\begin{pgfscope}%
\pgfpathrectangle{\pgfqpoint{0.017500in}{0.017500in}}{\pgfqpoint{6.218282in}{3.580372in}} %
\pgfusepath{clip}%
\pgfsetbuttcap%
\pgfsetroundjoin%
\definecolor{currentfill}{rgb}{0.711554,0.033337,0.154485}%
\pgfsetfillcolor{currentfill}%
\pgfsetlinewidth{0.000000pt}%
\definecolor{currentstroke}{rgb}{0.000000,0.000000,0.000000}%
\pgfsetstrokecolor{currentstroke}%
\pgfsetdash{}{0pt}%
\pgfpathmoveto{\pgfqpoint{3.372089in}{2.760957in}}%
\pgfpathlineto{\pgfqpoint{3.622269in}{2.896533in}}%
\pgfpathlineto{\pgfqpoint{3.372089in}{2.760957in}}%
\pgfpathclose%
\pgfusepath{fill}%
\end{pgfscope}%
\begin{pgfscope}%
\pgfpathrectangle{\pgfqpoint{0.017500in}{0.017500in}}{\pgfqpoint{6.218282in}{3.580372in}} %
\pgfusepath{clip}%
\pgfsetbuttcap%
\pgfsetroundjoin%
\definecolor{currentfill}{rgb}{0.711554,0.033337,0.154485}%
\pgfsetfillcolor{currentfill}%
\pgfsetlinewidth{0.000000pt}%
\definecolor{currentstroke}{rgb}{0.000000,0.000000,0.000000}%
\pgfsetstrokecolor{currentstroke}%
\pgfsetdash{}{0pt}%
\pgfpathmoveto{\pgfqpoint{3.372089in}{2.760957in}}%
\pgfpathlineto{\pgfqpoint{3.622269in}{2.896533in}}%
\pgfpathlineto{\pgfqpoint{3.372089in}{2.760957in}}%
\pgfpathclose%
\pgfusepath{fill}%
\end{pgfscope}%
\begin{pgfscope}%
\pgfpathrectangle{\pgfqpoint{0.017500in}{0.017500in}}{\pgfqpoint{6.218282in}{3.580372in}} %
\pgfusepath{clip}%
\pgfsetbuttcap%
\pgfsetroundjoin%
\definecolor{currentfill}{rgb}{0.711554,0.033337,0.154485}%
\pgfsetfillcolor{currentfill}%
\pgfsetlinewidth{0.000000pt}%
\definecolor{currentstroke}{rgb}{0.000000,0.000000,0.000000}%
\pgfsetstrokecolor{currentstroke}%
\pgfsetdash{}{0pt}%
\pgfpathmoveto{\pgfqpoint{3.372089in}{2.760957in}}%
\pgfpathlineto{\pgfqpoint{3.622269in}{2.896533in}}%
\pgfpathlineto{\pgfqpoint{3.372089in}{2.760957in}}%
\pgfpathclose%
\pgfusepath{fill}%
\end{pgfscope}%
\begin{pgfscope}%
\pgfpathrectangle{\pgfqpoint{0.017500in}{0.017500in}}{\pgfqpoint{6.218282in}{3.580372in}} %
\pgfusepath{clip}%
\pgfsetbuttcap%
\pgfsetroundjoin%
\definecolor{currentfill}{rgb}{0.809329,0.852974,0.922323}%
\pgfsetfillcolor{currentfill}%
\pgfsetlinewidth{0.000000pt}%
\definecolor{currentstroke}{rgb}{0.000000,0.000000,0.000000}%
\pgfsetstrokecolor{currentstroke}%
\pgfsetdash{}{0pt}%
\pgfpathmoveto{\pgfqpoint{3.622269in}{2.896533in}}%
\pgfpathlineto{\pgfqpoint{4.063757in}{2.816074in}}%
\pgfpathlineto{\pgfqpoint{2.789661in}{0.820654in}}%
\pgfpathlineto{\pgfqpoint{2.349703in}{0.878379in}}%
\pgfpathlineto{\pgfqpoint{3.622269in}{2.896533in}}%
\pgfpathclose%
\pgfusepath{fill}%
\end{pgfscope}%
\begin{pgfscope}%
\pgfpathrectangle{\pgfqpoint{0.017500in}{0.017500in}}{\pgfqpoint{6.218282in}{3.580372in}} %
\pgfusepath{clip}%
\pgfsetbuttcap%
\pgfsetroundjoin%
\definecolor{currentfill}{rgb}{0.809329,0.852974,0.922323}%
\pgfsetfillcolor{currentfill}%
\pgfsetlinewidth{0.000000pt}%
\definecolor{currentstroke}{rgb}{0.000000,0.000000,0.000000}%
\pgfsetstrokecolor{currentstroke}%
\pgfsetdash{}{0pt}%
\pgfpathmoveto{\pgfqpoint{3.622269in}{2.896533in}}%
\pgfpathlineto{\pgfqpoint{4.063757in}{2.816074in}}%
\pgfpathlineto{\pgfqpoint{2.789661in}{0.820654in}}%
\pgfpathlineto{\pgfqpoint{2.349703in}{0.878379in}}%
\pgfpathlineto{\pgfqpoint{3.622269in}{2.896533in}}%
\pgfpathclose%
\pgfusepath{fill}%
\end{pgfscope}%
\begin{pgfscope}%
\pgfpathrectangle{\pgfqpoint{0.017500in}{0.017500in}}{\pgfqpoint{6.218282in}{3.580372in}} %
\pgfusepath{clip}%
\pgfsetbuttcap%
\pgfsetroundjoin%
\definecolor{currentfill}{rgb}{0.809329,0.852974,0.922323}%
\pgfsetfillcolor{currentfill}%
\pgfsetlinewidth{0.000000pt}%
\definecolor{currentstroke}{rgb}{0.000000,0.000000,0.000000}%
\pgfsetstrokecolor{currentstroke}%
\pgfsetdash{}{0pt}%
\pgfpathmoveto{\pgfqpoint{3.622269in}{2.896533in}}%
\pgfpathlineto{\pgfqpoint{4.063757in}{2.816074in}}%
\pgfpathlineto{\pgfqpoint{2.789661in}{0.820654in}}%
\pgfpathlineto{\pgfqpoint{2.349703in}{0.878379in}}%
\pgfpathlineto{\pgfqpoint{3.622269in}{2.896533in}}%
\pgfpathclose%
\pgfusepath{fill}%
\end{pgfscope}%
\begin{pgfscope}%
\pgfpathrectangle{\pgfqpoint{0.017500in}{0.017500in}}{\pgfqpoint{6.218282in}{3.580372in}} %
\pgfusepath{clip}%
\pgfsetbuttcap%
\pgfsetroundjoin%
\definecolor{currentfill}{rgb}{0.809329,0.852974,0.922323}%
\pgfsetfillcolor{currentfill}%
\pgfsetlinewidth{0.000000pt}%
\definecolor{currentstroke}{rgb}{0.000000,0.000000,0.000000}%
\pgfsetstrokecolor{currentstroke}%
\pgfsetdash{}{0pt}%
\pgfpathmoveto{\pgfqpoint{3.622269in}{2.896533in}}%
\pgfpathlineto{\pgfqpoint{4.063757in}{2.816074in}}%
\pgfpathlineto{\pgfqpoint{2.789661in}{0.820654in}}%
\pgfpathlineto{\pgfqpoint{2.349703in}{0.878379in}}%
\pgfpathlineto{\pgfqpoint{3.622269in}{2.896533in}}%
\pgfpathclose%
\pgfusepath{fill}%
\end{pgfscope}%
\begin{pgfscope}%
\pgfpathrectangle{\pgfqpoint{0.017500in}{0.017500in}}{\pgfqpoint{6.218282in}{3.580372in}} %
\pgfusepath{clip}%
\pgfsetbuttcap%
\pgfsetroundjoin%
\definecolor{currentfill}{rgb}{0.809329,0.852974,0.922323}%
\pgfsetfillcolor{currentfill}%
\pgfsetlinewidth{0.000000pt}%
\definecolor{currentstroke}{rgb}{0.000000,0.000000,0.000000}%
\pgfsetstrokecolor{currentstroke}%
\pgfsetdash{}{0pt}%
\pgfpathmoveto{\pgfqpoint{3.622269in}{2.896533in}}%
\pgfpathlineto{\pgfqpoint{4.063757in}{2.816074in}}%
\pgfpathlineto{\pgfqpoint{2.789661in}{0.820654in}}%
\pgfpathlineto{\pgfqpoint{2.349703in}{0.878379in}}%
\pgfpathlineto{\pgfqpoint{3.622269in}{2.896533in}}%
\pgfpathclose%
\pgfusepath{fill}%
\end{pgfscope}%
\begin{pgfscope}%
\pgfpathrectangle{\pgfqpoint{0.017500in}{0.017500in}}{\pgfqpoint{6.218282in}{3.580372in}} %
\pgfusepath{clip}%
\pgfsetbuttcap%
\pgfsetroundjoin%
\definecolor{currentfill}{rgb}{0.711554,0.033337,0.154485}%
\pgfsetfillcolor{currentfill}%
\pgfsetlinewidth{0.000000pt}%
\definecolor{currentstroke}{rgb}{0.000000,0.000000,0.000000}%
\pgfsetstrokecolor{currentstroke}%
\pgfsetdash{}{0pt}%
\pgfpathmoveto{\pgfqpoint{3.372089in}{2.760957in}}%
\pgfpathlineto{\pgfqpoint{3.817122in}{2.696128in}}%
\pgfpathlineto{\pgfqpoint{4.063757in}{2.816074in}}%
\pgfpathlineto{\pgfqpoint{3.622269in}{2.896533in}}%
\pgfpathlineto{\pgfqpoint{3.372089in}{2.760957in}}%
\pgfpathclose%
\pgfusepath{fill}%
\end{pgfscope}%
\begin{pgfscope}%
\pgfpathrectangle{\pgfqpoint{0.017500in}{0.017500in}}{\pgfqpoint{6.218282in}{3.580372in}} %
\pgfusepath{clip}%
\pgfsetbuttcap%
\pgfsetroundjoin%
\definecolor{currentfill}{rgb}{0.711554,0.033337,0.154485}%
\pgfsetfillcolor{currentfill}%
\pgfsetlinewidth{0.000000pt}%
\definecolor{currentstroke}{rgb}{0.000000,0.000000,0.000000}%
\pgfsetstrokecolor{currentstroke}%
\pgfsetdash{}{0pt}%
\pgfpathmoveto{\pgfqpoint{3.372089in}{2.760957in}}%
\pgfpathlineto{\pgfqpoint{3.817122in}{2.696128in}}%
\pgfpathlineto{\pgfqpoint{4.063757in}{2.816074in}}%
\pgfpathlineto{\pgfqpoint{3.622269in}{2.896533in}}%
\pgfpathlineto{\pgfqpoint{3.372089in}{2.760957in}}%
\pgfpathclose%
\pgfusepath{fill}%
\end{pgfscope}%
\begin{pgfscope}%
\pgfpathrectangle{\pgfqpoint{0.017500in}{0.017500in}}{\pgfqpoint{6.218282in}{3.580372in}} %
\pgfusepath{clip}%
\pgfsetbuttcap%
\pgfsetroundjoin%
\definecolor{currentfill}{rgb}{0.711554,0.033337,0.154485}%
\pgfsetfillcolor{currentfill}%
\pgfsetlinewidth{0.000000pt}%
\definecolor{currentstroke}{rgb}{0.000000,0.000000,0.000000}%
\pgfsetstrokecolor{currentstroke}%
\pgfsetdash{}{0pt}%
\pgfpathmoveto{\pgfqpoint{3.372089in}{2.760957in}}%
\pgfpathlineto{\pgfqpoint{3.817122in}{2.696128in}}%
\pgfpathlineto{\pgfqpoint{4.063757in}{2.816074in}}%
\pgfpathlineto{\pgfqpoint{3.622269in}{2.896533in}}%
\pgfpathlineto{\pgfqpoint{3.372089in}{2.760957in}}%
\pgfpathclose%
\pgfusepath{fill}%
\end{pgfscope}%
\begin{pgfscope}%
\pgfpathrectangle{\pgfqpoint{0.017500in}{0.017500in}}{\pgfqpoint{6.218282in}{3.580372in}} %
\pgfusepath{clip}%
\pgfsetbuttcap%
\pgfsetroundjoin%
\definecolor{currentfill}{rgb}{0.711554,0.033337,0.154485}%
\pgfsetfillcolor{currentfill}%
\pgfsetlinewidth{0.000000pt}%
\definecolor{currentstroke}{rgb}{0.000000,0.000000,0.000000}%
\pgfsetstrokecolor{currentstroke}%
\pgfsetdash{}{0pt}%
\pgfpathmoveto{\pgfqpoint{3.372089in}{2.760957in}}%
\pgfpathlineto{\pgfqpoint{3.817122in}{2.696128in}}%
\pgfpathlineto{\pgfqpoint{4.063757in}{2.816074in}}%
\pgfpathlineto{\pgfqpoint{3.622269in}{2.896533in}}%
\pgfpathlineto{\pgfqpoint{3.372089in}{2.760957in}}%
\pgfpathclose%
\pgfusepath{fill}%
\end{pgfscope}%
\begin{pgfscope}%
\pgfpathrectangle{\pgfqpoint{0.017500in}{0.017500in}}{\pgfqpoint{6.218282in}{3.580372in}} %
\pgfusepath{clip}%
\pgfsetbuttcap%
\pgfsetroundjoin%
\definecolor{currentfill}{rgb}{0.711554,0.033337,0.154485}%
\pgfsetfillcolor{currentfill}%
\pgfsetlinewidth{0.000000pt}%
\definecolor{currentstroke}{rgb}{0.000000,0.000000,0.000000}%
\pgfsetstrokecolor{currentstroke}%
\pgfsetdash{}{0pt}%
\pgfpathmoveto{\pgfqpoint{3.372089in}{2.760957in}}%
\pgfpathlineto{\pgfqpoint{3.817122in}{2.696128in}}%
\pgfpathlineto{\pgfqpoint{4.063757in}{2.816074in}}%
\pgfpathlineto{\pgfqpoint{3.622269in}{2.896533in}}%
\pgfpathlineto{\pgfqpoint{3.372089in}{2.760957in}}%
\pgfpathclose%
\pgfusepath{fill}%
\end{pgfscope}%
\begin{pgfscope}%
\pgfpathrectangle{\pgfqpoint{0.017500in}{0.017500in}}{\pgfqpoint{6.218282in}{3.580372in}} %
\pgfusepath{clip}%
\pgfsetbuttcap%
\pgfsetroundjoin%
\definecolor{currentfill}{rgb}{0.711554,0.033337,0.154485}%
\pgfsetfillcolor{currentfill}%
\pgfsetlinewidth{0.000000pt}%
\definecolor{currentstroke}{rgb}{0.000000,0.000000,0.000000}%
\pgfsetstrokecolor{currentstroke}%
\pgfsetdash{}{0pt}%
\pgfpathmoveto{\pgfqpoint{3.372089in}{2.760957in}}%
\pgfpathlineto{\pgfqpoint{3.817122in}{2.696128in}}%
\pgfpathlineto{\pgfqpoint{4.063757in}{2.816074in}}%
\pgfpathlineto{\pgfqpoint{3.622269in}{2.896533in}}%
\pgfpathlineto{\pgfqpoint{3.372089in}{2.760957in}}%
\pgfpathclose%
\pgfusepath{fill}%
\end{pgfscope}%
\begin{pgfscope}%
\pgfpathrectangle{\pgfqpoint{0.017500in}{0.017500in}}{\pgfqpoint{6.218282in}{3.580372in}} %
\pgfusepath{clip}%
\pgfsetbuttcap%
\pgfsetroundjoin%
\definecolor{currentfill}{rgb}{0.229806,0.298718,0.753683}%
\pgfsetfillcolor{currentfill}%
\pgfsetlinewidth{0.000000pt}%
\definecolor{currentstroke}{rgb}{0.000000,0.000000,0.000000}%
\pgfsetstrokecolor{currentstroke}%
\pgfsetdash{}{0pt}%
\pgfpathmoveto{\pgfqpoint{1.486672in}{1.030836in}}%
\pgfpathlineto{\pgfqpoint{1.702428in}{2.361240in}}%
\pgfpathlineto{\pgfqpoint{1.486672in}{1.030836in}}%
\pgfpathclose%
\pgfusepath{fill}%
\end{pgfscope}%
\begin{pgfscope}%
\pgfpathrectangle{\pgfqpoint{0.017500in}{0.017500in}}{\pgfqpoint{6.218282in}{3.580372in}} %
\pgfusepath{clip}%
\pgfsetbuttcap%
\pgfsetroundjoin%
\definecolor{currentfill}{rgb}{0.229806,0.298718,0.753683}%
\pgfsetfillcolor{currentfill}%
\pgfsetlinewidth{0.000000pt}%
\definecolor{currentstroke}{rgb}{0.000000,0.000000,0.000000}%
\pgfsetstrokecolor{currentstroke}%
\pgfsetdash{}{0pt}%
\pgfpathmoveto{\pgfqpoint{1.486672in}{1.030836in}}%
\pgfpathlineto{\pgfqpoint{1.702428in}{2.361240in}}%
\pgfpathlineto{\pgfqpoint{1.486672in}{1.030836in}}%
\pgfpathclose%
\pgfusepath{fill}%
\end{pgfscope}%
\begin{pgfscope}%
\pgfpathrectangle{\pgfqpoint{0.017500in}{0.017500in}}{\pgfqpoint{6.218282in}{3.580372in}} %
\pgfusepath{clip}%
\pgfsetbuttcap%
\pgfsetroundjoin%
\definecolor{currentfill}{rgb}{0.229806,0.298718,0.753683}%
\pgfsetfillcolor{currentfill}%
\pgfsetlinewidth{0.000000pt}%
\definecolor{currentstroke}{rgb}{0.000000,0.000000,0.000000}%
\pgfsetstrokecolor{currentstroke}%
\pgfsetdash{}{0pt}%
\pgfpathmoveto{\pgfqpoint{1.486672in}{1.030836in}}%
\pgfpathlineto{\pgfqpoint{1.702428in}{2.361240in}}%
\pgfpathlineto{\pgfqpoint{1.486672in}{1.030836in}}%
\pgfpathclose%
\pgfusepath{fill}%
\end{pgfscope}%
\begin{pgfscope}%
\pgfpathrectangle{\pgfqpoint{0.017500in}{0.017500in}}{\pgfqpoint{6.218282in}{3.580372in}} %
\pgfusepath{clip}%
\pgfsetbuttcap%
\pgfsetroundjoin%
\definecolor{currentfill}{rgb}{0.229806,0.298718,0.753683}%
\pgfsetfillcolor{currentfill}%
\pgfsetlinewidth{0.000000pt}%
\definecolor{currentstroke}{rgb}{0.000000,0.000000,0.000000}%
\pgfsetstrokecolor{currentstroke}%
\pgfsetdash{}{0pt}%
\pgfpathmoveto{\pgfqpoint{1.486672in}{1.030836in}}%
\pgfpathlineto{\pgfqpoint{1.702428in}{2.361240in}}%
\pgfpathlineto{\pgfqpoint{1.486672in}{1.030836in}}%
\pgfpathclose%
\pgfusepath{fill}%
\end{pgfscope}%
\begin{pgfscope}%
\pgfpathrectangle{\pgfqpoint{0.017500in}{0.017500in}}{\pgfqpoint{6.218282in}{3.580372in}} %
\pgfusepath{clip}%
\pgfsetbuttcap%
\pgfsetroundjoin%
\definecolor{currentfill}{rgb}{0.229806,0.298718,0.753683}%
\pgfsetfillcolor{currentfill}%
\pgfsetlinewidth{0.000000pt}%
\definecolor{currentstroke}{rgb}{0.000000,0.000000,0.000000}%
\pgfsetstrokecolor{currentstroke}%
\pgfsetdash{}{0pt}%
\pgfpathmoveto{\pgfqpoint{1.486672in}{1.030836in}}%
\pgfpathlineto{\pgfqpoint{1.702428in}{2.361240in}}%
\pgfpathlineto{\pgfqpoint{1.486672in}{1.030836in}}%
\pgfpathclose%
\pgfusepath{fill}%
\end{pgfscope}%
\begin{pgfscope}%
\pgfpathrectangle{\pgfqpoint{0.017500in}{0.017500in}}{\pgfqpoint{6.218282in}{3.580372in}} %
\pgfusepath{clip}%
\pgfsetbuttcap%
\pgfsetroundjoin%
\definecolor{currentfill}{rgb}{0.229806,0.298718,0.753683}%
\pgfsetfillcolor{currentfill}%
\pgfsetlinewidth{0.000000pt}%
\definecolor{currentstroke}{rgb}{0.000000,0.000000,0.000000}%
\pgfsetstrokecolor{currentstroke}%
\pgfsetdash{}{0pt}%
\pgfpathmoveto{\pgfqpoint{1.486672in}{1.030836in}}%
\pgfpathlineto{\pgfqpoint{1.702428in}{2.361240in}}%
\pgfpathlineto{\pgfqpoint{1.486672in}{1.030836in}}%
\pgfpathclose%
\pgfusepath{fill}%
\end{pgfscope}%
\begin{pgfscope}%
\pgfpathrectangle{\pgfqpoint{0.017500in}{0.017500in}}{\pgfqpoint{6.218282in}{3.580372in}} %
\pgfusepath{clip}%
\pgfsetbuttcap%
\pgfsetroundjoin%
\definecolor{currentfill}{rgb}{0.229806,0.298718,0.753683}%
\pgfsetfillcolor{currentfill}%
\pgfsetlinewidth{0.000000pt}%
\definecolor{currentstroke}{rgb}{0.000000,0.000000,0.000000}%
\pgfsetstrokecolor{currentstroke}%
\pgfsetdash{}{0pt}%
\pgfpathmoveto{\pgfqpoint{1.486672in}{1.030836in}}%
\pgfpathlineto{\pgfqpoint{1.702428in}{2.361240in}}%
\pgfpathlineto{\pgfqpoint{1.486672in}{1.030836in}}%
\pgfpathclose%
\pgfusepath{fill}%
\end{pgfscope}%
\begin{pgfscope}%
\pgfpathrectangle{\pgfqpoint{0.017500in}{0.017500in}}{\pgfqpoint{6.218282in}{3.580372in}} %
\pgfusepath{clip}%
\pgfsetbuttcap%
\pgfsetroundjoin%
\definecolor{currentfill}{rgb}{0.229806,0.298718,0.753683}%
\pgfsetfillcolor{currentfill}%
\pgfsetlinewidth{0.000000pt}%
\definecolor{currentstroke}{rgb}{0.000000,0.000000,0.000000}%
\pgfsetstrokecolor{currentstroke}%
\pgfsetdash{}{0pt}%
\pgfpathmoveto{\pgfqpoint{1.486672in}{1.030836in}}%
\pgfpathlineto{\pgfqpoint{1.702428in}{2.361240in}}%
\pgfpathlineto{\pgfqpoint{1.486672in}{1.030836in}}%
\pgfpathclose%
\pgfusepath{fill}%
\end{pgfscope}%
\begin{pgfscope}%
\pgfpathrectangle{\pgfqpoint{0.017500in}{0.017500in}}{\pgfqpoint{6.218282in}{3.580372in}} %
\pgfusepath{clip}%
\pgfsetbuttcap%
\pgfsetroundjoin%
\definecolor{currentfill}{rgb}{0.229806,0.298718,0.753683}%
\pgfsetfillcolor{currentfill}%
\pgfsetlinewidth{0.000000pt}%
\definecolor{currentstroke}{rgb}{0.000000,0.000000,0.000000}%
\pgfsetstrokecolor{currentstroke}%
\pgfsetdash{}{0pt}%
\pgfpathmoveto{\pgfqpoint{1.486672in}{1.030836in}}%
\pgfpathlineto{\pgfqpoint{1.702428in}{2.361240in}}%
\pgfpathlineto{\pgfqpoint{1.486672in}{1.030836in}}%
\pgfpathclose%
\pgfusepath{fill}%
\end{pgfscope}%
\begin{pgfscope}%
\pgfpathrectangle{\pgfqpoint{0.017500in}{0.017500in}}{\pgfqpoint{6.218282in}{3.580372in}} %
\pgfusepath{clip}%
\pgfsetbuttcap%
\pgfsetroundjoin%
\definecolor{currentfill}{rgb}{0.229806,0.298718,0.753683}%
\pgfsetfillcolor{currentfill}%
\pgfsetlinewidth{0.000000pt}%
\definecolor{currentstroke}{rgb}{0.000000,0.000000,0.000000}%
\pgfsetstrokecolor{currentstroke}%
\pgfsetdash{}{0pt}%
\pgfpathmoveto{\pgfqpoint{1.486672in}{1.030836in}}%
\pgfpathlineto{\pgfqpoint{1.702428in}{2.361240in}}%
\pgfpathlineto{\pgfqpoint{1.486672in}{1.030836in}}%
\pgfpathclose%
\pgfusepath{fill}%
\end{pgfscope}%
\begin{pgfscope}%
\pgfpathrectangle{\pgfqpoint{0.017500in}{0.017500in}}{\pgfqpoint{6.218282in}{3.580372in}} %
\pgfusepath{clip}%
\pgfsetbuttcap%
\pgfsetroundjoin%
\definecolor{currentfill}{rgb}{0.229806,0.298718,0.753683}%
\pgfsetfillcolor{currentfill}%
\pgfsetlinewidth{0.000000pt}%
\definecolor{currentstroke}{rgb}{0.000000,0.000000,0.000000}%
\pgfsetstrokecolor{currentstroke}%
\pgfsetdash{}{0pt}%
\pgfpathmoveto{\pgfqpoint{1.486672in}{1.030836in}}%
\pgfpathlineto{\pgfqpoint{1.702428in}{2.361240in}}%
\pgfpathlineto{\pgfqpoint{1.486672in}{1.030836in}}%
\pgfpathclose%
\pgfusepath{fill}%
\end{pgfscope}%
\begin{pgfscope}%
\pgfpathrectangle{\pgfqpoint{0.017500in}{0.017500in}}{\pgfqpoint{6.218282in}{3.580372in}} %
\pgfusepath{clip}%
\pgfsetbuttcap%
\pgfsetroundjoin%
\definecolor{currentfill}{rgb}{0.229806,0.298718,0.753683}%
\pgfsetfillcolor{currentfill}%
\pgfsetlinewidth{0.000000pt}%
\definecolor{currentstroke}{rgb}{0.000000,0.000000,0.000000}%
\pgfsetstrokecolor{currentstroke}%
\pgfsetdash{}{0pt}%
\pgfpathmoveto{\pgfqpoint{1.486672in}{1.030836in}}%
\pgfpathlineto{\pgfqpoint{1.702428in}{2.361240in}}%
\pgfpathlineto{\pgfqpoint{1.486672in}{1.030836in}}%
\pgfpathclose%
\pgfusepath{fill}%
\end{pgfscope}%
\begin{pgfscope}%
\pgfpathrectangle{\pgfqpoint{0.017500in}{0.017500in}}{\pgfqpoint{6.218282in}{3.580372in}} %
\pgfusepath{clip}%
\pgfsetbuttcap%
\pgfsetroundjoin%
\definecolor{currentfill}{rgb}{0.229806,0.298718,0.753683}%
\pgfsetfillcolor{currentfill}%
\pgfsetlinewidth{0.000000pt}%
\definecolor{currentstroke}{rgb}{0.000000,0.000000,0.000000}%
\pgfsetstrokecolor{currentstroke}%
\pgfsetdash{}{0pt}%
\pgfpathmoveto{\pgfqpoint{1.486672in}{1.030836in}}%
\pgfpathlineto{\pgfqpoint{1.702428in}{2.361240in}}%
\pgfpathlineto{\pgfqpoint{1.486672in}{1.030836in}}%
\pgfpathclose%
\pgfusepath{fill}%
\end{pgfscope}%
\begin{pgfscope}%
\pgfpathrectangle{\pgfqpoint{0.017500in}{0.017500in}}{\pgfqpoint{6.218282in}{3.580372in}} %
\pgfusepath{clip}%
\pgfsetbuttcap%
\pgfsetroundjoin%
\definecolor{currentfill}{rgb}{0.229806,0.298718,0.753683}%
\pgfsetfillcolor{currentfill}%
\pgfsetlinewidth{0.000000pt}%
\definecolor{currentstroke}{rgb}{0.000000,0.000000,0.000000}%
\pgfsetstrokecolor{currentstroke}%
\pgfsetdash{}{0pt}%
\pgfpathmoveto{\pgfqpoint{1.486672in}{1.030836in}}%
\pgfpathlineto{\pgfqpoint{1.702428in}{2.361240in}}%
\pgfpathlineto{\pgfqpoint{1.486672in}{1.030836in}}%
\pgfpathclose%
\pgfusepath{fill}%
\end{pgfscope}%
\begin{pgfscope}%
\pgfpathrectangle{\pgfqpoint{0.017500in}{0.017500in}}{\pgfqpoint{6.218282in}{3.580372in}} %
\pgfusepath{clip}%
\pgfsetbuttcap%
\pgfsetroundjoin%
\definecolor{currentfill}{rgb}{0.229806,0.298718,0.753683}%
\pgfsetfillcolor{currentfill}%
\pgfsetlinewidth{0.000000pt}%
\definecolor{currentstroke}{rgb}{0.000000,0.000000,0.000000}%
\pgfsetstrokecolor{currentstroke}%
\pgfsetdash{}{0pt}%
\pgfpathmoveto{\pgfqpoint{1.486672in}{1.030836in}}%
\pgfpathlineto{\pgfqpoint{1.702428in}{2.361240in}}%
\pgfpathlineto{\pgfqpoint{1.486672in}{1.030836in}}%
\pgfpathclose%
\pgfusepath{fill}%
\end{pgfscope}%
\begin{pgfscope}%
\pgfpathrectangle{\pgfqpoint{0.017500in}{0.017500in}}{\pgfqpoint{6.218282in}{3.580372in}} %
\pgfusepath{clip}%
\pgfsetbuttcap%
\pgfsetroundjoin%
\definecolor{currentfill}{rgb}{0.229806,0.298718,0.753683}%
\pgfsetfillcolor{currentfill}%
\pgfsetlinewidth{0.000000pt}%
\definecolor{currentstroke}{rgb}{0.000000,0.000000,0.000000}%
\pgfsetstrokecolor{currentstroke}%
\pgfsetdash{}{0pt}%
\pgfpathmoveto{\pgfqpoint{1.486672in}{1.030836in}}%
\pgfpathlineto{\pgfqpoint{1.702428in}{2.361240in}}%
\pgfpathlineto{\pgfqpoint{1.486672in}{1.030836in}}%
\pgfpathclose%
\pgfusepath{fill}%
\end{pgfscope}%
\begin{pgfscope}%
\pgfpathrectangle{\pgfqpoint{0.017500in}{0.017500in}}{\pgfqpoint{6.218282in}{3.580372in}} %
\pgfusepath{clip}%
\pgfsetbuttcap%
\pgfsetroundjoin%
\definecolor{currentfill}{rgb}{0.229806,0.298718,0.753683}%
\pgfsetfillcolor{currentfill}%
\pgfsetlinewidth{0.000000pt}%
\definecolor{currentstroke}{rgb}{0.000000,0.000000,0.000000}%
\pgfsetstrokecolor{currentstroke}%
\pgfsetdash{}{0pt}%
\pgfpathmoveto{\pgfqpoint{1.486672in}{1.030836in}}%
\pgfpathlineto{\pgfqpoint{1.702428in}{2.361240in}}%
\pgfpathlineto{\pgfqpoint{1.486672in}{1.030836in}}%
\pgfpathclose%
\pgfusepath{fill}%
\end{pgfscope}%
\begin{pgfscope}%
\pgfpathrectangle{\pgfqpoint{0.017500in}{0.017500in}}{\pgfqpoint{6.218282in}{3.580372in}} %
\pgfusepath{clip}%
\pgfsetbuttcap%
\pgfsetroundjoin%
\definecolor{currentfill}{rgb}{0.229806,0.298718,0.753683}%
\pgfsetfillcolor{currentfill}%
\pgfsetlinewidth{0.000000pt}%
\definecolor{currentstroke}{rgb}{0.000000,0.000000,0.000000}%
\pgfsetstrokecolor{currentstroke}%
\pgfsetdash{}{0pt}%
\pgfpathmoveto{\pgfqpoint{1.486672in}{1.030836in}}%
\pgfpathlineto{\pgfqpoint{1.702428in}{2.361240in}}%
\pgfpathlineto{\pgfqpoint{1.486672in}{1.030836in}}%
\pgfpathclose%
\pgfusepath{fill}%
\end{pgfscope}%
\begin{pgfscope}%
\pgfpathrectangle{\pgfqpoint{0.017500in}{0.017500in}}{\pgfqpoint{6.218282in}{3.580372in}} %
\pgfusepath{clip}%
\pgfsetbuttcap%
\pgfsetroundjoin%
\definecolor{currentfill}{rgb}{0.229806,0.298718,0.753683}%
\pgfsetfillcolor{currentfill}%
\pgfsetlinewidth{0.000000pt}%
\definecolor{currentstroke}{rgb}{0.000000,0.000000,0.000000}%
\pgfsetstrokecolor{currentstroke}%
\pgfsetdash{}{0pt}%
\pgfpathmoveto{\pgfqpoint{1.486672in}{1.030836in}}%
\pgfpathlineto{\pgfqpoint{1.702428in}{2.361240in}}%
\pgfpathlineto{\pgfqpoint{1.486672in}{1.030836in}}%
\pgfpathclose%
\pgfusepath{fill}%
\end{pgfscope}%
\begin{pgfscope}%
\pgfpathrectangle{\pgfqpoint{0.017500in}{0.017500in}}{\pgfqpoint{6.218282in}{3.580372in}} %
\pgfusepath{clip}%
\pgfsetbuttcap%
\pgfsetroundjoin%
\definecolor{currentfill}{rgb}{0.229806,0.298718,0.753683}%
\pgfsetfillcolor{currentfill}%
\pgfsetlinewidth{0.000000pt}%
\definecolor{currentstroke}{rgb}{0.000000,0.000000,0.000000}%
\pgfsetstrokecolor{currentstroke}%
\pgfsetdash{}{0pt}%
\pgfpathmoveto{\pgfqpoint{1.486672in}{1.030836in}}%
\pgfpathlineto{\pgfqpoint{1.702428in}{2.361240in}}%
\pgfpathlineto{\pgfqpoint{1.486672in}{1.030836in}}%
\pgfpathclose%
\pgfusepath{fill}%
\end{pgfscope}%
\begin{pgfscope}%
\pgfpathrectangle{\pgfqpoint{0.017500in}{0.017500in}}{\pgfqpoint{6.218282in}{3.580372in}} %
\pgfusepath{clip}%
\pgfsetbuttcap%
\pgfsetroundjoin%
\definecolor{currentfill}{rgb}{0.229806,0.298718,0.753683}%
\pgfsetfillcolor{currentfill}%
\pgfsetlinewidth{0.000000pt}%
\definecolor{currentstroke}{rgb}{0.000000,0.000000,0.000000}%
\pgfsetstrokecolor{currentstroke}%
\pgfsetdash{}{0pt}%
\pgfpathmoveto{\pgfqpoint{1.486672in}{1.030836in}}%
\pgfpathlineto{\pgfqpoint{1.702428in}{2.361240in}}%
\pgfpathlineto{\pgfqpoint{1.486672in}{1.030836in}}%
\pgfpathclose%
\pgfusepath{fill}%
\end{pgfscope}%
\begin{pgfscope}%
\pgfpathrectangle{\pgfqpoint{0.017500in}{0.017500in}}{\pgfqpoint{6.218282in}{3.580372in}} %
\pgfusepath{clip}%
\pgfsetbuttcap%
\pgfsetroundjoin%
\definecolor{currentfill}{rgb}{0.229806,0.298718,0.753683}%
\pgfsetfillcolor{currentfill}%
\pgfsetlinewidth{0.000000pt}%
\definecolor{currentstroke}{rgb}{0.000000,0.000000,0.000000}%
\pgfsetstrokecolor{currentstroke}%
\pgfsetdash{}{0pt}%
\pgfpathmoveto{\pgfqpoint{1.486672in}{1.030836in}}%
\pgfpathlineto{\pgfqpoint{1.702428in}{2.361240in}}%
\pgfpathlineto{\pgfqpoint{1.486672in}{1.030836in}}%
\pgfpathclose%
\pgfusepath{fill}%
\end{pgfscope}%
\begin{pgfscope}%
\pgfpathrectangle{\pgfqpoint{0.017500in}{0.017500in}}{\pgfqpoint{6.218282in}{3.580372in}} %
\pgfusepath{clip}%
\pgfsetbuttcap%
\pgfsetroundjoin%
\definecolor{currentfill}{rgb}{0.229806,0.298718,0.753683}%
\pgfsetfillcolor{currentfill}%
\pgfsetlinewidth{0.000000pt}%
\definecolor{currentstroke}{rgb}{0.000000,0.000000,0.000000}%
\pgfsetstrokecolor{currentstroke}%
\pgfsetdash{}{0pt}%
\pgfpathmoveto{\pgfqpoint{1.486672in}{1.030836in}}%
\pgfpathlineto{\pgfqpoint{1.702428in}{2.361240in}}%
\pgfpathlineto{\pgfqpoint{1.486672in}{1.030836in}}%
\pgfpathclose%
\pgfusepath{fill}%
\end{pgfscope}%
\begin{pgfscope}%
\pgfpathrectangle{\pgfqpoint{0.017500in}{0.017500in}}{\pgfqpoint{6.218282in}{3.580372in}} %
\pgfusepath{clip}%
\pgfsetbuttcap%
\pgfsetroundjoin%
\definecolor{currentfill}{rgb}{0.229806,0.298718,0.753683}%
\pgfsetfillcolor{currentfill}%
\pgfsetlinewidth{0.000000pt}%
\definecolor{currentstroke}{rgb}{0.000000,0.000000,0.000000}%
\pgfsetstrokecolor{currentstroke}%
\pgfsetdash{}{0pt}%
\pgfpathmoveto{\pgfqpoint{1.486672in}{1.030836in}}%
\pgfpathlineto{\pgfqpoint{1.702428in}{2.361240in}}%
\pgfpathlineto{\pgfqpoint{1.486672in}{1.030836in}}%
\pgfpathclose%
\pgfusepath{fill}%
\end{pgfscope}%
\begin{pgfscope}%
\pgfpathrectangle{\pgfqpoint{0.017500in}{0.017500in}}{\pgfqpoint{6.218282in}{3.580372in}} %
\pgfusepath{clip}%
\pgfsetbuttcap%
\pgfsetroundjoin%
\definecolor{currentfill}{rgb}{0.229806,0.298718,0.753683}%
\pgfsetfillcolor{currentfill}%
\pgfsetlinewidth{0.000000pt}%
\definecolor{currentstroke}{rgb}{0.000000,0.000000,0.000000}%
\pgfsetstrokecolor{currentstroke}%
\pgfsetdash{}{0pt}%
\pgfpathmoveto{\pgfqpoint{1.486672in}{1.030836in}}%
\pgfpathlineto{\pgfqpoint{1.702428in}{2.361240in}}%
\pgfpathlineto{\pgfqpoint{1.486672in}{1.030836in}}%
\pgfpathclose%
\pgfusepath{fill}%
\end{pgfscope}%
\begin{pgfscope}%
\pgfpathrectangle{\pgfqpoint{0.017500in}{0.017500in}}{\pgfqpoint{6.218282in}{3.580372in}} %
\pgfusepath{clip}%
\pgfsetbuttcap%
\pgfsetroundjoin%
\definecolor{currentfill}{rgb}{0.229806,0.298718,0.753683}%
\pgfsetfillcolor{currentfill}%
\pgfsetlinewidth{0.000000pt}%
\definecolor{currentstroke}{rgb}{0.000000,0.000000,0.000000}%
\pgfsetstrokecolor{currentstroke}%
\pgfsetdash{}{0pt}%
\pgfpathmoveto{\pgfqpoint{1.486672in}{1.030836in}}%
\pgfpathlineto{\pgfqpoint{1.702428in}{2.361240in}}%
\pgfpathlineto{\pgfqpoint{1.486672in}{1.030836in}}%
\pgfpathclose%
\pgfusepath{fill}%
\end{pgfscope}%
\begin{pgfscope}%
\pgfpathrectangle{\pgfqpoint{0.017500in}{0.017500in}}{\pgfqpoint{6.218282in}{3.580372in}} %
\pgfusepath{clip}%
\pgfsetbuttcap%
\pgfsetroundjoin%
\definecolor{currentfill}{rgb}{0.229806,0.298718,0.753683}%
\pgfsetfillcolor{currentfill}%
\pgfsetlinewidth{0.000000pt}%
\definecolor{currentstroke}{rgb}{0.000000,0.000000,0.000000}%
\pgfsetstrokecolor{currentstroke}%
\pgfsetdash{}{0pt}%
\pgfpathmoveto{\pgfqpoint{1.486672in}{1.030836in}}%
\pgfpathlineto{\pgfqpoint{1.702428in}{2.361240in}}%
\pgfpathlineto{\pgfqpoint{1.486672in}{1.030836in}}%
\pgfpathclose%
\pgfusepath{fill}%
\end{pgfscope}%
\begin{pgfscope}%
\pgfpathrectangle{\pgfqpoint{0.017500in}{0.017500in}}{\pgfqpoint{6.218282in}{3.580372in}} %
\pgfusepath{clip}%
\pgfsetbuttcap%
\pgfsetroundjoin%
\definecolor{currentfill}{rgb}{0.229806,0.298718,0.753683}%
\pgfsetfillcolor{currentfill}%
\pgfsetlinewidth{0.000000pt}%
\definecolor{currentstroke}{rgb}{0.000000,0.000000,0.000000}%
\pgfsetstrokecolor{currentstroke}%
\pgfsetdash{}{0pt}%
\pgfpathmoveto{\pgfqpoint{1.486672in}{1.030836in}}%
\pgfpathlineto{\pgfqpoint{1.702428in}{2.361240in}}%
\pgfpathlineto{\pgfqpoint{1.486672in}{1.030836in}}%
\pgfpathclose%
\pgfusepath{fill}%
\end{pgfscope}%
\begin{pgfscope}%
\pgfpathrectangle{\pgfqpoint{0.017500in}{0.017500in}}{\pgfqpoint{6.218282in}{3.580372in}} %
\pgfusepath{clip}%
\pgfsetbuttcap%
\pgfsetroundjoin%
\definecolor{currentfill}{rgb}{0.229806,0.298718,0.753683}%
\pgfsetfillcolor{currentfill}%
\pgfsetlinewidth{0.000000pt}%
\definecolor{currentstroke}{rgb}{0.000000,0.000000,0.000000}%
\pgfsetstrokecolor{currentstroke}%
\pgfsetdash{}{0pt}%
\pgfpathmoveto{\pgfqpoint{1.486672in}{1.030836in}}%
\pgfpathlineto{\pgfqpoint{1.702428in}{2.361240in}}%
\pgfpathlineto{\pgfqpoint{1.486672in}{1.030836in}}%
\pgfpathclose%
\pgfusepath{fill}%
\end{pgfscope}%
\begin{pgfscope}%
\pgfpathrectangle{\pgfqpoint{0.017500in}{0.017500in}}{\pgfqpoint{6.218282in}{3.580372in}} %
\pgfusepath{clip}%
\pgfsetbuttcap%
\pgfsetroundjoin%
\definecolor{currentfill}{rgb}{0.229806,0.298718,0.753683}%
\pgfsetfillcolor{currentfill}%
\pgfsetlinewidth{0.000000pt}%
\definecolor{currentstroke}{rgb}{0.000000,0.000000,0.000000}%
\pgfsetstrokecolor{currentstroke}%
\pgfsetdash{}{0pt}%
\pgfpathmoveto{\pgfqpoint{1.486672in}{1.030836in}}%
\pgfpathlineto{\pgfqpoint{1.702428in}{2.361240in}}%
\pgfpathlineto{\pgfqpoint{1.486672in}{1.030836in}}%
\pgfpathclose%
\pgfusepath{fill}%
\end{pgfscope}%
\begin{pgfscope}%
\pgfpathrectangle{\pgfqpoint{0.017500in}{0.017500in}}{\pgfqpoint{6.218282in}{3.580372in}} %
\pgfusepath{clip}%
\pgfsetbuttcap%
\pgfsetroundjoin%
\definecolor{currentfill}{rgb}{0.746838,0.140021,0.179996}%
\pgfsetfillcolor{currentfill}%
\pgfsetlinewidth{0.000000pt}%
\definecolor{currentstroke}{rgb}{0.000000,0.000000,0.000000}%
\pgfsetstrokecolor{currentstroke}%
\pgfsetdash{}{0pt}%
\pgfpathmoveto{\pgfqpoint{2.674281in}{2.686821in}}%
\pgfpathlineto{\pgfqpoint{2.932971in}{2.820255in}}%
\pgfpathlineto{\pgfqpoint{2.674281in}{2.686821in}}%
\pgfpathclose%
\pgfusepath{fill}%
\end{pgfscope}%
\begin{pgfscope}%
\pgfpathrectangle{\pgfqpoint{0.017500in}{0.017500in}}{\pgfqpoint{6.218282in}{3.580372in}} %
\pgfusepath{clip}%
\pgfsetbuttcap%
\pgfsetroundjoin%
\definecolor{currentfill}{rgb}{0.746838,0.140021,0.179996}%
\pgfsetfillcolor{currentfill}%
\pgfsetlinewidth{0.000000pt}%
\definecolor{currentstroke}{rgb}{0.000000,0.000000,0.000000}%
\pgfsetstrokecolor{currentstroke}%
\pgfsetdash{}{0pt}%
\pgfpathmoveto{\pgfqpoint{2.674281in}{2.686821in}}%
\pgfpathlineto{\pgfqpoint{2.932971in}{2.820255in}}%
\pgfpathlineto{\pgfqpoint{2.674281in}{2.686821in}}%
\pgfpathclose%
\pgfusepath{fill}%
\end{pgfscope}%
\begin{pgfscope}%
\pgfpathrectangle{\pgfqpoint{0.017500in}{0.017500in}}{\pgfqpoint{6.218282in}{3.580372in}} %
\pgfusepath{clip}%
\pgfsetbuttcap%
\pgfsetroundjoin%
\definecolor{currentfill}{rgb}{0.746838,0.140021,0.179996}%
\pgfsetfillcolor{currentfill}%
\pgfsetlinewidth{0.000000pt}%
\definecolor{currentstroke}{rgb}{0.000000,0.000000,0.000000}%
\pgfsetstrokecolor{currentstroke}%
\pgfsetdash{}{0pt}%
\pgfpathmoveto{\pgfqpoint{2.674281in}{2.686821in}}%
\pgfpathlineto{\pgfqpoint{2.932971in}{2.820255in}}%
\pgfpathlineto{\pgfqpoint{2.674281in}{2.686821in}}%
\pgfpathclose%
\pgfusepath{fill}%
\end{pgfscope}%
\begin{pgfscope}%
\pgfpathrectangle{\pgfqpoint{0.017500in}{0.017500in}}{\pgfqpoint{6.218282in}{3.580372in}} %
\pgfusepath{clip}%
\pgfsetbuttcap%
\pgfsetroundjoin%
\definecolor{currentfill}{rgb}{0.746838,0.140021,0.179996}%
\pgfsetfillcolor{currentfill}%
\pgfsetlinewidth{0.000000pt}%
\definecolor{currentstroke}{rgb}{0.000000,0.000000,0.000000}%
\pgfsetstrokecolor{currentstroke}%
\pgfsetdash{}{0pt}%
\pgfpathmoveto{\pgfqpoint{2.674281in}{2.686821in}}%
\pgfpathlineto{\pgfqpoint{2.932971in}{2.820255in}}%
\pgfpathlineto{\pgfqpoint{2.674281in}{2.686821in}}%
\pgfpathclose%
\pgfusepath{fill}%
\end{pgfscope}%
\begin{pgfscope}%
\pgfpathrectangle{\pgfqpoint{0.017500in}{0.017500in}}{\pgfqpoint{6.218282in}{3.580372in}} %
\pgfusepath{clip}%
\pgfsetbuttcap%
\pgfsetroundjoin%
\definecolor{currentfill}{rgb}{0.746838,0.140021,0.179996}%
\pgfsetfillcolor{currentfill}%
\pgfsetlinewidth{0.000000pt}%
\definecolor{currentstroke}{rgb}{0.000000,0.000000,0.000000}%
\pgfsetstrokecolor{currentstroke}%
\pgfsetdash{}{0pt}%
\pgfpathmoveto{\pgfqpoint{2.674281in}{2.686821in}}%
\pgfpathlineto{\pgfqpoint{2.932971in}{2.820255in}}%
\pgfpathlineto{\pgfqpoint{2.674281in}{2.686821in}}%
\pgfpathclose%
\pgfusepath{fill}%
\end{pgfscope}%
\begin{pgfscope}%
\pgfpathrectangle{\pgfqpoint{0.017500in}{0.017500in}}{\pgfqpoint{6.218282in}{3.580372in}} %
\pgfusepath{clip}%
\pgfsetbuttcap%
\pgfsetroundjoin%
\definecolor{currentfill}{rgb}{0.746838,0.140021,0.179996}%
\pgfsetfillcolor{currentfill}%
\pgfsetlinewidth{0.000000pt}%
\definecolor{currentstroke}{rgb}{0.000000,0.000000,0.000000}%
\pgfsetstrokecolor{currentstroke}%
\pgfsetdash{}{0pt}%
\pgfpathmoveto{\pgfqpoint{2.674281in}{2.686821in}}%
\pgfpathlineto{\pgfqpoint{2.932971in}{2.820255in}}%
\pgfpathlineto{\pgfqpoint{2.674281in}{2.686821in}}%
\pgfpathclose%
\pgfusepath{fill}%
\end{pgfscope}%
\begin{pgfscope}%
\pgfpathrectangle{\pgfqpoint{0.017500in}{0.017500in}}{\pgfqpoint{6.218282in}{3.580372in}} %
\pgfusepath{clip}%
\pgfsetbuttcap%
\pgfsetroundjoin%
\definecolor{currentfill}{rgb}{0.746838,0.140021,0.179996}%
\pgfsetfillcolor{currentfill}%
\pgfsetlinewidth{0.000000pt}%
\definecolor{currentstroke}{rgb}{0.000000,0.000000,0.000000}%
\pgfsetstrokecolor{currentstroke}%
\pgfsetdash{}{0pt}%
\pgfpathmoveto{\pgfqpoint{2.674281in}{2.686821in}}%
\pgfpathlineto{\pgfqpoint{2.932971in}{2.820255in}}%
\pgfpathlineto{\pgfqpoint{2.674281in}{2.686821in}}%
\pgfpathclose%
\pgfusepath{fill}%
\end{pgfscope}%
\begin{pgfscope}%
\pgfpathrectangle{\pgfqpoint{0.017500in}{0.017500in}}{\pgfqpoint{6.218282in}{3.580372in}} %
\pgfusepath{clip}%
\pgfsetbuttcap%
\pgfsetroundjoin%
\definecolor{currentfill}{rgb}{0.746838,0.140021,0.179996}%
\pgfsetfillcolor{currentfill}%
\pgfsetlinewidth{0.000000pt}%
\definecolor{currentstroke}{rgb}{0.000000,0.000000,0.000000}%
\pgfsetstrokecolor{currentstroke}%
\pgfsetdash{}{0pt}%
\pgfpathmoveto{\pgfqpoint{2.674281in}{2.686821in}}%
\pgfpathlineto{\pgfqpoint{2.932971in}{2.820255in}}%
\pgfpathlineto{\pgfqpoint{2.674281in}{2.686821in}}%
\pgfpathclose%
\pgfusepath{fill}%
\end{pgfscope}%
\begin{pgfscope}%
\pgfpathrectangle{\pgfqpoint{0.017500in}{0.017500in}}{\pgfqpoint{6.218282in}{3.580372in}} %
\pgfusepath{clip}%
\pgfsetbuttcap%
\pgfsetroundjoin%
\definecolor{currentfill}{rgb}{0.746838,0.140021,0.179996}%
\pgfsetfillcolor{currentfill}%
\pgfsetlinewidth{0.000000pt}%
\definecolor{currentstroke}{rgb}{0.000000,0.000000,0.000000}%
\pgfsetstrokecolor{currentstroke}%
\pgfsetdash{}{0pt}%
\pgfpathmoveto{\pgfqpoint{2.674281in}{2.686821in}}%
\pgfpathlineto{\pgfqpoint{2.932971in}{2.820255in}}%
\pgfpathlineto{\pgfqpoint{2.674281in}{2.686821in}}%
\pgfpathclose%
\pgfusepath{fill}%
\end{pgfscope}%
\begin{pgfscope}%
\pgfpathrectangle{\pgfqpoint{0.017500in}{0.017500in}}{\pgfqpoint{6.218282in}{3.580372in}} %
\pgfusepath{clip}%
\pgfsetbuttcap%
\pgfsetroundjoin%
\definecolor{currentfill}{rgb}{0.746838,0.140021,0.179996}%
\pgfsetfillcolor{currentfill}%
\pgfsetlinewidth{0.000000pt}%
\definecolor{currentstroke}{rgb}{0.000000,0.000000,0.000000}%
\pgfsetstrokecolor{currentstroke}%
\pgfsetdash{}{0pt}%
\pgfpathmoveto{\pgfqpoint{2.674281in}{2.686821in}}%
\pgfpathlineto{\pgfqpoint{2.932971in}{2.820255in}}%
\pgfpathlineto{\pgfqpoint{2.674281in}{2.686821in}}%
\pgfpathclose%
\pgfusepath{fill}%
\end{pgfscope}%
\begin{pgfscope}%
\pgfpathrectangle{\pgfqpoint{0.017500in}{0.017500in}}{\pgfqpoint{6.218282in}{3.580372in}} %
\pgfusepath{clip}%
\pgfsetbuttcap%
\pgfsetroundjoin%
\definecolor{currentfill}{rgb}{0.746838,0.140021,0.179996}%
\pgfsetfillcolor{currentfill}%
\pgfsetlinewidth{0.000000pt}%
\definecolor{currentstroke}{rgb}{0.000000,0.000000,0.000000}%
\pgfsetstrokecolor{currentstroke}%
\pgfsetdash{}{0pt}%
\pgfpathmoveto{\pgfqpoint{2.674281in}{2.686821in}}%
\pgfpathlineto{\pgfqpoint{2.932971in}{2.820255in}}%
\pgfpathlineto{\pgfqpoint{2.674281in}{2.686821in}}%
\pgfpathclose%
\pgfusepath{fill}%
\end{pgfscope}%
\begin{pgfscope}%
\pgfpathrectangle{\pgfqpoint{0.017500in}{0.017500in}}{\pgfqpoint{6.218282in}{3.580372in}} %
\pgfusepath{clip}%
\pgfsetbuttcap%
\pgfsetroundjoin%
\definecolor{currentfill}{rgb}{0.746838,0.140021,0.179996}%
\pgfsetfillcolor{currentfill}%
\pgfsetlinewidth{0.000000pt}%
\definecolor{currentstroke}{rgb}{0.000000,0.000000,0.000000}%
\pgfsetstrokecolor{currentstroke}%
\pgfsetdash{}{0pt}%
\pgfpathmoveto{\pgfqpoint{2.674281in}{2.686821in}}%
\pgfpathlineto{\pgfqpoint{2.932971in}{2.820255in}}%
\pgfpathlineto{\pgfqpoint{2.674281in}{2.686821in}}%
\pgfpathclose%
\pgfusepath{fill}%
\end{pgfscope}%
\begin{pgfscope}%
\pgfpathrectangle{\pgfqpoint{0.017500in}{0.017500in}}{\pgfqpoint{6.218282in}{3.580372in}} %
\pgfusepath{clip}%
\pgfsetbuttcap%
\pgfsetroundjoin%
\definecolor{currentfill}{rgb}{0.746838,0.140021,0.179996}%
\pgfsetfillcolor{currentfill}%
\pgfsetlinewidth{0.000000pt}%
\definecolor{currentstroke}{rgb}{0.000000,0.000000,0.000000}%
\pgfsetstrokecolor{currentstroke}%
\pgfsetdash{}{0pt}%
\pgfpathmoveto{\pgfqpoint{2.674281in}{2.686821in}}%
\pgfpathlineto{\pgfqpoint{2.932971in}{2.820255in}}%
\pgfpathlineto{\pgfqpoint{2.674281in}{2.686821in}}%
\pgfpathclose%
\pgfusepath{fill}%
\end{pgfscope}%
\begin{pgfscope}%
\pgfpathrectangle{\pgfqpoint{0.017500in}{0.017500in}}{\pgfqpoint{6.218282in}{3.580372in}} %
\pgfusepath{clip}%
\pgfsetbuttcap%
\pgfsetroundjoin%
\definecolor{currentfill}{rgb}{0.746838,0.140021,0.179996}%
\pgfsetfillcolor{currentfill}%
\pgfsetlinewidth{0.000000pt}%
\definecolor{currentstroke}{rgb}{0.000000,0.000000,0.000000}%
\pgfsetstrokecolor{currentstroke}%
\pgfsetdash{}{0pt}%
\pgfpathmoveto{\pgfqpoint{2.674281in}{2.686821in}}%
\pgfpathlineto{\pgfqpoint{2.932971in}{2.820255in}}%
\pgfpathlineto{\pgfqpoint{2.674281in}{2.686821in}}%
\pgfpathclose%
\pgfusepath{fill}%
\end{pgfscope}%
\begin{pgfscope}%
\pgfpathrectangle{\pgfqpoint{0.017500in}{0.017500in}}{\pgfqpoint{6.218282in}{3.580372in}} %
\pgfusepath{clip}%
\pgfsetbuttcap%
\pgfsetroundjoin%
\definecolor{currentfill}{rgb}{0.746838,0.140021,0.179996}%
\pgfsetfillcolor{currentfill}%
\pgfsetlinewidth{0.000000pt}%
\definecolor{currentstroke}{rgb}{0.000000,0.000000,0.000000}%
\pgfsetstrokecolor{currentstroke}%
\pgfsetdash{}{0pt}%
\pgfpathmoveto{\pgfqpoint{2.674281in}{2.686821in}}%
\pgfpathlineto{\pgfqpoint{2.932971in}{2.820255in}}%
\pgfpathlineto{\pgfqpoint{2.674281in}{2.686821in}}%
\pgfpathclose%
\pgfusepath{fill}%
\end{pgfscope}%
\begin{pgfscope}%
\pgfpathrectangle{\pgfqpoint{0.017500in}{0.017500in}}{\pgfqpoint{6.218282in}{3.580372in}} %
\pgfusepath{clip}%
\pgfsetbuttcap%
\pgfsetroundjoin%
\definecolor{currentfill}{rgb}{0.746838,0.140021,0.179996}%
\pgfsetfillcolor{currentfill}%
\pgfsetlinewidth{0.000000pt}%
\definecolor{currentstroke}{rgb}{0.000000,0.000000,0.000000}%
\pgfsetstrokecolor{currentstroke}%
\pgfsetdash{}{0pt}%
\pgfpathmoveto{\pgfqpoint{2.674281in}{2.686821in}}%
\pgfpathlineto{\pgfqpoint{2.932971in}{2.820255in}}%
\pgfpathlineto{\pgfqpoint{2.674281in}{2.686821in}}%
\pgfpathclose%
\pgfusepath{fill}%
\end{pgfscope}%
\begin{pgfscope}%
\pgfpathrectangle{\pgfqpoint{0.017500in}{0.017500in}}{\pgfqpoint{6.218282in}{3.580372in}} %
\pgfusepath{clip}%
\pgfsetbuttcap%
\pgfsetroundjoin%
\definecolor{currentfill}{rgb}{0.746838,0.140021,0.179996}%
\pgfsetfillcolor{currentfill}%
\pgfsetlinewidth{0.000000pt}%
\definecolor{currentstroke}{rgb}{0.000000,0.000000,0.000000}%
\pgfsetstrokecolor{currentstroke}%
\pgfsetdash{}{0pt}%
\pgfpathmoveto{\pgfqpoint{2.674281in}{2.686821in}}%
\pgfpathlineto{\pgfqpoint{2.932971in}{2.820255in}}%
\pgfpathlineto{\pgfqpoint{2.674281in}{2.686821in}}%
\pgfpathclose%
\pgfusepath{fill}%
\end{pgfscope}%
\begin{pgfscope}%
\pgfpathrectangle{\pgfqpoint{0.017500in}{0.017500in}}{\pgfqpoint{6.218282in}{3.580372in}} %
\pgfusepath{clip}%
\pgfsetbuttcap%
\pgfsetroundjoin%
\definecolor{currentfill}{rgb}{0.746838,0.140021,0.179996}%
\pgfsetfillcolor{currentfill}%
\pgfsetlinewidth{0.000000pt}%
\definecolor{currentstroke}{rgb}{0.000000,0.000000,0.000000}%
\pgfsetstrokecolor{currentstroke}%
\pgfsetdash{}{0pt}%
\pgfpathmoveto{\pgfqpoint{2.674281in}{2.686821in}}%
\pgfpathlineto{\pgfqpoint{2.932971in}{2.820255in}}%
\pgfpathlineto{\pgfqpoint{2.674281in}{2.686821in}}%
\pgfpathclose%
\pgfusepath{fill}%
\end{pgfscope}%
\begin{pgfscope}%
\pgfpathrectangle{\pgfqpoint{0.017500in}{0.017500in}}{\pgfqpoint{6.218282in}{3.580372in}} %
\pgfusepath{clip}%
\pgfsetbuttcap%
\pgfsetroundjoin%
\definecolor{currentfill}{rgb}{0.746838,0.140021,0.179996}%
\pgfsetfillcolor{currentfill}%
\pgfsetlinewidth{0.000000pt}%
\definecolor{currentstroke}{rgb}{0.000000,0.000000,0.000000}%
\pgfsetstrokecolor{currentstroke}%
\pgfsetdash{}{0pt}%
\pgfpathmoveto{\pgfqpoint{2.674281in}{2.686821in}}%
\pgfpathlineto{\pgfqpoint{2.932971in}{2.820255in}}%
\pgfpathlineto{\pgfqpoint{2.674281in}{2.686821in}}%
\pgfpathclose%
\pgfusepath{fill}%
\end{pgfscope}%
\begin{pgfscope}%
\pgfpathrectangle{\pgfqpoint{0.017500in}{0.017500in}}{\pgfqpoint{6.218282in}{3.580372in}} %
\pgfusepath{clip}%
\pgfsetbuttcap%
\pgfsetroundjoin%
\definecolor{currentfill}{rgb}{0.746838,0.140021,0.179996}%
\pgfsetfillcolor{currentfill}%
\pgfsetlinewidth{0.000000pt}%
\definecolor{currentstroke}{rgb}{0.000000,0.000000,0.000000}%
\pgfsetstrokecolor{currentstroke}%
\pgfsetdash{}{0pt}%
\pgfpathmoveto{\pgfqpoint{2.674281in}{2.686821in}}%
\pgfpathlineto{\pgfqpoint{2.932971in}{2.820255in}}%
\pgfpathlineto{\pgfqpoint{2.674281in}{2.686821in}}%
\pgfpathclose%
\pgfusepath{fill}%
\end{pgfscope}%
\begin{pgfscope}%
\pgfpathrectangle{\pgfqpoint{0.017500in}{0.017500in}}{\pgfqpoint{6.218282in}{3.580372in}} %
\pgfusepath{clip}%
\pgfsetbuttcap%
\pgfsetroundjoin%
\definecolor{currentfill}{rgb}{0.746838,0.140021,0.179996}%
\pgfsetfillcolor{currentfill}%
\pgfsetlinewidth{0.000000pt}%
\definecolor{currentstroke}{rgb}{0.000000,0.000000,0.000000}%
\pgfsetstrokecolor{currentstroke}%
\pgfsetdash{}{0pt}%
\pgfpathmoveto{\pgfqpoint{2.674281in}{2.686821in}}%
\pgfpathlineto{\pgfqpoint{2.932971in}{2.820255in}}%
\pgfpathlineto{\pgfqpoint{2.674281in}{2.686821in}}%
\pgfpathclose%
\pgfusepath{fill}%
\end{pgfscope}%
\begin{pgfscope}%
\pgfpathrectangle{\pgfqpoint{0.017500in}{0.017500in}}{\pgfqpoint{6.218282in}{3.580372in}} %
\pgfusepath{clip}%
\pgfsetbuttcap%
\pgfsetroundjoin%
\definecolor{currentfill}{rgb}{0.746838,0.140021,0.179996}%
\pgfsetfillcolor{currentfill}%
\pgfsetlinewidth{0.000000pt}%
\definecolor{currentstroke}{rgb}{0.000000,0.000000,0.000000}%
\pgfsetstrokecolor{currentstroke}%
\pgfsetdash{}{0pt}%
\pgfpathmoveto{\pgfqpoint{2.674281in}{2.686821in}}%
\pgfpathlineto{\pgfqpoint{2.932971in}{2.820255in}}%
\pgfpathlineto{\pgfqpoint{2.674281in}{2.686821in}}%
\pgfpathclose%
\pgfusepath{fill}%
\end{pgfscope}%
\begin{pgfscope}%
\pgfpathrectangle{\pgfqpoint{0.017500in}{0.017500in}}{\pgfqpoint{6.218282in}{3.580372in}} %
\pgfusepath{clip}%
\pgfsetbuttcap%
\pgfsetroundjoin%
\definecolor{currentfill}{rgb}{0.746838,0.140021,0.179996}%
\pgfsetfillcolor{currentfill}%
\pgfsetlinewidth{0.000000pt}%
\definecolor{currentstroke}{rgb}{0.000000,0.000000,0.000000}%
\pgfsetstrokecolor{currentstroke}%
\pgfsetdash{}{0pt}%
\pgfpathmoveto{\pgfqpoint{2.674281in}{2.686821in}}%
\pgfpathlineto{\pgfqpoint{2.932971in}{2.820255in}}%
\pgfpathlineto{\pgfqpoint{2.674281in}{2.686821in}}%
\pgfpathclose%
\pgfusepath{fill}%
\end{pgfscope}%
\begin{pgfscope}%
\pgfpathrectangle{\pgfqpoint{0.017500in}{0.017500in}}{\pgfqpoint{6.218282in}{3.580372in}} %
\pgfusepath{clip}%
\pgfsetbuttcap%
\pgfsetroundjoin%
\definecolor{currentfill}{rgb}{0.746838,0.140021,0.179996}%
\pgfsetfillcolor{currentfill}%
\pgfsetlinewidth{0.000000pt}%
\definecolor{currentstroke}{rgb}{0.000000,0.000000,0.000000}%
\pgfsetstrokecolor{currentstroke}%
\pgfsetdash{}{0pt}%
\pgfpathmoveto{\pgfqpoint{2.674281in}{2.686821in}}%
\pgfpathlineto{\pgfqpoint{2.932971in}{2.820255in}}%
\pgfpathlineto{\pgfqpoint{2.674281in}{2.686821in}}%
\pgfpathclose%
\pgfusepath{fill}%
\end{pgfscope}%
\begin{pgfscope}%
\pgfpathrectangle{\pgfqpoint{0.017500in}{0.017500in}}{\pgfqpoint{6.218282in}{3.580372in}} %
\pgfusepath{clip}%
\pgfsetbuttcap%
\pgfsetroundjoin%
\definecolor{currentfill}{rgb}{0.746838,0.140021,0.179996}%
\pgfsetfillcolor{currentfill}%
\pgfsetlinewidth{0.000000pt}%
\definecolor{currentstroke}{rgb}{0.000000,0.000000,0.000000}%
\pgfsetstrokecolor{currentstroke}%
\pgfsetdash{}{0pt}%
\pgfpathmoveto{\pgfqpoint{2.674281in}{2.686821in}}%
\pgfpathlineto{\pgfqpoint{2.932971in}{2.820255in}}%
\pgfpathlineto{\pgfqpoint{2.674281in}{2.686821in}}%
\pgfpathclose%
\pgfusepath{fill}%
\end{pgfscope}%
\begin{pgfscope}%
\pgfpathrectangle{\pgfqpoint{0.017500in}{0.017500in}}{\pgfqpoint{6.218282in}{3.580372in}} %
\pgfusepath{clip}%
\pgfsetbuttcap%
\pgfsetroundjoin%
\definecolor{currentfill}{rgb}{0.746838,0.140021,0.179996}%
\pgfsetfillcolor{currentfill}%
\pgfsetlinewidth{0.000000pt}%
\definecolor{currentstroke}{rgb}{0.000000,0.000000,0.000000}%
\pgfsetstrokecolor{currentstroke}%
\pgfsetdash{}{0pt}%
\pgfpathmoveto{\pgfqpoint{2.674281in}{2.686821in}}%
\pgfpathlineto{\pgfqpoint{2.932971in}{2.820255in}}%
\pgfpathlineto{\pgfqpoint{2.674281in}{2.686821in}}%
\pgfpathclose%
\pgfusepath{fill}%
\end{pgfscope}%
\begin{pgfscope}%
\pgfpathrectangle{\pgfqpoint{0.017500in}{0.017500in}}{\pgfqpoint{6.218282in}{3.580372in}} %
\pgfusepath{clip}%
\pgfsetbuttcap%
\pgfsetroundjoin%
\definecolor{currentfill}{rgb}{0.746838,0.140021,0.179996}%
\pgfsetfillcolor{currentfill}%
\pgfsetlinewidth{0.000000pt}%
\definecolor{currentstroke}{rgb}{0.000000,0.000000,0.000000}%
\pgfsetstrokecolor{currentstroke}%
\pgfsetdash{}{0pt}%
\pgfpathmoveto{\pgfqpoint{2.674281in}{2.686821in}}%
\pgfpathlineto{\pgfqpoint{2.932971in}{2.820255in}}%
\pgfpathlineto{\pgfqpoint{2.674281in}{2.686821in}}%
\pgfpathclose%
\pgfusepath{fill}%
\end{pgfscope}%
\begin{pgfscope}%
\pgfpathrectangle{\pgfqpoint{0.017500in}{0.017500in}}{\pgfqpoint{6.218282in}{3.580372in}} %
\pgfusepath{clip}%
\pgfsetbuttcap%
\pgfsetroundjoin%
\definecolor{currentfill}{rgb}{0.746838,0.140021,0.179996}%
\pgfsetfillcolor{currentfill}%
\pgfsetlinewidth{0.000000pt}%
\definecolor{currentstroke}{rgb}{0.000000,0.000000,0.000000}%
\pgfsetstrokecolor{currentstroke}%
\pgfsetdash{}{0pt}%
\pgfpathmoveto{\pgfqpoint{2.674281in}{2.686821in}}%
\pgfpathlineto{\pgfqpoint{2.932971in}{2.820255in}}%
\pgfpathlineto{\pgfqpoint{2.674281in}{2.686821in}}%
\pgfpathclose%
\pgfusepath{fill}%
\end{pgfscope}%
\begin{pgfscope}%
\pgfpathrectangle{\pgfqpoint{0.017500in}{0.017500in}}{\pgfqpoint{6.218282in}{3.580372in}} %
\pgfusepath{clip}%
\pgfsetbuttcap%
\pgfsetroundjoin%
\definecolor{currentfill}{rgb}{0.746838,0.140021,0.179996}%
\pgfsetfillcolor{currentfill}%
\pgfsetlinewidth{0.000000pt}%
\definecolor{currentstroke}{rgb}{0.000000,0.000000,0.000000}%
\pgfsetstrokecolor{currentstroke}%
\pgfsetdash{}{0pt}%
\pgfpathmoveto{\pgfqpoint{2.674281in}{2.686821in}}%
\pgfpathlineto{\pgfqpoint{2.932971in}{2.820255in}}%
\pgfpathlineto{\pgfqpoint{2.674281in}{2.686821in}}%
\pgfpathclose%
\pgfusepath{fill}%
\end{pgfscope}%
\begin{pgfscope}%
\pgfpathrectangle{\pgfqpoint{0.017500in}{0.017500in}}{\pgfqpoint{6.218282in}{3.580372in}} %
\pgfusepath{clip}%
\pgfsetbuttcap%
\pgfsetroundjoin%
\definecolor{currentfill}{rgb}{0.746838,0.140021,0.179996}%
\pgfsetfillcolor{currentfill}%
\pgfsetlinewidth{0.000000pt}%
\definecolor{currentstroke}{rgb}{0.000000,0.000000,0.000000}%
\pgfsetstrokecolor{currentstroke}%
\pgfsetdash{}{0pt}%
\pgfpathmoveto{\pgfqpoint{2.674281in}{2.686821in}}%
\pgfpathlineto{\pgfqpoint{2.932971in}{2.820255in}}%
\pgfpathlineto{\pgfqpoint{2.674281in}{2.686821in}}%
\pgfpathclose%
\pgfusepath{fill}%
\end{pgfscope}%
\begin{pgfscope}%
\pgfpathrectangle{\pgfqpoint{0.017500in}{0.017500in}}{\pgfqpoint{6.218282in}{3.580372in}} %
\pgfusepath{clip}%
\pgfsetbuttcap%
\pgfsetroundjoin%
\definecolor{currentfill}{rgb}{0.902849,0.844796,0.811970}%
\pgfsetfillcolor{currentfill}%
\pgfsetlinewidth{0.000000pt}%
\definecolor{currentstroke}{rgb}{0.000000,0.000000,0.000000}%
\pgfsetstrokecolor{currentstroke}%
\pgfsetdash{}{0pt}%
\pgfpathmoveto{\pgfqpoint{4.063757in}{2.816074in}}%
\pgfpathlineto{\pgfqpoint{2.789661in}{0.820654in}}%
\pgfpathlineto{\pgfqpoint{4.063757in}{2.816074in}}%
\pgfpathclose%
\pgfusepath{fill}%
\end{pgfscope}%
\begin{pgfscope}%
\pgfpathrectangle{\pgfqpoint{0.017500in}{0.017500in}}{\pgfqpoint{6.218282in}{3.580372in}} %
\pgfusepath{clip}%
\pgfsetbuttcap%
\pgfsetroundjoin%
\definecolor{currentfill}{rgb}{0.902849,0.844796,0.811970}%
\pgfsetfillcolor{currentfill}%
\pgfsetlinewidth{0.000000pt}%
\definecolor{currentstroke}{rgb}{0.000000,0.000000,0.000000}%
\pgfsetstrokecolor{currentstroke}%
\pgfsetdash{}{0pt}%
\pgfpathmoveto{\pgfqpoint{4.063757in}{2.816074in}}%
\pgfpathlineto{\pgfqpoint{2.789661in}{0.820654in}}%
\pgfpathlineto{\pgfqpoint{4.063757in}{2.816074in}}%
\pgfpathclose%
\pgfusepath{fill}%
\end{pgfscope}%
\begin{pgfscope}%
\pgfpathrectangle{\pgfqpoint{0.017500in}{0.017500in}}{\pgfqpoint{6.218282in}{3.580372in}} %
\pgfusepath{clip}%
\pgfsetbuttcap%
\pgfsetroundjoin%
\definecolor{currentfill}{rgb}{0.902849,0.844796,0.811970}%
\pgfsetfillcolor{currentfill}%
\pgfsetlinewidth{0.000000pt}%
\definecolor{currentstroke}{rgb}{0.000000,0.000000,0.000000}%
\pgfsetstrokecolor{currentstroke}%
\pgfsetdash{}{0pt}%
\pgfpathmoveto{\pgfqpoint{4.063757in}{2.816074in}}%
\pgfpathlineto{\pgfqpoint{2.789661in}{0.820654in}}%
\pgfpathlineto{\pgfqpoint{4.063757in}{2.816074in}}%
\pgfpathclose%
\pgfusepath{fill}%
\end{pgfscope}%
\begin{pgfscope}%
\pgfpathrectangle{\pgfqpoint{0.017500in}{0.017500in}}{\pgfqpoint{6.218282in}{3.580372in}} %
\pgfusepath{clip}%
\pgfsetbuttcap%
\pgfsetroundjoin%
\definecolor{currentfill}{rgb}{0.902849,0.844796,0.811970}%
\pgfsetfillcolor{currentfill}%
\pgfsetlinewidth{0.000000pt}%
\definecolor{currentstroke}{rgb}{0.000000,0.000000,0.000000}%
\pgfsetstrokecolor{currentstroke}%
\pgfsetdash{}{0pt}%
\pgfpathmoveto{\pgfqpoint{4.063757in}{2.816074in}}%
\pgfpathlineto{\pgfqpoint{2.789661in}{0.820654in}}%
\pgfpathlineto{\pgfqpoint{4.063757in}{2.816074in}}%
\pgfpathclose%
\pgfusepath{fill}%
\end{pgfscope}%
\begin{pgfscope}%
\pgfpathrectangle{\pgfqpoint{0.017500in}{0.017500in}}{\pgfqpoint{6.218282in}{3.580372in}} %
\pgfusepath{clip}%
\pgfsetbuttcap%
\pgfsetroundjoin%
\definecolor{currentfill}{rgb}{0.902849,0.844796,0.811970}%
\pgfsetfillcolor{currentfill}%
\pgfsetlinewidth{0.000000pt}%
\definecolor{currentstroke}{rgb}{0.000000,0.000000,0.000000}%
\pgfsetstrokecolor{currentstroke}%
\pgfsetdash{}{0pt}%
\pgfpathmoveto{\pgfqpoint{4.063757in}{2.816074in}}%
\pgfpathlineto{\pgfqpoint{2.789661in}{0.820654in}}%
\pgfpathlineto{\pgfqpoint{4.063757in}{2.816074in}}%
\pgfpathclose%
\pgfusepath{fill}%
\end{pgfscope}%
\begin{pgfscope}%
\pgfpathrectangle{\pgfqpoint{0.017500in}{0.017500in}}{\pgfqpoint{6.218282in}{3.580372in}} %
\pgfusepath{clip}%
\pgfsetbuttcap%
\pgfsetroundjoin%
\definecolor{currentfill}{rgb}{0.902849,0.844796,0.811970}%
\pgfsetfillcolor{currentfill}%
\pgfsetlinewidth{0.000000pt}%
\definecolor{currentstroke}{rgb}{0.000000,0.000000,0.000000}%
\pgfsetstrokecolor{currentstroke}%
\pgfsetdash{}{0pt}%
\pgfpathmoveto{\pgfqpoint{4.063757in}{2.816074in}}%
\pgfpathlineto{\pgfqpoint{2.789661in}{0.820654in}}%
\pgfpathlineto{\pgfqpoint{4.063757in}{2.816074in}}%
\pgfpathclose%
\pgfusepath{fill}%
\end{pgfscope}%
\begin{pgfscope}%
\pgfpathrectangle{\pgfqpoint{0.017500in}{0.017500in}}{\pgfqpoint{6.218282in}{3.580372in}} %
\pgfusepath{clip}%
\pgfsetbuttcap%
\pgfsetroundjoin%
\definecolor{currentfill}{rgb}{0.902849,0.844796,0.811970}%
\pgfsetfillcolor{currentfill}%
\pgfsetlinewidth{0.000000pt}%
\definecolor{currentstroke}{rgb}{0.000000,0.000000,0.000000}%
\pgfsetstrokecolor{currentstroke}%
\pgfsetdash{}{0pt}%
\pgfpathmoveto{\pgfqpoint{4.063757in}{2.816074in}}%
\pgfpathlineto{\pgfqpoint{2.789661in}{0.820654in}}%
\pgfpathlineto{\pgfqpoint{4.063757in}{2.816074in}}%
\pgfpathclose%
\pgfusepath{fill}%
\end{pgfscope}%
\begin{pgfscope}%
\pgfpathrectangle{\pgfqpoint{0.017500in}{0.017500in}}{\pgfqpoint{6.218282in}{3.580372in}} %
\pgfusepath{clip}%
\pgfsetbuttcap%
\pgfsetroundjoin%
\definecolor{currentfill}{rgb}{0.902849,0.844796,0.811970}%
\pgfsetfillcolor{currentfill}%
\pgfsetlinewidth{0.000000pt}%
\definecolor{currentstroke}{rgb}{0.000000,0.000000,0.000000}%
\pgfsetstrokecolor{currentstroke}%
\pgfsetdash{}{0pt}%
\pgfpathmoveto{\pgfqpoint{4.063757in}{2.816074in}}%
\pgfpathlineto{\pgfqpoint{2.789661in}{0.820654in}}%
\pgfpathlineto{\pgfqpoint{4.063757in}{2.816074in}}%
\pgfpathclose%
\pgfusepath{fill}%
\end{pgfscope}%
\begin{pgfscope}%
\pgfpathrectangle{\pgfqpoint{0.017500in}{0.017500in}}{\pgfqpoint{6.218282in}{3.580372in}} %
\pgfusepath{clip}%
\pgfsetbuttcap%
\pgfsetroundjoin%
\definecolor{currentfill}{rgb}{0.902849,0.844796,0.811970}%
\pgfsetfillcolor{currentfill}%
\pgfsetlinewidth{0.000000pt}%
\definecolor{currentstroke}{rgb}{0.000000,0.000000,0.000000}%
\pgfsetstrokecolor{currentstroke}%
\pgfsetdash{}{0pt}%
\pgfpathmoveto{\pgfqpoint{4.063757in}{2.816074in}}%
\pgfpathlineto{\pgfqpoint{2.789661in}{0.820654in}}%
\pgfpathlineto{\pgfqpoint{4.063757in}{2.816074in}}%
\pgfpathclose%
\pgfusepath{fill}%
\end{pgfscope}%
\begin{pgfscope}%
\pgfpathrectangle{\pgfqpoint{0.017500in}{0.017500in}}{\pgfqpoint{6.218282in}{3.580372in}} %
\pgfusepath{clip}%
\pgfsetbuttcap%
\pgfsetroundjoin%
\definecolor{currentfill}{rgb}{0.902849,0.844796,0.811970}%
\pgfsetfillcolor{currentfill}%
\pgfsetlinewidth{0.000000pt}%
\definecolor{currentstroke}{rgb}{0.000000,0.000000,0.000000}%
\pgfsetstrokecolor{currentstroke}%
\pgfsetdash{}{0pt}%
\pgfpathmoveto{\pgfqpoint{4.063757in}{2.816074in}}%
\pgfpathlineto{\pgfqpoint{2.789661in}{0.820654in}}%
\pgfpathlineto{\pgfqpoint{4.063757in}{2.816074in}}%
\pgfpathclose%
\pgfusepath{fill}%
\end{pgfscope}%
\begin{pgfscope}%
\pgfpathrectangle{\pgfqpoint{0.017500in}{0.017500in}}{\pgfqpoint{6.218282in}{3.580372in}} %
\pgfusepath{clip}%
\pgfsetbuttcap%
\pgfsetroundjoin%
\definecolor{currentfill}{rgb}{0.902849,0.844796,0.811970}%
\pgfsetfillcolor{currentfill}%
\pgfsetlinewidth{0.000000pt}%
\definecolor{currentstroke}{rgb}{0.000000,0.000000,0.000000}%
\pgfsetstrokecolor{currentstroke}%
\pgfsetdash{}{0pt}%
\pgfpathmoveto{\pgfqpoint{4.063757in}{2.816074in}}%
\pgfpathlineto{\pgfqpoint{2.789661in}{0.820654in}}%
\pgfpathlineto{\pgfqpoint{4.063757in}{2.816074in}}%
\pgfpathclose%
\pgfusepath{fill}%
\end{pgfscope}%
\begin{pgfscope}%
\pgfpathrectangle{\pgfqpoint{0.017500in}{0.017500in}}{\pgfqpoint{6.218282in}{3.580372in}} %
\pgfusepath{clip}%
\pgfsetbuttcap%
\pgfsetroundjoin%
\definecolor{currentfill}{rgb}{0.902849,0.844796,0.811970}%
\pgfsetfillcolor{currentfill}%
\pgfsetlinewidth{0.000000pt}%
\definecolor{currentstroke}{rgb}{0.000000,0.000000,0.000000}%
\pgfsetstrokecolor{currentstroke}%
\pgfsetdash{}{0pt}%
\pgfpathmoveto{\pgfqpoint{4.063757in}{2.816074in}}%
\pgfpathlineto{\pgfqpoint{2.789661in}{0.820654in}}%
\pgfpathlineto{\pgfqpoint{4.063757in}{2.816074in}}%
\pgfpathclose%
\pgfusepath{fill}%
\end{pgfscope}%
\begin{pgfscope}%
\pgfpathrectangle{\pgfqpoint{0.017500in}{0.017500in}}{\pgfqpoint{6.218282in}{3.580372in}} %
\pgfusepath{clip}%
\pgfsetbuttcap%
\pgfsetroundjoin%
\definecolor{currentfill}{rgb}{0.902849,0.844796,0.811970}%
\pgfsetfillcolor{currentfill}%
\pgfsetlinewidth{0.000000pt}%
\definecolor{currentstroke}{rgb}{0.000000,0.000000,0.000000}%
\pgfsetstrokecolor{currentstroke}%
\pgfsetdash{}{0pt}%
\pgfpathmoveto{\pgfqpoint{4.063757in}{2.816074in}}%
\pgfpathlineto{\pgfqpoint{2.789661in}{0.820654in}}%
\pgfpathlineto{\pgfqpoint{4.063757in}{2.816074in}}%
\pgfpathclose%
\pgfusepath{fill}%
\end{pgfscope}%
\begin{pgfscope}%
\pgfpathrectangle{\pgfqpoint{0.017500in}{0.017500in}}{\pgfqpoint{6.218282in}{3.580372in}} %
\pgfusepath{clip}%
\pgfsetbuttcap%
\pgfsetroundjoin%
\definecolor{currentfill}{rgb}{0.902849,0.844796,0.811970}%
\pgfsetfillcolor{currentfill}%
\pgfsetlinewidth{0.000000pt}%
\definecolor{currentstroke}{rgb}{0.000000,0.000000,0.000000}%
\pgfsetstrokecolor{currentstroke}%
\pgfsetdash{}{0pt}%
\pgfpathmoveto{\pgfqpoint{4.063757in}{2.816074in}}%
\pgfpathlineto{\pgfqpoint{2.789661in}{0.820654in}}%
\pgfpathlineto{\pgfqpoint{4.063757in}{2.816074in}}%
\pgfpathclose%
\pgfusepath{fill}%
\end{pgfscope}%
\begin{pgfscope}%
\pgfpathrectangle{\pgfqpoint{0.017500in}{0.017500in}}{\pgfqpoint{6.218282in}{3.580372in}} %
\pgfusepath{clip}%
\pgfsetbuttcap%
\pgfsetroundjoin%
\definecolor{currentfill}{rgb}{0.902849,0.844796,0.811970}%
\pgfsetfillcolor{currentfill}%
\pgfsetlinewidth{0.000000pt}%
\definecolor{currentstroke}{rgb}{0.000000,0.000000,0.000000}%
\pgfsetstrokecolor{currentstroke}%
\pgfsetdash{}{0pt}%
\pgfpathmoveto{\pgfqpoint{4.063757in}{2.816074in}}%
\pgfpathlineto{\pgfqpoint{2.789661in}{0.820654in}}%
\pgfpathlineto{\pgfqpoint{4.063757in}{2.816074in}}%
\pgfpathclose%
\pgfusepath{fill}%
\end{pgfscope}%
\begin{pgfscope}%
\pgfpathrectangle{\pgfqpoint{0.017500in}{0.017500in}}{\pgfqpoint{6.218282in}{3.580372in}} %
\pgfusepath{clip}%
\pgfsetbuttcap%
\pgfsetroundjoin%
\definecolor{currentfill}{rgb}{0.902849,0.844796,0.811970}%
\pgfsetfillcolor{currentfill}%
\pgfsetlinewidth{0.000000pt}%
\definecolor{currentstroke}{rgb}{0.000000,0.000000,0.000000}%
\pgfsetstrokecolor{currentstroke}%
\pgfsetdash{}{0pt}%
\pgfpathmoveto{\pgfqpoint{4.063757in}{2.816074in}}%
\pgfpathlineto{\pgfqpoint{2.789661in}{0.820654in}}%
\pgfpathlineto{\pgfqpoint{4.063757in}{2.816074in}}%
\pgfpathclose%
\pgfusepath{fill}%
\end{pgfscope}%
\begin{pgfscope}%
\pgfpathrectangle{\pgfqpoint{0.017500in}{0.017500in}}{\pgfqpoint{6.218282in}{3.580372in}} %
\pgfusepath{clip}%
\pgfsetbuttcap%
\pgfsetroundjoin%
\definecolor{currentfill}{rgb}{0.902849,0.844796,0.811970}%
\pgfsetfillcolor{currentfill}%
\pgfsetlinewidth{0.000000pt}%
\definecolor{currentstroke}{rgb}{0.000000,0.000000,0.000000}%
\pgfsetstrokecolor{currentstroke}%
\pgfsetdash{}{0pt}%
\pgfpathmoveto{\pgfqpoint{4.063757in}{2.816074in}}%
\pgfpathlineto{\pgfqpoint{2.789661in}{0.820654in}}%
\pgfpathlineto{\pgfqpoint{4.063757in}{2.816074in}}%
\pgfpathclose%
\pgfusepath{fill}%
\end{pgfscope}%
\begin{pgfscope}%
\pgfpathrectangle{\pgfqpoint{0.017500in}{0.017500in}}{\pgfqpoint{6.218282in}{3.580372in}} %
\pgfusepath{clip}%
\pgfsetbuttcap%
\pgfsetroundjoin%
\definecolor{currentfill}{rgb}{0.902849,0.844796,0.811970}%
\pgfsetfillcolor{currentfill}%
\pgfsetlinewidth{0.000000pt}%
\definecolor{currentstroke}{rgb}{0.000000,0.000000,0.000000}%
\pgfsetstrokecolor{currentstroke}%
\pgfsetdash{}{0pt}%
\pgfpathmoveto{\pgfqpoint{4.063757in}{2.816074in}}%
\pgfpathlineto{\pgfqpoint{2.789661in}{0.820654in}}%
\pgfpathlineto{\pgfqpoint{4.063757in}{2.816074in}}%
\pgfpathclose%
\pgfusepath{fill}%
\end{pgfscope}%
\begin{pgfscope}%
\pgfpathrectangle{\pgfqpoint{0.017500in}{0.017500in}}{\pgfqpoint{6.218282in}{3.580372in}} %
\pgfusepath{clip}%
\pgfsetbuttcap%
\pgfsetroundjoin%
\definecolor{currentfill}{rgb}{0.902849,0.844796,0.811970}%
\pgfsetfillcolor{currentfill}%
\pgfsetlinewidth{0.000000pt}%
\definecolor{currentstroke}{rgb}{0.000000,0.000000,0.000000}%
\pgfsetstrokecolor{currentstroke}%
\pgfsetdash{}{0pt}%
\pgfpathmoveto{\pgfqpoint{4.063757in}{2.816074in}}%
\pgfpathlineto{\pgfqpoint{2.789661in}{0.820654in}}%
\pgfpathlineto{\pgfqpoint{4.063757in}{2.816074in}}%
\pgfpathclose%
\pgfusepath{fill}%
\end{pgfscope}%
\begin{pgfscope}%
\pgfpathrectangle{\pgfqpoint{0.017500in}{0.017500in}}{\pgfqpoint{6.218282in}{3.580372in}} %
\pgfusepath{clip}%
\pgfsetbuttcap%
\pgfsetroundjoin%
\definecolor{currentfill}{rgb}{0.902849,0.844796,0.811970}%
\pgfsetfillcolor{currentfill}%
\pgfsetlinewidth{0.000000pt}%
\definecolor{currentstroke}{rgb}{0.000000,0.000000,0.000000}%
\pgfsetstrokecolor{currentstroke}%
\pgfsetdash{}{0pt}%
\pgfpathmoveto{\pgfqpoint{4.063757in}{2.816074in}}%
\pgfpathlineto{\pgfqpoint{2.789661in}{0.820654in}}%
\pgfpathlineto{\pgfqpoint{4.063757in}{2.816074in}}%
\pgfpathclose%
\pgfusepath{fill}%
\end{pgfscope}%
\begin{pgfscope}%
\pgfpathrectangle{\pgfqpoint{0.017500in}{0.017500in}}{\pgfqpoint{6.218282in}{3.580372in}} %
\pgfusepath{clip}%
\pgfsetbuttcap%
\pgfsetroundjoin%
\definecolor{currentfill}{rgb}{0.902849,0.844796,0.811970}%
\pgfsetfillcolor{currentfill}%
\pgfsetlinewidth{0.000000pt}%
\definecolor{currentstroke}{rgb}{0.000000,0.000000,0.000000}%
\pgfsetstrokecolor{currentstroke}%
\pgfsetdash{}{0pt}%
\pgfpathmoveto{\pgfqpoint{4.063757in}{2.816074in}}%
\pgfpathlineto{\pgfqpoint{2.789661in}{0.820654in}}%
\pgfpathlineto{\pgfqpoint{4.063757in}{2.816074in}}%
\pgfpathclose%
\pgfusepath{fill}%
\end{pgfscope}%
\begin{pgfscope}%
\pgfpathrectangle{\pgfqpoint{0.017500in}{0.017500in}}{\pgfqpoint{6.218282in}{3.580372in}} %
\pgfusepath{clip}%
\pgfsetbuttcap%
\pgfsetroundjoin%
\definecolor{currentfill}{rgb}{0.902849,0.844796,0.811970}%
\pgfsetfillcolor{currentfill}%
\pgfsetlinewidth{0.000000pt}%
\definecolor{currentstroke}{rgb}{0.000000,0.000000,0.000000}%
\pgfsetstrokecolor{currentstroke}%
\pgfsetdash{}{0pt}%
\pgfpathmoveto{\pgfqpoint{4.063757in}{2.816074in}}%
\pgfpathlineto{\pgfqpoint{2.789661in}{0.820654in}}%
\pgfpathlineto{\pgfqpoint{4.063757in}{2.816074in}}%
\pgfpathclose%
\pgfusepath{fill}%
\end{pgfscope}%
\begin{pgfscope}%
\pgfpathrectangle{\pgfqpoint{0.017500in}{0.017500in}}{\pgfqpoint{6.218282in}{3.580372in}} %
\pgfusepath{clip}%
\pgfsetbuttcap%
\pgfsetroundjoin%
\definecolor{currentfill}{rgb}{0.902849,0.844796,0.811970}%
\pgfsetfillcolor{currentfill}%
\pgfsetlinewidth{0.000000pt}%
\definecolor{currentstroke}{rgb}{0.000000,0.000000,0.000000}%
\pgfsetstrokecolor{currentstroke}%
\pgfsetdash{}{0pt}%
\pgfpathmoveto{\pgfqpoint{4.063757in}{2.816074in}}%
\pgfpathlineto{\pgfqpoint{2.789661in}{0.820654in}}%
\pgfpathlineto{\pgfqpoint{4.063757in}{2.816074in}}%
\pgfpathclose%
\pgfusepath{fill}%
\end{pgfscope}%
\begin{pgfscope}%
\pgfpathrectangle{\pgfqpoint{0.017500in}{0.017500in}}{\pgfqpoint{6.218282in}{3.580372in}} %
\pgfusepath{clip}%
\pgfsetbuttcap%
\pgfsetroundjoin%
\definecolor{currentfill}{rgb}{0.902849,0.844796,0.811970}%
\pgfsetfillcolor{currentfill}%
\pgfsetlinewidth{0.000000pt}%
\definecolor{currentstroke}{rgb}{0.000000,0.000000,0.000000}%
\pgfsetstrokecolor{currentstroke}%
\pgfsetdash{}{0pt}%
\pgfpathmoveto{\pgfqpoint{4.063757in}{2.816074in}}%
\pgfpathlineto{\pgfqpoint{2.789661in}{0.820654in}}%
\pgfpathlineto{\pgfqpoint{4.063757in}{2.816074in}}%
\pgfpathclose%
\pgfusepath{fill}%
\end{pgfscope}%
\begin{pgfscope}%
\pgfpathrectangle{\pgfqpoint{0.017500in}{0.017500in}}{\pgfqpoint{6.218282in}{3.580372in}} %
\pgfusepath{clip}%
\pgfsetbuttcap%
\pgfsetroundjoin%
\definecolor{currentfill}{rgb}{0.902849,0.844796,0.811970}%
\pgfsetfillcolor{currentfill}%
\pgfsetlinewidth{0.000000pt}%
\definecolor{currentstroke}{rgb}{0.000000,0.000000,0.000000}%
\pgfsetstrokecolor{currentstroke}%
\pgfsetdash{}{0pt}%
\pgfpathmoveto{\pgfqpoint{4.063757in}{2.816074in}}%
\pgfpathlineto{\pgfqpoint{2.789661in}{0.820654in}}%
\pgfpathlineto{\pgfqpoint{4.063757in}{2.816074in}}%
\pgfpathclose%
\pgfusepath{fill}%
\end{pgfscope}%
\begin{pgfscope}%
\pgfpathrectangle{\pgfqpoint{0.017500in}{0.017500in}}{\pgfqpoint{6.218282in}{3.580372in}} %
\pgfusepath{clip}%
\pgfsetbuttcap%
\pgfsetroundjoin%
\definecolor{currentfill}{rgb}{0.740957,0.122240,0.175744}%
\pgfsetfillcolor{currentfill}%
\pgfsetlinewidth{0.000000pt}%
\definecolor{currentstroke}{rgb}{0.000000,0.000000,0.000000}%
\pgfsetstrokecolor{currentstroke}%
\pgfsetdash{}{0pt}%
\pgfpathmoveto{\pgfqpoint{2.674281in}{2.686821in}}%
\pgfpathlineto{\pgfqpoint{3.116444in}{2.629069in}}%
\pgfpathlineto{\pgfqpoint{3.372089in}{2.760957in}}%
\pgfpathlineto{\pgfqpoint{2.932971in}{2.820255in}}%
\pgfpathlineto{\pgfqpoint{2.674281in}{2.686821in}}%
\pgfpathclose%
\pgfusepath{fill}%
\end{pgfscope}%
\begin{pgfscope}%
\pgfpathrectangle{\pgfqpoint{0.017500in}{0.017500in}}{\pgfqpoint{6.218282in}{3.580372in}} %
\pgfusepath{clip}%
\pgfsetbuttcap%
\pgfsetroundjoin%
\definecolor{currentfill}{rgb}{0.740957,0.122240,0.175744}%
\pgfsetfillcolor{currentfill}%
\pgfsetlinewidth{0.000000pt}%
\definecolor{currentstroke}{rgb}{0.000000,0.000000,0.000000}%
\pgfsetstrokecolor{currentstroke}%
\pgfsetdash{}{0pt}%
\pgfpathmoveto{\pgfqpoint{2.674281in}{2.686821in}}%
\pgfpathlineto{\pgfqpoint{3.116444in}{2.629069in}}%
\pgfpathlineto{\pgfqpoint{3.372089in}{2.760957in}}%
\pgfpathlineto{\pgfqpoint{2.932971in}{2.820255in}}%
\pgfpathlineto{\pgfqpoint{2.674281in}{2.686821in}}%
\pgfpathclose%
\pgfusepath{fill}%
\end{pgfscope}%
\begin{pgfscope}%
\pgfpathrectangle{\pgfqpoint{0.017500in}{0.017500in}}{\pgfqpoint{6.218282in}{3.580372in}} %
\pgfusepath{clip}%
\pgfsetbuttcap%
\pgfsetroundjoin%
\definecolor{currentfill}{rgb}{0.740957,0.122240,0.175744}%
\pgfsetfillcolor{currentfill}%
\pgfsetlinewidth{0.000000pt}%
\definecolor{currentstroke}{rgb}{0.000000,0.000000,0.000000}%
\pgfsetstrokecolor{currentstroke}%
\pgfsetdash{}{0pt}%
\pgfpathmoveto{\pgfqpoint{2.674281in}{2.686821in}}%
\pgfpathlineto{\pgfqpoint{3.116444in}{2.629069in}}%
\pgfpathlineto{\pgfqpoint{3.372089in}{2.760957in}}%
\pgfpathlineto{\pgfqpoint{2.932971in}{2.820255in}}%
\pgfpathlineto{\pgfqpoint{2.674281in}{2.686821in}}%
\pgfpathclose%
\pgfusepath{fill}%
\end{pgfscope}%
\begin{pgfscope}%
\pgfpathrectangle{\pgfqpoint{0.017500in}{0.017500in}}{\pgfqpoint{6.218282in}{3.580372in}} %
\pgfusepath{clip}%
\pgfsetbuttcap%
\pgfsetroundjoin%
\definecolor{currentfill}{rgb}{0.740957,0.122240,0.175744}%
\pgfsetfillcolor{currentfill}%
\pgfsetlinewidth{0.000000pt}%
\definecolor{currentstroke}{rgb}{0.000000,0.000000,0.000000}%
\pgfsetstrokecolor{currentstroke}%
\pgfsetdash{}{0pt}%
\pgfpathmoveto{\pgfqpoint{2.674281in}{2.686821in}}%
\pgfpathlineto{\pgfqpoint{3.116444in}{2.629069in}}%
\pgfpathlineto{\pgfqpoint{3.372089in}{2.760957in}}%
\pgfpathlineto{\pgfqpoint{2.932971in}{2.820255in}}%
\pgfpathlineto{\pgfqpoint{2.674281in}{2.686821in}}%
\pgfpathclose%
\pgfusepath{fill}%
\end{pgfscope}%
\begin{pgfscope}%
\pgfpathrectangle{\pgfqpoint{0.017500in}{0.017500in}}{\pgfqpoint{6.218282in}{3.580372in}} %
\pgfusepath{clip}%
\pgfsetbuttcap%
\pgfsetroundjoin%
\definecolor{currentfill}{rgb}{0.740957,0.122240,0.175744}%
\pgfsetfillcolor{currentfill}%
\pgfsetlinewidth{0.000000pt}%
\definecolor{currentstroke}{rgb}{0.000000,0.000000,0.000000}%
\pgfsetstrokecolor{currentstroke}%
\pgfsetdash{}{0pt}%
\pgfpathmoveto{\pgfqpoint{2.674281in}{2.686821in}}%
\pgfpathlineto{\pgfqpoint{3.116444in}{2.629069in}}%
\pgfpathlineto{\pgfqpoint{3.372089in}{2.760957in}}%
\pgfpathlineto{\pgfqpoint{2.932971in}{2.820255in}}%
\pgfpathlineto{\pgfqpoint{2.674281in}{2.686821in}}%
\pgfpathclose%
\pgfusepath{fill}%
\end{pgfscope}%
\begin{pgfscope}%
\pgfpathrectangle{\pgfqpoint{0.017500in}{0.017500in}}{\pgfqpoint{6.218282in}{3.580372in}} %
\pgfusepath{clip}%
\pgfsetbuttcap%
\pgfsetroundjoin%
\definecolor{currentfill}{rgb}{0.740957,0.122240,0.175744}%
\pgfsetfillcolor{currentfill}%
\pgfsetlinewidth{0.000000pt}%
\definecolor{currentstroke}{rgb}{0.000000,0.000000,0.000000}%
\pgfsetstrokecolor{currentstroke}%
\pgfsetdash{}{0pt}%
\pgfpathmoveto{\pgfqpoint{2.674281in}{2.686821in}}%
\pgfpathlineto{\pgfqpoint{3.116444in}{2.629069in}}%
\pgfpathlineto{\pgfqpoint{3.372089in}{2.760957in}}%
\pgfpathlineto{\pgfqpoint{2.932971in}{2.820255in}}%
\pgfpathlineto{\pgfqpoint{2.674281in}{2.686821in}}%
\pgfpathclose%
\pgfusepath{fill}%
\end{pgfscope}%
\begin{pgfscope}%
\pgfpathrectangle{\pgfqpoint{0.017500in}{0.017500in}}{\pgfqpoint{6.218282in}{3.580372in}} %
\pgfusepath{clip}%
\pgfsetbuttcap%
\pgfsetroundjoin%
\definecolor{currentfill}{rgb}{0.717435,0.051118,0.158737}%
\pgfsetfillcolor{currentfill}%
\pgfsetlinewidth{0.000000pt}%
\definecolor{currentstroke}{rgb}{0.000000,0.000000,0.000000}%
\pgfsetstrokecolor{currentstroke}%
\pgfsetdash{}{0pt}%
\pgfpathmoveto{\pgfqpoint{3.817122in}{2.696128in}}%
\pgfpathlineto{\pgfqpoint{4.063757in}{2.816074in}}%
\pgfpathlineto{\pgfqpoint{3.817122in}{2.696128in}}%
\pgfpathclose%
\pgfusepath{fill}%
\end{pgfscope}%
\begin{pgfscope}%
\pgfpathrectangle{\pgfqpoint{0.017500in}{0.017500in}}{\pgfqpoint{6.218282in}{3.580372in}} %
\pgfusepath{clip}%
\pgfsetbuttcap%
\pgfsetroundjoin%
\definecolor{currentfill}{rgb}{0.717435,0.051118,0.158737}%
\pgfsetfillcolor{currentfill}%
\pgfsetlinewidth{0.000000pt}%
\definecolor{currentstroke}{rgb}{0.000000,0.000000,0.000000}%
\pgfsetstrokecolor{currentstroke}%
\pgfsetdash{}{0pt}%
\pgfpathmoveto{\pgfqpoint{3.817122in}{2.696128in}}%
\pgfpathlineto{\pgfqpoint{4.063757in}{2.816074in}}%
\pgfpathlineto{\pgfqpoint{3.817122in}{2.696128in}}%
\pgfpathclose%
\pgfusepath{fill}%
\end{pgfscope}%
\begin{pgfscope}%
\pgfpathrectangle{\pgfqpoint{0.017500in}{0.017500in}}{\pgfqpoint{6.218282in}{3.580372in}} %
\pgfusepath{clip}%
\pgfsetbuttcap%
\pgfsetroundjoin%
\definecolor{currentfill}{rgb}{0.717435,0.051118,0.158737}%
\pgfsetfillcolor{currentfill}%
\pgfsetlinewidth{0.000000pt}%
\definecolor{currentstroke}{rgb}{0.000000,0.000000,0.000000}%
\pgfsetstrokecolor{currentstroke}%
\pgfsetdash{}{0pt}%
\pgfpathmoveto{\pgfqpoint{3.817122in}{2.696128in}}%
\pgfpathlineto{\pgfqpoint{4.063757in}{2.816074in}}%
\pgfpathlineto{\pgfqpoint{3.817122in}{2.696128in}}%
\pgfpathclose%
\pgfusepath{fill}%
\end{pgfscope}%
\begin{pgfscope}%
\pgfpathrectangle{\pgfqpoint{0.017500in}{0.017500in}}{\pgfqpoint{6.218282in}{3.580372in}} %
\pgfusepath{clip}%
\pgfsetbuttcap%
\pgfsetroundjoin%
\definecolor{currentfill}{rgb}{0.717435,0.051118,0.158737}%
\pgfsetfillcolor{currentfill}%
\pgfsetlinewidth{0.000000pt}%
\definecolor{currentstroke}{rgb}{0.000000,0.000000,0.000000}%
\pgfsetstrokecolor{currentstroke}%
\pgfsetdash{}{0pt}%
\pgfpathmoveto{\pgfqpoint{3.817122in}{2.696128in}}%
\pgfpathlineto{\pgfqpoint{4.063757in}{2.816074in}}%
\pgfpathlineto{\pgfqpoint{3.817122in}{2.696128in}}%
\pgfpathclose%
\pgfusepath{fill}%
\end{pgfscope}%
\begin{pgfscope}%
\pgfpathrectangle{\pgfqpoint{0.017500in}{0.017500in}}{\pgfqpoint{6.218282in}{3.580372in}} %
\pgfusepath{clip}%
\pgfsetbuttcap%
\pgfsetroundjoin%
\definecolor{currentfill}{rgb}{0.717435,0.051118,0.158737}%
\pgfsetfillcolor{currentfill}%
\pgfsetlinewidth{0.000000pt}%
\definecolor{currentstroke}{rgb}{0.000000,0.000000,0.000000}%
\pgfsetstrokecolor{currentstroke}%
\pgfsetdash{}{0pt}%
\pgfpathmoveto{\pgfqpoint{3.817122in}{2.696128in}}%
\pgfpathlineto{\pgfqpoint{4.063757in}{2.816074in}}%
\pgfpathlineto{\pgfqpoint{3.817122in}{2.696128in}}%
\pgfpathclose%
\pgfusepath{fill}%
\end{pgfscope}%
\begin{pgfscope}%
\pgfpathrectangle{\pgfqpoint{0.017500in}{0.017500in}}{\pgfqpoint{6.218282in}{3.580372in}} %
\pgfusepath{clip}%
\pgfsetbuttcap%
\pgfsetroundjoin%
\definecolor{currentfill}{rgb}{0.717435,0.051118,0.158737}%
\pgfsetfillcolor{currentfill}%
\pgfsetlinewidth{0.000000pt}%
\definecolor{currentstroke}{rgb}{0.000000,0.000000,0.000000}%
\pgfsetstrokecolor{currentstroke}%
\pgfsetdash{}{0pt}%
\pgfpathmoveto{\pgfqpoint{3.817122in}{2.696128in}}%
\pgfpathlineto{\pgfqpoint{4.063757in}{2.816074in}}%
\pgfpathlineto{\pgfqpoint{3.817122in}{2.696128in}}%
\pgfpathclose%
\pgfusepath{fill}%
\end{pgfscope}%
\begin{pgfscope}%
\pgfpathrectangle{\pgfqpoint{0.017500in}{0.017500in}}{\pgfqpoint{6.218282in}{3.580372in}} %
\pgfusepath{clip}%
\pgfsetbuttcap%
\pgfsetroundjoin%
\definecolor{currentfill}{rgb}{0.717435,0.051118,0.158737}%
\pgfsetfillcolor{currentfill}%
\pgfsetlinewidth{0.000000pt}%
\definecolor{currentstroke}{rgb}{0.000000,0.000000,0.000000}%
\pgfsetstrokecolor{currentstroke}%
\pgfsetdash{}{0pt}%
\pgfpathmoveto{\pgfqpoint{3.817122in}{2.696128in}}%
\pgfpathlineto{\pgfqpoint{4.063757in}{2.816074in}}%
\pgfpathlineto{\pgfqpoint{3.817122in}{2.696128in}}%
\pgfpathclose%
\pgfusepath{fill}%
\end{pgfscope}%
\begin{pgfscope}%
\pgfpathrectangle{\pgfqpoint{0.017500in}{0.017500in}}{\pgfqpoint{6.218282in}{3.580372in}} %
\pgfusepath{clip}%
\pgfsetbuttcap%
\pgfsetroundjoin%
\definecolor{currentfill}{rgb}{0.717435,0.051118,0.158737}%
\pgfsetfillcolor{currentfill}%
\pgfsetlinewidth{0.000000pt}%
\definecolor{currentstroke}{rgb}{0.000000,0.000000,0.000000}%
\pgfsetstrokecolor{currentstroke}%
\pgfsetdash{}{0pt}%
\pgfpathmoveto{\pgfqpoint{3.817122in}{2.696128in}}%
\pgfpathlineto{\pgfqpoint{4.063757in}{2.816074in}}%
\pgfpathlineto{\pgfqpoint{3.817122in}{2.696128in}}%
\pgfpathclose%
\pgfusepath{fill}%
\end{pgfscope}%
\begin{pgfscope}%
\pgfpathrectangle{\pgfqpoint{0.017500in}{0.017500in}}{\pgfqpoint{6.218282in}{3.580372in}} %
\pgfusepath{clip}%
\pgfsetbuttcap%
\pgfsetroundjoin%
\definecolor{currentfill}{rgb}{0.717435,0.051118,0.158737}%
\pgfsetfillcolor{currentfill}%
\pgfsetlinewidth{0.000000pt}%
\definecolor{currentstroke}{rgb}{0.000000,0.000000,0.000000}%
\pgfsetstrokecolor{currentstroke}%
\pgfsetdash{}{0pt}%
\pgfpathmoveto{\pgfqpoint{3.817122in}{2.696128in}}%
\pgfpathlineto{\pgfqpoint{4.063757in}{2.816074in}}%
\pgfpathlineto{\pgfqpoint{3.817122in}{2.696128in}}%
\pgfpathclose%
\pgfusepath{fill}%
\end{pgfscope}%
\begin{pgfscope}%
\pgfpathrectangle{\pgfqpoint{0.017500in}{0.017500in}}{\pgfqpoint{6.218282in}{3.580372in}} %
\pgfusepath{clip}%
\pgfsetbuttcap%
\pgfsetroundjoin%
\definecolor{currentfill}{rgb}{0.717435,0.051118,0.158737}%
\pgfsetfillcolor{currentfill}%
\pgfsetlinewidth{0.000000pt}%
\definecolor{currentstroke}{rgb}{0.000000,0.000000,0.000000}%
\pgfsetstrokecolor{currentstroke}%
\pgfsetdash{}{0pt}%
\pgfpathmoveto{\pgfqpoint{3.817122in}{2.696128in}}%
\pgfpathlineto{\pgfqpoint{4.063757in}{2.816074in}}%
\pgfpathlineto{\pgfqpoint{3.817122in}{2.696128in}}%
\pgfpathclose%
\pgfusepath{fill}%
\end{pgfscope}%
\begin{pgfscope}%
\pgfpathrectangle{\pgfqpoint{0.017500in}{0.017500in}}{\pgfqpoint{6.218282in}{3.580372in}} %
\pgfusepath{clip}%
\pgfsetbuttcap%
\pgfsetroundjoin%
\definecolor{currentfill}{rgb}{0.717435,0.051118,0.158737}%
\pgfsetfillcolor{currentfill}%
\pgfsetlinewidth{0.000000pt}%
\definecolor{currentstroke}{rgb}{0.000000,0.000000,0.000000}%
\pgfsetstrokecolor{currentstroke}%
\pgfsetdash{}{0pt}%
\pgfpathmoveto{\pgfqpoint{3.817122in}{2.696128in}}%
\pgfpathlineto{\pgfqpoint{4.063757in}{2.816074in}}%
\pgfpathlineto{\pgfqpoint{3.817122in}{2.696128in}}%
\pgfpathclose%
\pgfusepath{fill}%
\end{pgfscope}%
\begin{pgfscope}%
\pgfpathrectangle{\pgfqpoint{0.017500in}{0.017500in}}{\pgfqpoint{6.218282in}{3.580372in}} %
\pgfusepath{clip}%
\pgfsetbuttcap%
\pgfsetroundjoin%
\definecolor{currentfill}{rgb}{0.717435,0.051118,0.158737}%
\pgfsetfillcolor{currentfill}%
\pgfsetlinewidth{0.000000pt}%
\definecolor{currentstroke}{rgb}{0.000000,0.000000,0.000000}%
\pgfsetstrokecolor{currentstroke}%
\pgfsetdash{}{0pt}%
\pgfpathmoveto{\pgfqpoint{3.817122in}{2.696128in}}%
\pgfpathlineto{\pgfqpoint{4.063757in}{2.816074in}}%
\pgfpathlineto{\pgfqpoint{3.817122in}{2.696128in}}%
\pgfpathclose%
\pgfusepath{fill}%
\end{pgfscope}%
\begin{pgfscope}%
\pgfpathrectangle{\pgfqpoint{0.017500in}{0.017500in}}{\pgfqpoint{6.218282in}{3.580372in}} %
\pgfusepath{clip}%
\pgfsetbuttcap%
\pgfsetroundjoin%
\definecolor{currentfill}{rgb}{0.717435,0.051118,0.158737}%
\pgfsetfillcolor{currentfill}%
\pgfsetlinewidth{0.000000pt}%
\definecolor{currentstroke}{rgb}{0.000000,0.000000,0.000000}%
\pgfsetstrokecolor{currentstroke}%
\pgfsetdash{}{0pt}%
\pgfpathmoveto{\pgfqpoint{3.817122in}{2.696128in}}%
\pgfpathlineto{\pgfqpoint{4.063757in}{2.816074in}}%
\pgfpathlineto{\pgfqpoint{3.817122in}{2.696128in}}%
\pgfpathclose%
\pgfusepath{fill}%
\end{pgfscope}%
\begin{pgfscope}%
\pgfpathrectangle{\pgfqpoint{0.017500in}{0.017500in}}{\pgfqpoint{6.218282in}{3.580372in}} %
\pgfusepath{clip}%
\pgfsetbuttcap%
\pgfsetroundjoin%
\definecolor{currentfill}{rgb}{0.717435,0.051118,0.158737}%
\pgfsetfillcolor{currentfill}%
\pgfsetlinewidth{0.000000pt}%
\definecolor{currentstroke}{rgb}{0.000000,0.000000,0.000000}%
\pgfsetstrokecolor{currentstroke}%
\pgfsetdash{}{0pt}%
\pgfpathmoveto{\pgfqpoint{3.817122in}{2.696128in}}%
\pgfpathlineto{\pgfqpoint{4.063757in}{2.816074in}}%
\pgfpathlineto{\pgfqpoint{3.817122in}{2.696128in}}%
\pgfpathclose%
\pgfusepath{fill}%
\end{pgfscope}%
\begin{pgfscope}%
\pgfpathrectangle{\pgfqpoint{0.017500in}{0.017500in}}{\pgfqpoint{6.218282in}{3.580372in}} %
\pgfusepath{clip}%
\pgfsetbuttcap%
\pgfsetroundjoin%
\definecolor{currentfill}{rgb}{0.717435,0.051118,0.158737}%
\pgfsetfillcolor{currentfill}%
\pgfsetlinewidth{0.000000pt}%
\definecolor{currentstroke}{rgb}{0.000000,0.000000,0.000000}%
\pgfsetstrokecolor{currentstroke}%
\pgfsetdash{}{0pt}%
\pgfpathmoveto{\pgfqpoint{3.817122in}{2.696128in}}%
\pgfpathlineto{\pgfqpoint{4.063757in}{2.816074in}}%
\pgfpathlineto{\pgfqpoint{3.817122in}{2.696128in}}%
\pgfpathclose%
\pgfusepath{fill}%
\end{pgfscope}%
\begin{pgfscope}%
\pgfpathrectangle{\pgfqpoint{0.017500in}{0.017500in}}{\pgfqpoint{6.218282in}{3.580372in}} %
\pgfusepath{clip}%
\pgfsetbuttcap%
\pgfsetroundjoin%
\definecolor{currentfill}{rgb}{0.717435,0.051118,0.158737}%
\pgfsetfillcolor{currentfill}%
\pgfsetlinewidth{0.000000pt}%
\definecolor{currentstroke}{rgb}{0.000000,0.000000,0.000000}%
\pgfsetstrokecolor{currentstroke}%
\pgfsetdash{}{0pt}%
\pgfpathmoveto{\pgfqpoint{3.817122in}{2.696128in}}%
\pgfpathlineto{\pgfqpoint{4.063757in}{2.816074in}}%
\pgfpathlineto{\pgfqpoint{3.817122in}{2.696128in}}%
\pgfpathclose%
\pgfusepath{fill}%
\end{pgfscope}%
\begin{pgfscope}%
\pgfpathrectangle{\pgfqpoint{0.017500in}{0.017500in}}{\pgfqpoint{6.218282in}{3.580372in}} %
\pgfusepath{clip}%
\pgfsetbuttcap%
\pgfsetroundjoin%
\definecolor{currentfill}{rgb}{0.717435,0.051118,0.158737}%
\pgfsetfillcolor{currentfill}%
\pgfsetlinewidth{0.000000pt}%
\definecolor{currentstroke}{rgb}{0.000000,0.000000,0.000000}%
\pgfsetstrokecolor{currentstroke}%
\pgfsetdash{}{0pt}%
\pgfpathmoveto{\pgfqpoint{3.817122in}{2.696128in}}%
\pgfpathlineto{\pgfqpoint{4.063757in}{2.816074in}}%
\pgfpathlineto{\pgfqpoint{3.817122in}{2.696128in}}%
\pgfpathclose%
\pgfusepath{fill}%
\end{pgfscope}%
\begin{pgfscope}%
\pgfpathrectangle{\pgfqpoint{0.017500in}{0.017500in}}{\pgfqpoint{6.218282in}{3.580372in}} %
\pgfusepath{clip}%
\pgfsetbuttcap%
\pgfsetroundjoin%
\definecolor{currentfill}{rgb}{0.717435,0.051118,0.158737}%
\pgfsetfillcolor{currentfill}%
\pgfsetlinewidth{0.000000pt}%
\definecolor{currentstroke}{rgb}{0.000000,0.000000,0.000000}%
\pgfsetstrokecolor{currentstroke}%
\pgfsetdash{}{0pt}%
\pgfpathmoveto{\pgfqpoint{3.817122in}{2.696128in}}%
\pgfpathlineto{\pgfqpoint{4.063757in}{2.816074in}}%
\pgfpathlineto{\pgfqpoint{3.817122in}{2.696128in}}%
\pgfpathclose%
\pgfusepath{fill}%
\end{pgfscope}%
\begin{pgfscope}%
\pgfpathrectangle{\pgfqpoint{0.017500in}{0.017500in}}{\pgfqpoint{6.218282in}{3.580372in}} %
\pgfusepath{clip}%
\pgfsetbuttcap%
\pgfsetroundjoin%
\definecolor{currentfill}{rgb}{0.717435,0.051118,0.158737}%
\pgfsetfillcolor{currentfill}%
\pgfsetlinewidth{0.000000pt}%
\definecolor{currentstroke}{rgb}{0.000000,0.000000,0.000000}%
\pgfsetstrokecolor{currentstroke}%
\pgfsetdash{}{0pt}%
\pgfpathmoveto{\pgfqpoint{3.817122in}{2.696128in}}%
\pgfpathlineto{\pgfqpoint{4.063757in}{2.816074in}}%
\pgfpathlineto{\pgfqpoint{3.817122in}{2.696128in}}%
\pgfpathclose%
\pgfusepath{fill}%
\end{pgfscope}%
\begin{pgfscope}%
\pgfpathrectangle{\pgfqpoint{0.017500in}{0.017500in}}{\pgfqpoint{6.218282in}{3.580372in}} %
\pgfusepath{clip}%
\pgfsetbuttcap%
\pgfsetroundjoin%
\definecolor{currentfill}{rgb}{0.717435,0.051118,0.158737}%
\pgfsetfillcolor{currentfill}%
\pgfsetlinewidth{0.000000pt}%
\definecolor{currentstroke}{rgb}{0.000000,0.000000,0.000000}%
\pgfsetstrokecolor{currentstroke}%
\pgfsetdash{}{0pt}%
\pgfpathmoveto{\pgfqpoint{3.817122in}{2.696128in}}%
\pgfpathlineto{\pgfqpoint{4.063757in}{2.816074in}}%
\pgfpathlineto{\pgfqpoint{3.817122in}{2.696128in}}%
\pgfpathclose%
\pgfusepath{fill}%
\end{pgfscope}%
\begin{pgfscope}%
\pgfpathrectangle{\pgfqpoint{0.017500in}{0.017500in}}{\pgfqpoint{6.218282in}{3.580372in}} %
\pgfusepath{clip}%
\pgfsetbuttcap%
\pgfsetroundjoin%
\definecolor{currentfill}{rgb}{0.717435,0.051118,0.158737}%
\pgfsetfillcolor{currentfill}%
\pgfsetlinewidth{0.000000pt}%
\definecolor{currentstroke}{rgb}{0.000000,0.000000,0.000000}%
\pgfsetstrokecolor{currentstroke}%
\pgfsetdash{}{0pt}%
\pgfpathmoveto{\pgfqpoint{3.817122in}{2.696128in}}%
\pgfpathlineto{\pgfqpoint{4.063757in}{2.816074in}}%
\pgfpathlineto{\pgfqpoint{3.817122in}{2.696128in}}%
\pgfpathclose%
\pgfusepath{fill}%
\end{pgfscope}%
\begin{pgfscope}%
\pgfpathrectangle{\pgfqpoint{0.017500in}{0.017500in}}{\pgfqpoint{6.218282in}{3.580372in}} %
\pgfusepath{clip}%
\pgfsetbuttcap%
\pgfsetroundjoin%
\definecolor{currentfill}{rgb}{0.717435,0.051118,0.158737}%
\pgfsetfillcolor{currentfill}%
\pgfsetlinewidth{0.000000pt}%
\definecolor{currentstroke}{rgb}{0.000000,0.000000,0.000000}%
\pgfsetstrokecolor{currentstroke}%
\pgfsetdash{}{0pt}%
\pgfpathmoveto{\pgfqpoint{3.817122in}{2.696128in}}%
\pgfpathlineto{\pgfqpoint{4.063757in}{2.816074in}}%
\pgfpathlineto{\pgfqpoint{3.817122in}{2.696128in}}%
\pgfpathclose%
\pgfusepath{fill}%
\end{pgfscope}%
\begin{pgfscope}%
\pgfpathrectangle{\pgfqpoint{0.017500in}{0.017500in}}{\pgfqpoint{6.218282in}{3.580372in}} %
\pgfusepath{clip}%
\pgfsetbuttcap%
\pgfsetroundjoin%
\definecolor{currentfill}{rgb}{0.717435,0.051118,0.158737}%
\pgfsetfillcolor{currentfill}%
\pgfsetlinewidth{0.000000pt}%
\definecolor{currentstroke}{rgb}{0.000000,0.000000,0.000000}%
\pgfsetstrokecolor{currentstroke}%
\pgfsetdash{}{0pt}%
\pgfpathmoveto{\pgfqpoint{3.817122in}{2.696128in}}%
\pgfpathlineto{\pgfqpoint{4.063757in}{2.816074in}}%
\pgfpathlineto{\pgfqpoint{3.817122in}{2.696128in}}%
\pgfpathclose%
\pgfusepath{fill}%
\end{pgfscope}%
\begin{pgfscope}%
\pgfpathrectangle{\pgfqpoint{0.017500in}{0.017500in}}{\pgfqpoint{6.218282in}{3.580372in}} %
\pgfusepath{clip}%
\pgfsetbuttcap%
\pgfsetroundjoin%
\definecolor{currentfill}{rgb}{0.717435,0.051118,0.158737}%
\pgfsetfillcolor{currentfill}%
\pgfsetlinewidth{0.000000pt}%
\definecolor{currentstroke}{rgb}{0.000000,0.000000,0.000000}%
\pgfsetstrokecolor{currentstroke}%
\pgfsetdash{}{0pt}%
\pgfpathmoveto{\pgfqpoint{3.817122in}{2.696128in}}%
\pgfpathlineto{\pgfqpoint{4.063757in}{2.816074in}}%
\pgfpathlineto{\pgfqpoint{3.817122in}{2.696128in}}%
\pgfpathclose%
\pgfusepath{fill}%
\end{pgfscope}%
\begin{pgfscope}%
\pgfpathrectangle{\pgfqpoint{0.017500in}{0.017500in}}{\pgfqpoint{6.218282in}{3.580372in}} %
\pgfusepath{clip}%
\pgfsetbuttcap%
\pgfsetroundjoin%
\definecolor{currentfill}{rgb}{0.717435,0.051118,0.158737}%
\pgfsetfillcolor{currentfill}%
\pgfsetlinewidth{0.000000pt}%
\definecolor{currentstroke}{rgb}{0.000000,0.000000,0.000000}%
\pgfsetstrokecolor{currentstroke}%
\pgfsetdash{}{0pt}%
\pgfpathmoveto{\pgfqpoint{3.817122in}{2.696128in}}%
\pgfpathlineto{\pgfqpoint{4.063757in}{2.816074in}}%
\pgfpathlineto{\pgfqpoint{3.817122in}{2.696128in}}%
\pgfpathclose%
\pgfusepath{fill}%
\end{pgfscope}%
\begin{pgfscope}%
\pgfpathrectangle{\pgfqpoint{0.017500in}{0.017500in}}{\pgfqpoint{6.218282in}{3.580372in}} %
\pgfusepath{clip}%
\pgfsetbuttcap%
\pgfsetroundjoin%
\definecolor{currentfill}{rgb}{0.717435,0.051118,0.158737}%
\pgfsetfillcolor{currentfill}%
\pgfsetlinewidth{0.000000pt}%
\definecolor{currentstroke}{rgb}{0.000000,0.000000,0.000000}%
\pgfsetstrokecolor{currentstroke}%
\pgfsetdash{}{0pt}%
\pgfpathmoveto{\pgfqpoint{3.817122in}{2.696128in}}%
\pgfpathlineto{\pgfqpoint{4.063757in}{2.816074in}}%
\pgfpathlineto{\pgfqpoint{3.817122in}{2.696128in}}%
\pgfpathclose%
\pgfusepath{fill}%
\end{pgfscope}%
\begin{pgfscope}%
\pgfpathrectangle{\pgfqpoint{0.017500in}{0.017500in}}{\pgfqpoint{6.218282in}{3.580372in}} %
\pgfusepath{clip}%
\pgfsetbuttcap%
\pgfsetroundjoin%
\definecolor{currentfill}{rgb}{0.717435,0.051118,0.158737}%
\pgfsetfillcolor{currentfill}%
\pgfsetlinewidth{0.000000pt}%
\definecolor{currentstroke}{rgb}{0.000000,0.000000,0.000000}%
\pgfsetstrokecolor{currentstroke}%
\pgfsetdash{}{0pt}%
\pgfpathmoveto{\pgfqpoint{3.817122in}{2.696128in}}%
\pgfpathlineto{\pgfqpoint{4.063757in}{2.816074in}}%
\pgfpathlineto{\pgfqpoint{3.817122in}{2.696128in}}%
\pgfpathclose%
\pgfusepath{fill}%
\end{pgfscope}%
\begin{pgfscope}%
\pgfpathrectangle{\pgfqpoint{0.017500in}{0.017500in}}{\pgfqpoint{6.218282in}{3.580372in}} %
\pgfusepath{clip}%
\pgfsetbuttcap%
\pgfsetroundjoin%
\definecolor{currentfill}{rgb}{0.717435,0.051118,0.158737}%
\pgfsetfillcolor{currentfill}%
\pgfsetlinewidth{0.000000pt}%
\definecolor{currentstroke}{rgb}{0.000000,0.000000,0.000000}%
\pgfsetstrokecolor{currentstroke}%
\pgfsetdash{}{0pt}%
\pgfpathmoveto{\pgfqpoint{3.817122in}{2.696128in}}%
\pgfpathlineto{\pgfqpoint{4.063757in}{2.816074in}}%
\pgfpathlineto{\pgfqpoint{3.817122in}{2.696128in}}%
\pgfpathclose%
\pgfusepath{fill}%
\end{pgfscope}%
\begin{pgfscope}%
\pgfpathrectangle{\pgfqpoint{0.017500in}{0.017500in}}{\pgfqpoint{6.218282in}{3.580372in}} %
\pgfusepath{clip}%
\pgfsetbuttcap%
\pgfsetroundjoin%
\definecolor{currentfill}{rgb}{0.717435,0.051118,0.158737}%
\pgfsetfillcolor{currentfill}%
\pgfsetlinewidth{0.000000pt}%
\definecolor{currentstroke}{rgb}{0.000000,0.000000,0.000000}%
\pgfsetstrokecolor{currentstroke}%
\pgfsetdash{}{0pt}%
\pgfpathmoveto{\pgfqpoint{3.817122in}{2.696128in}}%
\pgfpathlineto{\pgfqpoint{4.063757in}{2.816074in}}%
\pgfpathlineto{\pgfqpoint{3.817122in}{2.696128in}}%
\pgfpathclose%
\pgfusepath{fill}%
\end{pgfscope}%
\begin{pgfscope}%
\pgfpathrectangle{\pgfqpoint{0.017500in}{0.017500in}}{\pgfqpoint{6.218282in}{3.580372in}} %
\pgfusepath{clip}%
\pgfsetbuttcap%
\pgfsetroundjoin%
\definecolor{currentfill}{rgb}{0.717435,0.051118,0.158737}%
\pgfsetfillcolor{currentfill}%
\pgfsetlinewidth{0.000000pt}%
\definecolor{currentstroke}{rgb}{0.000000,0.000000,0.000000}%
\pgfsetstrokecolor{currentstroke}%
\pgfsetdash{}{0pt}%
\pgfpathmoveto{\pgfqpoint{3.817122in}{2.696128in}}%
\pgfpathlineto{\pgfqpoint{4.063757in}{2.816074in}}%
\pgfpathlineto{\pgfqpoint{3.817122in}{2.696128in}}%
\pgfpathclose%
\pgfusepath{fill}%
\end{pgfscope}%
\begin{pgfscope}%
\pgfpathrectangle{\pgfqpoint{0.017500in}{0.017500in}}{\pgfqpoint{6.218282in}{3.580372in}} %
\pgfusepath{clip}%
\pgfsetbuttcap%
\pgfsetroundjoin%
\definecolor{currentfill}{rgb}{0.820401,0.286765,0.245160}%
\pgfsetfillcolor{currentfill}%
\pgfsetlinewidth{0.000000pt}%
\definecolor{currentstroke}{rgb}{0.000000,0.000000,0.000000}%
\pgfsetstrokecolor{currentstroke}%
\pgfsetdash{}{0pt}%
\pgfpathmoveto{\pgfqpoint{1.972254in}{2.569602in}}%
\pgfpathlineto{\pgfqpoint{2.238213in}{2.742590in}}%
\pgfpathlineto{\pgfqpoint{1.972254in}{2.569602in}}%
\pgfpathclose%
\pgfusepath{fill}%
\end{pgfscope}%
\begin{pgfscope}%
\pgfpathrectangle{\pgfqpoint{0.017500in}{0.017500in}}{\pgfqpoint{6.218282in}{3.580372in}} %
\pgfusepath{clip}%
\pgfsetbuttcap%
\pgfsetroundjoin%
\definecolor{currentfill}{rgb}{0.820401,0.286765,0.245160}%
\pgfsetfillcolor{currentfill}%
\pgfsetlinewidth{0.000000pt}%
\definecolor{currentstroke}{rgb}{0.000000,0.000000,0.000000}%
\pgfsetstrokecolor{currentstroke}%
\pgfsetdash{}{0pt}%
\pgfpathmoveto{\pgfqpoint{1.972254in}{2.569602in}}%
\pgfpathlineto{\pgfqpoint{2.238213in}{2.742590in}}%
\pgfpathlineto{\pgfqpoint{1.972254in}{2.569602in}}%
\pgfpathclose%
\pgfusepath{fill}%
\end{pgfscope}%
\begin{pgfscope}%
\pgfpathrectangle{\pgfqpoint{0.017500in}{0.017500in}}{\pgfqpoint{6.218282in}{3.580372in}} %
\pgfusepath{clip}%
\pgfsetbuttcap%
\pgfsetroundjoin%
\definecolor{currentfill}{rgb}{0.820401,0.286765,0.245160}%
\pgfsetfillcolor{currentfill}%
\pgfsetlinewidth{0.000000pt}%
\definecolor{currentstroke}{rgb}{0.000000,0.000000,0.000000}%
\pgfsetstrokecolor{currentstroke}%
\pgfsetdash{}{0pt}%
\pgfpathmoveto{\pgfqpoint{1.972254in}{2.569602in}}%
\pgfpathlineto{\pgfqpoint{2.238213in}{2.742590in}}%
\pgfpathlineto{\pgfqpoint{1.972254in}{2.569602in}}%
\pgfpathclose%
\pgfusepath{fill}%
\end{pgfscope}%
\begin{pgfscope}%
\pgfpathrectangle{\pgfqpoint{0.017500in}{0.017500in}}{\pgfqpoint{6.218282in}{3.580372in}} %
\pgfusepath{clip}%
\pgfsetbuttcap%
\pgfsetroundjoin%
\definecolor{currentfill}{rgb}{0.820401,0.286765,0.245160}%
\pgfsetfillcolor{currentfill}%
\pgfsetlinewidth{0.000000pt}%
\definecolor{currentstroke}{rgb}{0.000000,0.000000,0.000000}%
\pgfsetstrokecolor{currentstroke}%
\pgfsetdash{}{0pt}%
\pgfpathmoveto{\pgfqpoint{1.972254in}{2.569602in}}%
\pgfpathlineto{\pgfqpoint{2.238213in}{2.742590in}}%
\pgfpathlineto{\pgfqpoint{1.972254in}{2.569602in}}%
\pgfpathclose%
\pgfusepath{fill}%
\end{pgfscope}%
\begin{pgfscope}%
\pgfpathrectangle{\pgfqpoint{0.017500in}{0.017500in}}{\pgfqpoint{6.218282in}{3.580372in}} %
\pgfusepath{clip}%
\pgfsetbuttcap%
\pgfsetroundjoin%
\definecolor{currentfill}{rgb}{0.820401,0.286765,0.245160}%
\pgfsetfillcolor{currentfill}%
\pgfsetlinewidth{0.000000pt}%
\definecolor{currentstroke}{rgb}{0.000000,0.000000,0.000000}%
\pgfsetstrokecolor{currentstroke}%
\pgfsetdash{}{0pt}%
\pgfpathmoveto{\pgfqpoint{1.972254in}{2.569602in}}%
\pgfpathlineto{\pgfqpoint{2.238213in}{2.742590in}}%
\pgfpathlineto{\pgfqpoint{1.972254in}{2.569602in}}%
\pgfpathclose%
\pgfusepath{fill}%
\end{pgfscope}%
\begin{pgfscope}%
\pgfpathrectangle{\pgfqpoint{0.017500in}{0.017500in}}{\pgfqpoint{6.218282in}{3.580372in}} %
\pgfusepath{clip}%
\pgfsetbuttcap%
\pgfsetroundjoin%
\definecolor{currentfill}{rgb}{0.820401,0.286765,0.245160}%
\pgfsetfillcolor{currentfill}%
\pgfsetlinewidth{0.000000pt}%
\definecolor{currentstroke}{rgb}{0.000000,0.000000,0.000000}%
\pgfsetstrokecolor{currentstroke}%
\pgfsetdash{}{0pt}%
\pgfpathmoveto{\pgfqpoint{1.972254in}{2.569602in}}%
\pgfpathlineto{\pgfqpoint{2.238213in}{2.742590in}}%
\pgfpathlineto{\pgfqpoint{1.972254in}{2.569602in}}%
\pgfpathclose%
\pgfusepath{fill}%
\end{pgfscope}%
\begin{pgfscope}%
\pgfpathrectangle{\pgfqpoint{0.017500in}{0.017500in}}{\pgfqpoint{6.218282in}{3.580372in}} %
\pgfusepath{clip}%
\pgfsetbuttcap%
\pgfsetroundjoin%
\definecolor{currentfill}{rgb}{0.820401,0.286765,0.245160}%
\pgfsetfillcolor{currentfill}%
\pgfsetlinewidth{0.000000pt}%
\definecolor{currentstroke}{rgb}{0.000000,0.000000,0.000000}%
\pgfsetstrokecolor{currentstroke}%
\pgfsetdash{}{0pt}%
\pgfpathmoveto{\pgfqpoint{1.972254in}{2.569602in}}%
\pgfpathlineto{\pgfqpoint{2.238213in}{2.742590in}}%
\pgfpathlineto{\pgfqpoint{1.972254in}{2.569602in}}%
\pgfpathclose%
\pgfusepath{fill}%
\end{pgfscope}%
\begin{pgfscope}%
\pgfpathrectangle{\pgfqpoint{0.017500in}{0.017500in}}{\pgfqpoint{6.218282in}{3.580372in}} %
\pgfusepath{clip}%
\pgfsetbuttcap%
\pgfsetroundjoin%
\definecolor{currentfill}{rgb}{0.820401,0.286765,0.245160}%
\pgfsetfillcolor{currentfill}%
\pgfsetlinewidth{0.000000pt}%
\definecolor{currentstroke}{rgb}{0.000000,0.000000,0.000000}%
\pgfsetstrokecolor{currentstroke}%
\pgfsetdash{}{0pt}%
\pgfpathmoveto{\pgfqpoint{1.972254in}{2.569602in}}%
\pgfpathlineto{\pgfqpoint{2.238213in}{2.742590in}}%
\pgfpathlineto{\pgfqpoint{1.972254in}{2.569602in}}%
\pgfpathclose%
\pgfusepath{fill}%
\end{pgfscope}%
\begin{pgfscope}%
\pgfpathrectangle{\pgfqpoint{0.017500in}{0.017500in}}{\pgfqpoint{6.218282in}{3.580372in}} %
\pgfusepath{clip}%
\pgfsetbuttcap%
\pgfsetroundjoin%
\definecolor{currentfill}{rgb}{0.820401,0.286765,0.245160}%
\pgfsetfillcolor{currentfill}%
\pgfsetlinewidth{0.000000pt}%
\definecolor{currentstroke}{rgb}{0.000000,0.000000,0.000000}%
\pgfsetstrokecolor{currentstroke}%
\pgfsetdash{}{0pt}%
\pgfpathmoveto{\pgfqpoint{1.972254in}{2.569602in}}%
\pgfpathlineto{\pgfqpoint{2.238213in}{2.742590in}}%
\pgfpathlineto{\pgfqpoint{1.972254in}{2.569602in}}%
\pgfpathclose%
\pgfusepath{fill}%
\end{pgfscope}%
\begin{pgfscope}%
\pgfpathrectangle{\pgfqpoint{0.017500in}{0.017500in}}{\pgfqpoint{6.218282in}{3.580372in}} %
\pgfusepath{clip}%
\pgfsetbuttcap%
\pgfsetroundjoin%
\definecolor{currentfill}{rgb}{0.820401,0.286765,0.245160}%
\pgfsetfillcolor{currentfill}%
\pgfsetlinewidth{0.000000pt}%
\definecolor{currentstroke}{rgb}{0.000000,0.000000,0.000000}%
\pgfsetstrokecolor{currentstroke}%
\pgfsetdash{}{0pt}%
\pgfpathmoveto{\pgfqpoint{1.972254in}{2.569602in}}%
\pgfpathlineto{\pgfqpoint{2.238213in}{2.742590in}}%
\pgfpathlineto{\pgfqpoint{1.972254in}{2.569602in}}%
\pgfpathclose%
\pgfusepath{fill}%
\end{pgfscope}%
\begin{pgfscope}%
\pgfpathrectangle{\pgfqpoint{0.017500in}{0.017500in}}{\pgfqpoint{6.218282in}{3.580372in}} %
\pgfusepath{clip}%
\pgfsetbuttcap%
\pgfsetroundjoin%
\definecolor{currentfill}{rgb}{0.820401,0.286765,0.245160}%
\pgfsetfillcolor{currentfill}%
\pgfsetlinewidth{0.000000pt}%
\definecolor{currentstroke}{rgb}{0.000000,0.000000,0.000000}%
\pgfsetstrokecolor{currentstroke}%
\pgfsetdash{}{0pt}%
\pgfpathmoveto{\pgfqpoint{1.972254in}{2.569602in}}%
\pgfpathlineto{\pgfqpoint{2.238213in}{2.742590in}}%
\pgfpathlineto{\pgfqpoint{1.972254in}{2.569602in}}%
\pgfpathclose%
\pgfusepath{fill}%
\end{pgfscope}%
\begin{pgfscope}%
\pgfpathrectangle{\pgfqpoint{0.017500in}{0.017500in}}{\pgfqpoint{6.218282in}{3.580372in}} %
\pgfusepath{clip}%
\pgfsetbuttcap%
\pgfsetroundjoin%
\definecolor{currentfill}{rgb}{0.820401,0.286765,0.245160}%
\pgfsetfillcolor{currentfill}%
\pgfsetlinewidth{0.000000pt}%
\definecolor{currentstroke}{rgb}{0.000000,0.000000,0.000000}%
\pgfsetstrokecolor{currentstroke}%
\pgfsetdash{}{0pt}%
\pgfpathmoveto{\pgfqpoint{1.972254in}{2.569602in}}%
\pgfpathlineto{\pgfqpoint{2.238213in}{2.742590in}}%
\pgfpathlineto{\pgfqpoint{1.972254in}{2.569602in}}%
\pgfpathclose%
\pgfusepath{fill}%
\end{pgfscope}%
\begin{pgfscope}%
\pgfpathrectangle{\pgfqpoint{0.017500in}{0.017500in}}{\pgfqpoint{6.218282in}{3.580372in}} %
\pgfusepath{clip}%
\pgfsetbuttcap%
\pgfsetroundjoin%
\definecolor{currentfill}{rgb}{0.820401,0.286765,0.245160}%
\pgfsetfillcolor{currentfill}%
\pgfsetlinewidth{0.000000pt}%
\definecolor{currentstroke}{rgb}{0.000000,0.000000,0.000000}%
\pgfsetstrokecolor{currentstroke}%
\pgfsetdash{}{0pt}%
\pgfpathmoveto{\pgfqpoint{1.972254in}{2.569602in}}%
\pgfpathlineto{\pgfqpoint{2.238213in}{2.742590in}}%
\pgfpathlineto{\pgfqpoint{1.972254in}{2.569602in}}%
\pgfpathclose%
\pgfusepath{fill}%
\end{pgfscope}%
\begin{pgfscope}%
\pgfpathrectangle{\pgfqpoint{0.017500in}{0.017500in}}{\pgfqpoint{6.218282in}{3.580372in}} %
\pgfusepath{clip}%
\pgfsetbuttcap%
\pgfsetroundjoin%
\definecolor{currentfill}{rgb}{0.820401,0.286765,0.245160}%
\pgfsetfillcolor{currentfill}%
\pgfsetlinewidth{0.000000pt}%
\definecolor{currentstroke}{rgb}{0.000000,0.000000,0.000000}%
\pgfsetstrokecolor{currentstroke}%
\pgfsetdash{}{0pt}%
\pgfpathmoveto{\pgfqpoint{1.972254in}{2.569602in}}%
\pgfpathlineto{\pgfqpoint{2.238213in}{2.742590in}}%
\pgfpathlineto{\pgfqpoint{1.972254in}{2.569602in}}%
\pgfpathclose%
\pgfusepath{fill}%
\end{pgfscope}%
\begin{pgfscope}%
\pgfpathrectangle{\pgfqpoint{0.017500in}{0.017500in}}{\pgfqpoint{6.218282in}{3.580372in}} %
\pgfusepath{clip}%
\pgfsetbuttcap%
\pgfsetroundjoin%
\definecolor{currentfill}{rgb}{0.820401,0.286765,0.245160}%
\pgfsetfillcolor{currentfill}%
\pgfsetlinewidth{0.000000pt}%
\definecolor{currentstroke}{rgb}{0.000000,0.000000,0.000000}%
\pgfsetstrokecolor{currentstroke}%
\pgfsetdash{}{0pt}%
\pgfpathmoveto{\pgfqpoint{1.972254in}{2.569602in}}%
\pgfpathlineto{\pgfqpoint{2.238213in}{2.742590in}}%
\pgfpathlineto{\pgfqpoint{1.972254in}{2.569602in}}%
\pgfpathclose%
\pgfusepath{fill}%
\end{pgfscope}%
\begin{pgfscope}%
\pgfpathrectangle{\pgfqpoint{0.017500in}{0.017500in}}{\pgfqpoint{6.218282in}{3.580372in}} %
\pgfusepath{clip}%
\pgfsetbuttcap%
\pgfsetroundjoin%
\definecolor{currentfill}{rgb}{0.820401,0.286765,0.245160}%
\pgfsetfillcolor{currentfill}%
\pgfsetlinewidth{0.000000pt}%
\definecolor{currentstroke}{rgb}{0.000000,0.000000,0.000000}%
\pgfsetstrokecolor{currentstroke}%
\pgfsetdash{}{0pt}%
\pgfpathmoveto{\pgfqpoint{1.972254in}{2.569602in}}%
\pgfpathlineto{\pgfqpoint{2.238213in}{2.742590in}}%
\pgfpathlineto{\pgfqpoint{1.972254in}{2.569602in}}%
\pgfpathclose%
\pgfusepath{fill}%
\end{pgfscope}%
\begin{pgfscope}%
\pgfpathrectangle{\pgfqpoint{0.017500in}{0.017500in}}{\pgfqpoint{6.218282in}{3.580372in}} %
\pgfusepath{clip}%
\pgfsetbuttcap%
\pgfsetroundjoin%
\definecolor{currentfill}{rgb}{0.820401,0.286765,0.245160}%
\pgfsetfillcolor{currentfill}%
\pgfsetlinewidth{0.000000pt}%
\definecolor{currentstroke}{rgb}{0.000000,0.000000,0.000000}%
\pgfsetstrokecolor{currentstroke}%
\pgfsetdash{}{0pt}%
\pgfpathmoveto{\pgfqpoint{1.972254in}{2.569602in}}%
\pgfpathlineto{\pgfqpoint{2.238213in}{2.742590in}}%
\pgfpathlineto{\pgfqpoint{1.972254in}{2.569602in}}%
\pgfpathclose%
\pgfusepath{fill}%
\end{pgfscope}%
\begin{pgfscope}%
\pgfpathrectangle{\pgfqpoint{0.017500in}{0.017500in}}{\pgfqpoint{6.218282in}{3.580372in}} %
\pgfusepath{clip}%
\pgfsetbuttcap%
\pgfsetroundjoin%
\definecolor{currentfill}{rgb}{0.820401,0.286765,0.245160}%
\pgfsetfillcolor{currentfill}%
\pgfsetlinewidth{0.000000pt}%
\definecolor{currentstroke}{rgb}{0.000000,0.000000,0.000000}%
\pgfsetstrokecolor{currentstroke}%
\pgfsetdash{}{0pt}%
\pgfpathmoveto{\pgfqpoint{1.972254in}{2.569602in}}%
\pgfpathlineto{\pgfqpoint{2.238213in}{2.742590in}}%
\pgfpathlineto{\pgfqpoint{1.972254in}{2.569602in}}%
\pgfpathclose%
\pgfusepath{fill}%
\end{pgfscope}%
\begin{pgfscope}%
\pgfpathrectangle{\pgfqpoint{0.017500in}{0.017500in}}{\pgfqpoint{6.218282in}{3.580372in}} %
\pgfusepath{clip}%
\pgfsetbuttcap%
\pgfsetroundjoin%
\definecolor{currentfill}{rgb}{0.820401,0.286765,0.245160}%
\pgfsetfillcolor{currentfill}%
\pgfsetlinewidth{0.000000pt}%
\definecolor{currentstroke}{rgb}{0.000000,0.000000,0.000000}%
\pgfsetstrokecolor{currentstroke}%
\pgfsetdash{}{0pt}%
\pgfpathmoveto{\pgfqpoint{1.972254in}{2.569602in}}%
\pgfpathlineto{\pgfqpoint{2.238213in}{2.742590in}}%
\pgfpathlineto{\pgfqpoint{1.972254in}{2.569602in}}%
\pgfpathclose%
\pgfusepath{fill}%
\end{pgfscope}%
\begin{pgfscope}%
\pgfpathrectangle{\pgfqpoint{0.017500in}{0.017500in}}{\pgfqpoint{6.218282in}{3.580372in}} %
\pgfusepath{clip}%
\pgfsetbuttcap%
\pgfsetroundjoin%
\definecolor{currentfill}{rgb}{0.820401,0.286765,0.245160}%
\pgfsetfillcolor{currentfill}%
\pgfsetlinewidth{0.000000pt}%
\definecolor{currentstroke}{rgb}{0.000000,0.000000,0.000000}%
\pgfsetstrokecolor{currentstroke}%
\pgfsetdash{}{0pt}%
\pgfpathmoveto{\pgfqpoint{1.972254in}{2.569602in}}%
\pgfpathlineto{\pgfqpoint{2.238213in}{2.742590in}}%
\pgfpathlineto{\pgfqpoint{1.972254in}{2.569602in}}%
\pgfpathclose%
\pgfusepath{fill}%
\end{pgfscope}%
\begin{pgfscope}%
\pgfpathrectangle{\pgfqpoint{0.017500in}{0.017500in}}{\pgfqpoint{6.218282in}{3.580372in}} %
\pgfusepath{clip}%
\pgfsetbuttcap%
\pgfsetroundjoin%
\definecolor{currentfill}{rgb}{0.820401,0.286765,0.245160}%
\pgfsetfillcolor{currentfill}%
\pgfsetlinewidth{0.000000pt}%
\definecolor{currentstroke}{rgb}{0.000000,0.000000,0.000000}%
\pgfsetstrokecolor{currentstroke}%
\pgfsetdash{}{0pt}%
\pgfpathmoveto{\pgfqpoint{1.972254in}{2.569602in}}%
\pgfpathlineto{\pgfqpoint{2.238213in}{2.742590in}}%
\pgfpathlineto{\pgfqpoint{1.972254in}{2.569602in}}%
\pgfpathclose%
\pgfusepath{fill}%
\end{pgfscope}%
\begin{pgfscope}%
\pgfpathrectangle{\pgfqpoint{0.017500in}{0.017500in}}{\pgfqpoint{6.218282in}{3.580372in}} %
\pgfusepath{clip}%
\pgfsetbuttcap%
\pgfsetroundjoin%
\definecolor{currentfill}{rgb}{0.820401,0.286765,0.245160}%
\pgfsetfillcolor{currentfill}%
\pgfsetlinewidth{0.000000pt}%
\definecolor{currentstroke}{rgb}{0.000000,0.000000,0.000000}%
\pgfsetstrokecolor{currentstroke}%
\pgfsetdash{}{0pt}%
\pgfpathmoveto{\pgfqpoint{1.972254in}{2.569602in}}%
\pgfpathlineto{\pgfqpoint{2.238213in}{2.742590in}}%
\pgfpathlineto{\pgfqpoint{1.972254in}{2.569602in}}%
\pgfpathclose%
\pgfusepath{fill}%
\end{pgfscope}%
\begin{pgfscope}%
\pgfpathrectangle{\pgfqpoint{0.017500in}{0.017500in}}{\pgfqpoint{6.218282in}{3.580372in}} %
\pgfusepath{clip}%
\pgfsetbuttcap%
\pgfsetroundjoin%
\definecolor{currentfill}{rgb}{0.820401,0.286765,0.245160}%
\pgfsetfillcolor{currentfill}%
\pgfsetlinewidth{0.000000pt}%
\definecolor{currentstroke}{rgb}{0.000000,0.000000,0.000000}%
\pgfsetstrokecolor{currentstroke}%
\pgfsetdash{}{0pt}%
\pgfpathmoveto{\pgfqpoint{1.972254in}{2.569602in}}%
\pgfpathlineto{\pgfqpoint{2.238213in}{2.742590in}}%
\pgfpathlineto{\pgfqpoint{1.972254in}{2.569602in}}%
\pgfpathclose%
\pgfusepath{fill}%
\end{pgfscope}%
\begin{pgfscope}%
\pgfpathrectangle{\pgfqpoint{0.017500in}{0.017500in}}{\pgfqpoint{6.218282in}{3.580372in}} %
\pgfusepath{clip}%
\pgfsetbuttcap%
\pgfsetroundjoin%
\definecolor{currentfill}{rgb}{0.820401,0.286765,0.245160}%
\pgfsetfillcolor{currentfill}%
\pgfsetlinewidth{0.000000pt}%
\definecolor{currentstroke}{rgb}{0.000000,0.000000,0.000000}%
\pgfsetstrokecolor{currentstroke}%
\pgfsetdash{}{0pt}%
\pgfpathmoveto{\pgfqpoint{1.972254in}{2.569602in}}%
\pgfpathlineto{\pgfqpoint{2.238213in}{2.742590in}}%
\pgfpathlineto{\pgfqpoint{1.972254in}{2.569602in}}%
\pgfpathclose%
\pgfusepath{fill}%
\end{pgfscope}%
\begin{pgfscope}%
\pgfpathrectangle{\pgfqpoint{0.017500in}{0.017500in}}{\pgfqpoint{6.218282in}{3.580372in}} %
\pgfusepath{clip}%
\pgfsetbuttcap%
\pgfsetroundjoin%
\definecolor{currentfill}{rgb}{0.820401,0.286765,0.245160}%
\pgfsetfillcolor{currentfill}%
\pgfsetlinewidth{0.000000pt}%
\definecolor{currentstroke}{rgb}{0.000000,0.000000,0.000000}%
\pgfsetstrokecolor{currentstroke}%
\pgfsetdash{}{0pt}%
\pgfpathmoveto{\pgfqpoint{1.972254in}{2.569602in}}%
\pgfpathlineto{\pgfqpoint{2.238213in}{2.742590in}}%
\pgfpathlineto{\pgfqpoint{1.972254in}{2.569602in}}%
\pgfpathclose%
\pgfusepath{fill}%
\end{pgfscope}%
\begin{pgfscope}%
\pgfpathrectangle{\pgfqpoint{0.017500in}{0.017500in}}{\pgfqpoint{6.218282in}{3.580372in}} %
\pgfusepath{clip}%
\pgfsetbuttcap%
\pgfsetroundjoin%
\definecolor{currentfill}{rgb}{0.820401,0.286765,0.245160}%
\pgfsetfillcolor{currentfill}%
\pgfsetlinewidth{0.000000pt}%
\definecolor{currentstroke}{rgb}{0.000000,0.000000,0.000000}%
\pgfsetstrokecolor{currentstroke}%
\pgfsetdash{}{0pt}%
\pgfpathmoveto{\pgfqpoint{1.972254in}{2.569602in}}%
\pgfpathlineto{\pgfqpoint{2.238213in}{2.742590in}}%
\pgfpathlineto{\pgfqpoint{1.972254in}{2.569602in}}%
\pgfpathclose%
\pgfusepath{fill}%
\end{pgfscope}%
\begin{pgfscope}%
\pgfpathrectangle{\pgfqpoint{0.017500in}{0.017500in}}{\pgfqpoint{6.218282in}{3.580372in}} %
\pgfusepath{clip}%
\pgfsetbuttcap%
\pgfsetroundjoin%
\definecolor{currentfill}{rgb}{0.820401,0.286765,0.245160}%
\pgfsetfillcolor{currentfill}%
\pgfsetlinewidth{0.000000pt}%
\definecolor{currentstroke}{rgb}{0.000000,0.000000,0.000000}%
\pgfsetstrokecolor{currentstroke}%
\pgfsetdash{}{0pt}%
\pgfpathmoveto{\pgfqpoint{1.972254in}{2.569602in}}%
\pgfpathlineto{\pgfqpoint{2.238213in}{2.742590in}}%
\pgfpathlineto{\pgfqpoint{1.972254in}{2.569602in}}%
\pgfpathclose%
\pgfusepath{fill}%
\end{pgfscope}%
\begin{pgfscope}%
\pgfpathrectangle{\pgfqpoint{0.017500in}{0.017500in}}{\pgfqpoint{6.218282in}{3.580372in}} %
\pgfusepath{clip}%
\pgfsetbuttcap%
\pgfsetroundjoin%
\definecolor{currentfill}{rgb}{0.820401,0.286765,0.245160}%
\pgfsetfillcolor{currentfill}%
\pgfsetlinewidth{0.000000pt}%
\definecolor{currentstroke}{rgb}{0.000000,0.000000,0.000000}%
\pgfsetstrokecolor{currentstroke}%
\pgfsetdash{}{0pt}%
\pgfpathmoveto{\pgfqpoint{1.972254in}{2.569602in}}%
\pgfpathlineto{\pgfqpoint{2.238213in}{2.742590in}}%
\pgfpathlineto{\pgfqpoint{1.972254in}{2.569602in}}%
\pgfpathclose%
\pgfusepath{fill}%
\end{pgfscope}%
\begin{pgfscope}%
\pgfpathrectangle{\pgfqpoint{0.017500in}{0.017500in}}{\pgfqpoint{6.218282in}{3.580372in}} %
\pgfusepath{clip}%
\pgfsetbuttcap%
\pgfsetroundjoin%
\definecolor{currentfill}{rgb}{0.820401,0.286765,0.245160}%
\pgfsetfillcolor{currentfill}%
\pgfsetlinewidth{0.000000pt}%
\definecolor{currentstroke}{rgb}{0.000000,0.000000,0.000000}%
\pgfsetstrokecolor{currentstroke}%
\pgfsetdash{}{0pt}%
\pgfpathmoveto{\pgfqpoint{1.972254in}{2.569602in}}%
\pgfpathlineto{\pgfqpoint{2.238213in}{2.742590in}}%
\pgfpathlineto{\pgfqpoint{1.972254in}{2.569602in}}%
\pgfpathclose%
\pgfusepath{fill}%
\end{pgfscope}%
\begin{pgfscope}%
\pgfpathrectangle{\pgfqpoint{0.017500in}{0.017500in}}{\pgfqpoint{6.218282in}{3.580372in}} %
\pgfusepath{clip}%
\pgfsetbuttcap%
\pgfsetroundjoin%
\definecolor{currentfill}{rgb}{0.820401,0.286765,0.245160}%
\pgfsetfillcolor{currentfill}%
\pgfsetlinewidth{0.000000pt}%
\definecolor{currentstroke}{rgb}{0.000000,0.000000,0.000000}%
\pgfsetstrokecolor{currentstroke}%
\pgfsetdash{}{0pt}%
\pgfpathmoveto{\pgfqpoint{1.972254in}{2.569602in}}%
\pgfpathlineto{\pgfqpoint{2.238213in}{2.742590in}}%
\pgfpathlineto{\pgfqpoint{1.972254in}{2.569602in}}%
\pgfpathclose%
\pgfusepath{fill}%
\end{pgfscope}%
\begin{pgfscope}%
\pgfpathrectangle{\pgfqpoint{0.017500in}{0.017500in}}{\pgfqpoint{6.218282in}{3.580372in}} %
\pgfusepath{clip}%
\pgfsetbuttcap%
\pgfsetroundjoin%
\definecolor{currentfill}{rgb}{0.333490,0.446265,0.874452}%
\pgfsetfillcolor{currentfill}%
\pgfsetlinewidth{0.000000pt}%
\definecolor{currentstroke}{rgb}{0.000000,0.000000,0.000000}%
\pgfsetstrokecolor{currentstroke}%
\pgfsetdash{}{0pt}%
\pgfpathmoveto{\pgfqpoint{1.486672in}{1.030836in}}%
\pgfpathlineto{\pgfqpoint{1.915196in}{0.958373in}}%
\pgfpathlineto{\pgfqpoint{2.142608in}{2.314596in}}%
\pgfpathlineto{\pgfqpoint{1.702428in}{2.361240in}}%
\pgfpathlineto{\pgfqpoint{1.486672in}{1.030836in}}%
\pgfpathclose%
\pgfusepath{fill}%
\end{pgfscope}%
\begin{pgfscope}%
\pgfpathrectangle{\pgfqpoint{0.017500in}{0.017500in}}{\pgfqpoint{6.218282in}{3.580372in}} %
\pgfusepath{clip}%
\pgfsetbuttcap%
\pgfsetroundjoin%
\definecolor{currentfill}{rgb}{0.333490,0.446265,0.874452}%
\pgfsetfillcolor{currentfill}%
\pgfsetlinewidth{0.000000pt}%
\definecolor{currentstroke}{rgb}{0.000000,0.000000,0.000000}%
\pgfsetstrokecolor{currentstroke}%
\pgfsetdash{}{0pt}%
\pgfpathmoveto{\pgfqpoint{1.486672in}{1.030836in}}%
\pgfpathlineto{\pgfqpoint{1.915196in}{0.958373in}}%
\pgfpathlineto{\pgfqpoint{2.142608in}{2.314596in}}%
\pgfpathlineto{\pgfqpoint{1.702428in}{2.361240in}}%
\pgfpathlineto{\pgfqpoint{1.486672in}{1.030836in}}%
\pgfpathclose%
\pgfusepath{fill}%
\end{pgfscope}%
\begin{pgfscope}%
\pgfpathrectangle{\pgfqpoint{0.017500in}{0.017500in}}{\pgfqpoint{6.218282in}{3.580372in}} %
\pgfusepath{clip}%
\pgfsetbuttcap%
\pgfsetroundjoin%
\definecolor{currentfill}{rgb}{0.333490,0.446265,0.874452}%
\pgfsetfillcolor{currentfill}%
\pgfsetlinewidth{0.000000pt}%
\definecolor{currentstroke}{rgb}{0.000000,0.000000,0.000000}%
\pgfsetstrokecolor{currentstroke}%
\pgfsetdash{}{0pt}%
\pgfpathmoveto{\pgfqpoint{1.486672in}{1.030836in}}%
\pgfpathlineto{\pgfqpoint{1.915196in}{0.958373in}}%
\pgfpathlineto{\pgfqpoint{2.142608in}{2.314596in}}%
\pgfpathlineto{\pgfqpoint{1.702428in}{2.361240in}}%
\pgfpathlineto{\pgfqpoint{1.486672in}{1.030836in}}%
\pgfpathclose%
\pgfusepath{fill}%
\end{pgfscope}%
\begin{pgfscope}%
\pgfpathrectangle{\pgfqpoint{0.017500in}{0.017500in}}{\pgfqpoint{6.218282in}{3.580372in}} %
\pgfusepath{clip}%
\pgfsetbuttcap%
\pgfsetroundjoin%
\definecolor{currentfill}{rgb}{0.333490,0.446265,0.874452}%
\pgfsetfillcolor{currentfill}%
\pgfsetlinewidth{0.000000pt}%
\definecolor{currentstroke}{rgb}{0.000000,0.000000,0.000000}%
\pgfsetstrokecolor{currentstroke}%
\pgfsetdash{}{0pt}%
\pgfpathmoveto{\pgfqpoint{1.486672in}{1.030836in}}%
\pgfpathlineto{\pgfqpoint{1.915196in}{0.958373in}}%
\pgfpathlineto{\pgfqpoint{2.142608in}{2.314596in}}%
\pgfpathlineto{\pgfqpoint{1.702428in}{2.361240in}}%
\pgfpathlineto{\pgfqpoint{1.486672in}{1.030836in}}%
\pgfpathclose%
\pgfusepath{fill}%
\end{pgfscope}%
\begin{pgfscope}%
\pgfpathrectangle{\pgfqpoint{0.017500in}{0.017500in}}{\pgfqpoint{6.218282in}{3.580372in}} %
\pgfusepath{clip}%
\pgfsetbuttcap%
\pgfsetroundjoin%
\definecolor{currentfill}{rgb}{0.333490,0.446265,0.874452}%
\pgfsetfillcolor{currentfill}%
\pgfsetlinewidth{0.000000pt}%
\definecolor{currentstroke}{rgb}{0.000000,0.000000,0.000000}%
\pgfsetstrokecolor{currentstroke}%
\pgfsetdash{}{0pt}%
\pgfpathmoveto{\pgfqpoint{1.486672in}{1.030836in}}%
\pgfpathlineto{\pgfqpoint{1.915196in}{0.958373in}}%
\pgfpathlineto{\pgfqpoint{2.142608in}{2.314596in}}%
\pgfpathlineto{\pgfqpoint{1.702428in}{2.361240in}}%
\pgfpathlineto{\pgfqpoint{1.486672in}{1.030836in}}%
\pgfpathclose%
\pgfusepath{fill}%
\end{pgfscope}%
\begin{pgfscope}%
\pgfpathrectangle{\pgfqpoint{0.017500in}{0.017500in}}{\pgfqpoint{6.218282in}{3.580372in}} %
\pgfusepath{clip}%
\pgfsetbuttcap%
\pgfsetroundjoin%
\definecolor{currentfill}{rgb}{0.333490,0.446265,0.874452}%
\pgfsetfillcolor{currentfill}%
\pgfsetlinewidth{0.000000pt}%
\definecolor{currentstroke}{rgb}{0.000000,0.000000,0.000000}%
\pgfsetstrokecolor{currentstroke}%
\pgfsetdash{}{0pt}%
\pgfpathmoveto{\pgfqpoint{1.486672in}{1.030836in}}%
\pgfpathlineto{\pgfqpoint{1.915196in}{0.958373in}}%
\pgfpathlineto{\pgfqpoint{2.142608in}{2.314596in}}%
\pgfpathlineto{\pgfqpoint{1.702428in}{2.361240in}}%
\pgfpathlineto{\pgfqpoint{1.486672in}{1.030836in}}%
\pgfpathclose%
\pgfusepath{fill}%
\end{pgfscope}%
\begin{pgfscope}%
\pgfpathrectangle{\pgfqpoint{0.017500in}{0.017500in}}{\pgfqpoint{6.218282in}{3.580372in}} %
\pgfusepath{clip}%
\pgfsetbuttcap%
\pgfsetroundjoin%
\definecolor{currentfill}{rgb}{0.804965,0.851666,0.926165}%
\pgfsetfillcolor{currentfill}%
\pgfsetlinewidth{0.000000pt}%
\definecolor{currentstroke}{rgb}{0.000000,0.000000,0.000000}%
\pgfsetstrokecolor{currentstroke}%
\pgfsetdash{}{0pt}%
\pgfpathmoveto{\pgfqpoint{4.063757in}{2.816074in}}%
\pgfpathlineto{\pgfqpoint{4.511715in}{2.762029in}}%
\pgfpathlineto{\pgfqpoint{3.236097in}{0.721171in}}%
\pgfpathlineto{\pgfqpoint{2.789661in}{0.820654in}}%
\pgfpathlineto{\pgfqpoint{4.063757in}{2.816074in}}%
\pgfpathclose%
\pgfusepath{fill}%
\end{pgfscope}%
\begin{pgfscope}%
\pgfpathrectangle{\pgfqpoint{0.017500in}{0.017500in}}{\pgfqpoint{6.218282in}{3.580372in}} %
\pgfusepath{clip}%
\pgfsetbuttcap%
\pgfsetroundjoin%
\definecolor{currentfill}{rgb}{0.804965,0.851666,0.926165}%
\pgfsetfillcolor{currentfill}%
\pgfsetlinewidth{0.000000pt}%
\definecolor{currentstroke}{rgb}{0.000000,0.000000,0.000000}%
\pgfsetstrokecolor{currentstroke}%
\pgfsetdash{}{0pt}%
\pgfpathmoveto{\pgfqpoint{4.063757in}{2.816074in}}%
\pgfpathlineto{\pgfqpoint{4.511715in}{2.762029in}}%
\pgfpathlineto{\pgfqpoint{3.236097in}{0.721171in}}%
\pgfpathlineto{\pgfqpoint{2.789661in}{0.820654in}}%
\pgfpathlineto{\pgfqpoint{4.063757in}{2.816074in}}%
\pgfpathclose%
\pgfusepath{fill}%
\end{pgfscope}%
\begin{pgfscope}%
\pgfpathrectangle{\pgfqpoint{0.017500in}{0.017500in}}{\pgfqpoint{6.218282in}{3.580372in}} %
\pgfusepath{clip}%
\pgfsetbuttcap%
\pgfsetroundjoin%
\definecolor{currentfill}{rgb}{0.804965,0.851666,0.926165}%
\pgfsetfillcolor{currentfill}%
\pgfsetlinewidth{0.000000pt}%
\definecolor{currentstroke}{rgb}{0.000000,0.000000,0.000000}%
\pgfsetstrokecolor{currentstroke}%
\pgfsetdash{}{0pt}%
\pgfpathmoveto{\pgfqpoint{4.063757in}{2.816074in}}%
\pgfpathlineto{\pgfqpoint{4.511715in}{2.762029in}}%
\pgfpathlineto{\pgfqpoint{3.236097in}{0.721171in}}%
\pgfpathlineto{\pgfqpoint{2.789661in}{0.820654in}}%
\pgfpathlineto{\pgfqpoint{4.063757in}{2.816074in}}%
\pgfpathclose%
\pgfusepath{fill}%
\end{pgfscope}%
\begin{pgfscope}%
\pgfpathrectangle{\pgfqpoint{0.017500in}{0.017500in}}{\pgfqpoint{6.218282in}{3.580372in}} %
\pgfusepath{clip}%
\pgfsetbuttcap%
\pgfsetroundjoin%
\definecolor{currentfill}{rgb}{0.804965,0.851666,0.926165}%
\pgfsetfillcolor{currentfill}%
\pgfsetlinewidth{0.000000pt}%
\definecolor{currentstroke}{rgb}{0.000000,0.000000,0.000000}%
\pgfsetstrokecolor{currentstroke}%
\pgfsetdash{}{0pt}%
\pgfpathmoveto{\pgfqpoint{4.063757in}{2.816074in}}%
\pgfpathlineto{\pgfqpoint{4.511715in}{2.762029in}}%
\pgfpathlineto{\pgfqpoint{3.236097in}{0.721171in}}%
\pgfpathlineto{\pgfqpoint{2.789661in}{0.820654in}}%
\pgfpathlineto{\pgfqpoint{4.063757in}{2.816074in}}%
\pgfpathclose%
\pgfusepath{fill}%
\end{pgfscope}%
\begin{pgfscope}%
\pgfpathrectangle{\pgfqpoint{0.017500in}{0.017500in}}{\pgfqpoint{6.218282in}{3.580372in}} %
\pgfusepath{clip}%
\pgfsetbuttcap%
\pgfsetroundjoin%
\definecolor{currentfill}{rgb}{0.804965,0.851666,0.926165}%
\pgfsetfillcolor{currentfill}%
\pgfsetlinewidth{0.000000pt}%
\definecolor{currentstroke}{rgb}{0.000000,0.000000,0.000000}%
\pgfsetstrokecolor{currentstroke}%
\pgfsetdash{}{0pt}%
\pgfpathmoveto{\pgfqpoint{4.063757in}{2.816074in}}%
\pgfpathlineto{\pgfqpoint{4.511715in}{2.762029in}}%
\pgfpathlineto{\pgfqpoint{3.236097in}{0.721171in}}%
\pgfpathlineto{\pgfqpoint{2.789661in}{0.820654in}}%
\pgfpathlineto{\pgfqpoint{4.063757in}{2.816074in}}%
\pgfpathclose%
\pgfusepath{fill}%
\end{pgfscope}%
\begin{pgfscope}%
\pgfpathrectangle{\pgfqpoint{0.017500in}{0.017500in}}{\pgfqpoint{6.218282in}{3.580372in}} %
\pgfusepath{clip}%
\pgfsetbuttcap%
\pgfsetroundjoin%
\definecolor{currentfill}{rgb}{0.711554,0.033337,0.154485}%
\pgfsetfillcolor{currentfill}%
\pgfsetlinewidth{0.000000pt}%
\definecolor{currentstroke}{rgb}{0.000000,0.000000,0.000000}%
\pgfsetstrokecolor{currentstroke}%
\pgfsetdash{}{0pt}%
\pgfpathmoveto{\pgfqpoint{3.817122in}{2.696128in}}%
\pgfpathlineto{\pgfqpoint{4.268228in}{2.632966in}}%
\pgfpathlineto{\pgfqpoint{4.511715in}{2.762029in}}%
\pgfpathlineto{\pgfqpoint{4.063757in}{2.816074in}}%
\pgfpathlineto{\pgfqpoint{3.817122in}{2.696128in}}%
\pgfpathclose%
\pgfusepath{fill}%
\end{pgfscope}%
\begin{pgfscope}%
\pgfpathrectangle{\pgfqpoint{0.017500in}{0.017500in}}{\pgfqpoint{6.218282in}{3.580372in}} %
\pgfusepath{clip}%
\pgfsetbuttcap%
\pgfsetroundjoin%
\definecolor{currentfill}{rgb}{0.711554,0.033337,0.154485}%
\pgfsetfillcolor{currentfill}%
\pgfsetlinewidth{0.000000pt}%
\definecolor{currentstroke}{rgb}{0.000000,0.000000,0.000000}%
\pgfsetstrokecolor{currentstroke}%
\pgfsetdash{}{0pt}%
\pgfpathmoveto{\pgfqpoint{3.817122in}{2.696128in}}%
\pgfpathlineto{\pgfqpoint{4.268228in}{2.632966in}}%
\pgfpathlineto{\pgfqpoint{4.511715in}{2.762029in}}%
\pgfpathlineto{\pgfqpoint{4.063757in}{2.816074in}}%
\pgfpathlineto{\pgfqpoint{3.817122in}{2.696128in}}%
\pgfpathclose%
\pgfusepath{fill}%
\end{pgfscope}%
\begin{pgfscope}%
\pgfpathrectangle{\pgfqpoint{0.017500in}{0.017500in}}{\pgfqpoint{6.218282in}{3.580372in}} %
\pgfusepath{clip}%
\pgfsetbuttcap%
\pgfsetroundjoin%
\definecolor{currentfill}{rgb}{0.711554,0.033337,0.154485}%
\pgfsetfillcolor{currentfill}%
\pgfsetlinewidth{0.000000pt}%
\definecolor{currentstroke}{rgb}{0.000000,0.000000,0.000000}%
\pgfsetstrokecolor{currentstroke}%
\pgfsetdash{}{0pt}%
\pgfpathmoveto{\pgfqpoint{3.817122in}{2.696128in}}%
\pgfpathlineto{\pgfqpoint{4.268228in}{2.632966in}}%
\pgfpathlineto{\pgfqpoint{4.511715in}{2.762029in}}%
\pgfpathlineto{\pgfqpoint{4.063757in}{2.816074in}}%
\pgfpathlineto{\pgfqpoint{3.817122in}{2.696128in}}%
\pgfpathclose%
\pgfusepath{fill}%
\end{pgfscope}%
\begin{pgfscope}%
\pgfpathrectangle{\pgfqpoint{0.017500in}{0.017500in}}{\pgfqpoint{6.218282in}{3.580372in}} %
\pgfusepath{clip}%
\pgfsetbuttcap%
\pgfsetroundjoin%
\definecolor{currentfill}{rgb}{0.711554,0.033337,0.154485}%
\pgfsetfillcolor{currentfill}%
\pgfsetlinewidth{0.000000pt}%
\definecolor{currentstroke}{rgb}{0.000000,0.000000,0.000000}%
\pgfsetstrokecolor{currentstroke}%
\pgfsetdash{}{0pt}%
\pgfpathmoveto{\pgfqpoint{3.817122in}{2.696128in}}%
\pgfpathlineto{\pgfqpoint{4.268228in}{2.632966in}}%
\pgfpathlineto{\pgfqpoint{4.511715in}{2.762029in}}%
\pgfpathlineto{\pgfqpoint{4.063757in}{2.816074in}}%
\pgfpathlineto{\pgfqpoint{3.817122in}{2.696128in}}%
\pgfpathclose%
\pgfusepath{fill}%
\end{pgfscope}%
\begin{pgfscope}%
\pgfpathrectangle{\pgfqpoint{0.017500in}{0.017500in}}{\pgfqpoint{6.218282in}{3.580372in}} %
\pgfusepath{clip}%
\pgfsetbuttcap%
\pgfsetroundjoin%
\definecolor{currentfill}{rgb}{0.711554,0.033337,0.154485}%
\pgfsetfillcolor{currentfill}%
\pgfsetlinewidth{0.000000pt}%
\definecolor{currentstroke}{rgb}{0.000000,0.000000,0.000000}%
\pgfsetstrokecolor{currentstroke}%
\pgfsetdash{}{0pt}%
\pgfpathmoveto{\pgfqpoint{3.817122in}{2.696128in}}%
\pgfpathlineto{\pgfqpoint{4.268228in}{2.632966in}}%
\pgfpathlineto{\pgfqpoint{4.511715in}{2.762029in}}%
\pgfpathlineto{\pgfqpoint{4.063757in}{2.816074in}}%
\pgfpathlineto{\pgfqpoint{3.817122in}{2.696128in}}%
\pgfpathclose%
\pgfusepath{fill}%
\end{pgfscope}%
\begin{pgfscope}%
\pgfpathrectangle{\pgfqpoint{0.017500in}{0.017500in}}{\pgfqpoint{6.218282in}{3.580372in}} %
\pgfusepath{clip}%
\pgfsetbuttcap%
\pgfsetroundjoin%
\definecolor{currentfill}{rgb}{0.711554,0.033337,0.154485}%
\pgfsetfillcolor{currentfill}%
\pgfsetlinewidth{0.000000pt}%
\definecolor{currentstroke}{rgb}{0.000000,0.000000,0.000000}%
\pgfsetstrokecolor{currentstroke}%
\pgfsetdash{}{0pt}%
\pgfpathmoveto{\pgfqpoint{3.817122in}{2.696128in}}%
\pgfpathlineto{\pgfqpoint{4.268228in}{2.632966in}}%
\pgfpathlineto{\pgfqpoint{4.511715in}{2.762029in}}%
\pgfpathlineto{\pgfqpoint{4.063757in}{2.816074in}}%
\pgfpathlineto{\pgfqpoint{3.817122in}{2.696128in}}%
\pgfpathclose%
\pgfusepath{fill}%
\end{pgfscope}%
\begin{pgfscope}%
\pgfpathrectangle{\pgfqpoint{0.017500in}{0.017500in}}{\pgfqpoint{6.218282in}{3.580372in}} %
\pgfusepath{clip}%
\pgfsetbuttcap%
\pgfsetroundjoin%
\definecolor{currentfill}{rgb}{0.805723,0.259813,0.230562}%
\pgfsetfillcolor{currentfill}%
\pgfsetlinewidth{0.000000pt}%
\definecolor{currentstroke}{rgb}{0.000000,0.000000,0.000000}%
\pgfsetstrokecolor{currentstroke}%
\pgfsetdash{}{0pt}%
\pgfpathmoveto{\pgfqpoint{1.972254in}{2.569602in}}%
\pgfpathlineto{\pgfqpoint{2.410172in}{2.537974in}}%
\pgfpathlineto{\pgfqpoint{2.674281in}{2.686821in}}%
\pgfpathlineto{\pgfqpoint{2.238213in}{2.742590in}}%
\pgfpathlineto{\pgfqpoint{1.972254in}{2.569602in}}%
\pgfpathclose%
\pgfusepath{fill}%
\end{pgfscope}%
\begin{pgfscope}%
\pgfpathrectangle{\pgfqpoint{0.017500in}{0.017500in}}{\pgfqpoint{6.218282in}{3.580372in}} %
\pgfusepath{clip}%
\pgfsetbuttcap%
\pgfsetroundjoin%
\definecolor{currentfill}{rgb}{0.805723,0.259813,0.230562}%
\pgfsetfillcolor{currentfill}%
\pgfsetlinewidth{0.000000pt}%
\definecolor{currentstroke}{rgb}{0.000000,0.000000,0.000000}%
\pgfsetstrokecolor{currentstroke}%
\pgfsetdash{}{0pt}%
\pgfpathmoveto{\pgfqpoint{1.972254in}{2.569602in}}%
\pgfpathlineto{\pgfqpoint{2.410172in}{2.537974in}}%
\pgfpathlineto{\pgfqpoint{2.674281in}{2.686821in}}%
\pgfpathlineto{\pgfqpoint{2.238213in}{2.742590in}}%
\pgfpathlineto{\pgfqpoint{1.972254in}{2.569602in}}%
\pgfpathclose%
\pgfusepath{fill}%
\end{pgfscope}%
\begin{pgfscope}%
\pgfpathrectangle{\pgfqpoint{0.017500in}{0.017500in}}{\pgfqpoint{6.218282in}{3.580372in}} %
\pgfusepath{clip}%
\pgfsetbuttcap%
\pgfsetroundjoin%
\definecolor{currentfill}{rgb}{0.805723,0.259813,0.230562}%
\pgfsetfillcolor{currentfill}%
\pgfsetlinewidth{0.000000pt}%
\definecolor{currentstroke}{rgb}{0.000000,0.000000,0.000000}%
\pgfsetstrokecolor{currentstroke}%
\pgfsetdash{}{0pt}%
\pgfpathmoveto{\pgfqpoint{1.972254in}{2.569602in}}%
\pgfpathlineto{\pgfqpoint{2.410172in}{2.537974in}}%
\pgfpathlineto{\pgfqpoint{2.674281in}{2.686821in}}%
\pgfpathlineto{\pgfqpoint{2.238213in}{2.742590in}}%
\pgfpathlineto{\pgfqpoint{1.972254in}{2.569602in}}%
\pgfpathclose%
\pgfusepath{fill}%
\end{pgfscope}%
\begin{pgfscope}%
\pgfpathrectangle{\pgfqpoint{0.017500in}{0.017500in}}{\pgfqpoint{6.218282in}{3.580372in}} %
\pgfusepath{clip}%
\pgfsetbuttcap%
\pgfsetroundjoin%
\definecolor{currentfill}{rgb}{0.805723,0.259813,0.230562}%
\pgfsetfillcolor{currentfill}%
\pgfsetlinewidth{0.000000pt}%
\definecolor{currentstroke}{rgb}{0.000000,0.000000,0.000000}%
\pgfsetstrokecolor{currentstroke}%
\pgfsetdash{}{0pt}%
\pgfpathmoveto{\pgfqpoint{1.972254in}{2.569602in}}%
\pgfpathlineto{\pgfqpoint{2.410172in}{2.537974in}}%
\pgfpathlineto{\pgfqpoint{2.674281in}{2.686821in}}%
\pgfpathlineto{\pgfqpoint{2.238213in}{2.742590in}}%
\pgfpathlineto{\pgfqpoint{1.972254in}{2.569602in}}%
\pgfpathclose%
\pgfusepath{fill}%
\end{pgfscope}%
\begin{pgfscope}%
\pgfpathrectangle{\pgfqpoint{0.017500in}{0.017500in}}{\pgfqpoint{6.218282in}{3.580372in}} %
\pgfusepath{clip}%
\pgfsetbuttcap%
\pgfsetroundjoin%
\definecolor{currentfill}{rgb}{0.805723,0.259813,0.230562}%
\pgfsetfillcolor{currentfill}%
\pgfsetlinewidth{0.000000pt}%
\definecolor{currentstroke}{rgb}{0.000000,0.000000,0.000000}%
\pgfsetstrokecolor{currentstroke}%
\pgfsetdash{}{0pt}%
\pgfpathmoveto{\pgfqpoint{1.972254in}{2.569602in}}%
\pgfpathlineto{\pgfqpoint{2.410172in}{2.537974in}}%
\pgfpathlineto{\pgfqpoint{2.674281in}{2.686821in}}%
\pgfpathlineto{\pgfqpoint{2.238213in}{2.742590in}}%
\pgfpathlineto{\pgfqpoint{1.972254in}{2.569602in}}%
\pgfpathclose%
\pgfusepath{fill}%
\end{pgfscope}%
\begin{pgfscope}%
\pgfpathrectangle{\pgfqpoint{0.017500in}{0.017500in}}{\pgfqpoint{6.218282in}{3.580372in}} %
\pgfusepath{clip}%
\pgfsetbuttcap%
\pgfsetroundjoin%
\definecolor{currentfill}{rgb}{0.805723,0.259813,0.230562}%
\pgfsetfillcolor{currentfill}%
\pgfsetlinewidth{0.000000pt}%
\definecolor{currentstroke}{rgb}{0.000000,0.000000,0.000000}%
\pgfsetstrokecolor{currentstroke}%
\pgfsetdash{}{0pt}%
\pgfpathmoveto{\pgfqpoint{1.972254in}{2.569602in}}%
\pgfpathlineto{\pgfqpoint{2.410172in}{2.537974in}}%
\pgfpathlineto{\pgfqpoint{2.674281in}{2.686821in}}%
\pgfpathlineto{\pgfqpoint{2.238213in}{2.742590in}}%
\pgfpathlineto{\pgfqpoint{1.972254in}{2.569602in}}%
\pgfpathclose%
\pgfusepath{fill}%
\end{pgfscope}%
\begin{pgfscope}%
\pgfpathrectangle{\pgfqpoint{0.017500in}{0.017500in}}{\pgfqpoint{6.218282in}{3.580372in}} %
\pgfusepath{clip}%
\pgfsetbuttcap%
\pgfsetroundjoin%
\definecolor{currentfill}{rgb}{0.243520,0.319189,0.771672}%
\pgfsetfillcolor{currentfill}%
\pgfsetlinewidth{0.000000pt}%
\definecolor{currentstroke}{rgb}{0.000000,0.000000,0.000000}%
\pgfsetstrokecolor{currentstroke}%
\pgfsetdash{}{0pt}%
\pgfpathmoveto{\pgfqpoint{1.915196in}{0.958373in}}%
\pgfpathlineto{\pgfqpoint{2.142608in}{2.314596in}}%
\pgfpathlineto{\pgfqpoint{1.915196in}{0.958373in}}%
\pgfpathclose%
\pgfusepath{fill}%
\end{pgfscope}%
\begin{pgfscope}%
\pgfpathrectangle{\pgfqpoint{0.017500in}{0.017500in}}{\pgfqpoint{6.218282in}{3.580372in}} %
\pgfusepath{clip}%
\pgfsetbuttcap%
\pgfsetroundjoin%
\definecolor{currentfill}{rgb}{0.243520,0.319189,0.771672}%
\pgfsetfillcolor{currentfill}%
\pgfsetlinewidth{0.000000pt}%
\definecolor{currentstroke}{rgb}{0.000000,0.000000,0.000000}%
\pgfsetstrokecolor{currentstroke}%
\pgfsetdash{}{0pt}%
\pgfpathmoveto{\pgfqpoint{1.915196in}{0.958373in}}%
\pgfpathlineto{\pgfqpoint{2.142608in}{2.314596in}}%
\pgfpathlineto{\pgfqpoint{1.915196in}{0.958373in}}%
\pgfpathclose%
\pgfusepath{fill}%
\end{pgfscope}%
\begin{pgfscope}%
\pgfpathrectangle{\pgfqpoint{0.017500in}{0.017500in}}{\pgfqpoint{6.218282in}{3.580372in}} %
\pgfusepath{clip}%
\pgfsetbuttcap%
\pgfsetroundjoin%
\definecolor{currentfill}{rgb}{0.243520,0.319189,0.771672}%
\pgfsetfillcolor{currentfill}%
\pgfsetlinewidth{0.000000pt}%
\definecolor{currentstroke}{rgb}{0.000000,0.000000,0.000000}%
\pgfsetstrokecolor{currentstroke}%
\pgfsetdash{}{0pt}%
\pgfpathmoveto{\pgfqpoint{1.915196in}{0.958373in}}%
\pgfpathlineto{\pgfqpoint{2.142608in}{2.314596in}}%
\pgfpathlineto{\pgfqpoint{1.915196in}{0.958373in}}%
\pgfpathclose%
\pgfusepath{fill}%
\end{pgfscope}%
\begin{pgfscope}%
\pgfpathrectangle{\pgfqpoint{0.017500in}{0.017500in}}{\pgfqpoint{6.218282in}{3.580372in}} %
\pgfusepath{clip}%
\pgfsetbuttcap%
\pgfsetroundjoin%
\definecolor{currentfill}{rgb}{0.243520,0.319189,0.771672}%
\pgfsetfillcolor{currentfill}%
\pgfsetlinewidth{0.000000pt}%
\definecolor{currentstroke}{rgb}{0.000000,0.000000,0.000000}%
\pgfsetstrokecolor{currentstroke}%
\pgfsetdash{}{0pt}%
\pgfpathmoveto{\pgfqpoint{1.915196in}{0.958373in}}%
\pgfpathlineto{\pgfqpoint{2.142608in}{2.314596in}}%
\pgfpathlineto{\pgfqpoint{1.915196in}{0.958373in}}%
\pgfpathclose%
\pgfusepath{fill}%
\end{pgfscope}%
\begin{pgfscope}%
\pgfpathrectangle{\pgfqpoint{0.017500in}{0.017500in}}{\pgfqpoint{6.218282in}{3.580372in}} %
\pgfusepath{clip}%
\pgfsetbuttcap%
\pgfsetroundjoin%
\definecolor{currentfill}{rgb}{0.243520,0.319189,0.771672}%
\pgfsetfillcolor{currentfill}%
\pgfsetlinewidth{0.000000pt}%
\definecolor{currentstroke}{rgb}{0.000000,0.000000,0.000000}%
\pgfsetstrokecolor{currentstroke}%
\pgfsetdash{}{0pt}%
\pgfpathmoveto{\pgfqpoint{1.915196in}{0.958373in}}%
\pgfpathlineto{\pgfqpoint{2.142608in}{2.314596in}}%
\pgfpathlineto{\pgfqpoint{1.915196in}{0.958373in}}%
\pgfpathclose%
\pgfusepath{fill}%
\end{pgfscope}%
\begin{pgfscope}%
\pgfpathrectangle{\pgfqpoint{0.017500in}{0.017500in}}{\pgfqpoint{6.218282in}{3.580372in}} %
\pgfusepath{clip}%
\pgfsetbuttcap%
\pgfsetroundjoin%
\definecolor{currentfill}{rgb}{0.243520,0.319189,0.771672}%
\pgfsetfillcolor{currentfill}%
\pgfsetlinewidth{0.000000pt}%
\definecolor{currentstroke}{rgb}{0.000000,0.000000,0.000000}%
\pgfsetstrokecolor{currentstroke}%
\pgfsetdash{}{0pt}%
\pgfpathmoveto{\pgfqpoint{1.915196in}{0.958373in}}%
\pgfpathlineto{\pgfqpoint{2.142608in}{2.314596in}}%
\pgfpathlineto{\pgfqpoint{1.915196in}{0.958373in}}%
\pgfpathclose%
\pgfusepath{fill}%
\end{pgfscope}%
\begin{pgfscope}%
\pgfpathrectangle{\pgfqpoint{0.017500in}{0.017500in}}{\pgfqpoint{6.218282in}{3.580372in}} %
\pgfusepath{clip}%
\pgfsetbuttcap%
\pgfsetroundjoin%
\definecolor{currentfill}{rgb}{0.243520,0.319189,0.771672}%
\pgfsetfillcolor{currentfill}%
\pgfsetlinewidth{0.000000pt}%
\definecolor{currentstroke}{rgb}{0.000000,0.000000,0.000000}%
\pgfsetstrokecolor{currentstroke}%
\pgfsetdash{}{0pt}%
\pgfpathmoveto{\pgfqpoint{1.915196in}{0.958373in}}%
\pgfpathlineto{\pgfqpoint{2.142608in}{2.314596in}}%
\pgfpathlineto{\pgfqpoint{1.915196in}{0.958373in}}%
\pgfpathclose%
\pgfusepath{fill}%
\end{pgfscope}%
\begin{pgfscope}%
\pgfpathrectangle{\pgfqpoint{0.017500in}{0.017500in}}{\pgfqpoint{6.218282in}{3.580372in}} %
\pgfusepath{clip}%
\pgfsetbuttcap%
\pgfsetroundjoin%
\definecolor{currentfill}{rgb}{0.243520,0.319189,0.771672}%
\pgfsetfillcolor{currentfill}%
\pgfsetlinewidth{0.000000pt}%
\definecolor{currentstroke}{rgb}{0.000000,0.000000,0.000000}%
\pgfsetstrokecolor{currentstroke}%
\pgfsetdash{}{0pt}%
\pgfpathmoveto{\pgfqpoint{1.915196in}{0.958373in}}%
\pgfpathlineto{\pgfqpoint{2.142608in}{2.314596in}}%
\pgfpathlineto{\pgfqpoint{1.915196in}{0.958373in}}%
\pgfpathclose%
\pgfusepath{fill}%
\end{pgfscope}%
\begin{pgfscope}%
\pgfpathrectangle{\pgfqpoint{0.017500in}{0.017500in}}{\pgfqpoint{6.218282in}{3.580372in}} %
\pgfusepath{clip}%
\pgfsetbuttcap%
\pgfsetroundjoin%
\definecolor{currentfill}{rgb}{0.243520,0.319189,0.771672}%
\pgfsetfillcolor{currentfill}%
\pgfsetlinewidth{0.000000pt}%
\definecolor{currentstroke}{rgb}{0.000000,0.000000,0.000000}%
\pgfsetstrokecolor{currentstroke}%
\pgfsetdash{}{0pt}%
\pgfpathmoveto{\pgfqpoint{1.915196in}{0.958373in}}%
\pgfpathlineto{\pgfqpoint{2.142608in}{2.314596in}}%
\pgfpathlineto{\pgfqpoint{1.915196in}{0.958373in}}%
\pgfpathclose%
\pgfusepath{fill}%
\end{pgfscope}%
\begin{pgfscope}%
\pgfpathrectangle{\pgfqpoint{0.017500in}{0.017500in}}{\pgfqpoint{6.218282in}{3.580372in}} %
\pgfusepath{clip}%
\pgfsetbuttcap%
\pgfsetroundjoin%
\definecolor{currentfill}{rgb}{0.243520,0.319189,0.771672}%
\pgfsetfillcolor{currentfill}%
\pgfsetlinewidth{0.000000pt}%
\definecolor{currentstroke}{rgb}{0.000000,0.000000,0.000000}%
\pgfsetstrokecolor{currentstroke}%
\pgfsetdash{}{0pt}%
\pgfpathmoveto{\pgfqpoint{1.915196in}{0.958373in}}%
\pgfpathlineto{\pgfqpoint{2.142608in}{2.314596in}}%
\pgfpathlineto{\pgfqpoint{1.915196in}{0.958373in}}%
\pgfpathclose%
\pgfusepath{fill}%
\end{pgfscope}%
\begin{pgfscope}%
\pgfpathrectangle{\pgfqpoint{0.017500in}{0.017500in}}{\pgfqpoint{6.218282in}{3.580372in}} %
\pgfusepath{clip}%
\pgfsetbuttcap%
\pgfsetroundjoin%
\definecolor{currentfill}{rgb}{0.243520,0.319189,0.771672}%
\pgfsetfillcolor{currentfill}%
\pgfsetlinewidth{0.000000pt}%
\definecolor{currentstroke}{rgb}{0.000000,0.000000,0.000000}%
\pgfsetstrokecolor{currentstroke}%
\pgfsetdash{}{0pt}%
\pgfpathmoveto{\pgfqpoint{1.915196in}{0.958373in}}%
\pgfpathlineto{\pgfqpoint{2.142608in}{2.314596in}}%
\pgfpathlineto{\pgfqpoint{1.915196in}{0.958373in}}%
\pgfpathclose%
\pgfusepath{fill}%
\end{pgfscope}%
\begin{pgfscope}%
\pgfpathrectangle{\pgfqpoint{0.017500in}{0.017500in}}{\pgfqpoint{6.218282in}{3.580372in}} %
\pgfusepath{clip}%
\pgfsetbuttcap%
\pgfsetroundjoin%
\definecolor{currentfill}{rgb}{0.243520,0.319189,0.771672}%
\pgfsetfillcolor{currentfill}%
\pgfsetlinewidth{0.000000pt}%
\definecolor{currentstroke}{rgb}{0.000000,0.000000,0.000000}%
\pgfsetstrokecolor{currentstroke}%
\pgfsetdash{}{0pt}%
\pgfpathmoveto{\pgfqpoint{1.915196in}{0.958373in}}%
\pgfpathlineto{\pgfqpoint{2.142608in}{2.314596in}}%
\pgfpathlineto{\pgfqpoint{1.915196in}{0.958373in}}%
\pgfpathclose%
\pgfusepath{fill}%
\end{pgfscope}%
\begin{pgfscope}%
\pgfpathrectangle{\pgfqpoint{0.017500in}{0.017500in}}{\pgfqpoint{6.218282in}{3.580372in}} %
\pgfusepath{clip}%
\pgfsetbuttcap%
\pgfsetroundjoin%
\definecolor{currentfill}{rgb}{0.243520,0.319189,0.771672}%
\pgfsetfillcolor{currentfill}%
\pgfsetlinewidth{0.000000pt}%
\definecolor{currentstroke}{rgb}{0.000000,0.000000,0.000000}%
\pgfsetstrokecolor{currentstroke}%
\pgfsetdash{}{0pt}%
\pgfpathmoveto{\pgfqpoint{1.915196in}{0.958373in}}%
\pgfpathlineto{\pgfqpoint{2.142608in}{2.314596in}}%
\pgfpathlineto{\pgfqpoint{1.915196in}{0.958373in}}%
\pgfpathclose%
\pgfusepath{fill}%
\end{pgfscope}%
\begin{pgfscope}%
\pgfpathrectangle{\pgfqpoint{0.017500in}{0.017500in}}{\pgfqpoint{6.218282in}{3.580372in}} %
\pgfusepath{clip}%
\pgfsetbuttcap%
\pgfsetroundjoin%
\definecolor{currentfill}{rgb}{0.243520,0.319189,0.771672}%
\pgfsetfillcolor{currentfill}%
\pgfsetlinewidth{0.000000pt}%
\definecolor{currentstroke}{rgb}{0.000000,0.000000,0.000000}%
\pgfsetstrokecolor{currentstroke}%
\pgfsetdash{}{0pt}%
\pgfpathmoveto{\pgfqpoint{1.915196in}{0.958373in}}%
\pgfpathlineto{\pgfqpoint{2.142608in}{2.314596in}}%
\pgfpathlineto{\pgfqpoint{1.915196in}{0.958373in}}%
\pgfpathclose%
\pgfusepath{fill}%
\end{pgfscope}%
\begin{pgfscope}%
\pgfpathrectangle{\pgfqpoint{0.017500in}{0.017500in}}{\pgfqpoint{6.218282in}{3.580372in}} %
\pgfusepath{clip}%
\pgfsetbuttcap%
\pgfsetroundjoin%
\definecolor{currentfill}{rgb}{0.243520,0.319189,0.771672}%
\pgfsetfillcolor{currentfill}%
\pgfsetlinewidth{0.000000pt}%
\definecolor{currentstroke}{rgb}{0.000000,0.000000,0.000000}%
\pgfsetstrokecolor{currentstroke}%
\pgfsetdash{}{0pt}%
\pgfpathmoveto{\pgfqpoint{1.915196in}{0.958373in}}%
\pgfpathlineto{\pgfqpoint{2.142608in}{2.314596in}}%
\pgfpathlineto{\pgfqpoint{1.915196in}{0.958373in}}%
\pgfpathclose%
\pgfusepath{fill}%
\end{pgfscope}%
\begin{pgfscope}%
\pgfpathrectangle{\pgfqpoint{0.017500in}{0.017500in}}{\pgfqpoint{6.218282in}{3.580372in}} %
\pgfusepath{clip}%
\pgfsetbuttcap%
\pgfsetroundjoin%
\definecolor{currentfill}{rgb}{0.243520,0.319189,0.771672}%
\pgfsetfillcolor{currentfill}%
\pgfsetlinewidth{0.000000pt}%
\definecolor{currentstroke}{rgb}{0.000000,0.000000,0.000000}%
\pgfsetstrokecolor{currentstroke}%
\pgfsetdash{}{0pt}%
\pgfpathmoveto{\pgfqpoint{1.915196in}{0.958373in}}%
\pgfpathlineto{\pgfqpoint{2.142608in}{2.314596in}}%
\pgfpathlineto{\pgfqpoint{1.915196in}{0.958373in}}%
\pgfpathclose%
\pgfusepath{fill}%
\end{pgfscope}%
\begin{pgfscope}%
\pgfpathrectangle{\pgfqpoint{0.017500in}{0.017500in}}{\pgfqpoint{6.218282in}{3.580372in}} %
\pgfusepath{clip}%
\pgfsetbuttcap%
\pgfsetroundjoin%
\definecolor{currentfill}{rgb}{0.243520,0.319189,0.771672}%
\pgfsetfillcolor{currentfill}%
\pgfsetlinewidth{0.000000pt}%
\definecolor{currentstroke}{rgb}{0.000000,0.000000,0.000000}%
\pgfsetstrokecolor{currentstroke}%
\pgfsetdash{}{0pt}%
\pgfpathmoveto{\pgfqpoint{1.915196in}{0.958373in}}%
\pgfpathlineto{\pgfqpoint{2.142608in}{2.314596in}}%
\pgfpathlineto{\pgfqpoint{1.915196in}{0.958373in}}%
\pgfpathclose%
\pgfusepath{fill}%
\end{pgfscope}%
\begin{pgfscope}%
\pgfpathrectangle{\pgfqpoint{0.017500in}{0.017500in}}{\pgfqpoint{6.218282in}{3.580372in}} %
\pgfusepath{clip}%
\pgfsetbuttcap%
\pgfsetroundjoin%
\definecolor{currentfill}{rgb}{0.243520,0.319189,0.771672}%
\pgfsetfillcolor{currentfill}%
\pgfsetlinewidth{0.000000pt}%
\definecolor{currentstroke}{rgb}{0.000000,0.000000,0.000000}%
\pgfsetstrokecolor{currentstroke}%
\pgfsetdash{}{0pt}%
\pgfpathmoveto{\pgfqpoint{1.915196in}{0.958373in}}%
\pgfpathlineto{\pgfqpoint{2.142608in}{2.314596in}}%
\pgfpathlineto{\pgfqpoint{1.915196in}{0.958373in}}%
\pgfpathclose%
\pgfusepath{fill}%
\end{pgfscope}%
\begin{pgfscope}%
\pgfpathrectangle{\pgfqpoint{0.017500in}{0.017500in}}{\pgfqpoint{6.218282in}{3.580372in}} %
\pgfusepath{clip}%
\pgfsetbuttcap%
\pgfsetroundjoin%
\definecolor{currentfill}{rgb}{0.243520,0.319189,0.771672}%
\pgfsetfillcolor{currentfill}%
\pgfsetlinewidth{0.000000pt}%
\definecolor{currentstroke}{rgb}{0.000000,0.000000,0.000000}%
\pgfsetstrokecolor{currentstroke}%
\pgfsetdash{}{0pt}%
\pgfpathmoveto{\pgfqpoint{1.915196in}{0.958373in}}%
\pgfpathlineto{\pgfqpoint{2.142608in}{2.314596in}}%
\pgfpathlineto{\pgfqpoint{1.915196in}{0.958373in}}%
\pgfpathclose%
\pgfusepath{fill}%
\end{pgfscope}%
\begin{pgfscope}%
\pgfpathrectangle{\pgfqpoint{0.017500in}{0.017500in}}{\pgfqpoint{6.218282in}{3.580372in}} %
\pgfusepath{clip}%
\pgfsetbuttcap%
\pgfsetroundjoin%
\definecolor{currentfill}{rgb}{0.243520,0.319189,0.771672}%
\pgfsetfillcolor{currentfill}%
\pgfsetlinewidth{0.000000pt}%
\definecolor{currentstroke}{rgb}{0.000000,0.000000,0.000000}%
\pgfsetstrokecolor{currentstroke}%
\pgfsetdash{}{0pt}%
\pgfpathmoveto{\pgfqpoint{1.915196in}{0.958373in}}%
\pgfpathlineto{\pgfqpoint{2.142608in}{2.314596in}}%
\pgfpathlineto{\pgfqpoint{1.915196in}{0.958373in}}%
\pgfpathclose%
\pgfusepath{fill}%
\end{pgfscope}%
\begin{pgfscope}%
\pgfpathrectangle{\pgfqpoint{0.017500in}{0.017500in}}{\pgfqpoint{6.218282in}{3.580372in}} %
\pgfusepath{clip}%
\pgfsetbuttcap%
\pgfsetroundjoin%
\definecolor{currentfill}{rgb}{0.243520,0.319189,0.771672}%
\pgfsetfillcolor{currentfill}%
\pgfsetlinewidth{0.000000pt}%
\definecolor{currentstroke}{rgb}{0.000000,0.000000,0.000000}%
\pgfsetstrokecolor{currentstroke}%
\pgfsetdash{}{0pt}%
\pgfpathmoveto{\pgfqpoint{1.915196in}{0.958373in}}%
\pgfpathlineto{\pgfqpoint{2.142608in}{2.314596in}}%
\pgfpathlineto{\pgfqpoint{1.915196in}{0.958373in}}%
\pgfpathclose%
\pgfusepath{fill}%
\end{pgfscope}%
\begin{pgfscope}%
\pgfpathrectangle{\pgfqpoint{0.017500in}{0.017500in}}{\pgfqpoint{6.218282in}{3.580372in}} %
\pgfusepath{clip}%
\pgfsetbuttcap%
\pgfsetroundjoin%
\definecolor{currentfill}{rgb}{0.243520,0.319189,0.771672}%
\pgfsetfillcolor{currentfill}%
\pgfsetlinewidth{0.000000pt}%
\definecolor{currentstroke}{rgb}{0.000000,0.000000,0.000000}%
\pgfsetstrokecolor{currentstroke}%
\pgfsetdash{}{0pt}%
\pgfpathmoveto{\pgfqpoint{1.915196in}{0.958373in}}%
\pgfpathlineto{\pgfqpoint{2.142608in}{2.314596in}}%
\pgfpathlineto{\pgfqpoint{1.915196in}{0.958373in}}%
\pgfpathclose%
\pgfusepath{fill}%
\end{pgfscope}%
\begin{pgfscope}%
\pgfpathrectangle{\pgfqpoint{0.017500in}{0.017500in}}{\pgfqpoint{6.218282in}{3.580372in}} %
\pgfusepath{clip}%
\pgfsetbuttcap%
\pgfsetroundjoin%
\definecolor{currentfill}{rgb}{0.243520,0.319189,0.771672}%
\pgfsetfillcolor{currentfill}%
\pgfsetlinewidth{0.000000pt}%
\definecolor{currentstroke}{rgb}{0.000000,0.000000,0.000000}%
\pgfsetstrokecolor{currentstroke}%
\pgfsetdash{}{0pt}%
\pgfpathmoveto{\pgfqpoint{1.915196in}{0.958373in}}%
\pgfpathlineto{\pgfqpoint{2.142608in}{2.314596in}}%
\pgfpathlineto{\pgfqpoint{1.915196in}{0.958373in}}%
\pgfpathclose%
\pgfusepath{fill}%
\end{pgfscope}%
\begin{pgfscope}%
\pgfpathrectangle{\pgfqpoint{0.017500in}{0.017500in}}{\pgfqpoint{6.218282in}{3.580372in}} %
\pgfusepath{clip}%
\pgfsetbuttcap%
\pgfsetroundjoin%
\definecolor{currentfill}{rgb}{0.243520,0.319189,0.771672}%
\pgfsetfillcolor{currentfill}%
\pgfsetlinewidth{0.000000pt}%
\definecolor{currentstroke}{rgb}{0.000000,0.000000,0.000000}%
\pgfsetstrokecolor{currentstroke}%
\pgfsetdash{}{0pt}%
\pgfpathmoveto{\pgfqpoint{1.915196in}{0.958373in}}%
\pgfpathlineto{\pgfqpoint{2.142608in}{2.314596in}}%
\pgfpathlineto{\pgfqpoint{1.915196in}{0.958373in}}%
\pgfpathclose%
\pgfusepath{fill}%
\end{pgfscope}%
\begin{pgfscope}%
\pgfpathrectangle{\pgfqpoint{0.017500in}{0.017500in}}{\pgfqpoint{6.218282in}{3.580372in}} %
\pgfusepath{clip}%
\pgfsetbuttcap%
\pgfsetroundjoin%
\definecolor{currentfill}{rgb}{0.243520,0.319189,0.771672}%
\pgfsetfillcolor{currentfill}%
\pgfsetlinewidth{0.000000pt}%
\definecolor{currentstroke}{rgb}{0.000000,0.000000,0.000000}%
\pgfsetstrokecolor{currentstroke}%
\pgfsetdash{}{0pt}%
\pgfpathmoveto{\pgfqpoint{1.915196in}{0.958373in}}%
\pgfpathlineto{\pgfqpoint{2.142608in}{2.314596in}}%
\pgfpathlineto{\pgfqpoint{1.915196in}{0.958373in}}%
\pgfpathclose%
\pgfusepath{fill}%
\end{pgfscope}%
\begin{pgfscope}%
\pgfpathrectangle{\pgfqpoint{0.017500in}{0.017500in}}{\pgfqpoint{6.218282in}{3.580372in}} %
\pgfusepath{clip}%
\pgfsetbuttcap%
\pgfsetroundjoin%
\definecolor{currentfill}{rgb}{0.243520,0.319189,0.771672}%
\pgfsetfillcolor{currentfill}%
\pgfsetlinewidth{0.000000pt}%
\definecolor{currentstroke}{rgb}{0.000000,0.000000,0.000000}%
\pgfsetstrokecolor{currentstroke}%
\pgfsetdash{}{0pt}%
\pgfpathmoveto{\pgfqpoint{1.915196in}{0.958373in}}%
\pgfpathlineto{\pgfqpoint{2.142608in}{2.314596in}}%
\pgfpathlineto{\pgfqpoint{1.915196in}{0.958373in}}%
\pgfpathclose%
\pgfusepath{fill}%
\end{pgfscope}%
\begin{pgfscope}%
\pgfpathrectangle{\pgfqpoint{0.017500in}{0.017500in}}{\pgfqpoint{6.218282in}{3.580372in}} %
\pgfusepath{clip}%
\pgfsetbuttcap%
\pgfsetroundjoin%
\definecolor{currentfill}{rgb}{0.243520,0.319189,0.771672}%
\pgfsetfillcolor{currentfill}%
\pgfsetlinewidth{0.000000pt}%
\definecolor{currentstroke}{rgb}{0.000000,0.000000,0.000000}%
\pgfsetstrokecolor{currentstroke}%
\pgfsetdash{}{0pt}%
\pgfpathmoveto{\pgfqpoint{1.915196in}{0.958373in}}%
\pgfpathlineto{\pgfqpoint{2.142608in}{2.314596in}}%
\pgfpathlineto{\pgfqpoint{1.915196in}{0.958373in}}%
\pgfpathclose%
\pgfusepath{fill}%
\end{pgfscope}%
\begin{pgfscope}%
\pgfpathrectangle{\pgfqpoint{0.017500in}{0.017500in}}{\pgfqpoint{6.218282in}{3.580372in}} %
\pgfusepath{clip}%
\pgfsetbuttcap%
\pgfsetroundjoin%
\definecolor{currentfill}{rgb}{0.243520,0.319189,0.771672}%
\pgfsetfillcolor{currentfill}%
\pgfsetlinewidth{0.000000pt}%
\definecolor{currentstroke}{rgb}{0.000000,0.000000,0.000000}%
\pgfsetstrokecolor{currentstroke}%
\pgfsetdash{}{0pt}%
\pgfpathmoveto{\pgfqpoint{1.915196in}{0.958373in}}%
\pgfpathlineto{\pgfqpoint{2.142608in}{2.314596in}}%
\pgfpathlineto{\pgfqpoint{1.915196in}{0.958373in}}%
\pgfpathclose%
\pgfusepath{fill}%
\end{pgfscope}%
\begin{pgfscope}%
\pgfpathrectangle{\pgfqpoint{0.017500in}{0.017500in}}{\pgfqpoint{6.218282in}{3.580372in}} %
\pgfusepath{clip}%
\pgfsetbuttcap%
\pgfsetroundjoin%
\definecolor{currentfill}{rgb}{0.243520,0.319189,0.771672}%
\pgfsetfillcolor{currentfill}%
\pgfsetlinewidth{0.000000pt}%
\definecolor{currentstroke}{rgb}{0.000000,0.000000,0.000000}%
\pgfsetstrokecolor{currentstroke}%
\pgfsetdash{}{0pt}%
\pgfpathmoveto{\pgfqpoint{1.915196in}{0.958373in}}%
\pgfpathlineto{\pgfqpoint{2.142608in}{2.314596in}}%
\pgfpathlineto{\pgfqpoint{1.915196in}{0.958373in}}%
\pgfpathclose%
\pgfusepath{fill}%
\end{pgfscope}%
\begin{pgfscope}%
\pgfpathrectangle{\pgfqpoint{0.017500in}{0.017500in}}{\pgfqpoint{6.218282in}{3.580372in}} %
\pgfusepath{clip}%
\pgfsetbuttcap%
\pgfsetroundjoin%
\definecolor{currentfill}{rgb}{0.243520,0.319189,0.771672}%
\pgfsetfillcolor{currentfill}%
\pgfsetlinewidth{0.000000pt}%
\definecolor{currentstroke}{rgb}{0.000000,0.000000,0.000000}%
\pgfsetstrokecolor{currentstroke}%
\pgfsetdash{}{0pt}%
\pgfpathmoveto{\pgfqpoint{1.915196in}{0.958373in}}%
\pgfpathlineto{\pgfqpoint{2.142608in}{2.314596in}}%
\pgfpathlineto{\pgfqpoint{1.915196in}{0.958373in}}%
\pgfpathclose%
\pgfusepath{fill}%
\end{pgfscope}%
\begin{pgfscope}%
\pgfpathrectangle{\pgfqpoint{0.017500in}{0.017500in}}{\pgfqpoint{6.218282in}{3.580372in}} %
\pgfusepath{clip}%
\pgfsetbuttcap%
\pgfsetroundjoin%
\definecolor{currentfill}{rgb}{0.735077,0.104460,0.171492}%
\pgfsetfillcolor{currentfill}%
\pgfsetlinewidth{0.000000pt}%
\definecolor{currentstroke}{rgb}{0.000000,0.000000,0.000000}%
\pgfsetstrokecolor{currentstroke}%
\pgfsetdash{}{0pt}%
\pgfpathmoveto{\pgfqpoint{3.116444in}{2.629069in}}%
\pgfpathlineto{\pgfqpoint{3.372089in}{2.760957in}}%
\pgfpathlineto{\pgfqpoint{3.116444in}{2.629069in}}%
\pgfpathclose%
\pgfusepath{fill}%
\end{pgfscope}%
\begin{pgfscope}%
\pgfpathrectangle{\pgfqpoint{0.017500in}{0.017500in}}{\pgfqpoint{6.218282in}{3.580372in}} %
\pgfusepath{clip}%
\pgfsetbuttcap%
\pgfsetroundjoin%
\definecolor{currentfill}{rgb}{0.735077,0.104460,0.171492}%
\pgfsetfillcolor{currentfill}%
\pgfsetlinewidth{0.000000pt}%
\definecolor{currentstroke}{rgb}{0.000000,0.000000,0.000000}%
\pgfsetstrokecolor{currentstroke}%
\pgfsetdash{}{0pt}%
\pgfpathmoveto{\pgfqpoint{3.116444in}{2.629069in}}%
\pgfpathlineto{\pgfqpoint{3.372089in}{2.760957in}}%
\pgfpathlineto{\pgfqpoint{3.116444in}{2.629069in}}%
\pgfpathclose%
\pgfusepath{fill}%
\end{pgfscope}%
\begin{pgfscope}%
\pgfpathrectangle{\pgfqpoint{0.017500in}{0.017500in}}{\pgfqpoint{6.218282in}{3.580372in}} %
\pgfusepath{clip}%
\pgfsetbuttcap%
\pgfsetroundjoin%
\definecolor{currentfill}{rgb}{0.735077,0.104460,0.171492}%
\pgfsetfillcolor{currentfill}%
\pgfsetlinewidth{0.000000pt}%
\definecolor{currentstroke}{rgb}{0.000000,0.000000,0.000000}%
\pgfsetstrokecolor{currentstroke}%
\pgfsetdash{}{0pt}%
\pgfpathmoveto{\pgfqpoint{3.116444in}{2.629069in}}%
\pgfpathlineto{\pgfqpoint{3.372089in}{2.760957in}}%
\pgfpathlineto{\pgfqpoint{3.116444in}{2.629069in}}%
\pgfpathclose%
\pgfusepath{fill}%
\end{pgfscope}%
\begin{pgfscope}%
\pgfpathrectangle{\pgfqpoint{0.017500in}{0.017500in}}{\pgfqpoint{6.218282in}{3.580372in}} %
\pgfusepath{clip}%
\pgfsetbuttcap%
\pgfsetroundjoin%
\definecolor{currentfill}{rgb}{0.735077,0.104460,0.171492}%
\pgfsetfillcolor{currentfill}%
\pgfsetlinewidth{0.000000pt}%
\definecolor{currentstroke}{rgb}{0.000000,0.000000,0.000000}%
\pgfsetstrokecolor{currentstroke}%
\pgfsetdash{}{0pt}%
\pgfpathmoveto{\pgfqpoint{3.116444in}{2.629069in}}%
\pgfpathlineto{\pgfqpoint{3.372089in}{2.760957in}}%
\pgfpathlineto{\pgfqpoint{3.116444in}{2.629069in}}%
\pgfpathclose%
\pgfusepath{fill}%
\end{pgfscope}%
\begin{pgfscope}%
\pgfpathrectangle{\pgfqpoint{0.017500in}{0.017500in}}{\pgfqpoint{6.218282in}{3.580372in}} %
\pgfusepath{clip}%
\pgfsetbuttcap%
\pgfsetroundjoin%
\definecolor{currentfill}{rgb}{0.735077,0.104460,0.171492}%
\pgfsetfillcolor{currentfill}%
\pgfsetlinewidth{0.000000pt}%
\definecolor{currentstroke}{rgb}{0.000000,0.000000,0.000000}%
\pgfsetstrokecolor{currentstroke}%
\pgfsetdash{}{0pt}%
\pgfpathmoveto{\pgfqpoint{3.116444in}{2.629069in}}%
\pgfpathlineto{\pgfqpoint{3.372089in}{2.760957in}}%
\pgfpathlineto{\pgfqpoint{3.116444in}{2.629069in}}%
\pgfpathclose%
\pgfusepath{fill}%
\end{pgfscope}%
\begin{pgfscope}%
\pgfpathrectangle{\pgfqpoint{0.017500in}{0.017500in}}{\pgfqpoint{6.218282in}{3.580372in}} %
\pgfusepath{clip}%
\pgfsetbuttcap%
\pgfsetroundjoin%
\definecolor{currentfill}{rgb}{0.735077,0.104460,0.171492}%
\pgfsetfillcolor{currentfill}%
\pgfsetlinewidth{0.000000pt}%
\definecolor{currentstroke}{rgb}{0.000000,0.000000,0.000000}%
\pgfsetstrokecolor{currentstroke}%
\pgfsetdash{}{0pt}%
\pgfpathmoveto{\pgfqpoint{3.116444in}{2.629069in}}%
\pgfpathlineto{\pgfqpoint{3.372089in}{2.760957in}}%
\pgfpathlineto{\pgfqpoint{3.116444in}{2.629069in}}%
\pgfpathclose%
\pgfusepath{fill}%
\end{pgfscope}%
\begin{pgfscope}%
\pgfpathrectangle{\pgfqpoint{0.017500in}{0.017500in}}{\pgfqpoint{6.218282in}{3.580372in}} %
\pgfusepath{clip}%
\pgfsetbuttcap%
\pgfsetroundjoin%
\definecolor{currentfill}{rgb}{0.735077,0.104460,0.171492}%
\pgfsetfillcolor{currentfill}%
\pgfsetlinewidth{0.000000pt}%
\definecolor{currentstroke}{rgb}{0.000000,0.000000,0.000000}%
\pgfsetstrokecolor{currentstroke}%
\pgfsetdash{}{0pt}%
\pgfpathmoveto{\pgfqpoint{3.116444in}{2.629069in}}%
\pgfpathlineto{\pgfqpoint{3.372089in}{2.760957in}}%
\pgfpathlineto{\pgfqpoint{3.116444in}{2.629069in}}%
\pgfpathclose%
\pgfusepath{fill}%
\end{pgfscope}%
\begin{pgfscope}%
\pgfpathrectangle{\pgfqpoint{0.017500in}{0.017500in}}{\pgfqpoint{6.218282in}{3.580372in}} %
\pgfusepath{clip}%
\pgfsetbuttcap%
\pgfsetroundjoin%
\definecolor{currentfill}{rgb}{0.735077,0.104460,0.171492}%
\pgfsetfillcolor{currentfill}%
\pgfsetlinewidth{0.000000pt}%
\definecolor{currentstroke}{rgb}{0.000000,0.000000,0.000000}%
\pgfsetstrokecolor{currentstroke}%
\pgfsetdash{}{0pt}%
\pgfpathmoveto{\pgfqpoint{3.116444in}{2.629069in}}%
\pgfpathlineto{\pgfqpoint{3.372089in}{2.760957in}}%
\pgfpathlineto{\pgfqpoint{3.116444in}{2.629069in}}%
\pgfpathclose%
\pgfusepath{fill}%
\end{pgfscope}%
\begin{pgfscope}%
\pgfpathrectangle{\pgfqpoint{0.017500in}{0.017500in}}{\pgfqpoint{6.218282in}{3.580372in}} %
\pgfusepath{clip}%
\pgfsetbuttcap%
\pgfsetroundjoin%
\definecolor{currentfill}{rgb}{0.735077,0.104460,0.171492}%
\pgfsetfillcolor{currentfill}%
\pgfsetlinewidth{0.000000pt}%
\definecolor{currentstroke}{rgb}{0.000000,0.000000,0.000000}%
\pgfsetstrokecolor{currentstroke}%
\pgfsetdash{}{0pt}%
\pgfpathmoveto{\pgfqpoint{3.116444in}{2.629069in}}%
\pgfpathlineto{\pgfqpoint{3.372089in}{2.760957in}}%
\pgfpathlineto{\pgfqpoint{3.116444in}{2.629069in}}%
\pgfpathclose%
\pgfusepath{fill}%
\end{pgfscope}%
\begin{pgfscope}%
\pgfpathrectangle{\pgfqpoint{0.017500in}{0.017500in}}{\pgfqpoint{6.218282in}{3.580372in}} %
\pgfusepath{clip}%
\pgfsetbuttcap%
\pgfsetroundjoin%
\definecolor{currentfill}{rgb}{0.735077,0.104460,0.171492}%
\pgfsetfillcolor{currentfill}%
\pgfsetlinewidth{0.000000pt}%
\definecolor{currentstroke}{rgb}{0.000000,0.000000,0.000000}%
\pgfsetstrokecolor{currentstroke}%
\pgfsetdash{}{0pt}%
\pgfpathmoveto{\pgfqpoint{3.116444in}{2.629069in}}%
\pgfpathlineto{\pgfqpoint{3.372089in}{2.760957in}}%
\pgfpathlineto{\pgfqpoint{3.116444in}{2.629069in}}%
\pgfpathclose%
\pgfusepath{fill}%
\end{pgfscope}%
\begin{pgfscope}%
\pgfpathrectangle{\pgfqpoint{0.017500in}{0.017500in}}{\pgfqpoint{6.218282in}{3.580372in}} %
\pgfusepath{clip}%
\pgfsetbuttcap%
\pgfsetroundjoin%
\definecolor{currentfill}{rgb}{0.735077,0.104460,0.171492}%
\pgfsetfillcolor{currentfill}%
\pgfsetlinewidth{0.000000pt}%
\definecolor{currentstroke}{rgb}{0.000000,0.000000,0.000000}%
\pgfsetstrokecolor{currentstroke}%
\pgfsetdash{}{0pt}%
\pgfpathmoveto{\pgfqpoint{3.116444in}{2.629069in}}%
\pgfpathlineto{\pgfqpoint{3.372089in}{2.760957in}}%
\pgfpathlineto{\pgfqpoint{3.116444in}{2.629069in}}%
\pgfpathclose%
\pgfusepath{fill}%
\end{pgfscope}%
\begin{pgfscope}%
\pgfpathrectangle{\pgfqpoint{0.017500in}{0.017500in}}{\pgfqpoint{6.218282in}{3.580372in}} %
\pgfusepath{clip}%
\pgfsetbuttcap%
\pgfsetroundjoin%
\definecolor{currentfill}{rgb}{0.735077,0.104460,0.171492}%
\pgfsetfillcolor{currentfill}%
\pgfsetlinewidth{0.000000pt}%
\definecolor{currentstroke}{rgb}{0.000000,0.000000,0.000000}%
\pgfsetstrokecolor{currentstroke}%
\pgfsetdash{}{0pt}%
\pgfpathmoveto{\pgfqpoint{3.116444in}{2.629069in}}%
\pgfpathlineto{\pgfqpoint{3.372089in}{2.760957in}}%
\pgfpathlineto{\pgfqpoint{3.116444in}{2.629069in}}%
\pgfpathclose%
\pgfusepath{fill}%
\end{pgfscope}%
\begin{pgfscope}%
\pgfpathrectangle{\pgfqpoint{0.017500in}{0.017500in}}{\pgfqpoint{6.218282in}{3.580372in}} %
\pgfusepath{clip}%
\pgfsetbuttcap%
\pgfsetroundjoin%
\definecolor{currentfill}{rgb}{0.735077,0.104460,0.171492}%
\pgfsetfillcolor{currentfill}%
\pgfsetlinewidth{0.000000pt}%
\definecolor{currentstroke}{rgb}{0.000000,0.000000,0.000000}%
\pgfsetstrokecolor{currentstroke}%
\pgfsetdash{}{0pt}%
\pgfpathmoveto{\pgfqpoint{3.116444in}{2.629069in}}%
\pgfpathlineto{\pgfqpoint{3.372089in}{2.760957in}}%
\pgfpathlineto{\pgfqpoint{3.116444in}{2.629069in}}%
\pgfpathclose%
\pgfusepath{fill}%
\end{pgfscope}%
\begin{pgfscope}%
\pgfpathrectangle{\pgfqpoint{0.017500in}{0.017500in}}{\pgfqpoint{6.218282in}{3.580372in}} %
\pgfusepath{clip}%
\pgfsetbuttcap%
\pgfsetroundjoin%
\definecolor{currentfill}{rgb}{0.735077,0.104460,0.171492}%
\pgfsetfillcolor{currentfill}%
\pgfsetlinewidth{0.000000pt}%
\definecolor{currentstroke}{rgb}{0.000000,0.000000,0.000000}%
\pgfsetstrokecolor{currentstroke}%
\pgfsetdash{}{0pt}%
\pgfpathmoveto{\pgfqpoint{3.116444in}{2.629069in}}%
\pgfpathlineto{\pgfqpoint{3.372089in}{2.760957in}}%
\pgfpathlineto{\pgfqpoint{3.116444in}{2.629069in}}%
\pgfpathclose%
\pgfusepath{fill}%
\end{pgfscope}%
\begin{pgfscope}%
\pgfpathrectangle{\pgfqpoint{0.017500in}{0.017500in}}{\pgfqpoint{6.218282in}{3.580372in}} %
\pgfusepath{clip}%
\pgfsetbuttcap%
\pgfsetroundjoin%
\definecolor{currentfill}{rgb}{0.735077,0.104460,0.171492}%
\pgfsetfillcolor{currentfill}%
\pgfsetlinewidth{0.000000pt}%
\definecolor{currentstroke}{rgb}{0.000000,0.000000,0.000000}%
\pgfsetstrokecolor{currentstroke}%
\pgfsetdash{}{0pt}%
\pgfpathmoveto{\pgfqpoint{3.116444in}{2.629069in}}%
\pgfpathlineto{\pgfqpoint{3.372089in}{2.760957in}}%
\pgfpathlineto{\pgfqpoint{3.116444in}{2.629069in}}%
\pgfpathclose%
\pgfusepath{fill}%
\end{pgfscope}%
\begin{pgfscope}%
\pgfpathrectangle{\pgfqpoint{0.017500in}{0.017500in}}{\pgfqpoint{6.218282in}{3.580372in}} %
\pgfusepath{clip}%
\pgfsetbuttcap%
\pgfsetroundjoin%
\definecolor{currentfill}{rgb}{0.735077,0.104460,0.171492}%
\pgfsetfillcolor{currentfill}%
\pgfsetlinewidth{0.000000pt}%
\definecolor{currentstroke}{rgb}{0.000000,0.000000,0.000000}%
\pgfsetstrokecolor{currentstroke}%
\pgfsetdash{}{0pt}%
\pgfpathmoveto{\pgfqpoint{3.116444in}{2.629069in}}%
\pgfpathlineto{\pgfqpoint{3.372089in}{2.760957in}}%
\pgfpathlineto{\pgfqpoint{3.116444in}{2.629069in}}%
\pgfpathclose%
\pgfusepath{fill}%
\end{pgfscope}%
\begin{pgfscope}%
\pgfpathrectangle{\pgfqpoint{0.017500in}{0.017500in}}{\pgfqpoint{6.218282in}{3.580372in}} %
\pgfusepath{clip}%
\pgfsetbuttcap%
\pgfsetroundjoin%
\definecolor{currentfill}{rgb}{0.735077,0.104460,0.171492}%
\pgfsetfillcolor{currentfill}%
\pgfsetlinewidth{0.000000pt}%
\definecolor{currentstroke}{rgb}{0.000000,0.000000,0.000000}%
\pgfsetstrokecolor{currentstroke}%
\pgfsetdash{}{0pt}%
\pgfpathmoveto{\pgfqpoint{3.116444in}{2.629069in}}%
\pgfpathlineto{\pgfqpoint{3.372089in}{2.760957in}}%
\pgfpathlineto{\pgfqpoint{3.116444in}{2.629069in}}%
\pgfpathclose%
\pgfusepath{fill}%
\end{pgfscope}%
\begin{pgfscope}%
\pgfpathrectangle{\pgfqpoint{0.017500in}{0.017500in}}{\pgfqpoint{6.218282in}{3.580372in}} %
\pgfusepath{clip}%
\pgfsetbuttcap%
\pgfsetroundjoin%
\definecolor{currentfill}{rgb}{0.735077,0.104460,0.171492}%
\pgfsetfillcolor{currentfill}%
\pgfsetlinewidth{0.000000pt}%
\definecolor{currentstroke}{rgb}{0.000000,0.000000,0.000000}%
\pgfsetstrokecolor{currentstroke}%
\pgfsetdash{}{0pt}%
\pgfpathmoveto{\pgfqpoint{3.116444in}{2.629069in}}%
\pgfpathlineto{\pgfqpoint{3.372089in}{2.760957in}}%
\pgfpathlineto{\pgfqpoint{3.116444in}{2.629069in}}%
\pgfpathclose%
\pgfusepath{fill}%
\end{pgfscope}%
\begin{pgfscope}%
\pgfpathrectangle{\pgfqpoint{0.017500in}{0.017500in}}{\pgfqpoint{6.218282in}{3.580372in}} %
\pgfusepath{clip}%
\pgfsetbuttcap%
\pgfsetroundjoin%
\definecolor{currentfill}{rgb}{0.735077,0.104460,0.171492}%
\pgfsetfillcolor{currentfill}%
\pgfsetlinewidth{0.000000pt}%
\definecolor{currentstroke}{rgb}{0.000000,0.000000,0.000000}%
\pgfsetstrokecolor{currentstroke}%
\pgfsetdash{}{0pt}%
\pgfpathmoveto{\pgfqpoint{3.116444in}{2.629069in}}%
\pgfpathlineto{\pgfqpoint{3.372089in}{2.760957in}}%
\pgfpathlineto{\pgfqpoint{3.116444in}{2.629069in}}%
\pgfpathclose%
\pgfusepath{fill}%
\end{pgfscope}%
\begin{pgfscope}%
\pgfpathrectangle{\pgfqpoint{0.017500in}{0.017500in}}{\pgfqpoint{6.218282in}{3.580372in}} %
\pgfusepath{clip}%
\pgfsetbuttcap%
\pgfsetroundjoin%
\definecolor{currentfill}{rgb}{0.735077,0.104460,0.171492}%
\pgfsetfillcolor{currentfill}%
\pgfsetlinewidth{0.000000pt}%
\definecolor{currentstroke}{rgb}{0.000000,0.000000,0.000000}%
\pgfsetstrokecolor{currentstroke}%
\pgfsetdash{}{0pt}%
\pgfpathmoveto{\pgfqpoint{3.116444in}{2.629069in}}%
\pgfpathlineto{\pgfqpoint{3.372089in}{2.760957in}}%
\pgfpathlineto{\pgfqpoint{3.116444in}{2.629069in}}%
\pgfpathclose%
\pgfusepath{fill}%
\end{pgfscope}%
\begin{pgfscope}%
\pgfpathrectangle{\pgfqpoint{0.017500in}{0.017500in}}{\pgfqpoint{6.218282in}{3.580372in}} %
\pgfusepath{clip}%
\pgfsetbuttcap%
\pgfsetroundjoin%
\definecolor{currentfill}{rgb}{0.735077,0.104460,0.171492}%
\pgfsetfillcolor{currentfill}%
\pgfsetlinewidth{0.000000pt}%
\definecolor{currentstroke}{rgb}{0.000000,0.000000,0.000000}%
\pgfsetstrokecolor{currentstroke}%
\pgfsetdash{}{0pt}%
\pgfpathmoveto{\pgfqpoint{3.116444in}{2.629069in}}%
\pgfpathlineto{\pgfqpoint{3.372089in}{2.760957in}}%
\pgfpathlineto{\pgfqpoint{3.116444in}{2.629069in}}%
\pgfpathclose%
\pgfusepath{fill}%
\end{pgfscope}%
\begin{pgfscope}%
\pgfpathrectangle{\pgfqpoint{0.017500in}{0.017500in}}{\pgfqpoint{6.218282in}{3.580372in}} %
\pgfusepath{clip}%
\pgfsetbuttcap%
\pgfsetroundjoin%
\definecolor{currentfill}{rgb}{0.735077,0.104460,0.171492}%
\pgfsetfillcolor{currentfill}%
\pgfsetlinewidth{0.000000pt}%
\definecolor{currentstroke}{rgb}{0.000000,0.000000,0.000000}%
\pgfsetstrokecolor{currentstroke}%
\pgfsetdash{}{0pt}%
\pgfpathmoveto{\pgfqpoint{3.116444in}{2.629069in}}%
\pgfpathlineto{\pgfqpoint{3.372089in}{2.760957in}}%
\pgfpathlineto{\pgfqpoint{3.116444in}{2.629069in}}%
\pgfpathclose%
\pgfusepath{fill}%
\end{pgfscope}%
\begin{pgfscope}%
\pgfpathrectangle{\pgfqpoint{0.017500in}{0.017500in}}{\pgfqpoint{6.218282in}{3.580372in}} %
\pgfusepath{clip}%
\pgfsetbuttcap%
\pgfsetroundjoin%
\definecolor{currentfill}{rgb}{0.735077,0.104460,0.171492}%
\pgfsetfillcolor{currentfill}%
\pgfsetlinewidth{0.000000pt}%
\definecolor{currentstroke}{rgb}{0.000000,0.000000,0.000000}%
\pgfsetstrokecolor{currentstroke}%
\pgfsetdash{}{0pt}%
\pgfpathmoveto{\pgfqpoint{3.116444in}{2.629069in}}%
\pgfpathlineto{\pgfqpoint{3.372089in}{2.760957in}}%
\pgfpathlineto{\pgfqpoint{3.116444in}{2.629069in}}%
\pgfpathclose%
\pgfusepath{fill}%
\end{pgfscope}%
\begin{pgfscope}%
\pgfpathrectangle{\pgfqpoint{0.017500in}{0.017500in}}{\pgfqpoint{6.218282in}{3.580372in}} %
\pgfusepath{clip}%
\pgfsetbuttcap%
\pgfsetroundjoin%
\definecolor{currentfill}{rgb}{0.735077,0.104460,0.171492}%
\pgfsetfillcolor{currentfill}%
\pgfsetlinewidth{0.000000pt}%
\definecolor{currentstroke}{rgb}{0.000000,0.000000,0.000000}%
\pgfsetstrokecolor{currentstroke}%
\pgfsetdash{}{0pt}%
\pgfpathmoveto{\pgfqpoint{3.116444in}{2.629069in}}%
\pgfpathlineto{\pgfqpoint{3.372089in}{2.760957in}}%
\pgfpathlineto{\pgfqpoint{3.116444in}{2.629069in}}%
\pgfpathclose%
\pgfusepath{fill}%
\end{pgfscope}%
\begin{pgfscope}%
\pgfpathrectangle{\pgfqpoint{0.017500in}{0.017500in}}{\pgfqpoint{6.218282in}{3.580372in}} %
\pgfusepath{clip}%
\pgfsetbuttcap%
\pgfsetroundjoin%
\definecolor{currentfill}{rgb}{0.735077,0.104460,0.171492}%
\pgfsetfillcolor{currentfill}%
\pgfsetlinewidth{0.000000pt}%
\definecolor{currentstroke}{rgb}{0.000000,0.000000,0.000000}%
\pgfsetstrokecolor{currentstroke}%
\pgfsetdash{}{0pt}%
\pgfpathmoveto{\pgfqpoint{3.116444in}{2.629069in}}%
\pgfpathlineto{\pgfqpoint{3.372089in}{2.760957in}}%
\pgfpathlineto{\pgfqpoint{3.116444in}{2.629069in}}%
\pgfpathclose%
\pgfusepath{fill}%
\end{pgfscope}%
\begin{pgfscope}%
\pgfpathrectangle{\pgfqpoint{0.017500in}{0.017500in}}{\pgfqpoint{6.218282in}{3.580372in}} %
\pgfusepath{clip}%
\pgfsetbuttcap%
\pgfsetroundjoin%
\definecolor{currentfill}{rgb}{0.735077,0.104460,0.171492}%
\pgfsetfillcolor{currentfill}%
\pgfsetlinewidth{0.000000pt}%
\definecolor{currentstroke}{rgb}{0.000000,0.000000,0.000000}%
\pgfsetstrokecolor{currentstroke}%
\pgfsetdash{}{0pt}%
\pgfpathmoveto{\pgfqpoint{3.116444in}{2.629069in}}%
\pgfpathlineto{\pgfqpoint{3.372089in}{2.760957in}}%
\pgfpathlineto{\pgfqpoint{3.116444in}{2.629069in}}%
\pgfpathclose%
\pgfusepath{fill}%
\end{pgfscope}%
\begin{pgfscope}%
\pgfpathrectangle{\pgfqpoint{0.017500in}{0.017500in}}{\pgfqpoint{6.218282in}{3.580372in}} %
\pgfusepath{clip}%
\pgfsetbuttcap%
\pgfsetroundjoin%
\definecolor{currentfill}{rgb}{0.735077,0.104460,0.171492}%
\pgfsetfillcolor{currentfill}%
\pgfsetlinewidth{0.000000pt}%
\definecolor{currentstroke}{rgb}{0.000000,0.000000,0.000000}%
\pgfsetstrokecolor{currentstroke}%
\pgfsetdash{}{0pt}%
\pgfpathmoveto{\pgfqpoint{3.116444in}{2.629069in}}%
\pgfpathlineto{\pgfqpoint{3.372089in}{2.760957in}}%
\pgfpathlineto{\pgfqpoint{3.116444in}{2.629069in}}%
\pgfpathclose%
\pgfusepath{fill}%
\end{pgfscope}%
\begin{pgfscope}%
\pgfpathrectangle{\pgfqpoint{0.017500in}{0.017500in}}{\pgfqpoint{6.218282in}{3.580372in}} %
\pgfusepath{clip}%
\pgfsetbuttcap%
\pgfsetroundjoin%
\definecolor{currentfill}{rgb}{0.735077,0.104460,0.171492}%
\pgfsetfillcolor{currentfill}%
\pgfsetlinewidth{0.000000pt}%
\definecolor{currentstroke}{rgb}{0.000000,0.000000,0.000000}%
\pgfsetstrokecolor{currentstroke}%
\pgfsetdash{}{0pt}%
\pgfpathmoveto{\pgfqpoint{3.116444in}{2.629069in}}%
\pgfpathlineto{\pgfqpoint{3.372089in}{2.760957in}}%
\pgfpathlineto{\pgfqpoint{3.116444in}{2.629069in}}%
\pgfpathclose%
\pgfusepath{fill}%
\end{pgfscope}%
\begin{pgfscope}%
\pgfpathrectangle{\pgfqpoint{0.017500in}{0.017500in}}{\pgfqpoint{6.218282in}{3.580372in}} %
\pgfusepath{clip}%
\pgfsetbuttcap%
\pgfsetroundjoin%
\definecolor{currentfill}{rgb}{0.735077,0.104460,0.171492}%
\pgfsetfillcolor{currentfill}%
\pgfsetlinewidth{0.000000pt}%
\definecolor{currentstroke}{rgb}{0.000000,0.000000,0.000000}%
\pgfsetstrokecolor{currentstroke}%
\pgfsetdash{}{0pt}%
\pgfpathmoveto{\pgfqpoint{3.116444in}{2.629069in}}%
\pgfpathlineto{\pgfqpoint{3.372089in}{2.760957in}}%
\pgfpathlineto{\pgfqpoint{3.116444in}{2.629069in}}%
\pgfpathclose%
\pgfusepath{fill}%
\end{pgfscope}%
\begin{pgfscope}%
\pgfpathrectangle{\pgfqpoint{0.017500in}{0.017500in}}{\pgfqpoint{6.218282in}{3.580372in}} %
\pgfusepath{clip}%
\pgfsetbuttcap%
\pgfsetroundjoin%
\definecolor{currentfill}{rgb}{0.735077,0.104460,0.171492}%
\pgfsetfillcolor{currentfill}%
\pgfsetlinewidth{0.000000pt}%
\definecolor{currentstroke}{rgb}{0.000000,0.000000,0.000000}%
\pgfsetstrokecolor{currentstroke}%
\pgfsetdash{}{0pt}%
\pgfpathmoveto{\pgfqpoint{3.116444in}{2.629069in}}%
\pgfpathlineto{\pgfqpoint{3.372089in}{2.760957in}}%
\pgfpathlineto{\pgfqpoint{3.116444in}{2.629069in}}%
\pgfpathclose%
\pgfusepath{fill}%
\end{pgfscope}%
\begin{pgfscope}%
\pgfpathrectangle{\pgfqpoint{0.017500in}{0.017500in}}{\pgfqpoint{6.218282in}{3.580372in}} %
\pgfusepath{clip}%
\pgfsetbuttcap%
\pgfsetroundjoin%
\definecolor{currentfill}{rgb}{0.906154,0.842091,0.806151}%
\pgfsetfillcolor{currentfill}%
\pgfsetlinewidth{0.000000pt}%
\definecolor{currentstroke}{rgb}{0.000000,0.000000,0.000000}%
\pgfsetstrokecolor{currentstroke}%
\pgfsetdash{}{0pt}%
\pgfpathmoveto{\pgfqpoint{4.511715in}{2.762029in}}%
\pgfpathlineto{\pgfqpoint{3.236097in}{0.721171in}}%
\pgfpathlineto{\pgfqpoint{4.511715in}{2.762029in}}%
\pgfpathclose%
\pgfusepath{fill}%
\end{pgfscope}%
\begin{pgfscope}%
\pgfpathrectangle{\pgfqpoint{0.017500in}{0.017500in}}{\pgfqpoint{6.218282in}{3.580372in}} %
\pgfusepath{clip}%
\pgfsetbuttcap%
\pgfsetroundjoin%
\definecolor{currentfill}{rgb}{0.906154,0.842091,0.806151}%
\pgfsetfillcolor{currentfill}%
\pgfsetlinewidth{0.000000pt}%
\definecolor{currentstroke}{rgb}{0.000000,0.000000,0.000000}%
\pgfsetstrokecolor{currentstroke}%
\pgfsetdash{}{0pt}%
\pgfpathmoveto{\pgfqpoint{4.511715in}{2.762029in}}%
\pgfpathlineto{\pgfqpoint{3.236097in}{0.721171in}}%
\pgfpathlineto{\pgfqpoint{4.511715in}{2.762029in}}%
\pgfpathclose%
\pgfusepath{fill}%
\end{pgfscope}%
\begin{pgfscope}%
\pgfpathrectangle{\pgfqpoint{0.017500in}{0.017500in}}{\pgfqpoint{6.218282in}{3.580372in}} %
\pgfusepath{clip}%
\pgfsetbuttcap%
\pgfsetroundjoin%
\definecolor{currentfill}{rgb}{0.906154,0.842091,0.806151}%
\pgfsetfillcolor{currentfill}%
\pgfsetlinewidth{0.000000pt}%
\definecolor{currentstroke}{rgb}{0.000000,0.000000,0.000000}%
\pgfsetstrokecolor{currentstroke}%
\pgfsetdash{}{0pt}%
\pgfpathmoveto{\pgfqpoint{4.511715in}{2.762029in}}%
\pgfpathlineto{\pgfqpoint{3.236097in}{0.721171in}}%
\pgfpathlineto{\pgfqpoint{4.511715in}{2.762029in}}%
\pgfpathclose%
\pgfusepath{fill}%
\end{pgfscope}%
\begin{pgfscope}%
\pgfpathrectangle{\pgfqpoint{0.017500in}{0.017500in}}{\pgfqpoint{6.218282in}{3.580372in}} %
\pgfusepath{clip}%
\pgfsetbuttcap%
\pgfsetroundjoin%
\definecolor{currentfill}{rgb}{0.906154,0.842091,0.806151}%
\pgfsetfillcolor{currentfill}%
\pgfsetlinewidth{0.000000pt}%
\definecolor{currentstroke}{rgb}{0.000000,0.000000,0.000000}%
\pgfsetstrokecolor{currentstroke}%
\pgfsetdash{}{0pt}%
\pgfpathmoveto{\pgfqpoint{4.511715in}{2.762029in}}%
\pgfpathlineto{\pgfqpoint{3.236097in}{0.721171in}}%
\pgfpathlineto{\pgfqpoint{4.511715in}{2.762029in}}%
\pgfpathclose%
\pgfusepath{fill}%
\end{pgfscope}%
\begin{pgfscope}%
\pgfpathrectangle{\pgfqpoint{0.017500in}{0.017500in}}{\pgfqpoint{6.218282in}{3.580372in}} %
\pgfusepath{clip}%
\pgfsetbuttcap%
\pgfsetroundjoin%
\definecolor{currentfill}{rgb}{0.906154,0.842091,0.806151}%
\pgfsetfillcolor{currentfill}%
\pgfsetlinewidth{0.000000pt}%
\definecolor{currentstroke}{rgb}{0.000000,0.000000,0.000000}%
\pgfsetstrokecolor{currentstroke}%
\pgfsetdash{}{0pt}%
\pgfpathmoveto{\pgfqpoint{4.511715in}{2.762029in}}%
\pgfpathlineto{\pgfqpoint{3.236097in}{0.721171in}}%
\pgfpathlineto{\pgfqpoint{4.511715in}{2.762029in}}%
\pgfpathclose%
\pgfusepath{fill}%
\end{pgfscope}%
\begin{pgfscope}%
\pgfpathrectangle{\pgfqpoint{0.017500in}{0.017500in}}{\pgfqpoint{6.218282in}{3.580372in}} %
\pgfusepath{clip}%
\pgfsetbuttcap%
\pgfsetroundjoin%
\definecolor{currentfill}{rgb}{0.906154,0.842091,0.806151}%
\pgfsetfillcolor{currentfill}%
\pgfsetlinewidth{0.000000pt}%
\definecolor{currentstroke}{rgb}{0.000000,0.000000,0.000000}%
\pgfsetstrokecolor{currentstroke}%
\pgfsetdash{}{0pt}%
\pgfpathmoveto{\pgfqpoint{4.511715in}{2.762029in}}%
\pgfpathlineto{\pgfqpoint{3.236097in}{0.721171in}}%
\pgfpathlineto{\pgfqpoint{4.511715in}{2.762029in}}%
\pgfpathclose%
\pgfusepath{fill}%
\end{pgfscope}%
\begin{pgfscope}%
\pgfpathrectangle{\pgfqpoint{0.017500in}{0.017500in}}{\pgfqpoint{6.218282in}{3.580372in}} %
\pgfusepath{clip}%
\pgfsetbuttcap%
\pgfsetroundjoin%
\definecolor{currentfill}{rgb}{0.906154,0.842091,0.806151}%
\pgfsetfillcolor{currentfill}%
\pgfsetlinewidth{0.000000pt}%
\definecolor{currentstroke}{rgb}{0.000000,0.000000,0.000000}%
\pgfsetstrokecolor{currentstroke}%
\pgfsetdash{}{0pt}%
\pgfpathmoveto{\pgfqpoint{4.511715in}{2.762029in}}%
\pgfpathlineto{\pgfqpoint{3.236097in}{0.721171in}}%
\pgfpathlineto{\pgfqpoint{4.511715in}{2.762029in}}%
\pgfpathclose%
\pgfusepath{fill}%
\end{pgfscope}%
\begin{pgfscope}%
\pgfpathrectangle{\pgfqpoint{0.017500in}{0.017500in}}{\pgfqpoint{6.218282in}{3.580372in}} %
\pgfusepath{clip}%
\pgfsetbuttcap%
\pgfsetroundjoin%
\definecolor{currentfill}{rgb}{0.906154,0.842091,0.806151}%
\pgfsetfillcolor{currentfill}%
\pgfsetlinewidth{0.000000pt}%
\definecolor{currentstroke}{rgb}{0.000000,0.000000,0.000000}%
\pgfsetstrokecolor{currentstroke}%
\pgfsetdash{}{0pt}%
\pgfpathmoveto{\pgfqpoint{4.511715in}{2.762029in}}%
\pgfpathlineto{\pgfqpoint{3.236097in}{0.721171in}}%
\pgfpathlineto{\pgfqpoint{4.511715in}{2.762029in}}%
\pgfpathclose%
\pgfusepath{fill}%
\end{pgfscope}%
\begin{pgfscope}%
\pgfpathrectangle{\pgfqpoint{0.017500in}{0.017500in}}{\pgfqpoint{6.218282in}{3.580372in}} %
\pgfusepath{clip}%
\pgfsetbuttcap%
\pgfsetroundjoin%
\definecolor{currentfill}{rgb}{0.906154,0.842091,0.806151}%
\pgfsetfillcolor{currentfill}%
\pgfsetlinewidth{0.000000pt}%
\definecolor{currentstroke}{rgb}{0.000000,0.000000,0.000000}%
\pgfsetstrokecolor{currentstroke}%
\pgfsetdash{}{0pt}%
\pgfpathmoveto{\pgfqpoint{4.511715in}{2.762029in}}%
\pgfpathlineto{\pgfqpoint{3.236097in}{0.721171in}}%
\pgfpathlineto{\pgfqpoint{4.511715in}{2.762029in}}%
\pgfpathclose%
\pgfusepath{fill}%
\end{pgfscope}%
\begin{pgfscope}%
\pgfpathrectangle{\pgfqpoint{0.017500in}{0.017500in}}{\pgfqpoint{6.218282in}{3.580372in}} %
\pgfusepath{clip}%
\pgfsetbuttcap%
\pgfsetroundjoin%
\definecolor{currentfill}{rgb}{0.906154,0.842091,0.806151}%
\pgfsetfillcolor{currentfill}%
\pgfsetlinewidth{0.000000pt}%
\definecolor{currentstroke}{rgb}{0.000000,0.000000,0.000000}%
\pgfsetstrokecolor{currentstroke}%
\pgfsetdash{}{0pt}%
\pgfpathmoveto{\pgfqpoint{4.511715in}{2.762029in}}%
\pgfpathlineto{\pgfqpoint{3.236097in}{0.721171in}}%
\pgfpathlineto{\pgfqpoint{4.511715in}{2.762029in}}%
\pgfpathclose%
\pgfusepath{fill}%
\end{pgfscope}%
\begin{pgfscope}%
\pgfpathrectangle{\pgfqpoint{0.017500in}{0.017500in}}{\pgfqpoint{6.218282in}{3.580372in}} %
\pgfusepath{clip}%
\pgfsetbuttcap%
\pgfsetroundjoin%
\definecolor{currentfill}{rgb}{0.906154,0.842091,0.806151}%
\pgfsetfillcolor{currentfill}%
\pgfsetlinewidth{0.000000pt}%
\definecolor{currentstroke}{rgb}{0.000000,0.000000,0.000000}%
\pgfsetstrokecolor{currentstroke}%
\pgfsetdash{}{0pt}%
\pgfpathmoveto{\pgfqpoint{4.511715in}{2.762029in}}%
\pgfpathlineto{\pgfqpoint{3.236097in}{0.721171in}}%
\pgfpathlineto{\pgfqpoint{4.511715in}{2.762029in}}%
\pgfpathclose%
\pgfusepath{fill}%
\end{pgfscope}%
\begin{pgfscope}%
\pgfpathrectangle{\pgfqpoint{0.017500in}{0.017500in}}{\pgfqpoint{6.218282in}{3.580372in}} %
\pgfusepath{clip}%
\pgfsetbuttcap%
\pgfsetroundjoin%
\definecolor{currentfill}{rgb}{0.906154,0.842091,0.806151}%
\pgfsetfillcolor{currentfill}%
\pgfsetlinewidth{0.000000pt}%
\definecolor{currentstroke}{rgb}{0.000000,0.000000,0.000000}%
\pgfsetstrokecolor{currentstroke}%
\pgfsetdash{}{0pt}%
\pgfpathmoveto{\pgfqpoint{4.511715in}{2.762029in}}%
\pgfpathlineto{\pgfqpoint{3.236097in}{0.721171in}}%
\pgfpathlineto{\pgfqpoint{4.511715in}{2.762029in}}%
\pgfpathclose%
\pgfusepath{fill}%
\end{pgfscope}%
\begin{pgfscope}%
\pgfpathrectangle{\pgfqpoint{0.017500in}{0.017500in}}{\pgfqpoint{6.218282in}{3.580372in}} %
\pgfusepath{clip}%
\pgfsetbuttcap%
\pgfsetroundjoin%
\definecolor{currentfill}{rgb}{0.906154,0.842091,0.806151}%
\pgfsetfillcolor{currentfill}%
\pgfsetlinewidth{0.000000pt}%
\definecolor{currentstroke}{rgb}{0.000000,0.000000,0.000000}%
\pgfsetstrokecolor{currentstroke}%
\pgfsetdash{}{0pt}%
\pgfpathmoveto{\pgfqpoint{4.511715in}{2.762029in}}%
\pgfpathlineto{\pgfqpoint{3.236097in}{0.721171in}}%
\pgfpathlineto{\pgfqpoint{4.511715in}{2.762029in}}%
\pgfpathclose%
\pgfusepath{fill}%
\end{pgfscope}%
\begin{pgfscope}%
\pgfpathrectangle{\pgfqpoint{0.017500in}{0.017500in}}{\pgfqpoint{6.218282in}{3.580372in}} %
\pgfusepath{clip}%
\pgfsetbuttcap%
\pgfsetroundjoin%
\definecolor{currentfill}{rgb}{0.906154,0.842091,0.806151}%
\pgfsetfillcolor{currentfill}%
\pgfsetlinewidth{0.000000pt}%
\definecolor{currentstroke}{rgb}{0.000000,0.000000,0.000000}%
\pgfsetstrokecolor{currentstroke}%
\pgfsetdash{}{0pt}%
\pgfpathmoveto{\pgfqpoint{4.511715in}{2.762029in}}%
\pgfpathlineto{\pgfqpoint{3.236097in}{0.721171in}}%
\pgfpathlineto{\pgfqpoint{4.511715in}{2.762029in}}%
\pgfpathclose%
\pgfusepath{fill}%
\end{pgfscope}%
\begin{pgfscope}%
\pgfpathrectangle{\pgfqpoint{0.017500in}{0.017500in}}{\pgfqpoint{6.218282in}{3.580372in}} %
\pgfusepath{clip}%
\pgfsetbuttcap%
\pgfsetroundjoin%
\definecolor{currentfill}{rgb}{0.906154,0.842091,0.806151}%
\pgfsetfillcolor{currentfill}%
\pgfsetlinewidth{0.000000pt}%
\definecolor{currentstroke}{rgb}{0.000000,0.000000,0.000000}%
\pgfsetstrokecolor{currentstroke}%
\pgfsetdash{}{0pt}%
\pgfpathmoveto{\pgfqpoint{4.511715in}{2.762029in}}%
\pgfpathlineto{\pgfqpoint{3.236097in}{0.721171in}}%
\pgfpathlineto{\pgfqpoint{4.511715in}{2.762029in}}%
\pgfpathclose%
\pgfusepath{fill}%
\end{pgfscope}%
\begin{pgfscope}%
\pgfpathrectangle{\pgfqpoint{0.017500in}{0.017500in}}{\pgfqpoint{6.218282in}{3.580372in}} %
\pgfusepath{clip}%
\pgfsetbuttcap%
\pgfsetroundjoin%
\definecolor{currentfill}{rgb}{0.906154,0.842091,0.806151}%
\pgfsetfillcolor{currentfill}%
\pgfsetlinewidth{0.000000pt}%
\definecolor{currentstroke}{rgb}{0.000000,0.000000,0.000000}%
\pgfsetstrokecolor{currentstroke}%
\pgfsetdash{}{0pt}%
\pgfpathmoveto{\pgfqpoint{4.511715in}{2.762029in}}%
\pgfpathlineto{\pgfqpoint{3.236097in}{0.721171in}}%
\pgfpathlineto{\pgfqpoint{4.511715in}{2.762029in}}%
\pgfpathclose%
\pgfusepath{fill}%
\end{pgfscope}%
\begin{pgfscope}%
\pgfpathrectangle{\pgfqpoint{0.017500in}{0.017500in}}{\pgfqpoint{6.218282in}{3.580372in}} %
\pgfusepath{clip}%
\pgfsetbuttcap%
\pgfsetroundjoin%
\definecolor{currentfill}{rgb}{0.906154,0.842091,0.806151}%
\pgfsetfillcolor{currentfill}%
\pgfsetlinewidth{0.000000pt}%
\definecolor{currentstroke}{rgb}{0.000000,0.000000,0.000000}%
\pgfsetstrokecolor{currentstroke}%
\pgfsetdash{}{0pt}%
\pgfpathmoveto{\pgfqpoint{4.511715in}{2.762029in}}%
\pgfpathlineto{\pgfqpoint{3.236097in}{0.721171in}}%
\pgfpathlineto{\pgfqpoint{4.511715in}{2.762029in}}%
\pgfpathclose%
\pgfusepath{fill}%
\end{pgfscope}%
\begin{pgfscope}%
\pgfpathrectangle{\pgfqpoint{0.017500in}{0.017500in}}{\pgfqpoint{6.218282in}{3.580372in}} %
\pgfusepath{clip}%
\pgfsetbuttcap%
\pgfsetroundjoin%
\definecolor{currentfill}{rgb}{0.906154,0.842091,0.806151}%
\pgfsetfillcolor{currentfill}%
\pgfsetlinewidth{0.000000pt}%
\definecolor{currentstroke}{rgb}{0.000000,0.000000,0.000000}%
\pgfsetstrokecolor{currentstroke}%
\pgfsetdash{}{0pt}%
\pgfpathmoveto{\pgfqpoint{4.511715in}{2.762029in}}%
\pgfpathlineto{\pgfqpoint{3.236097in}{0.721171in}}%
\pgfpathlineto{\pgfqpoint{4.511715in}{2.762029in}}%
\pgfpathclose%
\pgfusepath{fill}%
\end{pgfscope}%
\begin{pgfscope}%
\pgfpathrectangle{\pgfqpoint{0.017500in}{0.017500in}}{\pgfqpoint{6.218282in}{3.580372in}} %
\pgfusepath{clip}%
\pgfsetbuttcap%
\pgfsetroundjoin%
\definecolor{currentfill}{rgb}{0.906154,0.842091,0.806151}%
\pgfsetfillcolor{currentfill}%
\pgfsetlinewidth{0.000000pt}%
\definecolor{currentstroke}{rgb}{0.000000,0.000000,0.000000}%
\pgfsetstrokecolor{currentstroke}%
\pgfsetdash{}{0pt}%
\pgfpathmoveto{\pgfqpoint{4.511715in}{2.762029in}}%
\pgfpathlineto{\pgfqpoint{3.236097in}{0.721171in}}%
\pgfpathlineto{\pgfqpoint{4.511715in}{2.762029in}}%
\pgfpathclose%
\pgfusepath{fill}%
\end{pgfscope}%
\begin{pgfscope}%
\pgfpathrectangle{\pgfqpoint{0.017500in}{0.017500in}}{\pgfqpoint{6.218282in}{3.580372in}} %
\pgfusepath{clip}%
\pgfsetbuttcap%
\pgfsetroundjoin%
\definecolor{currentfill}{rgb}{0.906154,0.842091,0.806151}%
\pgfsetfillcolor{currentfill}%
\pgfsetlinewidth{0.000000pt}%
\definecolor{currentstroke}{rgb}{0.000000,0.000000,0.000000}%
\pgfsetstrokecolor{currentstroke}%
\pgfsetdash{}{0pt}%
\pgfpathmoveto{\pgfqpoint{4.511715in}{2.762029in}}%
\pgfpathlineto{\pgfqpoint{3.236097in}{0.721171in}}%
\pgfpathlineto{\pgfqpoint{4.511715in}{2.762029in}}%
\pgfpathclose%
\pgfusepath{fill}%
\end{pgfscope}%
\begin{pgfscope}%
\pgfpathrectangle{\pgfqpoint{0.017500in}{0.017500in}}{\pgfqpoint{6.218282in}{3.580372in}} %
\pgfusepath{clip}%
\pgfsetbuttcap%
\pgfsetroundjoin%
\definecolor{currentfill}{rgb}{0.906154,0.842091,0.806151}%
\pgfsetfillcolor{currentfill}%
\pgfsetlinewidth{0.000000pt}%
\definecolor{currentstroke}{rgb}{0.000000,0.000000,0.000000}%
\pgfsetstrokecolor{currentstroke}%
\pgfsetdash{}{0pt}%
\pgfpathmoveto{\pgfqpoint{4.511715in}{2.762029in}}%
\pgfpathlineto{\pgfqpoint{3.236097in}{0.721171in}}%
\pgfpathlineto{\pgfqpoint{4.511715in}{2.762029in}}%
\pgfpathclose%
\pgfusepath{fill}%
\end{pgfscope}%
\begin{pgfscope}%
\pgfpathrectangle{\pgfqpoint{0.017500in}{0.017500in}}{\pgfqpoint{6.218282in}{3.580372in}} %
\pgfusepath{clip}%
\pgfsetbuttcap%
\pgfsetroundjoin%
\definecolor{currentfill}{rgb}{0.906154,0.842091,0.806151}%
\pgfsetfillcolor{currentfill}%
\pgfsetlinewidth{0.000000pt}%
\definecolor{currentstroke}{rgb}{0.000000,0.000000,0.000000}%
\pgfsetstrokecolor{currentstroke}%
\pgfsetdash{}{0pt}%
\pgfpathmoveto{\pgfqpoint{4.511715in}{2.762029in}}%
\pgfpathlineto{\pgfqpoint{3.236097in}{0.721171in}}%
\pgfpathlineto{\pgfqpoint{4.511715in}{2.762029in}}%
\pgfpathclose%
\pgfusepath{fill}%
\end{pgfscope}%
\begin{pgfscope}%
\pgfpathrectangle{\pgfqpoint{0.017500in}{0.017500in}}{\pgfqpoint{6.218282in}{3.580372in}} %
\pgfusepath{clip}%
\pgfsetbuttcap%
\pgfsetroundjoin%
\definecolor{currentfill}{rgb}{0.906154,0.842091,0.806151}%
\pgfsetfillcolor{currentfill}%
\pgfsetlinewidth{0.000000pt}%
\definecolor{currentstroke}{rgb}{0.000000,0.000000,0.000000}%
\pgfsetstrokecolor{currentstroke}%
\pgfsetdash{}{0pt}%
\pgfpathmoveto{\pgfqpoint{4.511715in}{2.762029in}}%
\pgfpathlineto{\pgfqpoint{3.236097in}{0.721171in}}%
\pgfpathlineto{\pgfqpoint{4.511715in}{2.762029in}}%
\pgfpathclose%
\pgfusepath{fill}%
\end{pgfscope}%
\begin{pgfscope}%
\pgfpathrectangle{\pgfqpoint{0.017500in}{0.017500in}}{\pgfqpoint{6.218282in}{3.580372in}} %
\pgfusepath{clip}%
\pgfsetbuttcap%
\pgfsetroundjoin%
\definecolor{currentfill}{rgb}{0.906154,0.842091,0.806151}%
\pgfsetfillcolor{currentfill}%
\pgfsetlinewidth{0.000000pt}%
\definecolor{currentstroke}{rgb}{0.000000,0.000000,0.000000}%
\pgfsetstrokecolor{currentstroke}%
\pgfsetdash{}{0pt}%
\pgfpathmoveto{\pgfqpoint{4.511715in}{2.762029in}}%
\pgfpathlineto{\pgfqpoint{3.236097in}{0.721171in}}%
\pgfpathlineto{\pgfqpoint{4.511715in}{2.762029in}}%
\pgfpathclose%
\pgfusepath{fill}%
\end{pgfscope}%
\begin{pgfscope}%
\pgfpathrectangle{\pgfqpoint{0.017500in}{0.017500in}}{\pgfqpoint{6.218282in}{3.580372in}} %
\pgfusepath{clip}%
\pgfsetbuttcap%
\pgfsetroundjoin%
\definecolor{currentfill}{rgb}{0.906154,0.842091,0.806151}%
\pgfsetfillcolor{currentfill}%
\pgfsetlinewidth{0.000000pt}%
\definecolor{currentstroke}{rgb}{0.000000,0.000000,0.000000}%
\pgfsetstrokecolor{currentstroke}%
\pgfsetdash{}{0pt}%
\pgfpathmoveto{\pgfqpoint{4.511715in}{2.762029in}}%
\pgfpathlineto{\pgfqpoint{3.236097in}{0.721171in}}%
\pgfpathlineto{\pgfqpoint{4.511715in}{2.762029in}}%
\pgfpathclose%
\pgfusepath{fill}%
\end{pgfscope}%
\begin{pgfscope}%
\pgfpathrectangle{\pgfqpoint{0.017500in}{0.017500in}}{\pgfqpoint{6.218282in}{3.580372in}} %
\pgfusepath{clip}%
\pgfsetbuttcap%
\pgfsetroundjoin%
\definecolor{currentfill}{rgb}{0.735077,0.104460,0.171492}%
\pgfsetfillcolor{currentfill}%
\pgfsetlinewidth{0.000000pt}%
\definecolor{currentstroke}{rgb}{0.000000,0.000000,0.000000}%
\pgfsetstrokecolor{currentstroke}%
\pgfsetdash{}{0pt}%
\pgfpathmoveto{\pgfqpoint{3.116444in}{2.629069in}}%
\pgfpathlineto{\pgfqpoint{3.564605in}{2.552204in}}%
\pgfpathlineto{\pgfqpoint{3.817122in}{2.696128in}}%
\pgfpathlineto{\pgfqpoint{3.372089in}{2.760957in}}%
\pgfpathlineto{\pgfqpoint{3.116444in}{2.629069in}}%
\pgfpathclose%
\pgfusepath{fill}%
\end{pgfscope}%
\begin{pgfscope}%
\pgfpathrectangle{\pgfqpoint{0.017500in}{0.017500in}}{\pgfqpoint{6.218282in}{3.580372in}} %
\pgfusepath{clip}%
\pgfsetbuttcap%
\pgfsetroundjoin%
\definecolor{currentfill}{rgb}{0.735077,0.104460,0.171492}%
\pgfsetfillcolor{currentfill}%
\pgfsetlinewidth{0.000000pt}%
\definecolor{currentstroke}{rgb}{0.000000,0.000000,0.000000}%
\pgfsetstrokecolor{currentstroke}%
\pgfsetdash{}{0pt}%
\pgfpathmoveto{\pgfqpoint{3.116444in}{2.629069in}}%
\pgfpathlineto{\pgfqpoint{3.564605in}{2.552204in}}%
\pgfpathlineto{\pgfqpoint{3.817122in}{2.696128in}}%
\pgfpathlineto{\pgfqpoint{3.372089in}{2.760957in}}%
\pgfpathlineto{\pgfqpoint{3.116444in}{2.629069in}}%
\pgfpathclose%
\pgfusepath{fill}%
\end{pgfscope}%
\begin{pgfscope}%
\pgfpathrectangle{\pgfqpoint{0.017500in}{0.017500in}}{\pgfqpoint{6.218282in}{3.580372in}} %
\pgfusepath{clip}%
\pgfsetbuttcap%
\pgfsetroundjoin%
\definecolor{currentfill}{rgb}{0.735077,0.104460,0.171492}%
\pgfsetfillcolor{currentfill}%
\pgfsetlinewidth{0.000000pt}%
\definecolor{currentstroke}{rgb}{0.000000,0.000000,0.000000}%
\pgfsetstrokecolor{currentstroke}%
\pgfsetdash{}{0pt}%
\pgfpathmoveto{\pgfqpoint{3.116444in}{2.629069in}}%
\pgfpathlineto{\pgfqpoint{3.564605in}{2.552204in}}%
\pgfpathlineto{\pgfqpoint{3.817122in}{2.696128in}}%
\pgfpathlineto{\pgfqpoint{3.372089in}{2.760957in}}%
\pgfpathlineto{\pgfqpoint{3.116444in}{2.629069in}}%
\pgfpathclose%
\pgfusepath{fill}%
\end{pgfscope}%
\begin{pgfscope}%
\pgfpathrectangle{\pgfqpoint{0.017500in}{0.017500in}}{\pgfqpoint{6.218282in}{3.580372in}} %
\pgfusepath{clip}%
\pgfsetbuttcap%
\pgfsetroundjoin%
\definecolor{currentfill}{rgb}{0.735077,0.104460,0.171492}%
\pgfsetfillcolor{currentfill}%
\pgfsetlinewidth{0.000000pt}%
\definecolor{currentstroke}{rgb}{0.000000,0.000000,0.000000}%
\pgfsetstrokecolor{currentstroke}%
\pgfsetdash{}{0pt}%
\pgfpathmoveto{\pgfqpoint{3.116444in}{2.629069in}}%
\pgfpathlineto{\pgfqpoint{3.564605in}{2.552204in}}%
\pgfpathlineto{\pgfqpoint{3.817122in}{2.696128in}}%
\pgfpathlineto{\pgfqpoint{3.372089in}{2.760957in}}%
\pgfpathlineto{\pgfqpoint{3.116444in}{2.629069in}}%
\pgfpathclose%
\pgfusepath{fill}%
\end{pgfscope}%
\begin{pgfscope}%
\pgfpathrectangle{\pgfqpoint{0.017500in}{0.017500in}}{\pgfqpoint{6.218282in}{3.580372in}} %
\pgfusepath{clip}%
\pgfsetbuttcap%
\pgfsetroundjoin%
\definecolor{currentfill}{rgb}{0.735077,0.104460,0.171492}%
\pgfsetfillcolor{currentfill}%
\pgfsetlinewidth{0.000000pt}%
\definecolor{currentstroke}{rgb}{0.000000,0.000000,0.000000}%
\pgfsetstrokecolor{currentstroke}%
\pgfsetdash{}{0pt}%
\pgfpathmoveto{\pgfqpoint{3.116444in}{2.629069in}}%
\pgfpathlineto{\pgfqpoint{3.564605in}{2.552204in}}%
\pgfpathlineto{\pgfqpoint{3.817122in}{2.696128in}}%
\pgfpathlineto{\pgfqpoint{3.372089in}{2.760957in}}%
\pgfpathlineto{\pgfqpoint{3.116444in}{2.629069in}}%
\pgfpathclose%
\pgfusepath{fill}%
\end{pgfscope}%
\begin{pgfscope}%
\pgfpathrectangle{\pgfqpoint{0.017500in}{0.017500in}}{\pgfqpoint{6.218282in}{3.580372in}} %
\pgfusepath{clip}%
\pgfsetbuttcap%
\pgfsetroundjoin%
\definecolor{currentfill}{rgb}{0.735077,0.104460,0.171492}%
\pgfsetfillcolor{currentfill}%
\pgfsetlinewidth{0.000000pt}%
\definecolor{currentstroke}{rgb}{0.000000,0.000000,0.000000}%
\pgfsetstrokecolor{currentstroke}%
\pgfsetdash{}{0pt}%
\pgfpathmoveto{\pgfqpoint{3.116444in}{2.629069in}}%
\pgfpathlineto{\pgfqpoint{3.564605in}{2.552204in}}%
\pgfpathlineto{\pgfqpoint{3.817122in}{2.696128in}}%
\pgfpathlineto{\pgfqpoint{3.372089in}{2.760957in}}%
\pgfpathlineto{\pgfqpoint{3.116444in}{2.629069in}}%
\pgfpathclose%
\pgfusepath{fill}%
\end{pgfscope}%
\begin{pgfscope}%
\pgfpathrectangle{\pgfqpoint{0.017500in}{0.017500in}}{\pgfqpoint{6.218282in}{3.580372in}} %
\pgfusepath{clip}%
\pgfsetbuttcap%
\pgfsetroundjoin%
\definecolor{currentfill}{rgb}{0.705673,0.015556,0.150233}%
\pgfsetfillcolor{currentfill}%
\pgfsetlinewidth{0.000000pt}%
\definecolor{currentstroke}{rgb}{0.000000,0.000000,0.000000}%
\pgfsetstrokecolor{currentstroke}%
\pgfsetdash{}{0pt}%
\pgfpathmoveto{\pgfqpoint{4.268228in}{2.632966in}}%
\pgfpathlineto{\pgfqpoint{4.511715in}{2.762029in}}%
\pgfpathlineto{\pgfqpoint{4.268228in}{2.632966in}}%
\pgfpathclose%
\pgfusepath{fill}%
\end{pgfscope}%
\begin{pgfscope}%
\pgfpathrectangle{\pgfqpoint{0.017500in}{0.017500in}}{\pgfqpoint{6.218282in}{3.580372in}} %
\pgfusepath{clip}%
\pgfsetbuttcap%
\pgfsetroundjoin%
\definecolor{currentfill}{rgb}{0.705673,0.015556,0.150233}%
\pgfsetfillcolor{currentfill}%
\pgfsetlinewidth{0.000000pt}%
\definecolor{currentstroke}{rgb}{0.000000,0.000000,0.000000}%
\pgfsetstrokecolor{currentstroke}%
\pgfsetdash{}{0pt}%
\pgfpathmoveto{\pgfqpoint{4.268228in}{2.632966in}}%
\pgfpathlineto{\pgfqpoint{4.511715in}{2.762029in}}%
\pgfpathlineto{\pgfqpoint{4.268228in}{2.632966in}}%
\pgfpathclose%
\pgfusepath{fill}%
\end{pgfscope}%
\begin{pgfscope}%
\pgfpathrectangle{\pgfqpoint{0.017500in}{0.017500in}}{\pgfqpoint{6.218282in}{3.580372in}} %
\pgfusepath{clip}%
\pgfsetbuttcap%
\pgfsetroundjoin%
\definecolor{currentfill}{rgb}{0.705673,0.015556,0.150233}%
\pgfsetfillcolor{currentfill}%
\pgfsetlinewidth{0.000000pt}%
\definecolor{currentstroke}{rgb}{0.000000,0.000000,0.000000}%
\pgfsetstrokecolor{currentstroke}%
\pgfsetdash{}{0pt}%
\pgfpathmoveto{\pgfqpoint{4.268228in}{2.632966in}}%
\pgfpathlineto{\pgfqpoint{4.511715in}{2.762029in}}%
\pgfpathlineto{\pgfqpoint{4.268228in}{2.632966in}}%
\pgfpathclose%
\pgfusepath{fill}%
\end{pgfscope}%
\begin{pgfscope}%
\pgfpathrectangle{\pgfqpoint{0.017500in}{0.017500in}}{\pgfqpoint{6.218282in}{3.580372in}} %
\pgfusepath{clip}%
\pgfsetbuttcap%
\pgfsetroundjoin%
\definecolor{currentfill}{rgb}{0.705673,0.015556,0.150233}%
\pgfsetfillcolor{currentfill}%
\pgfsetlinewidth{0.000000pt}%
\definecolor{currentstroke}{rgb}{0.000000,0.000000,0.000000}%
\pgfsetstrokecolor{currentstroke}%
\pgfsetdash{}{0pt}%
\pgfpathmoveto{\pgfqpoint{4.268228in}{2.632966in}}%
\pgfpathlineto{\pgfqpoint{4.511715in}{2.762029in}}%
\pgfpathlineto{\pgfqpoint{4.268228in}{2.632966in}}%
\pgfpathclose%
\pgfusepath{fill}%
\end{pgfscope}%
\begin{pgfscope}%
\pgfpathrectangle{\pgfqpoint{0.017500in}{0.017500in}}{\pgfqpoint{6.218282in}{3.580372in}} %
\pgfusepath{clip}%
\pgfsetbuttcap%
\pgfsetroundjoin%
\definecolor{currentfill}{rgb}{0.705673,0.015556,0.150233}%
\pgfsetfillcolor{currentfill}%
\pgfsetlinewidth{0.000000pt}%
\definecolor{currentstroke}{rgb}{0.000000,0.000000,0.000000}%
\pgfsetstrokecolor{currentstroke}%
\pgfsetdash{}{0pt}%
\pgfpathmoveto{\pgfqpoint{4.268228in}{2.632966in}}%
\pgfpathlineto{\pgfqpoint{4.511715in}{2.762029in}}%
\pgfpathlineto{\pgfqpoint{4.268228in}{2.632966in}}%
\pgfpathclose%
\pgfusepath{fill}%
\end{pgfscope}%
\begin{pgfscope}%
\pgfpathrectangle{\pgfqpoint{0.017500in}{0.017500in}}{\pgfqpoint{6.218282in}{3.580372in}} %
\pgfusepath{clip}%
\pgfsetbuttcap%
\pgfsetroundjoin%
\definecolor{currentfill}{rgb}{0.705673,0.015556,0.150233}%
\pgfsetfillcolor{currentfill}%
\pgfsetlinewidth{0.000000pt}%
\definecolor{currentstroke}{rgb}{0.000000,0.000000,0.000000}%
\pgfsetstrokecolor{currentstroke}%
\pgfsetdash{}{0pt}%
\pgfpathmoveto{\pgfqpoint{4.268228in}{2.632966in}}%
\pgfpathlineto{\pgfqpoint{4.511715in}{2.762029in}}%
\pgfpathlineto{\pgfqpoint{4.268228in}{2.632966in}}%
\pgfpathclose%
\pgfusepath{fill}%
\end{pgfscope}%
\begin{pgfscope}%
\pgfpathrectangle{\pgfqpoint{0.017500in}{0.017500in}}{\pgfqpoint{6.218282in}{3.580372in}} %
\pgfusepath{clip}%
\pgfsetbuttcap%
\pgfsetroundjoin%
\definecolor{currentfill}{rgb}{0.705673,0.015556,0.150233}%
\pgfsetfillcolor{currentfill}%
\pgfsetlinewidth{0.000000pt}%
\definecolor{currentstroke}{rgb}{0.000000,0.000000,0.000000}%
\pgfsetstrokecolor{currentstroke}%
\pgfsetdash{}{0pt}%
\pgfpathmoveto{\pgfqpoint{4.268228in}{2.632966in}}%
\pgfpathlineto{\pgfqpoint{4.511715in}{2.762029in}}%
\pgfpathlineto{\pgfqpoint{4.268228in}{2.632966in}}%
\pgfpathclose%
\pgfusepath{fill}%
\end{pgfscope}%
\begin{pgfscope}%
\pgfpathrectangle{\pgfqpoint{0.017500in}{0.017500in}}{\pgfqpoint{6.218282in}{3.580372in}} %
\pgfusepath{clip}%
\pgfsetbuttcap%
\pgfsetroundjoin%
\definecolor{currentfill}{rgb}{0.705673,0.015556,0.150233}%
\pgfsetfillcolor{currentfill}%
\pgfsetlinewidth{0.000000pt}%
\definecolor{currentstroke}{rgb}{0.000000,0.000000,0.000000}%
\pgfsetstrokecolor{currentstroke}%
\pgfsetdash{}{0pt}%
\pgfpathmoveto{\pgfqpoint{4.268228in}{2.632966in}}%
\pgfpathlineto{\pgfqpoint{4.511715in}{2.762029in}}%
\pgfpathlineto{\pgfqpoint{4.268228in}{2.632966in}}%
\pgfpathclose%
\pgfusepath{fill}%
\end{pgfscope}%
\begin{pgfscope}%
\pgfpathrectangle{\pgfqpoint{0.017500in}{0.017500in}}{\pgfqpoint{6.218282in}{3.580372in}} %
\pgfusepath{clip}%
\pgfsetbuttcap%
\pgfsetroundjoin%
\definecolor{currentfill}{rgb}{0.705673,0.015556,0.150233}%
\pgfsetfillcolor{currentfill}%
\pgfsetlinewidth{0.000000pt}%
\definecolor{currentstroke}{rgb}{0.000000,0.000000,0.000000}%
\pgfsetstrokecolor{currentstroke}%
\pgfsetdash{}{0pt}%
\pgfpathmoveto{\pgfqpoint{4.268228in}{2.632966in}}%
\pgfpathlineto{\pgfqpoint{4.511715in}{2.762029in}}%
\pgfpathlineto{\pgfqpoint{4.268228in}{2.632966in}}%
\pgfpathclose%
\pgfusepath{fill}%
\end{pgfscope}%
\begin{pgfscope}%
\pgfpathrectangle{\pgfqpoint{0.017500in}{0.017500in}}{\pgfqpoint{6.218282in}{3.580372in}} %
\pgfusepath{clip}%
\pgfsetbuttcap%
\pgfsetroundjoin%
\definecolor{currentfill}{rgb}{0.705673,0.015556,0.150233}%
\pgfsetfillcolor{currentfill}%
\pgfsetlinewidth{0.000000pt}%
\definecolor{currentstroke}{rgb}{0.000000,0.000000,0.000000}%
\pgfsetstrokecolor{currentstroke}%
\pgfsetdash{}{0pt}%
\pgfpathmoveto{\pgfqpoint{4.268228in}{2.632966in}}%
\pgfpathlineto{\pgfqpoint{4.511715in}{2.762029in}}%
\pgfpathlineto{\pgfqpoint{4.268228in}{2.632966in}}%
\pgfpathclose%
\pgfusepath{fill}%
\end{pgfscope}%
\begin{pgfscope}%
\pgfpathrectangle{\pgfqpoint{0.017500in}{0.017500in}}{\pgfqpoint{6.218282in}{3.580372in}} %
\pgfusepath{clip}%
\pgfsetbuttcap%
\pgfsetroundjoin%
\definecolor{currentfill}{rgb}{0.705673,0.015556,0.150233}%
\pgfsetfillcolor{currentfill}%
\pgfsetlinewidth{0.000000pt}%
\definecolor{currentstroke}{rgb}{0.000000,0.000000,0.000000}%
\pgfsetstrokecolor{currentstroke}%
\pgfsetdash{}{0pt}%
\pgfpathmoveto{\pgfqpoint{4.268228in}{2.632966in}}%
\pgfpathlineto{\pgfqpoint{4.511715in}{2.762029in}}%
\pgfpathlineto{\pgfqpoint{4.268228in}{2.632966in}}%
\pgfpathclose%
\pgfusepath{fill}%
\end{pgfscope}%
\begin{pgfscope}%
\pgfpathrectangle{\pgfqpoint{0.017500in}{0.017500in}}{\pgfqpoint{6.218282in}{3.580372in}} %
\pgfusepath{clip}%
\pgfsetbuttcap%
\pgfsetroundjoin%
\definecolor{currentfill}{rgb}{0.705673,0.015556,0.150233}%
\pgfsetfillcolor{currentfill}%
\pgfsetlinewidth{0.000000pt}%
\definecolor{currentstroke}{rgb}{0.000000,0.000000,0.000000}%
\pgfsetstrokecolor{currentstroke}%
\pgfsetdash{}{0pt}%
\pgfpathmoveto{\pgfqpoint{4.268228in}{2.632966in}}%
\pgfpathlineto{\pgfqpoint{4.511715in}{2.762029in}}%
\pgfpathlineto{\pgfqpoint{4.268228in}{2.632966in}}%
\pgfpathclose%
\pgfusepath{fill}%
\end{pgfscope}%
\begin{pgfscope}%
\pgfpathrectangle{\pgfqpoint{0.017500in}{0.017500in}}{\pgfqpoint{6.218282in}{3.580372in}} %
\pgfusepath{clip}%
\pgfsetbuttcap%
\pgfsetroundjoin%
\definecolor{currentfill}{rgb}{0.705673,0.015556,0.150233}%
\pgfsetfillcolor{currentfill}%
\pgfsetlinewidth{0.000000pt}%
\definecolor{currentstroke}{rgb}{0.000000,0.000000,0.000000}%
\pgfsetstrokecolor{currentstroke}%
\pgfsetdash{}{0pt}%
\pgfpathmoveto{\pgfqpoint{4.268228in}{2.632966in}}%
\pgfpathlineto{\pgfqpoint{4.511715in}{2.762029in}}%
\pgfpathlineto{\pgfqpoint{4.268228in}{2.632966in}}%
\pgfpathclose%
\pgfusepath{fill}%
\end{pgfscope}%
\begin{pgfscope}%
\pgfpathrectangle{\pgfqpoint{0.017500in}{0.017500in}}{\pgfqpoint{6.218282in}{3.580372in}} %
\pgfusepath{clip}%
\pgfsetbuttcap%
\pgfsetroundjoin%
\definecolor{currentfill}{rgb}{0.705673,0.015556,0.150233}%
\pgfsetfillcolor{currentfill}%
\pgfsetlinewidth{0.000000pt}%
\definecolor{currentstroke}{rgb}{0.000000,0.000000,0.000000}%
\pgfsetstrokecolor{currentstroke}%
\pgfsetdash{}{0pt}%
\pgfpathmoveto{\pgfqpoint{4.268228in}{2.632966in}}%
\pgfpathlineto{\pgfqpoint{4.511715in}{2.762029in}}%
\pgfpathlineto{\pgfqpoint{4.268228in}{2.632966in}}%
\pgfpathclose%
\pgfusepath{fill}%
\end{pgfscope}%
\begin{pgfscope}%
\pgfpathrectangle{\pgfqpoint{0.017500in}{0.017500in}}{\pgfqpoint{6.218282in}{3.580372in}} %
\pgfusepath{clip}%
\pgfsetbuttcap%
\pgfsetroundjoin%
\definecolor{currentfill}{rgb}{0.705673,0.015556,0.150233}%
\pgfsetfillcolor{currentfill}%
\pgfsetlinewidth{0.000000pt}%
\definecolor{currentstroke}{rgb}{0.000000,0.000000,0.000000}%
\pgfsetstrokecolor{currentstroke}%
\pgfsetdash{}{0pt}%
\pgfpathmoveto{\pgfqpoint{4.268228in}{2.632966in}}%
\pgfpathlineto{\pgfqpoint{4.511715in}{2.762029in}}%
\pgfpathlineto{\pgfqpoint{4.268228in}{2.632966in}}%
\pgfpathclose%
\pgfusepath{fill}%
\end{pgfscope}%
\begin{pgfscope}%
\pgfpathrectangle{\pgfqpoint{0.017500in}{0.017500in}}{\pgfqpoint{6.218282in}{3.580372in}} %
\pgfusepath{clip}%
\pgfsetbuttcap%
\pgfsetroundjoin%
\definecolor{currentfill}{rgb}{0.705673,0.015556,0.150233}%
\pgfsetfillcolor{currentfill}%
\pgfsetlinewidth{0.000000pt}%
\definecolor{currentstroke}{rgb}{0.000000,0.000000,0.000000}%
\pgfsetstrokecolor{currentstroke}%
\pgfsetdash{}{0pt}%
\pgfpathmoveto{\pgfqpoint{4.268228in}{2.632966in}}%
\pgfpathlineto{\pgfqpoint{4.511715in}{2.762029in}}%
\pgfpathlineto{\pgfqpoint{4.268228in}{2.632966in}}%
\pgfpathclose%
\pgfusepath{fill}%
\end{pgfscope}%
\begin{pgfscope}%
\pgfpathrectangle{\pgfqpoint{0.017500in}{0.017500in}}{\pgfqpoint{6.218282in}{3.580372in}} %
\pgfusepath{clip}%
\pgfsetbuttcap%
\pgfsetroundjoin%
\definecolor{currentfill}{rgb}{0.705673,0.015556,0.150233}%
\pgfsetfillcolor{currentfill}%
\pgfsetlinewidth{0.000000pt}%
\definecolor{currentstroke}{rgb}{0.000000,0.000000,0.000000}%
\pgfsetstrokecolor{currentstroke}%
\pgfsetdash{}{0pt}%
\pgfpathmoveto{\pgfqpoint{4.268228in}{2.632966in}}%
\pgfpathlineto{\pgfqpoint{4.511715in}{2.762029in}}%
\pgfpathlineto{\pgfqpoint{4.268228in}{2.632966in}}%
\pgfpathclose%
\pgfusepath{fill}%
\end{pgfscope}%
\begin{pgfscope}%
\pgfpathrectangle{\pgfqpoint{0.017500in}{0.017500in}}{\pgfqpoint{6.218282in}{3.580372in}} %
\pgfusepath{clip}%
\pgfsetbuttcap%
\pgfsetroundjoin%
\definecolor{currentfill}{rgb}{0.705673,0.015556,0.150233}%
\pgfsetfillcolor{currentfill}%
\pgfsetlinewidth{0.000000pt}%
\definecolor{currentstroke}{rgb}{0.000000,0.000000,0.000000}%
\pgfsetstrokecolor{currentstroke}%
\pgfsetdash{}{0pt}%
\pgfpathmoveto{\pgfqpoint{4.268228in}{2.632966in}}%
\pgfpathlineto{\pgfqpoint{4.511715in}{2.762029in}}%
\pgfpathlineto{\pgfqpoint{4.268228in}{2.632966in}}%
\pgfpathclose%
\pgfusepath{fill}%
\end{pgfscope}%
\begin{pgfscope}%
\pgfpathrectangle{\pgfqpoint{0.017500in}{0.017500in}}{\pgfqpoint{6.218282in}{3.580372in}} %
\pgfusepath{clip}%
\pgfsetbuttcap%
\pgfsetroundjoin%
\definecolor{currentfill}{rgb}{0.705673,0.015556,0.150233}%
\pgfsetfillcolor{currentfill}%
\pgfsetlinewidth{0.000000pt}%
\definecolor{currentstroke}{rgb}{0.000000,0.000000,0.000000}%
\pgfsetstrokecolor{currentstroke}%
\pgfsetdash{}{0pt}%
\pgfpathmoveto{\pgfqpoint{4.268228in}{2.632966in}}%
\pgfpathlineto{\pgfqpoint{4.511715in}{2.762029in}}%
\pgfpathlineto{\pgfqpoint{4.268228in}{2.632966in}}%
\pgfpathclose%
\pgfusepath{fill}%
\end{pgfscope}%
\begin{pgfscope}%
\pgfpathrectangle{\pgfqpoint{0.017500in}{0.017500in}}{\pgfqpoint{6.218282in}{3.580372in}} %
\pgfusepath{clip}%
\pgfsetbuttcap%
\pgfsetroundjoin%
\definecolor{currentfill}{rgb}{0.705673,0.015556,0.150233}%
\pgfsetfillcolor{currentfill}%
\pgfsetlinewidth{0.000000pt}%
\definecolor{currentstroke}{rgb}{0.000000,0.000000,0.000000}%
\pgfsetstrokecolor{currentstroke}%
\pgfsetdash{}{0pt}%
\pgfpathmoveto{\pgfqpoint{4.268228in}{2.632966in}}%
\pgfpathlineto{\pgfqpoint{4.511715in}{2.762029in}}%
\pgfpathlineto{\pgfqpoint{4.268228in}{2.632966in}}%
\pgfpathclose%
\pgfusepath{fill}%
\end{pgfscope}%
\begin{pgfscope}%
\pgfpathrectangle{\pgfqpoint{0.017500in}{0.017500in}}{\pgfqpoint{6.218282in}{3.580372in}} %
\pgfusepath{clip}%
\pgfsetbuttcap%
\pgfsetroundjoin%
\definecolor{currentfill}{rgb}{0.705673,0.015556,0.150233}%
\pgfsetfillcolor{currentfill}%
\pgfsetlinewidth{0.000000pt}%
\definecolor{currentstroke}{rgb}{0.000000,0.000000,0.000000}%
\pgfsetstrokecolor{currentstroke}%
\pgfsetdash{}{0pt}%
\pgfpathmoveto{\pgfqpoint{4.268228in}{2.632966in}}%
\pgfpathlineto{\pgfqpoint{4.511715in}{2.762029in}}%
\pgfpathlineto{\pgfqpoint{4.268228in}{2.632966in}}%
\pgfpathclose%
\pgfusepath{fill}%
\end{pgfscope}%
\begin{pgfscope}%
\pgfpathrectangle{\pgfqpoint{0.017500in}{0.017500in}}{\pgfqpoint{6.218282in}{3.580372in}} %
\pgfusepath{clip}%
\pgfsetbuttcap%
\pgfsetroundjoin%
\definecolor{currentfill}{rgb}{0.705673,0.015556,0.150233}%
\pgfsetfillcolor{currentfill}%
\pgfsetlinewidth{0.000000pt}%
\definecolor{currentstroke}{rgb}{0.000000,0.000000,0.000000}%
\pgfsetstrokecolor{currentstroke}%
\pgfsetdash{}{0pt}%
\pgfpathmoveto{\pgfqpoint{4.268228in}{2.632966in}}%
\pgfpathlineto{\pgfqpoint{4.511715in}{2.762029in}}%
\pgfpathlineto{\pgfqpoint{4.268228in}{2.632966in}}%
\pgfpathclose%
\pgfusepath{fill}%
\end{pgfscope}%
\begin{pgfscope}%
\pgfpathrectangle{\pgfqpoint{0.017500in}{0.017500in}}{\pgfqpoint{6.218282in}{3.580372in}} %
\pgfusepath{clip}%
\pgfsetbuttcap%
\pgfsetroundjoin%
\definecolor{currentfill}{rgb}{0.705673,0.015556,0.150233}%
\pgfsetfillcolor{currentfill}%
\pgfsetlinewidth{0.000000pt}%
\definecolor{currentstroke}{rgb}{0.000000,0.000000,0.000000}%
\pgfsetstrokecolor{currentstroke}%
\pgfsetdash{}{0pt}%
\pgfpathmoveto{\pgfqpoint{4.268228in}{2.632966in}}%
\pgfpathlineto{\pgfqpoint{4.511715in}{2.762029in}}%
\pgfpathlineto{\pgfqpoint{4.268228in}{2.632966in}}%
\pgfpathclose%
\pgfusepath{fill}%
\end{pgfscope}%
\begin{pgfscope}%
\pgfpathrectangle{\pgfqpoint{0.017500in}{0.017500in}}{\pgfqpoint{6.218282in}{3.580372in}} %
\pgfusepath{clip}%
\pgfsetbuttcap%
\pgfsetroundjoin%
\definecolor{currentfill}{rgb}{0.705673,0.015556,0.150233}%
\pgfsetfillcolor{currentfill}%
\pgfsetlinewidth{0.000000pt}%
\definecolor{currentstroke}{rgb}{0.000000,0.000000,0.000000}%
\pgfsetstrokecolor{currentstroke}%
\pgfsetdash{}{0pt}%
\pgfpathmoveto{\pgfqpoint{4.268228in}{2.632966in}}%
\pgfpathlineto{\pgfqpoint{4.511715in}{2.762029in}}%
\pgfpathlineto{\pgfqpoint{4.268228in}{2.632966in}}%
\pgfpathclose%
\pgfusepath{fill}%
\end{pgfscope}%
\begin{pgfscope}%
\pgfpathrectangle{\pgfqpoint{0.017500in}{0.017500in}}{\pgfqpoint{6.218282in}{3.580372in}} %
\pgfusepath{clip}%
\pgfsetbuttcap%
\pgfsetroundjoin%
\definecolor{currentfill}{rgb}{0.705673,0.015556,0.150233}%
\pgfsetfillcolor{currentfill}%
\pgfsetlinewidth{0.000000pt}%
\definecolor{currentstroke}{rgb}{0.000000,0.000000,0.000000}%
\pgfsetstrokecolor{currentstroke}%
\pgfsetdash{}{0pt}%
\pgfpathmoveto{\pgfqpoint{4.268228in}{2.632966in}}%
\pgfpathlineto{\pgfqpoint{4.511715in}{2.762029in}}%
\pgfpathlineto{\pgfqpoint{4.268228in}{2.632966in}}%
\pgfpathclose%
\pgfusepath{fill}%
\end{pgfscope}%
\begin{pgfscope}%
\pgfpathrectangle{\pgfqpoint{0.017500in}{0.017500in}}{\pgfqpoint{6.218282in}{3.580372in}} %
\pgfusepath{clip}%
\pgfsetbuttcap%
\pgfsetroundjoin%
\definecolor{currentfill}{rgb}{0.705673,0.015556,0.150233}%
\pgfsetfillcolor{currentfill}%
\pgfsetlinewidth{0.000000pt}%
\definecolor{currentstroke}{rgb}{0.000000,0.000000,0.000000}%
\pgfsetstrokecolor{currentstroke}%
\pgfsetdash{}{0pt}%
\pgfpathmoveto{\pgfqpoint{4.268228in}{2.632966in}}%
\pgfpathlineto{\pgfqpoint{4.511715in}{2.762029in}}%
\pgfpathlineto{\pgfqpoint{4.268228in}{2.632966in}}%
\pgfpathclose%
\pgfusepath{fill}%
\end{pgfscope}%
\begin{pgfscope}%
\pgfpathrectangle{\pgfqpoint{0.017500in}{0.017500in}}{\pgfqpoint{6.218282in}{3.580372in}} %
\pgfusepath{clip}%
\pgfsetbuttcap%
\pgfsetroundjoin%
\definecolor{currentfill}{rgb}{0.705673,0.015556,0.150233}%
\pgfsetfillcolor{currentfill}%
\pgfsetlinewidth{0.000000pt}%
\definecolor{currentstroke}{rgb}{0.000000,0.000000,0.000000}%
\pgfsetstrokecolor{currentstroke}%
\pgfsetdash{}{0pt}%
\pgfpathmoveto{\pgfqpoint{4.268228in}{2.632966in}}%
\pgfpathlineto{\pgfqpoint{4.511715in}{2.762029in}}%
\pgfpathlineto{\pgfqpoint{4.268228in}{2.632966in}}%
\pgfpathclose%
\pgfusepath{fill}%
\end{pgfscope}%
\begin{pgfscope}%
\pgfpathrectangle{\pgfqpoint{0.017500in}{0.017500in}}{\pgfqpoint{6.218282in}{3.580372in}} %
\pgfusepath{clip}%
\pgfsetbuttcap%
\pgfsetroundjoin%
\definecolor{currentfill}{rgb}{0.705673,0.015556,0.150233}%
\pgfsetfillcolor{currentfill}%
\pgfsetlinewidth{0.000000pt}%
\definecolor{currentstroke}{rgb}{0.000000,0.000000,0.000000}%
\pgfsetstrokecolor{currentstroke}%
\pgfsetdash{}{0pt}%
\pgfpathmoveto{\pgfqpoint{4.268228in}{2.632966in}}%
\pgfpathlineto{\pgfqpoint{4.511715in}{2.762029in}}%
\pgfpathlineto{\pgfqpoint{4.268228in}{2.632966in}}%
\pgfpathclose%
\pgfusepath{fill}%
\end{pgfscope}%
\begin{pgfscope}%
\pgfpathrectangle{\pgfqpoint{0.017500in}{0.017500in}}{\pgfqpoint{6.218282in}{3.580372in}} %
\pgfusepath{clip}%
\pgfsetbuttcap%
\pgfsetroundjoin%
\definecolor{currentfill}{rgb}{0.705673,0.015556,0.150233}%
\pgfsetfillcolor{currentfill}%
\pgfsetlinewidth{0.000000pt}%
\definecolor{currentstroke}{rgb}{0.000000,0.000000,0.000000}%
\pgfsetstrokecolor{currentstroke}%
\pgfsetdash{}{0pt}%
\pgfpathmoveto{\pgfqpoint{4.268228in}{2.632966in}}%
\pgfpathlineto{\pgfqpoint{4.511715in}{2.762029in}}%
\pgfpathlineto{\pgfqpoint{4.268228in}{2.632966in}}%
\pgfpathclose%
\pgfusepath{fill}%
\end{pgfscope}%
\begin{pgfscope}%
\pgfpathrectangle{\pgfqpoint{0.017500in}{0.017500in}}{\pgfqpoint{6.218282in}{3.580372in}} %
\pgfusepath{clip}%
\pgfsetbuttcap%
\pgfsetroundjoin%
\definecolor{currentfill}{rgb}{0.705673,0.015556,0.150233}%
\pgfsetfillcolor{currentfill}%
\pgfsetlinewidth{0.000000pt}%
\definecolor{currentstroke}{rgb}{0.000000,0.000000,0.000000}%
\pgfsetstrokecolor{currentstroke}%
\pgfsetdash{}{0pt}%
\pgfpathmoveto{\pgfqpoint{4.268228in}{2.632966in}}%
\pgfpathlineto{\pgfqpoint{4.511715in}{2.762029in}}%
\pgfpathlineto{\pgfqpoint{4.268228in}{2.632966in}}%
\pgfpathclose%
\pgfusepath{fill}%
\end{pgfscope}%
\begin{pgfscope}%
\pgfpathrectangle{\pgfqpoint{0.017500in}{0.017500in}}{\pgfqpoint{6.218282in}{3.580372in}} %
\pgfusepath{clip}%
\pgfsetbuttcap%
\pgfsetroundjoin%
\definecolor{currentfill}{rgb}{0.338377,0.452819,0.879317}%
\pgfsetfillcolor{currentfill}%
\pgfsetlinewidth{0.000000pt}%
\definecolor{currentstroke}{rgb}{0.000000,0.000000,0.000000}%
\pgfsetstrokecolor{currentstroke}%
\pgfsetdash{}{0pt}%
\pgfpathmoveto{\pgfqpoint{1.915196in}{0.958373in}}%
\pgfpathlineto{\pgfqpoint{2.349703in}{0.878379in}}%
\pgfpathlineto{\pgfqpoint{2.590029in}{2.232729in}}%
\pgfpathlineto{\pgfqpoint{2.142608in}{2.314596in}}%
\pgfpathlineto{\pgfqpoint{1.915196in}{0.958373in}}%
\pgfpathclose%
\pgfusepath{fill}%
\end{pgfscope}%
\begin{pgfscope}%
\pgfpathrectangle{\pgfqpoint{0.017500in}{0.017500in}}{\pgfqpoint{6.218282in}{3.580372in}} %
\pgfusepath{clip}%
\pgfsetbuttcap%
\pgfsetroundjoin%
\definecolor{currentfill}{rgb}{0.338377,0.452819,0.879317}%
\pgfsetfillcolor{currentfill}%
\pgfsetlinewidth{0.000000pt}%
\definecolor{currentstroke}{rgb}{0.000000,0.000000,0.000000}%
\pgfsetstrokecolor{currentstroke}%
\pgfsetdash{}{0pt}%
\pgfpathmoveto{\pgfqpoint{1.915196in}{0.958373in}}%
\pgfpathlineto{\pgfqpoint{2.349703in}{0.878379in}}%
\pgfpathlineto{\pgfqpoint{2.590029in}{2.232729in}}%
\pgfpathlineto{\pgfqpoint{2.142608in}{2.314596in}}%
\pgfpathlineto{\pgfqpoint{1.915196in}{0.958373in}}%
\pgfpathclose%
\pgfusepath{fill}%
\end{pgfscope}%
\begin{pgfscope}%
\pgfpathrectangle{\pgfqpoint{0.017500in}{0.017500in}}{\pgfqpoint{6.218282in}{3.580372in}} %
\pgfusepath{clip}%
\pgfsetbuttcap%
\pgfsetroundjoin%
\definecolor{currentfill}{rgb}{0.338377,0.452819,0.879317}%
\pgfsetfillcolor{currentfill}%
\pgfsetlinewidth{0.000000pt}%
\definecolor{currentstroke}{rgb}{0.000000,0.000000,0.000000}%
\pgfsetstrokecolor{currentstroke}%
\pgfsetdash{}{0pt}%
\pgfpathmoveto{\pgfqpoint{1.915196in}{0.958373in}}%
\pgfpathlineto{\pgfqpoint{2.349703in}{0.878379in}}%
\pgfpathlineto{\pgfqpoint{2.590029in}{2.232729in}}%
\pgfpathlineto{\pgfqpoint{2.142608in}{2.314596in}}%
\pgfpathlineto{\pgfqpoint{1.915196in}{0.958373in}}%
\pgfpathclose%
\pgfusepath{fill}%
\end{pgfscope}%
\begin{pgfscope}%
\pgfpathrectangle{\pgfqpoint{0.017500in}{0.017500in}}{\pgfqpoint{6.218282in}{3.580372in}} %
\pgfusepath{clip}%
\pgfsetbuttcap%
\pgfsetroundjoin%
\definecolor{currentfill}{rgb}{0.338377,0.452819,0.879317}%
\pgfsetfillcolor{currentfill}%
\pgfsetlinewidth{0.000000pt}%
\definecolor{currentstroke}{rgb}{0.000000,0.000000,0.000000}%
\pgfsetstrokecolor{currentstroke}%
\pgfsetdash{}{0pt}%
\pgfpathmoveto{\pgfqpoint{1.915196in}{0.958373in}}%
\pgfpathlineto{\pgfqpoint{2.349703in}{0.878379in}}%
\pgfpathlineto{\pgfqpoint{2.590029in}{2.232729in}}%
\pgfpathlineto{\pgfqpoint{2.142608in}{2.314596in}}%
\pgfpathlineto{\pgfqpoint{1.915196in}{0.958373in}}%
\pgfpathclose%
\pgfusepath{fill}%
\end{pgfscope}%
\begin{pgfscope}%
\pgfpathrectangle{\pgfqpoint{0.017500in}{0.017500in}}{\pgfqpoint{6.218282in}{3.580372in}} %
\pgfusepath{clip}%
\pgfsetbuttcap%
\pgfsetroundjoin%
\definecolor{currentfill}{rgb}{0.338377,0.452819,0.879317}%
\pgfsetfillcolor{currentfill}%
\pgfsetlinewidth{0.000000pt}%
\definecolor{currentstroke}{rgb}{0.000000,0.000000,0.000000}%
\pgfsetstrokecolor{currentstroke}%
\pgfsetdash{}{0pt}%
\pgfpathmoveto{\pgfqpoint{1.915196in}{0.958373in}}%
\pgfpathlineto{\pgfqpoint{2.349703in}{0.878379in}}%
\pgfpathlineto{\pgfqpoint{2.590029in}{2.232729in}}%
\pgfpathlineto{\pgfqpoint{2.142608in}{2.314596in}}%
\pgfpathlineto{\pgfqpoint{1.915196in}{0.958373in}}%
\pgfpathclose%
\pgfusepath{fill}%
\end{pgfscope}%
\begin{pgfscope}%
\pgfpathrectangle{\pgfqpoint{0.017500in}{0.017500in}}{\pgfqpoint{6.218282in}{3.580372in}} %
\pgfusepath{clip}%
\pgfsetbuttcap%
\pgfsetroundjoin%
\definecolor{currentfill}{rgb}{0.338377,0.452819,0.879317}%
\pgfsetfillcolor{currentfill}%
\pgfsetlinewidth{0.000000pt}%
\definecolor{currentstroke}{rgb}{0.000000,0.000000,0.000000}%
\pgfsetstrokecolor{currentstroke}%
\pgfsetdash{}{0pt}%
\pgfpathmoveto{\pgfqpoint{1.915196in}{0.958373in}}%
\pgfpathlineto{\pgfqpoint{2.349703in}{0.878379in}}%
\pgfpathlineto{\pgfqpoint{2.590029in}{2.232729in}}%
\pgfpathlineto{\pgfqpoint{2.142608in}{2.314596in}}%
\pgfpathlineto{\pgfqpoint{1.915196in}{0.958373in}}%
\pgfpathclose%
\pgfusepath{fill}%
\end{pgfscope}%
\begin{pgfscope}%
\pgfpathrectangle{\pgfqpoint{0.017500in}{0.017500in}}{\pgfqpoint{6.218282in}{3.580372in}} %
\pgfusepath{clip}%
\pgfsetbuttcap%
\pgfsetroundjoin%
\definecolor{currentfill}{rgb}{0.785153,0.220851,0.211673}%
\pgfsetfillcolor{currentfill}%
\pgfsetlinewidth{0.000000pt}%
\definecolor{currentstroke}{rgb}{0.000000,0.000000,0.000000}%
\pgfsetstrokecolor{currentstroke}%
\pgfsetdash{}{0pt}%
\pgfpathmoveto{\pgfqpoint{2.410172in}{2.537974in}}%
\pgfpathlineto{\pgfqpoint{2.674281in}{2.686821in}}%
\pgfpathlineto{\pgfqpoint{2.410172in}{2.537974in}}%
\pgfpathclose%
\pgfusepath{fill}%
\end{pgfscope}%
\begin{pgfscope}%
\pgfpathrectangle{\pgfqpoint{0.017500in}{0.017500in}}{\pgfqpoint{6.218282in}{3.580372in}} %
\pgfusepath{clip}%
\pgfsetbuttcap%
\pgfsetroundjoin%
\definecolor{currentfill}{rgb}{0.785153,0.220851,0.211673}%
\pgfsetfillcolor{currentfill}%
\pgfsetlinewidth{0.000000pt}%
\definecolor{currentstroke}{rgb}{0.000000,0.000000,0.000000}%
\pgfsetstrokecolor{currentstroke}%
\pgfsetdash{}{0pt}%
\pgfpathmoveto{\pgfqpoint{2.410172in}{2.537974in}}%
\pgfpathlineto{\pgfqpoint{2.674281in}{2.686821in}}%
\pgfpathlineto{\pgfqpoint{2.410172in}{2.537974in}}%
\pgfpathclose%
\pgfusepath{fill}%
\end{pgfscope}%
\begin{pgfscope}%
\pgfpathrectangle{\pgfqpoint{0.017500in}{0.017500in}}{\pgfqpoint{6.218282in}{3.580372in}} %
\pgfusepath{clip}%
\pgfsetbuttcap%
\pgfsetroundjoin%
\definecolor{currentfill}{rgb}{0.785153,0.220851,0.211673}%
\pgfsetfillcolor{currentfill}%
\pgfsetlinewidth{0.000000pt}%
\definecolor{currentstroke}{rgb}{0.000000,0.000000,0.000000}%
\pgfsetstrokecolor{currentstroke}%
\pgfsetdash{}{0pt}%
\pgfpathmoveto{\pgfqpoint{2.410172in}{2.537974in}}%
\pgfpathlineto{\pgfqpoint{2.674281in}{2.686821in}}%
\pgfpathlineto{\pgfqpoint{2.410172in}{2.537974in}}%
\pgfpathclose%
\pgfusepath{fill}%
\end{pgfscope}%
\begin{pgfscope}%
\pgfpathrectangle{\pgfqpoint{0.017500in}{0.017500in}}{\pgfqpoint{6.218282in}{3.580372in}} %
\pgfusepath{clip}%
\pgfsetbuttcap%
\pgfsetroundjoin%
\definecolor{currentfill}{rgb}{0.785153,0.220851,0.211673}%
\pgfsetfillcolor{currentfill}%
\pgfsetlinewidth{0.000000pt}%
\definecolor{currentstroke}{rgb}{0.000000,0.000000,0.000000}%
\pgfsetstrokecolor{currentstroke}%
\pgfsetdash{}{0pt}%
\pgfpathmoveto{\pgfqpoint{2.410172in}{2.537974in}}%
\pgfpathlineto{\pgfqpoint{2.674281in}{2.686821in}}%
\pgfpathlineto{\pgfqpoint{2.410172in}{2.537974in}}%
\pgfpathclose%
\pgfusepath{fill}%
\end{pgfscope}%
\begin{pgfscope}%
\pgfpathrectangle{\pgfqpoint{0.017500in}{0.017500in}}{\pgfqpoint{6.218282in}{3.580372in}} %
\pgfusepath{clip}%
\pgfsetbuttcap%
\pgfsetroundjoin%
\definecolor{currentfill}{rgb}{0.785153,0.220851,0.211673}%
\pgfsetfillcolor{currentfill}%
\pgfsetlinewidth{0.000000pt}%
\definecolor{currentstroke}{rgb}{0.000000,0.000000,0.000000}%
\pgfsetstrokecolor{currentstroke}%
\pgfsetdash{}{0pt}%
\pgfpathmoveto{\pgfqpoint{2.410172in}{2.537974in}}%
\pgfpathlineto{\pgfqpoint{2.674281in}{2.686821in}}%
\pgfpathlineto{\pgfqpoint{2.410172in}{2.537974in}}%
\pgfpathclose%
\pgfusepath{fill}%
\end{pgfscope}%
\begin{pgfscope}%
\pgfpathrectangle{\pgfqpoint{0.017500in}{0.017500in}}{\pgfqpoint{6.218282in}{3.580372in}} %
\pgfusepath{clip}%
\pgfsetbuttcap%
\pgfsetroundjoin%
\definecolor{currentfill}{rgb}{0.785153,0.220851,0.211673}%
\pgfsetfillcolor{currentfill}%
\pgfsetlinewidth{0.000000pt}%
\definecolor{currentstroke}{rgb}{0.000000,0.000000,0.000000}%
\pgfsetstrokecolor{currentstroke}%
\pgfsetdash{}{0pt}%
\pgfpathmoveto{\pgfqpoint{2.410172in}{2.537974in}}%
\pgfpathlineto{\pgfqpoint{2.674281in}{2.686821in}}%
\pgfpathlineto{\pgfqpoint{2.410172in}{2.537974in}}%
\pgfpathclose%
\pgfusepath{fill}%
\end{pgfscope}%
\begin{pgfscope}%
\pgfpathrectangle{\pgfqpoint{0.017500in}{0.017500in}}{\pgfqpoint{6.218282in}{3.580372in}} %
\pgfusepath{clip}%
\pgfsetbuttcap%
\pgfsetroundjoin%
\definecolor{currentfill}{rgb}{0.785153,0.220851,0.211673}%
\pgfsetfillcolor{currentfill}%
\pgfsetlinewidth{0.000000pt}%
\definecolor{currentstroke}{rgb}{0.000000,0.000000,0.000000}%
\pgfsetstrokecolor{currentstroke}%
\pgfsetdash{}{0pt}%
\pgfpathmoveto{\pgfqpoint{2.410172in}{2.537974in}}%
\pgfpathlineto{\pgfqpoint{2.674281in}{2.686821in}}%
\pgfpathlineto{\pgfqpoint{2.410172in}{2.537974in}}%
\pgfpathclose%
\pgfusepath{fill}%
\end{pgfscope}%
\begin{pgfscope}%
\pgfpathrectangle{\pgfqpoint{0.017500in}{0.017500in}}{\pgfqpoint{6.218282in}{3.580372in}} %
\pgfusepath{clip}%
\pgfsetbuttcap%
\pgfsetroundjoin%
\definecolor{currentfill}{rgb}{0.785153,0.220851,0.211673}%
\pgfsetfillcolor{currentfill}%
\pgfsetlinewidth{0.000000pt}%
\definecolor{currentstroke}{rgb}{0.000000,0.000000,0.000000}%
\pgfsetstrokecolor{currentstroke}%
\pgfsetdash{}{0pt}%
\pgfpathmoveto{\pgfqpoint{2.410172in}{2.537974in}}%
\pgfpathlineto{\pgfqpoint{2.674281in}{2.686821in}}%
\pgfpathlineto{\pgfqpoint{2.410172in}{2.537974in}}%
\pgfpathclose%
\pgfusepath{fill}%
\end{pgfscope}%
\begin{pgfscope}%
\pgfpathrectangle{\pgfqpoint{0.017500in}{0.017500in}}{\pgfqpoint{6.218282in}{3.580372in}} %
\pgfusepath{clip}%
\pgfsetbuttcap%
\pgfsetroundjoin%
\definecolor{currentfill}{rgb}{0.785153,0.220851,0.211673}%
\pgfsetfillcolor{currentfill}%
\pgfsetlinewidth{0.000000pt}%
\definecolor{currentstroke}{rgb}{0.000000,0.000000,0.000000}%
\pgfsetstrokecolor{currentstroke}%
\pgfsetdash{}{0pt}%
\pgfpathmoveto{\pgfqpoint{2.410172in}{2.537974in}}%
\pgfpathlineto{\pgfqpoint{2.674281in}{2.686821in}}%
\pgfpathlineto{\pgfqpoint{2.410172in}{2.537974in}}%
\pgfpathclose%
\pgfusepath{fill}%
\end{pgfscope}%
\begin{pgfscope}%
\pgfpathrectangle{\pgfqpoint{0.017500in}{0.017500in}}{\pgfqpoint{6.218282in}{3.580372in}} %
\pgfusepath{clip}%
\pgfsetbuttcap%
\pgfsetroundjoin%
\definecolor{currentfill}{rgb}{0.785153,0.220851,0.211673}%
\pgfsetfillcolor{currentfill}%
\pgfsetlinewidth{0.000000pt}%
\definecolor{currentstroke}{rgb}{0.000000,0.000000,0.000000}%
\pgfsetstrokecolor{currentstroke}%
\pgfsetdash{}{0pt}%
\pgfpathmoveto{\pgfqpoint{2.410172in}{2.537974in}}%
\pgfpathlineto{\pgfqpoint{2.674281in}{2.686821in}}%
\pgfpathlineto{\pgfqpoint{2.410172in}{2.537974in}}%
\pgfpathclose%
\pgfusepath{fill}%
\end{pgfscope}%
\begin{pgfscope}%
\pgfpathrectangle{\pgfqpoint{0.017500in}{0.017500in}}{\pgfqpoint{6.218282in}{3.580372in}} %
\pgfusepath{clip}%
\pgfsetbuttcap%
\pgfsetroundjoin%
\definecolor{currentfill}{rgb}{0.785153,0.220851,0.211673}%
\pgfsetfillcolor{currentfill}%
\pgfsetlinewidth{0.000000pt}%
\definecolor{currentstroke}{rgb}{0.000000,0.000000,0.000000}%
\pgfsetstrokecolor{currentstroke}%
\pgfsetdash{}{0pt}%
\pgfpathmoveto{\pgfqpoint{2.410172in}{2.537974in}}%
\pgfpathlineto{\pgfqpoint{2.674281in}{2.686821in}}%
\pgfpathlineto{\pgfqpoint{2.410172in}{2.537974in}}%
\pgfpathclose%
\pgfusepath{fill}%
\end{pgfscope}%
\begin{pgfscope}%
\pgfpathrectangle{\pgfqpoint{0.017500in}{0.017500in}}{\pgfqpoint{6.218282in}{3.580372in}} %
\pgfusepath{clip}%
\pgfsetbuttcap%
\pgfsetroundjoin%
\definecolor{currentfill}{rgb}{0.785153,0.220851,0.211673}%
\pgfsetfillcolor{currentfill}%
\pgfsetlinewidth{0.000000pt}%
\definecolor{currentstroke}{rgb}{0.000000,0.000000,0.000000}%
\pgfsetstrokecolor{currentstroke}%
\pgfsetdash{}{0pt}%
\pgfpathmoveto{\pgfqpoint{2.410172in}{2.537974in}}%
\pgfpathlineto{\pgfqpoint{2.674281in}{2.686821in}}%
\pgfpathlineto{\pgfqpoint{2.410172in}{2.537974in}}%
\pgfpathclose%
\pgfusepath{fill}%
\end{pgfscope}%
\begin{pgfscope}%
\pgfpathrectangle{\pgfqpoint{0.017500in}{0.017500in}}{\pgfqpoint{6.218282in}{3.580372in}} %
\pgfusepath{clip}%
\pgfsetbuttcap%
\pgfsetroundjoin%
\definecolor{currentfill}{rgb}{0.785153,0.220851,0.211673}%
\pgfsetfillcolor{currentfill}%
\pgfsetlinewidth{0.000000pt}%
\definecolor{currentstroke}{rgb}{0.000000,0.000000,0.000000}%
\pgfsetstrokecolor{currentstroke}%
\pgfsetdash{}{0pt}%
\pgfpathmoveto{\pgfqpoint{2.410172in}{2.537974in}}%
\pgfpathlineto{\pgfqpoint{2.674281in}{2.686821in}}%
\pgfpathlineto{\pgfqpoint{2.410172in}{2.537974in}}%
\pgfpathclose%
\pgfusepath{fill}%
\end{pgfscope}%
\begin{pgfscope}%
\pgfpathrectangle{\pgfqpoint{0.017500in}{0.017500in}}{\pgfqpoint{6.218282in}{3.580372in}} %
\pgfusepath{clip}%
\pgfsetbuttcap%
\pgfsetroundjoin%
\definecolor{currentfill}{rgb}{0.785153,0.220851,0.211673}%
\pgfsetfillcolor{currentfill}%
\pgfsetlinewidth{0.000000pt}%
\definecolor{currentstroke}{rgb}{0.000000,0.000000,0.000000}%
\pgfsetstrokecolor{currentstroke}%
\pgfsetdash{}{0pt}%
\pgfpathmoveto{\pgfqpoint{2.410172in}{2.537974in}}%
\pgfpathlineto{\pgfqpoint{2.674281in}{2.686821in}}%
\pgfpathlineto{\pgfqpoint{2.410172in}{2.537974in}}%
\pgfpathclose%
\pgfusepath{fill}%
\end{pgfscope}%
\begin{pgfscope}%
\pgfpathrectangle{\pgfqpoint{0.017500in}{0.017500in}}{\pgfqpoint{6.218282in}{3.580372in}} %
\pgfusepath{clip}%
\pgfsetbuttcap%
\pgfsetroundjoin%
\definecolor{currentfill}{rgb}{0.785153,0.220851,0.211673}%
\pgfsetfillcolor{currentfill}%
\pgfsetlinewidth{0.000000pt}%
\definecolor{currentstroke}{rgb}{0.000000,0.000000,0.000000}%
\pgfsetstrokecolor{currentstroke}%
\pgfsetdash{}{0pt}%
\pgfpathmoveto{\pgfqpoint{2.410172in}{2.537974in}}%
\pgfpathlineto{\pgfqpoint{2.674281in}{2.686821in}}%
\pgfpathlineto{\pgfqpoint{2.410172in}{2.537974in}}%
\pgfpathclose%
\pgfusepath{fill}%
\end{pgfscope}%
\begin{pgfscope}%
\pgfpathrectangle{\pgfqpoint{0.017500in}{0.017500in}}{\pgfqpoint{6.218282in}{3.580372in}} %
\pgfusepath{clip}%
\pgfsetbuttcap%
\pgfsetroundjoin%
\definecolor{currentfill}{rgb}{0.785153,0.220851,0.211673}%
\pgfsetfillcolor{currentfill}%
\pgfsetlinewidth{0.000000pt}%
\definecolor{currentstroke}{rgb}{0.000000,0.000000,0.000000}%
\pgfsetstrokecolor{currentstroke}%
\pgfsetdash{}{0pt}%
\pgfpathmoveto{\pgfqpoint{2.410172in}{2.537974in}}%
\pgfpathlineto{\pgfqpoint{2.674281in}{2.686821in}}%
\pgfpathlineto{\pgfqpoint{2.410172in}{2.537974in}}%
\pgfpathclose%
\pgfusepath{fill}%
\end{pgfscope}%
\begin{pgfscope}%
\pgfpathrectangle{\pgfqpoint{0.017500in}{0.017500in}}{\pgfqpoint{6.218282in}{3.580372in}} %
\pgfusepath{clip}%
\pgfsetbuttcap%
\pgfsetroundjoin%
\definecolor{currentfill}{rgb}{0.785153,0.220851,0.211673}%
\pgfsetfillcolor{currentfill}%
\pgfsetlinewidth{0.000000pt}%
\definecolor{currentstroke}{rgb}{0.000000,0.000000,0.000000}%
\pgfsetstrokecolor{currentstroke}%
\pgfsetdash{}{0pt}%
\pgfpathmoveto{\pgfqpoint{2.410172in}{2.537974in}}%
\pgfpathlineto{\pgfqpoint{2.674281in}{2.686821in}}%
\pgfpathlineto{\pgfqpoint{2.410172in}{2.537974in}}%
\pgfpathclose%
\pgfusepath{fill}%
\end{pgfscope}%
\begin{pgfscope}%
\pgfpathrectangle{\pgfqpoint{0.017500in}{0.017500in}}{\pgfqpoint{6.218282in}{3.580372in}} %
\pgfusepath{clip}%
\pgfsetbuttcap%
\pgfsetroundjoin%
\definecolor{currentfill}{rgb}{0.785153,0.220851,0.211673}%
\pgfsetfillcolor{currentfill}%
\pgfsetlinewidth{0.000000pt}%
\definecolor{currentstroke}{rgb}{0.000000,0.000000,0.000000}%
\pgfsetstrokecolor{currentstroke}%
\pgfsetdash{}{0pt}%
\pgfpathmoveto{\pgfqpoint{2.410172in}{2.537974in}}%
\pgfpathlineto{\pgfqpoint{2.674281in}{2.686821in}}%
\pgfpathlineto{\pgfqpoint{2.410172in}{2.537974in}}%
\pgfpathclose%
\pgfusepath{fill}%
\end{pgfscope}%
\begin{pgfscope}%
\pgfpathrectangle{\pgfqpoint{0.017500in}{0.017500in}}{\pgfqpoint{6.218282in}{3.580372in}} %
\pgfusepath{clip}%
\pgfsetbuttcap%
\pgfsetroundjoin%
\definecolor{currentfill}{rgb}{0.785153,0.220851,0.211673}%
\pgfsetfillcolor{currentfill}%
\pgfsetlinewidth{0.000000pt}%
\definecolor{currentstroke}{rgb}{0.000000,0.000000,0.000000}%
\pgfsetstrokecolor{currentstroke}%
\pgfsetdash{}{0pt}%
\pgfpathmoveto{\pgfqpoint{2.410172in}{2.537974in}}%
\pgfpathlineto{\pgfqpoint{2.674281in}{2.686821in}}%
\pgfpathlineto{\pgfqpoint{2.410172in}{2.537974in}}%
\pgfpathclose%
\pgfusepath{fill}%
\end{pgfscope}%
\begin{pgfscope}%
\pgfpathrectangle{\pgfqpoint{0.017500in}{0.017500in}}{\pgfqpoint{6.218282in}{3.580372in}} %
\pgfusepath{clip}%
\pgfsetbuttcap%
\pgfsetroundjoin%
\definecolor{currentfill}{rgb}{0.785153,0.220851,0.211673}%
\pgfsetfillcolor{currentfill}%
\pgfsetlinewidth{0.000000pt}%
\definecolor{currentstroke}{rgb}{0.000000,0.000000,0.000000}%
\pgfsetstrokecolor{currentstroke}%
\pgfsetdash{}{0pt}%
\pgfpathmoveto{\pgfqpoint{2.410172in}{2.537974in}}%
\pgfpathlineto{\pgfqpoint{2.674281in}{2.686821in}}%
\pgfpathlineto{\pgfqpoint{2.410172in}{2.537974in}}%
\pgfpathclose%
\pgfusepath{fill}%
\end{pgfscope}%
\begin{pgfscope}%
\pgfpathrectangle{\pgfqpoint{0.017500in}{0.017500in}}{\pgfqpoint{6.218282in}{3.580372in}} %
\pgfusepath{clip}%
\pgfsetbuttcap%
\pgfsetroundjoin%
\definecolor{currentfill}{rgb}{0.785153,0.220851,0.211673}%
\pgfsetfillcolor{currentfill}%
\pgfsetlinewidth{0.000000pt}%
\definecolor{currentstroke}{rgb}{0.000000,0.000000,0.000000}%
\pgfsetstrokecolor{currentstroke}%
\pgfsetdash{}{0pt}%
\pgfpathmoveto{\pgfqpoint{2.410172in}{2.537974in}}%
\pgfpathlineto{\pgfqpoint{2.674281in}{2.686821in}}%
\pgfpathlineto{\pgfqpoint{2.410172in}{2.537974in}}%
\pgfpathclose%
\pgfusepath{fill}%
\end{pgfscope}%
\begin{pgfscope}%
\pgfpathrectangle{\pgfqpoint{0.017500in}{0.017500in}}{\pgfqpoint{6.218282in}{3.580372in}} %
\pgfusepath{clip}%
\pgfsetbuttcap%
\pgfsetroundjoin%
\definecolor{currentfill}{rgb}{0.785153,0.220851,0.211673}%
\pgfsetfillcolor{currentfill}%
\pgfsetlinewidth{0.000000pt}%
\definecolor{currentstroke}{rgb}{0.000000,0.000000,0.000000}%
\pgfsetstrokecolor{currentstroke}%
\pgfsetdash{}{0pt}%
\pgfpathmoveto{\pgfqpoint{2.410172in}{2.537974in}}%
\pgfpathlineto{\pgfqpoint{2.674281in}{2.686821in}}%
\pgfpathlineto{\pgfqpoint{2.410172in}{2.537974in}}%
\pgfpathclose%
\pgfusepath{fill}%
\end{pgfscope}%
\begin{pgfscope}%
\pgfpathrectangle{\pgfqpoint{0.017500in}{0.017500in}}{\pgfqpoint{6.218282in}{3.580372in}} %
\pgfusepath{clip}%
\pgfsetbuttcap%
\pgfsetroundjoin%
\definecolor{currentfill}{rgb}{0.785153,0.220851,0.211673}%
\pgfsetfillcolor{currentfill}%
\pgfsetlinewidth{0.000000pt}%
\definecolor{currentstroke}{rgb}{0.000000,0.000000,0.000000}%
\pgfsetstrokecolor{currentstroke}%
\pgfsetdash{}{0pt}%
\pgfpathmoveto{\pgfqpoint{2.410172in}{2.537974in}}%
\pgfpathlineto{\pgfqpoint{2.674281in}{2.686821in}}%
\pgfpathlineto{\pgfqpoint{2.410172in}{2.537974in}}%
\pgfpathclose%
\pgfusepath{fill}%
\end{pgfscope}%
\begin{pgfscope}%
\pgfpathrectangle{\pgfqpoint{0.017500in}{0.017500in}}{\pgfqpoint{6.218282in}{3.580372in}} %
\pgfusepath{clip}%
\pgfsetbuttcap%
\pgfsetroundjoin%
\definecolor{currentfill}{rgb}{0.785153,0.220851,0.211673}%
\pgfsetfillcolor{currentfill}%
\pgfsetlinewidth{0.000000pt}%
\definecolor{currentstroke}{rgb}{0.000000,0.000000,0.000000}%
\pgfsetstrokecolor{currentstroke}%
\pgfsetdash{}{0pt}%
\pgfpathmoveto{\pgfqpoint{2.410172in}{2.537974in}}%
\pgfpathlineto{\pgfqpoint{2.674281in}{2.686821in}}%
\pgfpathlineto{\pgfqpoint{2.410172in}{2.537974in}}%
\pgfpathclose%
\pgfusepath{fill}%
\end{pgfscope}%
\begin{pgfscope}%
\pgfpathrectangle{\pgfqpoint{0.017500in}{0.017500in}}{\pgfqpoint{6.218282in}{3.580372in}} %
\pgfusepath{clip}%
\pgfsetbuttcap%
\pgfsetroundjoin%
\definecolor{currentfill}{rgb}{0.785153,0.220851,0.211673}%
\pgfsetfillcolor{currentfill}%
\pgfsetlinewidth{0.000000pt}%
\definecolor{currentstroke}{rgb}{0.000000,0.000000,0.000000}%
\pgfsetstrokecolor{currentstroke}%
\pgfsetdash{}{0pt}%
\pgfpathmoveto{\pgfqpoint{2.410172in}{2.537974in}}%
\pgfpathlineto{\pgfqpoint{2.674281in}{2.686821in}}%
\pgfpathlineto{\pgfqpoint{2.410172in}{2.537974in}}%
\pgfpathclose%
\pgfusepath{fill}%
\end{pgfscope}%
\begin{pgfscope}%
\pgfpathrectangle{\pgfqpoint{0.017500in}{0.017500in}}{\pgfqpoint{6.218282in}{3.580372in}} %
\pgfusepath{clip}%
\pgfsetbuttcap%
\pgfsetroundjoin%
\definecolor{currentfill}{rgb}{0.785153,0.220851,0.211673}%
\pgfsetfillcolor{currentfill}%
\pgfsetlinewidth{0.000000pt}%
\definecolor{currentstroke}{rgb}{0.000000,0.000000,0.000000}%
\pgfsetstrokecolor{currentstroke}%
\pgfsetdash{}{0pt}%
\pgfpathmoveto{\pgfqpoint{2.410172in}{2.537974in}}%
\pgfpathlineto{\pgfqpoint{2.674281in}{2.686821in}}%
\pgfpathlineto{\pgfqpoint{2.410172in}{2.537974in}}%
\pgfpathclose%
\pgfusepath{fill}%
\end{pgfscope}%
\begin{pgfscope}%
\pgfpathrectangle{\pgfqpoint{0.017500in}{0.017500in}}{\pgfqpoint{6.218282in}{3.580372in}} %
\pgfusepath{clip}%
\pgfsetbuttcap%
\pgfsetroundjoin%
\definecolor{currentfill}{rgb}{0.785153,0.220851,0.211673}%
\pgfsetfillcolor{currentfill}%
\pgfsetlinewidth{0.000000pt}%
\definecolor{currentstroke}{rgb}{0.000000,0.000000,0.000000}%
\pgfsetstrokecolor{currentstroke}%
\pgfsetdash{}{0pt}%
\pgfpathmoveto{\pgfqpoint{2.410172in}{2.537974in}}%
\pgfpathlineto{\pgfqpoint{2.674281in}{2.686821in}}%
\pgfpathlineto{\pgfqpoint{2.410172in}{2.537974in}}%
\pgfpathclose%
\pgfusepath{fill}%
\end{pgfscope}%
\begin{pgfscope}%
\pgfpathrectangle{\pgfqpoint{0.017500in}{0.017500in}}{\pgfqpoint{6.218282in}{3.580372in}} %
\pgfusepath{clip}%
\pgfsetbuttcap%
\pgfsetroundjoin%
\definecolor{currentfill}{rgb}{0.785153,0.220851,0.211673}%
\pgfsetfillcolor{currentfill}%
\pgfsetlinewidth{0.000000pt}%
\definecolor{currentstroke}{rgb}{0.000000,0.000000,0.000000}%
\pgfsetstrokecolor{currentstroke}%
\pgfsetdash{}{0pt}%
\pgfpathmoveto{\pgfqpoint{2.410172in}{2.537974in}}%
\pgfpathlineto{\pgfqpoint{2.674281in}{2.686821in}}%
\pgfpathlineto{\pgfqpoint{2.410172in}{2.537974in}}%
\pgfpathclose%
\pgfusepath{fill}%
\end{pgfscope}%
\begin{pgfscope}%
\pgfpathrectangle{\pgfqpoint{0.017500in}{0.017500in}}{\pgfqpoint{6.218282in}{3.580372in}} %
\pgfusepath{clip}%
\pgfsetbuttcap%
\pgfsetroundjoin%
\definecolor{currentfill}{rgb}{0.785153,0.220851,0.211673}%
\pgfsetfillcolor{currentfill}%
\pgfsetlinewidth{0.000000pt}%
\definecolor{currentstroke}{rgb}{0.000000,0.000000,0.000000}%
\pgfsetstrokecolor{currentstroke}%
\pgfsetdash{}{0pt}%
\pgfpathmoveto{\pgfqpoint{2.410172in}{2.537974in}}%
\pgfpathlineto{\pgfqpoint{2.674281in}{2.686821in}}%
\pgfpathlineto{\pgfqpoint{2.410172in}{2.537974in}}%
\pgfpathclose%
\pgfusepath{fill}%
\end{pgfscope}%
\begin{pgfscope}%
\pgfpathrectangle{\pgfqpoint{0.017500in}{0.017500in}}{\pgfqpoint{6.218282in}{3.580372in}} %
\pgfusepath{clip}%
\pgfsetbuttcap%
\pgfsetroundjoin%
\definecolor{currentfill}{rgb}{0.785153,0.220851,0.211673}%
\pgfsetfillcolor{currentfill}%
\pgfsetlinewidth{0.000000pt}%
\definecolor{currentstroke}{rgb}{0.000000,0.000000,0.000000}%
\pgfsetstrokecolor{currentstroke}%
\pgfsetdash{}{0pt}%
\pgfpathmoveto{\pgfqpoint{2.410172in}{2.537974in}}%
\pgfpathlineto{\pgfqpoint{2.674281in}{2.686821in}}%
\pgfpathlineto{\pgfqpoint{2.410172in}{2.537974in}}%
\pgfpathclose%
\pgfusepath{fill}%
\end{pgfscope}%
\begin{pgfscope}%
\pgfpathrectangle{\pgfqpoint{0.017500in}{0.017500in}}{\pgfqpoint{6.218282in}{3.580372in}} %
\pgfusepath{clip}%
\pgfsetbuttcap%
\pgfsetroundjoin%
\definecolor{currentfill}{rgb}{0.809329,0.852974,0.922323}%
\pgfsetfillcolor{currentfill}%
\pgfsetlinewidth{0.000000pt}%
\definecolor{currentstroke}{rgb}{0.000000,0.000000,0.000000}%
\pgfsetstrokecolor{currentstroke}%
\pgfsetdash{}{0pt}%
\pgfpathmoveto{\pgfqpoint{4.511715in}{2.762029in}}%
\pgfpathlineto{\pgfqpoint{4.965375in}{2.696000in}}%
\pgfpathlineto{\pgfqpoint{3.688410in}{0.645663in}}%
\pgfpathlineto{\pgfqpoint{3.236097in}{0.721171in}}%
\pgfpathlineto{\pgfqpoint{4.511715in}{2.762029in}}%
\pgfpathclose%
\pgfusepath{fill}%
\end{pgfscope}%
\begin{pgfscope}%
\pgfpathrectangle{\pgfqpoint{0.017500in}{0.017500in}}{\pgfqpoint{6.218282in}{3.580372in}} %
\pgfusepath{clip}%
\pgfsetbuttcap%
\pgfsetroundjoin%
\definecolor{currentfill}{rgb}{0.809329,0.852974,0.922323}%
\pgfsetfillcolor{currentfill}%
\pgfsetlinewidth{0.000000pt}%
\definecolor{currentstroke}{rgb}{0.000000,0.000000,0.000000}%
\pgfsetstrokecolor{currentstroke}%
\pgfsetdash{}{0pt}%
\pgfpathmoveto{\pgfqpoint{4.511715in}{2.762029in}}%
\pgfpathlineto{\pgfqpoint{4.965375in}{2.696000in}}%
\pgfpathlineto{\pgfqpoint{3.688410in}{0.645663in}}%
\pgfpathlineto{\pgfqpoint{3.236097in}{0.721171in}}%
\pgfpathlineto{\pgfqpoint{4.511715in}{2.762029in}}%
\pgfpathclose%
\pgfusepath{fill}%
\end{pgfscope}%
\begin{pgfscope}%
\pgfpathrectangle{\pgfqpoint{0.017500in}{0.017500in}}{\pgfqpoint{6.218282in}{3.580372in}} %
\pgfusepath{clip}%
\pgfsetbuttcap%
\pgfsetroundjoin%
\definecolor{currentfill}{rgb}{0.809329,0.852974,0.922323}%
\pgfsetfillcolor{currentfill}%
\pgfsetlinewidth{0.000000pt}%
\definecolor{currentstroke}{rgb}{0.000000,0.000000,0.000000}%
\pgfsetstrokecolor{currentstroke}%
\pgfsetdash{}{0pt}%
\pgfpathmoveto{\pgfqpoint{4.511715in}{2.762029in}}%
\pgfpathlineto{\pgfqpoint{4.965375in}{2.696000in}}%
\pgfpathlineto{\pgfqpoint{3.688410in}{0.645663in}}%
\pgfpathlineto{\pgfqpoint{3.236097in}{0.721171in}}%
\pgfpathlineto{\pgfqpoint{4.511715in}{2.762029in}}%
\pgfpathclose%
\pgfusepath{fill}%
\end{pgfscope}%
\begin{pgfscope}%
\pgfpathrectangle{\pgfqpoint{0.017500in}{0.017500in}}{\pgfqpoint{6.218282in}{3.580372in}} %
\pgfusepath{clip}%
\pgfsetbuttcap%
\pgfsetroundjoin%
\definecolor{currentfill}{rgb}{0.809329,0.852974,0.922323}%
\pgfsetfillcolor{currentfill}%
\pgfsetlinewidth{0.000000pt}%
\definecolor{currentstroke}{rgb}{0.000000,0.000000,0.000000}%
\pgfsetstrokecolor{currentstroke}%
\pgfsetdash{}{0pt}%
\pgfpathmoveto{\pgfqpoint{4.511715in}{2.762029in}}%
\pgfpathlineto{\pgfqpoint{4.965375in}{2.696000in}}%
\pgfpathlineto{\pgfqpoint{3.688410in}{0.645663in}}%
\pgfpathlineto{\pgfqpoint{3.236097in}{0.721171in}}%
\pgfpathlineto{\pgfqpoint{4.511715in}{2.762029in}}%
\pgfpathclose%
\pgfusepath{fill}%
\end{pgfscope}%
\begin{pgfscope}%
\pgfpathrectangle{\pgfqpoint{0.017500in}{0.017500in}}{\pgfqpoint{6.218282in}{3.580372in}} %
\pgfusepath{clip}%
\pgfsetbuttcap%
\pgfsetroundjoin%
\definecolor{currentfill}{rgb}{0.809329,0.852974,0.922323}%
\pgfsetfillcolor{currentfill}%
\pgfsetlinewidth{0.000000pt}%
\definecolor{currentstroke}{rgb}{0.000000,0.000000,0.000000}%
\pgfsetstrokecolor{currentstroke}%
\pgfsetdash{}{0pt}%
\pgfpathmoveto{\pgfqpoint{4.511715in}{2.762029in}}%
\pgfpathlineto{\pgfqpoint{4.965375in}{2.696000in}}%
\pgfpathlineto{\pgfqpoint{3.688410in}{0.645663in}}%
\pgfpathlineto{\pgfqpoint{3.236097in}{0.721171in}}%
\pgfpathlineto{\pgfqpoint{4.511715in}{2.762029in}}%
\pgfpathclose%
\pgfusepath{fill}%
\end{pgfscope}%
\begin{pgfscope}%
\pgfpathrectangle{\pgfqpoint{0.017500in}{0.017500in}}{\pgfqpoint{6.218282in}{3.580372in}} %
\pgfusepath{clip}%
\pgfsetbuttcap%
\pgfsetroundjoin%
\definecolor{currentfill}{rgb}{0.705673,0.015556,0.150233}%
\pgfsetfillcolor{currentfill}%
\pgfsetlinewidth{0.000000pt}%
\definecolor{currentstroke}{rgb}{0.000000,0.000000,0.000000}%
\pgfsetstrokecolor{currentstroke}%
\pgfsetdash{}{0pt}%
\pgfpathmoveto{\pgfqpoint{4.268228in}{2.632966in}}%
\pgfpathlineto{\pgfqpoint{4.724973in}{2.556562in}}%
\pgfpathlineto{\pgfqpoint{4.965375in}{2.696000in}}%
\pgfpathlineto{\pgfqpoint{4.511715in}{2.762029in}}%
\pgfpathlineto{\pgfqpoint{4.268228in}{2.632966in}}%
\pgfpathclose%
\pgfusepath{fill}%
\end{pgfscope}%
\begin{pgfscope}%
\pgfpathrectangle{\pgfqpoint{0.017500in}{0.017500in}}{\pgfqpoint{6.218282in}{3.580372in}} %
\pgfusepath{clip}%
\pgfsetbuttcap%
\pgfsetroundjoin%
\definecolor{currentfill}{rgb}{0.705673,0.015556,0.150233}%
\pgfsetfillcolor{currentfill}%
\pgfsetlinewidth{0.000000pt}%
\definecolor{currentstroke}{rgb}{0.000000,0.000000,0.000000}%
\pgfsetstrokecolor{currentstroke}%
\pgfsetdash{}{0pt}%
\pgfpathmoveto{\pgfqpoint{4.268228in}{2.632966in}}%
\pgfpathlineto{\pgfqpoint{4.724973in}{2.556562in}}%
\pgfpathlineto{\pgfqpoint{4.965375in}{2.696000in}}%
\pgfpathlineto{\pgfqpoint{4.511715in}{2.762029in}}%
\pgfpathlineto{\pgfqpoint{4.268228in}{2.632966in}}%
\pgfpathclose%
\pgfusepath{fill}%
\end{pgfscope}%
\begin{pgfscope}%
\pgfpathrectangle{\pgfqpoint{0.017500in}{0.017500in}}{\pgfqpoint{6.218282in}{3.580372in}} %
\pgfusepath{clip}%
\pgfsetbuttcap%
\pgfsetroundjoin%
\definecolor{currentfill}{rgb}{0.705673,0.015556,0.150233}%
\pgfsetfillcolor{currentfill}%
\pgfsetlinewidth{0.000000pt}%
\definecolor{currentstroke}{rgb}{0.000000,0.000000,0.000000}%
\pgfsetstrokecolor{currentstroke}%
\pgfsetdash{}{0pt}%
\pgfpathmoveto{\pgfqpoint{4.268228in}{2.632966in}}%
\pgfpathlineto{\pgfqpoint{4.724973in}{2.556562in}}%
\pgfpathlineto{\pgfqpoint{4.965375in}{2.696000in}}%
\pgfpathlineto{\pgfqpoint{4.511715in}{2.762029in}}%
\pgfpathlineto{\pgfqpoint{4.268228in}{2.632966in}}%
\pgfpathclose%
\pgfusepath{fill}%
\end{pgfscope}%
\begin{pgfscope}%
\pgfpathrectangle{\pgfqpoint{0.017500in}{0.017500in}}{\pgfqpoint{6.218282in}{3.580372in}} %
\pgfusepath{clip}%
\pgfsetbuttcap%
\pgfsetroundjoin%
\definecolor{currentfill}{rgb}{0.705673,0.015556,0.150233}%
\pgfsetfillcolor{currentfill}%
\pgfsetlinewidth{0.000000pt}%
\definecolor{currentstroke}{rgb}{0.000000,0.000000,0.000000}%
\pgfsetstrokecolor{currentstroke}%
\pgfsetdash{}{0pt}%
\pgfpathmoveto{\pgfqpoint{4.268228in}{2.632966in}}%
\pgfpathlineto{\pgfqpoint{4.724973in}{2.556562in}}%
\pgfpathlineto{\pgfqpoint{4.965375in}{2.696000in}}%
\pgfpathlineto{\pgfqpoint{4.511715in}{2.762029in}}%
\pgfpathlineto{\pgfqpoint{4.268228in}{2.632966in}}%
\pgfpathclose%
\pgfusepath{fill}%
\end{pgfscope}%
\begin{pgfscope}%
\pgfpathrectangle{\pgfqpoint{0.017500in}{0.017500in}}{\pgfqpoint{6.218282in}{3.580372in}} %
\pgfusepath{clip}%
\pgfsetbuttcap%
\pgfsetroundjoin%
\definecolor{currentfill}{rgb}{0.705673,0.015556,0.150233}%
\pgfsetfillcolor{currentfill}%
\pgfsetlinewidth{0.000000pt}%
\definecolor{currentstroke}{rgb}{0.000000,0.000000,0.000000}%
\pgfsetstrokecolor{currentstroke}%
\pgfsetdash{}{0pt}%
\pgfpathmoveto{\pgfqpoint{4.268228in}{2.632966in}}%
\pgfpathlineto{\pgfqpoint{4.724973in}{2.556562in}}%
\pgfpathlineto{\pgfqpoint{4.965375in}{2.696000in}}%
\pgfpathlineto{\pgfqpoint{4.511715in}{2.762029in}}%
\pgfpathlineto{\pgfqpoint{4.268228in}{2.632966in}}%
\pgfpathclose%
\pgfusepath{fill}%
\end{pgfscope}%
\begin{pgfscope}%
\pgfpathrectangle{\pgfqpoint{0.017500in}{0.017500in}}{\pgfqpoint{6.218282in}{3.580372in}} %
\pgfusepath{clip}%
\pgfsetbuttcap%
\pgfsetroundjoin%
\definecolor{currentfill}{rgb}{0.705673,0.015556,0.150233}%
\pgfsetfillcolor{currentfill}%
\pgfsetlinewidth{0.000000pt}%
\definecolor{currentstroke}{rgb}{0.000000,0.000000,0.000000}%
\pgfsetstrokecolor{currentstroke}%
\pgfsetdash{}{0pt}%
\pgfpathmoveto{\pgfqpoint{4.268228in}{2.632966in}}%
\pgfpathlineto{\pgfqpoint{4.724973in}{2.556562in}}%
\pgfpathlineto{\pgfqpoint{4.965375in}{2.696000in}}%
\pgfpathlineto{\pgfqpoint{4.511715in}{2.762029in}}%
\pgfpathlineto{\pgfqpoint{4.268228in}{2.632966in}}%
\pgfpathclose%
\pgfusepath{fill}%
\end{pgfscope}%
\begin{pgfscope}%
\pgfpathrectangle{\pgfqpoint{0.017500in}{0.017500in}}{\pgfqpoint{6.218282in}{3.580372in}} %
\pgfusepath{clip}%
\pgfsetbuttcap%
\pgfsetroundjoin%
\definecolor{currentfill}{rgb}{0.899534,0.440692,0.344107}%
\pgfsetfillcolor{currentfill}%
\pgfsetlinewidth{0.000000pt}%
\definecolor{currentstroke}{rgb}{0.000000,0.000000,0.000000}%
\pgfsetstrokecolor{currentstroke}%
\pgfsetdash{}{0pt}%
\pgfpathmoveto{\pgfqpoint{1.702428in}{2.361240in}}%
\pgfpathlineto{\pgfqpoint{1.972254in}{2.569602in}}%
\pgfpathlineto{\pgfqpoint{1.702428in}{2.361240in}}%
\pgfpathclose%
\pgfusepath{fill}%
\end{pgfscope}%
\begin{pgfscope}%
\pgfpathrectangle{\pgfqpoint{0.017500in}{0.017500in}}{\pgfqpoint{6.218282in}{3.580372in}} %
\pgfusepath{clip}%
\pgfsetbuttcap%
\pgfsetroundjoin%
\definecolor{currentfill}{rgb}{0.899534,0.440692,0.344107}%
\pgfsetfillcolor{currentfill}%
\pgfsetlinewidth{0.000000pt}%
\definecolor{currentstroke}{rgb}{0.000000,0.000000,0.000000}%
\pgfsetstrokecolor{currentstroke}%
\pgfsetdash{}{0pt}%
\pgfpathmoveto{\pgfqpoint{1.702428in}{2.361240in}}%
\pgfpathlineto{\pgfqpoint{1.972254in}{2.569602in}}%
\pgfpathlineto{\pgfqpoint{1.702428in}{2.361240in}}%
\pgfpathclose%
\pgfusepath{fill}%
\end{pgfscope}%
\begin{pgfscope}%
\pgfpathrectangle{\pgfqpoint{0.017500in}{0.017500in}}{\pgfqpoint{6.218282in}{3.580372in}} %
\pgfusepath{clip}%
\pgfsetbuttcap%
\pgfsetroundjoin%
\definecolor{currentfill}{rgb}{0.899534,0.440692,0.344107}%
\pgfsetfillcolor{currentfill}%
\pgfsetlinewidth{0.000000pt}%
\definecolor{currentstroke}{rgb}{0.000000,0.000000,0.000000}%
\pgfsetstrokecolor{currentstroke}%
\pgfsetdash{}{0pt}%
\pgfpathmoveto{\pgfqpoint{1.702428in}{2.361240in}}%
\pgfpathlineto{\pgfqpoint{1.972254in}{2.569602in}}%
\pgfpathlineto{\pgfqpoint{1.702428in}{2.361240in}}%
\pgfpathclose%
\pgfusepath{fill}%
\end{pgfscope}%
\begin{pgfscope}%
\pgfpathrectangle{\pgfqpoint{0.017500in}{0.017500in}}{\pgfqpoint{6.218282in}{3.580372in}} %
\pgfusepath{clip}%
\pgfsetbuttcap%
\pgfsetroundjoin%
\definecolor{currentfill}{rgb}{0.899534,0.440692,0.344107}%
\pgfsetfillcolor{currentfill}%
\pgfsetlinewidth{0.000000pt}%
\definecolor{currentstroke}{rgb}{0.000000,0.000000,0.000000}%
\pgfsetstrokecolor{currentstroke}%
\pgfsetdash{}{0pt}%
\pgfpathmoveto{\pgfqpoint{1.702428in}{2.361240in}}%
\pgfpathlineto{\pgfqpoint{1.972254in}{2.569602in}}%
\pgfpathlineto{\pgfqpoint{1.702428in}{2.361240in}}%
\pgfpathclose%
\pgfusepath{fill}%
\end{pgfscope}%
\begin{pgfscope}%
\pgfpathrectangle{\pgfqpoint{0.017500in}{0.017500in}}{\pgfqpoint{6.218282in}{3.580372in}} %
\pgfusepath{clip}%
\pgfsetbuttcap%
\pgfsetroundjoin%
\definecolor{currentfill}{rgb}{0.899534,0.440692,0.344107}%
\pgfsetfillcolor{currentfill}%
\pgfsetlinewidth{0.000000pt}%
\definecolor{currentstroke}{rgb}{0.000000,0.000000,0.000000}%
\pgfsetstrokecolor{currentstroke}%
\pgfsetdash{}{0pt}%
\pgfpathmoveto{\pgfqpoint{1.702428in}{2.361240in}}%
\pgfpathlineto{\pgfqpoint{1.972254in}{2.569602in}}%
\pgfpathlineto{\pgfqpoint{1.702428in}{2.361240in}}%
\pgfpathclose%
\pgfusepath{fill}%
\end{pgfscope}%
\begin{pgfscope}%
\pgfpathrectangle{\pgfqpoint{0.017500in}{0.017500in}}{\pgfqpoint{6.218282in}{3.580372in}} %
\pgfusepath{clip}%
\pgfsetbuttcap%
\pgfsetroundjoin%
\definecolor{currentfill}{rgb}{0.899534,0.440692,0.344107}%
\pgfsetfillcolor{currentfill}%
\pgfsetlinewidth{0.000000pt}%
\definecolor{currentstroke}{rgb}{0.000000,0.000000,0.000000}%
\pgfsetstrokecolor{currentstroke}%
\pgfsetdash{}{0pt}%
\pgfpathmoveto{\pgfqpoint{1.702428in}{2.361240in}}%
\pgfpathlineto{\pgfqpoint{1.972254in}{2.569602in}}%
\pgfpathlineto{\pgfqpoint{1.702428in}{2.361240in}}%
\pgfpathclose%
\pgfusepath{fill}%
\end{pgfscope}%
\begin{pgfscope}%
\pgfpathrectangle{\pgfqpoint{0.017500in}{0.017500in}}{\pgfqpoint{6.218282in}{3.580372in}} %
\pgfusepath{clip}%
\pgfsetbuttcap%
\pgfsetroundjoin%
\definecolor{currentfill}{rgb}{0.899534,0.440692,0.344107}%
\pgfsetfillcolor{currentfill}%
\pgfsetlinewidth{0.000000pt}%
\definecolor{currentstroke}{rgb}{0.000000,0.000000,0.000000}%
\pgfsetstrokecolor{currentstroke}%
\pgfsetdash{}{0pt}%
\pgfpathmoveto{\pgfqpoint{1.702428in}{2.361240in}}%
\pgfpathlineto{\pgfqpoint{1.972254in}{2.569602in}}%
\pgfpathlineto{\pgfqpoint{1.702428in}{2.361240in}}%
\pgfpathclose%
\pgfusepath{fill}%
\end{pgfscope}%
\begin{pgfscope}%
\pgfpathrectangle{\pgfqpoint{0.017500in}{0.017500in}}{\pgfqpoint{6.218282in}{3.580372in}} %
\pgfusepath{clip}%
\pgfsetbuttcap%
\pgfsetroundjoin%
\definecolor{currentfill}{rgb}{0.899534,0.440692,0.344107}%
\pgfsetfillcolor{currentfill}%
\pgfsetlinewidth{0.000000pt}%
\definecolor{currentstroke}{rgb}{0.000000,0.000000,0.000000}%
\pgfsetstrokecolor{currentstroke}%
\pgfsetdash{}{0pt}%
\pgfpathmoveto{\pgfqpoint{1.702428in}{2.361240in}}%
\pgfpathlineto{\pgfqpoint{1.972254in}{2.569602in}}%
\pgfpathlineto{\pgfqpoint{1.702428in}{2.361240in}}%
\pgfpathclose%
\pgfusepath{fill}%
\end{pgfscope}%
\begin{pgfscope}%
\pgfpathrectangle{\pgfqpoint{0.017500in}{0.017500in}}{\pgfqpoint{6.218282in}{3.580372in}} %
\pgfusepath{clip}%
\pgfsetbuttcap%
\pgfsetroundjoin%
\definecolor{currentfill}{rgb}{0.899534,0.440692,0.344107}%
\pgfsetfillcolor{currentfill}%
\pgfsetlinewidth{0.000000pt}%
\definecolor{currentstroke}{rgb}{0.000000,0.000000,0.000000}%
\pgfsetstrokecolor{currentstroke}%
\pgfsetdash{}{0pt}%
\pgfpathmoveto{\pgfqpoint{1.702428in}{2.361240in}}%
\pgfpathlineto{\pgfqpoint{1.972254in}{2.569602in}}%
\pgfpathlineto{\pgfqpoint{1.702428in}{2.361240in}}%
\pgfpathclose%
\pgfusepath{fill}%
\end{pgfscope}%
\begin{pgfscope}%
\pgfpathrectangle{\pgfqpoint{0.017500in}{0.017500in}}{\pgfqpoint{6.218282in}{3.580372in}} %
\pgfusepath{clip}%
\pgfsetbuttcap%
\pgfsetroundjoin%
\definecolor{currentfill}{rgb}{0.899534,0.440692,0.344107}%
\pgfsetfillcolor{currentfill}%
\pgfsetlinewidth{0.000000pt}%
\definecolor{currentstroke}{rgb}{0.000000,0.000000,0.000000}%
\pgfsetstrokecolor{currentstroke}%
\pgfsetdash{}{0pt}%
\pgfpathmoveto{\pgfqpoint{1.702428in}{2.361240in}}%
\pgfpathlineto{\pgfqpoint{1.972254in}{2.569602in}}%
\pgfpathlineto{\pgfqpoint{1.702428in}{2.361240in}}%
\pgfpathclose%
\pgfusepath{fill}%
\end{pgfscope}%
\begin{pgfscope}%
\pgfpathrectangle{\pgfqpoint{0.017500in}{0.017500in}}{\pgfqpoint{6.218282in}{3.580372in}} %
\pgfusepath{clip}%
\pgfsetbuttcap%
\pgfsetroundjoin%
\definecolor{currentfill}{rgb}{0.899534,0.440692,0.344107}%
\pgfsetfillcolor{currentfill}%
\pgfsetlinewidth{0.000000pt}%
\definecolor{currentstroke}{rgb}{0.000000,0.000000,0.000000}%
\pgfsetstrokecolor{currentstroke}%
\pgfsetdash{}{0pt}%
\pgfpathmoveto{\pgfqpoint{1.702428in}{2.361240in}}%
\pgfpathlineto{\pgfqpoint{1.972254in}{2.569602in}}%
\pgfpathlineto{\pgfqpoint{1.702428in}{2.361240in}}%
\pgfpathclose%
\pgfusepath{fill}%
\end{pgfscope}%
\begin{pgfscope}%
\pgfpathrectangle{\pgfqpoint{0.017500in}{0.017500in}}{\pgfqpoint{6.218282in}{3.580372in}} %
\pgfusepath{clip}%
\pgfsetbuttcap%
\pgfsetroundjoin%
\definecolor{currentfill}{rgb}{0.899534,0.440692,0.344107}%
\pgfsetfillcolor{currentfill}%
\pgfsetlinewidth{0.000000pt}%
\definecolor{currentstroke}{rgb}{0.000000,0.000000,0.000000}%
\pgfsetstrokecolor{currentstroke}%
\pgfsetdash{}{0pt}%
\pgfpathmoveto{\pgfqpoint{1.702428in}{2.361240in}}%
\pgfpathlineto{\pgfqpoint{1.972254in}{2.569602in}}%
\pgfpathlineto{\pgfqpoint{1.702428in}{2.361240in}}%
\pgfpathclose%
\pgfusepath{fill}%
\end{pgfscope}%
\begin{pgfscope}%
\pgfpathrectangle{\pgfqpoint{0.017500in}{0.017500in}}{\pgfqpoint{6.218282in}{3.580372in}} %
\pgfusepath{clip}%
\pgfsetbuttcap%
\pgfsetroundjoin%
\definecolor{currentfill}{rgb}{0.899534,0.440692,0.344107}%
\pgfsetfillcolor{currentfill}%
\pgfsetlinewidth{0.000000pt}%
\definecolor{currentstroke}{rgb}{0.000000,0.000000,0.000000}%
\pgfsetstrokecolor{currentstroke}%
\pgfsetdash{}{0pt}%
\pgfpathmoveto{\pgfqpoint{1.702428in}{2.361240in}}%
\pgfpathlineto{\pgfqpoint{1.972254in}{2.569602in}}%
\pgfpathlineto{\pgfqpoint{1.702428in}{2.361240in}}%
\pgfpathclose%
\pgfusepath{fill}%
\end{pgfscope}%
\begin{pgfscope}%
\pgfpathrectangle{\pgfqpoint{0.017500in}{0.017500in}}{\pgfqpoint{6.218282in}{3.580372in}} %
\pgfusepath{clip}%
\pgfsetbuttcap%
\pgfsetroundjoin%
\definecolor{currentfill}{rgb}{0.899534,0.440692,0.344107}%
\pgfsetfillcolor{currentfill}%
\pgfsetlinewidth{0.000000pt}%
\definecolor{currentstroke}{rgb}{0.000000,0.000000,0.000000}%
\pgfsetstrokecolor{currentstroke}%
\pgfsetdash{}{0pt}%
\pgfpathmoveto{\pgfqpoint{1.702428in}{2.361240in}}%
\pgfpathlineto{\pgfqpoint{1.972254in}{2.569602in}}%
\pgfpathlineto{\pgfqpoint{1.702428in}{2.361240in}}%
\pgfpathclose%
\pgfusepath{fill}%
\end{pgfscope}%
\begin{pgfscope}%
\pgfpathrectangle{\pgfqpoint{0.017500in}{0.017500in}}{\pgfqpoint{6.218282in}{3.580372in}} %
\pgfusepath{clip}%
\pgfsetbuttcap%
\pgfsetroundjoin%
\definecolor{currentfill}{rgb}{0.899534,0.440692,0.344107}%
\pgfsetfillcolor{currentfill}%
\pgfsetlinewidth{0.000000pt}%
\definecolor{currentstroke}{rgb}{0.000000,0.000000,0.000000}%
\pgfsetstrokecolor{currentstroke}%
\pgfsetdash{}{0pt}%
\pgfpathmoveto{\pgfqpoint{1.702428in}{2.361240in}}%
\pgfpathlineto{\pgfqpoint{1.972254in}{2.569602in}}%
\pgfpathlineto{\pgfqpoint{1.702428in}{2.361240in}}%
\pgfpathclose%
\pgfusepath{fill}%
\end{pgfscope}%
\begin{pgfscope}%
\pgfpathrectangle{\pgfqpoint{0.017500in}{0.017500in}}{\pgfqpoint{6.218282in}{3.580372in}} %
\pgfusepath{clip}%
\pgfsetbuttcap%
\pgfsetroundjoin%
\definecolor{currentfill}{rgb}{0.899534,0.440692,0.344107}%
\pgfsetfillcolor{currentfill}%
\pgfsetlinewidth{0.000000pt}%
\definecolor{currentstroke}{rgb}{0.000000,0.000000,0.000000}%
\pgfsetstrokecolor{currentstroke}%
\pgfsetdash{}{0pt}%
\pgfpathmoveto{\pgfqpoint{1.702428in}{2.361240in}}%
\pgfpathlineto{\pgfqpoint{1.972254in}{2.569602in}}%
\pgfpathlineto{\pgfqpoint{1.702428in}{2.361240in}}%
\pgfpathclose%
\pgfusepath{fill}%
\end{pgfscope}%
\begin{pgfscope}%
\pgfpathrectangle{\pgfqpoint{0.017500in}{0.017500in}}{\pgfqpoint{6.218282in}{3.580372in}} %
\pgfusepath{clip}%
\pgfsetbuttcap%
\pgfsetroundjoin%
\definecolor{currentfill}{rgb}{0.899534,0.440692,0.344107}%
\pgfsetfillcolor{currentfill}%
\pgfsetlinewidth{0.000000pt}%
\definecolor{currentstroke}{rgb}{0.000000,0.000000,0.000000}%
\pgfsetstrokecolor{currentstroke}%
\pgfsetdash{}{0pt}%
\pgfpathmoveto{\pgfqpoint{1.702428in}{2.361240in}}%
\pgfpathlineto{\pgfqpoint{1.972254in}{2.569602in}}%
\pgfpathlineto{\pgfqpoint{1.702428in}{2.361240in}}%
\pgfpathclose%
\pgfusepath{fill}%
\end{pgfscope}%
\begin{pgfscope}%
\pgfpathrectangle{\pgfqpoint{0.017500in}{0.017500in}}{\pgfqpoint{6.218282in}{3.580372in}} %
\pgfusepath{clip}%
\pgfsetbuttcap%
\pgfsetroundjoin%
\definecolor{currentfill}{rgb}{0.899534,0.440692,0.344107}%
\pgfsetfillcolor{currentfill}%
\pgfsetlinewidth{0.000000pt}%
\definecolor{currentstroke}{rgb}{0.000000,0.000000,0.000000}%
\pgfsetstrokecolor{currentstroke}%
\pgfsetdash{}{0pt}%
\pgfpathmoveto{\pgfqpoint{1.702428in}{2.361240in}}%
\pgfpathlineto{\pgfqpoint{1.972254in}{2.569602in}}%
\pgfpathlineto{\pgfqpoint{1.702428in}{2.361240in}}%
\pgfpathclose%
\pgfusepath{fill}%
\end{pgfscope}%
\begin{pgfscope}%
\pgfpathrectangle{\pgfqpoint{0.017500in}{0.017500in}}{\pgfqpoint{6.218282in}{3.580372in}} %
\pgfusepath{clip}%
\pgfsetbuttcap%
\pgfsetroundjoin%
\definecolor{currentfill}{rgb}{0.899534,0.440692,0.344107}%
\pgfsetfillcolor{currentfill}%
\pgfsetlinewidth{0.000000pt}%
\definecolor{currentstroke}{rgb}{0.000000,0.000000,0.000000}%
\pgfsetstrokecolor{currentstroke}%
\pgfsetdash{}{0pt}%
\pgfpathmoveto{\pgfqpoint{1.702428in}{2.361240in}}%
\pgfpathlineto{\pgfqpoint{1.972254in}{2.569602in}}%
\pgfpathlineto{\pgfqpoint{1.702428in}{2.361240in}}%
\pgfpathclose%
\pgfusepath{fill}%
\end{pgfscope}%
\begin{pgfscope}%
\pgfpathrectangle{\pgfqpoint{0.017500in}{0.017500in}}{\pgfqpoint{6.218282in}{3.580372in}} %
\pgfusepath{clip}%
\pgfsetbuttcap%
\pgfsetroundjoin%
\definecolor{currentfill}{rgb}{0.899534,0.440692,0.344107}%
\pgfsetfillcolor{currentfill}%
\pgfsetlinewidth{0.000000pt}%
\definecolor{currentstroke}{rgb}{0.000000,0.000000,0.000000}%
\pgfsetstrokecolor{currentstroke}%
\pgfsetdash{}{0pt}%
\pgfpathmoveto{\pgfqpoint{1.702428in}{2.361240in}}%
\pgfpathlineto{\pgfqpoint{1.972254in}{2.569602in}}%
\pgfpathlineto{\pgfqpoint{1.702428in}{2.361240in}}%
\pgfpathclose%
\pgfusepath{fill}%
\end{pgfscope}%
\begin{pgfscope}%
\pgfpathrectangle{\pgfqpoint{0.017500in}{0.017500in}}{\pgfqpoint{6.218282in}{3.580372in}} %
\pgfusepath{clip}%
\pgfsetbuttcap%
\pgfsetroundjoin%
\definecolor{currentfill}{rgb}{0.899534,0.440692,0.344107}%
\pgfsetfillcolor{currentfill}%
\pgfsetlinewidth{0.000000pt}%
\definecolor{currentstroke}{rgb}{0.000000,0.000000,0.000000}%
\pgfsetstrokecolor{currentstroke}%
\pgfsetdash{}{0pt}%
\pgfpathmoveto{\pgfqpoint{1.702428in}{2.361240in}}%
\pgfpathlineto{\pgfqpoint{1.972254in}{2.569602in}}%
\pgfpathlineto{\pgfqpoint{1.702428in}{2.361240in}}%
\pgfpathclose%
\pgfusepath{fill}%
\end{pgfscope}%
\begin{pgfscope}%
\pgfpathrectangle{\pgfqpoint{0.017500in}{0.017500in}}{\pgfqpoint{6.218282in}{3.580372in}} %
\pgfusepath{clip}%
\pgfsetbuttcap%
\pgfsetroundjoin%
\definecolor{currentfill}{rgb}{0.899534,0.440692,0.344107}%
\pgfsetfillcolor{currentfill}%
\pgfsetlinewidth{0.000000pt}%
\definecolor{currentstroke}{rgb}{0.000000,0.000000,0.000000}%
\pgfsetstrokecolor{currentstroke}%
\pgfsetdash{}{0pt}%
\pgfpathmoveto{\pgfqpoint{1.702428in}{2.361240in}}%
\pgfpathlineto{\pgfqpoint{1.972254in}{2.569602in}}%
\pgfpathlineto{\pgfqpoint{1.702428in}{2.361240in}}%
\pgfpathclose%
\pgfusepath{fill}%
\end{pgfscope}%
\begin{pgfscope}%
\pgfpathrectangle{\pgfqpoint{0.017500in}{0.017500in}}{\pgfqpoint{6.218282in}{3.580372in}} %
\pgfusepath{clip}%
\pgfsetbuttcap%
\pgfsetroundjoin%
\definecolor{currentfill}{rgb}{0.899534,0.440692,0.344107}%
\pgfsetfillcolor{currentfill}%
\pgfsetlinewidth{0.000000pt}%
\definecolor{currentstroke}{rgb}{0.000000,0.000000,0.000000}%
\pgfsetstrokecolor{currentstroke}%
\pgfsetdash{}{0pt}%
\pgfpathmoveto{\pgfqpoint{1.702428in}{2.361240in}}%
\pgfpathlineto{\pgfqpoint{1.972254in}{2.569602in}}%
\pgfpathlineto{\pgfqpoint{1.702428in}{2.361240in}}%
\pgfpathclose%
\pgfusepath{fill}%
\end{pgfscope}%
\begin{pgfscope}%
\pgfpathrectangle{\pgfqpoint{0.017500in}{0.017500in}}{\pgfqpoint{6.218282in}{3.580372in}} %
\pgfusepath{clip}%
\pgfsetbuttcap%
\pgfsetroundjoin%
\definecolor{currentfill}{rgb}{0.899534,0.440692,0.344107}%
\pgfsetfillcolor{currentfill}%
\pgfsetlinewidth{0.000000pt}%
\definecolor{currentstroke}{rgb}{0.000000,0.000000,0.000000}%
\pgfsetstrokecolor{currentstroke}%
\pgfsetdash{}{0pt}%
\pgfpathmoveto{\pgfqpoint{1.702428in}{2.361240in}}%
\pgfpathlineto{\pgfqpoint{1.972254in}{2.569602in}}%
\pgfpathlineto{\pgfqpoint{1.702428in}{2.361240in}}%
\pgfpathclose%
\pgfusepath{fill}%
\end{pgfscope}%
\begin{pgfscope}%
\pgfpathrectangle{\pgfqpoint{0.017500in}{0.017500in}}{\pgfqpoint{6.218282in}{3.580372in}} %
\pgfusepath{clip}%
\pgfsetbuttcap%
\pgfsetroundjoin%
\definecolor{currentfill}{rgb}{0.899534,0.440692,0.344107}%
\pgfsetfillcolor{currentfill}%
\pgfsetlinewidth{0.000000pt}%
\definecolor{currentstroke}{rgb}{0.000000,0.000000,0.000000}%
\pgfsetstrokecolor{currentstroke}%
\pgfsetdash{}{0pt}%
\pgfpathmoveto{\pgfqpoint{1.702428in}{2.361240in}}%
\pgfpathlineto{\pgfqpoint{1.972254in}{2.569602in}}%
\pgfpathlineto{\pgfqpoint{1.702428in}{2.361240in}}%
\pgfpathclose%
\pgfusepath{fill}%
\end{pgfscope}%
\begin{pgfscope}%
\pgfpathrectangle{\pgfqpoint{0.017500in}{0.017500in}}{\pgfqpoint{6.218282in}{3.580372in}} %
\pgfusepath{clip}%
\pgfsetbuttcap%
\pgfsetroundjoin%
\definecolor{currentfill}{rgb}{0.899534,0.440692,0.344107}%
\pgfsetfillcolor{currentfill}%
\pgfsetlinewidth{0.000000pt}%
\definecolor{currentstroke}{rgb}{0.000000,0.000000,0.000000}%
\pgfsetstrokecolor{currentstroke}%
\pgfsetdash{}{0pt}%
\pgfpathmoveto{\pgfqpoint{1.702428in}{2.361240in}}%
\pgfpathlineto{\pgfqpoint{1.972254in}{2.569602in}}%
\pgfpathlineto{\pgfqpoint{1.702428in}{2.361240in}}%
\pgfpathclose%
\pgfusepath{fill}%
\end{pgfscope}%
\begin{pgfscope}%
\pgfpathrectangle{\pgfqpoint{0.017500in}{0.017500in}}{\pgfqpoint{6.218282in}{3.580372in}} %
\pgfusepath{clip}%
\pgfsetbuttcap%
\pgfsetroundjoin%
\definecolor{currentfill}{rgb}{0.899534,0.440692,0.344107}%
\pgfsetfillcolor{currentfill}%
\pgfsetlinewidth{0.000000pt}%
\definecolor{currentstroke}{rgb}{0.000000,0.000000,0.000000}%
\pgfsetstrokecolor{currentstroke}%
\pgfsetdash{}{0pt}%
\pgfpathmoveto{\pgfqpoint{1.702428in}{2.361240in}}%
\pgfpathlineto{\pgfqpoint{1.972254in}{2.569602in}}%
\pgfpathlineto{\pgfqpoint{1.702428in}{2.361240in}}%
\pgfpathclose%
\pgfusepath{fill}%
\end{pgfscope}%
\begin{pgfscope}%
\pgfpathrectangle{\pgfqpoint{0.017500in}{0.017500in}}{\pgfqpoint{6.218282in}{3.580372in}} %
\pgfusepath{clip}%
\pgfsetbuttcap%
\pgfsetroundjoin%
\definecolor{currentfill}{rgb}{0.899534,0.440692,0.344107}%
\pgfsetfillcolor{currentfill}%
\pgfsetlinewidth{0.000000pt}%
\definecolor{currentstroke}{rgb}{0.000000,0.000000,0.000000}%
\pgfsetstrokecolor{currentstroke}%
\pgfsetdash{}{0pt}%
\pgfpathmoveto{\pgfqpoint{1.702428in}{2.361240in}}%
\pgfpathlineto{\pgfqpoint{1.972254in}{2.569602in}}%
\pgfpathlineto{\pgfqpoint{1.702428in}{2.361240in}}%
\pgfpathclose%
\pgfusepath{fill}%
\end{pgfscope}%
\begin{pgfscope}%
\pgfpathrectangle{\pgfqpoint{0.017500in}{0.017500in}}{\pgfqpoint{6.218282in}{3.580372in}} %
\pgfusepath{clip}%
\pgfsetbuttcap%
\pgfsetroundjoin%
\definecolor{currentfill}{rgb}{0.899534,0.440692,0.344107}%
\pgfsetfillcolor{currentfill}%
\pgfsetlinewidth{0.000000pt}%
\definecolor{currentstroke}{rgb}{0.000000,0.000000,0.000000}%
\pgfsetstrokecolor{currentstroke}%
\pgfsetdash{}{0pt}%
\pgfpathmoveto{\pgfqpoint{1.702428in}{2.361240in}}%
\pgfpathlineto{\pgfqpoint{1.972254in}{2.569602in}}%
\pgfpathlineto{\pgfqpoint{1.702428in}{2.361240in}}%
\pgfpathclose%
\pgfusepath{fill}%
\end{pgfscope}%
\begin{pgfscope}%
\pgfpathrectangle{\pgfqpoint{0.017500in}{0.017500in}}{\pgfqpoint{6.218282in}{3.580372in}} %
\pgfusepath{clip}%
\pgfsetbuttcap%
\pgfsetroundjoin%
\definecolor{currentfill}{rgb}{0.899534,0.440692,0.344107}%
\pgfsetfillcolor{currentfill}%
\pgfsetlinewidth{0.000000pt}%
\definecolor{currentstroke}{rgb}{0.000000,0.000000,0.000000}%
\pgfsetstrokecolor{currentstroke}%
\pgfsetdash{}{0pt}%
\pgfpathmoveto{\pgfqpoint{1.702428in}{2.361240in}}%
\pgfpathlineto{\pgfqpoint{1.972254in}{2.569602in}}%
\pgfpathlineto{\pgfqpoint{1.702428in}{2.361240in}}%
\pgfpathclose%
\pgfusepath{fill}%
\end{pgfscope}%
\begin{pgfscope}%
\pgfpathrectangle{\pgfqpoint{0.017500in}{0.017500in}}{\pgfqpoint{6.218282in}{3.580372in}} %
\pgfusepath{clip}%
\pgfsetbuttcap%
\pgfsetroundjoin%
\definecolor{currentfill}{rgb}{0.238948,0.312365,0.765676}%
\pgfsetfillcolor{currentfill}%
\pgfsetlinewidth{0.000000pt}%
\definecolor{currentstroke}{rgb}{0.000000,0.000000,0.000000}%
\pgfsetstrokecolor{currentstroke}%
\pgfsetdash{}{0pt}%
\pgfpathmoveto{\pgfqpoint{2.349703in}{0.878379in}}%
\pgfpathlineto{\pgfqpoint{2.590029in}{2.232729in}}%
\pgfpathlineto{\pgfqpoint{2.349703in}{0.878379in}}%
\pgfpathclose%
\pgfusepath{fill}%
\end{pgfscope}%
\begin{pgfscope}%
\pgfpathrectangle{\pgfqpoint{0.017500in}{0.017500in}}{\pgfqpoint{6.218282in}{3.580372in}} %
\pgfusepath{clip}%
\pgfsetbuttcap%
\pgfsetroundjoin%
\definecolor{currentfill}{rgb}{0.238948,0.312365,0.765676}%
\pgfsetfillcolor{currentfill}%
\pgfsetlinewidth{0.000000pt}%
\definecolor{currentstroke}{rgb}{0.000000,0.000000,0.000000}%
\pgfsetstrokecolor{currentstroke}%
\pgfsetdash{}{0pt}%
\pgfpathmoveto{\pgfqpoint{2.349703in}{0.878379in}}%
\pgfpathlineto{\pgfqpoint{2.590029in}{2.232729in}}%
\pgfpathlineto{\pgfqpoint{2.349703in}{0.878379in}}%
\pgfpathclose%
\pgfusepath{fill}%
\end{pgfscope}%
\begin{pgfscope}%
\pgfpathrectangle{\pgfqpoint{0.017500in}{0.017500in}}{\pgfqpoint{6.218282in}{3.580372in}} %
\pgfusepath{clip}%
\pgfsetbuttcap%
\pgfsetroundjoin%
\definecolor{currentfill}{rgb}{0.238948,0.312365,0.765676}%
\pgfsetfillcolor{currentfill}%
\pgfsetlinewidth{0.000000pt}%
\definecolor{currentstroke}{rgb}{0.000000,0.000000,0.000000}%
\pgfsetstrokecolor{currentstroke}%
\pgfsetdash{}{0pt}%
\pgfpathmoveto{\pgfqpoint{2.349703in}{0.878379in}}%
\pgfpathlineto{\pgfqpoint{2.590029in}{2.232729in}}%
\pgfpathlineto{\pgfqpoint{2.349703in}{0.878379in}}%
\pgfpathclose%
\pgfusepath{fill}%
\end{pgfscope}%
\begin{pgfscope}%
\pgfpathrectangle{\pgfqpoint{0.017500in}{0.017500in}}{\pgfqpoint{6.218282in}{3.580372in}} %
\pgfusepath{clip}%
\pgfsetbuttcap%
\pgfsetroundjoin%
\definecolor{currentfill}{rgb}{0.238948,0.312365,0.765676}%
\pgfsetfillcolor{currentfill}%
\pgfsetlinewidth{0.000000pt}%
\definecolor{currentstroke}{rgb}{0.000000,0.000000,0.000000}%
\pgfsetstrokecolor{currentstroke}%
\pgfsetdash{}{0pt}%
\pgfpathmoveto{\pgfqpoint{2.349703in}{0.878379in}}%
\pgfpathlineto{\pgfqpoint{2.590029in}{2.232729in}}%
\pgfpathlineto{\pgfqpoint{2.349703in}{0.878379in}}%
\pgfpathclose%
\pgfusepath{fill}%
\end{pgfscope}%
\begin{pgfscope}%
\pgfpathrectangle{\pgfqpoint{0.017500in}{0.017500in}}{\pgfqpoint{6.218282in}{3.580372in}} %
\pgfusepath{clip}%
\pgfsetbuttcap%
\pgfsetroundjoin%
\definecolor{currentfill}{rgb}{0.238948,0.312365,0.765676}%
\pgfsetfillcolor{currentfill}%
\pgfsetlinewidth{0.000000pt}%
\definecolor{currentstroke}{rgb}{0.000000,0.000000,0.000000}%
\pgfsetstrokecolor{currentstroke}%
\pgfsetdash{}{0pt}%
\pgfpathmoveto{\pgfqpoint{2.349703in}{0.878379in}}%
\pgfpathlineto{\pgfqpoint{2.590029in}{2.232729in}}%
\pgfpathlineto{\pgfqpoint{2.349703in}{0.878379in}}%
\pgfpathclose%
\pgfusepath{fill}%
\end{pgfscope}%
\begin{pgfscope}%
\pgfpathrectangle{\pgfqpoint{0.017500in}{0.017500in}}{\pgfqpoint{6.218282in}{3.580372in}} %
\pgfusepath{clip}%
\pgfsetbuttcap%
\pgfsetroundjoin%
\definecolor{currentfill}{rgb}{0.238948,0.312365,0.765676}%
\pgfsetfillcolor{currentfill}%
\pgfsetlinewidth{0.000000pt}%
\definecolor{currentstroke}{rgb}{0.000000,0.000000,0.000000}%
\pgfsetstrokecolor{currentstroke}%
\pgfsetdash{}{0pt}%
\pgfpathmoveto{\pgfqpoint{2.349703in}{0.878379in}}%
\pgfpathlineto{\pgfqpoint{2.590029in}{2.232729in}}%
\pgfpathlineto{\pgfqpoint{2.349703in}{0.878379in}}%
\pgfpathclose%
\pgfusepath{fill}%
\end{pgfscope}%
\begin{pgfscope}%
\pgfpathrectangle{\pgfqpoint{0.017500in}{0.017500in}}{\pgfqpoint{6.218282in}{3.580372in}} %
\pgfusepath{clip}%
\pgfsetbuttcap%
\pgfsetroundjoin%
\definecolor{currentfill}{rgb}{0.238948,0.312365,0.765676}%
\pgfsetfillcolor{currentfill}%
\pgfsetlinewidth{0.000000pt}%
\definecolor{currentstroke}{rgb}{0.000000,0.000000,0.000000}%
\pgfsetstrokecolor{currentstroke}%
\pgfsetdash{}{0pt}%
\pgfpathmoveto{\pgfqpoint{2.349703in}{0.878379in}}%
\pgfpathlineto{\pgfqpoint{2.590029in}{2.232729in}}%
\pgfpathlineto{\pgfqpoint{2.349703in}{0.878379in}}%
\pgfpathclose%
\pgfusepath{fill}%
\end{pgfscope}%
\begin{pgfscope}%
\pgfpathrectangle{\pgfqpoint{0.017500in}{0.017500in}}{\pgfqpoint{6.218282in}{3.580372in}} %
\pgfusepath{clip}%
\pgfsetbuttcap%
\pgfsetroundjoin%
\definecolor{currentfill}{rgb}{0.238948,0.312365,0.765676}%
\pgfsetfillcolor{currentfill}%
\pgfsetlinewidth{0.000000pt}%
\definecolor{currentstroke}{rgb}{0.000000,0.000000,0.000000}%
\pgfsetstrokecolor{currentstroke}%
\pgfsetdash{}{0pt}%
\pgfpathmoveto{\pgfqpoint{2.349703in}{0.878379in}}%
\pgfpathlineto{\pgfqpoint{2.590029in}{2.232729in}}%
\pgfpathlineto{\pgfqpoint{2.349703in}{0.878379in}}%
\pgfpathclose%
\pgfusepath{fill}%
\end{pgfscope}%
\begin{pgfscope}%
\pgfpathrectangle{\pgfqpoint{0.017500in}{0.017500in}}{\pgfqpoint{6.218282in}{3.580372in}} %
\pgfusepath{clip}%
\pgfsetbuttcap%
\pgfsetroundjoin%
\definecolor{currentfill}{rgb}{0.238948,0.312365,0.765676}%
\pgfsetfillcolor{currentfill}%
\pgfsetlinewidth{0.000000pt}%
\definecolor{currentstroke}{rgb}{0.000000,0.000000,0.000000}%
\pgfsetstrokecolor{currentstroke}%
\pgfsetdash{}{0pt}%
\pgfpathmoveto{\pgfqpoint{2.349703in}{0.878379in}}%
\pgfpathlineto{\pgfqpoint{2.590029in}{2.232729in}}%
\pgfpathlineto{\pgfqpoint{2.349703in}{0.878379in}}%
\pgfpathclose%
\pgfusepath{fill}%
\end{pgfscope}%
\begin{pgfscope}%
\pgfpathrectangle{\pgfqpoint{0.017500in}{0.017500in}}{\pgfqpoint{6.218282in}{3.580372in}} %
\pgfusepath{clip}%
\pgfsetbuttcap%
\pgfsetroundjoin%
\definecolor{currentfill}{rgb}{0.238948,0.312365,0.765676}%
\pgfsetfillcolor{currentfill}%
\pgfsetlinewidth{0.000000pt}%
\definecolor{currentstroke}{rgb}{0.000000,0.000000,0.000000}%
\pgfsetstrokecolor{currentstroke}%
\pgfsetdash{}{0pt}%
\pgfpathmoveto{\pgfqpoint{2.349703in}{0.878379in}}%
\pgfpathlineto{\pgfqpoint{2.590029in}{2.232729in}}%
\pgfpathlineto{\pgfqpoint{2.349703in}{0.878379in}}%
\pgfpathclose%
\pgfusepath{fill}%
\end{pgfscope}%
\begin{pgfscope}%
\pgfpathrectangle{\pgfqpoint{0.017500in}{0.017500in}}{\pgfqpoint{6.218282in}{3.580372in}} %
\pgfusepath{clip}%
\pgfsetbuttcap%
\pgfsetroundjoin%
\definecolor{currentfill}{rgb}{0.238948,0.312365,0.765676}%
\pgfsetfillcolor{currentfill}%
\pgfsetlinewidth{0.000000pt}%
\definecolor{currentstroke}{rgb}{0.000000,0.000000,0.000000}%
\pgfsetstrokecolor{currentstroke}%
\pgfsetdash{}{0pt}%
\pgfpathmoveto{\pgfqpoint{2.349703in}{0.878379in}}%
\pgfpathlineto{\pgfqpoint{2.590029in}{2.232729in}}%
\pgfpathlineto{\pgfqpoint{2.349703in}{0.878379in}}%
\pgfpathclose%
\pgfusepath{fill}%
\end{pgfscope}%
\begin{pgfscope}%
\pgfpathrectangle{\pgfqpoint{0.017500in}{0.017500in}}{\pgfqpoint{6.218282in}{3.580372in}} %
\pgfusepath{clip}%
\pgfsetbuttcap%
\pgfsetroundjoin%
\definecolor{currentfill}{rgb}{0.238948,0.312365,0.765676}%
\pgfsetfillcolor{currentfill}%
\pgfsetlinewidth{0.000000pt}%
\definecolor{currentstroke}{rgb}{0.000000,0.000000,0.000000}%
\pgfsetstrokecolor{currentstroke}%
\pgfsetdash{}{0pt}%
\pgfpathmoveto{\pgfqpoint{2.349703in}{0.878379in}}%
\pgfpathlineto{\pgfqpoint{2.590029in}{2.232729in}}%
\pgfpathlineto{\pgfqpoint{2.349703in}{0.878379in}}%
\pgfpathclose%
\pgfusepath{fill}%
\end{pgfscope}%
\begin{pgfscope}%
\pgfpathrectangle{\pgfqpoint{0.017500in}{0.017500in}}{\pgfqpoint{6.218282in}{3.580372in}} %
\pgfusepath{clip}%
\pgfsetbuttcap%
\pgfsetroundjoin%
\definecolor{currentfill}{rgb}{0.238948,0.312365,0.765676}%
\pgfsetfillcolor{currentfill}%
\pgfsetlinewidth{0.000000pt}%
\definecolor{currentstroke}{rgb}{0.000000,0.000000,0.000000}%
\pgfsetstrokecolor{currentstroke}%
\pgfsetdash{}{0pt}%
\pgfpathmoveto{\pgfqpoint{2.349703in}{0.878379in}}%
\pgfpathlineto{\pgfqpoint{2.590029in}{2.232729in}}%
\pgfpathlineto{\pgfqpoint{2.349703in}{0.878379in}}%
\pgfpathclose%
\pgfusepath{fill}%
\end{pgfscope}%
\begin{pgfscope}%
\pgfpathrectangle{\pgfqpoint{0.017500in}{0.017500in}}{\pgfqpoint{6.218282in}{3.580372in}} %
\pgfusepath{clip}%
\pgfsetbuttcap%
\pgfsetroundjoin%
\definecolor{currentfill}{rgb}{0.238948,0.312365,0.765676}%
\pgfsetfillcolor{currentfill}%
\pgfsetlinewidth{0.000000pt}%
\definecolor{currentstroke}{rgb}{0.000000,0.000000,0.000000}%
\pgfsetstrokecolor{currentstroke}%
\pgfsetdash{}{0pt}%
\pgfpathmoveto{\pgfqpoint{2.349703in}{0.878379in}}%
\pgfpathlineto{\pgfqpoint{2.590029in}{2.232729in}}%
\pgfpathlineto{\pgfqpoint{2.349703in}{0.878379in}}%
\pgfpathclose%
\pgfusepath{fill}%
\end{pgfscope}%
\begin{pgfscope}%
\pgfpathrectangle{\pgfqpoint{0.017500in}{0.017500in}}{\pgfqpoint{6.218282in}{3.580372in}} %
\pgfusepath{clip}%
\pgfsetbuttcap%
\pgfsetroundjoin%
\definecolor{currentfill}{rgb}{0.238948,0.312365,0.765676}%
\pgfsetfillcolor{currentfill}%
\pgfsetlinewidth{0.000000pt}%
\definecolor{currentstroke}{rgb}{0.000000,0.000000,0.000000}%
\pgfsetstrokecolor{currentstroke}%
\pgfsetdash{}{0pt}%
\pgfpathmoveto{\pgfqpoint{2.349703in}{0.878379in}}%
\pgfpathlineto{\pgfqpoint{2.590029in}{2.232729in}}%
\pgfpathlineto{\pgfqpoint{2.349703in}{0.878379in}}%
\pgfpathclose%
\pgfusepath{fill}%
\end{pgfscope}%
\begin{pgfscope}%
\pgfpathrectangle{\pgfqpoint{0.017500in}{0.017500in}}{\pgfqpoint{6.218282in}{3.580372in}} %
\pgfusepath{clip}%
\pgfsetbuttcap%
\pgfsetroundjoin%
\definecolor{currentfill}{rgb}{0.238948,0.312365,0.765676}%
\pgfsetfillcolor{currentfill}%
\pgfsetlinewidth{0.000000pt}%
\definecolor{currentstroke}{rgb}{0.000000,0.000000,0.000000}%
\pgfsetstrokecolor{currentstroke}%
\pgfsetdash{}{0pt}%
\pgfpathmoveto{\pgfqpoint{2.349703in}{0.878379in}}%
\pgfpathlineto{\pgfqpoint{2.590029in}{2.232729in}}%
\pgfpathlineto{\pgfqpoint{2.349703in}{0.878379in}}%
\pgfpathclose%
\pgfusepath{fill}%
\end{pgfscope}%
\begin{pgfscope}%
\pgfpathrectangle{\pgfqpoint{0.017500in}{0.017500in}}{\pgfqpoint{6.218282in}{3.580372in}} %
\pgfusepath{clip}%
\pgfsetbuttcap%
\pgfsetroundjoin%
\definecolor{currentfill}{rgb}{0.238948,0.312365,0.765676}%
\pgfsetfillcolor{currentfill}%
\pgfsetlinewidth{0.000000pt}%
\definecolor{currentstroke}{rgb}{0.000000,0.000000,0.000000}%
\pgfsetstrokecolor{currentstroke}%
\pgfsetdash{}{0pt}%
\pgfpathmoveto{\pgfqpoint{2.349703in}{0.878379in}}%
\pgfpathlineto{\pgfqpoint{2.590029in}{2.232729in}}%
\pgfpathlineto{\pgfqpoint{2.349703in}{0.878379in}}%
\pgfpathclose%
\pgfusepath{fill}%
\end{pgfscope}%
\begin{pgfscope}%
\pgfpathrectangle{\pgfqpoint{0.017500in}{0.017500in}}{\pgfqpoint{6.218282in}{3.580372in}} %
\pgfusepath{clip}%
\pgfsetbuttcap%
\pgfsetroundjoin%
\definecolor{currentfill}{rgb}{0.238948,0.312365,0.765676}%
\pgfsetfillcolor{currentfill}%
\pgfsetlinewidth{0.000000pt}%
\definecolor{currentstroke}{rgb}{0.000000,0.000000,0.000000}%
\pgfsetstrokecolor{currentstroke}%
\pgfsetdash{}{0pt}%
\pgfpathmoveto{\pgfqpoint{2.349703in}{0.878379in}}%
\pgfpathlineto{\pgfqpoint{2.590029in}{2.232729in}}%
\pgfpathlineto{\pgfqpoint{2.349703in}{0.878379in}}%
\pgfpathclose%
\pgfusepath{fill}%
\end{pgfscope}%
\begin{pgfscope}%
\pgfpathrectangle{\pgfqpoint{0.017500in}{0.017500in}}{\pgfqpoint{6.218282in}{3.580372in}} %
\pgfusepath{clip}%
\pgfsetbuttcap%
\pgfsetroundjoin%
\definecolor{currentfill}{rgb}{0.238948,0.312365,0.765676}%
\pgfsetfillcolor{currentfill}%
\pgfsetlinewidth{0.000000pt}%
\definecolor{currentstroke}{rgb}{0.000000,0.000000,0.000000}%
\pgfsetstrokecolor{currentstroke}%
\pgfsetdash{}{0pt}%
\pgfpathmoveto{\pgfqpoint{2.349703in}{0.878379in}}%
\pgfpathlineto{\pgfqpoint{2.590029in}{2.232729in}}%
\pgfpathlineto{\pgfqpoint{2.349703in}{0.878379in}}%
\pgfpathclose%
\pgfusepath{fill}%
\end{pgfscope}%
\begin{pgfscope}%
\pgfpathrectangle{\pgfqpoint{0.017500in}{0.017500in}}{\pgfqpoint{6.218282in}{3.580372in}} %
\pgfusepath{clip}%
\pgfsetbuttcap%
\pgfsetroundjoin%
\definecolor{currentfill}{rgb}{0.238948,0.312365,0.765676}%
\pgfsetfillcolor{currentfill}%
\pgfsetlinewidth{0.000000pt}%
\definecolor{currentstroke}{rgb}{0.000000,0.000000,0.000000}%
\pgfsetstrokecolor{currentstroke}%
\pgfsetdash{}{0pt}%
\pgfpathmoveto{\pgfqpoint{2.349703in}{0.878379in}}%
\pgfpathlineto{\pgfqpoint{2.590029in}{2.232729in}}%
\pgfpathlineto{\pgfqpoint{2.349703in}{0.878379in}}%
\pgfpathclose%
\pgfusepath{fill}%
\end{pgfscope}%
\begin{pgfscope}%
\pgfpathrectangle{\pgfqpoint{0.017500in}{0.017500in}}{\pgfqpoint{6.218282in}{3.580372in}} %
\pgfusepath{clip}%
\pgfsetbuttcap%
\pgfsetroundjoin%
\definecolor{currentfill}{rgb}{0.238948,0.312365,0.765676}%
\pgfsetfillcolor{currentfill}%
\pgfsetlinewidth{0.000000pt}%
\definecolor{currentstroke}{rgb}{0.000000,0.000000,0.000000}%
\pgfsetstrokecolor{currentstroke}%
\pgfsetdash{}{0pt}%
\pgfpathmoveto{\pgfqpoint{2.349703in}{0.878379in}}%
\pgfpathlineto{\pgfqpoint{2.590029in}{2.232729in}}%
\pgfpathlineto{\pgfqpoint{2.349703in}{0.878379in}}%
\pgfpathclose%
\pgfusepath{fill}%
\end{pgfscope}%
\begin{pgfscope}%
\pgfpathrectangle{\pgfqpoint{0.017500in}{0.017500in}}{\pgfqpoint{6.218282in}{3.580372in}} %
\pgfusepath{clip}%
\pgfsetbuttcap%
\pgfsetroundjoin%
\definecolor{currentfill}{rgb}{0.238948,0.312365,0.765676}%
\pgfsetfillcolor{currentfill}%
\pgfsetlinewidth{0.000000pt}%
\definecolor{currentstroke}{rgb}{0.000000,0.000000,0.000000}%
\pgfsetstrokecolor{currentstroke}%
\pgfsetdash{}{0pt}%
\pgfpathmoveto{\pgfqpoint{2.349703in}{0.878379in}}%
\pgfpathlineto{\pgfqpoint{2.590029in}{2.232729in}}%
\pgfpathlineto{\pgfqpoint{2.349703in}{0.878379in}}%
\pgfpathclose%
\pgfusepath{fill}%
\end{pgfscope}%
\begin{pgfscope}%
\pgfpathrectangle{\pgfqpoint{0.017500in}{0.017500in}}{\pgfqpoint{6.218282in}{3.580372in}} %
\pgfusepath{clip}%
\pgfsetbuttcap%
\pgfsetroundjoin%
\definecolor{currentfill}{rgb}{0.238948,0.312365,0.765676}%
\pgfsetfillcolor{currentfill}%
\pgfsetlinewidth{0.000000pt}%
\definecolor{currentstroke}{rgb}{0.000000,0.000000,0.000000}%
\pgfsetstrokecolor{currentstroke}%
\pgfsetdash{}{0pt}%
\pgfpathmoveto{\pgfqpoint{2.349703in}{0.878379in}}%
\pgfpathlineto{\pgfqpoint{2.590029in}{2.232729in}}%
\pgfpathlineto{\pgfqpoint{2.349703in}{0.878379in}}%
\pgfpathclose%
\pgfusepath{fill}%
\end{pgfscope}%
\begin{pgfscope}%
\pgfpathrectangle{\pgfqpoint{0.017500in}{0.017500in}}{\pgfqpoint{6.218282in}{3.580372in}} %
\pgfusepath{clip}%
\pgfsetbuttcap%
\pgfsetroundjoin%
\definecolor{currentfill}{rgb}{0.238948,0.312365,0.765676}%
\pgfsetfillcolor{currentfill}%
\pgfsetlinewidth{0.000000pt}%
\definecolor{currentstroke}{rgb}{0.000000,0.000000,0.000000}%
\pgfsetstrokecolor{currentstroke}%
\pgfsetdash{}{0pt}%
\pgfpathmoveto{\pgfqpoint{2.349703in}{0.878379in}}%
\pgfpathlineto{\pgfqpoint{2.590029in}{2.232729in}}%
\pgfpathlineto{\pgfqpoint{2.349703in}{0.878379in}}%
\pgfpathclose%
\pgfusepath{fill}%
\end{pgfscope}%
\begin{pgfscope}%
\pgfpathrectangle{\pgfqpoint{0.017500in}{0.017500in}}{\pgfqpoint{6.218282in}{3.580372in}} %
\pgfusepath{clip}%
\pgfsetbuttcap%
\pgfsetroundjoin%
\definecolor{currentfill}{rgb}{0.238948,0.312365,0.765676}%
\pgfsetfillcolor{currentfill}%
\pgfsetlinewidth{0.000000pt}%
\definecolor{currentstroke}{rgb}{0.000000,0.000000,0.000000}%
\pgfsetstrokecolor{currentstroke}%
\pgfsetdash{}{0pt}%
\pgfpathmoveto{\pgfqpoint{2.349703in}{0.878379in}}%
\pgfpathlineto{\pgfqpoint{2.590029in}{2.232729in}}%
\pgfpathlineto{\pgfqpoint{2.349703in}{0.878379in}}%
\pgfpathclose%
\pgfusepath{fill}%
\end{pgfscope}%
\begin{pgfscope}%
\pgfpathrectangle{\pgfqpoint{0.017500in}{0.017500in}}{\pgfqpoint{6.218282in}{3.580372in}} %
\pgfusepath{clip}%
\pgfsetbuttcap%
\pgfsetroundjoin%
\definecolor{currentfill}{rgb}{0.238948,0.312365,0.765676}%
\pgfsetfillcolor{currentfill}%
\pgfsetlinewidth{0.000000pt}%
\definecolor{currentstroke}{rgb}{0.000000,0.000000,0.000000}%
\pgfsetstrokecolor{currentstroke}%
\pgfsetdash{}{0pt}%
\pgfpathmoveto{\pgfqpoint{2.349703in}{0.878379in}}%
\pgfpathlineto{\pgfqpoint{2.590029in}{2.232729in}}%
\pgfpathlineto{\pgfqpoint{2.349703in}{0.878379in}}%
\pgfpathclose%
\pgfusepath{fill}%
\end{pgfscope}%
\begin{pgfscope}%
\pgfpathrectangle{\pgfqpoint{0.017500in}{0.017500in}}{\pgfqpoint{6.218282in}{3.580372in}} %
\pgfusepath{clip}%
\pgfsetbuttcap%
\pgfsetroundjoin%
\definecolor{currentfill}{rgb}{0.238948,0.312365,0.765676}%
\pgfsetfillcolor{currentfill}%
\pgfsetlinewidth{0.000000pt}%
\definecolor{currentstroke}{rgb}{0.000000,0.000000,0.000000}%
\pgfsetstrokecolor{currentstroke}%
\pgfsetdash{}{0pt}%
\pgfpathmoveto{\pgfqpoint{2.349703in}{0.878379in}}%
\pgfpathlineto{\pgfqpoint{2.590029in}{2.232729in}}%
\pgfpathlineto{\pgfqpoint{2.349703in}{0.878379in}}%
\pgfpathclose%
\pgfusepath{fill}%
\end{pgfscope}%
\begin{pgfscope}%
\pgfpathrectangle{\pgfqpoint{0.017500in}{0.017500in}}{\pgfqpoint{6.218282in}{3.580372in}} %
\pgfusepath{clip}%
\pgfsetbuttcap%
\pgfsetroundjoin%
\definecolor{currentfill}{rgb}{0.238948,0.312365,0.765676}%
\pgfsetfillcolor{currentfill}%
\pgfsetlinewidth{0.000000pt}%
\definecolor{currentstroke}{rgb}{0.000000,0.000000,0.000000}%
\pgfsetstrokecolor{currentstroke}%
\pgfsetdash{}{0pt}%
\pgfpathmoveto{\pgfqpoint{2.349703in}{0.878379in}}%
\pgfpathlineto{\pgfqpoint{2.590029in}{2.232729in}}%
\pgfpathlineto{\pgfqpoint{2.349703in}{0.878379in}}%
\pgfpathclose%
\pgfusepath{fill}%
\end{pgfscope}%
\begin{pgfscope}%
\pgfpathrectangle{\pgfqpoint{0.017500in}{0.017500in}}{\pgfqpoint{6.218282in}{3.580372in}} %
\pgfusepath{clip}%
\pgfsetbuttcap%
\pgfsetroundjoin%
\definecolor{currentfill}{rgb}{0.238948,0.312365,0.765676}%
\pgfsetfillcolor{currentfill}%
\pgfsetlinewidth{0.000000pt}%
\definecolor{currentstroke}{rgb}{0.000000,0.000000,0.000000}%
\pgfsetstrokecolor{currentstroke}%
\pgfsetdash{}{0pt}%
\pgfpathmoveto{\pgfqpoint{2.349703in}{0.878379in}}%
\pgfpathlineto{\pgfqpoint{2.590029in}{2.232729in}}%
\pgfpathlineto{\pgfqpoint{2.349703in}{0.878379in}}%
\pgfpathclose%
\pgfusepath{fill}%
\end{pgfscope}%
\begin{pgfscope}%
\pgfpathrectangle{\pgfqpoint{0.017500in}{0.017500in}}{\pgfqpoint{6.218282in}{3.580372in}} %
\pgfusepath{clip}%
\pgfsetbuttcap%
\pgfsetroundjoin%
\definecolor{currentfill}{rgb}{0.238948,0.312365,0.765676}%
\pgfsetfillcolor{currentfill}%
\pgfsetlinewidth{0.000000pt}%
\definecolor{currentstroke}{rgb}{0.000000,0.000000,0.000000}%
\pgfsetstrokecolor{currentstroke}%
\pgfsetdash{}{0pt}%
\pgfpathmoveto{\pgfqpoint{2.349703in}{0.878379in}}%
\pgfpathlineto{\pgfqpoint{2.590029in}{2.232729in}}%
\pgfpathlineto{\pgfqpoint{2.349703in}{0.878379in}}%
\pgfpathclose%
\pgfusepath{fill}%
\end{pgfscope}%
\begin{pgfscope}%
\pgfpathrectangle{\pgfqpoint{0.017500in}{0.017500in}}{\pgfqpoint{6.218282in}{3.580372in}} %
\pgfusepath{clip}%
\pgfsetbuttcap%
\pgfsetroundjoin%
\definecolor{currentfill}{rgb}{0.779745,0.210305,0.207104}%
\pgfsetfillcolor{currentfill}%
\pgfsetlinewidth{0.000000pt}%
\definecolor{currentstroke}{rgb}{0.000000,0.000000,0.000000}%
\pgfsetstrokecolor{currentstroke}%
\pgfsetdash{}{0pt}%
\pgfpathmoveto{\pgfqpoint{2.410172in}{2.537974in}}%
\pgfpathlineto{\pgfqpoint{2.855457in}{2.456853in}}%
\pgfpathlineto{\pgfqpoint{3.116444in}{2.629069in}}%
\pgfpathlineto{\pgfqpoint{2.674281in}{2.686821in}}%
\pgfpathlineto{\pgfqpoint{2.410172in}{2.537974in}}%
\pgfpathclose%
\pgfusepath{fill}%
\end{pgfscope}%
\begin{pgfscope}%
\pgfpathrectangle{\pgfqpoint{0.017500in}{0.017500in}}{\pgfqpoint{6.218282in}{3.580372in}} %
\pgfusepath{clip}%
\pgfsetbuttcap%
\pgfsetroundjoin%
\definecolor{currentfill}{rgb}{0.779745,0.210305,0.207104}%
\pgfsetfillcolor{currentfill}%
\pgfsetlinewidth{0.000000pt}%
\definecolor{currentstroke}{rgb}{0.000000,0.000000,0.000000}%
\pgfsetstrokecolor{currentstroke}%
\pgfsetdash{}{0pt}%
\pgfpathmoveto{\pgfqpoint{2.410172in}{2.537974in}}%
\pgfpathlineto{\pgfqpoint{2.855457in}{2.456853in}}%
\pgfpathlineto{\pgfqpoint{3.116444in}{2.629069in}}%
\pgfpathlineto{\pgfqpoint{2.674281in}{2.686821in}}%
\pgfpathlineto{\pgfqpoint{2.410172in}{2.537974in}}%
\pgfpathclose%
\pgfusepath{fill}%
\end{pgfscope}%
\begin{pgfscope}%
\pgfpathrectangle{\pgfqpoint{0.017500in}{0.017500in}}{\pgfqpoint{6.218282in}{3.580372in}} %
\pgfusepath{clip}%
\pgfsetbuttcap%
\pgfsetroundjoin%
\definecolor{currentfill}{rgb}{0.779745,0.210305,0.207104}%
\pgfsetfillcolor{currentfill}%
\pgfsetlinewidth{0.000000pt}%
\definecolor{currentstroke}{rgb}{0.000000,0.000000,0.000000}%
\pgfsetstrokecolor{currentstroke}%
\pgfsetdash{}{0pt}%
\pgfpathmoveto{\pgfqpoint{2.410172in}{2.537974in}}%
\pgfpathlineto{\pgfqpoint{2.855457in}{2.456853in}}%
\pgfpathlineto{\pgfqpoint{3.116444in}{2.629069in}}%
\pgfpathlineto{\pgfqpoint{2.674281in}{2.686821in}}%
\pgfpathlineto{\pgfqpoint{2.410172in}{2.537974in}}%
\pgfpathclose%
\pgfusepath{fill}%
\end{pgfscope}%
\begin{pgfscope}%
\pgfpathrectangle{\pgfqpoint{0.017500in}{0.017500in}}{\pgfqpoint{6.218282in}{3.580372in}} %
\pgfusepath{clip}%
\pgfsetbuttcap%
\pgfsetroundjoin%
\definecolor{currentfill}{rgb}{0.779745,0.210305,0.207104}%
\pgfsetfillcolor{currentfill}%
\pgfsetlinewidth{0.000000pt}%
\definecolor{currentstroke}{rgb}{0.000000,0.000000,0.000000}%
\pgfsetstrokecolor{currentstroke}%
\pgfsetdash{}{0pt}%
\pgfpathmoveto{\pgfqpoint{2.410172in}{2.537974in}}%
\pgfpathlineto{\pgfqpoint{2.855457in}{2.456853in}}%
\pgfpathlineto{\pgfqpoint{3.116444in}{2.629069in}}%
\pgfpathlineto{\pgfqpoint{2.674281in}{2.686821in}}%
\pgfpathlineto{\pgfqpoint{2.410172in}{2.537974in}}%
\pgfpathclose%
\pgfusepath{fill}%
\end{pgfscope}%
\begin{pgfscope}%
\pgfpathrectangle{\pgfqpoint{0.017500in}{0.017500in}}{\pgfqpoint{6.218282in}{3.580372in}} %
\pgfusepath{clip}%
\pgfsetbuttcap%
\pgfsetroundjoin%
\definecolor{currentfill}{rgb}{0.779745,0.210305,0.207104}%
\pgfsetfillcolor{currentfill}%
\pgfsetlinewidth{0.000000pt}%
\definecolor{currentstroke}{rgb}{0.000000,0.000000,0.000000}%
\pgfsetstrokecolor{currentstroke}%
\pgfsetdash{}{0pt}%
\pgfpathmoveto{\pgfqpoint{2.410172in}{2.537974in}}%
\pgfpathlineto{\pgfqpoint{2.855457in}{2.456853in}}%
\pgfpathlineto{\pgfqpoint{3.116444in}{2.629069in}}%
\pgfpathlineto{\pgfqpoint{2.674281in}{2.686821in}}%
\pgfpathlineto{\pgfqpoint{2.410172in}{2.537974in}}%
\pgfpathclose%
\pgfusepath{fill}%
\end{pgfscope}%
\begin{pgfscope}%
\pgfpathrectangle{\pgfqpoint{0.017500in}{0.017500in}}{\pgfqpoint{6.218282in}{3.580372in}} %
\pgfusepath{clip}%
\pgfsetbuttcap%
\pgfsetroundjoin%
\definecolor{currentfill}{rgb}{0.779745,0.210305,0.207104}%
\pgfsetfillcolor{currentfill}%
\pgfsetlinewidth{0.000000pt}%
\definecolor{currentstroke}{rgb}{0.000000,0.000000,0.000000}%
\pgfsetstrokecolor{currentstroke}%
\pgfsetdash{}{0pt}%
\pgfpathmoveto{\pgfqpoint{2.410172in}{2.537974in}}%
\pgfpathlineto{\pgfqpoint{2.855457in}{2.456853in}}%
\pgfpathlineto{\pgfqpoint{3.116444in}{2.629069in}}%
\pgfpathlineto{\pgfqpoint{2.674281in}{2.686821in}}%
\pgfpathlineto{\pgfqpoint{2.410172in}{2.537974in}}%
\pgfpathclose%
\pgfusepath{fill}%
\end{pgfscope}%
\begin{pgfscope}%
\pgfpathrectangle{\pgfqpoint{0.017500in}{0.017500in}}{\pgfqpoint{6.218282in}{3.580372in}} %
\pgfusepath{clip}%
\pgfsetbuttcap%
\pgfsetroundjoin%
\definecolor{currentfill}{rgb}{0.740957,0.122240,0.175744}%
\pgfsetfillcolor{currentfill}%
\pgfsetlinewidth{0.000000pt}%
\definecolor{currentstroke}{rgb}{0.000000,0.000000,0.000000}%
\pgfsetstrokecolor{currentstroke}%
\pgfsetdash{}{0pt}%
\pgfpathmoveto{\pgfqpoint{3.564605in}{2.552204in}}%
\pgfpathlineto{\pgfqpoint{3.817122in}{2.696128in}}%
\pgfpathlineto{\pgfqpoint{3.564605in}{2.552204in}}%
\pgfpathclose%
\pgfusepath{fill}%
\end{pgfscope}%
\begin{pgfscope}%
\pgfpathrectangle{\pgfqpoint{0.017500in}{0.017500in}}{\pgfqpoint{6.218282in}{3.580372in}} %
\pgfusepath{clip}%
\pgfsetbuttcap%
\pgfsetroundjoin%
\definecolor{currentfill}{rgb}{0.740957,0.122240,0.175744}%
\pgfsetfillcolor{currentfill}%
\pgfsetlinewidth{0.000000pt}%
\definecolor{currentstroke}{rgb}{0.000000,0.000000,0.000000}%
\pgfsetstrokecolor{currentstroke}%
\pgfsetdash{}{0pt}%
\pgfpathmoveto{\pgfqpoint{3.564605in}{2.552204in}}%
\pgfpathlineto{\pgfqpoint{3.817122in}{2.696128in}}%
\pgfpathlineto{\pgfqpoint{3.564605in}{2.552204in}}%
\pgfpathclose%
\pgfusepath{fill}%
\end{pgfscope}%
\begin{pgfscope}%
\pgfpathrectangle{\pgfqpoint{0.017500in}{0.017500in}}{\pgfqpoint{6.218282in}{3.580372in}} %
\pgfusepath{clip}%
\pgfsetbuttcap%
\pgfsetroundjoin%
\definecolor{currentfill}{rgb}{0.740957,0.122240,0.175744}%
\pgfsetfillcolor{currentfill}%
\pgfsetlinewidth{0.000000pt}%
\definecolor{currentstroke}{rgb}{0.000000,0.000000,0.000000}%
\pgfsetstrokecolor{currentstroke}%
\pgfsetdash{}{0pt}%
\pgfpathmoveto{\pgfqpoint{3.564605in}{2.552204in}}%
\pgfpathlineto{\pgfqpoint{3.817122in}{2.696128in}}%
\pgfpathlineto{\pgfqpoint{3.564605in}{2.552204in}}%
\pgfpathclose%
\pgfusepath{fill}%
\end{pgfscope}%
\begin{pgfscope}%
\pgfpathrectangle{\pgfqpoint{0.017500in}{0.017500in}}{\pgfqpoint{6.218282in}{3.580372in}} %
\pgfusepath{clip}%
\pgfsetbuttcap%
\pgfsetroundjoin%
\definecolor{currentfill}{rgb}{0.740957,0.122240,0.175744}%
\pgfsetfillcolor{currentfill}%
\pgfsetlinewidth{0.000000pt}%
\definecolor{currentstroke}{rgb}{0.000000,0.000000,0.000000}%
\pgfsetstrokecolor{currentstroke}%
\pgfsetdash{}{0pt}%
\pgfpathmoveto{\pgfqpoint{3.564605in}{2.552204in}}%
\pgfpathlineto{\pgfqpoint{3.817122in}{2.696128in}}%
\pgfpathlineto{\pgfqpoint{3.564605in}{2.552204in}}%
\pgfpathclose%
\pgfusepath{fill}%
\end{pgfscope}%
\begin{pgfscope}%
\pgfpathrectangle{\pgfqpoint{0.017500in}{0.017500in}}{\pgfqpoint{6.218282in}{3.580372in}} %
\pgfusepath{clip}%
\pgfsetbuttcap%
\pgfsetroundjoin%
\definecolor{currentfill}{rgb}{0.740957,0.122240,0.175744}%
\pgfsetfillcolor{currentfill}%
\pgfsetlinewidth{0.000000pt}%
\definecolor{currentstroke}{rgb}{0.000000,0.000000,0.000000}%
\pgfsetstrokecolor{currentstroke}%
\pgfsetdash{}{0pt}%
\pgfpathmoveto{\pgfqpoint{3.564605in}{2.552204in}}%
\pgfpathlineto{\pgfqpoint{3.817122in}{2.696128in}}%
\pgfpathlineto{\pgfqpoint{3.564605in}{2.552204in}}%
\pgfpathclose%
\pgfusepath{fill}%
\end{pgfscope}%
\begin{pgfscope}%
\pgfpathrectangle{\pgfqpoint{0.017500in}{0.017500in}}{\pgfqpoint{6.218282in}{3.580372in}} %
\pgfusepath{clip}%
\pgfsetbuttcap%
\pgfsetroundjoin%
\definecolor{currentfill}{rgb}{0.740957,0.122240,0.175744}%
\pgfsetfillcolor{currentfill}%
\pgfsetlinewidth{0.000000pt}%
\definecolor{currentstroke}{rgb}{0.000000,0.000000,0.000000}%
\pgfsetstrokecolor{currentstroke}%
\pgfsetdash{}{0pt}%
\pgfpathmoveto{\pgfqpoint{3.564605in}{2.552204in}}%
\pgfpathlineto{\pgfqpoint{3.817122in}{2.696128in}}%
\pgfpathlineto{\pgfqpoint{3.564605in}{2.552204in}}%
\pgfpathclose%
\pgfusepath{fill}%
\end{pgfscope}%
\begin{pgfscope}%
\pgfpathrectangle{\pgfqpoint{0.017500in}{0.017500in}}{\pgfqpoint{6.218282in}{3.580372in}} %
\pgfusepath{clip}%
\pgfsetbuttcap%
\pgfsetroundjoin%
\definecolor{currentfill}{rgb}{0.740957,0.122240,0.175744}%
\pgfsetfillcolor{currentfill}%
\pgfsetlinewidth{0.000000pt}%
\definecolor{currentstroke}{rgb}{0.000000,0.000000,0.000000}%
\pgfsetstrokecolor{currentstroke}%
\pgfsetdash{}{0pt}%
\pgfpathmoveto{\pgfqpoint{3.564605in}{2.552204in}}%
\pgfpathlineto{\pgfqpoint{3.817122in}{2.696128in}}%
\pgfpathlineto{\pgfqpoint{3.564605in}{2.552204in}}%
\pgfpathclose%
\pgfusepath{fill}%
\end{pgfscope}%
\begin{pgfscope}%
\pgfpathrectangle{\pgfqpoint{0.017500in}{0.017500in}}{\pgfqpoint{6.218282in}{3.580372in}} %
\pgfusepath{clip}%
\pgfsetbuttcap%
\pgfsetroundjoin%
\definecolor{currentfill}{rgb}{0.740957,0.122240,0.175744}%
\pgfsetfillcolor{currentfill}%
\pgfsetlinewidth{0.000000pt}%
\definecolor{currentstroke}{rgb}{0.000000,0.000000,0.000000}%
\pgfsetstrokecolor{currentstroke}%
\pgfsetdash{}{0pt}%
\pgfpathmoveto{\pgfqpoint{3.564605in}{2.552204in}}%
\pgfpathlineto{\pgfqpoint{3.817122in}{2.696128in}}%
\pgfpathlineto{\pgfqpoint{3.564605in}{2.552204in}}%
\pgfpathclose%
\pgfusepath{fill}%
\end{pgfscope}%
\begin{pgfscope}%
\pgfpathrectangle{\pgfqpoint{0.017500in}{0.017500in}}{\pgfqpoint{6.218282in}{3.580372in}} %
\pgfusepath{clip}%
\pgfsetbuttcap%
\pgfsetroundjoin%
\definecolor{currentfill}{rgb}{0.740957,0.122240,0.175744}%
\pgfsetfillcolor{currentfill}%
\pgfsetlinewidth{0.000000pt}%
\definecolor{currentstroke}{rgb}{0.000000,0.000000,0.000000}%
\pgfsetstrokecolor{currentstroke}%
\pgfsetdash{}{0pt}%
\pgfpathmoveto{\pgfqpoint{3.564605in}{2.552204in}}%
\pgfpathlineto{\pgfqpoint{3.817122in}{2.696128in}}%
\pgfpathlineto{\pgfqpoint{3.564605in}{2.552204in}}%
\pgfpathclose%
\pgfusepath{fill}%
\end{pgfscope}%
\begin{pgfscope}%
\pgfpathrectangle{\pgfqpoint{0.017500in}{0.017500in}}{\pgfqpoint{6.218282in}{3.580372in}} %
\pgfusepath{clip}%
\pgfsetbuttcap%
\pgfsetroundjoin%
\definecolor{currentfill}{rgb}{0.740957,0.122240,0.175744}%
\pgfsetfillcolor{currentfill}%
\pgfsetlinewidth{0.000000pt}%
\definecolor{currentstroke}{rgb}{0.000000,0.000000,0.000000}%
\pgfsetstrokecolor{currentstroke}%
\pgfsetdash{}{0pt}%
\pgfpathmoveto{\pgfqpoint{3.564605in}{2.552204in}}%
\pgfpathlineto{\pgfqpoint{3.817122in}{2.696128in}}%
\pgfpathlineto{\pgfqpoint{3.564605in}{2.552204in}}%
\pgfpathclose%
\pgfusepath{fill}%
\end{pgfscope}%
\begin{pgfscope}%
\pgfpathrectangle{\pgfqpoint{0.017500in}{0.017500in}}{\pgfqpoint{6.218282in}{3.580372in}} %
\pgfusepath{clip}%
\pgfsetbuttcap%
\pgfsetroundjoin%
\definecolor{currentfill}{rgb}{0.740957,0.122240,0.175744}%
\pgfsetfillcolor{currentfill}%
\pgfsetlinewidth{0.000000pt}%
\definecolor{currentstroke}{rgb}{0.000000,0.000000,0.000000}%
\pgfsetstrokecolor{currentstroke}%
\pgfsetdash{}{0pt}%
\pgfpathmoveto{\pgfqpoint{3.564605in}{2.552204in}}%
\pgfpathlineto{\pgfqpoint{3.817122in}{2.696128in}}%
\pgfpathlineto{\pgfqpoint{3.564605in}{2.552204in}}%
\pgfpathclose%
\pgfusepath{fill}%
\end{pgfscope}%
\begin{pgfscope}%
\pgfpathrectangle{\pgfqpoint{0.017500in}{0.017500in}}{\pgfqpoint{6.218282in}{3.580372in}} %
\pgfusepath{clip}%
\pgfsetbuttcap%
\pgfsetroundjoin%
\definecolor{currentfill}{rgb}{0.740957,0.122240,0.175744}%
\pgfsetfillcolor{currentfill}%
\pgfsetlinewidth{0.000000pt}%
\definecolor{currentstroke}{rgb}{0.000000,0.000000,0.000000}%
\pgfsetstrokecolor{currentstroke}%
\pgfsetdash{}{0pt}%
\pgfpathmoveto{\pgfqpoint{3.564605in}{2.552204in}}%
\pgfpathlineto{\pgfqpoint{3.817122in}{2.696128in}}%
\pgfpathlineto{\pgfqpoint{3.564605in}{2.552204in}}%
\pgfpathclose%
\pgfusepath{fill}%
\end{pgfscope}%
\begin{pgfscope}%
\pgfpathrectangle{\pgfqpoint{0.017500in}{0.017500in}}{\pgfqpoint{6.218282in}{3.580372in}} %
\pgfusepath{clip}%
\pgfsetbuttcap%
\pgfsetroundjoin%
\definecolor{currentfill}{rgb}{0.740957,0.122240,0.175744}%
\pgfsetfillcolor{currentfill}%
\pgfsetlinewidth{0.000000pt}%
\definecolor{currentstroke}{rgb}{0.000000,0.000000,0.000000}%
\pgfsetstrokecolor{currentstroke}%
\pgfsetdash{}{0pt}%
\pgfpathmoveto{\pgfqpoint{3.564605in}{2.552204in}}%
\pgfpathlineto{\pgfqpoint{3.817122in}{2.696128in}}%
\pgfpathlineto{\pgfqpoint{3.564605in}{2.552204in}}%
\pgfpathclose%
\pgfusepath{fill}%
\end{pgfscope}%
\begin{pgfscope}%
\pgfpathrectangle{\pgfqpoint{0.017500in}{0.017500in}}{\pgfqpoint{6.218282in}{3.580372in}} %
\pgfusepath{clip}%
\pgfsetbuttcap%
\pgfsetroundjoin%
\definecolor{currentfill}{rgb}{0.740957,0.122240,0.175744}%
\pgfsetfillcolor{currentfill}%
\pgfsetlinewidth{0.000000pt}%
\definecolor{currentstroke}{rgb}{0.000000,0.000000,0.000000}%
\pgfsetstrokecolor{currentstroke}%
\pgfsetdash{}{0pt}%
\pgfpathmoveto{\pgfqpoint{3.564605in}{2.552204in}}%
\pgfpathlineto{\pgfqpoint{3.817122in}{2.696128in}}%
\pgfpathlineto{\pgfqpoint{3.564605in}{2.552204in}}%
\pgfpathclose%
\pgfusepath{fill}%
\end{pgfscope}%
\begin{pgfscope}%
\pgfpathrectangle{\pgfqpoint{0.017500in}{0.017500in}}{\pgfqpoint{6.218282in}{3.580372in}} %
\pgfusepath{clip}%
\pgfsetbuttcap%
\pgfsetroundjoin%
\definecolor{currentfill}{rgb}{0.740957,0.122240,0.175744}%
\pgfsetfillcolor{currentfill}%
\pgfsetlinewidth{0.000000pt}%
\definecolor{currentstroke}{rgb}{0.000000,0.000000,0.000000}%
\pgfsetstrokecolor{currentstroke}%
\pgfsetdash{}{0pt}%
\pgfpathmoveto{\pgfqpoint{3.564605in}{2.552204in}}%
\pgfpathlineto{\pgfqpoint{3.817122in}{2.696128in}}%
\pgfpathlineto{\pgfqpoint{3.564605in}{2.552204in}}%
\pgfpathclose%
\pgfusepath{fill}%
\end{pgfscope}%
\begin{pgfscope}%
\pgfpathrectangle{\pgfqpoint{0.017500in}{0.017500in}}{\pgfqpoint{6.218282in}{3.580372in}} %
\pgfusepath{clip}%
\pgfsetbuttcap%
\pgfsetroundjoin%
\definecolor{currentfill}{rgb}{0.740957,0.122240,0.175744}%
\pgfsetfillcolor{currentfill}%
\pgfsetlinewidth{0.000000pt}%
\definecolor{currentstroke}{rgb}{0.000000,0.000000,0.000000}%
\pgfsetstrokecolor{currentstroke}%
\pgfsetdash{}{0pt}%
\pgfpathmoveto{\pgfqpoint{3.564605in}{2.552204in}}%
\pgfpathlineto{\pgfqpoint{3.817122in}{2.696128in}}%
\pgfpathlineto{\pgfqpoint{3.564605in}{2.552204in}}%
\pgfpathclose%
\pgfusepath{fill}%
\end{pgfscope}%
\begin{pgfscope}%
\pgfpathrectangle{\pgfqpoint{0.017500in}{0.017500in}}{\pgfqpoint{6.218282in}{3.580372in}} %
\pgfusepath{clip}%
\pgfsetbuttcap%
\pgfsetroundjoin%
\definecolor{currentfill}{rgb}{0.740957,0.122240,0.175744}%
\pgfsetfillcolor{currentfill}%
\pgfsetlinewidth{0.000000pt}%
\definecolor{currentstroke}{rgb}{0.000000,0.000000,0.000000}%
\pgfsetstrokecolor{currentstroke}%
\pgfsetdash{}{0pt}%
\pgfpathmoveto{\pgfqpoint{3.564605in}{2.552204in}}%
\pgfpathlineto{\pgfqpoint{3.817122in}{2.696128in}}%
\pgfpathlineto{\pgfqpoint{3.564605in}{2.552204in}}%
\pgfpathclose%
\pgfusepath{fill}%
\end{pgfscope}%
\begin{pgfscope}%
\pgfpathrectangle{\pgfqpoint{0.017500in}{0.017500in}}{\pgfqpoint{6.218282in}{3.580372in}} %
\pgfusepath{clip}%
\pgfsetbuttcap%
\pgfsetroundjoin%
\definecolor{currentfill}{rgb}{0.740957,0.122240,0.175744}%
\pgfsetfillcolor{currentfill}%
\pgfsetlinewidth{0.000000pt}%
\definecolor{currentstroke}{rgb}{0.000000,0.000000,0.000000}%
\pgfsetstrokecolor{currentstroke}%
\pgfsetdash{}{0pt}%
\pgfpathmoveto{\pgfqpoint{3.564605in}{2.552204in}}%
\pgfpathlineto{\pgfqpoint{3.817122in}{2.696128in}}%
\pgfpathlineto{\pgfqpoint{3.564605in}{2.552204in}}%
\pgfpathclose%
\pgfusepath{fill}%
\end{pgfscope}%
\begin{pgfscope}%
\pgfpathrectangle{\pgfqpoint{0.017500in}{0.017500in}}{\pgfqpoint{6.218282in}{3.580372in}} %
\pgfusepath{clip}%
\pgfsetbuttcap%
\pgfsetroundjoin%
\definecolor{currentfill}{rgb}{0.740957,0.122240,0.175744}%
\pgfsetfillcolor{currentfill}%
\pgfsetlinewidth{0.000000pt}%
\definecolor{currentstroke}{rgb}{0.000000,0.000000,0.000000}%
\pgfsetstrokecolor{currentstroke}%
\pgfsetdash{}{0pt}%
\pgfpathmoveto{\pgfqpoint{3.564605in}{2.552204in}}%
\pgfpathlineto{\pgfqpoint{3.817122in}{2.696128in}}%
\pgfpathlineto{\pgfqpoint{3.564605in}{2.552204in}}%
\pgfpathclose%
\pgfusepath{fill}%
\end{pgfscope}%
\begin{pgfscope}%
\pgfpathrectangle{\pgfqpoint{0.017500in}{0.017500in}}{\pgfqpoint{6.218282in}{3.580372in}} %
\pgfusepath{clip}%
\pgfsetbuttcap%
\pgfsetroundjoin%
\definecolor{currentfill}{rgb}{0.740957,0.122240,0.175744}%
\pgfsetfillcolor{currentfill}%
\pgfsetlinewidth{0.000000pt}%
\definecolor{currentstroke}{rgb}{0.000000,0.000000,0.000000}%
\pgfsetstrokecolor{currentstroke}%
\pgfsetdash{}{0pt}%
\pgfpathmoveto{\pgfqpoint{3.564605in}{2.552204in}}%
\pgfpathlineto{\pgfqpoint{3.817122in}{2.696128in}}%
\pgfpathlineto{\pgfqpoint{3.564605in}{2.552204in}}%
\pgfpathclose%
\pgfusepath{fill}%
\end{pgfscope}%
\begin{pgfscope}%
\pgfpathrectangle{\pgfqpoint{0.017500in}{0.017500in}}{\pgfqpoint{6.218282in}{3.580372in}} %
\pgfusepath{clip}%
\pgfsetbuttcap%
\pgfsetroundjoin%
\definecolor{currentfill}{rgb}{0.740957,0.122240,0.175744}%
\pgfsetfillcolor{currentfill}%
\pgfsetlinewidth{0.000000pt}%
\definecolor{currentstroke}{rgb}{0.000000,0.000000,0.000000}%
\pgfsetstrokecolor{currentstroke}%
\pgfsetdash{}{0pt}%
\pgfpathmoveto{\pgfqpoint{3.564605in}{2.552204in}}%
\pgfpathlineto{\pgfqpoint{3.817122in}{2.696128in}}%
\pgfpathlineto{\pgfqpoint{3.564605in}{2.552204in}}%
\pgfpathclose%
\pgfusepath{fill}%
\end{pgfscope}%
\begin{pgfscope}%
\pgfpathrectangle{\pgfqpoint{0.017500in}{0.017500in}}{\pgfqpoint{6.218282in}{3.580372in}} %
\pgfusepath{clip}%
\pgfsetbuttcap%
\pgfsetroundjoin%
\definecolor{currentfill}{rgb}{0.740957,0.122240,0.175744}%
\pgfsetfillcolor{currentfill}%
\pgfsetlinewidth{0.000000pt}%
\definecolor{currentstroke}{rgb}{0.000000,0.000000,0.000000}%
\pgfsetstrokecolor{currentstroke}%
\pgfsetdash{}{0pt}%
\pgfpathmoveto{\pgfqpoint{3.564605in}{2.552204in}}%
\pgfpathlineto{\pgfqpoint{3.817122in}{2.696128in}}%
\pgfpathlineto{\pgfqpoint{3.564605in}{2.552204in}}%
\pgfpathclose%
\pgfusepath{fill}%
\end{pgfscope}%
\begin{pgfscope}%
\pgfpathrectangle{\pgfqpoint{0.017500in}{0.017500in}}{\pgfqpoint{6.218282in}{3.580372in}} %
\pgfusepath{clip}%
\pgfsetbuttcap%
\pgfsetroundjoin%
\definecolor{currentfill}{rgb}{0.740957,0.122240,0.175744}%
\pgfsetfillcolor{currentfill}%
\pgfsetlinewidth{0.000000pt}%
\definecolor{currentstroke}{rgb}{0.000000,0.000000,0.000000}%
\pgfsetstrokecolor{currentstroke}%
\pgfsetdash{}{0pt}%
\pgfpathmoveto{\pgfqpoint{3.564605in}{2.552204in}}%
\pgfpathlineto{\pgfqpoint{3.817122in}{2.696128in}}%
\pgfpathlineto{\pgfqpoint{3.564605in}{2.552204in}}%
\pgfpathclose%
\pgfusepath{fill}%
\end{pgfscope}%
\begin{pgfscope}%
\pgfpathrectangle{\pgfqpoint{0.017500in}{0.017500in}}{\pgfqpoint{6.218282in}{3.580372in}} %
\pgfusepath{clip}%
\pgfsetbuttcap%
\pgfsetroundjoin%
\definecolor{currentfill}{rgb}{0.740957,0.122240,0.175744}%
\pgfsetfillcolor{currentfill}%
\pgfsetlinewidth{0.000000pt}%
\definecolor{currentstroke}{rgb}{0.000000,0.000000,0.000000}%
\pgfsetstrokecolor{currentstroke}%
\pgfsetdash{}{0pt}%
\pgfpathmoveto{\pgfqpoint{3.564605in}{2.552204in}}%
\pgfpathlineto{\pgfqpoint{3.817122in}{2.696128in}}%
\pgfpathlineto{\pgfqpoint{3.564605in}{2.552204in}}%
\pgfpathclose%
\pgfusepath{fill}%
\end{pgfscope}%
\begin{pgfscope}%
\pgfpathrectangle{\pgfqpoint{0.017500in}{0.017500in}}{\pgfqpoint{6.218282in}{3.580372in}} %
\pgfusepath{clip}%
\pgfsetbuttcap%
\pgfsetroundjoin%
\definecolor{currentfill}{rgb}{0.740957,0.122240,0.175744}%
\pgfsetfillcolor{currentfill}%
\pgfsetlinewidth{0.000000pt}%
\definecolor{currentstroke}{rgb}{0.000000,0.000000,0.000000}%
\pgfsetstrokecolor{currentstroke}%
\pgfsetdash{}{0pt}%
\pgfpathmoveto{\pgfqpoint{3.564605in}{2.552204in}}%
\pgfpathlineto{\pgfqpoint{3.817122in}{2.696128in}}%
\pgfpathlineto{\pgfqpoint{3.564605in}{2.552204in}}%
\pgfpathclose%
\pgfusepath{fill}%
\end{pgfscope}%
\begin{pgfscope}%
\pgfpathrectangle{\pgfqpoint{0.017500in}{0.017500in}}{\pgfqpoint{6.218282in}{3.580372in}} %
\pgfusepath{clip}%
\pgfsetbuttcap%
\pgfsetroundjoin%
\definecolor{currentfill}{rgb}{0.740957,0.122240,0.175744}%
\pgfsetfillcolor{currentfill}%
\pgfsetlinewidth{0.000000pt}%
\definecolor{currentstroke}{rgb}{0.000000,0.000000,0.000000}%
\pgfsetstrokecolor{currentstroke}%
\pgfsetdash{}{0pt}%
\pgfpathmoveto{\pgfqpoint{3.564605in}{2.552204in}}%
\pgfpathlineto{\pgfqpoint{3.817122in}{2.696128in}}%
\pgfpathlineto{\pgfqpoint{3.564605in}{2.552204in}}%
\pgfpathclose%
\pgfusepath{fill}%
\end{pgfscope}%
\begin{pgfscope}%
\pgfpathrectangle{\pgfqpoint{0.017500in}{0.017500in}}{\pgfqpoint{6.218282in}{3.580372in}} %
\pgfusepath{clip}%
\pgfsetbuttcap%
\pgfsetroundjoin%
\definecolor{currentfill}{rgb}{0.740957,0.122240,0.175744}%
\pgfsetfillcolor{currentfill}%
\pgfsetlinewidth{0.000000pt}%
\definecolor{currentstroke}{rgb}{0.000000,0.000000,0.000000}%
\pgfsetstrokecolor{currentstroke}%
\pgfsetdash{}{0pt}%
\pgfpathmoveto{\pgfqpoint{3.564605in}{2.552204in}}%
\pgfpathlineto{\pgfqpoint{3.817122in}{2.696128in}}%
\pgfpathlineto{\pgfqpoint{3.564605in}{2.552204in}}%
\pgfpathclose%
\pgfusepath{fill}%
\end{pgfscope}%
\begin{pgfscope}%
\pgfpathrectangle{\pgfqpoint{0.017500in}{0.017500in}}{\pgfqpoint{6.218282in}{3.580372in}} %
\pgfusepath{clip}%
\pgfsetbuttcap%
\pgfsetroundjoin%
\definecolor{currentfill}{rgb}{0.740957,0.122240,0.175744}%
\pgfsetfillcolor{currentfill}%
\pgfsetlinewidth{0.000000pt}%
\definecolor{currentstroke}{rgb}{0.000000,0.000000,0.000000}%
\pgfsetstrokecolor{currentstroke}%
\pgfsetdash{}{0pt}%
\pgfpathmoveto{\pgfqpoint{3.564605in}{2.552204in}}%
\pgfpathlineto{\pgfqpoint{3.817122in}{2.696128in}}%
\pgfpathlineto{\pgfqpoint{3.564605in}{2.552204in}}%
\pgfpathclose%
\pgfusepath{fill}%
\end{pgfscope}%
\begin{pgfscope}%
\pgfpathrectangle{\pgfqpoint{0.017500in}{0.017500in}}{\pgfqpoint{6.218282in}{3.580372in}} %
\pgfusepath{clip}%
\pgfsetbuttcap%
\pgfsetroundjoin%
\definecolor{currentfill}{rgb}{0.740957,0.122240,0.175744}%
\pgfsetfillcolor{currentfill}%
\pgfsetlinewidth{0.000000pt}%
\definecolor{currentstroke}{rgb}{0.000000,0.000000,0.000000}%
\pgfsetstrokecolor{currentstroke}%
\pgfsetdash{}{0pt}%
\pgfpathmoveto{\pgfqpoint{3.564605in}{2.552204in}}%
\pgfpathlineto{\pgfqpoint{3.817122in}{2.696128in}}%
\pgfpathlineto{\pgfqpoint{3.564605in}{2.552204in}}%
\pgfpathclose%
\pgfusepath{fill}%
\end{pgfscope}%
\begin{pgfscope}%
\pgfpathrectangle{\pgfqpoint{0.017500in}{0.017500in}}{\pgfqpoint{6.218282in}{3.580372in}} %
\pgfusepath{clip}%
\pgfsetbuttcap%
\pgfsetroundjoin%
\definecolor{currentfill}{rgb}{0.740957,0.122240,0.175744}%
\pgfsetfillcolor{currentfill}%
\pgfsetlinewidth{0.000000pt}%
\definecolor{currentstroke}{rgb}{0.000000,0.000000,0.000000}%
\pgfsetstrokecolor{currentstroke}%
\pgfsetdash{}{0pt}%
\pgfpathmoveto{\pgfqpoint{3.564605in}{2.552204in}}%
\pgfpathlineto{\pgfqpoint{3.817122in}{2.696128in}}%
\pgfpathlineto{\pgfqpoint{3.564605in}{2.552204in}}%
\pgfpathclose%
\pgfusepath{fill}%
\end{pgfscope}%
\begin{pgfscope}%
\pgfpathrectangle{\pgfqpoint{0.017500in}{0.017500in}}{\pgfqpoint{6.218282in}{3.580372in}} %
\pgfusepath{clip}%
\pgfsetbuttcap%
\pgfsetroundjoin%
\definecolor{currentfill}{rgb}{0.909460,0.839386,0.800331}%
\pgfsetfillcolor{currentfill}%
\pgfsetlinewidth{0.000000pt}%
\definecolor{currentstroke}{rgb}{0.000000,0.000000,0.000000}%
\pgfsetstrokecolor{currentstroke}%
\pgfsetdash{}{0pt}%
\pgfpathmoveto{\pgfqpoint{4.965375in}{2.696000in}}%
\pgfpathlineto{\pgfqpoint{3.688410in}{0.645663in}}%
\pgfpathlineto{\pgfqpoint{4.965375in}{2.696000in}}%
\pgfpathclose%
\pgfusepath{fill}%
\end{pgfscope}%
\begin{pgfscope}%
\pgfpathrectangle{\pgfqpoint{0.017500in}{0.017500in}}{\pgfqpoint{6.218282in}{3.580372in}} %
\pgfusepath{clip}%
\pgfsetbuttcap%
\pgfsetroundjoin%
\definecolor{currentfill}{rgb}{0.909460,0.839386,0.800331}%
\pgfsetfillcolor{currentfill}%
\pgfsetlinewidth{0.000000pt}%
\definecolor{currentstroke}{rgb}{0.000000,0.000000,0.000000}%
\pgfsetstrokecolor{currentstroke}%
\pgfsetdash{}{0pt}%
\pgfpathmoveto{\pgfqpoint{4.965375in}{2.696000in}}%
\pgfpathlineto{\pgfqpoint{3.688410in}{0.645663in}}%
\pgfpathlineto{\pgfqpoint{4.965375in}{2.696000in}}%
\pgfpathclose%
\pgfusepath{fill}%
\end{pgfscope}%
\begin{pgfscope}%
\pgfpathrectangle{\pgfqpoint{0.017500in}{0.017500in}}{\pgfqpoint{6.218282in}{3.580372in}} %
\pgfusepath{clip}%
\pgfsetbuttcap%
\pgfsetroundjoin%
\definecolor{currentfill}{rgb}{0.909460,0.839386,0.800331}%
\pgfsetfillcolor{currentfill}%
\pgfsetlinewidth{0.000000pt}%
\definecolor{currentstroke}{rgb}{0.000000,0.000000,0.000000}%
\pgfsetstrokecolor{currentstroke}%
\pgfsetdash{}{0pt}%
\pgfpathmoveto{\pgfqpoint{4.965375in}{2.696000in}}%
\pgfpathlineto{\pgfqpoint{3.688410in}{0.645663in}}%
\pgfpathlineto{\pgfqpoint{4.965375in}{2.696000in}}%
\pgfpathclose%
\pgfusepath{fill}%
\end{pgfscope}%
\begin{pgfscope}%
\pgfpathrectangle{\pgfqpoint{0.017500in}{0.017500in}}{\pgfqpoint{6.218282in}{3.580372in}} %
\pgfusepath{clip}%
\pgfsetbuttcap%
\pgfsetroundjoin%
\definecolor{currentfill}{rgb}{0.909460,0.839386,0.800331}%
\pgfsetfillcolor{currentfill}%
\pgfsetlinewidth{0.000000pt}%
\definecolor{currentstroke}{rgb}{0.000000,0.000000,0.000000}%
\pgfsetstrokecolor{currentstroke}%
\pgfsetdash{}{0pt}%
\pgfpathmoveto{\pgfqpoint{4.965375in}{2.696000in}}%
\pgfpathlineto{\pgfqpoint{3.688410in}{0.645663in}}%
\pgfpathlineto{\pgfqpoint{4.965375in}{2.696000in}}%
\pgfpathclose%
\pgfusepath{fill}%
\end{pgfscope}%
\begin{pgfscope}%
\pgfpathrectangle{\pgfqpoint{0.017500in}{0.017500in}}{\pgfqpoint{6.218282in}{3.580372in}} %
\pgfusepath{clip}%
\pgfsetbuttcap%
\pgfsetroundjoin%
\definecolor{currentfill}{rgb}{0.909460,0.839386,0.800331}%
\pgfsetfillcolor{currentfill}%
\pgfsetlinewidth{0.000000pt}%
\definecolor{currentstroke}{rgb}{0.000000,0.000000,0.000000}%
\pgfsetstrokecolor{currentstroke}%
\pgfsetdash{}{0pt}%
\pgfpathmoveto{\pgfqpoint{4.965375in}{2.696000in}}%
\pgfpathlineto{\pgfqpoint{3.688410in}{0.645663in}}%
\pgfpathlineto{\pgfqpoint{4.965375in}{2.696000in}}%
\pgfpathclose%
\pgfusepath{fill}%
\end{pgfscope}%
\begin{pgfscope}%
\pgfpathrectangle{\pgfqpoint{0.017500in}{0.017500in}}{\pgfqpoint{6.218282in}{3.580372in}} %
\pgfusepath{clip}%
\pgfsetbuttcap%
\pgfsetroundjoin%
\definecolor{currentfill}{rgb}{0.909460,0.839386,0.800331}%
\pgfsetfillcolor{currentfill}%
\pgfsetlinewidth{0.000000pt}%
\definecolor{currentstroke}{rgb}{0.000000,0.000000,0.000000}%
\pgfsetstrokecolor{currentstroke}%
\pgfsetdash{}{0pt}%
\pgfpathmoveto{\pgfqpoint{4.965375in}{2.696000in}}%
\pgfpathlineto{\pgfqpoint{3.688410in}{0.645663in}}%
\pgfpathlineto{\pgfqpoint{4.965375in}{2.696000in}}%
\pgfpathclose%
\pgfusepath{fill}%
\end{pgfscope}%
\begin{pgfscope}%
\pgfpathrectangle{\pgfqpoint{0.017500in}{0.017500in}}{\pgfqpoint{6.218282in}{3.580372in}} %
\pgfusepath{clip}%
\pgfsetbuttcap%
\pgfsetroundjoin%
\definecolor{currentfill}{rgb}{0.909460,0.839386,0.800331}%
\pgfsetfillcolor{currentfill}%
\pgfsetlinewidth{0.000000pt}%
\definecolor{currentstroke}{rgb}{0.000000,0.000000,0.000000}%
\pgfsetstrokecolor{currentstroke}%
\pgfsetdash{}{0pt}%
\pgfpathmoveto{\pgfqpoint{4.965375in}{2.696000in}}%
\pgfpathlineto{\pgfqpoint{3.688410in}{0.645663in}}%
\pgfpathlineto{\pgfqpoint{4.965375in}{2.696000in}}%
\pgfpathclose%
\pgfusepath{fill}%
\end{pgfscope}%
\begin{pgfscope}%
\pgfpathrectangle{\pgfqpoint{0.017500in}{0.017500in}}{\pgfqpoint{6.218282in}{3.580372in}} %
\pgfusepath{clip}%
\pgfsetbuttcap%
\pgfsetroundjoin%
\definecolor{currentfill}{rgb}{0.909460,0.839386,0.800331}%
\pgfsetfillcolor{currentfill}%
\pgfsetlinewidth{0.000000pt}%
\definecolor{currentstroke}{rgb}{0.000000,0.000000,0.000000}%
\pgfsetstrokecolor{currentstroke}%
\pgfsetdash{}{0pt}%
\pgfpathmoveto{\pgfqpoint{4.965375in}{2.696000in}}%
\pgfpathlineto{\pgfqpoint{3.688410in}{0.645663in}}%
\pgfpathlineto{\pgfqpoint{4.965375in}{2.696000in}}%
\pgfpathclose%
\pgfusepath{fill}%
\end{pgfscope}%
\begin{pgfscope}%
\pgfpathrectangle{\pgfqpoint{0.017500in}{0.017500in}}{\pgfqpoint{6.218282in}{3.580372in}} %
\pgfusepath{clip}%
\pgfsetbuttcap%
\pgfsetroundjoin%
\definecolor{currentfill}{rgb}{0.909460,0.839386,0.800331}%
\pgfsetfillcolor{currentfill}%
\pgfsetlinewidth{0.000000pt}%
\definecolor{currentstroke}{rgb}{0.000000,0.000000,0.000000}%
\pgfsetstrokecolor{currentstroke}%
\pgfsetdash{}{0pt}%
\pgfpathmoveto{\pgfqpoint{4.965375in}{2.696000in}}%
\pgfpathlineto{\pgfqpoint{3.688410in}{0.645663in}}%
\pgfpathlineto{\pgfqpoint{4.965375in}{2.696000in}}%
\pgfpathclose%
\pgfusepath{fill}%
\end{pgfscope}%
\begin{pgfscope}%
\pgfpathrectangle{\pgfqpoint{0.017500in}{0.017500in}}{\pgfqpoint{6.218282in}{3.580372in}} %
\pgfusepath{clip}%
\pgfsetbuttcap%
\pgfsetroundjoin%
\definecolor{currentfill}{rgb}{0.909460,0.839386,0.800331}%
\pgfsetfillcolor{currentfill}%
\pgfsetlinewidth{0.000000pt}%
\definecolor{currentstroke}{rgb}{0.000000,0.000000,0.000000}%
\pgfsetstrokecolor{currentstroke}%
\pgfsetdash{}{0pt}%
\pgfpathmoveto{\pgfqpoint{4.965375in}{2.696000in}}%
\pgfpathlineto{\pgfqpoint{3.688410in}{0.645663in}}%
\pgfpathlineto{\pgfqpoint{4.965375in}{2.696000in}}%
\pgfpathclose%
\pgfusepath{fill}%
\end{pgfscope}%
\begin{pgfscope}%
\pgfpathrectangle{\pgfqpoint{0.017500in}{0.017500in}}{\pgfqpoint{6.218282in}{3.580372in}} %
\pgfusepath{clip}%
\pgfsetbuttcap%
\pgfsetroundjoin%
\definecolor{currentfill}{rgb}{0.909460,0.839386,0.800331}%
\pgfsetfillcolor{currentfill}%
\pgfsetlinewidth{0.000000pt}%
\definecolor{currentstroke}{rgb}{0.000000,0.000000,0.000000}%
\pgfsetstrokecolor{currentstroke}%
\pgfsetdash{}{0pt}%
\pgfpathmoveto{\pgfqpoint{4.965375in}{2.696000in}}%
\pgfpathlineto{\pgfqpoint{3.688410in}{0.645663in}}%
\pgfpathlineto{\pgfqpoint{4.965375in}{2.696000in}}%
\pgfpathclose%
\pgfusepath{fill}%
\end{pgfscope}%
\begin{pgfscope}%
\pgfpathrectangle{\pgfqpoint{0.017500in}{0.017500in}}{\pgfqpoint{6.218282in}{3.580372in}} %
\pgfusepath{clip}%
\pgfsetbuttcap%
\pgfsetroundjoin%
\definecolor{currentfill}{rgb}{0.909460,0.839386,0.800331}%
\pgfsetfillcolor{currentfill}%
\pgfsetlinewidth{0.000000pt}%
\definecolor{currentstroke}{rgb}{0.000000,0.000000,0.000000}%
\pgfsetstrokecolor{currentstroke}%
\pgfsetdash{}{0pt}%
\pgfpathmoveto{\pgfqpoint{4.965375in}{2.696000in}}%
\pgfpathlineto{\pgfqpoint{3.688410in}{0.645663in}}%
\pgfpathlineto{\pgfqpoint{4.965375in}{2.696000in}}%
\pgfpathclose%
\pgfusepath{fill}%
\end{pgfscope}%
\begin{pgfscope}%
\pgfpathrectangle{\pgfqpoint{0.017500in}{0.017500in}}{\pgfqpoint{6.218282in}{3.580372in}} %
\pgfusepath{clip}%
\pgfsetbuttcap%
\pgfsetroundjoin%
\definecolor{currentfill}{rgb}{0.909460,0.839386,0.800331}%
\pgfsetfillcolor{currentfill}%
\pgfsetlinewidth{0.000000pt}%
\definecolor{currentstroke}{rgb}{0.000000,0.000000,0.000000}%
\pgfsetstrokecolor{currentstroke}%
\pgfsetdash{}{0pt}%
\pgfpathmoveto{\pgfqpoint{4.965375in}{2.696000in}}%
\pgfpathlineto{\pgfqpoint{3.688410in}{0.645663in}}%
\pgfpathlineto{\pgfqpoint{4.965375in}{2.696000in}}%
\pgfpathclose%
\pgfusepath{fill}%
\end{pgfscope}%
\begin{pgfscope}%
\pgfpathrectangle{\pgfqpoint{0.017500in}{0.017500in}}{\pgfqpoint{6.218282in}{3.580372in}} %
\pgfusepath{clip}%
\pgfsetbuttcap%
\pgfsetroundjoin%
\definecolor{currentfill}{rgb}{0.909460,0.839386,0.800331}%
\pgfsetfillcolor{currentfill}%
\pgfsetlinewidth{0.000000pt}%
\definecolor{currentstroke}{rgb}{0.000000,0.000000,0.000000}%
\pgfsetstrokecolor{currentstroke}%
\pgfsetdash{}{0pt}%
\pgfpathmoveto{\pgfqpoint{4.965375in}{2.696000in}}%
\pgfpathlineto{\pgfqpoint{3.688410in}{0.645663in}}%
\pgfpathlineto{\pgfqpoint{4.965375in}{2.696000in}}%
\pgfpathclose%
\pgfusepath{fill}%
\end{pgfscope}%
\begin{pgfscope}%
\pgfpathrectangle{\pgfqpoint{0.017500in}{0.017500in}}{\pgfqpoint{6.218282in}{3.580372in}} %
\pgfusepath{clip}%
\pgfsetbuttcap%
\pgfsetroundjoin%
\definecolor{currentfill}{rgb}{0.909460,0.839386,0.800331}%
\pgfsetfillcolor{currentfill}%
\pgfsetlinewidth{0.000000pt}%
\definecolor{currentstroke}{rgb}{0.000000,0.000000,0.000000}%
\pgfsetstrokecolor{currentstroke}%
\pgfsetdash{}{0pt}%
\pgfpathmoveto{\pgfqpoint{4.965375in}{2.696000in}}%
\pgfpathlineto{\pgfqpoint{3.688410in}{0.645663in}}%
\pgfpathlineto{\pgfqpoint{4.965375in}{2.696000in}}%
\pgfpathclose%
\pgfusepath{fill}%
\end{pgfscope}%
\begin{pgfscope}%
\pgfpathrectangle{\pgfqpoint{0.017500in}{0.017500in}}{\pgfqpoint{6.218282in}{3.580372in}} %
\pgfusepath{clip}%
\pgfsetbuttcap%
\pgfsetroundjoin%
\definecolor{currentfill}{rgb}{0.909460,0.839386,0.800331}%
\pgfsetfillcolor{currentfill}%
\pgfsetlinewidth{0.000000pt}%
\definecolor{currentstroke}{rgb}{0.000000,0.000000,0.000000}%
\pgfsetstrokecolor{currentstroke}%
\pgfsetdash{}{0pt}%
\pgfpathmoveto{\pgfqpoint{4.965375in}{2.696000in}}%
\pgfpathlineto{\pgfqpoint{3.688410in}{0.645663in}}%
\pgfpathlineto{\pgfqpoint{4.965375in}{2.696000in}}%
\pgfpathclose%
\pgfusepath{fill}%
\end{pgfscope}%
\begin{pgfscope}%
\pgfpathrectangle{\pgfqpoint{0.017500in}{0.017500in}}{\pgfqpoint{6.218282in}{3.580372in}} %
\pgfusepath{clip}%
\pgfsetbuttcap%
\pgfsetroundjoin%
\definecolor{currentfill}{rgb}{0.909460,0.839386,0.800331}%
\pgfsetfillcolor{currentfill}%
\pgfsetlinewidth{0.000000pt}%
\definecolor{currentstroke}{rgb}{0.000000,0.000000,0.000000}%
\pgfsetstrokecolor{currentstroke}%
\pgfsetdash{}{0pt}%
\pgfpathmoveto{\pgfqpoint{4.965375in}{2.696000in}}%
\pgfpathlineto{\pgfqpoint{3.688410in}{0.645663in}}%
\pgfpathlineto{\pgfqpoint{4.965375in}{2.696000in}}%
\pgfpathclose%
\pgfusepath{fill}%
\end{pgfscope}%
\begin{pgfscope}%
\pgfpathrectangle{\pgfqpoint{0.017500in}{0.017500in}}{\pgfqpoint{6.218282in}{3.580372in}} %
\pgfusepath{clip}%
\pgfsetbuttcap%
\pgfsetroundjoin%
\definecolor{currentfill}{rgb}{0.909460,0.839386,0.800331}%
\pgfsetfillcolor{currentfill}%
\pgfsetlinewidth{0.000000pt}%
\definecolor{currentstroke}{rgb}{0.000000,0.000000,0.000000}%
\pgfsetstrokecolor{currentstroke}%
\pgfsetdash{}{0pt}%
\pgfpathmoveto{\pgfqpoint{4.965375in}{2.696000in}}%
\pgfpathlineto{\pgfqpoint{3.688410in}{0.645663in}}%
\pgfpathlineto{\pgfqpoint{4.965375in}{2.696000in}}%
\pgfpathclose%
\pgfusepath{fill}%
\end{pgfscope}%
\begin{pgfscope}%
\pgfpathrectangle{\pgfqpoint{0.017500in}{0.017500in}}{\pgfqpoint{6.218282in}{3.580372in}} %
\pgfusepath{clip}%
\pgfsetbuttcap%
\pgfsetroundjoin%
\definecolor{currentfill}{rgb}{0.909460,0.839386,0.800331}%
\pgfsetfillcolor{currentfill}%
\pgfsetlinewidth{0.000000pt}%
\definecolor{currentstroke}{rgb}{0.000000,0.000000,0.000000}%
\pgfsetstrokecolor{currentstroke}%
\pgfsetdash{}{0pt}%
\pgfpathmoveto{\pgfqpoint{4.965375in}{2.696000in}}%
\pgfpathlineto{\pgfqpoint{3.688410in}{0.645663in}}%
\pgfpathlineto{\pgfqpoint{4.965375in}{2.696000in}}%
\pgfpathclose%
\pgfusepath{fill}%
\end{pgfscope}%
\begin{pgfscope}%
\pgfpathrectangle{\pgfqpoint{0.017500in}{0.017500in}}{\pgfqpoint{6.218282in}{3.580372in}} %
\pgfusepath{clip}%
\pgfsetbuttcap%
\pgfsetroundjoin%
\definecolor{currentfill}{rgb}{0.909460,0.839386,0.800331}%
\pgfsetfillcolor{currentfill}%
\pgfsetlinewidth{0.000000pt}%
\definecolor{currentstroke}{rgb}{0.000000,0.000000,0.000000}%
\pgfsetstrokecolor{currentstroke}%
\pgfsetdash{}{0pt}%
\pgfpathmoveto{\pgfqpoint{4.965375in}{2.696000in}}%
\pgfpathlineto{\pgfqpoint{3.688410in}{0.645663in}}%
\pgfpathlineto{\pgfqpoint{4.965375in}{2.696000in}}%
\pgfpathclose%
\pgfusepath{fill}%
\end{pgfscope}%
\begin{pgfscope}%
\pgfpathrectangle{\pgfqpoint{0.017500in}{0.017500in}}{\pgfqpoint{6.218282in}{3.580372in}} %
\pgfusepath{clip}%
\pgfsetbuttcap%
\pgfsetroundjoin%
\definecolor{currentfill}{rgb}{0.909460,0.839386,0.800331}%
\pgfsetfillcolor{currentfill}%
\pgfsetlinewidth{0.000000pt}%
\definecolor{currentstroke}{rgb}{0.000000,0.000000,0.000000}%
\pgfsetstrokecolor{currentstroke}%
\pgfsetdash{}{0pt}%
\pgfpathmoveto{\pgfqpoint{4.965375in}{2.696000in}}%
\pgfpathlineto{\pgfqpoint{3.688410in}{0.645663in}}%
\pgfpathlineto{\pgfqpoint{4.965375in}{2.696000in}}%
\pgfpathclose%
\pgfusepath{fill}%
\end{pgfscope}%
\begin{pgfscope}%
\pgfpathrectangle{\pgfqpoint{0.017500in}{0.017500in}}{\pgfqpoint{6.218282in}{3.580372in}} %
\pgfusepath{clip}%
\pgfsetbuttcap%
\pgfsetroundjoin%
\definecolor{currentfill}{rgb}{0.909460,0.839386,0.800331}%
\pgfsetfillcolor{currentfill}%
\pgfsetlinewidth{0.000000pt}%
\definecolor{currentstroke}{rgb}{0.000000,0.000000,0.000000}%
\pgfsetstrokecolor{currentstroke}%
\pgfsetdash{}{0pt}%
\pgfpathmoveto{\pgfqpoint{4.965375in}{2.696000in}}%
\pgfpathlineto{\pgfqpoint{3.688410in}{0.645663in}}%
\pgfpathlineto{\pgfqpoint{4.965375in}{2.696000in}}%
\pgfpathclose%
\pgfusepath{fill}%
\end{pgfscope}%
\begin{pgfscope}%
\pgfpathrectangle{\pgfqpoint{0.017500in}{0.017500in}}{\pgfqpoint{6.218282in}{3.580372in}} %
\pgfusepath{clip}%
\pgfsetbuttcap%
\pgfsetroundjoin%
\definecolor{currentfill}{rgb}{0.909460,0.839386,0.800331}%
\pgfsetfillcolor{currentfill}%
\pgfsetlinewidth{0.000000pt}%
\definecolor{currentstroke}{rgb}{0.000000,0.000000,0.000000}%
\pgfsetstrokecolor{currentstroke}%
\pgfsetdash{}{0pt}%
\pgfpathmoveto{\pgfqpoint{4.965375in}{2.696000in}}%
\pgfpathlineto{\pgfqpoint{3.688410in}{0.645663in}}%
\pgfpathlineto{\pgfqpoint{4.965375in}{2.696000in}}%
\pgfpathclose%
\pgfusepath{fill}%
\end{pgfscope}%
\begin{pgfscope}%
\pgfpathrectangle{\pgfqpoint{0.017500in}{0.017500in}}{\pgfqpoint{6.218282in}{3.580372in}} %
\pgfusepath{clip}%
\pgfsetbuttcap%
\pgfsetroundjoin%
\definecolor{currentfill}{rgb}{0.909460,0.839386,0.800331}%
\pgfsetfillcolor{currentfill}%
\pgfsetlinewidth{0.000000pt}%
\definecolor{currentstroke}{rgb}{0.000000,0.000000,0.000000}%
\pgfsetstrokecolor{currentstroke}%
\pgfsetdash{}{0pt}%
\pgfpathmoveto{\pgfqpoint{4.965375in}{2.696000in}}%
\pgfpathlineto{\pgfqpoint{3.688410in}{0.645663in}}%
\pgfpathlineto{\pgfqpoint{4.965375in}{2.696000in}}%
\pgfpathclose%
\pgfusepath{fill}%
\end{pgfscope}%
\begin{pgfscope}%
\pgfpathrectangle{\pgfqpoint{0.017500in}{0.017500in}}{\pgfqpoint{6.218282in}{3.580372in}} %
\pgfusepath{clip}%
\pgfsetbuttcap%
\pgfsetroundjoin%
\definecolor{currentfill}{rgb}{0.909460,0.839386,0.800331}%
\pgfsetfillcolor{currentfill}%
\pgfsetlinewidth{0.000000pt}%
\definecolor{currentstroke}{rgb}{0.000000,0.000000,0.000000}%
\pgfsetstrokecolor{currentstroke}%
\pgfsetdash{}{0pt}%
\pgfpathmoveto{\pgfqpoint{4.965375in}{2.696000in}}%
\pgfpathlineto{\pgfqpoint{3.688410in}{0.645663in}}%
\pgfpathlineto{\pgfqpoint{4.965375in}{2.696000in}}%
\pgfpathclose%
\pgfusepath{fill}%
\end{pgfscope}%
\begin{pgfscope}%
\pgfpathrectangle{\pgfqpoint{0.017500in}{0.017500in}}{\pgfqpoint{6.218282in}{3.580372in}} %
\pgfusepath{clip}%
\pgfsetbuttcap%
\pgfsetroundjoin%
\definecolor{currentfill}{rgb}{0.884643,0.410017,0.322507}%
\pgfsetfillcolor{currentfill}%
\pgfsetlinewidth{0.000000pt}%
\definecolor{currentstroke}{rgb}{0.000000,0.000000,0.000000}%
\pgfsetstrokecolor{currentstroke}%
\pgfsetdash{}{0pt}%
\pgfpathmoveto{\pgfqpoint{1.702428in}{2.361240in}}%
\pgfpathlineto{\pgfqpoint{2.142608in}{2.314596in}}%
\pgfpathlineto{\pgfqpoint{2.410172in}{2.537974in}}%
\pgfpathlineto{\pgfqpoint{1.972254in}{2.569602in}}%
\pgfpathlineto{\pgfqpoint{1.702428in}{2.361240in}}%
\pgfpathclose%
\pgfusepath{fill}%
\end{pgfscope}%
\begin{pgfscope}%
\pgfpathrectangle{\pgfqpoint{0.017500in}{0.017500in}}{\pgfqpoint{6.218282in}{3.580372in}} %
\pgfusepath{clip}%
\pgfsetbuttcap%
\pgfsetroundjoin%
\definecolor{currentfill}{rgb}{0.884643,0.410017,0.322507}%
\pgfsetfillcolor{currentfill}%
\pgfsetlinewidth{0.000000pt}%
\definecolor{currentstroke}{rgb}{0.000000,0.000000,0.000000}%
\pgfsetstrokecolor{currentstroke}%
\pgfsetdash{}{0pt}%
\pgfpathmoveto{\pgfqpoint{1.702428in}{2.361240in}}%
\pgfpathlineto{\pgfqpoint{2.142608in}{2.314596in}}%
\pgfpathlineto{\pgfqpoint{2.410172in}{2.537974in}}%
\pgfpathlineto{\pgfqpoint{1.972254in}{2.569602in}}%
\pgfpathlineto{\pgfqpoint{1.702428in}{2.361240in}}%
\pgfpathclose%
\pgfusepath{fill}%
\end{pgfscope}%
\begin{pgfscope}%
\pgfpathrectangle{\pgfqpoint{0.017500in}{0.017500in}}{\pgfqpoint{6.218282in}{3.580372in}} %
\pgfusepath{clip}%
\pgfsetbuttcap%
\pgfsetroundjoin%
\definecolor{currentfill}{rgb}{0.884643,0.410017,0.322507}%
\pgfsetfillcolor{currentfill}%
\pgfsetlinewidth{0.000000pt}%
\definecolor{currentstroke}{rgb}{0.000000,0.000000,0.000000}%
\pgfsetstrokecolor{currentstroke}%
\pgfsetdash{}{0pt}%
\pgfpathmoveto{\pgfqpoint{1.702428in}{2.361240in}}%
\pgfpathlineto{\pgfqpoint{2.142608in}{2.314596in}}%
\pgfpathlineto{\pgfqpoint{2.410172in}{2.537974in}}%
\pgfpathlineto{\pgfqpoint{1.972254in}{2.569602in}}%
\pgfpathlineto{\pgfqpoint{1.702428in}{2.361240in}}%
\pgfpathclose%
\pgfusepath{fill}%
\end{pgfscope}%
\begin{pgfscope}%
\pgfpathrectangle{\pgfqpoint{0.017500in}{0.017500in}}{\pgfqpoint{6.218282in}{3.580372in}} %
\pgfusepath{clip}%
\pgfsetbuttcap%
\pgfsetroundjoin%
\definecolor{currentfill}{rgb}{0.884643,0.410017,0.322507}%
\pgfsetfillcolor{currentfill}%
\pgfsetlinewidth{0.000000pt}%
\definecolor{currentstroke}{rgb}{0.000000,0.000000,0.000000}%
\pgfsetstrokecolor{currentstroke}%
\pgfsetdash{}{0pt}%
\pgfpathmoveto{\pgfqpoint{1.702428in}{2.361240in}}%
\pgfpathlineto{\pgfqpoint{2.142608in}{2.314596in}}%
\pgfpathlineto{\pgfqpoint{2.410172in}{2.537974in}}%
\pgfpathlineto{\pgfqpoint{1.972254in}{2.569602in}}%
\pgfpathlineto{\pgfqpoint{1.702428in}{2.361240in}}%
\pgfpathclose%
\pgfusepath{fill}%
\end{pgfscope}%
\begin{pgfscope}%
\pgfpathrectangle{\pgfqpoint{0.017500in}{0.017500in}}{\pgfqpoint{6.218282in}{3.580372in}} %
\pgfusepath{clip}%
\pgfsetbuttcap%
\pgfsetroundjoin%
\definecolor{currentfill}{rgb}{0.884643,0.410017,0.322507}%
\pgfsetfillcolor{currentfill}%
\pgfsetlinewidth{0.000000pt}%
\definecolor{currentstroke}{rgb}{0.000000,0.000000,0.000000}%
\pgfsetstrokecolor{currentstroke}%
\pgfsetdash{}{0pt}%
\pgfpathmoveto{\pgfqpoint{1.702428in}{2.361240in}}%
\pgfpathlineto{\pgfqpoint{2.142608in}{2.314596in}}%
\pgfpathlineto{\pgfqpoint{2.410172in}{2.537974in}}%
\pgfpathlineto{\pgfqpoint{1.972254in}{2.569602in}}%
\pgfpathlineto{\pgfqpoint{1.702428in}{2.361240in}}%
\pgfpathclose%
\pgfusepath{fill}%
\end{pgfscope}%
\begin{pgfscope}%
\pgfpathrectangle{\pgfqpoint{0.017500in}{0.017500in}}{\pgfqpoint{6.218282in}{3.580372in}} %
\pgfusepath{clip}%
\pgfsetbuttcap%
\pgfsetroundjoin%
\definecolor{currentfill}{rgb}{0.884643,0.410017,0.322507}%
\pgfsetfillcolor{currentfill}%
\pgfsetlinewidth{0.000000pt}%
\definecolor{currentstroke}{rgb}{0.000000,0.000000,0.000000}%
\pgfsetstrokecolor{currentstroke}%
\pgfsetdash{}{0pt}%
\pgfpathmoveto{\pgfqpoint{1.702428in}{2.361240in}}%
\pgfpathlineto{\pgfqpoint{2.142608in}{2.314596in}}%
\pgfpathlineto{\pgfqpoint{2.410172in}{2.537974in}}%
\pgfpathlineto{\pgfqpoint{1.972254in}{2.569602in}}%
\pgfpathlineto{\pgfqpoint{1.702428in}{2.361240in}}%
\pgfpathclose%
\pgfusepath{fill}%
\end{pgfscope}%
\begin{pgfscope}%
\pgfpathrectangle{\pgfqpoint{0.017500in}{0.017500in}}{\pgfqpoint{6.218282in}{3.580372in}} %
\pgfusepath{clip}%
\pgfsetbuttcap%
\pgfsetroundjoin%
\definecolor{currentfill}{rgb}{0.740957,0.122240,0.175744}%
\pgfsetfillcolor{currentfill}%
\pgfsetlinewidth{0.000000pt}%
\definecolor{currentstroke}{rgb}{0.000000,0.000000,0.000000}%
\pgfsetstrokecolor{currentstroke}%
\pgfsetdash{}{0pt}%
\pgfpathmoveto{\pgfqpoint{3.564605in}{2.552204in}}%
\pgfpathlineto{\pgfqpoint{4.018512in}{2.470906in}}%
\pgfpathlineto{\pgfqpoint{4.268228in}{2.632966in}}%
\pgfpathlineto{\pgfqpoint{3.817122in}{2.696128in}}%
\pgfpathlineto{\pgfqpoint{3.564605in}{2.552204in}}%
\pgfpathclose%
\pgfusepath{fill}%
\end{pgfscope}%
\begin{pgfscope}%
\pgfpathrectangle{\pgfqpoint{0.017500in}{0.017500in}}{\pgfqpoint{6.218282in}{3.580372in}} %
\pgfusepath{clip}%
\pgfsetbuttcap%
\pgfsetroundjoin%
\definecolor{currentfill}{rgb}{0.740957,0.122240,0.175744}%
\pgfsetfillcolor{currentfill}%
\pgfsetlinewidth{0.000000pt}%
\definecolor{currentstroke}{rgb}{0.000000,0.000000,0.000000}%
\pgfsetstrokecolor{currentstroke}%
\pgfsetdash{}{0pt}%
\pgfpathmoveto{\pgfqpoint{3.564605in}{2.552204in}}%
\pgfpathlineto{\pgfqpoint{4.018512in}{2.470906in}}%
\pgfpathlineto{\pgfqpoint{4.268228in}{2.632966in}}%
\pgfpathlineto{\pgfqpoint{3.817122in}{2.696128in}}%
\pgfpathlineto{\pgfqpoint{3.564605in}{2.552204in}}%
\pgfpathclose%
\pgfusepath{fill}%
\end{pgfscope}%
\begin{pgfscope}%
\pgfpathrectangle{\pgfqpoint{0.017500in}{0.017500in}}{\pgfqpoint{6.218282in}{3.580372in}} %
\pgfusepath{clip}%
\pgfsetbuttcap%
\pgfsetroundjoin%
\definecolor{currentfill}{rgb}{0.740957,0.122240,0.175744}%
\pgfsetfillcolor{currentfill}%
\pgfsetlinewidth{0.000000pt}%
\definecolor{currentstroke}{rgb}{0.000000,0.000000,0.000000}%
\pgfsetstrokecolor{currentstroke}%
\pgfsetdash{}{0pt}%
\pgfpathmoveto{\pgfqpoint{3.564605in}{2.552204in}}%
\pgfpathlineto{\pgfqpoint{4.018512in}{2.470906in}}%
\pgfpathlineto{\pgfqpoint{4.268228in}{2.632966in}}%
\pgfpathlineto{\pgfqpoint{3.817122in}{2.696128in}}%
\pgfpathlineto{\pgfqpoint{3.564605in}{2.552204in}}%
\pgfpathclose%
\pgfusepath{fill}%
\end{pgfscope}%
\begin{pgfscope}%
\pgfpathrectangle{\pgfqpoint{0.017500in}{0.017500in}}{\pgfqpoint{6.218282in}{3.580372in}} %
\pgfusepath{clip}%
\pgfsetbuttcap%
\pgfsetroundjoin%
\definecolor{currentfill}{rgb}{0.740957,0.122240,0.175744}%
\pgfsetfillcolor{currentfill}%
\pgfsetlinewidth{0.000000pt}%
\definecolor{currentstroke}{rgb}{0.000000,0.000000,0.000000}%
\pgfsetstrokecolor{currentstroke}%
\pgfsetdash{}{0pt}%
\pgfpathmoveto{\pgfqpoint{3.564605in}{2.552204in}}%
\pgfpathlineto{\pgfqpoint{4.018512in}{2.470906in}}%
\pgfpathlineto{\pgfqpoint{4.268228in}{2.632966in}}%
\pgfpathlineto{\pgfqpoint{3.817122in}{2.696128in}}%
\pgfpathlineto{\pgfqpoint{3.564605in}{2.552204in}}%
\pgfpathclose%
\pgfusepath{fill}%
\end{pgfscope}%
\begin{pgfscope}%
\pgfpathrectangle{\pgfqpoint{0.017500in}{0.017500in}}{\pgfqpoint{6.218282in}{3.580372in}} %
\pgfusepath{clip}%
\pgfsetbuttcap%
\pgfsetroundjoin%
\definecolor{currentfill}{rgb}{0.740957,0.122240,0.175744}%
\pgfsetfillcolor{currentfill}%
\pgfsetlinewidth{0.000000pt}%
\definecolor{currentstroke}{rgb}{0.000000,0.000000,0.000000}%
\pgfsetstrokecolor{currentstroke}%
\pgfsetdash{}{0pt}%
\pgfpathmoveto{\pgfqpoint{3.564605in}{2.552204in}}%
\pgfpathlineto{\pgfqpoint{4.018512in}{2.470906in}}%
\pgfpathlineto{\pgfqpoint{4.268228in}{2.632966in}}%
\pgfpathlineto{\pgfqpoint{3.817122in}{2.696128in}}%
\pgfpathlineto{\pgfqpoint{3.564605in}{2.552204in}}%
\pgfpathclose%
\pgfusepath{fill}%
\end{pgfscope}%
\begin{pgfscope}%
\pgfpathrectangle{\pgfqpoint{0.017500in}{0.017500in}}{\pgfqpoint{6.218282in}{3.580372in}} %
\pgfusepath{clip}%
\pgfsetbuttcap%
\pgfsetroundjoin%
\definecolor{currentfill}{rgb}{0.740957,0.122240,0.175744}%
\pgfsetfillcolor{currentfill}%
\pgfsetlinewidth{0.000000pt}%
\definecolor{currentstroke}{rgb}{0.000000,0.000000,0.000000}%
\pgfsetstrokecolor{currentstroke}%
\pgfsetdash{}{0pt}%
\pgfpathmoveto{\pgfqpoint{3.564605in}{2.552204in}}%
\pgfpathlineto{\pgfqpoint{4.018512in}{2.470906in}}%
\pgfpathlineto{\pgfqpoint{4.268228in}{2.632966in}}%
\pgfpathlineto{\pgfqpoint{3.817122in}{2.696128in}}%
\pgfpathlineto{\pgfqpoint{3.564605in}{2.552204in}}%
\pgfpathclose%
\pgfusepath{fill}%
\end{pgfscope}%
\begin{pgfscope}%
\pgfpathrectangle{\pgfqpoint{0.017500in}{0.017500in}}{\pgfqpoint{6.218282in}{3.580372in}} %
\pgfusepath{clip}%
\pgfsetbuttcap%
\pgfsetroundjoin%
\definecolor{currentfill}{rgb}{0.705673,0.015556,0.150233}%
\pgfsetfillcolor{currentfill}%
\pgfsetlinewidth{0.000000pt}%
\definecolor{currentstroke}{rgb}{0.000000,0.000000,0.000000}%
\pgfsetstrokecolor{currentstroke}%
\pgfsetdash{}{0pt}%
\pgfpathmoveto{\pgfqpoint{4.724973in}{2.556562in}}%
\pgfpathlineto{\pgfqpoint{4.965375in}{2.696000in}}%
\pgfpathlineto{\pgfqpoint{4.724973in}{2.556562in}}%
\pgfpathclose%
\pgfusepath{fill}%
\end{pgfscope}%
\begin{pgfscope}%
\pgfpathrectangle{\pgfqpoint{0.017500in}{0.017500in}}{\pgfqpoint{6.218282in}{3.580372in}} %
\pgfusepath{clip}%
\pgfsetbuttcap%
\pgfsetroundjoin%
\definecolor{currentfill}{rgb}{0.705673,0.015556,0.150233}%
\pgfsetfillcolor{currentfill}%
\pgfsetlinewidth{0.000000pt}%
\definecolor{currentstroke}{rgb}{0.000000,0.000000,0.000000}%
\pgfsetstrokecolor{currentstroke}%
\pgfsetdash{}{0pt}%
\pgfpathmoveto{\pgfqpoint{4.724973in}{2.556562in}}%
\pgfpathlineto{\pgfqpoint{4.965375in}{2.696000in}}%
\pgfpathlineto{\pgfqpoint{4.724973in}{2.556562in}}%
\pgfpathclose%
\pgfusepath{fill}%
\end{pgfscope}%
\begin{pgfscope}%
\pgfpathrectangle{\pgfqpoint{0.017500in}{0.017500in}}{\pgfqpoint{6.218282in}{3.580372in}} %
\pgfusepath{clip}%
\pgfsetbuttcap%
\pgfsetroundjoin%
\definecolor{currentfill}{rgb}{0.705673,0.015556,0.150233}%
\pgfsetfillcolor{currentfill}%
\pgfsetlinewidth{0.000000pt}%
\definecolor{currentstroke}{rgb}{0.000000,0.000000,0.000000}%
\pgfsetstrokecolor{currentstroke}%
\pgfsetdash{}{0pt}%
\pgfpathmoveto{\pgfqpoint{4.724973in}{2.556562in}}%
\pgfpathlineto{\pgfqpoint{4.965375in}{2.696000in}}%
\pgfpathlineto{\pgfqpoint{4.724973in}{2.556562in}}%
\pgfpathclose%
\pgfusepath{fill}%
\end{pgfscope}%
\begin{pgfscope}%
\pgfpathrectangle{\pgfqpoint{0.017500in}{0.017500in}}{\pgfqpoint{6.218282in}{3.580372in}} %
\pgfusepath{clip}%
\pgfsetbuttcap%
\pgfsetroundjoin%
\definecolor{currentfill}{rgb}{0.705673,0.015556,0.150233}%
\pgfsetfillcolor{currentfill}%
\pgfsetlinewidth{0.000000pt}%
\definecolor{currentstroke}{rgb}{0.000000,0.000000,0.000000}%
\pgfsetstrokecolor{currentstroke}%
\pgfsetdash{}{0pt}%
\pgfpathmoveto{\pgfqpoint{4.724973in}{2.556562in}}%
\pgfpathlineto{\pgfqpoint{4.965375in}{2.696000in}}%
\pgfpathlineto{\pgfqpoint{4.724973in}{2.556562in}}%
\pgfpathclose%
\pgfusepath{fill}%
\end{pgfscope}%
\begin{pgfscope}%
\pgfpathrectangle{\pgfqpoint{0.017500in}{0.017500in}}{\pgfqpoint{6.218282in}{3.580372in}} %
\pgfusepath{clip}%
\pgfsetbuttcap%
\pgfsetroundjoin%
\definecolor{currentfill}{rgb}{0.705673,0.015556,0.150233}%
\pgfsetfillcolor{currentfill}%
\pgfsetlinewidth{0.000000pt}%
\definecolor{currentstroke}{rgb}{0.000000,0.000000,0.000000}%
\pgfsetstrokecolor{currentstroke}%
\pgfsetdash{}{0pt}%
\pgfpathmoveto{\pgfqpoint{4.724973in}{2.556562in}}%
\pgfpathlineto{\pgfqpoint{4.965375in}{2.696000in}}%
\pgfpathlineto{\pgfqpoint{4.724973in}{2.556562in}}%
\pgfpathclose%
\pgfusepath{fill}%
\end{pgfscope}%
\begin{pgfscope}%
\pgfpathrectangle{\pgfqpoint{0.017500in}{0.017500in}}{\pgfqpoint{6.218282in}{3.580372in}} %
\pgfusepath{clip}%
\pgfsetbuttcap%
\pgfsetroundjoin%
\definecolor{currentfill}{rgb}{0.705673,0.015556,0.150233}%
\pgfsetfillcolor{currentfill}%
\pgfsetlinewidth{0.000000pt}%
\definecolor{currentstroke}{rgb}{0.000000,0.000000,0.000000}%
\pgfsetstrokecolor{currentstroke}%
\pgfsetdash{}{0pt}%
\pgfpathmoveto{\pgfqpoint{4.724973in}{2.556562in}}%
\pgfpathlineto{\pgfqpoint{4.965375in}{2.696000in}}%
\pgfpathlineto{\pgfqpoint{4.724973in}{2.556562in}}%
\pgfpathclose%
\pgfusepath{fill}%
\end{pgfscope}%
\begin{pgfscope}%
\pgfpathrectangle{\pgfqpoint{0.017500in}{0.017500in}}{\pgfqpoint{6.218282in}{3.580372in}} %
\pgfusepath{clip}%
\pgfsetbuttcap%
\pgfsetroundjoin%
\definecolor{currentfill}{rgb}{0.705673,0.015556,0.150233}%
\pgfsetfillcolor{currentfill}%
\pgfsetlinewidth{0.000000pt}%
\definecolor{currentstroke}{rgb}{0.000000,0.000000,0.000000}%
\pgfsetstrokecolor{currentstroke}%
\pgfsetdash{}{0pt}%
\pgfpathmoveto{\pgfqpoint{4.724973in}{2.556562in}}%
\pgfpathlineto{\pgfqpoint{4.965375in}{2.696000in}}%
\pgfpathlineto{\pgfqpoint{4.724973in}{2.556562in}}%
\pgfpathclose%
\pgfusepath{fill}%
\end{pgfscope}%
\begin{pgfscope}%
\pgfpathrectangle{\pgfqpoint{0.017500in}{0.017500in}}{\pgfqpoint{6.218282in}{3.580372in}} %
\pgfusepath{clip}%
\pgfsetbuttcap%
\pgfsetroundjoin%
\definecolor{currentfill}{rgb}{0.705673,0.015556,0.150233}%
\pgfsetfillcolor{currentfill}%
\pgfsetlinewidth{0.000000pt}%
\definecolor{currentstroke}{rgb}{0.000000,0.000000,0.000000}%
\pgfsetstrokecolor{currentstroke}%
\pgfsetdash{}{0pt}%
\pgfpathmoveto{\pgfqpoint{4.724973in}{2.556562in}}%
\pgfpathlineto{\pgfqpoint{4.965375in}{2.696000in}}%
\pgfpathlineto{\pgfqpoint{4.724973in}{2.556562in}}%
\pgfpathclose%
\pgfusepath{fill}%
\end{pgfscope}%
\begin{pgfscope}%
\pgfpathrectangle{\pgfqpoint{0.017500in}{0.017500in}}{\pgfqpoint{6.218282in}{3.580372in}} %
\pgfusepath{clip}%
\pgfsetbuttcap%
\pgfsetroundjoin%
\definecolor{currentfill}{rgb}{0.705673,0.015556,0.150233}%
\pgfsetfillcolor{currentfill}%
\pgfsetlinewidth{0.000000pt}%
\definecolor{currentstroke}{rgb}{0.000000,0.000000,0.000000}%
\pgfsetstrokecolor{currentstroke}%
\pgfsetdash{}{0pt}%
\pgfpathmoveto{\pgfqpoint{4.724973in}{2.556562in}}%
\pgfpathlineto{\pgfqpoint{4.965375in}{2.696000in}}%
\pgfpathlineto{\pgfqpoint{4.724973in}{2.556562in}}%
\pgfpathclose%
\pgfusepath{fill}%
\end{pgfscope}%
\begin{pgfscope}%
\pgfpathrectangle{\pgfqpoint{0.017500in}{0.017500in}}{\pgfqpoint{6.218282in}{3.580372in}} %
\pgfusepath{clip}%
\pgfsetbuttcap%
\pgfsetroundjoin%
\definecolor{currentfill}{rgb}{0.705673,0.015556,0.150233}%
\pgfsetfillcolor{currentfill}%
\pgfsetlinewidth{0.000000pt}%
\definecolor{currentstroke}{rgb}{0.000000,0.000000,0.000000}%
\pgfsetstrokecolor{currentstroke}%
\pgfsetdash{}{0pt}%
\pgfpathmoveto{\pgfqpoint{4.724973in}{2.556562in}}%
\pgfpathlineto{\pgfqpoint{4.965375in}{2.696000in}}%
\pgfpathlineto{\pgfqpoint{4.724973in}{2.556562in}}%
\pgfpathclose%
\pgfusepath{fill}%
\end{pgfscope}%
\begin{pgfscope}%
\pgfpathrectangle{\pgfqpoint{0.017500in}{0.017500in}}{\pgfqpoint{6.218282in}{3.580372in}} %
\pgfusepath{clip}%
\pgfsetbuttcap%
\pgfsetroundjoin%
\definecolor{currentfill}{rgb}{0.705673,0.015556,0.150233}%
\pgfsetfillcolor{currentfill}%
\pgfsetlinewidth{0.000000pt}%
\definecolor{currentstroke}{rgb}{0.000000,0.000000,0.000000}%
\pgfsetstrokecolor{currentstroke}%
\pgfsetdash{}{0pt}%
\pgfpathmoveto{\pgfqpoint{4.724973in}{2.556562in}}%
\pgfpathlineto{\pgfqpoint{4.965375in}{2.696000in}}%
\pgfpathlineto{\pgfqpoint{4.724973in}{2.556562in}}%
\pgfpathclose%
\pgfusepath{fill}%
\end{pgfscope}%
\begin{pgfscope}%
\pgfpathrectangle{\pgfqpoint{0.017500in}{0.017500in}}{\pgfqpoint{6.218282in}{3.580372in}} %
\pgfusepath{clip}%
\pgfsetbuttcap%
\pgfsetroundjoin%
\definecolor{currentfill}{rgb}{0.705673,0.015556,0.150233}%
\pgfsetfillcolor{currentfill}%
\pgfsetlinewidth{0.000000pt}%
\definecolor{currentstroke}{rgb}{0.000000,0.000000,0.000000}%
\pgfsetstrokecolor{currentstroke}%
\pgfsetdash{}{0pt}%
\pgfpathmoveto{\pgfqpoint{4.724973in}{2.556562in}}%
\pgfpathlineto{\pgfqpoint{4.965375in}{2.696000in}}%
\pgfpathlineto{\pgfqpoint{4.724973in}{2.556562in}}%
\pgfpathclose%
\pgfusepath{fill}%
\end{pgfscope}%
\begin{pgfscope}%
\pgfpathrectangle{\pgfqpoint{0.017500in}{0.017500in}}{\pgfqpoint{6.218282in}{3.580372in}} %
\pgfusepath{clip}%
\pgfsetbuttcap%
\pgfsetroundjoin%
\definecolor{currentfill}{rgb}{0.705673,0.015556,0.150233}%
\pgfsetfillcolor{currentfill}%
\pgfsetlinewidth{0.000000pt}%
\definecolor{currentstroke}{rgb}{0.000000,0.000000,0.000000}%
\pgfsetstrokecolor{currentstroke}%
\pgfsetdash{}{0pt}%
\pgfpathmoveto{\pgfqpoint{4.724973in}{2.556562in}}%
\pgfpathlineto{\pgfqpoint{4.965375in}{2.696000in}}%
\pgfpathlineto{\pgfqpoint{4.724973in}{2.556562in}}%
\pgfpathclose%
\pgfusepath{fill}%
\end{pgfscope}%
\begin{pgfscope}%
\pgfpathrectangle{\pgfqpoint{0.017500in}{0.017500in}}{\pgfqpoint{6.218282in}{3.580372in}} %
\pgfusepath{clip}%
\pgfsetbuttcap%
\pgfsetroundjoin%
\definecolor{currentfill}{rgb}{0.705673,0.015556,0.150233}%
\pgfsetfillcolor{currentfill}%
\pgfsetlinewidth{0.000000pt}%
\definecolor{currentstroke}{rgb}{0.000000,0.000000,0.000000}%
\pgfsetstrokecolor{currentstroke}%
\pgfsetdash{}{0pt}%
\pgfpathmoveto{\pgfqpoint{4.724973in}{2.556562in}}%
\pgfpathlineto{\pgfqpoint{4.965375in}{2.696000in}}%
\pgfpathlineto{\pgfqpoint{4.724973in}{2.556562in}}%
\pgfpathclose%
\pgfusepath{fill}%
\end{pgfscope}%
\begin{pgfscope}%
\pgfpathrectangle{\pgfqpoint{0.017500in}{0.017500in}}{\pgfqpoint{6.218282in}{3.580372in}} %
\pgfusepath{clip}%
\pgfsetbuttcap%
\pgfsetroundjoin%
\definecolor{currentfill}{rgb}{0.705673,0.015556,0.150233}%
\pgfsetfillcolor{currentfill}%
\pgfsetlinewidth{0.000000pt}%
\definecolor{currentstroke}{rgb}{0.000000,0.000000,0.000000}%
\pgfsetstrokecolor{currentstroke}%
\pgfsetdash{}{0pt}%
\pgfpathmoveto{\pgfqpoint{4.724973in}{2.556562in}}%
\pgfpathlineto{\pgfqpoint{4.965375in}{2.696000in}}%
\pgfpathlineto{\pgfqpoint{4.724973in}{2.556562in}}%
\pgfpathclose%
\pgfusepath{fill}%
\end{pgfscope}%
\begin{pgfscope}%
\pgfpathrectangle{\pgfqpoint{0.017500in}{0.017500in}}{\pgfqpoint{6.218282in}{3.580372in}} %
\pgfusepath{clip}%
\pgfsetbuttcap%
\pgfsetroundjoin%
\definecolor{currentfill}{rgb}{0.705673,0.015556,0.150233}%
\pgfsetfillcolor{currentfill}%
\pgfsetlinewidth{0.000000pt}%
\definecolor{currentstroke}{rgb}{0.000000,0.000000,0.000000}%
\pgfsetstrokecolor{currentstroke}%
\pgfsetdash{}{0pt}%
\pgfpathmoveto{\pgfqpoint{4.724973in}{2.556562in}}%
\pgfpathlineto{\pgfqpoint{4.965375in}{2.696000in}}%
\pgfpathlineto{\pgfqpoint{4.724973in}{2.556562in}}%
\pgfpathclose%
\pgfusepath{fill}%
\end{pgfscope}%
\begin{pgfscope}%
\pgfpathrectangle{\pgfqpoint{0.017500in}{0.017500in}}{\pgfqpoint{6.218282in}{3.580372in}} %
\pgfusepath{clip}%
\pgfsetbuttcap%
\pgfsetroundjoin%
\definecolor{currentfill}{rgb}{0.705673,0.015556,0.150233}%
\pgfsetfillcolor{currentfill}%
\pgfsetlinewidth{0.000000pt}%
\definecolor{currentstroke}{rgb}{0.000000,0.000000,0.000000}%
\pgfsetstrokecolor{currentstroke}%
\pgfsetdash{}{0pt}%
\pgfpathmoveto{\pgfqpoint{4.724973in}{2.556562in}}%
\pgfpathlineto{\pgfqpoint{4.965375in}{2.696000in}}%
\pgfpathlineto{\pgfqpoint{4.724973in}{2.556562in}}%
\pgfpathclose%
\pgfusepath{fill}%
\end{pgfscope}%
\begin{pgfscope}%
\pgfpathrectangle{\pgfqpoint{0.017500in}{0.017500in}}{\pgfqpoint{6.218282in}{3.580372in}} %
\pgfusepath{clip}%
\pgfsetbuttcap%
\pgfsetroundjoin%
\definecolor{currentfill}{rgb}{0.705673,0.015556,0.150233}%
\pgfsetfillcolor{currentfill}%
\pgfsetlinewidth{0.000000pt}%
\definecolor{currentstroke}{rgb}{0.000000,0.000000,0.000000}%
\pgfsetstrokecolor{currentstroke}%
\pgfsetdash{}{0pt}%
\pgfpathmoveto{\pgfqpoint{4.724973in}{2.556562in}}%
\pgfpathlineto{\pgfqpoint{4.965375in}{2.696000in}}%
\pgfpathlineto{\pgfqpoint{4.724973in}{2.556562in}}%
\pgfpathclose%
\pgfusepath{fill}%
\end{pgfscope}%
\begin{pgfscope}%
\pgfpathrectangle{\pgfqpoint{0.017500in}{0.017500in}}{\pgfqpoint{6.218282in}{3.580372in}} %
\pgfusepath{clip}%
\pgfsetbuttcap%
\pgfsetroundjoin%
\definecolor{currentfill}{rgb}{0.705673,0.015556,0.150233}%
\pgfsetfillcolor{currentfill}%
\pgfsetlinewidth{0.000000pt}%
\definecolor{currentstroke}{rgb}{0.000000,0.000000,0.000000}%
\pgfsetstrokecolor{currentstroke}%
\pgfsetdash{}{0pt}%
\pgfpathmoveto{\pgfqpoint{4.724973in}{2.556562in}}%
\pgfpathlineto{\pgfqpoint{4.965375in}{2.696000in}}%
\pgfpathlineto{\pgfqpoint{4.724973in}{2.556562in}}%
\pgfpathclose%
\pgfusepath{fill}%
\end{pgfscope}%
\begin{pgfscope}%
\pgfpathrectangle{\pgfqpoint{0.017500in}{0.017500in}}{\pgfqpoint{6.218282in}{3.580372in}} %
\pgfusepath{clip}%
\pgfsetbuttcap%
\pgfsetroundjoin%
\definecolor{currentfill}{rgb}{0.705673,0.015556,0.150233}%
\pgfsetfillcolor{currentfill}%
\pgfsetlinewidth{0.000000pt}%
\definecolor{currentstroke}{rgb}{0.000000,0.000000,0.000000}%
\pgfsetstrokecolor{currentstroke}%
\pgfsetdash{}{0pt}%
\pgfpathmoveto{\pgfqpoint{4.724973in}{2.556562in}}%
\pgfpathlineto{\pgfqpoint{4.965375in}{2.696000in}}%
\pgfpathlineto{\pgfqpoint{4.724973in}{2.556562in}}%
\pgfpathclose%
\pgfusepath{fill}%
\end{pgfscope}%
\begin{pgfscope}%
\pgfpathrectangle{\pgfqpoint{0.017500in}{0.017500in}}{\pgfqpoint{6.218282in}{3.580372in}} %
\pgfusepath{clip}%
\pgfsetbuttcap%
\pgfsetroundjoin%
\definecolor{currentfill}{rgb}{0.705673,0.015556,0.150233}%
\pgfsetfillcolor{currentfill}%
\pgfsetlinewidth{0.000000pt}%
\definecolor{currentstroke}{rgb}{0.000000,0.000000,0.000000}%
\pgfsetstrokecolor{currentstroke}%
\pgfsetdash{}{0pt}%
\pgfpathmoveto{\pgfqpoint{4.724973in}{2.556562in}}%
\pgfpathlineto{\pgfqpoint{4.965375in}{2.696000in}}%
\pgfpathlineto{\pgfqpoint{4.724973in}{2.556562in}}%
\pgfpathclose%
\pgfusepath{fill}%
\end{pgfscope}%
\begin{pgfscope}%
\pgfpathrectangle{\pgfqpoint{0.017500in}{0.017500in}}{\pgfqpoint{6.218282in}{3.580372in}} %
\pgfusepath{clip}%
\pgfsetbuttcap%
\pgfsetroundjoin%
\definecolor{currentfill}{rgb}{0.705673,0.015556,0.150233}%
\pgfsetfillcolor{currentfill}%
\pgfsetlinewidth{0.000000pt}%
\definecolor{currentstroke}{rgb}{0.000000,0.000000,0.000000}%
\pgfsetstrokecolor{currentstroke}%
\pgfsetdash{}{0pt}%
\pgfpathmoveto{\pgfqpoint{4.724973in}{2.556562in}}%
\pgfpathlineto{\pgfqpoint{4.965375in}{2.696000in}}%
\pgfpathlineto{\pgfqpoint{4.724973in}{2.556562in}}%
\pgfpathclose%
\pgfusepath{fill}%
\end{pgfscope}%
\begin{pgfscope}%
\pgfpathrectangle{\pgfqpoint{0.017500in}{0.017500in}}{\pgfqpoint{6.218282in}{3.580372in}} %
\pgfusepath{clip}%
\pgfsetbuttcap%
\pgfsetroundjoin%
\definecolor{currentfill}{rgb}{0.705673,0.015556,0.150233}%
\pgfsetfillcolor{currentfill}%
\pgfsetlinewidth{0.000000pt}%
\definecolor{currentstroke}{rgb}{0.000000,0.000000,0.000000}%
\pgfsetstrokecolor{currentstroke}%
\pgfsetdash{}{0pt}%
\pgfpathmoveto{\pgfqpoint{4.724973in}{2.556562in}}%
\pgfpathlineto{\pgfqpoint{4.965375in}{2.696000in}}%
\pgfpathlineto{\pgfqpoint{4.724973in}{2.556562in}}%
\pgfpathclose%
\pgfusepath{fill}%
\end{pgfscope}%
\begin{pgfscope}%
\pgfpathrectangle{\pgfqpoint{0.017500in}{0.017500in}}{\pgfqpoint{6.218282in}{3.580372in}} %
\pgfusepath{clip}%
\pgfsetbuttcap%
\pgfsetroundjoin%
\definecolor{currentfill}{rgb}{0.705673,0.015556,0.150233}%
\pgfsetfillcolor{currentfill}%
\pgfsetlinewidth{0.000000pt}%
\definecolor{currentstroke}{rgb}{0.000000,0.000000,0.000000}%
\pgfsetstrokecolor{currentstroke}%
\pgfsetdash{}{0pt}%
\pgfpathmoveto{\pgfqpoint{4.724973in}{2.556562in}}%
\pgfpathlineto{\pgfqpoint{4.965375in}{2.696000in}}%
\pgfpathlineto{\pgfqpoint{4.724973in}{2.556562in}}%
\pgfpathclose%
\pgfusepath{fill}%
\end{pgfscope}%
\begin{pgfscope}%
\pgfpathrectangle{\pgfqpoint{0.017500in}{0.017500in}}{\pgfqpoint{6.218282in}{3.580372in}} %
\pgfusepath{clip}%
\pgfsetbuttcap%
\pgfsetroundjoin%
\definecolor{currentfill}{rgb}{0.705673,0.015556,0.150233}%
\pgfsetfillcolor{currentfill}%
\pgfsetlinewidth{0.000000pt}%
\definecolor{currentstroke}{rgb}{0.000000,0.000000,0.000000}%
\pgfsetstrokecolor{currentstroke}%
\pgfsetdash{}{0pt}%
\pgfpathmoveto{\pgfqpoint{4.724973in}{2.556562in}}%
\pgfpathlineto{\pgfqpoint{4.965375in}{2.696000in}}%
\pgfpathlineto{\pgfqpoint{4.724973in}{2.556562in}}%
\pgfpathclose%
\pgfusepath{fill}%
\end{pgfscope}%
\begin{pgfscope}%
\pgfpathrectangle{\pgfqpoint{0.017500in}{0.017500in}}{\pgfqpoint{6.218282in}{3.580372in}} %
\pgfusepath{clip}%
\pgfsetbuttcap%
\pgfsetroundjoin%
\definecolor{currentfill}{rgb}{0.705673,0.015556,0.150233}%
\pgfsetfillcolor{currentfill}%
\pgfsetlinewidth{0.000000pt}%
\definecolor{currentstroke}{rgb}{0.000000,0.000000,0.000000}%
\pgfsetstrokecolor{currentstroke}%
\pgfsetdash{}{0pt}%
\pgfpathmoveto{\pgfqpoint{4.724973in}{2.556562in}}%
\pgfpathlineto{\pgfqpoint{4.965375in}{2.696000in}}%
\pgfpathlineto{\pgfqpoint{4.724973in}{2.556562in}}%
\pgfpathclose%
\pgfusepath{fill}%
\end{pgfscope}%
\begin{pgfscope}%
\pgfpathrectangle{\pgfqpoint{0.017500in}{0.017500in}}{\pgfqpoint{6.218282in}{3.580372in}} %
\pgfusepath{clip}%
\pgfsetbuttcap%
\pgfsetroundjoin%
\definecolor{currentfill}{rgb}{0.705673,0.015556,0.150233}%
\pgfsetfillcolor{currentfill}%
\pgfsetlinewidth{0.000000pt}%
\definecolor{currentstroke}{rgb}{0.000000,0.000000,0.000000}%
\pgfsetstrokecolor{currentstroke}%
\pgfsetdash{}{0pt}%
\pgfpathmoveto{\pgfqpoint{4.724973in}{2.556562in}}%
\pgfpathlineto{\pgfqpoint{4.965375in}{2.696000in}}%
\pgfpathlineto{\pgfqpoint{4.724973in}{2.556562in}}%
\pgfpathclose%
\pgfusepath{fill}%
\end{pgfscope}%
\begin{pgfscope}%
\pgfpathrectangle{\pgfqpoint{0.017500in}{0.017500in}}{\pgfqpoint{6.218282in}{3.580372in}} %
\pgfusepath{clip}%
\pgfsetbuttcap%
\pgfsetroundjoin%
\definecolor{currentfill}{rgb}{0.705673,0.015556,0.150233}%
\pgfsetfillcolor{currentfill}%
\pgfsetlinewidth{0.000000pt}%
\definecolor{currentstroke}{rgb}{0.000000,0.000000,0.000000}%
\pgfsetstrokecolor{currentstroke}%
\pgfsetdash{}{0pt}%
\pgfpathmoveto{\pgfqpoint{4.724973in}{2.556562in}}%
\pgfpathlineto{\pgfqpoint{4.965375in}{2.696000in}}%
\pgfpathlineto{\pgfqpoint{4.724973in}{2.556562in}}%
\pgfpathclose%
\pgfusepath{fill}%
\end{pgfscope}%
\begin{pgfscope}%
\pgfpathrectangle{\pgfqpoint{0.017500in}{0.017500in}}{\pgfqpoint{6.218282in}{3.580372in}} %
\pgfusepath{clip}%
\pgfsetbuttcap%
\pgfsetroundjoin%
\definecolor{currentfill}{rgb}{0.705673,0.015556,0.150233}%
\pgfsetfillcolor{currentfill}%
\pgfsetlinewidth{0.000000pt}%
\definecolor{currentstroke}{rgb}{0.000000,0.000000,0.000000}%
\pgfsetstrokecolor{currentstroke}%
\pgfsetdash{}{0pt}%
\pgfpathmoveto{\pgfqpoint{4.724973in}{2.556562in}}%
\pgfpathlineto{\pgfqpoint{4.965375in}{2.696000in}}%
\pgfpathlineto{\pgfqpoint{4.724973in}{2.556562in}}%
\pgfpathclose%
\pgfusepath{fill}%
\end{pgfscope}%
\begin{pgfscope}%
\pgfpathrectangle{\pgfqpoint{0.017500in}{0.017500in}}{\pgfqpoint{6.218282in}{3.580372in}} %
\pgfusepath{clip}%
\pgfsetbuttcap%
\pgfsetroundjoin%
\definecolor{currentfill}{rgb}{0.705673,0.015556,0.150233}%
\pgfsetfillcolor{currentfill}%
\pgfsetlinewidth{0.000000pt}%
\definecolor{currentstroke}{rgb}{0.000000,0.000000,0.000000}%
\pgfsetstrokecolor{currentstroke}%
\pgfsetdash{}{0pt}%
\pgfpathmoveto{\pgfqpoint{4.724973in}{2.556562in}}%
\pgfpathlineto{\pgfqpoint{4.965375in}{2.696000in}}%
\pgfpathlineto{\pgfqpoint{4.724973in}{2.556562in}}%
\pgfpathclose%
\pgfusepath{fill}%
\end{pgfscope}%
\begin{pgfscope}%
\pgfpathrectangle{\pgfqpoint{0.017500in}{0.017500in}}{\pgfqpoint{6.218282in}{3.580372in}} %
\pgfusepath{clip}%
\pgfsetbuttcap%
\pgfsetroundjoin%
\definecolor{currentfill}{rgb}{0.343278,0.459354,0.884122}%
\pgfsetfillcolor{currentfill}%
\pgfsetlinewidth{0.000000pt}%
\definecolor{currentstroke}{rgb}{0.000000,0.000000,0.000000}%
\pgfsetstrokecolor{currentstroke}%
\pgfsetdash{}{0pt}%
\pgfpathmoveto{\pgfqpoint{2.349703in}{0.878379in}}%
\pgfpathlineto{\pgfqpoint{2.789661in}{0.820654in}}%
\pgfpathlineto{\pgfqpoint{3.043201in}{2.166904in}}%
\pgfpathlineto{\pgfqpoint{2.590029in}{2.232729in}}%
\pgfpathlineto{\pgfqpoint{2.349703in}{0.878379in}}%
\pgfpathclose%
\pgfusepath{fill}%
\end{pgfscope}%
\begin{pgfscope}%
\pgfpathrectangle{\pgfqpoint{0.017500in}{0.017500in}}{\pgfqpoint{6.218282in}{3.580372in}} %
\pgfusepath{clip}%
\pgfsetbuttcap%
\pgfsetroundjoin%
\definecolor{currentfill}{rgb}{0.343278,0.459354,0.884122}%
\pgfsetfillcolor{currentfill}%
\pgfsetlinewidth{0.000000pt}%
\definecolor{currentstroke}{rgb}{0.000000,0.000000,0.000000}%
\pgfsetstrokecolor{currentstroke}%
\pgfsetdash{}{0pt}%
\pgfpathmoveto{\pgfqpoint{2.349703in}{0.878379in}}%
\pgfpathlineto{\pgfqpoint{2.789661in}{0.820654in}}%
\pgfpathlineto{\pgfqpoint{3.043201in}{2.166904in}}%
\pgfpathlineto{\pgfqpoint{2.590029in}{2.232729in}}%
\pgfpathlineto{\pgfqpoint{2.349703in}{0.878379in}}%
\pgfpathclose%
\pgfusepath{fill}%
\end{pgfscope}%
\begin{pgfscope}%
\pgfpathrectangle{\pgfqpoint{0.017500in}{0.017500in}}{\pgfqpoint{6.218282in}{3.580372in}} %
\pgfusepath{clip}%
\pgfsetbuttcap%
\pgfsetroundjoin%
\definecolor{currentfill}{rgb}{0.343278,0.459354,0.884122}%
\pgfsetfillcolor{currentfill}%
\pgfsetlinewidth{0.000000pt}%
\definecolor{currentstroke}{rgb}{0.000000,0.000000,0.000000}%
\pgfsetstrokecolor{currentstroke}%
\pgfsetdash{}{0pt}%
\pgfpathmoveto{\pgfqpoint{2.349703in}{0.878379in}}%
\pgfpathlineto{\pgfqpoint{2.789661in}{0.820654in}}%
\pgfpathlineto{\pgfqpoint{3.043201in}{2.166904in}}%
\pgfpathlineto{\pgfqpoint{2.590029in}{2.232729in}}%
\pgfpathlineto{\pgfqpoint{2.349703in}{0.878379in}}%
\pgfpathclose%
\pgfusepath{fill}%
\end{pgfscope}%
\begin{pgfscope}%
\pgfpathrectangle{\pgfqpoint{0.017500in}{0.017500in}}{\pgfqpoint{6.218282in}{3.580372in}} %
\pgfusepath{clip}%
\pgfsetbuttcap%
\pgfsetroundjoin%
\definecolor{currentfill}{rgb}{0.343278,0.459354,0.884122}%
\pgfsetfillcolor{currentfill}%
\pgfsetlinewidth{0.000000pt}%
\definecolor{currentstroke}{rgb}{0.000000,0.000000,0.000000}%
\pgfsetstrokecolor{currentstroke}%
\pgfsetdash{}{0pt}%
\pgfpathmoveto{\pgfqpoint{2.349703in}{0.878379in}}%
\pgfpathlineto{\pgfqpoint{2.789661in}{0.820654in}}%
\pgfpathlineto{\pgfqpoint{3.043201in}{2.166904in}}%
\pgfpathlineto{\pgfqpoint{2.590029in}{2.232729in}}%
\pgfpathlineto{\pgfqpoint{2.349703in}{0.878379in}}%
\pgfpathclose%
\pgfusepath{fill}%
\end{pgfscope}%
\begin{pgfscope}%
\pgfpathrectangle{\pgfqpoint{0.017500in}{0.017500in}}{\pgfqpoint{6.218282in}{3.580372in}} %
\pgfusepath{clip}%
\pgfsetbuttcap%
\pgfsetroundjoin%
\definecolor{currentfill}{rgb}{0.343278,0.459354,0.884122}%
\pgfsetfillcolor{currentfill}%
\pgfsetlinewidth{0.000000pt}%
\definecolor{currentstroke}{rgb}{0.000000,0.000000,0.000000}%
\pgfsetstrokecolor{currentstroke}%
\pgfsetdash{}{0pt}%
\pgfpathmoveto{\pgfqpoint{2.349703in}{0.878379in}}%
\pgfpathlineto{\pgfqpoint{2.789661in}{0.820654in}}%
\pgfpathlineto{\pgfqpoint{3.043201in}{2.166904in}}%
\pgfpathlineto{\pgfqpoint{2.590029in}{2.232729in}}%
\pgfpathlineto{\pgfqpoint{2.349703in}{0.878379in}}%
\pgfpathclose%
\pgfusepath{fill}%
\end{pgfscope}%
\begin{pgfscope}%
\pgfpathrectangle{\pgfqpoint{0.017500in}{0.017500in}}{\pgfqpoint{6.218282in}{3.580372in}} %
\pgfusepath{clip}%
\pgfsetbuttcap%
\pgfsetroundjoin%
\definecolor{currentfill}{rgb}{0.343278,0.459354,0.884122}%
\pgfsetfillcolor{currentfill}%
\pgfsetlinewidth{0.000000pt}%
\definecolor{currentstroke}{rgb}{0.000000,0.000000,0.000000}%
\pgfsetstrokecolor{currentstroke}%
\pgfsetdash{}{0pt}%
\pgfpathmoveto{\pgfqpoint{2.349703in}{0.878379in}}%
\pgfpathlineto{\pgfqpoint{2.789661in}{0.820654in}}%
\pgfpathlineto{\pgfqpoint{3.043201in}{2.166904in}}%
\pgfpathlineto{\pgfqpoint{2.590029in}{2.232729in}}%
\pgfpathlineto{\pgfqpoint{2.349703in}{0.878379in}}%
\pgfpathclose%
\pgfusepath{fill}%
\end{pgfscope}%
\begin{pgfscope}%
\pgfpathrectangle{\pgfqpoint{0.017500in}{0.017500in}}{\pgfqpoint{6.218282in}{3.580372in}} %
\pgfusepath{clip}%
\pgfsetbuttcap%
\pgfsetroundjoin%
\definecolor{currentfill}{rgb}{0.790562,0.231397,0.216242}%
\pgfsetfillcolor{currentfill}%
\pgfsetlinewidth{0.000000pt}%
\definecolor{currentstroke}{rgb}{0.000000,0.000000,0.000000}%
\pgfsetstrokecolor{currentstroke}%
\pgfsetdash{}{0pt}%
\pgfpathmoveto{\pgfqpoint{2.855457in}{2.456853in}}%
\pgfpathlineto{\pgfqpoint{3.116444in}{2.629069in}}%
\pgfpathlineto{\pgfqpoint{2.855457in}{2.456853in}}%
\pgfpathclose%
\pgfusepath{fill}%
\end{pgfscope}%
\begin{pgfscope}%
\pgfpathrectangle{\pgfqpoint{0.017500in}{0.017500in}}{\pgfqpoint{6.218282in}{3.580372in}} %
\pgfusepath{clip}%
\pgfsetbuttcap%
\pgfsetroundjoin%
\definecolor{currentfill}{rgb}{0.790562,0.231397,0.216242}%
\pgfsetfillcolor{currentfill}%
\pgfsetlinewidth{0.000000pt}%
\definecolor{currentstroke}{rgb}{0.000000,0.000000,0.000000}%
\pgfsetstrokecolor{currentstroke}%
\pgfsetdash{}{0pt}%
\pgfpathmoveto{\pgfqpoint{2.855457in}{2.456853in}}%
\pgfpathlineto{\pgfqpoint{3.116444in}{2.629069in}}%
\pgfpathlineto{\pgfqpoint{2.855457in}{2.456853in}}%
\pgfpathclose%
\pgfusepath{fill}%
\end{pgfscope}%
\begin{pgfscope}%
\pgfpathrectangle{\pgfqpoint{0.017500in}{0.017500in}}{\pgfqpoint{6.218282in}{3.580372in}} %
\pgfusepath{clip}%
\pgfsetbuttcap%
\pgfsetroundjoin%
\definecolor{currentfill}{rgb}{0.790562,0.231397,0.216242}%
\pgfsetfillcolor{currentfill}%
\pgfsetlinewidth{0.000000pt}%
\definecolor{currentstroke}{rgb}{0.000000,0.000000,0.000000}%
\pgfsetstrokecolor{currentstroke}%
\pgfsetdash{}{0pt}%
\pgfpathmoveto{\pgfqpoint{2.855457in}{2.456853in}}%
\pgfpathlineto{\pgfqpoint{3.116444in}{2.629069in}}%
\pgfpathlineto{\pgfqpoint{2.855457in}{2.456853in}}%
\pgfpathclose%
\pgfusepath{fill}%
\end{pgfscope}%
\begin{pgfscope}%
\pgfpathrectangle{\pgfqpoint{0.017500in}{0.017500in}}{\pgfqpoint{6.218282in}{3.580372in}} %
\pgfusepath{clip}%
\pgfsetbuttcap%
\pgfsetroundjoin%
\definecolor{currentfill}{rgb}{0.790562,0.231397,0.216242}%
\pgfsetfillcolor{currentfill}%
\pgfsetlinewidth{0.000000pt}%
\definecolor{currentstroke}{rgb}{0.000000,0.000000,0.000000}%
\pgfsetstrokecolor{currentstroke}%
\pgfsetdash{}{0pt}%
\pgfpathmoveto{\pgfqpoint{2.855457in}{2.456853in}}%
\pgfpathlineto{\pgfqpoint{3.116444in}{2.629069in}}%
\pgfpathlineto{\pgfqpoint{2.855457in}{2.456853in}}%
\pgfpathclose%
\pgfusepath{fill}%
\end{pgfscope}%
\begin{pgfscope}%
\pgfpathrectangle{\pgfqpoint{0.017500in}{0.017500in}}{\pgfqpoint{6.218282in}{3.580372in}} %
\pgfusepath{clip}%
\pgfsetbuttcap%
\pgfsetroundjoin%
\definecolor{currentfill}{rgb}{0.790562,0.231397,0.216242}%
\pgfsetfillcolor{currentfill}%
\pgfsetlinewidth{0.000000pt}%
\definecolor{currentstroke}{rgb}{0.000000,0.000000,0.000000}%
\pgfsetstrokecolor{currentstroke}%
\pgfsetdash{}{0pt}%
\pgfpathmoveto{\pgfqpoint{2.855457in}{2.456853in}}%
\pgfpathlineto{\pgfqpoint{3.116444in}{2.629069in}}%
\pgfpathlineto{\pgfqpoint{2.855457in}{2.456853in}}%
\pgfpathclose%
\pgfusepath{fill}%
\end{pgfscope}%
\begin{pgfscope}%
\pgfpathrectangle{\pgfqpoint{0.017500in}{0.017500in}}{\pgfqpoint{6.218282in}{3.580372in}} %
\pgfusepath{clip}%
\pgfsetbuttcap%
\pgfsetroundjoin%
\definecolor{currentfill}{rgb}{0.790562,0.231397,0.216242}%
\pgfsetfillcolor{currentfill}%
\pgfsetlinewidth{0.000000pt}%
\definecolor{currentstroke}{rgb}{0.000000,0.000000,0.000000}%
\pgfsetstrokecolor{currentstroke}%
\pgfsetdash{}{0pt}%
\pgfpathmoveto{\pgfqpoint{2.855457in}{2.456853in}}%
\pgfpathlineto{\pgfqpoint{3.116444in}{2.629069in}}%
\pgfpathlineto{\pgfqpoint{2.855457in}{2.456853in}}%
\pgfpathclose%
\pgfusepath{fill}%
\end{pgfscope}%
\begin{pgfscope}%
\pgfpathrectangle{\pgfqpoint{0.017500in}{0.017500in}}{\pgfqpoint{6.218282in}{3.580372in}} %
\pgfusepath{clip}%
\pgfsetbuttcap%
\pgfsetroundjoin%
\definecolor{currentfill}{rgb}{0.790562,0.231397,0.216242}%
\pgfsetfillcolor{currentfill}%
\pgfsetlinewidth{0.000000pt}%
\definecolor{currentstroke}{rgb}{0.000000,0.000000,0.000000}%
\pgfsetstrokecolor{currentstroke}%
\pgfsetdash{}{0pt}%
\pgfpathmoveto{\pgfqpoint{2.855457in}{2.456853in}}%
\pgfpathlineto{\pgfqpoint{3.116444in}{2.629069in}}%
\pgfpathlineto{\pgfqpoint{2.855457in}{2.456853in}}%
\pgfpathclose%
\pgfusepath{fill}%
\end{pgfscope}%
\begin{pgfscope}%
\pgfpathrectangle{\pgfqpoint{0.017500in}{0.017500in}}{\pgfqpoint{6.218282in}{3.580372in}} %
\pgfusepath{clip}%
\pgfsetbuttcap%
\pgfsetroundjoin%
\definecolor{currentfill}{rgb}{0.790562,0.231397,0.216242}%
\pgfsetfillcolor{currentfill}%
\pgfsetlinewidth{0.000000pt}%
\definecolor{currentstroke}{rgb}{0.000000,0.000000,0.000000}%
\pgfsetstrokecolor{currentstroke}%
\pgfsetdash{}{0pt}%
\pgfpathmoveto{\pgfqpoint{2.855457in}{2.456853in}}%
\pgfpathlineto{\pgfqpoint{3.116444in}{2.629069in}}%
\pgfpathlineto{\pgfqpoint{2.855457in}{2.456853in}}%
\pgfpathclose%
\pgfusepath{fill}%
\end{pgfscope}%
\begin{pgfscope}%
\pgfpathrectangle{\pgfqpoint{0.017500in}{0.017500in}}{\pgfqpoint{6.218282in}{3.580372in}} %
\pgfusepath{clip}%
\pgfsetbuttcap%
\pgfsetroundjoin%
\definecolor{currentfill}{rgb}{0.790562,0.231397,0.216242}%
\pgfsetfillcolor{currentfill}%
\pgfsetlinewidth{0.000000pt}%
\definecolor{currentstroke}{rgb}{0.000000,0.000000,0.000000}%
\pgfsetstrokecolor{currentstroke}%
\pgfsetdash{}{0pt}%
\pgfpathmoveto{\pgfqpoint{2.855457in}{2.456853in}}%
\pgfpathlineto{\pgfqpoint{3.116444in}{2.629069in}}%
\pgfpathlineto{\pgfqpoint{2.855457in}{2.456853in}}%
\pgfpathclose%
\pgfusepath{fill}%
\end{pgfscope}%
\begin{pgfscope}%
\pgfpathrectangle{\pgfqpoint{0.017500in}{0.017500in}}{\pgfqpoint{6.218282in}{3.580372in}} %
\pgfusepath{clip}%
\pgfsetbuttcap%
\pgfsetroundjoin%
\definecolor{currentfill}{rgb}{0.790562,0.231397,0.216242}%
\pgfsetfillcolor{currentfill}%
\pgfsetlinewidth{0.000000pt}%
\definecolor{currentstroke}{rgb}{0.000000,0.000000,0.000000}%
\pgfsetstrokecolor{currentstroke}%
\pgfsetdash{}{0pt}%
\pgfpathmoveto{\pgfqpoint{2.855457in}{2.456853in}}%
\pgfpathlineto{\pgfqpoint{3.116444in}{2.629069in}}%
\pgfpathlineto{\pgfqpoint{2.855457in}{2.456853in}}%
\pgfpathclose%
\pgfusepath{fill}%
\end{pgfscope}%
\begin{pgfscope}%
\pgfpathrectangle{\pgfqpoint{0.017500in}{0.017500in}}{\pgfqpoint{6.218282in}{3.580372in}} %
\pgfusepath{clip}%
\pgfsetbuttcap%
\pgfsetroundjoin%
\definecolor{currentfill}{rgb}{0.790562,0.231397,0.216242}%
\pgfsetfillcolor{currentfill}%
\pgfsetlinewidth{0.000000pt}%
\definecolor{currentstroke}{rgb}{0.000000,0.000000,0.000000}%
\pgfsetstrokecolor{currentstroke}%
\pgfsetdash{}{0pt}%
\pgfpathmoveto{\pgfqpoint{2.855457in}{2.456853in}}%
\pgfpathlineto{\pgfqpoint{3.116444in}{2.629069in}}%
\pgfpathlineto{\pgfqpoint{2.855457in}{2.456853in}}%
\pgfpathclose%
\pgfusepath{fill}%
\end{pgfscope}%
\begin{pgfscope}%
\pgfpathrectangle{\pgfqpoint{0.017500in}{0.017500in}}{\pgfqpoint{6.218282in}{3.580372in}} %
\pgfusepath{clip}%
\pgfsetbuttcap%
\pgfsetroundjoin%
\definecolor{currentfill}{rgb}{0.790562,0.231397,0.216242}%
\pgfsetfillcolor{currentfill}%
\pgfsetlinewidth{0.000000pt}%
\definecolor{currentstroke}{rgb}{0.000000,0.000000,0.000000}%
\pgfsetstrokecolor{currentstroke}%
\pgfsetdash{}{0pt}%
\pgfpathmoveto{\pgfqpoint{2.855457in}{2.456853in}}%
\pgfpathlineto{\pgfqpoint{3.116444in}{2.629069in}}%
\pgfpathlineto{\pgfqpoint{2.855457in}{2.456853in}}%
\pgfpathclose%
\pgfusepath{fill}%
\end{pgfscope}%
\begin{pgfscope}%
\pgfpathrectangle{\pgfqpoint{0.017500in}{0.017500in}}{\pgfqpoint{6.218282in}{3.580372in}} %
\pgfusepath{clip}%
\pgfsetbuttcap%
\pgfsetroundjoin%
\definecolor{currentfill}{rgb}{0.790562,0.231397,0.216242}%
\pgfsetfillcolor{currentfill}%
\pgfsetlinewidth{0.000000pt}%
\definecolor{currentstroke}{rgb}{0.000000,0.000000,0.000000}%
\pgfsetstrokecolor{currentstroke}%
\pgfsetdash{}{0pt}%
\pgfpathmoveto{\pgfqpoint{2.855457in}{2.456853in}}%
\pgfpathlineto{\pgfqpoint{3.116444in}{2.629069in}}%
\pgfpathlineto{\pgfqpoint{2.855457in}{2.456853in}}%
\pgfpathclose%
\pgfusepath{fill}%
\end{pgfscope}%
\begin{pgfscope}%
\pgfpathrectangle{\pgfqpoint{0.017500in}{0.017500in}}{\pgfqpoint{6.218282in}{3.580372in}} %
\pgfusepath{clip}%
\pgfsetbuttcap%
\pgfsetroundjoin%
\definecolor{currentfill}{rgb}{0.790562,0.231397,0.216242}%
\pgfsetfillcolor{currentfill}%
\pgfsetlinewidth{0.000000pt}%
\definecolor{currentstroke}{rgb}{0.000000,0.000000,0.000000}%
\pgfsetstrokecolor{currentstroke}%
\pgfsetdash{}{0pt}%
\pgfpathmoveto{\pgfqpoint{2.855457in}{2.456853in}}%
\pgfpathlineto{\pgfqpoint{3.116444in}{2.629069in}}%
\pgfpathlineto{\pgfqpoint{2.855457in}{2.456853in}}%
\pgfpathclose%
\pgfusepath{fill}%
\end{pgfscope}%
\begin{pgfscope}%
\pgfpathrectangle{\pgfqpoint{0.017500in}{0.017500in}}{\pgfqpoint{6.218282in}{3.580372in}} %
\pgfusepath{clip}%
\pgfsetbuttcap%
\pgfsetroundjoin%
\definecolor{currentfill}{rgb}{0.790562,0.231397,0.216242}%
\pgfsetfillcolor{currentfill}%
\pgfsetlinewidth{0.000000pt}%
\definecolor{currentstroke}{rgb}{0.000000,0.000000,0.000000}%
\pgfsetstrokecolor{currentstroke}%
\pgfsetdash{}{0pt}%
\pgfpathmoveto{\pgfqpoint{2.855457in}{2.456853in}}%
\pgfpathlineto{\pgfqpoint{3.116444in}{2.629069in}}%
\pgfpathlineto{\pgfqpoint{2.855457in}{2.456853in}}%
\pgfpathclose%
\pgfusepath{fill}%
\end{pgfscope}%
\begin{pgfscope}%
\pgfpathrectangle{\pgfqpoint{0.017500in}{0.017500in}}{\pgfqpoint{6.218282in}{3.580372in}} %
\pgfusepath{clip}%
\pgfsetbuttcap%
\pgfsetroundjoin%
\definecolor{currentfill}{rgb}{0.790562,0.231397,0.216242}%
\pgfsetfillcolor{currentfill}%
\pgfsetlinewidth{0.000000pt}%
\definecolor{currentstroke}{rgb}{0.000000,0.000000,0.000000}%
\pgfsetstrokecolor{currentstroke}%
\pgfsetdash{}{0pt}%
\pgfpathmoveto{\pgfqpoint{2.855457in}{2.456853in}}%
\pgfpathlineto{\pgfqpoint{3.116444in}{2.629069in}}%
\pgfpathlineto{\pgfqpoint{2.855457in}{2.456853in}}%
\pgfpathclose%
\pgfusepath{fill}%
\end{pgfscope}%
\begin{pgfscope}%
\pgfpathrectangle{\pgfqpoint{0.017500in}{0.017500in}}{\pgfqpoint{6.218282in}{3.580372in}} %
\pgfusepath{clip}%
\pgfsetbuttcap%
\pgfsetroundjoin%
\definecolor{currentfill}{rgb}{0.790562,0.231397,0.216242}%
\pgfsetfillcolor{currentfill}%
\pgfsetlinewidth{0.000000pt}%
\definecolor{currentstroke}{rgb}{0.000000,0.000000,0.000000}%
\pgfsetstrokecolor{currentstroke}%
\pgfsetdash{}{0pt}%
\pgfpathmoveto{\pgfqpoint{2.855457in}{2.456853in}}%
\pgfpathlineto{\pgfqpoint{3.116444in}{2.629069in}}%
\pgfpathlineto{\pgfqpoint{2.855457in}{2.456853in}}%
\pgfpathclose%
\pgfusepath{fill}%
\end{pgfscope}%
\begin{pgfscope}%
\pgfpathrectangle{\pgfqpoint{0.017500in}{0.017500in}}{\pgfqpoint{6.218282in}{3.580372in}} %
\pgfusepath{clip}%
\pgfsetbuttcap%
\pgfsetroundjoin%
\definecolor{currentfill}{rgb}{0.790562,0.231397,0.216242}%
\pgfsetfillcolor{currentfill}%
\pgfsetlinewidth{0.000000pt}%
\definecolor{currentstroke}{rgb}{0.000000,0.000000,0.000000}%
\pgfsetstrokecolor{currentstroke}%
\pgfsetdash{}{0pt}%
\pgfpathmoveto{\pgfqpoint{2.855457in}{2.456853in}}%
\pgfpathlineto{\pgfqpoint{3.116444in}{2.629069in}}%
\pgfpathlineto{\pgfqpoint{2.855457in}{2.456853in}}%
\pgfpathclose%
\pgfusepath{fill}%
\end{pgfscope}%
\begin{pgfscope}%
\pgfpathrectangle{\pgfqpoint{0.017500in}{0.017500in}}{\pgfqpoint{6.218282in}{3.580372in}} %
\pgfusepath{clip}%
\pgfsetbuttcap%
\pgfsetroundjoin%
\definecolor{currentfill}{rgb}{0.790562,0.231397,0.216242}%
\pgfsetfillcolor{currentfill}%
\pgfsetlinewidth{0.000000pt}%
\definecolor{currentstroke}{rgb}{0.000000,0.000000,0.000000}%
\pgfsetstrokecolor{currentstroke}%
\pgfsetdash{}{0pt}%
\pgfpathmoveto{\pgfqpoint{2.855457in}{2.456853in}}%
\pgfpathlineto{\pgfqpoint{3.116444in}{2.629069in}}%
\pgfpathlineto{\pgfqpoint{2.855457in}{2.456853in}}%
\pgfpathclose%
\pgfusepath{fill}%
\end{pgfscope}%
\begin{pgfscope}%
\pgfpathrectangle{\pgfqpoint{0.017500in}{0.017500in}}{\pgfqpoint{6.218282in}{3.580372in}} %
\pgfusepath{clip}%
\pgfsetbuttcap%
\pgfsetroundjoin%
\definecolor{currentfill}{rgb}{0.790562,0.231397,0.216242}%
\pgfsetfillcolor{currentfill}%
\pgfsetlinewidth{0.000000pt}%
\definecolor{currentstroke}{rgb}{0.000000,0.000000,0.000000}%
\pgfsetstrokecolor{currentstroke}%
\pgfsetdash{}{0pt}%
\pgfpathmoveto{\pgfqpoint{2.855457in}{2.456853in}}%
\pgfpathlineto{\pgfqpoint{3.116444in}{2.629069in}}%
\pgfpathlineto{\pgfqpoint{2.855457in}{2.456853in}}%
\pgfpathclose%
\pgfusepath{fill}%
\end{pgfscope}%
\begin{pgfscope}%
\pgfpathrectangle{\pgfqpoint{0.017500in}{0.017500in}}{\pgfqpoint{6.218282in}{3.580372in}} %
\pgfusepath{clip}%
\pgfsetbuttcap%
\pgfsetroundjoin%
\definecolor{currentfill}{rgb}{0.790562,0.231397,0.216242}%
\pgfsetfillcolor{currentfill}%
\pgfsetlinewidth{0.000000pt}%
\definecolor{currentstroke}{rgb}{0.000000,0.000000,0.000000}%
\pgfsetstrokecolor{currentstroke}%
\pgfsetdash{}{0pt}%
\pgfpathmoveto{\pgfqpoint{2.855457in}{2.456853in}}%
\pgfpathlineto{\pgfqpoint{3.116444in}{2.629069in}}%
\pgfpathlineto{\pgfqpoint{2.855457in}{2.456853in}}%
\pgfpathclose%
\pgfusepath{fill}%
\end{pgfscope}%
\begin{pgfscope}%
\pgfpathrectangle{\pgfqpoint{0.017500in}{0.017500in}}{\pgfqpoint{6.218282in}{3.580372in}} %
\pgfusepath{clip}%
\pgfsetbuttcap%
\pgfsetroundjoin%
\definecolor{currentfill}{rgb}{0.790562,0.231397,0.216242}%
\pgfsetfillcolor{currentfill}%
\pgfsetlinewidth{0.000000pt}%
\definecolor{currentstroke}{rgb}{0.000000,0.000000,0.000000}%
\pgfsetstrokecolor{currentstroke}%
\pgfsetdash{}{0pt}%
\pgfpathmoveto{\pgfqpoint{2.855457in}{2.456853in}}%
\pgfpathlineto{\pgfqpoint{3.116444in}{2.629069in}}%
\pgfpathlineto{\pgfqpoint{2.855457in}{2.456853in}}%
\pgfpathclose%
\pgfusepath{fill}%
\end{pgfscope}%
\begin{pgfscope}%
\pgfpathrectangle{\pgfqpoint{0.017500in}{0.017500in}}{\pgfqpoint{6.218282in}{3.580372in}} %
\pgfusepath{clip}%
\pgfsetbuttcap%
\pgfsetroundjoin%
\definecolor{currentfill}{rgb}{0.790562,0.231397,0.216242}%
\pgfsetfillcolor{currentfill}%
\pgfsetlinewidth{0.000000pt}%
\definecolor{currentstroke}{rgb}{0.000000,0.000000,0.000000}%
\pgfsetstrokecolor{currentstroke}%
\pgfsetdash{}{0pt}%
\pgfpathmoveto{\pgfqpoint{2.855457in}{2.456853in}}%
\pgfpathlineto{\pgfqpoint{3.116444in}{2.629069in}}%
\pgfpathlineto{\pgfqpoint{2.855457in}{2.456853in}}%
\pgfpathclose%
\pgfusepath{fill}%
\end{pgfscope}%
\begin{pgfscope}%
\pgfpathrectangle{\pgfqpoint{0.017500in}{0.017500in}}{\pgfqpoint{6.218282in}{3.580372in}} %
\pgfusepath{clip}%
\pgfsetbuttcap%
\pgfsetroundjoin%
\definecolor{currentfill}{rgb}{0.790562,0.231397,0.216242}%
\pgfsetfillcolor{currentfill}%
\pgfsetlinewidth{0.000000pt}%
\definecolor{currentstroke}{rgb}{0.000000,0.000000,0.000000}%
\pgfsetstrokecolor{currentstroke}%
\pgfsetdash{}{0pt}%
\pgfpathmoveto{\pgfqpoint{2.855457in}{2.456853in}}%
\pgfpathlineto{\pgfqpoint{3.116444in}{2.629069in}}%
\pgfpathlineto{\pgfqpoint{2.855457in}{2.456853in}}%
\pgfpathclose%
\pgfusepath{fill}%
\end{pgfscope}%
\begin{pgfscope}%
\pgfpathrectangle{\pgfqpoint{0.017500in}{0.017500in}}{\pgfqpoint{6.218282in}{3.580372in}} %
\pgfusepath{clip}%
\pgfsetbuttcap%
\pgfsetroundjoin%
\definecolor{currentfill}{rgb}{0.790562,0.231397,0.216242}%
\pgfsetfillcolor{currentfill}%
\pgfsetlinewidth{0.000000pt}%
\definecolor{currentstroke}{rgb}{0.000000,0.000000,0.000000}%
\pgfsetstrokecolor{currentstroke}%
\pgfsetdash{}{0pt}%
\pgfpathmoveto{\pgfqpoint{2.855457in}{2.456853in}}%
\pgfpathlineto{\pgfqpoint{3.116444in}{2.629069in}}%
\pgfpathlineto{\pgfqpoint{2.855457in}{2.456853in}}%
\pgfpathclose%
\pgfusepath{fill}%
\end{pgfscope}%
\begin{pgfscope}%
\pgfpathrectangle{\pgfqpoint{0.017500in}{0.017500in}}{\pgfqpoint{6.218282in}{3.580372in}} %
\pgfusepath{clip}%
\pgfsetbuttcap%
\pgfsetroundjoin%
\definecolor{currentfill}{rgb}{0.790562,0.231397,0.216242}%
\pgfsetfillcolor{currentfill}%
\pgfsetlinewidth{0.000000pt}%
\definecolor{currentstroke}{rgb}{0.000000,0.000000,0.000000}%
\pgfsetstrokecolor{currentstroke}%
\pgfsetdash{}{0pt}%
\pgfpathmoveto{\pgfqpoint{2.855457in}{2.456853in}}%
\pgfpathlineto{\pgfqpoint{3.116444in}{2.629069in}}%
\pgfpathlineto{\pgfqpoint{2.855457in}{2.456853in}}%
\pgfpathclose%
\pgfusepath{fill}%
\end{pgfscope}%
\begin{pgfscope}%
\pgfpathrectangle{\pgfqpoint{0.017500in}{0.017500in}}{\pgfqpoint{6.218282in}{3.580372in}} %
\pgfusepath{clip}%
\pgfsetbuttcap%
\pgfsetroundjoin%
\definecolor{currentfill}{rgb}{0.790562,0.231397,0.216242}%
\pgfsetfillcolor{currentfill}%
\pgfsetlinewidth{0.000000pt}%
\definecolor{currentstroke}{rgb}{0.000000,0.000000,0.000000}%
\pgfsetstrokecolor{currentstroke}%
\pgfsetdash{}{0pt}%
\pgfpathmoveto{\pgfqpoint{2.855457in}{2.456853in}}%
\pgfpathlineto{\pgfqpoint{3.116444in}{2.629069in}}%
\pgfpathlineto{\pgfqpoint{2.855457in}{2.456853in}}%
\pgfpathclose%
\pgfusepath{fill}%
\end{pgfscope}%
\begin{pgfscope}%
\pgfpathrectangle{\pgfqpoint{0.017500in}{0.017500in}}{\pgfqpoint{6.218282in}{3.580372in}} %
\pgfusepath{clip}%
\pgfsetbuttcap%
\pgfsetroundjoin%
\definecolor{currentfill}{rgb}{0.790562,0.231397,0.216242}%
\pgfsetfillcolor{currentfill}%
\pgfsetlinewidth{0.000000pt}%
\definecolor{currentstroke}{rgb}{0.000000,0.000000,0.000000}%
\pgfsetstrokecolor{currentstroke}%
\pgfsetdash{}{0pt}%
\pgfpathmoveto{\pgfqpoint{2.855457in}{2.456853in}}%
\pgfpathlineto{\pgfqpoint{3.116444in}{2.629069in}}%
\pgfpathlineto{\pgfqpoint{2.855457in}{2.456853in}}%
\pgfpathclose%
\pgfusepath{fill}%
\end{pgfscope}%
\begin{pgfscope}%
\pgfpathrectangle{\pgfqpoint{0.017500in}{0.017500in}}{\pgfqpoint{6.218282in}{3.580372in}} %
\pgfusepath{clip}%
\pgfsetbuttcap%
\pgfsetroundjoin%
\definecolor{currentfill}{rgb}{0.790562,0.231397,0.216242}%
\pgfsetfillcolor{currentfill}%
\pgfsetlinewidth{0.000000pt}%
\definecolor{currentstroke}{rgb}{0.000000,0.000000,0.000000}%
\pgfsetstrokecolor{currentstroke}%
\pgfsetdash{}{0pt}%
\pgfpathmoveto{\pgfqpoint{2.855457in}{2.456853in}}%
\pgfpathlineto{\pgfqpoint{3.116444in}{2.629069in}}%
\pgfpathlineto{\pgfqpoint{2.855457in}{2.456853in}}%
\pgfpathclose%
\pgfusepath{fill}%
\end{pgfscope}%
\begin{pgfscope}%
\pgfpathrectangle{\pgfqpoint{0.017500in}{0.017500in}}{\pgfqpoint{6.218282in}{3.580372in}} %
\pgfusepath{clip}%
\pgfsetbuttcap%
\pgfsetroundjoin%
\definecolor{currentfill}{rgb}{0.790562,0.231397,0.216242}%
\pgfsetfillcolor{currentfill}%
\pgfsetlinewidth{0.000000pt}%
\definecolor{currentstroke}{rgb}{0.000000,0.000000,0.000000}%
\pgfsetstrokecolor{currentstroke}%
\pgfsetdash{}{0pt}%
\pgfpathmoveto{\pgfqpoint{2.855457in}{2.456853in}}%
\pgfpathlineto{\pgfqpoint{3.116444in}{2.629069in}}%
\pgfpathlineto{\pgfqpoint{2.855457in}{2.456853in}}%
\pgfpathclose%
\pgfusepath{fill}%
\end{pgfscope}%
\begin{pgfscope}%
\pgfpathrectangle{\pgfqpoint{0.017500in}{0.017500in}}{\pgfqpoint{6.218282in}{3.580372in}} %
\pgfusepath{clip}%
\pgfsetbuttcap%
\pgfsetroundjoin%
\definecolor{currentfill}{rgb}{0.785153,0.220851,0.211673}%
\pgfsetfillcolor{currentfill}%
\pgfsetlinewidth{0.000000pt}%
\definecolor{currentstroke}{rgb}{0.000000,0.000000,0.000000}%
\pgfsetstrokecolor{currentstroke}%
\pgfsetdash{}{0pt}%
\pgfpathmoveto{\pgfqpoint{2.855457in}{2.456853in}}%
\pgfpathlineto{\pgfqpoint{3.306522in}{2.395850in}}%
\pgfpathlineto{\pgfqpoint{3.564605in}{2.552204in}}%
\pgfpathlineto{\pgfqpoint{3.116444in}{2.629069in}}%
\pgfpathlineto{\pgfqpoint{2.855457in}{2.456853in}}%
\pgfpathclose%
\pgfusepath{fill}%
\end{pgfscope}%
\begin{pgfscope}%
\pgfpathrectangle{\pgfqpoint{0.017500in}{0.017500in}}{\pgfqpoint{6.218282in}{3.580372in}} %
\pgfusepath{clip}%
\pgfsetbuttcap%
\pgfsetroundjoin%
\definecolor{currentfill}{rgb}{0.785153,0.220851,0.211673}%
\pgfsetfillcolor{currentfill}%
\pgfsetlinewidth{0.000000pt}%
\definecolor{currentstroke}{rgb}{0.000000,0.000000,0.000000}%
\pgfsetstrokecolor{currentstroke}%
\pgfsetdash{}{0pt}%
\pgfpathmoveto{\pgfqpoint{2.855457in}{2.456853in}}%
\pgfpathlineto{\pgfqpoint{3.306522in}{2.395850in}}%
\pgfpathlineto{\pgfqpoint{3.564605in}{2.552204in}}%
\pgfpathlineto{\pgfqpoint{3.116444in}{2.629069in}}%
\pgfpathlineto{\pgfqpoint{2.855457in}{2.456853in}}%
\pgfpathclose%
\pgfusepath{fill}%
\end{pgfscope}%
\begin{pgfscope}%
\pgfpathrectangle{\pgfqpoint{0.017500in}{0.017500in}}{\pgfqpoint{6.218282in}{3.580372in}} %
\pgfusepath{clip}%
\pgfsetbuttcap%
\pgfsetroundjoin%
\definecolor{currentfill}{rgb}{0.785153,0.220851,0.211673}%
\pgfsetfillcolor{currentfill}%
\pgfsetlinewidth{0.000000pt}%
\definecolor{currentstroke}{rgb}{0.000000,0.000000,0.000000}%
\pgfsetstrokecolor{currentstroke}%
\pgfsetdash{}{0pt}%
\pgfpathmoveto{\pgfqpoint{2.855457in}{2.456853in}}%
\pgfpathlineto{\pgfqpoint{3.306522in}{2.395850in}}%
\pgfpathlineto{\pgfqpoint{3.564605in}{2.552204in}}%
\pgfpathlineto{\pgfqpoint{3.116444in}{2.629069in}}%
\pgfpathlineto{\pgfqpoint{2.855457in}{2.456853in}}%
\pgfpathclose%
\pgfusepath{fill}%
\end{pgfscope}%
\begin{pgfscope}%
\pgfpathrectangle{\pgfqpoint{0.017500in}{0.017500in}}{\pgfqpoint{6.218282in}{3.580372in}} %
\pgfusepath{clip}%
\pgfsetbuttcap%
\pgfsetroundjoin%
\definecolor{currentfill}{rgb}{0.785153,0.220851,0.211673}%
\pgfsetfillcolor{currentfill}%
\pgfsetlinewidth{0.000000pt}%
\definecolor{currentstroke}{rgb}{0.000000,0.000000,0.000000}%
\pgfsetstrokecolor{currentstroke}%
\pgfsetdash{}{0pt}%
\pgfpathmoveto{\pgfqpoint{2.855457in}{2.456853in}}%
\pgfpathlineto{\pgfqpoint{3.306522in}{2.395850in}}%
\pgfpathlineto{\pgfqpoint{3.564605in}{2.552204in}}%
\pgfpathlineto{\pgfqpoint{3.116444in}{2.629069in}}%
\pgfpathlineto{\pgfqpoint{2.855457in}{2.456853in}}%
\pgfpathclose%
\pgfusepath{fill}%
\end{pgfscope}%
\begin{pgfscope}%
\pgfpathrectangle{\pgfqpoint{0.017500in}{0.017500in}}{\pgfqpoint{6.218282in}{3.580372in}} %
\pgfusepath{clip}%
\pgfsetbuttcap%
\pgfsetroundjoin%
\definecolor{currentfill}{rgb}{0.785153,0.220851,0.211673}%
\pgfsetfillcolor{currentfill}%
\pgfsetlinewidth{0.000000pt}%
\definecolor{currentstroke}{rgb}{0.000000,0.000000,0.000000}%
\pgfsetstrokecolor{currentstroke}%
\pgfsetdash{}{0pt}%
\pgfpathmoveto{\pgfqpoint{2.855457in}{2.456853in}}%
\pgfpathlineto{\pgfqpoint{3.306522in}{2.395850in}}%
\pgfpathlineto{\pgfqpoint{3.564605in}{2.552204in}}%
\pgfpathlineto{\pgfqpoint{3.116444in}{2.629069in}}%
\pgfpathlineto{\pgfqpoint{2.855457in}{2.456853in}}%
\pgfpathclose%
\pgfusepath{fill}%
\end{pgfscope}%
\begin{pgfscope}%
\pgfpathrectangle{\pgfqpoint{0.017500in}{0.017500in}}{\pgfqpoint{6.218282in}{3.580372in}} %
\pgfusepath{clip}%
\pgfsetbuttcap%
\pgfsetroundjoin%
\definecolor{currentfill}{rgb}{0.785153,0.220851,0.211673}%
\pgfsetfillcolor{currentfill}%
\pgfsetlinewidth{0.000000pt}%
\definecolor{currentstroke}{rgb}{0.000000,0.000000,0.000000}%
\pgfsetstrokecolor{currentstroke}%
\pgfsetdash{}{0pt}%
\pgfpathmoveto{\pgfqpoint{2.855457in}{2.456853in}}%
\pgfpathlineto{\pgfqpoint{3.306522in}{2.395850in}}%
\pgfpathlineto{\pgfqpoint{3.564605in}{2.552204in}}%
\pgfpathlineto{\pgfqpoint{3.116444in}{2.629069in}}%
\pgfpathlineto{\pgfqpoint{2.855457in}{2.456853in}}%
\pgfpathclose%
\pgfusepath{fill}%
\end{pgfscope}%
\begin{pgfscope}%
\pgfpathrectangle{\pgfqpoint{0.017500in}{0.017500in}}{\pgfqpoint{6.218282in}{3.580372in}} %
\pgfusepath{clip}%
\pgfsetbuttcap%
\pgfsetroundjoin%
\definecolor{currentfill}{rgb}{0.257234,0.339661,0.789661}%
\pgfsetfillcolor{currentfill}%
\pgfsetlinewidth{0.000000pt}%
\definecolor{currentstroke}{rgb}{0.000000,0.000000,0.000000}%
\pgfsetstrokecolor{currentstroke}%
\pgfsetdash{}{0pt}%
\pgfpathmoveto{\pgfqpoint{2.789661in}{0.820654in}}%
\pgfpathlineto{\pgfqpoint{3.043201in}{2.166904in}}%
\pgfpathlineto{\pgfqpoint{2.789661in}{0.820654in}}%
\pgfpathclose%
\pgfusepath{fill}%
\end{pgfscope}%
\begin{pgfscope}%
\pgfpathrectangle{\pgfqpoint{0.017500in}{0.017500in}}{\pgfqpoint{6.218282in}{3.580372in}} %
\pgfusepath{clip}%
\pgfsetbuttcap%
\pgfsetroundjoin%
\definecolor{currentfill}{rgb}{0.257234,0.339661,0.789661}%
\pgfsetfillcolor{currentfill}%
\pgfsetlinewidth{0.000000pt}%
\definecolor{currentstroke}{rgb}{0.000000,0.000000,0.000000}%
\pgfsetstrokecolor{currentstroke}%
\pgfsetdash{}{0pt}%
\pgfpathmoveto{\pgfqpoint{2.789661in}{0.820654in}}%
\pgfpathlineto{\pgfqpoint{3.043201in}{2.166904in}}%
\pgfpathlineto{\pgfqpoint{2.789661in}{0.820654in}}%
\pgfpathclose%
\pgfusepath{fill}%
\end{pgfscope}%
\begin{pgfscope}%
\pgfpathrectangle{\pgfqpoint{0.017500in}{0.017500in}}{\pgfqpoint{6.218282in}{3.580372in}} %
\pgfusepath{clip}%
\pgfsetbuttcap%
\pgfsetroundjoin%
\definecolor{currentfill}{rgb}{0.257234,0.339661,0.789661}%
\pgfsetfillcolor{currentfill}%
\pgfsetlinewidth{0.000000pt}%
\definecolor{currentstroke}{rgb}{0.000000,0.000000,0.000000}%
\pgfsetstrokecolor{currentstroke}%
\pgfsetdash{}{0pt}%
\pgfpathmoveto{\pgfqpoint{2.789661in}{0.820654in}}%
\pgfpathlineto{\pgfqpoint{3.043201in}{2.166904in}}%
\pgfpathlineto{\pgfqpoint{2.789661in}{0.820654in}}%
\pgfpathclose%
\pgfusepath{fill}%
\end{pgfscope}%
\begin{pgfscope}%
\pgfpathrectangle{\pgfqpoint{0.017500in}{0.017500in}}{\pgfqpoint{6.218282in}{3.580372in}} %
\pgfusepath{clip}%
\pgfsetbuttcap%
\pgfsetroundjoin%
\definecolor{currentfill}{rgb}{0.257234,0.339661,0.789661}%
\pgfsetfillcolor{currentfill}%
\pgfsetlinewidth{0.000000pt}%
\definecolor{currentstroke}{rgb}{0.000000,0.000000,0.000000}%
\pgfsetstrokecolor{currentstroke}%
\pgfsetdash{}{0pt}%
\pgfpathmoveto{\pgfqpoint{2.789661in}{0.820654in}}%
\pgfpathlineto{\pgfqpoint{3.043201in}{2.166904in}}%
\pgfpathlineto{\pgfqpoint{2.789661in}{0.820654in}}%
\pgfpathclose%
\pgfusepath{fill}%
\end{pgfscope}%
\begin{pgfscope}%
\pgfpathrectangle{\pgfqpoint{0.017500in}{0.017500in}}{\pgfqpoint{6.218282in}{3.580372in}} %
\pgfusepath{clip}%
\pgfsetbuttcap%
\pgfsetroundjoin%
\definecolor{currentfill}{rgb}{0.257234,0.339661,0.789661}%
\pgfsetfillcolor{currentfill}%
\pgfsetlinewidth{0.000000pt}%
\definecolor{currentstroke}{rgb}{0.000000,0.000000,0.000000}%
\pgfsetstrokecolor{currentstroke}%
\pgfsetdash{}{0pt}%
\pgfpathmoveto{\pgfqpoint{2.789661in}{0.820654in}}%
\pgfpathlineto{\pgfqpoint{3.043201in}{2.166904in}}%
\pgfpathlineto{\pgfqpoint{2.789661in}{0.820654in}}%
\pgfpathclose%
\pgfusepath{fill}%
\end{pgfscope}%
\begin{pgfscope}%
\pgfpathrectangle{\pgfqpoint{0.017500in}{0.017500in}}{\pgfqpoint{6.218282in}{3.580372in}} %
\pgfusepath{clip}%
\pgfsetbuttcap%
\pgfsetroundjoin%
\definecolor{currentfill}{rgb}{0.257234,0.339661,0.789661}%
\pgfsetfillcolor{currentfill}%
\pgfsetlinewidth{0.000000pt}%
\definecolor{currentstroke}{rgb}{0.000000,0.000000,0.000000}%
\pgfsetstrokecolor{currentstroke}%
\pgfsetdash{}{0pt}%
\pgfpathmoveto{\pgfqpoint{2.789661in}{0.820654in}}%
\pgfpathlineto{\pgfqpoint{3.043201in}{2.166904in}}%
\pgfpathlineto{\pgfqpoint{2.789661in}{0.820654in}}%
\pgfpathclose%
\pgfusepath{fill}%
\end{pgfscope}%
\begin{pgfscope}%
\pgfpathrectangle{\pgfqpoint{0.017500in}{0.017500in}}{\pgfqpoint{6.218282in}{3.580372in}} %
\pgfusepath{clip}%
\pgfsetbuttcap%
\pgfsetroundjoin%
\definecolor{currentfill}{rgb}{0.257234,0.339661,0.789661}%
\pgfsetfillcolor{currentfill}%
\pgfsetlinewidth{0.000000pt}%
\definecolor{currentstroke}{rgb}{0.000000,0.000000,0.000000}%
\pgfsetstrokecolor{currentstroke}%
\pgfsetdash{}{0pt}%
\pgfpathmoveto{\pgfqpoint{2.789661in}{0.820654in}}%
\pgfpathlineto{\pgfqpoint{3.043201in}{2.166904in}}%
\pgfpathlineto{\pgfqpoint{2.789661in}{0.820654in}}%
\pgfpathclose%
\pgfusepath{fill}%
\end{pgfscope}%
\begin{pgfscope}%
\pgfpathrectangle{\pgfqpoint{0.017500in}{0.017500in}}{\pgfqpoint{6.218282in}{3.580372in}} %
\pgfusepath{clip}%
\pgfsetbuttcap%
\pgfsetroundjoin%
\definecolor{currentfill}{rgb}{0.257234,0.339661,0.789661}%
\pgfsetfillcolor{currentfill}%
\pgfsetlinewidth{0.000000pt}%
\definecolor{currentstroke}{rgb}{0.000000,0.000000,0.000000}%
\pgfsetstrokecolor{currentstroke}%
\pgfsetdash{}{0pt}%
\pgfpathmoveto{\pgfqpoint{2.789661in}{0.820654in}}%
\pgfpathlineto{\pgfqpoint{3.043201in}{2.166904in}}%
\pgfpathlineto{\pgfqpoint{2.789661in}{0.820654in}}%
\pgfpathclose%
\pgfusepath{fill}%
\end{pgfscope}%
\begin{pgfscope}%
\pgfpathrectangle{\pgfqpoint{0.017500in}{0.017500in}}{\pgfqpoint{6.218282in}{3.580372in}} %
\pgfusepath{clip}%
\pgfsetbuttcap%
\pgfsetroundjoin%
\definecolor{currentfill}{rgb}{0.257234,0.339661,0.789661}%
\pgfsetfillcolor{currentfill}%
\pgfsetlinewidth{0.000000pt}%
\definecolor{currentstroke}{rgb}{0.000000,0.000000,0.000000}%
\pgfsetstrokecolor{currentstroke}%
\pgfsetdash{}{0pt}%
\pgfpathmoveto{\pgfqpoint{2.789661in}{0.820654in}}%
\pgfpathlineto{\pgfqpoint{3.043201in}{2.166904in}}%
\pgfpathlineto{\pgfqpoint{2.789661in}{0.820654in}}%
\pgfpathclose%
\pgfusepath{fill}%
\end{pgfscope}%
\begin{pgfscope}%
\pgfpathrectangle{\pgfqpoint{0.017500in}{0.017500in}}{\pgfqpoint{6.218282in}{3.580372in}} %
\pgfusepath{clip}%
\pgfsetbuttcap%
\pgfsetroundjoin%
\definecolor{currentfill}{rgb}{0.257234,0.339661,0.789661}%
\pgfsetfillcolor{currentfill}%
\pgfsetlinewidth{0.000000pt}%
\definecolor{currentstroke}{rgb}{0.000000,0.000000,0.000000}%
\pgfsetstrokecolor{currentstroke}%
\pgfsetdash{}{0pt}%
\pgfpathmoveto{\pgfqpoint{2.789661in}{0.820654in}}%
\pgfpathlineto{\pgfqpoint{3.043201in}{2.166904in}}%
\pgfpathlineto{\pgfqpoint{2.789661in}{0.820654in}}%
\pgfpathclose%
\pgfusepath{fill}%
\end{pgfscope}%
\begin{pgfscope}%
\pgfpathrectangle{\pgfqpoint{0.017500in}{0.017500in}}{\pgfqpoint{6.218282in}{3.580372in}} %
\pgfusepath{clip}%
\pgfsetbuttcap%
\pgfsetroundjoin%
\definecolor{currentfill}{rgb}{0.257234,0.339661,0.789661}%
\pgfsetfillcolor{currentfill}%
\pgfsetlinewidth{0.000000pt}%
\definecolor{currentstroke}{rgb}{0.000000,0.000000,0.000000}%
\pgfsetstrokecolor{currentstroke}%
\pgfsetdash{}{0pt}%
\pgfpathmoveto{\pgfqpoint{2.789661in}{0.820654in}}%
\pgfpathlineto{\pgfqpoint{3.043201in}{2.166904in}}%
\pgfpathlineto{\pgfqpoint{2.789661in}{0.820654in}}%
\pgfpathclose%
\pgfusepath{fill}%
\end{pgfscope}%
\begin{pgfscope}%
\pgfpathrectangle{\pgfqpoint{0.017500in}{0.017500in}}{\pgfqpoint{6.218282in}{3.580372in}} %
\pgfusepath{clip}%
\pgfsetbuttcap%
\pgfsetroundjoin%
\definecolor{currentfill}{rgb}{0.257234,0.339661,0.789661}%
\pgfsetfillcolor{currentfill}%
\pgfsetlinewidth{0.000000pt}%
\definecolor{currentstroke}{rgb}{0.000000,0.000000,0.000000}%
\pgfsetstrokecolor{currentstroke}%
\pgfsetdash{}{0pt}%
\pgfpathmoveto{\pgfqpoint{2.789661in}{0.820654in}}%
\pgfpathlineto{\pgfqpoint{3.043201in}{2.166904in}}%
\pgfpathlineto{\pgfqpoint{2.789661in}{0.820654in}}%
\pgfpathclose%
\pgfusepath{fill}%
\end{pgfscope}%
\begin{pgfscope}%
\pgfpathrectangle{\pgfqpoint{0.017500in}{0.017500in}}{\pgfqpoint{6.218282in}{3.580372in}} %
\pgfusepath{clip}%
\pgfsetbuttcap%
\pgfsetroundjoin%
\definecolor{currentfill}{rgb}{0.257234,0.339661,0.789661}%
\pgfsetfillcolor{currentfill}%
\pgfsetlinewidth{0.000000pt}%
\definecolor{currentstroke}{rgb}{0.000000,0.000000,0.000000}%
\pgfsetstrokecolor{currentstroke}%
\pgfsetdash{}{0pt}%
\pgfpathmoveto{\pgfqpoint{2.789661in}{0.820654in}}%
\pgfpathlineto{\pgfqpoint{3.043201in}{2.166904in}}%
\pgfpathlineto{\pgfqpoint{2.789661in}{0.820654in}}%
\pgfpathclose%
\pgfusepath{fill}%
\end{pgfscope}%
\begin{pgfscope}%
\pgfpathrectangle{\pgfqpoint{0.017500in}{0.017500in}}{\pgfqpoint{6.218282in}{3.580372in}} %
\pgfusepath{clip}%
\pgfsetbuttcap%
\pgfsetroundjoin%
\definecolor{currentfill}{rgb}{0.257234,0.339661,0.789661}%
\pgfsetfillcolor{currentfill}%
\pgfsetlinewidth{0.000000pt}%
\definecolor{currentstroke}{rgb}{0.000000,0.000000,0.000000}%
\pgfsetstrokecolor{currentstroke}%
\pgfsetdash{}{0pt}%
\pgfpathmoveto{\pgfqpoint{2.789661in}{0.820654in}}%
\pgfpathlineto{\pgfqpoint{3.043201in}{2.166904in}}%
\pgfpathlineto{\pgfqpoint{2.789661in}{0.820654in}}%
\pgfpathclose%
\pgfusepath{fill}%
\end{pgfscope}%
\begin{pgfscope}%
\pgfpathrectangle{\pgfqpoint{0.017500in}{0.017500in}}{\pgfqpoint{6.218282in}{3.580372in}} %
\pgfusepath{clip}%
\pgfsetbuttcap%
\pgfsetroundjoin%
\definecolor{currentfill}{rgb}{0.257234,0.339661,0.789661}%
\pgfsetfillcolor{currentfill}%
\pgfsetlinewidth{0.000000pt}%
\definecolor{currentstroke}{rgb}{0.000000,0.000000,0.000000}%
\pgfsetstrokecolor{currentstroke}%
\pgfsetdash{}{0pt}%
\pgfpathmoveto{\pgfqpoint{2.789661in}{0.820654in}}%
\pgfpathlineto{\pgfqpoint{3.043201in}{2.166904in}}%
\pgfpathlineto{\pgfqpoint{2.789661in}{0.820654in}}%
\pgfpathclose%
\pgfusepath{fill}%
\end{pgfscope}%
\begin{pgfscope}%
\pgfpathrectangle{\pgfqpoint{0.017500in}{0.017500in}}{\pgfqpoint{6.218282in}{3.580372in}} %
\pgfusepath{clip}%
\pgfsetbuttcap%
\pgfsetroundjoin%
\definecolor{currentfill}{rgb}{0.257234,0.339661,0.789661}%
\pgfsetfillcolor{currentfill}%
\pgfsetlinewidth{0.000000pt}%
\definecolor{currentstroke}{rgb}{0.000000,0.000000,0.000000}%
\pgfsetstrokecolor{currentstroke}%
\pgfsetdash{}{0pt}%
\pgfpathmoveto{\pgfqpoint{2.789661in}{0.820654in}}%
\pgfpathlineto{\pgfqpoint{3.043201in}{2.166904in}}%
\pgfpathlineto{\pgfqpoint{2.789661in}{0.820654in}}%
\pgfpathclose%
\pgfusepath{fill}%
\end{pgfscope}%
\begin{pgfscope}%
\pgfpathrectangle{\pgfqpoint{0.017500in}{0.017500in}}{\pgfqpoint{6.218282in}{3.580372in}} %
\pgfusepath{clip}%
\pgfsetbuttcap%
\pgfsetroundjoin%
\definecolor{currentfill}{rgb}{0.257234,0.339661,0.789661}%
\pgfsetfillcolor{currentfill}%
\pgfsetlinewidth{0.000000pt}%
\definecolor{currentstroke}{rgb}{0.000000,0.000000,0.000000}%
\pgfsetstrokecolor{currentstroke}%
\pgfsetdash{}{0pt}%
\pgfpathmoveto{\pgfqpoint{2.789661in}{0.820654in}}%
\pgfpathlineto{\pgfqpoint{3.043201in}{2.166904in}}%
\pgfpathlineto{\pgfqpoint{2.789661in}{0.820654in}}%
\pgfpathclose%
\pgfusepath{fill}%
\end{pgfscope}%
\begin{pgfscope}%
\pgfpathrectangle{\pgfqpoint{0.017500in}{0.017500in}}{\pgfqpoint{6.218282in}{3.580372in}} %
\pgfusepath{clip}%
\pgfsetbuttcap%
\pgfsetroundjoin%
\definecolor{currentfill}{rgb}{0.257234,0.339661,0.789661}%
\pgfsetfillcolor{currentfill}%
\pgfsetlinewidth{0.000000pt}%
\definecolor{currentstroke}{rgb}{0.000000,0.000000,0.000000}%
\pgfsetstrokecolor{currentstroke}%
\pgfsetdash{}{0pt}%
\pgfpathmoveto{\pgfqpoint{2.789661in}{0.820654in}}%
\pgfpathlineto{\pgfqpoint{3.043201in}{2.166904in}}%
\pgfpathlineto{\pgfqpoint{2.789661in}{0.820654in}}%
\pgfpathclose%
\pgfusepath{fill}%
\end{pgfscope}%
\begin{pgfscope}%
\pgfpathrectangle{\pgfqpoint{0.017500in}{0.017500in}}{\pgfqpoint{6.218282in}{3.580372in}} %
\pgfusepath{clip}%
\pgfsetbuttcap%
\pgfsetroundjoin%
\definecolor{currentfill}{rgb}{0.257234,0.339661,0.789661}%
\pgfsetfillcolor{currentfill}%
\pgfsetlinewidth{0.000000pt}%
\definecolor{currentstroke}{rgb}{0.000000,0.000000,0.000000}%
\pgfsetstrokecolor{currentstroke}%
\pgfsetdash{}{0pt}%
\pgfpathmoveto{\pgfqpoint{2.789661in}{0.820654in}}%
\pgfpathlineto{\pgfqpoint{3.043201in}{2.166904in}}%
\pgfpathlineto{\pgfqpoint{2.789661in}{0.820654in}}%
\pgfpathclose%
\pgfusepath{fill}%
\end{pgfscope}%
\begin{pgfscope}%
\pgfpathrectangle{\pgfqpoint{0.017500in}{0.017500in}}{\pgfqpoint{6.218282in}{3.580372in}} %
\pgfusepath{clip}%
\pgfsetbuttcap%
\pgfsetroundjoin%
\definecolor{currentfill}{rgb}{0.257234,0.339661,0.789661}%
\pgfsetfillcolor{currentfill}%
\pgfsetlinewidth{0.000000pt}%
\definecolor{currentstroke}{rgb}{0.000000,0.000000,0.000000}%
\pgfsetstrokecolor{currentstroke}%
\pgfsetdash{}{0pt}%
\pgfpathmoveto{\pgfqpoint{2.789661in}{0.820654in}}%
\pgfpathlineto{\pgfqpoint{3.043201in}{2.166904in}}%
\pgfpathlineto{\pgfqpoint{2.789661in}{0.820654in}}%
\pgfpathclose%
\pgfusepath{fill}%
\end{pgfscope}%
\begin{pgfscope}%
\pgfpathrectangle{\pgfqpoint{0.017500in}{0.017500in}}{\pgfqpoint{6.218282in}{3.580372in}} %
\pgfusepath{clip}%
\pgfsetbuttcap%
\pgfsetroundjoin%
\definecolor{currentfill}{rgb}{0.257234,0.339661,0.789661}%
\pgfsetfillcolor{currentfill}%
\pgfsetlinewidth{0.000000pt}%
\definecolor{currentstroke}{rgb}{0.000000,0.000000,0.000000}%
\pgfsetstrokecolor{currentstroke}%
\pgfsetdash{}{0pt}%
\pgfpathmoveto{\pgfqpoint{2.789661in}{0.820654in}}%
\pgfpathlineto{\pgfqpoint{3.043201in}{2.166904in}}%
\pgfpathlineto{\pgfqpoint{2.789661in}{0.820654in}}%
\pgfpathclose%
\pgfusepath{fill}%
\end{pgfscope}%
\begin{pgfscope}%
\pgfpathrectangle{\pgfqpoint{0.017500in}{0.017500in}}{\pgfqpoint{6.218282in}{3.580372in}} %
\pgfusepath{clip}%
\pgfsetbuttcap%
\pgfsetroundjoin%
\definecolor{currentfill}{rgb}{0.257234,0.339661,0.789661}%
\pgfsetfillcolor{currentfill}%
\pgfsetlinewidth{0.000000pt}%
\definecolor{currentstroke}{rgb}{0.000000,0.000000,0.000000}%
\pgfsetstrokecolor{currentstroke}%
\pgfsetdash{}{0pt}%
\pgfpathmoveto{\pgfqpoint{2.789661in}{0.820654in}}%
\pgfpathlineto{\pgfqpoint{3.043201in}{2.166904in}}%
\pgfpathlineto{\pgfqpoint{2.789661in}{0.820654in}}%
\pgfpathclose%
\pgfusepath{fill}%
\end{pgfscope}%
\begin{pgfscope}%
\pgfpathrectangle{\pgfqpoint{0.017500in}{0.017500in}}{\pgfqpoint{6.218282in}{3.580372in}} %
\pgfusepath{clip}%
\pgfsetbuttcap%
\pgfsetroundjoin%
\definecolor{currentfill}{rgb}{0.257234,0.339661,0.789661}%
\pgfsetfillcolor{currentfill}%
\pgfsetlinewidth{0.000000pt}%
\definecolor{currentstroke}{rgb}{0.000000,0.000000,0.000000}%
\pgfsetstrokecolor{currentstroke}%
\pgfsetdash{}{0pt}%
\pgfpathmoveto{\pgfqpoint{2.789661in}{0.820654in}}%
\pgfpathlineto{\pgfqpoint{3.043201in}{2.166904in}}%
\pgfpathlineto{\pgfqpoint{2.789661in}{0.820654in}}%
\pgfpathclose%
\pgfusepath{fill}%
\end{pgfscope}%
\begin{pgfscope}%
\pgfpathrectangle{\pgfqpoint{0.017500in}{0.017500in}}{\pgfqpoint{6.218282in}{3.580372in}} %
\pgfusepath{clip}%
\pgfsetbuttcap%
\pgfsetroundjoin%
\definecolor{currentfill}{rgb}{0.257234,0.339661,0.789661}%
\pgfsetfillcolor{currentfill}%
\pgfsetlinewidth{0.000000pt}%
\definecolor{currentstroke}{rgb}{0.000000,0.000000,0.000000}%
\pgfsetstrokecolor{currentstroke}%
\pgfsetdash{}{0pt}%
\pgfpathmoveto{\pgfqpoint{2.789661in}{0.820654in}}%
\pgfpathlineto{\pgfqpoint{3.043201in}{2.166904in}}%
\pgfpathlineto{\pgfqpoint{2.789661in}{0.820654in}}%
\pgfpathclose%
\pgfusepath{fill}%
\end{pgfscope}%
\begin{pgfscope}%
\pgfpathrectangle{\pgfqpoint{0.017500in}{0.017500in}}{\pgfqpoint{6.218282in}{3.580372in}} %
\pgfusepath{clip}%
\pgfsetbuttcap%
\pgfsetroundjoin%
\definecolor{currentfill}{rgb}{0.257234,0.339661,0.789661}%
\pgfsetfillcolor{currentfill}%
\pgfsetlinewidth{0.000000pt}%
\definecolor{currentstroke}{rgb}{0.000000,0.000000,0.000000}%
\pgfsetstrokecolor{currentstroke}%
\pgfsetdash{}{0pt}%
\pgfpathmoveto{\pgfqpoint{2.789661in}{0.820654in}}%
\pgfpathlineto{\pgfqpoint{3.043201in}{2.166904in}}%
\pgfpathlineto{\pgfqpoint{2.789661in}{0.820654in}}%
\pgfpathclose%
\pgfusepath{fill}%
\end{pgfscope}%
\begin{pgfscope}%
\pgfpathrectangle{\pgfqpoint{0.017500in}{0.017500in}}{\pgfqpoint{6.218282in}{3.580372in}} %
\pgfusepath{clip}%
\pgfsetbuttcap%
\pgfsetroundjoin%
\definecolor{currentfill}{rgb}{0.257234,0.339661,0.789661}%
\pgfsetfillcolor{currentfill}%
\pgfsetlinewidth{0.000000pt}%
\definecolor{currentstroke}{rgb}{0.000000,0.000000,0.000000}%
\pgfsetstrokecolor{currentstroke}%
\pgfsetdash{}{0pt}%
\pgfpathmoveto{\pgfqpoint{2.789661in}{0.820654in}}%
\pgfpathlineto{\pgfqpoint{3.043201in}{2.166904in}}%
\pgfpathlineto{\pgfqpoint{2.789661in}{0.820654in}}%
\pgfpathclose%
\pgfusepath{fill}%
\end{pgfscope}%
\begin{pgfscope}%
\pgfpathrectangle{\pgfqpoint{0.017500in}{0.017500in}}{\pgfqpoint{6.218282in}{3.580372in}} %
\pgfusepath{clip}%
\pgfsetbuttcap%
\pgfsetroundjoin%
\definecolor{currentfill}{rgb}{0.257234,0.339661,0.789661}%
\pgfsetfillcolor{currentfill}%
\pgfsetlinewidth{0.000000pt}%
\definecolor{currentstroke}{rgb}{0.000000,0.000000,0.000000}%
\pgfsetstrokecolor{currentstroke}%
\pgfsetdash{}{0pt}%
\pgfpathmoveto{\pgfqpoint{2.789661in}{0.820654in}}%
\pgfpathlineto{\pgfqpoint{3.043201in}{2.166904in}}%
\pgfpathlineto{\pgfqpoint{2.789661in}{0.820654in}}%
\pgfpathclose%
\pgfusepath{fill}%
\end{pgfscope}%
\begin{pgfscope}%
\pgfpathrectangle{\pgfqpoint{0.017500in}{0.017500in}}{\pgfqpoint{6.218282in}{3.580372in}} %
\pgfusepath{clip}%
\pgfsetbuttcap%
\pgfsetroundjoin%
\definecolor{currentfill}{rgb}{0.257234,0.339661,0.789661}%
\pgfsetfillcolor{currentfill}%
\pgfsetlinewidth{0.000000pt}%
\definecolor{currentstroke}{rgb}{0.000000,0.000000,0.000000}%
\pgfsetstrokecolor{currentstroke}%
\pgfsetdash{}{0pt}%
\pgfpathmoveto{\pgfqpoint{2.789661in}{0.820654in}}%
\pgfpathlineto{\pgfqpoint{3.043201in}{2.166904in}}%
\pgfpathlineto{\pgfqpoint{2.789661in}{0.820654in}}%
\pgfpathclose%
\pgfusepath{fill}%
\end{pgfscope}%
\begin{pgfscope}%
\pgfpathrectangle{\pgfqpoint{0.017500in}{0.017500in}}{\pgfqpoint{6.218282in}{3.580372in}} %
\pgfusepath{clip}%
\pgfsetbuttcap%
\pgfsetroundjoin%
\definecolor{currentfill}{rgb}{0.257234,0.339661,0.789661}%
\pgfsetfillcolor{currentfill}%
\pgfsetlinewidth{0.000000pt}%
\definecolor{currentstroke}{rgb}{0.000000,0.000000,0.000000}%
\pgfsetstrokecolor{currentstroke}%
\pgfsetdash{}{0pt}%
\pgfpathmoveto{\pgfqpoint{2.789661in}{0.820654in}}%
\pgfpathlineto{\pgfqpoint{3.043201in}{2.166904in}}%
\pgfpathlineto{\pgfqpoint{2.789661in}{0.820654in}}%
\pgfpathclose%
\pgfusepath{fill}%
\end{pgfscope}%
\begin{pgfscope}%
\pgfpathrectangle{\pgfqpoint{0.017500in}{0.017500in}}{\pgfqpoint{6.218282in}{3.580372in}} %
\pgfusepath{clip}%
\pgfsetbuttcap%
\pgfsetroundjoin%
\definecolor{currentfill}{rgb}{0.257234,0.339661,0.789661}%
\pgfsetfillcolor{currentfill}%
\pgfsetlinewidth{0.000000pt}%
\definecolor{currentstroke}{rgb}{0.000000,0.000000,0.000000}%
\pgfsetstrokecolor{currentstroke}%
\pgfsetdash{}{0pt}%
\pgfpathmoveto{\pgfqpoint{2.789661in}{0.820654in}}%
\pgfpathlineto{\pgfqpoint{3.043201in}{2.166904in}}%
\pgfpathlineto{\pgfqpoint{2.789661in}{0.820654in}}%
\pgfpathclose%
\pgfusepath{fill}%
\end{pgfscope}%
\begin{pgfscope}%
\pgfpathrectangle{\pgfqpoint{0.017500in}{0.017500in}}{\pgfqpoint{6.218282in}{3.580372in}} %
\pgfusepath{clip}%
\pgfsetbuttcap%
\pgfsetroundjoin%
\definecolor{currentfill}{rgb}{0.746838,0.140021,0.179996}%
\pgfsetfillcolor{currentfill}%
\pgfsetlinewidth{0.000000pt}%
\definecolor{currentstroke}{rgb}{0.000000,0.000000,0.000000}%
\pgfsetstrokecolor{currentstroke}%
\pgfsetdash{}{0pt}%
\pgfpathmoveto{\pgfqpoint{4.018512in}{2.470906in}}%
\pgfpathlineto{\pgfqpoint{4.268228in}{2.632966in}}%
\pgfpathlineto{\pgfqpoint{4.018512in}{2.470906in}}%
\pgfpathclose%
\pgfusepath{fill}%
\end{pgfscope}%
\begin{pgfscope}%
\pgfpathrectangle{\pgfqpoint{0.017500in}{0.017500in}}{\pgfqpoint{6.218282in}{3.580372in}} %
\pgfusepath{clip}%
\pgfsetbuttcap%
\pgfsetroundjoin%
\definecolor{currentfill}{rgb}{0.746838,0.140021,0.179996}%
\pgfsetfillcolor{currentfill}%
\pgfsetlinewidth{0.000000pt}%
\definecolor{currentstroke}{rgb}{0.000000,0.000000,0.000000}%
\pgfsetstrokecolor{currentstroke}%
\pgfsetdash{}{0pt}%
\pgfpathmoveto{\pgfqpoint{4.018512in}{2.470906in}}%
\pgfpathlineto{\pgfqpoint{4.268228in}{2.632966in}}%
\pgfpathlineto{\pgfqpoint{4.018512in}{2.470906in}}%
\pgfpathclose%
\pgfusepath{fill}%
\end{pgfscope}%
\begin{pgfscope}%
\pgfpathrectangle{\pgfqpoint{0.017500in}{0.017500in}}{\pgfqpoint{6.218282in}{3.580372in}} %
\pgfusepath{clip}%
\pgfsetbuttcap%
\pgfsetroundjoin%
\definecolor{currentfill}{rgb}{0.746838,0.140021,0.179996}%
\pgfsetfillcolor{currentfill}%
\pgfsetlinewidth{0.000000pt}%
\definecolor{currentstroke}{rgb}{0.000000,0.000000,0.000000}%
\pgfsetstrokecolor{currentstroke}%
\pgfsetdash{}{0pt}%
\pgfpathmoveto{\pgfqpoint{4.018512in}{2.470906in}}%
\pgfpathlineto{\pgfqpoint{4.268228in}{2.632966in}}%
\pgfpathlineto{\pgfqpoint{4.018512in}{2.470906in}}%
\pgfpathclose%
\pgfusepath{fill}%
\end{pgfscope}%
\begin{pgfscope}%
\pgfpathrectangle{\pgfqpoint{0.017500in}{0.017500in}}{\pgfqpoint{6.218282in}{3.580372in}} %
\pgfusepath{clip}%
\pgfsetbuttcap%
\pgfsetroundjoin%
\definecolor{currentfill}{rgb}{0.746838,0.140021,0.179996}%
\pgfsetfillcolor{currentfill}%
\pgfsetlinewidth{0.000000pt}%
\definecolor{currentstroke}{rgb}{0.000000,0.000000,0.000000}%
\pgfsetstrokecolor{currentstroke}%
\pgfsetdash{}{0pt}%
\pgfpathmoveto{\pgfqpoint{4.018512in}{2.470906in}}%
\pgfpathlineto{\pgfqpoint{4.268228in}{2.632966in}}%
\pgfpathlineto{\pgfqpoint{4.018512in}{2.470906in}}%
\pgfpathclose%
\pgfusepath{fill}%
\end{pgfscope}%
\begin{pgfscope}%
\pgfpathrectangle{\pgfqpoint{0.017500in}{0.017500in}}{\pgfqpoint{6.218282in}{3.580372in}} %
\pgfusepath{clip}%
\pgfsetbuttcap%
\pgfsetroundjoin%
\definecolor{currentfill}{rgb}{0.746838,0.140021,0.179996}%
\pgfsetfillcolor{currentfill}%
\pgfsetlinewidth{0.000000pt}%
\definecolor{currentstroke}{rgb}{0.000000,0.000000,0.000000}%
\pgfsetstrokecolor{currentstroke}%
\pgfsetdash{}{0pt}%
\pgfpathmoveto{\pgfqpoint{4.018512in}{2.470906in}}%
\pgfpathlineto{\pgfqpoint{4.268228in}{2.632966in}}%
\pgfpathlineto{\pgfqpoint{4.018512in}{2.470906in}}%
\pgfpathclose%
\pgfusepath{fill}%
\end{pgfscope}%
\begin{pgfscope}%
\pgfpathrectangle{\pgfqpoint{0.017500in}{0.017500in}}{\pgfqpoint{6.218282in}{3.580372in}} %
\pgfusepath{clip}%
\pgfsetbuttcap%
\pgfsetroundjoin%
\definecolor{currentfill}{rgb}{0.746838,0.140021,0.179996}%
\pgfsetfillcolor{currentfill}%
\pgfsetlinewidth{0.000000pt}%
\definecolor{currentstroke}{rgb}{0.000000,0.000000,0.000000}%
\pgfsetstrokecolor{currentstroke}%
\pgfsetdash{}{0pt}%
\pgfpathmoveto{\pgfqpoint{4.018512in}{2.470906in}}%
\pgfpathlineto{\pgfqpoint{4.268228in}{2.632966in}}%
\pgfpathlineto{\pgfqpoint{4.018512in}{2.470906in}}%
\pgfpathclose%
\pgfusepath{fill}%
\end{pgfscope}%
\begin{pgfscope}%
\pgfpathrectangle{\pgfqpoint{0.017500in}{0.017500in}}{\pgfqpoint{6.218282in}{3.580372in}} %
\pgfusepath{clip}%
\pgfsetbuttcap%
\pgfsetroundjoin%
\definecolor{currentfill}{rgb}{0.746838,0.140021,0.179996}%
\pgfsetfillcolor{currentfill}%
\pgfsetlinewidth{0.000000pt}%
\definecolor{currentstroke}{rgb}{0.000000,0.000000,0.000000}%
\pgfsetstrokecolor{currentstroke}%
\pgfsetdash{}{0pt}%
\pgfpathmoveto{\pgfqpoint{4.018512in}{2.470906in}}%
\pgfpathlineto{\pgfqpoint{4.268228in}{2.632966in}}%
\pgfpathlineto{\pgfqpoint{4.018512in}{2.470906in}}%
\pgfpathclose%
\pgfusepath{fill}%
\end{pgfscope}%
\begin{pgfscope}%
\pgfpathrectangle{\pgfqpoint{0.017500in}{0.017500in}}{\pgfqpoint{6.218282in}{3.580372in}} %
\pgfusepath{clip}%
\pgfsetbuttcap%
\pgfsetroundjoin%
\definecolor{currentfill}{rgb}{0.746838,0.140021,0.179996}%
\pgfsetfillcolor{currentfill}%
\pgfsetlinewidth{0.000000pt}%
\definecolor{currentstroke}{rgb}{0.000000,0.000000,0.000000}%
\pgfsetstrokecolor{currentstroke}%
\pgfsetdash{}{0pt}%
\pgfpathmoveto{\pgfqpoint{4.018512in}{2.470906in}}%
\pgfpathlineto{\pgfqpoint{4.268228in}{2.632966in}}%
\pgfpathlineto{\pgfqpoint{4.018512in}{2.470906in}}%
\pgfpathclose%
\pgfusepath{fill}%
\end{pgfscope}%
\begin{pgfscope}%
\pgfpathrectangle{\pgfqpoint{0.017500in}{0.017500in}}{\pgfqpoint{6.218282in}{3.580372in}} %
\pgfusepath{clip}%
\pgfsetbuttcap%
\pgfsetroundjoin%
\definecolor{currentfill}{rgb}{0.746838,0.140021,0.179996}%
\pgfsetfillcolor{currentfill}%
\pgfsetlinewidth{0.000000pt}%
\definecolor{currentstroke}{rgb}{0.000000,0.000000,0.000000}%
\pgfsetstrokecolor{currentstroke}%
\pgfsetdash{}{0pt}%
\pgfpathmoveto{\pgfqpoint{4.018512in}{2.470906in}}%
\pgfpathlineto{\pgfqpoint{4.268228in}{2.632966in}}%
\pgfpathlineto{\pgfqpoint{4.018512in}{2.470906in}}%
\pgfpathclose%
\pgfusepath{fill}%
\end{pgfscope}%
\begin{pgfscope}%
\pgfpathrectangle{\pgfqpoint{0.017500in}{0.017500in}}{\pgfqpoint{6.218282in}{3.580372in}} %
\pgfusepath{clip}%
\pgfsetbuttcap%
\pgfsetroundjoin%
\definecolor{currentfill}{rgb}{0.746838,0.140021,0.179996}%
\pgfsetfillcolor{currentfill}%
\pgfsetlinewidth{0.000000pt}%
\definecolor{currentstroke}{rgb}{0.000000,0.000000,0.000000}%
\pgfsetstrokecolor{currentstroke}%
\pgfsetdash{}{0pt}%
\pgfpathmoveto{\pgfqpoint{4.018512in}{2.470906in}}%
\pgfpathlineto{\pgfqpoint{4.268228in}{2.632966in}}%
\pgfpathlineto{\pgfqpoint{4.018512in}{2.470906in}}%
\pgfpathclose%
\pgfusepath{fill}%
\end{pgfscope}%
\begin{pgfscope}%
\pgfpathrectangle{\pgfqpoint{0.017500in}{0.017500in}}{\pgfqpoint{6.218282in}{3.580372in}} %
\pgfusepath{clip}%
\pgfsetbuttcap%
\pgfsetroundjoin%
\definecolor{currentfill}{rgb}{0.746838,0.140021,0.179996}%
\pgfsetfillcolor{currentfill}%
\pgfsetlinewidth{0.000000pt}%
\definecolor{currentstroke}{rgb}{0.000000,0.000000,0.000000}%
\pgfsetstrokecolor{currentstroke}%
\pgfsetdash{}{0pt}%
\pgfpathmoveto{\pgfqpoint{4.018512in}{2.470906in}}%
\pgfpathlineto{\pgfqpoint{4.268228in}{2.632966in}}%
\pgfpathlineto{\pgfqpoint{4.018512in}{2.470906in}}%
\pgfpathclose%
\pgfusepath{fill}%
\end{pgfscope}%
\begin{pgfscope}%
\pgfpathrectangle{\pgfqpoint{0.017500in}{0.017500in}}{\pgfqpoint{6.218282in}{3.580372in}} %
\pgfusepath{clip}%
\pgfsetbuttcap%
\pgfsetroundjoin%
\definecolor{currentfill}{rgb}{0.746838,0.140021,0.179996}%
\pgfsetfillcolor{currentfill}%
\pgfsetlinewidth{0.000000pt}%
\definecolor{currentstroke}{rgb}{0.000000,0.000000,0.000000}%
\pgfsetstrokecolor{currentstroke}%
\pgfsetdash{}{0pt}%
\pgfpathmoveto{\pgfqpoint{4.018512in}{2.470906in}}%
\pgfpathlineto{\pgfqpoint{4.268228in}{2.632966in}}%
\pgfpathlineto{\pgfqpoint{4.018512in}{2.470906in}}%
\pgfpathclose%
\pgfusepath{fill}%
\end{pgfscope}%
\begin{pgfscope}%
\pgfpathrectangle{\pgfqpoint{0.017500in}{0.017500in}}{\pgfqpoint{6.218282in}{3.580372in}} %
\pgfusepath{clip}%
\pgfsetbuttcap%
\pgfsetroundjoin%
\definecolor{currentfill}{rgb}{0.746838,0.140021,0.179996}%
\pgfsetfillcolor{currentfill}%
\pgfsetlinewidth{0.000000pt}%
\definecolor{currentstroke}{rgb}{0.000000,0.000000,0.000000}%
\pgfsetstrokecolor{currentstroke}%
\pgfsetdash{}{0pt}%
\pgfpathmoveto{\pgfqpoint{4.018512in}{2.470906in}}%
\pgfpathlineto{\pgfqpoint{4.268228in}{2.632966in}}%
\pgfpathlineto{\pgfqpoint{4.018512in}{2.470906in}}%
\pgfpathclose%
\pgfusepath{fill}%
\end{pgfscope}%
\begin{pgfscope}%
\pgfpathrectangle{\pgfqpoint{0.017500in}{0.017500in}}{\pgfqpoint{6.218282in}{3.580372in}} %
\pgfusepath{clip}%
\pgfsetbuttcap%
\pgfsetroundjoin%
\definecolor{currentfill}{rgb}{0.746838,0.140021,0.179996}%
\pgfsetfillcolor{currentfill}%
\pgfsetlinewidth{0.000000pt}%
\definecolor{currentstroke}{rgb}{0.000000,0.000000,0.000000}%
\pgfsetstrokecolor{currentstroke}%
\pgfsetdash{}{0pt}%
\pgfpathmoveto{\pgfqpoint{4.018512in}{2.470906in}}%
\pgfpathlineto{\pgfqpoint{4.268228in}{2.632966in}}%
\pgfpathlineto{\pgfqpoint{4.018512in}{2.470906in}}%
\pgfpathclose%
\pgfusepath{fill}%
\end{pgfscope}%
\begin{pgfscope}%
\pgfpathrectangle{\pgfqpoint{0.017500in}{0.017500in}}{\pgfqpoint{6.218282in}{3.580372in}} %
\pgfusepath{clip}%
\pgfsetbuttcap%
\pgfsetroundjoin%
\definecolor{currentfill}{rgb}{0.746838,0.140021,0.179996}%
\pgfsetfillcolor{currentfill}%
\pgfsetlinewidth{0.000000pt}%
\definecolor{currentstroke}{rgb}{0.000000,0.000000,0.000000}%
\pgfsetstrokecolor{currentstroke}%
\pgfsetdash{}{0pt}%
\pgfpathmoveto{\pgfqpoint{4.018512in}{2.470906in}}%
\pgfpathlineto{\pgfqpoint{4.268228in}{2.632966in}}%
\pgfpathlineto{\pgfqpoint{4.018512in}{2.470906in}}%
\pgfpathclose%
\pgfusepath{fill}%
\end{pgfscope}%
\begin{pgfscope}%
\pgfpathrectangle{\pgfqpoint{0.017500in}{0.017500in}}{\pgfqpoint{6.218282in}{3.580372in}} %
\pgfusepath{clip}%
\pgfsetbuttcap%
\pgfsetroundjoin%
\definecolor{currentfill}{rgb}{0.746838,0.140021,0.179996}%
\pgfsetfillcolor{currentfill}%
\pgfsetlinewidth{0.000000pt}%
\definecolor{currentstroke}{rgb}{0.000000,0.000000,0.000000}%
\pgfsetstrokecolor{currentstroke}%
\pgfsetdash{}{0pt}%
\pgfpathmoveto{\pgfqpoint{4.018512in}{2.470906in}}%
\pgfpathlineto{\pgfqpoint{4.268228in}{2.632966in}}%
\pgfpathlineto{\pgfqpoint{4.018512in}{2.470906in}}%
\pgfpathclose%
\pgfusepath{fill}%
\end{pgfscope}%
\begin{pgfscope}%
\pgfpathrectangle{\pgfqpoint{0.017500in}{0.017500in}}{\pgfqpoint{6.218282in}{3.580372in}} %
\pgfusepath{clip}%
\pgfsetbuttcap%
\pgfsetroundjoin%
\definecolor{currentfill}{rgb}{0.746838,0.140021,0.179996}%
\pgfsetfillcolor{currentfill}%
\pgfsetlinewidth{0.000000pt}%
\definecolor{currentstroke}{rgb}{0.000000,0.000000,0.000000}%
\pgfsetstrokecolor{currentstroke}%
\pgfsetdash{}{0pt}%
\pgfpathmoveto{\pgfqpoint{4.018512in}{2.470906in}}%
\pgfpathlineto{\pgfqpoint{4.268228in}{2.632966in}}%
\pgfpathlineto{\pgfqpoint{4.018512in}{2.470906in}}%
\pgfpathclose%
\pgfusepath{fill}%
\end{pgfscope}%
\begin{pgfscope}%
\pgfpathrectangle{\pgfqpoint{0.017500in}{0.017500in}}{\pgfqpoint{6.218282in}{3.580372in}} %
\pgfusepath{clip}%
\pgfsetbuttcap%
\pgfsetroundjoin%
\definecolor{currentfill}{rgb}{0.746838,0.140021,0.179996}%
\pgfsetfillcolor{currentfill}%
\pgfsetlinewidth{0.000000pt}%
\definecolor{currentstroke}{rgb}{0.000000,0.000000,0.000000}%
\pgfsetstrokecolor{currentstroke}%
\pgfsetdash{}{0pt}%
\pgfpathmoveto{\pgfqpoint{4.018512in}{2.470906in}}%
\pgfpathlineto{\pgfqpoint{4.268228in}{2.632966in}}%
\pgfpathlineto{\pgfqpoint{4.018512in}{2.470906in}}%
\pgfpathclose%
\pgfusepath{fill}%
\end{pgfscope}%
\begin{pgfscope}%
\pgfpathrectangle{\pgfqpoint{0.017500in}{0.017500in}}{\pgfqpoint{6.218282in}{3.580372in}} %
\pgfusepath{clip}%
\pgfsetbuttcap%
\pgfsetroundjoin%
\definecolor{currentfill}{rgb}{0.746838,0.140021,0.179996}%
\pgfsetfillcolor{currentfill}%
\pgfsetlinewidth{0.000000pt}%
\definecolor{currentstroke}{rgb}{0.000000,0.000000,0.000000}%
\pgfsetstrokecolor{currentstroke}%
\pgfsetdash{}{0pt}%
\pgfpathmoveto{\pgfqpoint{4.018512in}{2.470906in}}%
\pgfpathlineto{\pgfqpoint{4.268228in}{2.632966in}}%
\pgfpathlineto{\pgfqpoint{4.018512in}{2.470906in}}%
\pgfpathclose%
\pgfusepath{fill}%
\end{pgfscope}%
\begin{pgfscope}%
\pgfpathrectangle{\pgfqpoint{0.017500in}{0.017500in}}{\pgfqpoint{6.218282in}{3.580372in}} %
\pgfusepath{clip}%
\pgfsetbuttcap%
\pgfsetroundjoin%
\definecolor{currentfill}{rgb}{0.746838,0.140021,0.179996}%
\pgfsetfillcolor{currentfill}%
\pgfsetlinewidth{0.000000pt}%
\definecolor{currentstroke}{rgb}{0.000000,0.000000,0.000000}%
\pgfsetstrokecolor{currentstroke}%
\pgfsetdash{}{0pt}%
\pgfpathmoveto{\pgfqpoint{4.018512in}{2.470906in}}%
\pgfpathlineto{\pgfqpoint{4.268228in}{2.632966in}}%
\pgfpathlineto{\pgfqpoint{4.018512in}{2.470906in}}%
\pgfpathclose%
\pgfusepath{fill}%
\end{pgfscope}%
\begin{pgfscope}%
\pgfpathrectangle{\pgfqpoint{0.017500in}{0.017500in}}{\pgfqpoint{6.218282in}{3.580372in}} %
\pgfusepath{clip}%
\pgfsetbuttcap%
\pgfsetroundjoin%
\definecolor{currentfill}{rgb}{0.746838,0.140021,0.179996}%
\pgfsetfillcolor{currentfill}%
\pgfsetlinewidth{0.000000pt}%
\definecolor{currentstroke}{rgb}{0.000000,0.000000,0.000000}%
\pgfsetstrokecolor{currentstroke}%
\pgfsetdash{}{0pt}%
\pgfpathmoveto{\pgfqpoint{4.018512in}{2.470906in}}%
\pgfpathlineto{\pgfqpoint{4.268228in}{2.632966in}}%
\pgfpathlineto{\pgfqpoint{4.018512in}{2.470906in}}%
\pgfpathclose%
\pgfusepath{fill}%
\end{pgfscope}%
\begin{pgfscope}%
\pgfpathrectangle{\pgfqpoint{0.017500in}{0.017500in}}{\pgfqpoint{6.218282in}{3.580372in}} %
\pgfusepath{clip}%
\pgfsetbuttcap%
\pgfsetroundjoin%
\definecolor{currentfill}{rgb}{0.746838,0.140021,0.179996}%
\pgfsetfillcolor{currentfill}%
\pgfsetlinewidth{0.000000pt}%
\definecolor{currentstroke}{rgb}{0.000000,0.000000,0.000000}%
\pgfsetstrokecolor{currentstroke}%
\pgfsetdash{}{0pt}%
\pgfpathmoveto{\pgfqpoint{4.018512in}{2.470906in}}%
\pgfpathlineto{\pgfqpoint{4.268228in}{2.632966in}}%
\pgfpathlineto{\pgfqpoint{4.018512in}{2.470906in}}%
\pgfpathclose%
\pgfusepath{fill}%
\end{pgfscope}%
\begin{pgfscope}%
\pgfpathrectangle{\pgfqpoint{0.017500in}{0.017500in}}{\pgfqpoint{6.218282in}{3.580372in}} %
\pgfusepath{clip}%
\pgfsetbuttcap%
\pgfsetroundjoin%
\definecolor{currentfill}{rgb}{0.746838,0.140021,0.179996}%
\pgfsetfillcolor{currentfill}%
\pgfsetlinewidth{0.000000pt}%
\definecolor{currentstroke}{rgb}{0.000000,0.000000,0.000000}%
\pgfsetstrokecolor{currentstroke}%
\pgfsetdash{}{0pt}%
\pgfpathmoveto{\pgfqpoint{4.018512in}{2.470906in}}%
\pgfpathlineto{\pgfqpoint{4.268228in}{2.632966in}}%
\pgfpathlineto{\pgfqpoint{4.018512in}{2.470906in}}%
\pgfpathclose%
\pgfusepath{fill}%
\end{pgfscope}%
\begin{pgfscope}%
\pgfpathrectangle{\pgfqpoint{0.017500in}{0.017500in}}{\pgfqpoint{6.218282in}{3.580372in}} %
\pgfusepath{clip}%
\pgfsetbuttcap%
\pgfsetroundjoin%
\definecolor{currentfill}{rgb}{0.746838,0.140021,0.179996}%
\pgfsetfillcolor{currentfill}%
\pgfsetlinewidth{0.000000pt}%
\definecolor{currentstroke}{rgb}{0.000000,0.000000,0.000000}%
\pgfsetstrokecolor{currentstroke}%
\pgfsetdash{}{0pt}%
\pgfpathmoveto{\pgfqpoint{4.018512in}{2.470906in}}%
\pgfpathlineto{\pgfqpoint{4.268228in}{2.632966in}}%
\pgfpathlineto{\pgfqpoint{4.018512in}{2.470906in}}%
\pgfpathclose%
\pgfusepath{fill}%
\end{pgfscope}%
\begin{pgfscope}%
\pgfpathrectangle{\pgfqpoint{0.017500in}{0.017500in}}{\pgfqpoint{6.218282in}{3.580372in}} %
\pgfusepath{clip}%
\pgfsetbuttcap%
\pgfsetroundjoin%
\definecolor{currentfill}{rgb}{0.746838,0.140021,0.179996}%
\pgfsetfillcolor{currentfill}%
\pgfsetlinewidth{0.000000pt}%
\definecolor{currentstroke}{rgb}{0.000000,0.000000,0.000000}%
\pgfsetstrokecolor{currentstroke}%
\pgfsetdash{}{0pt}%
\pgfpathmoveto{\pgfqpoint{4.018512in}{2.470906in}}%
\pgfpathlineto{\pgfqpoint{4.268228in}{2.632966in}}%
\pgfpathlineto{\pgfqpoint{4.018512in}{2.470906in}}%
\pgfpathclose%
\pgfusepath{fill}%
\end{pgfscope}%
\begin{pgfscope}%
\pgfpathrectangle{\pgfqpoint{0.017500in}{0.017500in}}{\pgfqpoint{6.218282in}{3.580372in}} %
\pgfusepath{clip}%
\pgfsetbuttcap%
\pgfsetroundjoin%
\definecolor{currentfill}{rgb}{0.746838,0.140021,0.179996}%
\pgfsetfillcolor{currentfill}%
\pgfsetlinewidth{0.000000pt}%
\definecolor{currentstroke}{rgb}{0.000000,0.000000,0.000000}%
\pgfsetstrokecolor{currentstroke}%
\pgfsetdash{}{0pt}%
\pgfpathmoveto{\pgfqpoint{4.018512in}{2.470906in}}%
\pgfpathlineto{\pgfqpoint{4.268228in}{2.632966in}}%
\pgfpathlineto{\pgfqpoint{4.018512in}{2.470906in}}%
\pgfpathclose%
\pgfusepath{fill}%
\end{pgfscope}%
\begin{pgfscope}%
\pgfpathrectangle{\pgfqpoint{0.017500in}{0.017500in}}{\pgfqpoint{6.218282in}{3.580372in}} %
\pgfusepath{clip}%
\pgfsetbuttcap%
\pgfsetroundjoin%
\definecolor{currentfill}{rgb}{0.746838,0.140021,0.179996}%
\pgfsetfillcolor{currentfill}%
\pgfsetlinewidth{0.000000pt}%
\definecolor{currentstroke}{rgb}{0.000000,0.000000,0.000000}%
\pgfsetstrokecolor{currentstroke}%
\pgfsetdash{}{0pt}%
\pgfpathmoveto{\pgfqpoint{4.018512in}{2.470906in}}%
\pgfpathlineto{\pgfqpoint{4.268228in}{2.632966in}}%
\pgfpathlineto{\pgfqpoint{4.018512in}{2.470906in}}%
\pgfpathclose%
\pgfusepath{fill}%
\end{pgfscope}%
\begin{pgfscope}%
\pgfpathrectangle{\pgfqpoint{0.017500in}{0.017500in}}{\pgfqpoint{6.218282in}{3.580372in}} %
\pgfusepath{clip}%
\pgfsetbuttcap%
\pgfsetroundjoin%
\definecolor{currentfill}{rgb}{0.746838,0.140021,0.179996}%
\pgfsetfillcolor{currentfill}%
\pgfsetlinewidth{0.000000pt}%
\definecolor{currentstroke}{rgb}{0.000000,0.000000,0.000000}%
\pgfsetstrokecolor{currentstroke}%
\pgfsetdash{}{0pt}%
\pgfpathmoveto{\pgfqpoint{4.018512in}{2.470906in}}%
\pgfpathlineto{\pgfqpoint{4.268228in}{2.632966in}}%
\pgfpathlineto{\pgfqpoint{4.018512in}{2.470906in}}%
\pgfpathclose%
\pgfusepath{fill}%
\end{pgfscope}%
\begin{pgfscope}%
\pgfpathrectangle{\pgfqpoint{0.017500in}{0.017500in}}{\pgfqpoint{6.218282in}{3.580372in}} %
\pgfusepath{clip}%
\pgfsetbuttcap%
\pgfsetroundjoin%
\definecolor{currentfill}{rgb}{0.746838,0.140021,0.179996}%
\pgfsetfillcolor{currentfill}%
\pgfsetlinewidth{0.000000pt}%
\definecolor{currentstroke}{rgb}{0.000000,0.000000,0.000000}%
\pgfsetstrokecolor{currentstroke}%
\pgfsetdash{}{0pt}%
\pgfpathmoveto{\pgfqpoint{4.018512in}{2.470906in}}%
\pgfpathlineto{\pgfqpoint{4.268228in}{2.632966in}}%
\pgfpathlineto{\pgfqpoint{4.018512in}{2.470906in}}%
\pgfpathclose%
\pgfusepath{fill}%
\end{pgfscope}%
\begin{pgfscope}%
\pgfpathrectangle{\pgfqpoint{0.017500in}{0.017500in}}{\pgfqpoint{6.218282in}{3.580372in}} %
\pgfusepath{clip}%
\pgfsetbuttcap%
\pgfsetroundjoin%
\definecolor{currentfill}{rgb}{0.746838,0.140021,0.179996}%
\pgfsetfillcolor{currentfill}%
\pgfsetlinewidth{0.000000pt}%
\definecolor{currentstroke}{rgb}{0.000000,0.000000,0.000000}%
\pgfsetstrokecolor{currentstroke}%
\pgfsetdash{}{0pt}%
\pgfpathmoveto{\pgfqpoint{4.018512in}{2.470906in}}%
\pgfpathlineto{\pgfqpoint{4.268228in}{2.632966in}}%
\pgfpathlineto{\pgfqpoint{4.018512in}{2.470906in}}%
\pgfpathclose%
\pgfusepath{fill}%
\end{pgfscope}%
\begin{pgfscope}%
\pgfpathrectangle{\pgfqpoint{0.017500in}{0.017500in}}{\pgfqpoint{6.218282in}{3.580372in}} %
\pgfusepath{clip}%
\pgfsetbuttcap%
\pgfsetroundjoin%
\definecolor{currentfill}{rgb}{0.873402,0.386960,0.306332}%
\pgfsetfillcolor{currentfill}%
\pgfsetlinewidth{0.000000pt}%
\definecolor{currentstroke}{rgb}{0.000000,0.000000,0.000000}%
\pgfsetstrokecolor{currentstroke}%
\pgfsetdash{}{0pt}%
\pgfpathmoveto{\pgfqpoint{2.142608in}{2.314596in}}%
\pgfpathlineto{\pgfqpoint{2.410172in}{2.537974in}}%
\pgfpathlineto{\pgfqpoint{2.142608in}{2.314596in}}%
\pgfpathclose%
\pgfusepath{fill}%
\end{pgfscope}%
\begin{pgfscope}%
\pgfpathrectangle{\pgfqpoint{0.017500in}{0.017500in}}{\pgfqpoint{6.218282in}{3.580372in}} %
\pgfusepath{clip}%
\pgfsetbuttcap%
\pgfsetroundjoin%
\definecolor{currentfill}{rgb}{0.873402,0.386960,0.306332}%
\pgfsetfillcolor{currentfill}%
\pgfsetlinewidth{0.000000pt}%
\definecolor{currentstroke}{rgb}{0.000000,0.000000,0.000000}%
\pgfsetstrokecolor{currentstroke}%
\pgfsetdash{}{0pt}%
\pgfpathmoveto{\pgfqpoint{2.142608in}{2.314596in}}%
\pgfpathlineto{\pgfqpoint{2.410172in}{2.537974in}}%
\pgfpathlineto{\pgfqpoint{2.142608in}{2.314596in}}%
\pgfpathclose%
\pgfusepath{fill}%
\end{pgfscope}%
\begin{pgfscope}%
\pgfpathrectangle{\pgfqpoint{0.017500in}{0.017500in}}{\pgfqpoint{6.218282in}{3.580372in}} %
\pgfusepath{clip}%
\pgfsetbuttcap%
\pgfsetroundjoin%
\definecolor{currentfill}{rgb}{0.873402,0.386960,0.306332}%
\pgfsetfillcolor{currentfill}%
\pgfsetlinewidth{0.000000pt}%
\definecolor{currentstroke}{rgb}{0.000000,0.000000,0.000000}%
\pgfsetstrokecolor{currentstroke}%
\pgfsetdash{}{0pt}%
\pgfpathmoveto{\pgfqpoint{2.142608in}{2.314596in}}%
\pgfpathlineto{\pgfqpoint{2.410172in}{2.537974in}}%
\pgfpathlineto{\pgfqpoint{2.142608in}{2.314596in}}%
\pgfpathclose%
\pgfusepath{fill}%
\end{pgfscope}%
\begin{pgfscope}%
\pgfpathrectangle{\pgfqpoint{0.017500in}{0.017500in}}{\pgfqpoint{6.218282in}{3.580372in}} %
\pgfusepath{clip}%
\pgfsetbuttcap%
\pgfsetroundjoin%
\definecolor{currentfill}{rgb}{0.873402,0.386960,0.306332}%
\pgfsetfillcolor{currentfill}%
\pgfsetlinewidth{0.000000pt}%
\definecolor{currentstroke}{rgb}{0.000000,0.000000,0.000000}%
\pgfsetstrokecolor{currentstroke}%
\pgfsetdash{}{0pt}%
\pgfpathmoveto{\pgfqpoint{2.142608in}{2.314596in}}%
\pgfpathlineto{\pgfqpoint{2.410172in}{2.537974in}}%
\pgfpathlineto{\pgfqpoint{2.142608in}{2.314596in}}%
\pgfpathclose%
\pgfusepath{fill}%
\end{pgfscope}%
\begin{pgfscope}%
\pgfpathrectangle{\pgfqpoint{0.017500in}{0.017500in}}{\pgfqpoint{6.218282in}{3.580372in}} %
\pgfusepath{clip}%
\pgfsetbuttcap%
\pgfsetroundjoin%
\definecolor{currentfill}{rgb}{0.873402,0.386960,0.306332}%
\pgfsetfillcolor{currentfill}%
\pgfsetlinewidth{0.000000pt}%
\definecolor{currentstroke}{rgb}{0.000000,0.000000,0.000000}%
\pgfsetstrokecolor{currentstroke}%
\pgfsetdash{}{0pt}%
\pgfpathmoveto{\pgfqpoint{2.142608in}{2.314596in}}%
\pgfpathlineto{\pgfqpoint{2.410172in}{2.537974in}}%
\pgfpathlineto{\pgfqpoint{2.142608in}{2.314596in}}%
\pgfpathclose%
\pgfusepath{fill}%
\end{pgfscope}%
\begin{pgfscope}%
\pgfpathrectangle{\pgfqpoint{0.017500in}{0.017500in}}{\pgfqpoint{6.218282in}{3.580372in}} %
\pgfusepath{clip}%
\pgfsetbuttcap%
\pgfsetroundjoin%
\definecolor{currentfill}{rgb}{0.873402,0.386960,0.306332}%
\pgfsetfillcolor{currentfill}%
\pgfsetlinewidth{0.000000pt}%
\definecolor{currentstroke}{rgb}{0.000000,0.000000,0.000000}%
\pgfsetstrokecolor{currentstroke}%
\pgfsetdash{}{0pt}%
\pgfpathmoveto{\pgfqpoint{2.142608in}{2.314596in}}%
\pgfpathlineto{\pgfqpoint{2.410172in}{2.537974in}}%
\pgfpathlineto{\pgfqpoint{2.142608in}{2.314596in}}%
\pgfpathclose%
\pgfusepath{fill}%
\end{pgfscope}%
\begin{pgfscope}%
\pgfpathrectangle{\pgfqpoint{0.017500in}{0.017500in}}{\pgfqpoint{6.218282in}{3.580372in}} %
\pgfusepath{clip}%
\pgfsetbuttcap%
\pgfsetroundjoin%
\definecolor{currentfill}{rgb}{0.873402,0.386960,0.306332}%
\pgfsetfillcolor{currentfill}%
\pgfsetlinewidth{0.000000pt}%
\definecolor{currentstroke}{rgb}{0.000000,0.000000,0.000000}%
\pgfsetstrokecolor{currentstroke}%
\pgfsetdash{}{0pt}%
\pgfpathmoveto{\pgfqpoint{2.142608in}{2.314596in}}%
\pgfpathlineto{\pgfqpoint{2.410172in}{2.537974in}}%
\pgfpathlineto{\pgfqpoint{2.142608in}{2.314596in}}%
\pgfpathclose%
\pgfusepath{fill}%
\end{pgfscope}%
\begin{pgfscope}%
\pgfpathrectangle{\pgfqpoint{0.017500in}{0.017500in}}{\pgfqpoint{6.218282in}{3.580372in}} %
\pgfusepath{clip}%
\pgfsetbuttcap%
\pgfsetroundjoin%
\definecolor{currentfill}{rgb}{0.873402,0.386960,0.306332}%
\pgfsetfillcolor{currentfill}%
\pgfsetlinewidth{0.000000pt}%
\definecolor{currentstroke}{rgb}{0.000000,0.000000,0.000000}%
\pgfsetstrokecolor{currentstroke}%
\pgfsetdash{}{0pt}%
\pgfpathmoveto{\pgfqpoint{2.142608in}{2.314596in}}%
\pgfpathlineto{\pgfqpoint{2.410172in}{2.537974in}}%
\pgfpathlineto{\pgfqpoint{2.142608in}{2.314596in}}%
\pgfpathclose%
\pgfusepath{fill}%
\end{pgfscope}%
\begin{pgfscope}%
\pgfpathrectangle{\pgfqpoint{0.017500in}{0.017500in}}{\pgfqpoint{6.218282in}{3.580372in}} %
\pgfusepath{clip}%
\pgfsetbuttcap%
\pgfsetroundjoin%
\definecolor{currentfill}{rgb}{0.873402,0.386960,0.306332}%
\pgfsetfillcolor{currentfill}%
\pgfsetlinewidth{0.000000pt}%
\definecolor{currentstroke}{rgb}{0.000000,0.000000,0.000000}%
\pgfsetstrokecolor{currentstroke}%
\pgfsetdash{}{0pt}%
\pgfpathmoveto{\pgfqpoint{2.142608in}{2.314596in}}%
\pgfpathlineto{\pgfqpoint{2.410172in}{2.537974in}}%
\pgfpathlineto{\pgfqpoint{2.142608in}{2.314596in}}%
\pgfpathclose%
\pgfusepath{fill}%
\end{pgfscope}%
\begin{pgfscope}%
\pgfpathrectangle{\pgfqpoint{0.017500in}{0.017500in}}{\pgfqpoint{6.218282in}{3.580372in}} %
\pgfusepath{clip}%
\pgfsetbuttcap%
\pgfsetroundjoin%
\definecolor{currentfill}{rgb}{0.873402,0.386960,0.306332}%
\pgfsetfillcolor{currentfill}%
\pgfsetlinewidth{0.000000pt}%
\definecolor{currentstroke}{rgb}{0.000000,0.000000,0.000000}%
\pgfsetstrokecolor{currentstroke}%
\pgfsetdash{}{0pt}%
\pgfpathmoveto{\pgfqpoint{2.142608in}{2.314596in}}%
\pgfpathlineto{\pgfqpoint{2.410172in}{2.537974in}}%
\pgfpathlineto{\pgfqpoint{2.142608in}{2.314596in}}%
\pgfpathclose%
\pgfusepath{fill}%
\end{pgfscope}%
\begin{pgfscope}%
\pgfpathrectangle{\pgfqpoint{0.017500in}{0.017500in}}{\pgfqpoint{6.218282in}{3.580372in}} %
\pgfusepath{clip}%
\pgfsetbuttcap%
\pgfsetroundjoin%
\definecolor{currentfill}{rgb}{0.873402,0.386960,0.306332}%
\pgfsetfillcolor{currentfill}%
\pgfsetlinewidth{0.000000pt}%
\definecolor{currentstroke}{rgb}{0.000000,0.000000,0.000000}%
\pgfsetstrokecolor{currentstroke}%
\pgfsetdash{}{0pt}%
\pgfpathmoveto{\pgfqpoint{2.142608in}{2.314596in}}%
\pgfpathlineto{\pgfqpoint{2.410172in}{2.537974in}}%
\pgfpathlineto{\pgfqpoint{2.142608in}{2.314596in}}%
\pgfpathclose%
\pgfusepath{fill}%
\end{pgfscope}%
\begin{pgfscope}%
\pgfpathrectangle{\pgfqpoint{0.017500in}{0.017500in}}{\pgfqpoint{6.218282in}{3.580372in}} %
\pgfusepath{clip}%
\pgfsetbuttcap%
\pgfsetroundjoin%
\definecolor{currentfill}{rgb}{0.873402,0.386960,0.306332}%
\pgfsetfillcolor{currentfill}%
\pgfsetlinewidth{0.000000pt}%
\definecolor{currentstroke}{rgb}{0.000000,0.000000,0.000000}%
\pgfsetstrokecolor{currentstroke}%
\pgfsetdash{}{0pt}%
\pgfpathmoveto{\pgfqpoint{2.142608in}{2.314596in}}%
\pgfpathlineto{\pgfqpoint{2.410172in}{2.537974in}}%
\pgfpathlineto{\pgfqpoint{2.142608in}{2.314596in}}%
\pgfpathclose%
\pgfusepath{fill}%
\end{pgfscope}%
\begin{pgfscope}%
\pgfpathrectangle{\pgfqpoint{0.017500in}{0.017500in}}{\pgfqpoint{6.218282in}{3.580372in}} %
\pgfusepath{clip}%
\pgfsetbuttcap%
\pgfsetroundjoin%
\definecolor{currentfill}{rgb}{0.873402,0.386960,0.306332}%
\pgfsetfillcolor{currentfill}%
\pgfsetlinewidth{0.000000pt}%
\definecolor{currentstroke}{rgb}{0.000000,0.000000,0.000000}%
\pgfsetstrokecolor{currentstroke}%
\pgfsetdash{}{0pt}%
\pgfpathmoveto{\pgfqpoint{2.142608in}{2.314596in}}%
\pgfpathlineto{\pgfqpoint{2.410172in}{2.537974in}}%
\pgfpathlineto{\pgfqpoint{2.142608in}{2.314596in}}%
\pgfpathclose%
\pgfusepath{fill}%
\end{pgfscope}%
\begin{pgfscope}%
\pgfpathrectangle{\pgfqpoint{0.017500in}{0.017500in}}{\pgfqpoint{6.218282in}{3.580372in}} %
\pgfusepath{clip}%
\pgfsetbuttcap%
\pgfsetroundjoin%
\definecolor{currentfill}{rgb}{0.873402,0.386960,0.306332}%
\pgfsetfillcolor{currentfill}%
\pgfsetlinewidth{0.000000pt}%
\definecolor{currentstroke}{rgb}{0.000000,0.000000,0.000000}%
\pgfsetstrokecolor{currentstroke}%
\pgfsetdash{}{0pt}%
\pgfpathmoveto{\pgfqpoint{2.142608in}{2.314596in}}%
\pgfpathlineto{\pgfqpoint{2.410172in}{2.537974in}}%
\pgfpathlineto{\pgfqpoint{2.142608in}{2.314596in}}%
\pgfpathclose%
\pgfusepath{fill}%
\end{pgfscope}%
\begin{pgfscope}%
\pgfpathrectangle{\pgfqpoint{0.017500in}{0.017500in}}{\pgfqpoint{6.218282in}{3.580372in}} %
\pgfusepath{clip}%
\pgfsetbuttcap%
\pgfsetroundjoin%
\definecolor{currentfill}{rgb}{0.873402,0.386960,0.306332}%
\pgfsetfillcolor{currentfill}%
\pgfsetlinewidth{0.000000pt}%
\definecolor{currentstroke}{rgb}{0.000000,0.000000,0.000000}%
\pgfsetstrokecolor{currentstroke}%
\pgfsetdash{}{0pt}%
\pgfpathmoveto{\pgfqpoint{2.142608in}{2.314596in}}%
\pgfpathlineto{\pgfqpoint{2.410172in}{2.537974in}}%
\pgfpathlineto{\pgfqpoint{2.142608in}{2.314596in}}%
\pgfpathclose%
\pgfusepath{fill}%
\end{pgfscope}%
\begin{pgfscope}%
\pgfpathrectangle{\pgfqpoint{0.017500in}{0.017500in}}{\pgfqpoint{6.218282in}{3.580372in}} %
\pgfusepath{clip}%
\pgfsetbuttcap%
\pgfsetroundjoin%
\definecolor{currentfill}{rgb}{0.873402,0.386960,0.306332}%
\pgfsetfillcolor{currentfill}%
\pgfsetlinewidth{0.000000pt}%
\definecolor{currentstroke}{rgb}{0.000000,0.000000,0.000000}%
\pgfsetstrokecolor{currentstroke}%
\pgfsetdash{}{0pt}%
\pgfpathmoveto{\pgfqpoint{2.142608in}{2.314596in}}%
\pgfpathlineto{\pgfqpoint{2.410172in}{2.537974in}}%
\pgfpathlineto{\pgfqpoint{2.142608in}{2.314596in}}%
\pgfpathclose%
\pgfusepath{fill}%
\end{pgfscope}%
\begin{pgfscope}%
\pgfpathrectangle{\pgfqpoint{0.017500in}{0.017500in}}{\pgfqpoint{6.218282in}{3.580372in}} %
\pgfusepath{clip}%
\pgfsetbuttcap%
\pgfsetroundjoin%
\definecolor{currentfill}{rgb}{0.873402,0.386960,0.306332}%
\pgfsetfillcolor{currentfill}%
\pgfsetlinewidth{0.000000pt}%
\definecolor{currentstroke}{rgb}{0.000000,0.000000,0.000000}%
\pgfsetstrokecolor{currentstroke}%
\pgfsetdash{}{0pt}%
\pgfpathmoveto{\pgfqpoint{2.142608in}{2.314596in}}%
\pgfpathlineto{\pgfqpoint{2.410172in}{2.537974in}}%
\pgfpathlineto{\pgfqpoint{2.142608in}{2.314596in}}%
\pgfpathclose%
\pgfusepath{fill}%
\end{pgfscope}%
\begin{pgfscope}%
\pgfpathrectangle{\pgfqpoint{0.017500in}{0.017500in}}{\pgfqpoint{6.218282in}{3.580372in}} %
\pgfusepath{clip}%
\pgfsetbuttcap%
\pgfsetroundjoin%
\definecolor{currentfill}{rgb}{0.873402,0.386960,0.306332}%
\pgfsetfillcolor{currentfill}%
\pgfsetlinewidth{0.000000pt}%
\definecolor{currentstroke}{rgb}{0.000000,0.000000,0.000000}%
\pgfsetstrokecolor{currentstroke}%
\pgfsetdash{}{0pt}%
\pgfpathmoveto{\pgfqpoint{2.142608in}{2.314596in}}%
\pgfpathlineto{\pgfqpoint{2.410172in}{2.537974in}}%
\pgfpathlineto{\pgfqpoint{2.142608in}{2.314596in}}%
\pgfpathclose%
\pgfusepath{fill}%
\end{pgfscope}%
\begin{pgfscope}%
\pgfpathrectangle{\pgfqpoint{0.017500in}{0.017500in}}{\pgfqpoint{6.218282in}{3.580372in}} %
\pgfusepath{clip}%
\pgfsetbuttcap%
\pgfsetroundjoin%
\definecolor{currentfill}{rgb}{0.873402,0.386960,0.306332}%
\pgfsetfillcolor{currentfill}%
\pgfsetlinewidth{0.000000pt}%
\definecolor{currentstroke}{rgb}{0.000000,0.000000,0.000000}%
\pgfsetstrokecolor{currentstroke}%
\pgfsetdash{}{0pt}%
\pgfpathmoveto{\pgfqpoint{2.142608in}{2.314596in}}%
\pgfpathlineto{\pgfqpoint{2.410172in}{2.537974in}}%
\pgfpathlineto{\pgfqpoint{2.142608in}{2.314596in}}%
\pgfpathclose%
\pgfusepath{fill}%
\end{pgfscope}%
\begin{pgfscope}%
\pgfpathrectangle{\pgfqpoint{0.017500in}{0.017500in}}{\pgfqpoint{6.218282in}{3.580372in}} %
\pgfusepath{clip}%
\pgfsetbuttcap%
\pgfsetroundjoin%
\definecolor{currentfill}{rgb}{0.873402,0.386960,0.306332}%
\pgfsetfillcolor{currentfill}%
\pgfsetlinewidth{0.000000pt}%
\definecolor{currentstroke}{rgb}{0.000000,0.000000,0.000000}%
\pgfsetstrokecolor{currentstroke}%
\pgfsetdash{}{0pt}%
\pgfpathmoveto{\pgfqpoint{2.142608in}{2.314596in}}%
\pgfpathlineto{\pgfqpoint{2.410172in}{2.537974in}}%
\pgfpathlineto{\pgfqpoint{2.142608in}{2.314596in}}%
\pgfpathclose%
\pgfusepath{fill}%
\end{pgfscope}%
\begin{pgfscope}%
\pgfpathrectangle{\pgfqpoint{0.017500in}{0.017500in}}{\pgfqpoint{6.218282in}{3.580372in}} %
\pgfusepath{clip}%
\pgfsetbuttcap%
\pgfsetroundjoin%
\definecolor{currentfill}{rgb}{0.873402,0.386960,0.306332}%
\pgfsetfillcolor{currentfill}%
\pgfsetlinewidth{0.000000pt}%
\definecolor{currentstroke}{rgb}{0.000000,0.000000,0.000000}%
\pgfsetstrokecolor{currentstroke}%
\pgfsetdash{}{0pt}%
\pgfpathmoveto{\pgfqpoint{2.142608in}{2.314596in}}%
\pgfpathlineto{\pgfqpoint{2.410172in}{2.537974in}}%
\pgfpathlineto{\pgfqpoint{2.142608in}{2.314596in}}%
\pgfpathclose%
\pgfusepath{fill}%
\end{pgfscope}%
\begin{pgfscope}%
\pgfpathrectangle{\pgfqpoint{0.017500in}{0.017500in}}{\pgfqpoint{6.218282in}{3.580372in}} %
\pgfusepath{clip}%
\pgfsetbuttcap%
\pgfsetroundjoin%
\definecolor{currentfill}{rgb}{0.873402,0.386960,0.306332}%
\pgfsetfillcolor{currentfill}%
\pgfsetlinewidth{0.000000pt}%
\definecolor{currentstroke}{rgb}{0.000000,0.000000,0.000000}%
\pgfsetstrokecolor{currentstroke}%
\pgfsetdash{}{0pt}%
\pgfpathmoveto{\pgfqpoint{2.142608in}{2.314596in}}%
\pgfpathlineto{\pgfqpoint{2.410172in}{2.537974in}}%
\pgfpathlineto{\pgfqpoint{2.142608in}{2.314596in}}%
\pgfpathclose%
\pgfusepath{fill}%
\end{pgfscope}%
\begin{pgfscope}%
\pgfpathrectangle{\pgfqpoint{0.017500in}{0.017500in}}{\pgfqpoint{6.218282in}{3.580372in}} %
\pgfusepath{clip}%
\pgfsetbuttcap%
\pgfsetroundjoin%
\definecolor{currentfill}{rgb}{0.873402,0.386960,0.306332}%
\pgfsetfillcolor{currentfill}%
\pgfsetlinewidth{0.000000pt}%
\definecolor{currentstroke}{rgb}{0.000000,0.000000,0.000000}%
\pgfsetstrokecolor{currentstroke}%
\pgfsetdash{}{0pt}%
\pgfpathmoveto{\pgfqpoint{2.142608in}{2.314596in}}%
\pgfpathlineto{\pgfqpoint{2.410172in}{2.537974in}}%
\pgfpathlineto{\pgfqpoint{2.142608in}{2.314596in}}%
\pgfpathclose%
\pgfusepath{fill}%
\end{pgfscope}%
\begin{pgfscope}%
\pgfpathrectangle{\pgfqpoint{0.017500in}{0.017500in}}{\pgfqpoint{6.218282in}{3.580372in}} %
\pgfusepath{clip}%
\pgfsetbuttcap%
\pgfsetroundjoin%
\definecolor{currentfill}{rgb}{0.873402,0.386960,0.306332}%
\pgfsetfillcolor{currentfill}%
\pgfsetlinewidth{0.000000pt}%
\definecolor{currentstroke}{rgb}{0.000000,0.000000,0.000000}%
\pgfsetstrokecolor{currentstroke}%
\pgfsetdash{}{0pt}%
\pgfpathmoveto{\pgfqpoint{2.142608in}{2.314596in}}%
\pgfpathlineto{\pgfqpoint{2.410172in}{2.537974in}}%
\pgfpathlineto{\pgfqpoint{2.142608in}{2.314596in}}%
\pgfpathclose%
\pgfusepath{fill}%
\end{pgfscope}%
\begin{pgfscope}%
\pgfpathrectangle{\pgfqpoint{0.017500in}{0.017500in}}{\pgfqpoint{6.218282in}{3.580372in}} %
\pgfusepath{clip}%
\pgfsetbuttcap%
\pgfsetroundjoin%
\definecolor{currentfill}{rgb}{0.873402,0.386960,0.306332}%
\pgfsetfillcolor{currentfill}%
\pgfsetlinewidth{0.000000pt}%
\definecolor{currentstroke}{rgb}{0.000000,0.000000,0.000000}%
\pgfsetstrokecolor{currentstroke}%
\pgfsetdash{}{0pt}%
\pgfpathmoveto{\pgfqpoint{2.142608in}{2.314596in}}%
\pgfpathlineto{\pgfqpoint{2.410172in}{2.537974in}}%
\pgfpathlineto{\pgfqpoint{2.142608in}{2.314596in}}%
\pgfpathclose%
\pgfusepath{fill}%
\end{pgfscope}%
\begin{pgfscope}%
\pgfpathrectangle{\pgfqpoint{0.017500in}{0.017500in}}{\pgfqpoint{6.218282in}{3.580372in}} %
\pgfusepath{clip}%
\pgfsetbuttcap%
\pgfsetroundjoin%
\definecolor{currentfill}{rgb}{0.873402,0.386960,0.306332}%
\pgfsetfillcolor{currentfill}%
\pgfsetlinewidth{0.000000pt}%
\definecolor{currentstroke}{rgb}{0.000000,0.000000,0.000000}%
\pgfsetstrokecolor{currentstroke}%
\pgfsetdash{}{0pt}%
\pgfpathmoveto{\pgfqpoint{2.142608in}{2.314596in}}%
\pgfpathlineto{\pgfqpoint{2.410172in}{2.537974in}}%
\pgfpathlineto{\pgfqpoint{2.142608in}{2.314596in}}%
\pgfpathclose%
\pgfusepath{fill}%
\end{pgfscope}%
\begin{pgfscope}%
\pgfpathrectangle{\pgfqpoint{0.017500in}{0.017500in}}{\pgfqpoint{6.218282in}{3.580372in}} %
\pgfusepath{clip}%
\pgfsetbuttcap%
\pgfsetroundjoin%
\definecolor{currentfill}{rgb}{0.873402,0.386960,0.306332}%
\pgfsetfillcolor{currentfill}%
\pgfsetlinewidth{0.000000pt}%
\definecolor{currentstroke}{rgb}{0.000000,0.000000,0.000000}%
\pgfsetstrokecolor{currentstroke}%
\pgfsetdash{}{0pt}%
\pgfpathmoveto{\pgfqpoint{2.142608in}{2.314596in}}%
\pgfpathlineto{\pgfqpoint{2.410172in}{2.537974in}}%
\pgfpathlineto{\pgfqpoint{2.142608in}{2.314596in}}%
\pgfpathclose%
\pgfusepath{fill}%
\end{pgfscope}%
\begin{pgfscope}%
\pgfpathrectangle{\pgfqpoint{0.017500in}{0.017500in}}{\pgfqpoint{6.218282in}{3.580372in}} %
\pgfusepath{clip}%
\pgfsetbuttcap%
\pgfsetroundjoin%
\definecolor{currentfill}{rgb}{0.873402,0.386960,0.306332}%
\pgfsetfillcolor{currentfill}%
\pgfsetlinewidth{0.000000pt}%
\definecolor{currentstroke}{rgb}{0.000000,0.000000,0.000000}%
\pgfsetstrokecolor{currentstroke}%
\pgfsetdash{}{0pt}%
\pgfpathmoveto{\pgfqpoint{2.142608in}{2.314596in}}%
\pgfpathlineto{\pgfqpoint{2.410172in}{2.537974in}}%
\pgfpathlineto{\pgfqpoint{2.142608in}{2.314596in}}%
\pgfpathclose%
\pgfusepath{fill}%
\end{pgfscope}%
\begin{pgfscope}%
\pgfpathrectangle{\pgfqpoint{0.017500in}{0.017500in}}{\pgfqpoint{6.218282in}{3.580372in}} %
\pgfusepath{clip}%
\pgfsetbuttcap%
\pgfsetroundjoin%
\definecolor{currentfill}{rgb}{0.873402,0.386960,0.306332}%
\pgfsetfillcolor{currentfill}%
\pgfsetlinewidth{0.000000pt}%
\definecolor{currentstroke}{rgb}{0.000000,0.000000,0.000000}%
\pgfsetstrokecolor{currentstroke}%
\pgfsetdash{}{0pt}%
\pgfpathmoveto{\pgfqpoint{2.142608in}{2.314596in}}%
\pgfpathlineto{\pgfqpoint{2.410172in}{2.537974in}}%
\pgfpathlineto{\pgfqpoint{2.142608in}{2.314596in}}%
\pgfpathclose%
\pgfusepath{fill}%
\end{pgfscope}%
\begin{pgfscope}%
\pgfpathrectangle{\pgfqpoint{0.017500in}{0.017500in}}{\pgfqpoint{6.218282in}{3.580372in}} %
\pgfusepath{clip}%
\pgfsetbuttcap%
\pgfsetroundjoin%
\definecolor{currentfill}{rgb}{0.873402,0.386960,0.306332}%
\pgfsetfillcolor{currentfill}%
\pgfsetlinewidth{0.000000pt}%
\definecolor{currentstroke}{rgb}{0.000000,0.000000,0.000000}%
\pgfsetstrokecolor{currentstroke}%
\pgfsetdash{}{0pt}%
\pgfpathmoveto{\pgfqpoint{2.142608in}{2.314596in}}%
\pgfpathlineto{\pgfqpoint{2.410172in}{2.537974in}}%
\pgfpathlineto{\pgfqpoint{2.142608in}{2.314596in}}%
\pgfpathclose%
\pgfusepath{fill}%
\end{pgfscope}%
\begin{pgfscope}%
\pgfpathrectangle{\pgfqpoint{0.017500in}{0.017500in}}{\pgfqpoint{6.218282in}{3.580372in}} %
\pgfusepath{clip}%
\pgfsetbuttcap%
\pgfsetroundjoin%
\definecolor{currentfill}{rgb}{0.746838,0.140021,0.179996}%
\pgfsetfillcolor{currentfill}%
\pgfsetlinewidth{0.000000pt}%
\definecolor{currentstroke}{rgb}{0.000000,0.000000,0.000000}%
\pgfsetstrokecolor{currentstroke}%
\pgfsetdash{}{0pt}%
\pgfpathmoveto{\pgfqpoint{4.018512in}{2.470906in}}%
\pgfpathlineto{\pgfqpoint{4.478797in}{2.402809in}}%
\pgfpathlineto{\pgfqpoint{4.724973in}{2.556562in}}%
\pgfpathlineto{\pgfqpoint{4.268228in}{2.632966in}}%
\pgfpathlineto{\pgfqpoint{4.018512in}{2.470906in}}%
\pgfpathclose%
\pgfusepath{fill}%
\end{pgfscope}%
\begin{pgfscope}%
\pgfpathrectangle{\pgfqpoint{0.017500in}{0.017500in}}{\pgfqpoint{6.218282in}{3.580372in}} %
\pgfusepath{clip}%
\pgfsetbuttcap%
\pgfsetroundjoin%
\definecolor{currentfill}{rgb}{0.746838,0.140021,0.179996}%
\pgfsetfillcolor{currentfill}%
\pgfsetlinewidth{0.000000pt}%
\definecolor{currentstroke}{rgb}{0.000000,0.000000,0.000000}%
\pgfsetstrokecolor{currentstroke}%
\pgfsetdash{}{0pt}%
\pgfpathmoveto{\pgfqpoint{4.018512in}{2.470906in}}%
\pgfpathlineto{\pgfqpoint{4.478797in}{2.402809in}}%
\pgfpathlineto{\pgfqpoint{4.724973in}{2.556562in}}%
\pgfpathlineto{\pgfqpoint{4.268228in}{2.632966in}}%
\pgfpathlineto{\pgfqpoint{4.018512in}{2.470906in}}%
\pgfpathclose%
\pgfusepath{fill}%
\end{pgfscope}%
\begin{pgfscope}%
\pgfpathrectangle{\pgfqpoint{0.017500in}{0.017500in}}{\pgfqpoint{6.218282in}{3.580372in}} %
\pgfusepath{clip}%
\pgfsetbuttcap%
\pgfsetroundjoin%
\definecolor{currentfill}{rgb}{0.746838,0.140021,0.179996}%
\pgfsetfillcolor{currentfill}%
\pgfsetlinewidth{0.000000pt}%
\definecolor{currentstroke}{rgb}{0.000000,0.000000,0.000000}%
\pgfsetstrokecolor{currentstroke}%
\pgfsetdash{}{0pt}%
\pgfpathmoveto{\pgfqpoint{4.018512in}{2.470906in}}%
\pgfpathlineto{\pgfqpoint{4.478797in}{2.402809in}}%
\pgfpathlineto{\pgfqpoint{4.724973in}{2.556562in}}%
\pgfpathlineto{\pgfqpoint{4.268228in}{2.632966in}}%
\pgfpathlineto{\pgfqpoint{4.018512in}{2.470906in}}%
\pgfpathclose%
\pgfusepath{fill}%
\end{pgfscope}%
\begin{pgfscope}%
\pgfpathrectangle{\pgfqpoint{0.017500in}{0.017500in}}{\pgfqpoint{6.218282in}{3.580372in}} %
\pgfusepath{clip}%
\pgfsetbuttcap%
\pgfsetroundjoin%
\definecolor{currentfill}{rgb}{0.746838,0.140021,0.179996}%
\pgfsetfillcolor{currentfill}%
\pgfsetlinewidth{0.000000pt}%
\definecolor{currentstroke}{rgb}{0.000000,0.000000,0.000000}%
\pgfsetstrokecolor{currentstroke}%
\pgfsetdash{}{0pt}%
\pgfpathmoveto{\pgfqpoint{4.018512in}{2.470906in}}%
\pgfpathlineto{\pgfqpoint{4.478797in}{2.402809in}}%
\pgfpathlineto{\pgfqpoint{4.724973in}{2.556562in}}%
\pgfpathlineto{\pgfqpoint{4.268228in}{2.632966in}}%
\pgfpathlineto{\pgfqpoint{4.018512in}{2.470906in}}%
\pgfpathclose%
\pgfusepath{fill}%
\end{pgfscope}%
\begin{pgfscope}%
\pgfpathrectangle{\pgfqpoint{0.017500in}{0.017500in}}{\pgfqpoint{6.218282in}{3.580372in}} %
\pgfusepath{clip}%
\pgfsetbuttcap%
\pgfsetroundjoin%
\definecolor{currentfill}{rgb}{0.746838,0.140021,0.179996}%
\pgfsetfillcolor{currentfill}%
\pgfsetlinewidth{0.000000pt}%
\definecolor{currentstroke}{rgb}{0.000000,0.000000,0.000000}%
\pgfsetstrokecolor{currentstroke}%
\pgfsetdash{}{0pt}%
\pgfpathmoveto{\pgfqpoint{4.018512in}{2.470906in}}%
\pgfpathlineto{\pgfqpoint{4.478797in}{2.402809in}}%
\pgfpathlineto{\pgfqpoint{4.724973in}{2.556562in}}%
\pgfpathlineto{\pgfqpoint{4.268228in}{2.632966in}}%
\pgfpathlineto{\pgfqpoint{4.018512in}{2.470906in}}%
\pgfpathclose%
\pgfusepath{fill}%
\end{pgfscope}%
\begin{pgfscope}%
\pgfpathrectangle{\pgfqpoint{0.017500in}{0.017500in}}{\pgfqpoint{6.218282in}{3.580372in}} %
\pgfusepath{clip}%
\pgfsetbuttcap%
\pgfsetroundjoin%
\definecolor{currentfill}{rgb}{0.746838,0.140021,0.179996}%
\pgfsetfillcolor{currentfill}%
\pgfsetlinewidth{0.000000pt}%
\definecolor{currentstroke}{rgb}{0.000000,0.000000,0.000000}%
\pgfsetstrokecolor{currentstroke}%
\pgfsetdash{}{0pt}%
\pgfpathmoveto{\pgfqpoint{4.018512in}{2.470906in}}%
\pgfpathlineto{\pgfqpoint{4.478797in}{2.402809in}}%
\pgfpathlineto{\pgfqpoint{4.724973in}{2.556562in}}%
\pgfpathlineto{\pgfqpoint{4.268228in}{2.632966in}}%
\pgfpathlineto{\pgfqpoint{4.018512in}{2.470906in}}%
\pgfpathclose%
\pgfusepath{fill}%
\end{pgfscope}%
\begin{pgfscope}%
\pgfpathrectangle{\pgfqpoint{0.017500in}{0.017500in}}{\pgfqpoint{6.218282in}{3.580372in}} %
\pgfusepath{clip}%
\pgfsetbuttcap%
\pgfsetroundjoin%
\definecolor{currentfill}{rgb}{0.869655,0.379274,0.300941}%
\pgfsetfillcolor{currentfill}%
\pgfsetlinewidth{0.000000pt}%
\definecolor{currentstroke}{rgb}{0.000000,0.000000,0.000000}%
\pgfsetstrokecolor{currentstroke}%
\pgfsetdash{}{0pt}%
\pgfpathmoveto{\pgfqpoint{2.142608in}{2.314596in}}%
\pgfpathlineto{\pgfqpoint{2.590029in}{2.232729in}}%
\pgfpathlineto{\pgfqpoint{2.855457in}{2.456853in}}%
\pgfpathlineto{\pgfqpoint{2.410172in}{2.537974in}}%
\pgfpathlineto{\pgfqpoint{2.142608in}{2.314596in}}%
\pgfpathclose%
\pgfusepath{fill}%
\end{pgfscope}%
\begin{pgfscope}%
\pgfpathrectangle{\pgfqpoint{0.017500in}{0.017500in}}{\pgfqpoint{6.218282in}{3.580372in}} %
\pgfusepath{clip}%
\pgfsetbuttcap%
\pgfsetroundjoin%
\definecolor{currentfill}{rgb}{0.869655,0.379274,0.300941}%
\pgfsetfillcolor{currentfill}%
\pgfsetlinewidth{0.000000pt}%
\definecolor{currentstroke}{rgb}{0.000000,0.000000,0.000000}%
\pgfsetstrokecolor{currentstroke}%
\pgfsetdash{}{0pt}%
\pgfpathmoveto{\pgfqpoint{2.142608in}{2.314596in}}%
\pgfpathlineto{\pgfqpoint{2.590029in}{2.232729in}}%
\pgfpathlineto{\pgfqpoint{2.855457in}{2.456853in}}%
\pgfpathlineto{\pgfqpoint{2.410172in}{2.537974in}}%
\pgfpathlineto{\pgfqpoint{2.142608in}{2.314596in}}%
\pgfpathclose%
\pgfusepath{fill}%
\end{pgfscope}%
\begin{pgfscope}%
\pgfpathrectangle{\pgfqpoint{0.017500in}{0.017500in}}{\pgfqpoint{6.218282in}{3.580372in}} %
\pgfusepath{clip}%
\pgfsetbuttcap%
\pgfsetroundjoin%
\definecolor{currentfill}{rgb}{0.869655,0.379274,0.300941}%
\pgfsetfillcolor{currentfill}%
\pgfsetlinewidth{0.000000pt}%
\definecolor{currentstroke}{rgb}{0.000000,0.000000,0.000000}%
\pgfsetstrokecolor{currentstroke}%
\pgfsetdash{}{0pt}%
\pgfpathmoveto{\pgfqpoint{2.142608in}{2.314596in}}%
\pgfpathlineto{\pgfqpoint{2.590029in}{2.232729in}}%
\pgfpathlineto{\pgfqpoint{2.855457in}{2.456853in}}%
\pgfpathlineto{\pgfqpoint{2.410172in}{2.537974in}}%
\pgfpathlineto{\pgfqpoint{2.142608in}{2.314596in}}%
\pgfpathclose%
\pgfusepath{fill}%
\end{pgfscope}%
\begin{pgfscope}%
\pgfpathrectangle{\pgfqpoint{0.017500in}{0.017500in}}{\pgfqpoint{6.218282in}{3.580372in}} %
\pgfusepath{clip}%
\pgfsetbuttcap%
\pgfsetroundjoin%
\definecolor{currentfill}{rgb}{0.869655,0.379274,0.300941}%
\pgfsetfillcolor{currentfill}%
\pgfsetlinewidth{0.000000pt}%
\definecolor{currentstroke}{rgb}{0.000000,0.000000,0.000000}%
\pgfsetstrokecolor{currentstroke}%
\pgfsetdash{}{0pt}%
\pgfpathmoveto{\pgfqpoint{2.142608in}{2.314596in}}%
\pgfpathlineto{\pgfqpoint{2.590029in}{2.232729in}}%
\pgfpathlineto{\pgfqpoint{2.855457in}{2.456853in}}%
\pgfpathlineto{\pgfqpoint{2.410172in}{2.537974in}}%
\pgfpathlineto{\pgfqpoint{2.142608in}{2.314596in}}%
\pgfpathclose%
\pgfusepath{fill}%
\end{pgfscope}%
\begin{pgfscope}%
\pgfpathrectangle{\pgfqpoint{0.017500in}{0.017500in}}{\pgfqpoint{6.218282in}{3.580372in}} %
\pgfusepath{clip}%
\pgfsetbuttcap%
\pgfsetroundjoin%
\definecolor{currentfill}{rgb}{0.869655,0.379274,0.300941}%
\pgfsetfillcolor{currentfill}%
\pgfsetlinewidth{0.000000pt}%
\definecolor{currentstroke}{rgb}{0.000000,0.000000,0.000000}%
\pgfsetstrokecolor{currentstroke}%
\pgfsetdash{}{0pt}%
\pgfpathmoveto{\pgfqpoint{2.142608in}{2.314596in}}%
\pgfpathlineto{\pgfqpoint{2.590029in}{2.232729in}}%
\pgfpathlineto{\pgfqpoint{2.855457in}{2.456853in}}%
\pgfpathlineto{\pgfqpoint{2.410172in}{2.537974in}}%
\pgfpathlineto{\pgfqpoint{2.142608in}{2.314596in}}%
\pgfpathclose%
\pgfusepath{fill}%
\end{pgfscope}%
\begin{pgfscope}%
\pgfpathrectangle{\pgfqpoint{0.017500in}{0.017500in}}{\pgfqpoint{6.218282in}{3.580372in}} %
\pgfusepath{clip}%
\pgfsetbuttcap%
\pgfsetroundjoin%
\definecolor{currentfill}{rgb}{0.869655,0.379274,0.300941}%
\pgfsetfillcolor{currentfill}%
\pgfsetlinewidth{0.000000pt}%
\definecolor{currentstroke}{rgb}{0.000000,0.000000,0.000000}%
\pgfsetstrokecolor{currentstroke}%
\pgfsetdash{}{0pt}%
\pgfpathmoveto{\pgfqpoint{2.142608in}{2.314596in}}%
\pgfpathlineto{\pgfqpoint{2.590029in}{2.232729in}}%
\pgfpathlineto{\pgfqpoint{2.855457in}{2.456853in}}%
\pgfpathlineto{\pgfqpoint{2.410172in}{2.537974in}}%
\pgfpathlineto{\pgfqpoint{2.142608in}{2.314596in}}%
\pgfpathclose%
\pgfusepath{fill}%
\end{pgfscope}%
\begin{pgfscope}%
\pgfpathrectangle{\pgfqpoint{0.017500in}{0.017500in}}{\pgfqpoint{6.218282in}{3.580372in}} %
\pgfusepath{clip}%
\pgfsetbuttcap%
\pgfsetroundjoin%
\definecolor{currentfill}{rgb}{0.353369,0.472069,0.892570}%
\pgfsetfillcolor{currentfill}%
\pgfsetlinewidth{0.000000pt}%
\definecolor{currentstroke}{rgb}{0.000000,0.000000,0.000000}%
\pgfsetstrokecolor{currentstroke}%
\pgfsetdash{}{0pt}%
\pgfpathmoveto{\pgfqpoint{2.789661in}{0.820654in}}%
\pgfpathlineto{\pgfqpoint{3.236097in}{0.721171in}}%
\pgfpathlineto{\pgfqpoint{3.502717in}{2.091450in}}%
\pgfpathlineto{\pgfqpoint{3.043201in}{2.166904in}}%
\pgfpathlineto{\pgfqpoint{2.789661in}{0.820654in}}%
\pgfpathclose%
\pgfusepath{fill}%
\end{pgfscope}%
\begin{pgfscope}%
\pgfpathrectangle{\pgfqpoint{0.017500in}{0.017500in}}{\pgfqpoint{6.218282in}{3.580372in}} %
\pgfusepath{clip}%
\pgfsetbuttcap%
\pgfsetroundjoin%
\definecolor{currentfill}{rgb}{0.353369,0.472069,0.892570}%
\pgfsetfillcolor{currentfill}%
\pgfsetlinewidth{0.000000pt}%
\definecolor{currentstroke}{rgb}{0.000000,0.000000,0.000000}%
\pgfsetstrokecolor{currentstroke}%
\pgfsetdash{}{0pt}%
\pgfpathmoveto{\pgfqpoint{2.789661in}{0.820654in}}%
\pgfpathlineto{\pgfqpoint{3.236097in}{0.721171in}}%
\pgfpathlineto{\pgfqpoint{3.502717in}{2.091450in}}%
\pgfpathlineto{\pgfqpoint{3.043201in}{2.166904in}}%
\pgfpathlineto{\pgfqpoint{2.789661in}{0.820654in}}%
\pgfpathclose%
\pgfusepath{fill}%
\end{pgfscope}%
\begin{pgfscope}%
\pgfpathrectangle{\pgfqpoint{0.017500in}{0.017500in}}{\pgfqpoint{6.218282in}{3.580372in}} %
\pgfusepath{clip}%
\pgfsetbuttcap%
\pgfsetroundjoin%
\definecolor{currentfill}{rgb}{0.353369,0.472069,0.892570}%
\pgfsetfillcolor{currentfill}%
\pgfsetlinewidth{0.000000pt}%
\definecolor{currentstroke}{rgb}{0.000000,0.000000,0.000000}%
\pgfsetstrokecolor{currentstroke}%
\pgfsetdash{}{0pt}%
\pgfpathmoveto{\pgfqpoint{2.789661in}{0.820654in}}%
\pgfpathlineto{\pgfqpoint{3.236097in}{0.721171in}}%
\pgfpathlineto{\pgfqpoint{3.502717in}{2.091450in}}%
\pgfpathlineto{\pgfqpoint{3.043201in}{2.166904in}}%
\pgfpathlineto{\pgfqpoint{2.789661in}{0.820654in}}%
\pgfpathclose%
\pgfusepath{fill}%
\end{pgfscope}%
\begin{pgfscope}%
\pgfpathrectangle{\pgfqpoint{0.017500in}{0.017500in}}{\pgfqpoint{6.218282in}{3.580372in}} %
\pgfusepath{clip}%
\pgfsetbuttcap%
\pgfsetroundjoin%
\definecolor{currentfill}{rgb}{0.353369,0.472069,0.892570}%
\pgfsetfillcolor{currentfill}%
\pgfsetlinewidth{0.000000pt}%
\definecolor{currentstroke}{rgb}{0.000000,0.000000,0.000000}%
\pgfsetstrokecolor{currentstroke}%
\pgfsetdash{}{0pt}%
\pgfpathmoveto{\pgfqpoint{2.789661in}{0.820654in}}%
\pgfpathlineto{\pgfqpoint{3.236097in}{0.721171in}}%
\pgfpathlineto{\pgfqpoint{3.502717in}{2.091450in}}%
\pgfpathlineto{\pgfqpoint{3.043201in}{2.166904in}}%
\pgfpathlineto{\pgfqpoint{2.789661in}{0.820654in}}%
\pgfpathclose%
\pgfusepath{fill}%
\end{pgfscope}%
\begin{pgfscope}%
\pgfpathrectangle{\pgfqpoint{0.017500in}{0.017500in}}{\pgfqpoint{6.218282in}{3.580372in}} %
\pgfusepath{clip}%
\pgfsetbuttcap%
\pgfsetroundjoin%
\definecolor{currentfill}{rgb}{0.353369,0.472069,0.892570}%
\pgfsetfillcolor{currentfill}%
\pgfsetlinewidth{0.000000pt}%
\definecolor{currentstroke}{rgb}{0.000000,0.000000,0.000000}%
\pgfsetstrokecolor{currentstroke}%
\pgfsetdash{}{0pt}%
\pgfpathmoveto{\pgfqpoint{2.789661in}{0.820654in}}%
\pgfpathlineto{\pgfqpoint{3.236097in}{0.721171in}}%
\pgfpathlineto{\pgfqpoint{3.502717in}{2.091450in}}%
\pgfpathlineto{\pgfqpoint{3.043201in}{2.166904in}}%
\pgfpathlineto{\pgfqpoint{2.789661in}{0.820654in}}%
\pgfpathclose%
\pgfusepath{fill}%
\end{pgfscope}%
\begin{pgfscope}%
\pgfpathrectangle{\pgfqpoint{0.017500in}{0.017500in}}{\pgfqpoint{6.218282in}{3.580372in}} %
\pgfusepath{clip}%
\pgfsetbuttcap%
\pgfsetroundjoin%
\definecolor{currentfill}{rgb}{0.353369,0.472069,0.892570}%
\pgfsetfillcolor{currentfill}%
\pgfsetlinewidth{0.000000pt}%
\definecolor{currentstroke}{rgb}{0.000000,0.000000,0.000000}%
\pgfsetstrokecolor{currentstroke}%
\pgfsetdash{}{0pt}%
\pgfpathmoveto{\pgfqpoint{2.789661in}{0.820654in}}%
\pgfpathlineto{\pgfqpoint{3.236097in}{0.721171in}}%
\pgfpathlineto{\pgfqpoint{3.502717in}{2.091450in}}%
\pgfpathlineto{\pgfqpoint{3.043201in}{2.166904in}}%
\pgfpathlineto{\pgfqpoint{2.789661in}{0.820654in}}%
\pgfpathclose%
\pgfusepath{fill}%
\end{pgfscope}%
\begin{pgfscope}%
\pgfpathrectangle{\pgfqpoint{0.017500in}{0.017500in}}{\pgfqpoint{6.218282in}{3.580372in}} %
\pgfusepath{clip}%
\pgfsetbuttcap%
\pgfsetroundjoin%
\definecolor{currentfill}{rgb}{0.779745,0.210305,0.207104}%
\pgfsetfillcolor{currentfill}%
\pgfsetlinewidth{0.000000pt}%
\definecolor{currentstroke}{rgb}{0.000000,0.000000,0.000000}%
\pgfsetstrokecolor{currentstroke}%
\pgfsetdash{}{0pt}%
\pgfpathmoveto{\pgfqpoint{3.306522in}{2.395850in}}%
\pgfpathlineto{\pgfqpoint{3.564605in}{2.552204in}}%
\pgfpathlineto{\pgfqpoint{3.306522in}{2.395850in}}%
\pgfpathclose%
\pgfusepath{fill}%
\end{pgfscope}%
\begin{pgfscope}%
\pgfpathrectangle{\pgfqpoint{0.017500in}{0.017500in}}{\pgfqpoint{6.218282in}{3.580372in}} %
\pgfusepath{clip}%
\pgfsetbuttcap%
\pgfsetroundjoin%
\definecolor{currentfill}{rgb}{0.779745,0.210305,0.207104}%
\pgfsetfillcolor{currentfill}%
\pgfsetlinewidth{0.000000pt}%
\definecolor{currentstroke}{rgb}{0.000000,0.000000,0.000000}%
\pgfsetstrokecolor{currentstroke}%
\pgfsetdash{}{0pt}%
\pgfpathmoveto{\pgfqpoint{3.306522in}{2.395850in}}%
\pgfpathlineto{\pgfqpoint{3.564605in}{2.552204in}}%
\pgfpathlineto{\pgfqpoint{3.306522in}{2.395850in}}%
\pgfpathclose%
\pgfusepath{fill}%
\end{pgfscope}%
\begin{pgfscope}%
\pgfpathrectangle{\pgfqpoint{0.017500in}{0.017500in}}{\pgfqpoint{6.218282in}{3.580372in}} %
\pgfusepath{clip}%
\pgfsetbuttcap%
\pgfsetroundjoin%
\definecolor{currentfill}{rgb}{0.779745,0.210305,0.207104}%
\pgfsetfillcolor{currentfill}%
\pgfsetlinewidth{0.000000pt}%
\definecolor{currentstroke}{rgb}{0.000000,0.000000,0.000000}%
\pgfsetstrokecolor{currentstroke}%
\pgfsetdash{}{0pt}%
\pgfpathmoveto{\pgfqpoint{3.306522in}{2.395850in}}%
\pgfpathlineto{\pgfqpoint{3.564605in}{2.552204in}}%
\pgfpathlineto{\pgfqpoint{3.306522in}{2.395850in}}%
\pgfpathclose%
\pgfusepath{fill}%
\end{pgfscope}%
\begin{pgfscope}%
\pgfpathrectangle{\pgfqpoint{0.017500in}{0.017500in}}{\pgfqpoint{6.218282in}{3.580372in}} %
\pgfusepath{clip}%
\pgfsetbuttcap%
\pgfsetroundjoin%
\definecolor{currentfill}{rgb}{0.779745,0.210305,0.207104}%
\pgfsetfillcolor{currentfill}%
\pgfsetlinewidth{0.000000pt}%
\definecolor{currentstroke}{rgb}{0.000000,0.000000,0.000000}%
\pgfsetstrokecolor{currentstroke}%
\pgfsetdash{}{0pt}%
\pgfpathmoveto{\pgfqpoint{3.306522in}{2.395850in}}%
\pgfpathlineto{\pgfqpoint{3.564605in}{2.552204in}}%
\pgfpathlineto{\pgfqpoint{3.306522in}{2.395850in}}%
\pgfpathclose%
\pgfusepath{fill}%
\end{pgfscope}%
\begin{pgfscope}%
\pgfpathrectangle{\pgfqpoint{0.017500in}{0.017500in}}{\pgfqpoint{6.218282in}{3.580372in}} %
\pgfusepath{clip}%
\pgfsetbuttcap%
\pgfsetroundjoin%
\definecolor{currentfill}{rgb}{0.779745,0.210305,0.207104}%
\pgfsetfillcolor{currentfill}%
\pgfsetlinewidth{0.000000pt}%
\definecolor{currentstroke}{rgb}{0.000000,0.000000,0.000000}%
\pgfsetstrokecolor{currentstroke}%
\pgfsetdash{}{0pt}%
\pgfpathmoveto{\pgfqpoint{3.306522in}{2.395850in}}%
\pgfpathlineto{\pgfqpoint{3.564605in}{2.552204in}}%
\pgfpathlineto{\pgfqpoint{3.306522in}{2.395850in}}%
\pgfpathclose%
\pgfusepath{fill}%
\end{pgfscope}%
\begin{pgfscope}%
\pgfpathrectangle{\pgfqpoint{0.017500in}{0.017500in}}{\pgfqpoint{6.218282in}{3.580372in}} %
\pgfusepath{clip}%
\pgfsetbuttcap%
\pgfsetroundjoin%
\definecolor{currentfill}{rgb}{0.779745,0.210305,0.207104}%
\pgfsetfillcolor{currentfill}%
\pgfsetlinewidth{0.000000pt}%
\definecolor{currentstroke}{rgb}{0.000000,0.000000,0.000000}%
\pgfsetstrokecolor{currentstroke}%
\pgfsetdash{}{0pt}%
\pgfpathmoveto{\pgfqpoint{3.306522in}{2.395850in}}%
\pgfpathlineto{\pgfqpoint{3.564605in}{2.552204in}}%
\pgfpathlineto{\pgfqpoint{3.306522in}{2.395850in}}%
\pgfpathclose%
\pgfusepath{fill}%
\end{pgfscope}%
\begin{pgfscope}%
\pgfpathrectangle{\pgfqpoint{0.017500in}{0.017500in}}{\pgfqpoint{6.218282in}{3.580372in}} %
\pgfusepath{clip}%
\pgfsetbuttcap%
\pgfsetroundjoin%
\definecolor{currentfill}{rgb}{0.779745,0.210305,0.207104}%
\pgfsetfillcolor{currentfill}%
\pgfsetlinewidth{0.000000pt}%
\definecolor{currentstroke}{rgb}{0.000000,0.000000,0.000000}%
\pgfsetstrokecolor{currentstroke}%
\pgfsetdash{}{0pt}%
\pgfpathmoveto{\pgfqpoint{3.306522in}{2.395850in}}%
\pgfpathlineto{\pgfqpoint{3.564605in}{2.552204in}}%
\pgfpathlineto{\pgfqpoint{3.306522in}{2.395850in}}%
\pgfpathclose%
\pgfusepath{fill}%
\end{pgfscope}%
\begin{pgfscope}%
\pgfpathrectangle{\pgfqpoint{0.017500in}{0.017500in}}{\pgfqpoint{6.218282in}{3.580372in}} %
\pgfusepath{clip}%
\pgfsetbuttcap%
\pgfsetroundjoin%
\definecolor{currentfill}{rgb}{0.779745,0.210305,0.207104}%
\pgfsetfillcolor{currentfill}%
\pgfsetlinewidth{0.000000pt}%
\definecolor{currentstroke}{rgb}{0.000000,0.000000,0.000000}%
\pgfsetstrokecolor{currentstroke}%
\pgfsetdash{}{0pt}%
\pgfpathmoveto{\pgfqpoint{3.306522in}{2.395850in}}%
\pgfpathlineto{\pgfqpoint{3.564605in}{2.552204in}}%
\pgfpathlineto{\pgfqpoint{3.306522in}{2.395850in}}%
\pgfpathclose%
\pgfusepath{fill}%
\end{pgfscope}%
\begin{pgfscope}%
\pgfpathrectangle{\pgfqpoint{0.017500in}{0.017500in}}{\pgfqpoint{6.218282in}{3.580372in}} %
\pgfusepath{clip}%
\pgfsetbuttcap%
\pgfsetroundjoin%
\definecolor{currentfill}{rgb}{0.779745,0.210305,0.207104}%
\pgfsetfillcolor{currentfill}%
\pgfsetlinewidth{0.000000pt}%
\definecolor{currentstroke}{rgb}{0.000000,0.000000,0.000000}%
\pgfsetstrokecolor{currentstroke}%
\pgfsetdash{}{0pt}%
\pgfpathmoveto{\pgfqpoint{3.306522in}{2.395850in}}%
\pgfpathlineto{\pgfqpoint{3.564605in}{2.552204in}}%
\pgfpathlineto{\pgfqpoint{3.306522in}{2.395850in}}%
\pgfpathclose%
\pgfusepath{fill}%
\end{pgfscope}%
\begin{pgfscope}%
\pgfpathrectangle{\pgfqpoint{0.017500in}{0.017500in}}{\pgfqpoint{6.218282in}{3.580372in}} %
\pgfusepath{clip}%
\pgfsetbuttcap%
\pgfsetroundjoin%
\definecolor{currentfill}{rgb}{0.779745,0.210305,0.207104}%
\pgfsetfillcolor{currentfill}%
\pgfsetlinewidth{0.000000pt}%
\definecolor{currentstroke}{rgb}{0.000000,0.000000,0.000000}%
\pgfsetstrokecolor{currentstroke}%
\pgfsetdash{}{0pt}%
\pgfpathmoveto{\pgfqpoint{3.306522in}{2.395850in}}%
\pgfpathlineto{\pgfqpoint{3.564605in}{2.552204in}}%
\pgfpathlineto{\pgfqpoint{3.306522in}{2.395850in}}%
\pgfpathclose%
\pgfusepath{fill}%
\end{pgfscope}%
\begin{pgfscope}%
\pgfpathrectangle{\pgfqpoint{0.017500in}{0.017500in}}{\pgfqpoint{6.218282in}{3.580372in}} %
\pgfusepath{clip}%
\pgfsetbuttcap%
\pgfsetroundjoin%
\definecolor{currentfill}{rgb}{0.779745,0.210305,0.207104}%
\pgfsetfillcolor{currentfill}%
\pgfsetlinewidth{0.000000pt}%
\definecolor{currentstroke}{rgb}{0.000000,0.000000,0.000000}%
\pgfsetstrokecolor{currentstroke}%
\pgfsetdash{}{0pt}%
\pgfpathmoveto{\pgfqpoint{3.306522in}{2.395850in}}%
\pgfpathlineto{\pgfqpoint{3.564605in}{2.552204in}}%
\pgfpathlineto{\pgfqpoint{3.306522in}{2.395850in}}%
\pgfpathclose%
\pgfusepath{fill}%
\end{pgfscope}%
\begin{pgfscope}%
\pgfpathrectangle{\pgfqpoint{0.017500in}{0.017500in}}{\pgfqpoint{6.218282in}{3.580372in}} %
\pgfusepath{clip}%
\pgfsetbuttcap%
\pgfsetroundjoin%
\definecolor{currentfill}{rgb}{0.779745,0.210305,0.207104}%
\pgfsetfillcolor{currentfill}%
\pgfsetlinewidth{0.000000pt}%
\definecolor{currentstroke}{rgb}{0.000000,0.000000,0.000000}%
\pgfsetstrokecolor{currentstroke}%
\pgfsetdash{}{0pt}%
\pgfpathmoveto{\pgfqpoint{3.306522in}{2.395850in}}%
\pgfpathlineto{\pgfqpoint{3.564605in}{2.552204in}}%
\pgfpathlineto{\pgfqpoint{3.306522in}{2.395850in}}%
\pgfpathclose%
\pgfusepath{fill}%
\end{pgfscope}%
\begin{pgfscope}%
\pgfpathrectangle{\pgfqpoint{0.017500in}{0.017500in}}{\pgfqpoint{6.218282in}{3.580372in}} %
\pgfusepath{clip}%
\pgfsetbuttcap%
\pgfsetroundjoin%
\definecolor{currentfill}{rgb}{0.779745,0.210305,0.207104}%
\pgfsetfillcolor{currentfill}%
\pgfsetlinewidth{0.000000pt}%
\definecolor{currentstroke}{rgb}{0.000000,0.000000,0.000000}%
\pgfsetstrokecolor{currentstroke}%
\pgfsetdash{}{0pt}%
\pgfpathmoveto{\pgfqpoint{3.306522in}{2.395850in}}%
\pgfpathlineto{\pgfqpoint{3.564605in}{2.552204in}}%
\pgfpathlineto{\pgfqpoint{3.306522in}{2.395850in}}%
\pgfpathclose%
\pgfusepath{fill}%
\end{pgfscope}%
\begin{pgfscope}%
\pgfpathrectangle{\pgfqpoint{0.017500in}{0.017500in}}{\pgfqpoint{6.218282in}{3.580372in}} %
\pgfusepath{clip}%
\pgfsetbuttcap%
\pgfsetroundjoin%
\definecolor{currentfill}{rgb}{0.779745,0.210305,0.207104}%
\pgfsetfillcolor{currentfill}%
\pgfsetlinewidth{0.000000pt}%
\definecolor{currentstroke}{rgb}{0.000000,0.000000,0.000000}%
\pgfsetstrokecolor{currentstroke}%
\pgfsetdash{}{0pt}%
\pgfpathmoveto{\pgfqpoint{3.306522in}{2.395850in}}%
\pgfpathlineto{\pgfqpoint{3.564605in}{2.552204in}}%
\pgfpathlineto{\pgfqpoint{3.306522in}{2.395850in}}%
\pgfpathclose%
\pgfusepath{fill}%
\end{pgfscope}%
\begin{pgfscope}%
\pgfpathrectangle{\pgfqpoint{0.017500in}{0.017500in}}{\pgfqpoint{6.218282in}{3.580372in}} %
\pgfusepath{clip}%
\pgfsetbuttcap%
\pgfsetroundjoin%
\definecolor{currentfill}{rgb}{0.779745,0.210305,0.207104}%
\pgfsetfillcolor{currentfill}%
\pgfsetlinewidth{0.000000pt}%
\definecolor{currentstroke}{rgb}{0.000000,0.000000,0.000000}%
\pgfsetstrokecolor{currentstroke}%
\pgfsetdash{}{0pt}%
\pgfpathmoveto{\pgfqpoint{3.306522in}{2.395850in}}%
\pgfpathlineto{\pgfqpoint{3.564605in}{2.552204in}}%
\pgfpathlineto{\pgfqpoint{3.306522in}{2.395850in}}%
\pgfpathclose%
\pgfusepath{fill}%
\end{pgfscope}%
\begin{pgfscope}%
\pgfpathrectangle{\pgfqpoint{0.017500in}{0.017500in}}{\pgfqpoint{6.218282in}{3.580372in}} %
\pgfusepath{clip}%
\pgfsetbuttcap%
\pgfsetroundjoin%
\definecolor{currentfill}{rgb}{0.779745,0.210305,0.207104}%
\pgfsetfillcolor{currentfill}%
\pgfsetlinewidth{0.000000pt}%
\definecolor{currentstroke}{rgb}{0.000000,0.000000,0.000000}%
\pgfsetstrokecolor{currentstroke}%
\pgfsetdash{}{0pt}%
\pgfpathmoveto{\pgfqpoint{3.306522in}{2.395850in}}%
\pgfpathlineto{\pgfqpoint{3.564605in}{2.552204in}}%
\pgfpathlineto{\pgfqpoint{3.306522in}{2.395850in}}%
\pgfpathclose%
\pgfusepath{fill}%
\end{pgfscope}%
\begin{pgfscope}%
\pgfpathrectangle{\pgfqpoint{0.017500in}{0.017500in}}{\pgfqpoint{6.218282in}{3.580372in}} %
\pgfusepath{clip}%
\pgfsetbuttcap%
\pgfsetroundjoin%
\definecolor{currentfill}{rgb}{0.779745,0.210305,0.207104}%
\pgfsetfillcolor{currentfill}%
\pgfsetlinewidth{0.000000pt}%
\definecolor{currentstroke}{rgb}{0.000000,0.000000,0.000000}%
\pgfsetstrokecolor{currentstroke}%
\pgfsetdash{}{0pt}%
\pgfpathmoveto{\pgfqpoint{3.306522in}{2.395850in}}%
\pgfpathlineto{\pgfqpoint{3.564605in}{2.552204in}}%
\pgfpathlineto{\pgfqpoint{3.306522in}{2.395850in}}%
\pgfpathclose%
\pgfusepath{fill}%
\end{pgfscope}%
\begin{pgfscope}%
\pgfpathrectangle{\pgfqpoint{0.017500in}{0.017500in}}{\pgfqpoint{6.218282in}{3.580372in}} %
\pgfusepath{clip}%
\pgfsetbuttcap%
\pgfsetroundjoin%
\definecolor{currentfill}{rgb}{0.779745,0.210305,0.207104}%
\pgfsetfillcolor{currentfill}%
\pgfsetlinewidth{0.000000pt}%
\definecolor{currentstroke}{rgb}{0.000000,0.000000,0.000000}%
\pgfsetstrokecolor{currentstroke}%
\pgfsetdash{}{0pt}%
\pgfpathmoveto{\pgfqpoint{3.306522in}{2.395850in}}%
\pgfpathlineto{\pgfqpoint{3.564605in}{2.552204in}}%
\pgfpathlineto{\pgfqpoint{3.306522in}{2.395850in}}%
\pgfpathclose%
\pgfusepath{fill}%
\end{pgfscope}%
\begin{pgfscope}%
\pgfpathrectangle{\pgfqpoint{0.017500in}{0.017500in}}{\pgfqpoint{6.218282in}{3.580372in}} %
\pgfusepath{clip}%
\pgfsetbuttcap%
\pgfsetroundjoin%
\definecolor{currentfill}{rgb}{0.779745,0.210305,0.207104}%
\pgfsetfillcolor{currentfill}%
\pgfsetlinewidth{0.000000pt}%
\definecolor{currentstroke}{rgb}{0.000000,0.000000,0.000000}%
\pgfsetstrokecolor{currentstroke}%
\pgfsetdash{}{0pt}%
\pgfpathmoveto{\pgfqpoint{3.306522in}{2.395850in}}%
\pgfpathlineto{\pgfqpoint{3.564605in}{2.552204in}}%
\pgfpathlineto{\pgfqpoint{3.306522in}{2.395850in}}%
\pgfpathclose%
\pgfusepath{fill}%
\end{pgfscope}%
\begin{pgfscope}%
\pgfpathrectangle{\pgfqpoint{0.017500in}{0.017500in}}{\pgfqpoint{6.218282in}{3.580372in}} %
\pgfusepath{clip}%
\pgfsetbuttcap%
\pgfsetroundjoin%
\definecolor{currentfill}{rgb}{0.779745,0.210305,0.207104}%
\pgfsetfillcolor{currentfill}%
\pgfsetlinewidth{0.000000pt}%
\definecolor{currentstroke}{rgb}{0.000000,0.000000,0.000000}%
\pgfsetstrokecolor{currentstroke}%
\pgfsetdash{}{0pt}%
\pgfpathmoveto{\pgfqpoint{3.306522in}{2.395850in}}%
\pgfpathlineto{\pgfqpoint{3.564605in}{2.552204in}}%
\pgfpathlineto{\pgfqpoint{3.306522in}{2.395850in}}%
\pgfpathclose%
\pgfusepath{fill}%
\end{pgfscope}%
\begin{pgfscope}%
\pgfpathrectangle{\pgfqpoint{0.017500in}{0.017500in}}{\pgfqpoint{6.218282in}{3.580372in}} %
\pgfusepath{clip}%
\pgfsetbuttcap%
\pgfsetroundjoin%
\definecolor{currentfill}{rgb}{0.779745,0.210305,0.207104}%
\pgfsetfillcolor{currentfill}%
\pgfsetlinewidth{0.000000pt}%
\definecolor{currentstroke}{rgb}{0.000000,0.000000,0.000000}%
\pgfsetstrokecolor{currentstroke}%
\pgfsetdash{}{0pt}%
\pgfpathmoveto{\pgfqpoint{3.306522in}{2.395850in}}%
\pgfpathlineto{\pgfqpoint{3.564605in}{2.552204in}}%
\pgfpathlineto{\pgfqpoint{3.306522in}{2.395850in}}%
\pgfpathclose%
\pgfusepath{fill}%
\end{pgfscope}%
\begin{pgfscope}%
\pgfpathrectangle{\pgfqpoint{0.017500in}{0.017500in}}{\pgfqpoint{6.218282in}{3.580372in}} %
\pgfusepath{clip}%
\pgfsetbuttcap%
\pgfsetroundjoin%
\definecolor{currentfill}{rgb}{0.779745,0.210305,0.207104}%
\pgfsetfillcolor{currentfill}%
\pgfsetlinewidth{0.000000pt}%
\definecolor{currentstroke}{rgb}{0.000000,0.000000,0.000000}%
\pgfsetstrokecolor{currentstroke}%
\pgfsetdash{}{0pt}%
\pgfpathmoveto{\pgfqpoint{3.306522in}{2.395850in}}%
\pgfpathlineto{\pgfqpoint{3.564605in}{2.552204in}}%
\pgfpathlineto{\pgfqpoint{3.306522in}{2.395850in}}%
\pgfpathclose%
\pgfusepath{fill}%
\end{pgfscope}%
\begin{pgfscope}%
\pgfpathrectangle{\pgfqpoint{0.017500in}{0.017500in}}{\pgfqpoint{6.218282in}{3.580372in}} %
\pgfusepath{clip}%
\pgfsetbuttcap%
\pgfsetroundjoin%
\definecolor{currentfill}{rgb}{0.779745,0.210305,0.207104}%
\pgfsetfillcolor{currentfill}%
\pgfsetlinewidth{0.000000pt}%
\definecolor{currentstroke}{rgb}{0.000000,0.000000,0.000000}%
\pgfsetstrokecolor{currentstroke}%
\pgfsetdash{}{0pt}%
\pgfpathmoveto{\pgfqpoint{3.306522in}{2.395850in}}%
\pgfpathlineto{\pgfqpoint{3.564605in}{2.552204in}}%
\pgfpathlineto{\pgfqpoint{3.306522in}{2.395850in}}%
\pgfpathclose%
\pgfusepath{fill}%
\end{pgfscope}%
\begin{pgfscope}%
\pgfpathrectangle{\pgfqpoint{0.017500in}{0.017500in}}{\pgfqpoint{6.218282in}{3.580372in}} %
\pgfusepath{clip}%
\pgfsetbuttcap%
\pgfsetroundjoin%
\definecolor{currentfill}{rgb}{0.779745,0.210305,0.207104}%
\pgfsetfillcolor{currentfill}%
\pgfsetlinewidth{0.000000pt}%
\definecolor{currentstroke}{rgb}{0.000000,0.000000,0.000000}%
\pgfsetstrokecolor{currentstroke}%
\pgfsetdash{}{0pt}%
\pgfpathmoveto{\pgfqpoint{3.306522in}{2.395850in}}%
\pgfpathlineto{\pgfqpoint{3.564605in}{2.552204in}}%
\pgfpathlineto{\pgfqpoint{3.306522in}{2.395850in}}%
\pgfpathclose%
\pgfusepath{fill}%
\end{pgfscope}%
\begin{pgfscope}%
\pgfpathrectangle{\pgfqpoint{0.017500in}{0.017500in}}{\pgfqpoint{6.218282in}{3.580372in}} %
\pgfusepath{clip}%
\pgfsetbuttcap%
\pgfsetroundjoin%
\definecolor{currentfill}{rgb}{0.779745,0.210305,0.207104}%
\pgfsetfillcolor{currentfill}%
\pgfsetlinewidth{0.000000pt}%
\definecolor{currentstroke}{rgb}{0.000000,0.000000,0.000000}%
\pgfsetstrokecolor{currentstroke}%
\pgfsetdash{}{0pt}%
\pgfpathmoveto{\pgfqpoint{3.306522in}{2.395850in}}%
\pgfpathlineto{\pgfqpoint{3.564605in}{2.552204in}}%
\pgfpathlineto{\pgfqpoint{3.306522in}{2.395850in}}%
\pgfpathclose%
\pgfusepath{fill}%
\end{pgfscope}%
\begin{pgfscope}%
\pgfpathrectangle{\pgfqpoint{0.017500in}{0.017500in}}{\pgfqpoint{6.218282in}{3.580372in}} %
\pgfusepath{clip}%
\pgfsetbuttcap%
\pgfsetroundjoin%
\definecolor{currentfill}{rgb}{0.779745,0.210305,0.207104}%
\pgfsetfillcolor{currentfill}%
\pgfsetlinewidth{0.000000pt}%
\definecolor{currentstroke}{rgb}{0.000000,0.000000,0.000000}%
\pgfsetstrokecolor{currentstroke}%
\pgfsetdash{}{0pt}%
\pgfpathmoveto{\pgfqpoint{3.306522in}{2.395850in}}%
\pgfpathlineto{\pgfqpoint{3.564605in}{2.552204in}}%
\pgfpathlineto{\pgfqpoint{3.306522in}{2.395850in}}%
\pgfpathclose%
\pgfusepath{fill}%
\end{pgfscope}%
\begin{pgfscope}%
\pgfpathrectangle{\pgfqpoint{0.017500in}{0.017500in}}{\pgfqpoint{6.218282in}{3.580372in}} %
\pgfusepath{clip}%
\pgfsetbuttcap%
\pgfsetroundjoin%
\definecolor{currentfill}{rgb}{0.779745,0.210305,0.207104}%
\pgfsetfillcolor{currentfill}%
\pgfsetlinewidth{0.000000pt}%
\definecolor{currentstroke}{rgb}{0.000000,0.000000,0.000000}%
\pgfsetstrokecolor{currentstroke}%
\pgfsetdash{}{0pt}%
\pgfpathmoveto{\pgfqpoint{3.306522in}{2.395850in}}%
\pgfpathlineto{\pgfqpoint{3.564605in}{2.552204in}}%
\pgfpathlineto{\pgfqpoint{3.306522in}{2.395850in}}%
\pgfpathclose%
\pgfusepath{fill}%
\end{pgfscope}%
\begin{pgfscope}%
\pgfpathrectangle{\pgfqpoint{0.017500in}{0.017500in}}{\pgfqpoint{6.218282in}{3.580372in}} %
\pgfusepath{clip}%
\pgfsetbuttcap%
\pgfsetroundjoin%
\definecolor{currentfill}{rgb}{0.779745,0.210305,0.207104}%
\pgfsetfillcolor{currentfill}%
\pgfsetlinewidth{0.000000pt}%
\definecolor{currentstroke}{rgb}{0.000000,0.000000,0.000000}%
\pgfsetstrokecolor{currentstroke}%
\pgfsetdash{}{0pt}%
\pgfpathmoveto{\pgfqpoint{3.306522in}{2.395850in}}%
\pgfpathlineto{\pgfqpoint{3.564605in}{2.552204in}}%
\pgfpathlineto{\pgfqpoint{3.306522in}{2.395850in}}%
\pgfpathclose%
\pgfusepath{fill}%
\end{pgfscope}%
\begin{pgfscope}%
\pgfpathrectangle{\pgfqpoint{0.017500in}{0.017500in}}{\pgfqpoint{6.218282in}{3.580372in}} %
\pgfusepath{clip}%
\pgfsetbuttcap%
\pgfsetroundjoin%
\definecolor{currentfill}{rgb}{0.779745,0.210305,0.207104}%
\pgfsetfillcolor{currentfill}%
\pgfsetlinewidth{0.000000pt}%
\definecolor{currentstroke}{rgb}{0.000000,0.000000,0.000000}%
\pgfsetstrokecolor{currentstroke}%
\pgfsetdash{}{0pt}%
\pgfpathmoveto{\pgfqpoint{3.306522in}{2.395850in}}%
\pgfpathlineto{\pgfqpoint{3.564605in}{2.552204in}}%
\pgfpathlineto{\pgfqpoint{3.306522in}{2.395850in}}%
\pgfpathclose%
\pgfusepath{fill}%
\end{pgfscope}%
\begin{pgfscope}%
\pgfpathrectangle{\pgfqpoint{0.017500in}{0.017500in}}{\pgfqpoint{6.218282in}{3.580372in}} %
\pgfusepath{clip}%
\pgfsetbuttcap%
\pgfsetroundjoin%
\definecolor{currentfill}{rgb}{0.779745,0.210305,0.207104}%
\pgfsetfillcolor{currentfill}%
\pgfsetlinewidth{0.000000pt}%
\definecolor{currentstroke}{rgb}{0.000000,0.000000,0.000000}%
\pgfsetstrokecolor{currentstroke}%
\pgfsetdash{}{0pt}%
\pgfpathmoveto{\pgfqpoint{3.306522in}{2.395850in}}%
\pgfpathlineto{\pgfqpoint{3.564605in}{2.552204in}}%
\pgfpathlineto{\pgfqpoint{3.306522in}{2.395850in}}%
\pgfpathclose%
\pgfusepath{fill}%
\end{pgfscope}%
\begin{pgfscope}%
\pgfpathrectangle{\pgfqpoint{0.017500in}{0.017500in}}{\pgfqpoint{6.218282in}{3.580372in}} %
\pgfusepath{clip}%
\pgfsetbuttcap%
\pgfsetroundjoin%
\definecolor{currentfill}{rgb}{0.785153,0.220851,0.211673}%
\pgfsetfillcolor{currentfill}%
\pgfsetlinewidth{0.000000pt}%
\definecolor{currentstroke}{rgb}{0.000000,0.000000,0.000000}%
\pgfsetstrokecolor{currentstroke}%
\pgfsetdash{}{0pt}%
\pgfpathmoveto{\pgfqpoint{3.306522in}{2.395850in}}%
\pgfpathlineto{\pgfqpoint{3.763681in}{2.314857in}}%
\pgfpathlineto{\pgfqpoint{4.018512in}{2.470906in}}%
\pgfpathlineto{\pgfqpoint{3.564605in}{2.552204in}}%
\pgfpathlineto{\pgfqpoint{3.306522in}{2.395850in}}%
\pgfpathclose%
\pgfusepath{fill}%
\end{pgfscope}%
\begin{pgfscope}%
\pgfpathrectangle{\pgfqpoint{0.017500in}{0.017500in}}{\pgfqpoint{6.218282in}{3.580372in}} %
\pgfusepath{clip}%
\pgfsetbuttcap%
\pgfsetroundjoin%
\definecolor{currentfill}{rgb}{0.785153,0.220851,0.211673}%
\pgfsetfillcolor{currentfill}%
\pgfsetlinewidth{0.000000pt}%
\definecolor{currentstroke}{rgb}{0.000000,0.000000,0.000000}%
\pgfsetstrokecolor{currentstroke}%
\pgfsetdash{}{0pt}%
\pgfpathmoveto{\pgfqpoint{3.306522in}{2.395850in}}%
\pgfpathlineto{\pgfqpoint{3.763681in}{2.314857in}}%
\pgfpathlineto{\pgfqpoint{4.018512in}{2.470906in}}%
\pgfpathlineto{\pgfqpoint{3.564605in}{2.552204in}}%
\pgfpathlineto{\pgfqpoint{3.306522in}{2.395850in}}%
\pgfpathclose%
\pgfusepath{fill}%
\end{pgfscope}%
\begin{pgfscope}%
\pgfpathrectangle{\pgfqpoint{0.017500in}{0.017500in}}{\pgfqpoint{6.218282in}{3.580372in}} %
\pgfusepath{clip}%
\pgfsetbuttcap%
\pgfsetroundjoin%
\definecolor{currentfill}{rgb}{0.785153,0.220851,0.211673}%
\pgfsetfillcolor{currentfill}%
\pgfsetlinewidth{0.000000pt}%
\definecolor{currentstroke}{rgb}{0.000000,0.000000,0.000000}%
\pgfsetstrokecolor{currentstroke}%
\pgfsetdash{}{0pt}%
\pgfpathmoveto{\pgfqpoint{3.306522in}{2.395850in}}%
\pgfpathlineto{\pgfqpoint{3.763681in}{2.314857in}}%
\pgfpathlineto{\pgfqpoint{4.018512in}{2.470906in}}%
\pgfpathlineto{\pgfqpoint{3.564605in}{2.552204in}}%
\pgfpathlineto{\pgfqpoint{3.306522in}{2.395850in}}%
\pgfpathclose%
\pgfusepath{fill}%
\end{pgfscope}%
\begin{pgfscope}%
\pgfpathrectangle{\pgfqpoint{0.017500in}{0.017500in}}{\pgfqpoint{6.218282in}{3.580372in}} %
\pgfusepath{clip}%
\pgfsetbuttcap%
\pgfsetroundjoin%
\definecolor{currentfill}{rgb}{0.785153,0.220851,0.211673}%
\pgfsetfillcolor{currentfill}%
\pgfsetlinewidth{0.000000pt}%
\definecolor{currentstroke}{rgb}{0.000000,0.000000,0.000000}%
\pgfsetstrokecolor{currentstroke}%
\pgfsetdash{}{0pt}%
\pgfpathmoveto{\pgfqpoint{3.306522in}{2.395850in}}%
\pgfpathlineto{\pgfqpoint{3.763681in}{2.314857in}}%
\pgfpathlineto{\pgfqpoint{4.018512in}{2.470906in}}%
\pgfpathlineto{\pgfqpoint{3.564605in}{2.552204in}}%
\pgfpathlineto{\pgfqpoint{3.306522in}{2.395850in}}%
\pgfpathclose%
\pgfusepath{fill}%
\end{pgfscope}%
\begin{pgfscope}%
\pgfpathrectangle{\pgfqpoint{0.017500in}{0.017500in}}{\pgfqpoint{6.218282in}{3.580372in}} %
\pgfusepath{clip}%
\pgfsetbuttcap%
\pgfsetroundjoin%
\definecolor{currentfill}{rgb}{0.785153,0.220851,0.211673}%
\pgfsetfillcolor{currentfill}%
\pgfsetlinewidth{0.000000pt}%
\definecolor{currentstroke}{rgb}{0.000000,0.000000,0.000000}%
\pgfsetstrokecolor{currentstroke}%
\pgfsetdash{}{0pt}%
\pgfpathmoveto{\pgfqpoint{3.306522in}{2.395850in}}%
\pgfpathlineto{\pgfqpoint{3.763681in}{2.314857in}}%
\pgfpathlineto{\pgfqpoint{4.018512in}{2.470906in}}%
\pgfpathlineto{\pgfqpoint{3.564605in}{2.552204in}}%
\pgfpathlineto{\pgfqpoint{3.306522in}{2.395850in}}%
\pgfpathclose%
\pgfusepath{fill}%
\end{pgfscope}%
\begin{pgfscope}%
\pgfpathrectangle{\pgfqpoint{0.017500in}{0.017500in}}{\pgfqpoint{6.218282in}{3.580372in}} %
\pgfusepath{clip}%
\pgfsetbuttcap%
\pgfsetroundjoin%
\definecolor{currentfill}{rgb}{0.785153,0.220851,0.211673}%
\pgfsetfillcolor{currentfill}%
\pgfsetlinewidth{0.000000pt}%
\definecolor{currentstroke}{rgb}{0.000000,0.000000,0.000000}%
\pgfsetstrokecolor{currentstroke}%
\pgfsetdash{}{0pt}%
\pgfpathmoveto{\pgfqpoint{3.306522in}{2.395850in}}%
\pgfpathlineto{\pgfqpoint{3.763681in}{2.314857in}}%
\pgfpathlineto{\pgfqpoint{4.018512in}{2.470906in}}%
\pgfpathlineto{\pgfqpoint{3.564605in}{2.552204in}}%
\pgfpathlineto{\pgfqpoint{3.306522in}{2.395850in}}%
\pgfpathclose%
\pgfusepath{fill}%
\end{pgfscope}%
\begin{pgfscope}%
\pgfpathrectangle{\pgfqpoint{0.017500in}{0.017500in}}{\pgfqpoint{6.218282in}{3.580372in}} %
\pgfusepath{clip}%
\pgfsetbuttcap%
\pgfsetroundjoin%
\definecolor{currentfill}{rgb}{0.243520,0.319189,0.771672}%
\pgfsetfillcolor{currentfill}%
\pgfsetlinewidth{0.000000pt}%
\definecolor{currentstroke}{rgb}{0.000000,0.000000,0.000000}%
\pgfsetstrokecolor{currentstroke}%
\pgfsetdash{}{0pt}%
\pgfpathmoveto{\pgfqpoint{3.236097in}{0.721171in}}%
\pgfpathlineto{\pgfqpoint{3.502717in}{2.091450in}}%
\pgfpathlineto{\pgfqpoint{3.236097in}{0.721171in}}%
\pgfpathclose%
\pgfusepath{fill}%
\end{pgfscope}%
\begin{pgfscope}%
\pgfpathrectangle{\pgfqpoint{0.017500in}{0.017500in}}{\pgfqpoint{6.218282in}{3.580372in}} %
\pgfusepath{clip}%
\pgfsetbuttcap%
\pgfsetroundjoin%
\definecolor{currentfill}{rgb}{0.243520,0.319189,0.771672}%
\pgfsetfillcolor{currentfill}%
\pgfsetlinewidth{0.000000pt}%
\definecolor{currentstroke}{rgb}{0.000000,0.000000,0.000000}%
\pgfsetstrokecolor{currentstroke}%
\pgfsetdash{}{0pt}%
\pgfpathmoveto{\pgfqpoint{3.236097in}{0.721171in}}%
\pgfpathlineto{\pgfqpoint{3.502717in}{2.091450in}}%
\pgfpathlineto{\pgfqpoint{3.236097in}{0.721171in}}%
\pgfpathclose%
\pgfusepath{fill}%
\end{pgfscope}%
\begin{pgfscope}%
\pgfpathrectangle{\pgfqpoint{0.017500in}{0.017500in}}{\pgfqpoint{6.218282in}{3.580372in}} %
\pgfusepath{clip}%
\pgfsetbuttcap%
\pgfsetroundjoin%
\definecolor{currentfill}{rgb}{0.243520,0.319189,0.771672}%
\pgfsetfillcolor{currentfill}%
\pgfsetlinewidth{0.000000pt}%
\definecolor{currentstroke}{rgb}{0.000000,0.000000,0.000000}%
\pgfsetstrokecolor{currentstroke}%
\pgfsetdash{}{0pt}%
\pgfpathmoveto{\pgfqpoint{3.236097in}{0.721171in}}%
\pgfpathlineto{\pgfqpoint{3.502717in}{2.091450in}}%
\pgfpathlineto{\pgfqpoint{3.236097in}{0.721171in}}%
\pgfpathclose%
\pgfusepath{fill}%
\end{pgfscope}%
\begin{pgfscope}%
\pgfpathrectangle{\pgfqpoint{0.017500in}{0.017500in}}{\pgfqpoint{6.218282in}{3.580372in}} %
\pgfusepath{clip}%
\pgfsetbuttcap%
\pgfsetroundjoin%
\definecolor{currentfill}{rgb}{0.243520,0.319189,0.771672}%
\pgfsetfillcolor{currentfill}%
\pgfsetlinewidth{0.000000pt}%
\definecolor{currentstroke}{rgb}{0.000000,0.000000,0.000000}%
\pgfsetstrokecolor{currentstroke}%
\pgfsetdash{}{0pt}%
\pgfpathmoveto{\pgfqpoint{3.236097in}{0.721171in}}%
\pgfpathlineto{\pgfqpoint{3.502717in}{2.091450in}}%
\pgfpathlineto{\pgfqpoint{3.236097in}{0.721171in}}%
\pgfpathclose%
\pgfusepath{fill}%
\end{pgfscope}%
\begin{pgfscope}%
\pgfpathrectangle{\pgfqpoint{0.017500in}{0.017500in}}{\pgfqpoint{6.218282in}{3.580372in}} %
\pgfusepath{clip}%
\pgfsetbuttcap%
\pgfsetroundjoin%
\definecolor{currentfill}{rgb}{0.243520,0.319189,0.771672}%
\pgfsetfillcolor{currentfill}%
\pgfsetlinewidth{0.000000pt}%
\definecolor{currentstroke}{rgb}{0.000000,0.000000,0.000000}%
\pgfsetstrokecolor{currentstroke}%
\pgfsetdash{}{0pt}%
\pgfpathmoveto{\pgfqpoint{3.236097in}{0.721171in}}%
\pgfpathlineto{\pgfqpoint{3.502717in}{2.091450in}}%
\pgfpathlineto{\pgfqpoint{3.236097in}{0.721171in}}%
\pgfpathclose%
\pgfusepath{fill}%
\end{pgfscope}%
\begin{pgfscope}%
\pgfpathrectangle{\pgfqpoint{0.017500in}{0.017500in}}{\pgfqpoint{6.218282in}{3.580372in}} %
\pgfusepath{clip}%
\pgfsetbuttcap%
\pgfsetroundjoin%
\definecolor{currentfill}{rgb}{0.243520,0.319189,0.771672}%
\pgfsetfillcolor{currentfill}%
\pgfsetlinewidth{0.000000pt}%
\definecolor{currentstroke}{rgb}{0.000000,0.000000,0.000000}%
\pgfsetstrokecolor{currentstroke}%
\pgfsetdash{}{0pt}%
\pgfpathmoveto{\pgfqpoint{3.236097in}{0.721171in}}%
\pgfpathlineto{\pgfqpoint{3.502717in}{2.091450in}}%
\pgfpathlineto{\pgfqpoint{3.236097in}{0.721171in}}%
\pgfpathclose%
\pgfusepath{fill}%
\end{pgfscope}%
\begin{pgfscope}%
\pgfpathrectangle{\pgfqpoint{0.017500in}{0.017500in}}{\pgfqpoint{6.218282in}{3.580372in}} %
\pgfusepath{clip}%
\pgfsetbuttcap%
\pgfsetroundjoin%
\definecolor{currentfill}{rgb}{0.243520,0.319189,0.771672}%
\pgfsetfillcolor{currentfill}%
\pgfsetlinewidth{0.000000pt}%
\definecolor{currentstroke}{rgb}{0.000000,0.000000,0.000000}%
\pgfsetstrokecolor{currentstroke}%
\pgfsetdash{}{0pt}%
\pgfpathmoveto{\pgfqpoint{3.236097in}{0.721171in}}%
\pgfpathlineto{\pgfqpoint{3.502717in}{2.091450in}}%
\pgfpathlineto{\pgfqpoint{3.236097in}{0.721171in}}%
\pgfpathclose%
\pgfusepath{fill}%
\end{pgfscope}%
\begin{pgfscope}%
\pgfpathrectangle{\pgfqpoint{0.017500in}{0.017500in}}{\pgfqpoint{6.218282in}{3.580372in}} %
\pgfusepath{clip}%
\pgfsetbuttcap%
\pgfsetroundjoin%
\definecolor{currentfill}{rgb}{0.243520,0.319189,0.771672}%
\pgfsetfillcolor{currentfill}%
\pgfsetlinewidth{0.000000pt}%
\definecolor{currentstroke}{rgb}{0.000000,0.000000,0.000000}%
\pgfsetstrokecolor{currentstroke}%
\pgfsetdash{}{0pt}%
\pgfpathmoveto{\pgfqpoint{3.236097in}{0.721171in}}%
\pgfpathlineto{\pgfqpoint{3.502717in}{2.091450in}}%
\pgfpathlineto{\pgfqpoint{3.236097in}{0.721171in}}%
\pgfpathclose%
\pgfusepath{fill}%
\end{pgfscope}%
\begin{pgfscope}%
\pgfpathrectangle{\pgfqpoint{0.017500in}{0.017500in}}{\pgfqpoint{6.218282in}{3.580372in}} %
\pgfusepath{clip}%
\pgfsetbuttcap%
\pgfsetroundjoin%
\definecolor{currentfill}{rgb}{0.243520,0.319189,0.771672}%
\pgfsetfillcolor{currentfill}%
\pgfsetlinewidth{0.000000pt}%
\definecolor{currentstroke}{rgb}{0.000000,0.000000,0.000000}%
\pgfsetstrokecolor{currentstroke}%
\pgfsetdash{}{0pt}%
\pgfpathmoveto{\pgfqpoint{3.236097in}{0.721171in}}%
\pgfpathlineto{\pgfqpoint{3.502717in}{2.091450in}}%
\pgfpathlineto{\pgfqpoint{3.236097in}{0.721171in}}%
\pgfpathclose%
\pgfusepath{fill}%
\end{pgfscope}%
\begin{pgfscope}%
\pgfpathrectangle{\pgfqpoint{0.017500in}{0.017500in}}{\pgfqpoint{6.218282in}{3.580372in}} %
\pgfusepath{clip}%
\pgfsetbuttcap%
\pgfsetroundjoin%
\definecolor{currentfill}{rgb}{0.243520,0.319189,0.771672}%
\pgfsetfillcolor{currentfill}%
\pgfsetlinewidth{0.000000pt}%
\definecolor{currentstroke}{rgb}{0.000000,0.000000,0.000000}%
\pgfsetstrokecolor{currentstroke}%
\pgfsetdash{}{0pt}%
\pgfpathmoveto{\pgfqpoint{3.236097in}{0.721171in}}%
\pgfpathlineto{\pgfqpoint{3.502717in}{2.091450in}}%
\pgfpathlineto{\pgfqpoint{3.236097in}{0.721171in}}%
\pgfpathclose%
\pgfusepath{fill}%
\end{pgfscope}%
\begin{pgfscope}%
\pgfpathrectangle{\pgfqpoint{0.017500in}{0.017500in}}{\pgfqpoint{6.218282in}{3.580372in}} %
\pgfusepath{clip}%
\pgfsetbuttcap%
\pgfsetroundjoin%
\definecolor{currentfill}{rgb}{0.243520,0.319189,0.771672}%
\pgfsetfillcolor{currentfill}%
\pgfsetlinewidth{0.000000pt}%
\definecolor{currentstroke}{rgb}{0.000000,0.000000,0.000000}%
\pgfsetstrokecolor{currentstroke}%
\pgfsetdash{}{0pt}%
\pgfpathmoveto{\pgfqpoint{3.236097in}{0.721171in}}%
\pgfpathlineto{\pgfqpoint{3.502717in}{2.091450in}}%
\pgfpathlineto{\pgfqpoint{3.236097in}{0.721171in}}%
\pgfpathclose%
\pgfusepath{fill}%
\end{pgfscope}%
\begin{pgfscope}%
\pgfpathrectangle{\pgfqpoint{0.017500in}{0.017500in}}{\pgfqpoint{6.218282in}{3.580372in}} %
\pgfusepath{clip}%
\pgfsetbuttcap%
\pgfsetroundjoin%
\definecolor{currentfill}{rgb}{0.243520,0.319189,0.771672}%
\pgfsetfillcolor{currentfill}%
\pgfsetlinewidth{0.000000pt}%
\definecolor{currentstroke}{rgb}{0.000000,0.000000,0.000000}%
\pgfsetstrokecolor{currentstroke}%
\pgfsetdash{}{0pt}%
\pgfpathmoveto{\pgfqpoint{3.236097in}{0.721171in}}%
\pgfpathlineto{\pgfqpoint{3.502717in}{2.091450in}}%
\pgfpathlineto{\pgfqpoint{3.236097in}{0.721171in}}%
\pgfpathclose%
\pgfusepath{fill}%
\end{pgfscope}%
\begin{pgfscope}%
\pgfpathrectangle{\pgfqpoint{0.017500in}{0.017500in}}{\pgfqpoint{6.218282in}{3.580372in}} %
\pgfusepath{clip}%
\pgfsetbuttcap%
\pgfsetroundjoin%
\definecolor{currentfill}{rgb}{0.243520,0.319189,0.771672}%
\pgfsetfillcolor{currentfill}%
\pgfsetlinewidth{0.000000pt}%
\definecolor{currentstroke}{rgb}{0.000000,0.000000,0.000000}%
\pgfsetstrokecolor{currentstroke}%
\pgfsetdash{}{0pt}%
\pgfpathmoveto{\pgfqpoint{3.236097in}{0.721171in}}%
\pgfpathlineto{\pgfqpoint{3.502717in}{2.091450in}}%
\pgfpathlineto{\pgfqpoint{3.236097in}{0.721171in}}%
\pgfpathclose%
\pgfusepath{fill}%
\end{pgfscope}%
\begin{pgfscope}%
\pgfpathrectangle{\pgfqpoint{0.017500in}{0.017500in}}{\pgfqpoint{6.218282in}{3.580372in}} %
\pgfusepath{clip}%
\pgfsetbuttcap%
\pgfsetroundjoin%
\definecolor{currentfill}{rgb}{0.243520,0.319189,0.771672}%
\pgfsetfillcolor{currentfill}%
\pgfsetlinewidth{0.000000pt}%
\definecolor{currentstroke}{rgb}{0.000000,0.000000,0.000000}%
\pgfsetstrokecolor{currentstroke}%
\pgfsetdash{}{0pt}%
\pgfpathmoveto{\pgfqpoint{3.236097in}{0.721171in}}%
\pgfpathlineto{\pgfqpoint{3.502717in}{2.091450in}}%
\pgfpathlineto{\pgfqpoint{3.236097in}{0.721171in}}%
\pgfpathclose%
\pgfusepath{fill}%
\end{pgfscope}%
\begin{pgfscope}%
\pgfpathrectangle{\pgfqpoint{0.017500in}{0.017500in}}{\pgfqpoint{6.218282in}{3.580372in}} %
\pgfusepath{clip}%
\pgfsetbuttcap%
\pgfsetroundjoin%
\definecolor{currentfill}{rgb}{0.243520,0.319189,0.771672}%
\pgfsetfillcolor{currentfill}%
\pgfsetlinewidth{0.000000pt}%
\definecolor{currentstroke}{rgb}{0.000000,0.000000,0.000000}%
\pgfsetstrokecolor{currentstroke}%
\pgfsetdash{}{0pt}%
\pgfpathmoveto{\pgfqpoint{3.236097in}{0.721171in}}%
\pgfpathlineto{\pgfqpoint{3.502717in}{2.091450in}}%
\pgfpathlineto{\pgfqpoint{3.236097in}{0.721171in}}%
\pgfpathclose%
\pgfusepath{fill}%
\end{pgfscope}%
\begin{pgfscope}%
\pgfpathrectangle{\pgfqpoint{0.017500in}{0.017500in}}{\pgfqpoint{6.218282in}{3.580372in}} %
\pgfusepath{clip}%
\pgfsetbuttcap%
\pgfsetroundjoin%
\definecolor{currentfill}{rgb}{0.243520,0.319189,0.771672}%
\pgfsetfillcolor{currentfill}%
\pgfsetlinewidth{0.000000pt}%
\definecolor{currentstroke}{rgb}{0.000000,0.000000,0.000000}%
\pgfsetstrokecolor{currentstroke}%
\pgfsetdash{}{0pt}%
\pgfpathmoveto{\pgfqpoint{3.236097in}{0.721171in}}%
\pgfpathlineto{\pgfqpoint{3.502717in}{2.091450in}}%
\pgfpathlineto{\pgfqpoint{3.236097in}{0.721171in}}%
\pgfpathclose%
\pgfusepath{fill}%
\end{pgfscope}%
\begin{pgfscope}%
\pgfpathrectangle{\pgfqpoint{0.017500in}{0.017500in}}{\pgfqpoint{6.218282in}{3.580372in}} %
\pgfusepath{clip}%
\pgfsetbuttcap%
\pgfsetroundjoin%
\definecolor{currentfill}{rgb}{0.243520,0.319189,0.771672}%
\pgfsetfillcolor{currentfill}%
\pgfsetlinewidth{0.000000pt}%
\definecolor{currentstroke}{rgb}{0.000000,0.000000,0.000000}%
\pgfsetstrokecolor{currentstroke}%
\pgfsetdash{}{0pt}%
\pgfpathmoveto{\pgfqpoint{3.236097in}{0.721171in}}%
\pgfpathlineto{\pgfqpoint{3.502717in}{2.091450in}}%
\pgfpathlineto{\pgfqpoint{3.236097in}{0.721171in}}%
\pgfpathclose%
\pgfusepath{fill}%
\end{pgfscope}%
\begin{pgfscope}%
\pgfpathrectangle{\pgfqpoint{0.017500in}{0.017500in}}{\pgfqpoint{6.218282in}{3.580372in}} %
\pgfusepath{clip}%
\pgfsetbuttcap%
\pgfsetroundjoin%
\definecolor{currentfill}{rgb}{0.243520,0.319189,0.771672}%
\pgfsetfillcolor{currentfill}%
\pgfsetlinewidth{0.000000pt}%
\definecolor{currentstroke}{rgb}{0.000000,0.000000,0.000000}%
\pgfsetstrokecolor{currentstroke}%
\pgfsetdash{}{0pt}%
\pgfpathmoveto{\pgfqpoint{3.236097in}{0.721171in}}%
\pgfpathlineto{\pgfqpoint{3.502717in}{2.091450in}}%
\pgfpathlineto{\pgfqpoint{3.236097in}{0.721171in}}%
\pgfpathclose%
\pgfusepath{fill}%
\end{pgfscope}%
\begin{pgfscope}%
\pgfpathrectangle{\pgfqpoint{0.017500in}{0.017500in}}{\pgfqpoint{6.218282in}{3.580372in}} %
\pgfusepath{clip}%
\pgfsetbuttcap%
\pgfsetroundjoin%
\definecolor{currentfill}{rgb}{0.243520,0.319189,0.771672}%
\pgfsetfillcolor{currentfill}%
\pgfsetlinewidth{0.000000pt}%
\definecolor{currentstroke}{rgb}{0.000000,0.000000,0.000000}%
\pgfsetstrokecolor{currentstroke}%
\pgfsetdash{}{0pt}%
\pgfpathmoveto{\pgfqpoint{3.236097in}{0.721171in}}%
\pgfpathlineto{\pgfqpoint{3.502717in}{2.091450in}}%
\pgfpathlineto{\pgfqpoint{3.236097in}{0.721171in}}%
\pgfpathclose%
\pgfusepath{fill}%
\end{pgfscope}%
\begin{pgfscope}%
\pgfpathrectangle{\pgfqpoint{0.017500in}{0.017500in}}{\pgfqpoint{6.218282in}{3.580372in}} %
\pgfusepath{clip}%
\pgfsetbuttcap%
\pgfsetroundjoin%
\definecolor{currentfill}{rgb}{0.243520,0.319189,0.771672}%
\pgfsetfillcolor{currentfill}%
\pgfsetlinewidth{0.000000pt}%
\definecolor{currentstroke}{rgb}{0.000000,0.000000,0.000000}%
\pgfsetstrokecolor{currentstroke}%
\pgfsetdash{}{0pt}%
\pgfpathmoveto{\pgfqpoint{3.236097in}{0.721171in}}%
\pgfpathlineto{\pgfqpoint{3.502717in}{2.091450in}}%
\pgfpathlineto{\pgfqpoint{3.236097in}{0.721171in}}%
\pgfpathclose%
\pgfusepath{fill}%
\end{pgfscope}%
\begin{pgfscope}%
\pgfpathrectangle{\pgfqpoint{0.017500in}{0.017500in}}{\pgfqpoint{6.218282in}{3.580372in}} %
\pgfusepath{clip}%
\pgfsetbuttcap%
\pgfsetroundjoin%
\definecolor{currentfill}{rgb}{0.243520,0.319189,0.771672}%
\pgfsetfillcolor{currentfill}%
\pgfsetlinewidth{0.000000pt}%
\definecolor{currentstroke}{rgb}{0.000000,0.000000,0.000000}%
\pgfsetstrokecolor{currentstroke}%
\pgfsetdash{}{0pt}%
\pgfpathmoveto{\pgfqpoint{3.236097in}{0.721171in}}%
\pgfpathlineto{\pgfqpoint{3.502717in}{2.091450in}}%
\pgfpathlineto{\pgfqpoint{3.236097in}{0.721171in}}%
\pgfpathclose%
\pgfusepath{fill}%
\end{pgfscope}%
\begin{pgfscope}%
\pgfpathrectangle{\pgfqpoint{0.017500in}{0.017500in}}{\pgfqpoint{6.218282in}{3.580372in}} %
\pgfusepath{clip}%
\pgfsetbuttcap%
\pgfsetroundjoin%
\definecolor{currentfill}{rgb}{0.243520,0.319189,0.771672}%
\pgfsetfillcolor{currentfill}%
\pgfsetlinewidth{0.000000pt}%
\definecolor{currentstroke}{rgb}{0.000000,0.000000,0.000000}%
\pgfsetstrokecolor{currentstroke}%
\pgfsetdash{}{0pt}%
\pgfpathmoveto{\pgfqpoint{3.236097in}{0.721171in}}%
\pgfpathlineto{\pgfqpoint{3.502717in}{2.091450in}}%
\pgfpathlineto{\pgfqpoint{3.236097in}{0.721171in}}%
\pgfpathclose%
\pgfusepath{fill}%
\end{pgfscope}%
\begin{pgfscope}%
\pgfpathrectangle{\pgfqpoint{0.017500in}{0.017500in}}{\pgfqpoint{6.218282in}{3.580372in}} %
\pgfusepath{clip}%
\pgfsetbuttcap%
\pgfsetroundjoin%
\definecolor{currentfill}{rgb}{0.243520,0.319189,0.771672}%
\pgfsetfillcolor{currentfill}%
\pgfsetlinewidth{0.000000pt}%
\definecolor{currentstroke}{rgb}{0.000000,0.000000,0.000000}%
\pgfsetstrokecolor{currentstroke}%
\pgfsetdash{}{0pt}%
\pgfpathmoveto{\pgfqpoint{3.236097in}{0.721171in}}%
\pgfpathlineto{\pgfqpoint{3.502717in}{2.091450in}}%
\pgfpathlineto{\pgfqpoint{3.236097in}{0.721171in}}%
\pgfpathclose%
\pgfusepath{fill}%
\end{pgfscope}%
\begin{pgfscope}%
\pgfpathrectangle{\pgfqpoint{0.017500in}{0.017500in}}{\pgfqpoint{6.218282in}{3.580372in}} %
\pgfusepath{clip}%
\pgfsetbuttcap%
\pgfsetroundjoin%
\definecolor{currentfill}{rgb}{0.243520,0.319189,0.771672}%
\pgfsetfillcolor{currentfill}%
\pgfsetlinewidth{0.000000pt}%
\definecolor{currentstroke}{rgb}{0.000000,0.000000,0.000000}%
\pgfsetstrokecolor{currentstroke}%
\pgfsetdash{}{0pt}%
\pgfpathmoveto{\pgfqpoint{3.236097in}{0.721171in}}%
\pgfpathlineto{\pgfqpoint{3.502717in}{2.091450in}}%
\pgfpathlineto{\pgfqpoint{3.236097in}{0.721171in}}%
\pgfpathclose%
\pgfusepath{fill}%
\end{pgfscope}%
\begin{pgfscope}%
\pgfpathrectangle{\pgfqpoint{0.017500in}{0.017500in}}{\pgfqpoint{6.218282in}{3.580372in}} %
\pgfusepath{clip}%
\pgfsetbuttcap%
\pgfsetroundjoin%
\definecolor{currentfill}{rgb}{0.243520,0.319189,0.771672}%
\pgfsetfillcolor{currentfill}%
\pgfsetlinewidth{0.000000pt}%
\definecolor{currentstroke}{rgb}{0.000000,0.000000,0.000000}%
\pgfsetstrokecolor{currentstroke}%
\pgfsetdash{}{0pt}%
\pgfpathmoveto{\pgfqpoint{3.236097in}{0.721171in}}%
\pgfpathlineto{\pgfqpoint{3.502717in}{2.091450in}}%
\pgfpathlineto{\pgfqpoint{3.236097in}{0.721171in}}%
\pgfpathclose%
\pgfusepath{fill}%
\end{pgfscope}%
\begin{pgfscope}%
\pgfpathrectangle{\pgfqpoint{0.017500in}{0.017500in}}{\pgfqpoint{6.218282in}{3.580372in}} %
\pgfusepath{clip}%
\pgfsetbuttcap%
\pgfsetroundjoin%
\definecolor{currentfill}{rgb}{0.243520,0.319189,0.771672}%
\pgfsetfillcolor{currentfill}%
\pgfsetlinewidth{0.000000pt}%
\definecolor{currentstroke}{rgb}{0.000000,0.000000,0.000000}%
\pgfsetstrokecolor{currentstroke}%
\pgfsetdash{}{0pt}%
\pgfpathmoveto{\pgfqpoint{3.236097in}{0.721171in}}%
\pgfpathlineto{\pgfqpoint{3.502717in}{2.091450in}}%
\pgfpathlineto{\pgfqpoint{3.236097in}{0.721171in}}%
\pgfpathclose%
\pgfusepath{fill}%
\end{pgfscope}%
\begin{pgfscope}%
\pgfpathrectangle{\pgfqpoint{0.017500in}{0.017500in}}{\pgfqpoint{6.218282in}{3.580372in}} %
\pgfusepath{clip}%
\pgfsetbuttcap%
\pgfsetroundjoin%
\definecolor{currentfill}{rgb}{0.243520,0.319189,0.771672}%
\pgfsetfillcolor{currentfill}%
\pgfsetlinewidth{0.000000pt}%
\definecolor{currentstroke}{rgb}{0.000000,0.000000,0.000000}%
\pgfsetstrokecolor{currentstroke}%
\pgfsetdash{}{0pt}%
\pgfpathmoveto{\pgfqpoint{3.236097in}{0.721171in}}%
\pgfpathlineto{\pgfqpoint{3.502717in}{2.091450in}}%
\pgfpathlineto{\pgfqpoint{3.236097in}{0.721171in}}%
\pgfpathclose%
\pgfusepath{fill}%
\end{pgfscope}%
\begin{pgfscope}%
\pgfpathrectangle{\pgfqpoint{0.017500in}{0.017500in}}{\pgfqpoint{6.218282in}{3.580372in}} %
\pgfusepath{clip}%
\pgfsetbuttcap%
\pgfsetroundjoin%
\definecolor{currentfill}{rgb}{0.243520,0.319189,0.771672}%
\pgfsetfillcolor{currentfill}%
\pgfsetlinewidth{0.000000pt}%
\definecolor{currentstroke}{rgb}{0.000000,0.000000,0.000000}%
\pgfsetstrokecolor{currentstroke}%
\pgfsetdash{}{0pt}%
\pgfpathmoveto{\pgfqpoint{3.236097in}{0.721171in}}%
\pgfpathlineto{\pgfqpoint{3.502717in}{2.091450in}}%
\pgfpathlineto{\pgfqpoint{3.236097in}{0.721171in}}%
\pgfpathclose%
\pgfusepath{fill}%
\end{pgfscope}%
\begin{pgfscope}%
\pgfpathrectangle{\pgfqpoint{0.017500in}{0.017500in}}{\pgfqpoint{6.218282in}{3.580372in}} %
\pgfusepath{clip}%
\pgfsetbuttcap%
\pgfsetroundjoin%
\definecolor{currentfill}{rgb}{0.243520,0.319189,0.771672}%
\pgfsetfillcolor{currentfill}%
\pgfsetlinewidth{0.000000pt}%
\definecolor{currentstroke}{rgb}{0.000000,0.000000,0.000000}%
\pgfsetstrokecolor{currentstroke}%
\pgfsetdash{}{0pt}%
\pgfpathmoveto{\pgfqpoint{3.236097in}{0.721171in}}%
\pgfpathlineto{\pgfqpoint{3.502717in}{2.091450in}}%
\pgfpathlineto{\pgfqpoint{3.236097in}{0.721171in}}%
\pgfpathclose%
\pgfusepath{fill}%
\end{pgfscope}%
\begin{pgfscope}%
\pgfpathrectangle{\pgfqpoint{0.017500in}{0.017500in}}{\pgfqpoint{6.218282in}{3.580372in}} %
\pgfusepath{clip}%
\pgfsetbuttcap%
\pgfsetroundjoin%
\definecolor{currentfill}{rgb}{0.243520,0.319189,0.771672}%
\pgfsetfillcolor{currentfill}%
\pgfsetlinewidth{0.000000pt}%
\definecolor{currentstroke}{rgb}{0.000000,0.000000,0.000000}%
\pgfsetstrokecolor{currentstroke}%
\pgfsetdash{}{0pt}%
\pgfpathmoveto{\pgfqpoint{3.236097in}{0.721171in}}%
\pgfpathlineto{\pgfqpoint{3.502717in}{2.091450in}}%
\pgfpathlineto{\pgfqpoint{3.236097in}{0.721171in}}%
\pgfpathclose%
\pgfusepath{fill}%
\end{pgfscope}%
\begin{pgfscope}%
\pgfpathrectangle{\pgfqpoint{0.017500in}{0.017500in}}{\pgfqpoint{6.218282in}{3.580372in}} %
\pgfusepath{clip}%
\pgfsetbuttcap%
\pgfsetroundjoin%
\definecolor{currentfill}{rgb}{0.880896,0.402331,0.317115}%
\pgfsetfillcolor{currentfill}%
\pgfsetlinewidth{0.000000pt}%
\definecolor{currentstroke}{rgb}{0.000000,0.000000,0.000000}%
\pgfsetstrokecolor{currentstroke}%
\pgfsetdash{}{0pt}%
\pgfpathmoveto{\pgfqpoint{2.590029in}{2.232729in}}%
\pgfpathlineto{\pgfqpoint{2.855457in}{2.456853in}}%
\pgfpathlineto{\pgfqpoint{2.590029in}{2.232729in}}%
\pgfpathclose%
\pgfusepath{fill}%
\end{pgfscope}%
\begin{pgfscope}%
\pgfpathrectangle{\pgfqpoint{0.017500in}{0.017500in}}{\pgfqpoint{6.218282in}{3.580372in}} %
\pgfusepath{clip}%
\pgfsetbuttcap%
\pgfsetroundjoin%
\definecolor{currentfill}{rgb}{0.880896,0.402331,0.317115}%
\pgfsetfillcolor{currentfill}%
\pgfsetlinewidth{0.000000pt}%
\definecolor{currentstroke}{rgb}{0.000000,0.000000,0.000000}%
\pgfsetstrokecolor{currentstroke}%
\pgfsetdash{}{0pt}%
\pgfpathmoveto{\pgfqpoint{2.590029in}{2.232729in}}%
\pgfpathlineto{\pgfqpoint{2.855457in}{2.456853in}}%
\pgfpathlineto{\pgfqpoint{2.590029in}{2.232729in}}%
\pgfpathclose%
\pgfusepath{fill}%
\end{pgfscope}%
\begin{pgfscope}%
\pgfpathrectangle{\pgfqpoint{0.017500in}{0.017500in}}{\pgfqpoint{6.218282in}{3.580372in}} %
\pgfusepath{clip}%
\pgfsetbuttcap%
\pgfsetroundjoin%
\definecolor{currentfill}{rgb}{0.880896,0.402331,0.317115}%
\pgfsetfillcolor{currentfill}%
\pgfsetlinewidth{0.000000pt}%
\definecolor{currentstroke}{rgb}{0.000000,0.000000,0.000000}%
\pgfsetstrokecolor{currentstroke}%
\pgfsetdash{}{0pt}%
\pgfpathmoveto{\pgfqpoint{2.590029in}{2.232729in}}%
\pgfpathlineto{\pgfqpoint{2.855457in}{2.456853in}}%
\pgfpathlineto{\pgfqpoint{2.590029in}{2.232729in}}%
\pgfpathclose%
\pgfusepath{fill}%
\end{pgfscope}%
\begin{pgfscope}%
\pgfpathrectangle{\pgfqpoint{0.017500in}{0.017500in}}{\pgfqpoint{6.218282in}{3.580372in}} %
\pgfusepath{clip}%
\pgfsetbuttcap%
\pgfsetroundjoin%
\definecolor{currentfill}{rgb}{0.880896,0.402331,0.317115}%
\pgfsetfillcolor{currentfill}%
\pgfsetlinewidth{0.000000pt}%
\definecolor{currentstroke}{rgb}{0.000000,0.000000,0.000000}%
\pgfsetstrokecolor{currentstroke}%
\pgfsetdash{}{0pt}%
\pgfpathmoveto{\pgfqpoint{2.590029in}{2.232729in}}%
\pgfpathlineto{\pgfqpoint{2.855457in}{2.456853in}}%
\pgfpathlineto{\pgfqpoint{2.590029in}{2.232729in}}%
\pgfpathclose%
\pgfusepath{fill}%
\end{pgfscope}%
\begin{pgfscope}%
\pgfpathrectangle{\pgfqpoint{0.017500in}{0.017500in}}{\pgfqpoint{6.218282in}{3.580372in}} %
\pgfusepath{clip}%
\pgfsetbuttcap%
\pgfsetroundjoin%
\definecolor{currentfill}{rgb}{0.880896,0.402331,0.317115}%
\pgfsetfillcolor{currentfill}%
\pgfsetlinewidth{0.000000pt}%
\definecolor{currentstroke}{rgb}{0.000000,0.000000,0.000000}%
\pgfsetstrokecolor{currentstroke}%
\pgfsetdash{}{0pt}%
\pgfpathmoveto{\pgfqpoint{2.590029in}{2.232729in}}%
\pgfpathlineto{\pgfqpoint{2.855457in}{2.456853in}}%
\pgfpathlineto{\pgfqpoint{2.590029in}{2.232729in}}%
\pgfpathclose%
\pgfusepath{fill}%
\end{pgfscope}%
\begin{pgfscope}%
\pgfpathrectangle{\pgfqpoint{0.017500in}{0.017500in}}{\pgfqpoint{6.218282in}{3.580372in}} %
\pgfusepath{clip}%
\pgfsetbuttcap%
\pgfsetroundjoin%
\definecolor{currentfill}{rgb}{0.880896,0.402331,0.317115}%
\pgfsetfillcolor{currentfill}%
\pgfsetlinewidth{0.000000pt}%
\definecolor{currentstroke}{rgb}{0.000000,0.000000,0.000000}%
\pgfsetstrokecolor{currentstroke}%
\pgfsetdash{}{0pt}%
\pgfpathmoveto{\pgfqpoint{2.590029in}{2.232729in}}%
\pgfpathlineto{\pgfqpoint{2.855457in}{2.456853in}}%
\pgfpathlineto{\pgfqpoint{2.590029in}{2.232729in}}%
\pgfpathclose%
\pgfusepath{fill}%
\end{pgfscope}%
\begin{pgfscope}%
\pgfpathrectangle{\pgfqpoint{0.017500in}{0.017500in}}{\pgfqpoint{6.218282in}{3.580372in}} %
\pgfusepath{clip}%
\pgfsetbuttcap%
\pgfsetroundjoin%
\definecolor{currentfill}{rgb}{0.880896,0.402331,0.317115}%
\pgfsetfillcolor{currentfill}%
\pgfsetlinewidth{0.000000pt}%
\definecolor{currentstroke}{rgb}{0.000000,0.000000,0.000000}%
\pgfsetstrokecolor{currentstroke}%
\pgfsetdash{}{0pt}%
\pgfpathmoveto{\pgfqpoint{2.590029in}{2.232729in}}%
\pgfpathlineto{\pgfqpoint{2.855457in}{2.456853in}}%
\pgfpathlineto{\pgfqpoint{2.590029in}{2.232729in}}%
\pgfpathclose%
\pgfusepath{fill}%
\end{pgfscope}%
\begin{pgfscope}%
\pgfpathrectangle{\pgfqpoint{0.017500in}{0.017500in}}{\pgfqpoint{6.218282in}{3.580372in}} %
\pgfusepath{clip}%
\pgfsetbuttcap%
\pgfsetroundjoin%
\definecolor{currentfill}{rgb}{0.880896,0.402331,0.317115}%
\pgfsetfillcolor{currentfill}%
\pgfsetlinewidth{0.000000pt}%
\definecolor{currentstroke}{rgb}{0.000000,0.000000,0.000000}%
\pgfsetstrokecolor{currentstroke}%
\pgfsetdash{}{0pt}%
\pgfpathmoveto{\pgfqpoint{2.590029in}{2.232729in}}%
\pgfpathlineto{\pgfqpoint{2.855457in}{2.456853in}}%
\pgfpathlineto{\pgfqpoint{2.590029in}{2.232729in}}%
\pgfpathclose%
\pgfusepath{fill}%
\end{pgfscope}%
\begin{pgfscope}%
\pgfpathrectangle{\pgfqpoint{0.017500in}{0.017500in}}{\pgfqpoint{6.218282in}{3.580372in}} %
\pgfusepath{clip}%
\pgfsetbuttcap%
\pgfsetroundjoin%
\definecolor{currentfill}{rgb}{0.880896,0.402331,0.317115}%
\pgfsetfillcolor{currentfill}%
\pgfsetlinewidth{0.000000pt}%
\definecolor{currentstroke}{rgb}{0.000000,0.000000,0.000000}%
\pgfsetstrokecolor{currentstroke}%
\pgfsetdash{}{0pt}%
\pgfpathmoveto{\pgfqpoint{2.590029in}{2.232729in}}%
\pgfpathlineto{\pgfqpoint{2.855457in}{2.456853in}}%
\pgfpathlineto{\pgfqpoint{2.590029in}{2.232729in}}%
\pgfpathclose%
\pgfusepath{fill}%
\end{pgfscope}%
\begin{pgfscope}%
\pgfpathrectangle{\pgfqpoint{0.017500in}{0.017500in}}{\pgfqpoint{6.218282in}{3.580372in}} %
\pgfusepath{clip}%
\pgfsetbuttcap%
\pgfsetroundjoin%
\definecolor{currentfill}{rgb}{0.880896,0.402331,0.317115}%
\pgfsetfillcolor{currentfill}%
\pgfsetlinewidth{0.000000pt}%
\definecolor{currentstroke}{rgb}{0.000000,0.000000,0.000000}%
\pgfsetstrokecolor{currentstroke}%
\pgfsetdash{}{0pt}%
\pgfpathmoveto{\pgfqpoint{2.590029in}{2.232729in}}%
\pgfpathlineto{\pgfqpoint{2.855457in}{2.456853in}}%
\pgfpathlineto{\pgfqpoint{2.590029in}{2.232729in}}%
\pgfpathclose%
\pgfusepath{fill}%
\end{pgfscope}%
\begin{pgfscope}%
\pgfpathrectangle{\pgfqpoint{0.017500in}{0.017500in}}{\pgfqpoint{6.218282in}{3.580372in}} %
\pgfusepath{clip}%
\pgfsetbuttcap%
\pgfsetroundjoin%
\definecolor{currentfill}{rgb}{0.880896,0.402331,0.317115}%
\pgfsetfillcolor{currentfill}%
\pgfsetlinewidth{0.000000pt}%
\definecolor{currentstroke}{rgb}{0.000000,0.000000,0.000000}%
\pgfsetstrokecolor{currentstroke}%
\pgfsetdash{}{0pt}%
\pgfpathmoveto{\pgfqpoint{2.590029in}{2.232729in}}%
\pgfpathlineto{\pgfqpoint{2.855457in}{2.456853in}}%
\pgfpathlineto{\pgfqpoint{2.590029in}{2.232729in}}%
\pgfpathclose%
\pgfusepath{fill}%
\end{pgfscope}%
\begin{pgfscope}%
\pgfpathrectangle{\pgfqpoint{0.017500in}{0.017500in}}{\pgfqpoint{6.218282in}{3.580372in}} %
\pgfusepath{clip}%
\pgfsetbuttcap%
\pgfsetroundjoin%
\definecolor{currentfill}{rgb}{0.880896,0.402331,0.317115}%
\pgfsetfillcolor{currentfill}%
\pgfsetlinewidth{0.000000pt}%
\definecolor{currentstroke}{rgb}{0.000000,0.000000,0.000000}%
\pgfsetstrokecolor{currentstroke}%
\pgfsetdash{}{0pt}%
\pgfpathmoveto{\pgfqpoint{2.590029in}{2.232729in}}%
\pgfpathlineto{\pgfqpoint{2.855457in}{2.456853in}}%
\pgfpathlineto{\pgfqpoint{2.590029in}{2.232729in}}%
\pgfpathclose%
\pgfusepath{fill}%
\end{pgfscope}%
\begin{pgfscope}%
\pgfpathrectangle{\pgfqpoint{0.017500in}{0.017500in}}{\pgfqpoint{6.218282in}{3.580372in}} %
\pgfusepath{clip}%
\pgfsetbuttcap%
\pgfsetroundjoin%
\definecolor{currentfill}{rgb}{0.880896,0.402331,0.317115}%
\pgfsetfillcolor{currentfill}%
\pgfsetlinewidth{0.000000pt}%
\definecolor{currentstroke}{rgb}{0.000000,0.000000,0.000000}%
\pgfsetstrokecolor{currentstroke}%
\pgfsetdash{}{0pt}%
\pgfpathmoveto{\pgfqpoint{2.590029in}{2.232729in}}%
\pgfpathlineto{\pgfqpoint{2.855457in}{2.456853in}}%
\pgfpathlineto{\pgfqpoint{2.590029in}{2.232729in}}%
\pgfpathclose%
\pgfusepath{fill}%
\end{pgfscope}%
\begin{pgfscope}%
\pgfpathrectangle{\pgfqpoint{0.017500in}{0.017500in}}{\pgfqpoint{6.218282in}{3.580372in}} %
\pgfusepath{clip}%
\pgfsetbuttcap%
\pgfsetroundjoin%
\definecolor{currentfill}{rgb}{0.880896,0.402331,0.317115}%
\pgfsetfillcolor{currentfill}%
\pgfsetlinewidth{0.000000pt}%
\definecolor{currentstroke}{rgb}{0.000000,0.000000,0.000000}%
\pgfsetstrokecolor{currentstroke}%
\pgfsetdash{}{0pt}%
\pgfpathmoveto{\pgfqpoint{2.590029in}{2.232729in}}%
\pgfpathlineto{\pgfqpoint{2.855457in}{2.456853in}}%
\pgfpathlineto{\pgfqpoint{2.590029in}{2.232729in}}%
\pgfpathclose%
\pgfusepath{fill}%
\end{pgfscope}%
\begin{pgfscope}%
\pgfpathrectangle{\pgfqpoint{0.017500in}{0.017500in}}{\pgfqpoint{6.218282in}{3.580372in}} %
\pgfusepath{clip}%
\pgfsetbuttcap%
\pgfsetroundjoin%
\definecolor{currentfill}{rgb}{0.880896,0.402331,0.317115}%
\pgfsetfillcolor{currentfill}%
\pgfsetlinewidth{0.000000pt}%
\definecolor{currentstroke}{rgb}{0.000000,0.000000,0.000000}%
\pgfsetstrokecolor{currentstroke}%
\pgfsetdash{}{0pt}%
\pgfpathmoveto{\pgfqpoint{2.590029in}{2.232729in}}%
\pgfpathlineto{\pgfqpoint{2.855457in}{2.456853in}}%
\pgfpathlineto{\pgfqpoint{2.590029in}{2.232729in}}%
\pgfpathclose%
\pgfusepath{fill}%
\end{pgfscope}%
\begin{pgfscope}%
\pgfpathrectangle{\pgfqpoint{0.017500in}{0.017500in}}{\pgfqpoint{6.218282in}{3.580372in}} %
\pgfusepath{clip}%
\pgfsetbuttcap%
\pgfsetroundjoin%
\definecolor{currentfill}{rgb}{0.880896,0.402331,0.317115}%
\pgfsetfillcolor{currentfill}%
\pgfsetlinewidth{0.000000pt}%
\definecolor{currentstroke}{rgb}{0.000000,0.000000,0.000000}%
\pgfsetstrokecolor{currentstroke}%
\pgfsetdash{}{0pt}%
\pgfpathmoveto{\pgfqpoint{2.590029in}{2.232729in}}%
\pgfpathlineto{\pgfqpoint{2.855457in}{2.456853in}}%
\pgfpathlineto{\pgfqpoint{2.590029in}{2.232729in}}%
\pgfpathclose%
\pgfusepath{fill}%
\end{pgfscope}%
\begin{pgfscope}%
\pgfpathrectangle{\pgfqpoint{0.017500in}{0.017500in}}{\pgfqpoint{6.218282in}{3.580372in}} %
\pgfusepath{clip}%
\pgfsetbuttcap%
\pgfsetroundjoin%
\definecolor{currentfill}{rgb}{0.880896,0.402331,0.317115}%
\pgfsetfillcolor{currentfill}%
\pgfsetlinewidth{0.000000pt}%
\definecolor{currentstroke}{rgb}{0.000000,0.000000,0.000000}%
\pgfsetstrokecolor{currentstroke}%
\pgfsetdash{}{0pt}%
\pgfpathmoveto{\pgfqpoint{2.590029in}{2.232729in}}%
\pgfpathlineto{\pgfqpoint{2.855457in}{2.456853in}}%
\pgfpathlineto{\pgfqpoint{2.590029in}{2.232729in}}%
\pgfpathclose%
\pgfusepath{fill}%
\end{pgfscope}%
\begin{pgfscope}%
\pgfpathrectangle{\pgfqpoint{0.017500in}{0.017500in}}{\pgfqpoint{6.218282in}{3.580372in}} %
\pgfusepath{clip}%
\pgfsetbuttcap%
\pgfsetroundjoin%
\definecolor{currentfill}{rgb}{0.880896,0.402331,0.317115}%
\pgfsetfillcolor{currentfill}%
\pgfsetlinewidth{0.000000pt}%
\definecolor{currentstroke}{rgb}{0.000000,0.000000,0.000000}%
\pgfsetstrokecolor{currentstroke}%
\pgfsetdash{}{0pt}%
\pgfpathmoveto{\pgfqpoint{2.590029in}{2.232729in}}%
\pgfpathlineto{\pgfqpoint{2.855457in}{2.456853in}}%
\pgfpathlineto{\pgfqpoint{2.590029in}{2.232729in}}%
\pgfpathclose%
\pgfusepath{fill}%
\end{pgfscope}%
\begin{pgfscope}%
\pgfpathrectangle{\pgfqpoint{0.017500in}{0.017500in}}{\pgfqpoint{6.218282in}{3.580372in}} %
\pgfusepath{clip}%
\pgfsetbuttcap%
\pgfsetroundjoin%
\definecolor{currentfill}{rgb}{0.880896,0.402331,0.317115}%
\pgfsetfillcolor{currentfill}%
\pgfsetlinewidth{0.000000pt}%
\definecolor{currentstroke}{rgb}{0.000000,0.000000,0.000000}%
\pgfsetstrokecolor{currentstroke}%
\pgfsetdash{}{0pt}%
\pgfpathmoveto{\pgfqpoint{2.590029in}{2.232729in}}%
\pgfpathlineto{\pgfqpoint{2.855457in}{2.456853in}}%
\pgfpathlineto{\pgfqpoint{2.590029in}{2.232729in}}%
\pgfpathclose%
\pgfusepath{fill}%
\end{pgfscope}%
\begin{pgfscope}%
\pgfpathrectangle{\pgfqpoint{0.017500in}{0.017500in}}{\pgfqpoint{6.218282in}{3.580372in}} %
\pgfusepath{clip}%
\pgfsetbuttcap%
\pgfsetroundjoin%
\definecolor{currentfill}{rgb}{0.880896,0.402331,0.317115}%
\pgfsetfillcolor{currentfill}%
\pgfsetlinewidth{0.000000pt}%
\definecolor{currentstroke}{rgb}{0.000000,0.000000,0.000000}%
\pgfsetstrokecolor{currentstroke}%
\pgfsetdash{}{0pt}%
\pgfpathmoveto{\pgfqpoint{2.590029in}{2.232729in}}%
\pgfpathlineto{\pgfqpoint{2.855457in}{2.456853in}}%
\pgfpathlineto{\pgfqpoint{2.590029in}{2.232729in}}%
\pgfpathclose%
\pgfusepath{fill}%
\end{pgfscope}%
\begin{pgfscope}%
\pgfpathrectangle{\pgfqpoint{0.017500in}{0.017500in}}{\pgfqpoint{6.218282in}{3.580372in}} %
\pgfusepath{clip}%
\pgfsetbuttcap%
\pgfsetroundjoin%
\definecolor{currentfill}{rgb}{0.880896,0.402331,0.317115}%
\pgfsetfillcolor{currentfill}%
\pgfsetlinewidth{0.000000pt}%
\definecolor{currentstroke}{rgb}{0.000000,0.000000,0.000000}%
\pgfsetstrokecolor{currentstroke}%
\pgfsetdash{}{0pt}%
\pgfpathmoveto{\pgfqpoint{2.590029in}{2.232729in}}%
\pgfpathlineto{\pgfqpoint{2.855457in}{2.456853in}}%
\pgfpathlineto{\pgfqpoint{2.590029in}{2.232729in}}%
\pgfpathclose%
\pgfusepath{fill}%
\end{pgfscope}%
\begin{pgfscope}%
\pgfpathrectangle{\pgfqpoint{0.017500in}{0.017500in}}{\pgfqpoint{6.218282in}{3.580372in}} %
\pgfusepath{clip}%
\pgfsetbuttcap%
\pgfsetroundjoin%
\definecolor{currentfill}{rgb}{0.880896,0.402331,0.317115}%
\pgfsetfillcolor{currentfill}%
\pgfsetlinewidth{0.000000pt}%
\definecolor{currentstroke}{rgb}{0.000000,0.000000,0.000000}%
\pgfsetstrokecolor{currentstroke}%
\pgfsetdash{}{0pt}%
\pgfpathmoveto{\pgfqpoint{2.590029in}{2.232729in}}%
\pgfpathlineto{\pgfqpoint{2.855457in}{2.456853in}}%
\pgfpathlineto{\pgfqpoint{2.590029in}{2.232729in}}%
\pgfpathclose%
\pgfusepath{fill}%
\end{pgfscope}%
\begin{pgfscope}%
\pgfpathrectangle{\pgfqpoint{0.017500in}{0.017500in}}{\pgfqpoint{6.218282in}{3.580372in}} %
\pgfusepath{clip}%
\pgfsetbuttcap%
\pgfsetroundjoin%
\definecolor{currentfill}{rgb}{0.880896,0.402331,0.317115}%
\pgfsetfillcolor{currentfill}%
\pgfsetlinewidth{0.000000pt}%
\definecolor{currentstroke}{rgb}{0.000000,0.000000,0.000000}%
\pgfsetstrokecolor{currentstroke}%
\pgfsetdash{}{0pt}%
\pgfpathmoveto{\pgfqpoint{2.590029in}{2.232729in}}%
\pgfpathlineto{\pgfqpoint{2.855457in}{2.456853in}}%
\pgfpathlineto{\pgfqpoint{2.590029in}{2.232729in}}%
\pgfpathclose%
\pgfusepath{fill}%
\end{pgfscope}%
\begin{pgfscope}%
\pgfpathrectangle{\pgfqpoint{0.017500in}{0.017500in}}{\pgfqpoint{6.218282in}{3.580372in}} %
\pgfusepath{clip}%
\pgfsetbuttcap%
\pgfsetroundjoin%
\definecolor{currentfill}{rgb}{0.880896,0.402331,0.317115}%
\pgfsetfillcolor{currentfill}%
\pgfsetlinewidth{0.000000pt}%
\definecolor{currentstroke}{rgb}{0.000000,0.000000,0.000000}%
\pgfsetstrokecolor{currentstroke}%
\pgfsetdash{}{0pt}%
\pgfpathmoveto{\pgfqpoint{2.590029in}{2.232729in}}%
\pgfpathlineto{\pgfqpoint{2.855457in}{2.456853in}}%
\pgfpathlineto{\pgfqpoint{2.590029in}{2.232729in}}%
\pgfpathclose%
\pgfusepath{fill}%
\end{pgfscope}%
\begin{pgfscope}%
\pgfpathrectangle{\pgfqpoint{0.017500in}{0.017500in}}{\pgfqpoint{6.218282in}{3.580372in}} %
\pgfusepath{clip}%
\pgfsetbuttcap%
\pgfsetroundjoin%
\definecolor{currentfill}{rgb}{0.880896,0.402331,0.317115}%
\pgfsetfillcolor{currentfill}%
\pgfsetlinewidth{0.000000pt}%
\definecolor{currentstroke}{rgb}{0.000000,0.000000,0.000000}%
\pgfsetstrokecolor{currentstroke}%
\pgfsetdash{}{0pt}%
\pgfpathmoveto{\pgfqpoint{2.590029in}{2.232729in}}%
\pgfpathlineto{\pgfqpoint{2.855457in}{2.456853in}}%
\pgfpathlineto{\pgfqpoint{2.590029in}{2.232729in}}%
\pgfpathclose%
\pgfusepath{fill}%
\end{pgfscope}%
\begin{pgfscope}%
\pgfpathrectangle{\pgfqpoint{0.017500in}{0.017500in}}{\pgfqpoint{6.218282in}{3.580372in}} %
\pgfusepath{clip}%
\pgfsetbuttcap%
\pgfsetroundjoin%
\definecolor{currentfill}{rgb}{0.880896,0.402331,0.317115}%
\pgfsetfillcolor{currentfill}%
\pgfsetlinewidth{0.000000pt}%
\definecolor{currentstroke}{rgb}{0.000000,0.000000,0.000000}%
\pgfsetstrokecolor{currentstroke}%
\pgfsetdash{}{0pt}%
\pgfpathmoveto{\pgfqpoint{2.590029in}{2.232729in}}%
\pgfpathlineto{\pgfqpoint{2.855457in}{2.456853in}}%
\pgfpathlineto{\pgfqpoint{2.590029in}{2.232729in}}%
\pgfpathclose%
\pgfusepath{fill}%
\end{pgfscope}%
\begin{pgfscope}%
\pgfpathrectangle{\pgfqpoint{0.017500in}{0.017500in}}{\pgfqpoint{6.218282in}{3.580372in}} %
\pgfusepath{clip}%
\pgfsetbuttcap%
\pgfsetroundjoin%
\definecolor{currentfill}{rgb}{0.880896,0.402331,0.317115}%
\pgfsetfillcolor{currentfill}%
\pgfsetlinewidth{0.000000pt}%
\definecolor{currentstroke}{rgb}{0.000000,0.000000,0.000000}%
\pgfsetstrokecolor{currentstroke}%
\pgfsetdash{}{0pt}%
\pgfpathmoveto{\pgfqpoint{2.590029in}{2.232729in}}%
\pgfpathlineto{\pgfqpoint{2.855457in}{2.456853in}}%
\pgfpathlineto{\pgfqpoint{2.590029in}{2.232729in}}%
\pgfpathclose%
\pgfusepath{fill}%
\end{pgfscope}%
\begin{pgfscope}%
\pgfpathrectangle{\pgfqpoint{0.017500in}{0.017500in}}{\pgfqpoint{6.218282in}{3.580372in}} %
\pgfusepath{clip}%
\pgfsetbuttcap%
\pgfsetroundjoin%
\definecolor{currentfill}{rgb}{0.880896,0.402331,0.317115}%
\pgfsetfillcolor{currentfill}%
\pgfsetlinewidth{0.000000pt}%
\definecolor{currentstroke}{rgb}{0.000000,0.000000,0.000000}%
\pgfsetstrokecolor{currentstroke}%
\pgfsetdash{}{0pt}%
\pgfpathmoveto{\pgfqpoint{2.590029in}{2.232729in}}%
\pgfpathlineto{\pgfqpoint{2.855457in}{2.456853in}}%
\pgfpathlineto{\pgfqpoint{2.590029in}{2.232729in}}%
\pgfpathclose%
\pgfusepath{fill}%
\end{pgfscope}%
\begin{pgfscope}%
\pgfpathrectangle{\pgfqpoint{0.017500in}{0.017500in}}{\pgfqpoint{6.218282in}{3.580372in}} %
\pgfusepath{clip}%
\pgfsetbuttcap%
\pgfsetroundjoin%
\definecolor{currentfill}{rgb}{0.880896,0.402331,0.317115}%
\pgfsetfillcolor{currentfill}%
\pgfsetlinewidth{0.000000pt}%
\definecolor{currentstroke}{rgb}{0.000000,0.000000,0.000000}%
\pgfsetstrokecolor{currentstroke}%
\pgfsetdash{}{0pt}%
\pgfpathmoveto{\pgfqpoint{2.590029in}{2.232729in}}%
\pgfpathlineto{\pgfqpoint{2.855457in}{2.456853in}}%
\pgfpathlineto{\pgfqpoint{2.590029in}{2.232729in}}%
\pgfpathclose%
\pgfusepath{fill}%
\end{pgfscope}%
\begin{pgfscope}%
\pgfpathrectangle{\pgfqpoint{0.017500in}{0.017500in}}{\pgfqpoint{6.218282in}{3.580372in}} %
\pgfusepath{clip}%
\pgfsetbuttcap%
\pgfsetroundjoin%
\definecolor{currentfill}{rgb}{0.880896,0.402331,0.317115}%
\pgfsetfillcolor{currentfill}%
\pgfsetlinewidth{0.000000pt}%
\definecolor{currentstroke}{rgb}{0.000000,0.000000,0.000000}%
\pgfsetstrokecolor{currentstroke}%
\pgfsetdash{}{0pt}%
\pgfpathmoveto{\pgfqpoint{2.590029in}{2.232729in}}%
\pgfpathlineto{\pgfqpoint{2.855457in}{2.456853in}}%
\pgfpathlineto{\pgfqpoint{2.590029in}{2.232729in}}%
\pgfpathclose%
\pgfusepath{fill}%
\end{pgfscope}%
\begin{pgfscope}%
\pgfpathrectangle{\pgfqpoint{0.017500in}{0.017500in}}{\pgfqpoint{6.218282in}{3.580372in}} %
\pgfusepath{clip}%
\pgfsetbuttcap%
\pgfsetroundjoin%
\definecolor{currentfill}{rgb}{0.746838,0.140021,0.179996}%
\pgfsetfillcolor{currentfill}%
\pgfsetlinewidth{0.000000pt}%
\definecolor{currentstroke}{rgb}{0.000000,0.000000,0.000000}%
\pgfsetstrokecolor{currentstroke}%
\pgfsetdash{}{0pt}%
\pgfpathmoveto{\pgfqpoint{4.478797in}{2.402809in}}%
\pgfpathlineto{\pgfqpoint{4.724973in}{2.556562in}}%
\pgfpathlineto{\pgfqpoint{4.478797in}{2.402809in}}%
\pgfpathclose%
\pgfusepath{fill}%
\end{pgfscope}%
\begin{pgfscope}%
\pgfpathrectangle{\pgfqpoint{0.017500in}{0.017500in}}{\pgfqpoint{6.218282in}{3.580372in}} %
\pgfusepath{clip}%
\pgfsetbuttcap%
\pgfsetroundjoin%
\definecolor{currentfill}{rgb}{0.746838,0.140021,0.179996}%
\pgfsetfillcolor{currentfill}%
\pgfsetlinewidth{0.000000pt}%
\definecolor{currentstroke}{rgb}{0.000000,0.000000,0.000000}%
\pgfsetstrokecolor{currentstroke}%
\pgfsetdash{}{0pt}%
\pgfpathmoveto{\pgfqpoint{4.478797in}{2.402809in}}%
\pgfpathlineto{\pgfqpoint{4.724973in}{2.556562in}}%
\pgfpathlineto{\pgfqpoint{4.478797in}{2.402809in}}%
\pgfpathclose%
\pgfusepath{fill}%
\end{pgfscope}%
\begin{pgfscope}%
\pgfpathrectangle{\pgfqpoint{0.017500in}{0.017500in}}{\pgfqpoint{6.218282in}{3.580372in}} %
\pgfusepath{clip}%
\pgfsetbuttcap%
\pgfsetroundjoin%
\definecolor{currentfill}{rgb}{0.746838,0.140021,0.179996}%
\pgfsetfillcolor{currentfill}%
\pgfsetlinewidth{0.000000pt}%
\definecolor{currentstroke}{rgb}{0.000000,0.000000,0.000000}%
\pgfsetstrokecolor{currentstroke}%
\pgfsetdash{}{0pt}%
\pgfpathmoveto{\pgfqpoint{4.478797in}{2.402809in}}%
\pgfpathlineto{\pgfqpoint{4.724973in}{2.556562in}}%
\pgfpathlineto{\pgfqpoint{4.478797in}{2.402809in}}%
\pgfpathclose%
\pgfusepath{fill}%
\end{pgfscope}%
\begin{pgfscope}%
\pgfpathrectangle{\pgfqpoint{0.017500in}{0.017500in}}{\pgfqpoint{6.218282in}{3.580372in}} %
\pgfusepath{clip}%
\pgfsetbuttcap%
\pgfsetroundjoin%
\definecolor{currentfill}{rgb}{0.746838,0.140021,0.179996}%
\pgfsetfillcolor{currentfill}%
\pgfsetlinewidth{0.000000pt}%
\definecolor{currentstroke}{rgb}{0.000000,0.000000,0.000000}%
\pgfsetstrokecolor{currentstroke}%
\pgfsetdash{}{0pt}%
\pgfpathmoveto{\pgfqpoint{4.478797in}{2.402809in}}%
\pgfpathlineto{\pgfqpoint{4.724973in}{2.556562in}}%
\pgfpathlineto{\pgfqpoint{4.478797in}{2.402809in}}%
\pgfpathclose%
\pgfusepath{fill}%
\end{pgfscope}%
\begin{pgfscope}%
\pgfpathrectangle{\pgfqpoint{0.017500in}{0.017500in}}{\pgfqpoint{6.218282in}{3.580372in}} %
\pgfusepath{clip}%
\pgfsetbuttcap%
\pgfsetroundjoin%
\definecolor{currentfill}{rgb}{0.746838,0.140021,0.179996}%
\pgfsetfillcolor{currentfill}%
\pgfsetlinewidth{0.000000pt}%
\definecolor{currentstroke}{rgb}{0.000000,0.000000,0.000000}%
\pgfsetstrokecolor{currentstroke}%
\pgfsetdash{}{0pt}%
\pgfpathmoveto{\pgfqpoint{4.478797in}{2.402809in}}%
\pgfpathlineto{\pgfqpoint{4.724973in}{2.556562in}}%
\pgfpathlineto{\pgfqpoint{4.478797in}{2.402809in}}%
\pgfpathclose%
\pgfusepath{fill}%
\end{pgfscope}%
\begin{pgfscope}%
\pgfpathrectangle{\pgfqpoint{0.017500in}{0.017500in}}{\pgfqpoint{6.218282in}{3.580372in}} %
\pgfusepath{clip}%
\pgfsetbuttcap%
\pgfsetroundjoin%
\definecolor{currentfill}{rgb}{0.746838,0.140021,0.179996}%
\pgfsetfillcolor{currentfill}%
\pgfsetlinewidth{0.000000pt}%
\definecolor{currentstroke}{rgb}{0.000000,0.000000,0.000000}%
\pgfsetstrokecolor{currentstroke}%
\pgfsetdash{}{0pt}%
\pgfpathmoveto{\pgfqpoint{4.478797in}{2.402809in}}%
\pgfpathlineto{\pgfqpoint{4.724973in}{2.556562in}}%
\pgfpathlineto{\pgfqpoint{4.478797in}{2.402809in}}%
\pgfpathclose%
\pgfusepath{fill}%
\end{pgfscope}%
\begin{pgfscope}%
\pgfpathrectangle{\pgfqpoint{0.017500in}{0.017500in}}{\pgfqpoint{6.218282in}{3.580372in}} %
\pgfusepath{clip}%
\pgfsetbuttcap%
\pgfsetroundjoin%
\definecolor{currentfill}{rgb}{0.746838,0.140021,0.179996}%
\pgfsetfillcolor{currentfill}%
\pgfsetlinewidth{0.000000pt}%
\definecolor{currentstroke}{rgb}{0.000000,0.000000,0.000000}%
\pgfsetstrokecolor{currentstroke}%
\pgfsetdash{}{0pt}%
\pgfpathmoveto{\pgfqpoint{4.478797in}{2.402809in}}%
\pgfpathlineto{\pgfqpoint{4.724973in}{2.556562in}}%
\pgfpathlineto{\pgfqpoint{4.478797in}{2.402809in}}%
\pgfpathclose%
\pgfusepath{fill}%
\end{pgfscope}%
\begin{pgfscope}%
\pgfpathrectangle{\pgfqpoint{0.017500in}{0.017500in}}{\pgfqpoint{6.218282in}{3.580372in}} %
\pgfusepath{clip}%
\pgfsetbuttcap%
\pgfsetroundjoin%
\definecolor{currentfill}{rgb}{0.746838,0.140021,0.179996}%
\pgfsetfillcolor{currentfill}%
\pgfsetlinewidth{0.000000pt}%
\definecolor{currentstroke}{rgb}{0.000000,0.000000,0.000000}%
\pgfsetstrokecolor{currentstroke}%
\pgfsetdash{}{0pt}%
\pgfpathmoveto{\pgfqpoint{4.478797in}{2.402809in}}%
\pgfpathlineto{\pgfqpoint{4.724973in}{2.556562in}}%
\pgfpathlineto{\pgfqpoint{4.478797in}{2.402809in}}%
\pgfpathclose%
\pgfusepath{fill}%
\end{pgfscope}%
\begin{pgfscope}%
\pgfpathrectangle{\pgfqpoint{0.017500in}{0.017500in}}{\pgfqpoint{6.218282in}{3.580372in}} %
\pgfusepath{clip}%
\pgfsetbuttcap%
\pgfsetroundjoin%
\definecolor{currentfill}{rgb}{0.746838,0.140021,0.179996}%
\pgfsetfillcolor{currentfill}%
\pgfsetlinewidth{0.000000pt}%
\definecolor{currentstroke}{rgb}{0.000000,0.000000,0.000000}%
\pgfsetstrokecolor{currentstroke}%
\pgfsetdash{}{0pt}%
\pgfpathmoveto{\pgfqpoint{4.478797in}{2.402809in}}%
\pgfpathlineto{\pgfqpoint{4.724973in}{2.556562in}}%
\pgfpathlineto{\pgfqpoint{4.478797in}{2.402809in}}%
\pgfpathclose%
\pgfusepath{fill}%
\end{pgfscope}%
\begin{pgfscope}%
\pgfpathrectangle{\pgfqpoint{0.017500in}{0.017500in}}{\pgfqpoint{6.218282in}{3.580372in}} %
\pgfusepath{clip}%
\pgfsetbuttcap%
\pgfsetroundjoin%
\definecolor{currentfill}{rgb}{0.746838,0.140021,0.179996}%
\pgfsetfillcolor{currentfill}%
\pgfsetlinewidth{0.000000pt}%
\definecolor{currentstroke}{rgb}{0.000000,0.000000,0.000000}%
\pgfsetstrokecolor{currentstroke}%
\pgfsetdash{}{0pt}%
\pgfpathmoveto{\pgfqpoint{4.478797in}{2.402809in}}%
\pgfpathlineto{\pgfqpoint{4.724973in}{2.556562in}}%
\pgfpathlineto{\pgfqpoint{4.478797in}{2.402809in}}%
\pgfpathclose%
\pgfusepath{fill}%
\end{pgfscope}%
\begin{pgfscope}%
\pgfpathrectangle{\pgfqpoint{0.017500in}{0.017500in}}{\pgfqpoint{6.218282in}{3.580372in}} %
\pgfusepath{clip}%
\pgfsetbuttcap%
\pgfsetroundjoin%
\definecolor{currentfill}{rgb}{0.746838,0.140021,0.179996}%
\pgfsetfillcolor{currentfill}%
\pgfsetlinewidth{0.000000pt}%
\definecolor{currentstroke}{rgb}{0.000000,0.000000,0.000000}%
\pgfsetstrokecolor{currentstroke}%
\pgfsetdash{}{0pt}%
\pgfpathmoveto{\pgfqpoint{4.478797in}{2.402809in}}%
\pgfpathlineto{\pgfqpoint{4.724973in}{2.556562in}}%
\pgfpathlineto{\pgfqpoint{4.478797in}{2.402809in}}%
\pgfpathclose%
\pgfusepath{fill}%
\end{pgfscope}%
\begin{pgfscope}%
\pgfpathrectangle{\pgfqpoint{0.017500in}{0.017500in}}{\pgfqpoint{6.218282in}{3.580372in}} %
\pgfusepath{clip}%
\pgfsetbuttcap%
\pgfsetroundjoin%
\definecolor{currentfill}{rgb}{0.746838,0.140021,0.179996}%
\pgfsetfillcolor{currentfill}%
\pgfsetlinewidth{0.000000pt}%
\definecolor{currentstroke}{rgb}{0.000000,0.000000,0.000000}%
\pgfsetstrokecolor{currentstroke}%
\pgfsetdash{}{0pt}%
\pgfpathmoveto{\pgfqpoint{4.478797in}{2.402809in}}%
\pgfpathlineto{\pgfqpoint{4.724973in}{2.556562in}}%
\pgfpathlineto{\pgfqpoint{4.478797in}{2.402809in}}%
\pgfpathclose%
\pgfusepath{fill}%
\end{pgfscope}%
\begin{pgfscope}%
\pgfpathrectangle{\pgfqpoint{0.017500in}{0.017500in}}{\pgfqpoint{6.218282in}{3.580372in}} %
\pgfusepath{clip}%
\pgfsetbuttcap%
\pgfsetroundjoin%
\definecolor{currentfill}{rgb}{0.746838,0.140021,0.179996}%
\pgfsetfillcolor{currentfill}%
\pgfsetlinewidth{0.000000pt}%
\definecolor{currentstroke}{rgb}{0.000000,0.000000,0.000000}%
\pgfsetstrokecolor{currentstroke}%
\pgfsetdash{}{0pt}%
\pgfpathmoveto{\pgfqpoint{4.478797in}{2.402809in}}%
\pgfpathlineto{\pgfqpoint{4.724973in}{2.556562in}}%
\pgfpathlineto{\pgfqpoint{4.478797in}{2.402809in}}%
\pgfpathclose%
\pgfusepath{fill}%
\end{pgfscope}%
\begin{pgfscope}%
\pgfpathrectangle{\pgfqpoint{0.017500in}{0.017500in}}{\pgfqpoint{6.218282in}{3.580372in}} %
\pgfusepath{clip}%
\pgfsetbuttcap%
\pgfsetroundjoin%
\definecolor{currentfill}{rgb}{0.746838,0.140021,0.179996}%
\pgfsetfillcolor{currentfill}%
\pgfsetlinewidth{0.000000pt}%
\definecolor{currentstroke}{rgb}{0.000000,0.000000,0.000000}%
\pgfsetstrokecolor{currentstroke}%
\pgfsetdash{}{0pt}%
\pgfpathmoveto{\pgfqpoint{4.478797in}{2.402809in}}%
\pgfpathlineto{\pgfqpoint{4.724973in}{2.556562in}}%
\pgfpathlineto{\pgfqpoint{4.478797in}{2.402809in}}%
\pgfpathclose%
\pgfusepath{fill}%
\end{pgfscope}%
\begin{pgfscope}%
\pgfpathrectangle{\pgfqpoint{0.017500in}{0.017500in}}{\pgfqpoint{6.218282in}{3.580372in}} %
\pgfusepath{clip}%
\pgfsetbuttcap%
\pgfsetroundjoin%
\definecolor{currentfill}{rgb}{0.746838,0.140021,0.179996}%
\pgfsetfillcolor{currentfill}%
\pgfsetlinewidth{0.000000pt}%
\definecolor{currentstroke}{rgb}{0.000000,0.000000,0.000000}%
\pgfsetstrokecolor{currentstroke}%
\pgfsetdash{}{0pt}%
\pgfpathmoveto{\pgfqpoint{4.478797in}{2.402809in}}%
\pgfpathlineto{\pgfqpoint{4.724973in}{2.556562in}}%
\pgfpathlineto{\pgfqpoint{4.478797in}{2.402809in}}%
\pgfpathclose%
\pgfusepath{fill}%
\end{pgfscope}%
\begin{pgfscope}%
\pgfpathrectangle{\pgfqpoint{0.017500in}{0.017500in}}{\pgfqpoint{6.218282in}{3.580372in}} %
\pgfusepath{clip}%
\pgfsetbuttcap%
\pgfsetroundjoin%
\definecolor{currentfill}{rgb}{0.746838,0.140021,0.179996}%
\pgfsetfillcolor{currentfill}%
\pgfsetlinewidth{0.000000pt}%
\definecolor{currentstroke}{rgb}{0.000000,0.000000,0.000000}%
\pgfsetstrokecolor{currentstroke}%
\pgfsetdash{}{0pt}%
\pgfpathmoveto{\pgfqpoint{4.478797in}{2.402809in}}%
\pgfpathlineto{\pgfqpoint{4.724973in}{2.556562in}}%
\pgfpathlineto{\pgfqpoint{4.478797in}{2.402809in}}%
\pgfpathclose%
\pgfusepath{fill}%
\end{pgfscope}%
\begin{pgfscope}%
\pgfpathrectangle{\pgfqpoint{0.017500in}{0.017500in}}{\pgfqpoint{6.218282in}{3.580372in}} %
\pgfusepath{clip}%
\pgfsetbuttcap%
\pgfsetroundjoin%
\definecolor{currentfill}{rgb}{0.746838,0.140021,0.179996}%
\pgfsetfillcolor{currentfill}%
\pgfsetlinewidth{0.000000pt}%
\definecolor{currentstroke}{rgb}{0.000000,0.000000,0.000000}%
\pgfsetstrokecolor{currentstroke}%
\pgfsetdash{}{0pt}%
\pgfpathmoveto{\pgfqpoint{4.478797in}{2.402809in}}%
\pgfpathlineto{\pgfqpoint{4.724973in}{2.556562in}}%
\pgfpathlineto{\pgfqpoint{4.478797in}{2.402809in}}%
\pgfpathclose%
\pgfusepath{fill}%
\end{pgfscope}%
\begin{pgfscope}%
\pgfpathrectangle{\pgfqpoint{0.017500in}{0.017500in}}{\pgfqpoint{6.218282in}{3.580372in}} %
\pgfusepath{clip}%
\pgfsetbuttcap%
\pgfsetroundjoin%
\definecolor{currentfill}{rgb}{0.746838,0.140021,0.179996}%
\pgfsetfillcolor{currentfill}%
\pgfsetlinewidth{0.000000pt}%
\definecolor{currentstroke}{rgb}{0.000000,0.000000,0.000000}%
\pgfsetstrokecolor{currentstroke}%
\pgfsetdash{}{0pt}%
\pgfpathmoveto{\pgfqpoint{4.478797in}{2.402809in}}%
\pgfpathlineto{\pgfqpoint{4.724973in}{2.556562in}}%
\pgfpathlineto{\pgfqpoint{4.478797in}{2.402809in}}%
\pgfpathclose%
\pgfusepath{fill}%
\end{pgfscope}%
\begin{pgfscope}%
\pgfpathrectangle{\pgfqpoint{0.017500in}{0.017500in}}{\pgfqpoint{6.218282in}{3.580372in}} %
\pgfusepath{clip}%
\pgfsetbuttcap%
\pgfsetroundjoin%
\definecolor{currentfill}{rgb}{0.746838,0.140021,0.179996}%
\pgfsetfillcolor{currentfill}%
\pgfsetlinewidth{0.000000pt}%
\definecolor{currentstroke}{rgb}{0.000000,0.000000,0.000000}%
\pgfsetstrokecolor{currentstroke}%
\pgfsetdash{}{0pt}%
\pgfpathmoveto{\pgfqpoint{4.478797in}{2.402809in}}%
\pgfpathlineto{\pgfqpoint{4.724973in}{2.556562in}}%
\pgfpathlineto{\pgfqpoint{4.478797in}{2.402809in}}%
\pgfpathclose%
\pgfusepath{fill}%
\end{pgfscope}%
\begin{pgfscope}%
\pgfpathrectangle{\pgfqpoint{0.017500in}{0.017500in}}{\pgfqpoint{6.218282in}{3.580372in}} %
\pgfusepath{clip}%
\pgfsetbuttcap%
\pgfsetroundjoin%
\definecolor{currentfill}{rgb}{0.746838,0.140021,0.179996}%
\pgfsetfillcolor{currentfill}%
\pgfsetlinewidth{0.000000pt}%
\definecolor{currentstroke}{rgb}{0.000000,0.000000,0.000000}%
\pgfsetstrokecolor{currentstroke}%
\pgfsetdash{}{0pt}%
\pgfpathmoveto{\pgfqpoint{4.478797in}{2.402809in}}%
\pgfpathlineto{\pgfqpoint{4.724973in}{2.556562in}}%
\pgfpathlineto{\pgfqpoint{4.478797in}{2.402809in}}%
\pgfpathclose%
\pgfusepath{fill}%
\end{pgfscope}%
\begin{pgfscope}%
\pgfpathrectangle{\pgfqpoint{0.017500in}{0.017500in}}{\pgfqpoint{6.218282in}{3.580372in}} %
\pgfusepath{clip}%
\pgfsetbuttcap%
\pgfsetroundjoin%
\definecolor{currentfill}{rgb}{0.746838,0.140021,0.179996}%
\pgfsetfillcolor{currentfill}%
\pgfsetlinewidth{0.000000pt}%
\definecolor{currentstroke}{rgb}{0.000000,0.000000,0.000000}%
\pgfsetstrokecolor{currentstroke}%
\pgfsetdash{}{0pt}%
\pgfpathmoveto{\pgfqpoint{4.478797in}{2.402809in}}%
\pgfpathlineto{\pgfqpoint{4.724973in}{2.556562in}}%
\pgfpathlineto{\pgfqpoint{4.478797in}{2.402809in}}%
\pgfpathclose%
\pgfusepath{fill}%
\end{pgfscope}%
\begin{pgfscope}%
\pgfpathrectangle{\pgfqpoint{0.017500in}{0.017500in}}{\pgfqpoint{6.218282in}{3.580372in}} %
\pgfusepath{clip}%
\pgfsetbuttcap%
\pgfsetroundjoin%
\definecolor{currentfill}{rgb}{0.746838,0.140021,0.179996}%
\pgfsetfillcolor{currentfill}%
\pgfsetlinewidth{0.000000pt}%
\definecolor{currentstroke}{rgb}{0.000000,0.000000,0.000000}%
\pgfsetstrokecolor{currentstroke}%
\pgfsetdash{}{0pt}%
\pgfpathmoveto{\pgfqpoint{4.478797in}{2.402809in}}%
\pgfpathlineto{\pgfqpoint{4.724973in}{2.556562in}}%
\pgfpathlineto{\pgfqpoint{4.478797in}{2.402809in}}%
\pgfpathclose%
\pgfusepath{fill}%
\end{pgfscope}%
\begin{pgfscope}%
\pgfpathrectangle{\pgfqpoint{0.017500in}{0.017500in}}{\pgfqpoint{6.218282in}{3.580372in}} %
\pgfusepath{clip}%
\pgfsetbuttcap%
\pgfsetroundjoin%
\definecolor{currentfill}{rgb}{0.746838,0.140021,0.179996}%
\pgfsetfillcolor{currentfill}%
\pgfsetlinewidth{0.000000pt}%
\definecolor{currentstroke}{rgb}{0.000000,0.000000,0.000000}%
\pgfsetstrokecolor{currentstroke}%
\pgfsetdash{}{0pt}%
\pgfpathmoveto{\pgfqpoint{4.478797in}{2.402809in}}%
\pgfpathlineto{\pgfqpoint{4.724973in}{2.556562in}}%
\pgfpathlineto{\pgfqpoint{4.478797in}{2.402809in}}%
\pgfpathclose%
\pgfusepath{fill}%
\end{pgfscope}%
\begin{pgfscope}%
\pgfpathrectangle{\pgfqpoint{0.017500in}{0.017500in}}{\pgfqpoint{6.218282in}{3.580372in}} %
\pgfusepath{clip}%
\pgfsetbuttcap%
\pgfsetroundjoin%
\definecolor{currentfill}{rgb}{0.746838,0.140021,0.179996}%
\pgfsetfillcolor{currentfill}%
\pgfsetlinewidth{0.000000pt}%
\definecolor{currentstroke}{rgb}{0.000000,0.000000,0.000000}%
\pgfsetstrokecolor{currentstroke}%
\pgfsetdash{}{0pt}%
\pgfpathmoveto{\pgfqpoint{4.478797in}{2.402809in}}%
\pgfpathlineto{\pgfqpoint{4.724973in}{2.556562in}}%
\pgfpathlineto{\pgfqpoint{4.478797in}{2.402809in}}%
\pgfpathclose%
\pgfusepath{fill}%
\end{pgfscope}%
\begin{pgfscope}%
\pgfpathrectangle{\pgfqpoint{0.017500in}{0.017500in}}{\pgfqpoint{6.218282in}{3.580372in}} %
\pgfusepath{clip}%
\pgfsetbuttcap%
\pgfsetroundjoin%
\definecolor{currentfill}{rgb}{0.746838,0.140021,0.179996}%
\pgfsetfillcolor{currentfill}%
\pgfsetlinewidth{0.000000pt}%
\definecolor{currentstroke}{rgb}{0.000000,0.000000,0.000000}%
\pgfsetstrokecolor{currentstroke}%
\pgfsetdash{}{0pt}%
\pgfpathmoveto{\pgfqpoint{4.478797in}{2.402809in}}%
\pgfpathlineto{\pgfqpoint{4.724973in}{2.556562in}}%
\pgfpathlineto{\pgfqpoint{4.478797in}{2.402809in}}%
\pgfpathclose%
\pgfusepath{fill}%
\end{pgfscope}%
\begin{pgfscope}%
\pgfpathrectangle{\pgfqpoint{0.017500in}{0.017500in}}{\pgfqpoint{6.218282in}{3.580372in}} %
\pgfusepath{clip}%
\pgfsetbuttcap%
\pgfsetroundjoin%
\definecolor{currentfill}{rgb}{0.746838,0.140021,0.179996}%
\pgfsetfillcolor{currentfill}%
\pgfsetlinewidth{0.000000pt}%
\definecolor{currentstroke}{rgb}{0.000000,0.000000,0.000000}%
\pgfsetstrokecolor{currentstroke}%
\pgfsetdash{}{0pt}%
\pgfpathmoveto{\pgfqpoint{4.478797in}{2.402809in}}%
\pgfpathlineto{\pgfqpoint{4.724973in}{2.556562in}}%
\pgfpathlineto{\pgfqpoint{4.478797in}{2.402809in}}%
\pgfpathclose%
\pgfusepath{fill}%
\end{pgfscope}%
\begin{pgfscope}%
\pgfpathrectangle{\pgfqpoint{0.017500in}{0.017500in}}{\pgfqpoint{6.218282in}{3.580372in}} %
\pgfusepath{clip}%
\pgfsetbuttcap%
\pgfsetroundjoin%
\definecolor{currentfill}{rgb}{0.746838,0.140021,0.179996}%
\pgfsetfillcolor{currentfill}%
\pgfsetlinewidth{0.000000pt}%
\definecolor{currentstroke}{rgb}{0.000000,0.000000,0.000000}%
\pgfsetstrokecolor{currentstroke}%
\pgfsetdash{}{0pt}%
\pgfpathmoveto{\pgfqpoint{4.478797in}{2.402809in}}%
\pgfpathlineto{\pgfqpoint{4.724973in}{2.556562in}}%
\pgfpathlineto{\pgfqpoint{4.478797in}{2.402809in}}%
\pgfpathclose%
\pgfusepath{fill}%
\end{pgfscope}%
\begin{pgfscope}%
\pgfpathrectangle{\pgfqpoint{0.017500in}{0.017500in}}{\pgfqpoint{6.218282in}{3.580372in}} %
\pgfusepath{clip}%
\pgfsetbuttcap%
\pgfsetroundjoin%
\definecolor{currentfill}{rgb}{0.746838,0.140021,0.179996}%
\pgfsetfillcolor{currentfill}%
\pgfsetlinewidth{0.000000pt}%
\definecolor{currentstroke}{rgb}{0.000000,0.000000,0.000000}%
\pgfsetstrokecolor{currentstroke}%
\pgfsetdash{}{0pt}%
\pgfpathmoveto{\pgfqpoint{4.478797in}{2.402809in}}%
\pgfpathlineto{\pgfqpoint{4.724973in}{2.556562in}}%
\pgfpathlineto{\pgfqpoint{4.478797in}{2.402809in}}%
\pgfpathclose%
\pgfusepath{fill}%
\end{pgfscope}%
\begin{pgfscope}%
\pgfpathrectangle{\pgfqpoint{0.017500in}{0.017500in}}{\pgfqpoint{6.218282in}{3.580372in}} %
\pgfusepath{clip}%
\pgfsetbuttcap%
\pgfsetroundjoin%
\definecolor{currentfill}{rgb}{0.746838,0.140021,0.179996}%
\pgfsetfillcolor{currentfill}%
\pgfsetlinewidth{0.000000pt}%
\definecolor{currentstroke}{rgb}{0.000000,0.000000,0.000000}%
\pgfsetstrokecolor{currentstroke}%
\pgfsetdash{}{0pt}%
\pgfpathmoveto{\pgfqpoint{4.478797in}{2.402809in}}%
\pgfpathlineto{\pgfqpoint{4.724973in}{2.556562in}}%
\pgfpathlineto{\pgfqpoint{4.478797in}{2.402809in}}%
\pgfpathclose%
\pgfusepath{fill}%
\end{pgfscope}%
\begin{pgfscope}%
\pgfpathrectangle{\pgfqpoint{0.017500in}{0.017500in}}{\pgfqpoint{6.218282in}{3.580372in}} %
\pgfusepath{clip}%
\pgfsetbuttcap%
\pgfsetroundjoin%
\definecolor{currentfill}{rgb}{0.746838,0.140021,0.179996}%
\pgfsetfillcolor{currentfill}%
\pgfsetlinewidth{0.000000pt}%
\definecolor{currentstroke}{rgb}{0.000000,0.000000,0.000000}%
\pgfsetstrokecolor{currentstroke}%
\pgfsetdash{}{0pt}%
\pgfpathmoveto{\pgfqpoint{4.478797in}{2.402809in}}%
\pgfpathlineto{\pgfqpoint{4.724973in}{2.556562in}}%
\pgfpathlineto{\pgfqpoint{4.478797in}{2.402809in}}%
\pgfpathclose%
\pgfusepath{fill}%
\end{pgfscope}%
\begin{pgfscope}%
\pgfpathrectangle{\pgfqpoint{0.017500in}{0.017500in}}{\pgfqpoint{6.218282in}{3.580372in}} %
\pgfusepath{clip}%
\pgfsetbuttcap%
\pgfsetroundjoin%
\definecolor{currentfill}{rgb}{0.873402,0.386960,0.306332}%
\pgfsetfillcolor{currentfill}%
\pgfsetlinewidth{0.000000pt}%
\definecolor{currentstroke}{rgb}{0.000000,0.000000,0.000000}%
\pgfsetstrokecolor{currentstroke}%
\pgfsetdash{}{0pt}%
\pgfpathmoveto{\pgfqpoint{2.590029in}{2.232729in}}%
\pgfpathlineto{\pgfqpoint{3.043201in}{2.166904in}}%
\pgfpathlineto{\pgfqpoint{3.306522in}{2.395850in}}%
\pgfpathlineto{\pgfqpoint{2.855457in}{2.456853in}}%
\pgfpathlineto{\pgfqpoint{2.590029in}{2.232729in}}%
\pgfpathclose%
\pgfusepath{fill}%
\end{pgfscope}%
\begin{pgfscope}%
\pgfpathrectangle{\pgfqpoint{0.017500in}{0.017500in}}{\pgfqpoint{6.218282in}{3.580372in}} %
\pgfusepath{clip}%
\pgfsetbuttcap%
\pgfsetroundjoin%
\definecolor{currentfill}{rgb}{0.873402,0.386960,0.306332}%
\pgfsetfillcolor{currentfill}%
\pgfsetlinewidth{0.000000pt}%
\definecolor{currentstroke}{rgb}{0.000000,0.000000,0.000000}%
\pgfsetstrokecolor{currentstroke}%
\pgfsetdash{}{0pt}%
\pgfpathmoveto{\pgfqpoint{2.590029in}{2.232729in}}%
\pgfpathlineto{\pgfqpoint{3.043201in}{2.166904in}}%
\pgfpathlineto{\pgfqpoint{3.306522in}{2.395850in}}%
\pgfpathlineto{\pgfqpoint{2.855457in}{2.456853in}}%
\pgfpathlineto{\pgfqpoint{2.590029in}{2.232729in}}%
\pgfpathclose%
\pgfusepath{fill}%
\end{pgfscope}%
\begin{pgfscope}%
\pgfpathrectangle{\pgfqpoint{0.017500in}{0.017500in}}{\pgfqpoint{6.218282in}{3.580372in}} %
\pgfusepath{clip}%
\pgfsetbuttcap%
\pgfsetroundjoin%
\definecolor{currentfill}{rgb}{0.873402,0.386960,0.306332}%
\pgfsetfillcolor{currentfill}%
\pgfsetlinewidth{0.000000pt}%
\definecolor{currentstroke}{rgb}{0.000000,0.000000,0.000000}%
\pgfsetstrokecolor{currentstroke}%
\pgfsetdash{}{0pt}%
\pgfpathmoveto{\pgfqpoint{2.590029in}{2.232729in}}%
\pgfpathlineto{\pgfqpoint{3.043201in}{2.166904in}}%
\pgfpathlineto{\pgfqpoint{3.306522in}{2.395850in}}%
\pgfpathlineto{\pgfqpoint{2.855457in}{2.456853in}}%
\pgfpathlineto{\pgfqpoint{2.590029in}{2.232729in}}%
\pgfpathclose%
\pgfusepath{fill}%
\end{pgfscope}%
\begin{pgfscope}%
\pgfpathrectangle{\pgfqpoint{0.017500in}{0.017500in}}{\pgfqpoint{6.218282in}{3.580372in}} %
\pgfusepath{clip}%
\pgfsetbuttcap%
\pgfsetroundjoin%
\definecolor{currentfill}{rgb}{0.873402,0.386960,0.306332}%
\pgfsetfillcolor{currentfill}%
\pgfsetlinewidth{0.000000pt}%
\definecolor{currentstroke}{rgb}{0.000000,0.000000,0.000000}%
\pgfsetstrokecolor{currentstroke}%
\pgfsetdash{}{0pt}%
\pgfpathmoveto{\pgfqpoint{2.590029in}{2.232729in}}%
\pgfpathlineto{\pgfqpoint{3.043201in}{2.166904in}}%
\pgfpathlineto{\pgfqpoint{3.306522in}{2.395850in}}%
\pgfpathlineto{\pgfqpoint{2.855457in}{2.456853in}}%
\pgfpathlineto{\pgfqpoint{2.590029in}{2.232729in}}%
\pgfpathclose%
\pgfusepath{fill}%
\end{pgfscope}%
\begin{pgfscope}%
\pgfpathrectangle{\pgfqpoint{0.017500in}{0.017500in}}{\pgfqpoint{6.218282in}{3.580372in}} %
\pgfusepath{clip}%
\pgfsetbuttcap%
\pgfsetroundjoin%
\definecolor{currentfill}{rgb}{0.873402,0.386960,0.306332}%
\pgfsetfillcolor{currentfill}%
\pgfsetlinewidth{0.000000pt}%
\definecolor{currentstroke}{rgb}{0.000000,0.000000,0.000000}%
\pgfsetstrokecolor{currentstroke}%
\pgfsetdash{}{0pt}%
\pgfpathmoveto{\pgfqpoint{2.590029in}{2.232729in}}%
\pgfpathlineto{\pgfqpoint{3.043201in}{2.166904in}}%
\pgfpathlineto{\pgfqpoint{3.306522in}{2.395850in}}%
\pgfpathlineto{\pgfqpoint{2.855457in}{2.456853in}}%
\pgfpathlineto{\pgfqpoint{2.590029in}{2.232729in}}%
\pgfpathclose%
\pgfusepath{fill}%
\end{pgfscope}%
\begin{pgfscope}%
\pgfpathrectangle{\pgfqpoint{0.017500in}{0.017500in}}{\pgfqpoint{6.218282in}{3.580372in}} %
\pgfusepath{clip}%
\pgfsetbuttcap%
\pgfsetroundjoin%
\definecolor{currentfill}{rgb}{0.873402,0.386960,0.306332}%
\pgfsetfillcolor{currentfill}%
\pgfsetlinewidth{0.000000pt}%
\definecolor{currentstroke}{rgb}{0.000000,0.000000,0.000000}%
\pgfsetstrokecolor{currentstroke}%
\pgfsetdash{}{0pt}%
\pgfpathmoveto{\pgfqpoint{2.590029in}{2.232729in}}%
\pgfpathlineto{\pgfqpoint{3.043201in}{2.166904in}}%
\pgfpathlineto{\pgfqpoint{3.306522in}{2.395850in}}%
\pgfpathlineto{\pgfqpoint{2.855457in}{2.456853in}}%
\pgfpathlineto{\pgfqpoint{2.590029in}{2.232729in}}%
\pgfpathclose%
\pgfusepath{fill}%
\end{pgfscope}%
\begin{pgfscope}%
\pgfpathrectangle{\pgfqpoint{0.017500in}{0.017500in}}{\pgfqpoint{6.218282in}{3.580372in}} %
\pgfusepath{clip}%
\pgfsetbuttcap%
\pgfsetroundjoin%
\definecolor{currentfill}{rgb}{0.795938,0.241845,0.220830}%
\pgfsetfillcolor{currentfill}%
\pgfsetlinewidth{0.000000pt}%
\definecolor{currentstroke}{rgb}{0.000000,0.000000,0.000000}%
\pgfsetstrokecolor{currentstroke}%
\pgfsetdash{}{0pt}%
\pgfpathmoveto{\pgfqpoint{3.763681in}{2.314857in}}%
\pgfpathlineto{\pgfqpoint{4.018512in}{2.470906in}}%
\pgfpathlineto{\pgfqpoint{3.763681in}{2.314857in}}%
\pgfpathclose%
\pgfusepath{fill}%
\end{pgfscope}%
\begin{pgfscope}%
\pgfpathrectangle{\pgfqpoint{0.017500in}{0.017500in}}{\pgfqpoint{6.218282in}{3.580372in}} %
\pgfusepath{clip}%
\pgfsetbuttcap%
\pgfsetroundjoin%
\definecolor{currentfill}{rgb}{0.795938,0.241845,0.220830}%
\pgfsetfillcolor{currentfill}%
\pgfsetlinewidth{0.000000pt}%
\definecolor{currentstroke}{rgb}{0.000000,0.000000,0.000000}%
\pgfsetstrokecolor{currentstroke}%
\pgfsetdash{}{0pt}%
\pgfpathmoveto{\pgfqpoint{3.763681in}{2.314857in}}%
\pgfpathlineto{\pgfqpoint{4.018512in}{2.470906in}}%
\pgfpathlineto{\pgfqpoint{3.763681in}{2.314857in}}%
\pgfpathclose%
\pgfusepath{fill}%
\end{pgfscope}%
\begin{pgfscope}%
\pgfpathrectangle{\pgfqpoint{0.017500in}{0.017500in}}{\pgfqpoint{6.218282in}{3.580372in}} %
\pgfusepath{clip}%
\pgfsetbuttcap%
\pgfsetroundjoin%
\definecolor{currentfill}{rgb}{0.795938,0.241845,0.220830}%
\pgfsetfillcolor{currentfill}%
\pgfsetlinewidth{0.000000pt}%
\definecolor{currentstroke}{rgb}{0.000000,0.000000,0.000000}%
\pgfsetstrokecolor{currentstroke}%
\pgfsetdash{}{0pt}%
\pgfpathmoveto{\pgfqpoint{3.763681in}{2.314857in}}%
\pgfpathlineto{\pgfqpoint{4.018512in}{2.470906in}}%
\pgfpathlineto{\pgfqpoint{3.763681in}{2.314857in}}%
\pgfpathclose%
\pgfusepath{fill}%
\end{pgfscope}%
\begin{pgfscope}%
\pgfpathrectangle{\pgfqpoint{0.017500in}{0.017500in}}{\pgfqpoint{6.218282in}{3.580372in}} %
\pgfusepath{clip}%
\pgfsetbuttcap%
\pgfsetroundjoin%
\definecolor{currentfill}{rgb}{0.795938,0.241845,0.220830}%
\pgfsetfillcolor{currentfill}%
\pgfsetlinewidth{0.000000pt}%
\definecolor{currentstroke}{rgb}{0.000000,0.000000,0.000000}%
\pgfsetstrokecolor{currentstroke}%
\pgfsetdash{}{0pt}%
\pgfpathmoveto{\pgfqpoint{3.763681in}{2.314857in}}%
\pgfpathlineto{\pgfqpoint{4.018512in}{2.470906in}}%
\pgfpathlineto{\pgfqpoint{3.763681in}{2.314857in}}%
\pgfpathclose%
\pgfusepath{fill}%
\end{pgfscope}%
\begin{pgfscope}%
\pgfpathrectangle{\pgfqpoint{0.017500in}{0.017500in}}{\pgfqpoint{6.218282in}{3.580372in}} %
\pgfusepath{clip}%
\pgfsetbuttcap%
\pgfsetroundjoin%
\definecolor{currentfill}{rgb}{0.795938,0.241845,0.220830}%
\pgfsetfillcolor{currentfill}%
\pgfsetlinewidth{0.000000pt}%
\definecolor{currentstroke}{rgb}{0.000000,0.000000,0.000000}%
\pgfsetstrokecolor{currentstroke}%
\pgfsetdash{}{0pt}%
\pgfpathmoveto{\pgfqpoint{3.763681in}{2.314857in}}%
\pgfpathlineto{\pgfqpoint{4.018512in}{2.470906in}}%
\pgfpathlineto{\pgfqpoint{3.763681in}{2.314857in}}%
\pgfpathclose%
\pgfusepath{fill}%
\end{pgfscope}%
\begin{pgfscope}%
\pgfpathrectangle{\pgfqpoint{0.017500in}{0.017500in}}{\pgfqpoint{6.218282in}{3.580372in}} %
\pgfusepath{clip}%
\pgfsetbuttcap%
\pgfsetroundjoin%
\definecolor{currentfill}{rgb}{0.795938,0.241845,0.220830}%
\pgfsetfillcolor{currentfill}%
\pgfsetlinewidth{0.000000pt}%
\definecolor{currentstroke}{rgb}{0.000000,0.000000,0.000000}%
\pgfsetstrokecolor{currentstroke}%
\pgfsetdash{}{0pt}%
\pgfpathmoveto{\pgfqpoint{3.763681in}{2.314857in}}%
\pgfpathlineto{\pgfqpoint{4.018512in}{2.470906in}}%
\pgfpathlineto{\pgfqpoint{3.763681in}{2.314857in}}%
\pgfpathclose%
\pgfusepath{fill}%
\end{pgfscope}%
\begin{pgfscope}%
\pgfpathrectangle{\pgfqpoint{0.017500in}{0.017500in}}{\pgfqpoint{6.218282in}{3.580372in}} %
\pgfusepath{clip}%
\pgfsetbuttcap%
\pgfsetroundjoin%
\definecolor{currentfill}{rgb}{0.795938,0.241845,0.220830}%
\pgfsetfillcolor{currentfill}%
\pgfsetlinewidth{0.000000pt}%
\definecolor{currentstroke}{rgb}{0.000000,0.000000,0.000000}%
\pgfsetstrokecolor{currentstroke}%
\pgfsetdash{}{0pt}%
\pgfpathmoveto{\pgfqpoint{3.763681in}{2.314857in}}%
\pgfpathlineto{\pgfqpoint{4.018512in}{2.470906in}}%
\pgfpathlineto{\pgfqpoint{3.763681in}{2.314857in}}%
\pgfpathclose%
\pgfusepath{fill}%
\end{pgfscope}%
\begin{pgfscope}%
\pgfpathrectangle{\pgfqpoint{0.017500in}{0.017500in}}{\pgfqpoint{6.218282in}{3.580372in}} %
\pgfusepath{clip}%
\pgfsetbuttcap%
\pgfsetroundjoin%
\definecolor{currentfill}{rgb}{0.795938,0.241845,0.220830}%
\pgfsetfillcolor{currentfill}%
\pgfsetlinewidth{0.000000pt}%
\definecolor{currentstroke}{rgb}{0.000000,0.000000,0.000000}%
\pgfsetstrokecolor{currentstroke}%
\pgfsetdash{}{0pt}%
\pgfpathmoveto{\pgfqpoint{3.763681in}{2.314857in}}%
\pgfpathlineto{\pgfqpoint{4.018512in}{2.470906in}}%
\pgfpathlineto{\pgfqpoint{3.763681in}{2.314857in}}%
\pgfpathclose%
\pgfusepath{fill}%
\end{pgfscope}%
\begin{pgfscope}%
\pgfpathrectangle{\pgfqpoint{0.017500in}{0.017500in}}{\pgfqpoint{6.218282in}{3.580372in}} %
\pgfusepath{clip}%
\pgfsetbuttcap%
\pgfsetroundjoin%
\definecolor{currentfill}{rgb}{0.795938,0.241845,0.220830}%
\pgfsetfillcolor{currentfill}%
\pgfsetlinewidth{0.000000pt}%
\definecolor{currentstroke}{rgb}{0.000000,0.000000,0.000000}%
\pgfsetstrokecolor{currentstroke}%
\pgfsetdash{}{0pt}%
\pgfpathmoveto{\pgfqpoint{3.763681in}{2.314857in}}%
\pgfpathlineto{\pgfqpoint{4.018512in}{2.470906in}}%
\pgfpathlineto{\pgfqpoint{3.763681in}{2.314857in}}%
\pgfpathclose%
\pgfusepath{fill}%
\end{pgfscope}%
\begin{pgfscope}%
\pgfpathrectangle{\pgfqpoint{0.017500in}{0.017500in}}{\pgfqpoint{6.218282in}{3.580372in}} %
\pgfusepath{clip}%
\pgfsetbuttcap%
\pgfsetroundjoin%
\definecolor{currentfill}{rgb}{0.795938,0.241845,0.220830}%
\pgfsetfillcolor{currentfill}%
\pgfsetlinewidth{0.000000pt}%
\definecolor{currentstroke}{rgb}{0.000000,0.000000,0.000000}%
\pgfsetstrokecolor{currentstroke}%
\pgfsetdash{}{0pt}%
\pgfpathmoveto{\pgfqpoint{3.763681in}{2.314857in}}%
\pgfpathlineto{\pgfqpoint{4.018512in}{2.470906in}}%
\pgfpathlineto{\pgfqpoint{3.763681in}{2.314857in}}%
\pgfpathclose%
\pgfusepath{fill}%
\end{pgfscope}%
\begin{pgfscope}%
\pgfpathrectangle{\pgfqpoint{0.017500in}{0.017500in}}{\pgfqpoint{6.218282in}{3.580372in}} %
\pgfusepath{clip}%
\pgfsetbuttcap%
\pgfsetroundjoin%
\definecolor{currentfill}{rgb}{0.795938,0.241845,0.220830}%
\pgfsetfillcolor{currentfill}%
\pgfsetlinewidth{0.000000pt}%
\definecolor{currentstroke}{rgb}{0.000000,0.000000,0.000000}%
\pgfsetstrokecolor{currentstroke}%
\pgfsetdash{}{0pt}%
\pgfpathmoveto{\pgfqpoint{3.763681in}{2.314857in}}%
\pgfpathlineto{\pgfqpoint{4.018512in}{2.470906in}}%
\pgfpathlineto{\pgfqpoint{3.763681in}{2.314857in}}%
\pgfpathclose%
\pgfusepath{fill}%
\end{pgfscope}%
\begin{pgfscope}%
\pgfpathrectangle{\pgfqpoint{0.017500in}{0.017500in}}{\pgfqpoint{6.218282in}{3.580372in}} %
\pgfusepath{clip}%
\pgfsetbuttcap%
\pgfsetroundjoin%
\definecolor{currentfill}{rgb}{0.795938,0.241845,0.220830}%
\pgfsetfillcolor{currentfill}%
\pgfsetlinewidth{0.000000pt}%
\definecolor{currentstroke}{rgb}{0.000000,0.000000,0.000000}%
\pgfsetstrokecolor{currentstroke}%
\pgfsetdash{}{0pt}%
\pgfpathmoveto{\pgfqpoint{3.763681in}{2.314857in}}%
\pgfpathlineto{\pgfqpoint{4.018512in}{2.470906in}}%
\pgfpathlineto{\pgfqpoint{3.763681in}{2.314857in}}%
\pgfpathclose%
\pgfusepath{fill}%
\end{pgfscope}%
\begin{pgfscope}%
\pgfpathrectangle{\pgfqpoint{0.017500in}{0.017500in}}{\pgfqpoint{6.218282in}{3.580372in}} %
\pgfusepath{clip}%
\pgfsetbuttcap%
\pgfsetroundjoin%
\definecolor{currentfill}{rgb}{0.795938,0.241845,0.220830}%
\pgfsetfillcolor{currentfill}%
\pgfsetlinewidth{0.000000pt}%
\definecolor{currentstroke}{rgb}{0.000000,0.000000,0.000000}%
\pgfsetstrokecolor{currentstroke}%
\pgfsetdash{}{0pt}%
\pgfpathmoveto{\pgfqpoint{3.763681in}{2.314857in}}%
\pgfpathlineto{\pgfqpoint{4.018512in}{2.470906in}}%
\pgfpathlineto{\pgfqpoint{3.763681in}{2.314857in}}%
\pgfpathclose%
\pgfusepath{fill}%
\end{pgfscope}%
\begin{pgfscope}%
\pgfpathrectangle{\pgfqpoint{0.017500in}{0.017500in}}{\pgfqpoint{6.218282in}{3.580372in}} %
\pgfusepath{clip}%
\pgfsetbuttcap%
\pgfsetroundjoin%
\definecolor{currentfill}{rgb}{0.795938,0.241845,0.220830}%
\pgfsetfillcolor{currentfill}%
\pgfsetlinewidth{0.000000pt}%
\definecolor{currentstroke}{rgb}{0.000000,0.000000,0.000000}%
\pgfsetstrokecolor{currentstroke}%
\pgfsetdash{}{0pt}%
\pgfpathmoveto{\pgfqpoint{3.763681in}{2.314857in}}%
\pgfpathlineto{\pgfqpoint{4.018512in}{2.470906in}}%
\pgfpathlineto{\pgfqpoint{3.763681in}{2.314857in}}%
\pgfpathclose%
\pgfusepath{fill}%
\end{pgfscope}%
\begin{pgfscope}%
\pgfpathrectangle{\pgfqpoint{0.017500in}{0.017500in}}{\pgfqpoint{6.218282in}{3.580372in}} %
\pgfusepath{clip}%
\pgfsetbuttcap%
\pgfsetroundjoin%
\definecolor{currentfill}{rgb}{0.795938,0.241845,0.220830}%
\pgfsetfillcolor{currentfill}%
\pgfsetlinewidth{0.000000pt}%
\definecolor{currentstroke}{rgb}{0.000000,0.000000,0.000000}%
\pgfsetstrokecolor{currentstroke}%
\pgfsetdash{}{0pt}%
\pgfpathmoveto{\pgfqpoint{3.763681in}{2.314857in}}%
\pgfpathlineto{\pgfqpoint{4.018512in}{2.470906in}}%
\pgfpathlineto{\pgfqpoint{3.763681in}{2.314857in}}%
\pgfpathclose%
\pgfusepath{fill}%
\end{pgfscope}%
\begin{pgfscope}%
\pgfpathrectangle{\pgfqpoint{0.017500in}{0.017500in}}{\pgfqpoint{6.218282in}{3.580372in}} %
\pgfusepath{clip}%
\pgfsetbuttcap%
\pgfsetroundjoin%
\definecolor{currentfill}{rgb}{0.795938,0.241845,0.220830}%
\pgfsetfillcolor{currentfill}%
\pgfsetlinewidth{0.000000pt}%
\definecolor{currentstroke}{rgb}{0.000000,0.000000,0.000000}%
\pgfsetstrokecolor{currentstroke}%
\pgfsetdash{}{0pt}%
\pgfpathmoveto{\pgfqpoint{3.763681in}{2.314857in}}%
\pgfpathlineto{\pgfqpoint{4.018512in}{2.470906in}}%
\pgfpathlineto{\pgfqpoint{3.763681in}{2.314857in}}%
\pgfpathclose%
\pgfusepath{fill}%
\end{pgfscope}%
\begin{pgfscope}%
\pgfpathrectangle{\pgfqpoint{0.017500in}{0.017500in}}{\pgfqpoint{6.218282in}{3.580372in}} %
\pgfusepath{clip}%
\pgfsetbuttcap%
\pgfsetroundjoin%
\definecolor{currentfill}{rgb}{0.795938,0.241845,0.220830}%
\pgfsetfillcolor{currentfill}%
\pgfsetlinewidth{0.000000pt}%
\definecolor{currentstroke}{rgb}{0.000000,0.000000,0.000000}%
\pgfsetstrokecolor{currentstroke}%
\pgfsetdash{}{0pt}%
\pgfpathmoveto{\pgfqpoint{3.763681in}{2.314857in}}%
\pgfpathlineto{\pgfqpoint{4.018512in}{2.470906in}}%
\pgfpathlineto{\pgfqpoint{3.763681in}{2.314857in}}%
\pgfpathclose%
\pgfusepath{fill}%
\end{pgfscope}%
\begin{pgfscope}%
\pgfpathrectangle{\pgfqpoint{0.017500in}{0.017500in}}{\pgfqpoint{6.218282in}{3.580372in}} %
\pgfusepath{clip}%
\pgfsetbuttcap%
\pgfsetroundjoin%
\definecolor{currentfill}{rgb}{0.795938,0.241845,0.220830}%
\pgfsetfillcolor{currentfill}%
\pgfsetlinewidth{0.000000pt}%
\definecolor{currentstroke}{rgb}{0.000000,0.000000,0.000000}%
\pgfsetstrokecolor{currentstroke}%
\pgfsetdash{}{0pt}%
\pgfpathmoveto{\pgfqpoint{3.763681in}{2.314857in}}%
\pgfpathlineto{\pgfqpoint{4.018512in}{2.470906in}}%
\pgfpathlineto{\pgfqpoint{3.763681in}{2.314857in}}%
\pgfpathclose%
\pgfusepath{fill}%
\end{pgfscope}%
\begin{pgfscope}%
\pgfpathrectangle{\pgfqpoint{0.017500in}{0.017500in}}{\pgfqpoint{6.218282in}{3.580372in}} %
\pgfusepath{clip}%
\pgfsetbuttcap%
\pgfsetroundjoin%
\definecolor{currentfill}{rgb}{0.795938,0.241845,0.220830}%
\pgfsetfillcolor{currentfill}%
\pgfsetlinewidth{0.000000pt}%
\definecolor{currentstroke}{rgb}{0.000000,0.000000,0.000000}%
\pgfsetstrokecolor{currentstroke}%
\pgfsetdash{}{0pt}%
\pgfpathmoveto{\pgfqpoint{3.763681in}{2.314857in}}%
\pgfpathlineto{\pgfqpoint{4.018512in}{2.470906in}}%
\pgfpathlineto{\pgfqpoint{3.763681in}{2.314857in}}%
\pgfpathclose%
\pgfusepath{fill}%
\end{pgfscope}%
\begin{pgfscope}%
\pgfpathrectangle{\pgfqpoint{0.017500in}{0.017500in}}{\pgfqpoint{6.218282in}{3.580372in}} %
\pgfusepath{clip}%
\pgfsetbuttcap%
\pgfsetroundjoin%
\definecolor{currentfill}{rgb}{0.795938,0.241845,0.220830}%
\pgfsetfillcolor{currentfill}%
\pgfsetlinewidth{0.000000pt}%
\definecolor{currentstroke}{rgb}{0.000000,0.000000,0.000000}%
\pgfsetstrokecolor{currentstroke}%
\pgfsetdash{}{0pt}%
\pgfpathmoveto{\pgfqpoint{3.763681in}{2.314857in}}%
\pgfpathlineto{\pgfqpoint{4.018512in}{2.470906in}}%
\pgfpathlineto{\pgfqpoint{3.763681in}{2.314857in}}%
\pgfpathclose%
\pgfusepath{fill}%
\end{pgfscope}%
\begin{pgfscope}%
\pgfpathrectangle{\pgfqpoint{0.017500in}{0.017500in}}{\pgfqpoint{6.218282in}{3.580372in}} %
\pgfusepath{clip}%
\pgfsetbuttcap%
\pgfsetroundjoin%
\definecolor{currentfill}{rgb}{0.795938,0.241845,0.220830}%
\pgfsetfillcolor{currentfill}%
\pgfsetlinewidth{0.000000pt}%
\definecolor{currentstroke}{rgb}{0.000000,0.000000,0.000000}%
\pgfsetstrokecolor{currentstroke}%
\pgfsetdash{}{0pt}%
\pgfpathmoveto{\pgfqpoint{3.763681in}{2.314857in}}%
\pgfpathlineto{\pgfqpoint{4.018512in}{2.470906in}}%
\pgfpathlineto{\pgfqpoint{3.763681in}{2.314857in}}%
\pgfpathclose%
\pgfusepath{fill}%
\end{pgfscope}%
\begin{pgfscope}%
\pgfpathrectangle{\pgfqpoint{0.017500in}{0.017500in}}{\pgfqpoint{6.218282in}{3.580372in}} %
\pgfusepath{clip}%
\pgfsetbuttcap%
\pgfsetroundjoin%
\definecolor{currentfill}{rgb}{0.795938,0.241845,0.220830}%
\pgfsetfillcolor{currentfill}%
\pgfsetlinewidth{0.000000pt}%
\definecolor{currentstroke}{rgb}{0.000000,0.000000,0.000000}%
\pgfsetstrokecolor{currentstroke}%
\pgfsetdash{}{0pt}%
\pgfpathmoveto{\pgfqpoint{3.763681in}{2.314857in}}%
\pgfpathlineto{\pgfqpoint{4.018512in}{2.470906in}}%
\pgfpathlineto{\pgfqpoint{3.763681in}{2.314857in}}%
\pgfpathclose%
\pgfusepath{fill}%
\end{pgfscope}%
\begin{pgfscope}%
\pgfpathrectangle{\pgfqpoint{0.017500in}{0.017500in}}{\pgfqpoint{6.218282in}{3.580372in}} %
\pgfusepath{clip}%
\pgfsetbuttcap%
\pgfsetroundjoin%
\definecolor{currentfill}{rgb}{0.795938,0.241845,0.220830}%
\pgfsetfillcolor{currentfill}%
\pgfsetlinewidth{0.000000pt}%
\definecolor{currentstroke}{rgb}{0.000000,0.000000,0.000000}%
\pgfsetstrokecolor{currentstroke}%
\pgfsetdash{}{0pt}%
\pgfpathmoveto{\pgfqpoint{3.763681in}{2.314857in}}%
\pgfpathlineto{\pgfqpoint{4.018512in}{2.470906in}}%
\pgfpathlineto{\pgfqpoint{3.763681in}{2.314857in}}%
\pgfpathclose%
\pgfusepath{fill}%
\end{pgfscope}%
\begin{pgfscope}%
\pgfpathrectangle{\pgfqpoint{0.017500in}{0.017500in}}{\pgfqpoint{6.218282in}{3.580372in}} %
\pgfusepath{clip}%
\pgfsetbuttcap%
\pgfsetroundjoin%
\definecolor{currentfill}{rgb}{0.795938,0.241845,0.220830}%
\pgfsetfillcolor{currentfill}%
\pgfsetlinewidth{0.000000pt}%
\definecolor{currentstroke}{rgb}{0.000000,0.000000,0.000000}%
\pgfsetstrokecolor{currentstroke}%
\pgfsetdash{}{0pt}%
\pgfpathmoveto{\pgfqpoint{3.763681in}{2.314857in}}%
\pgfpathlineto{\pgfqpoint{4.018512in}{2.470906in}}%
\pgfpathlineto{\pgfqpoint{3.763681in}{2.314857in}}%
\pgfpathclose%
\pgfusepath{fill}%
\end{pgfscope}%
\begin{pgfscope}%
\pgfpathrectangle{\pgfqpoint{0.017500in}{0.017500in}}{\pgfqpoint{6.218282in}{3.580372in}} %
\pgfusepath{clip}%
\pgfsetbuttcap%
\pgfsetroundjoin%
\definecolor{currentfill}{rgb}{0.795938,0.241845,0.220830}%
\pgfsetfillcolor{currentfill}%
\pgfsetlinewidth{0.000000pt}%
\definecolor{currentstroke}{rgb}{0.000000,0.000000,0.000000}%
\pgfsetstrokecolor{currentstroke}%
\pgfsetdash{}{0pt}%
\pgfpathmoveto{\pgfqpoint{3.763681in}{2.314857in}}%
\pgfpathlineto{\pgfqpoint{4.018512in}{2.470906in}}%
\pgfpathlineto{\pgfqpoint{3.763681in}{2.314857in}}%
\pgfpathclose%
\pgfusepath{fill}%
\end{pgfscope}%
\begin{pgfscope}%
\pgfpathrectangle{\pgfqpoint{0.017500in}{0.017500in}}{\pgfqpoint{6.218282in}{3.580372in}} %
\pgfusepath{clip}%
\pgfsetbuttcap%
\pgfsetroundjoin%
\definecolor{currentfill}{rgb}{0.795938,0.241845,0.220830}%
\pgfsetfillcolor{currentfill}%
\pgfsetlinewidth{0.000000pt}%
\definecolor{currentstroke}{rgb}{0.000000,0.000000,0.000000}%
\pgfsetstrokecolor{currentstroke}%
\pgfsetdash{}{0pt}%
\pgfpathmoveto{\pgfqpoint{3.763681in}{2.314857in}}%
\pgfpathlineto{\pgfqpoint{4.018512in}{2.470906in}}%
\pgfpathlineto{\pgfqpoint{3.763681in}{2.314857in}}%
\pgfpathclose%
\pgfusepath{fill}%
\end{pgfscope}%
\begin{pgfscope}%
\pgfpathrectangle{\pgfqpoint{0.017500in}{0.017500in}}{\pgfqpoint{6.218282in}{3.580372in}} %
\pgfusepath{clip}%
\pgfsetbuttcap%
\pgfsetroundjoin%
\definecolor{currentfill}{rgb}{0.795938,0.241845,0.220830}%
\pgfsetfillcolor{currentfill}%
\pgfsetlinewidth{0.000000pt}%
\definecolor{currentstroke}{rgb}{0.000000,0.000000,0.000000}%
\pgfsetstrokecolor{currentstroke}%
\pgfsetdash{}{0pt}%
\pgfpathmoveto{\pgfqpoint{3.763681in}{2.314857in}}%
\pgfpathlineto{\pgfqpoint{4.018512in}{2.470906in}}%
\pgfpathlineto{\pgfqpoint{3.763681in}{2.314857in}}%
\pgfpathclose%
\pgfusepath{fill}%
\end{pgfscope}%
\begin{pgfscope}%
\pgfpathrectangle{\pgfqpoint{0.017500in}{0.017500in}}{\pgfqpoint{6.218282in}{3.580372in}} %
\pgfusepath{clip}%
\pgfsetbuttcap%
\pgfsetroundjoin%
\definecolor{currentfill}{rgb}{0.795938,0.241845,0.220830}%
\pgfsetfillcolor{currentfill}%
\pgfsetlinewidth{0.000000pt}%
\definecolor{currentstroke}{rgb}{0.000000,0.000000,0.000000}%
\pgfsetstrokecolor{currentstroke}%
\pgfsetdash{}{0pt}%
\pgfpathmoveto{\pgfqpoint{3.763681in}{2.314857in}}%
\pgfpathlineto{\pgfqpoint{4.018512in}{2.470906in}}%
\pgfpathlineto{\pgfqpoint{3.763681in}{2.314857in}}%
\pgfpathclose%
\pgfusepath{fill}%
\end{pgfscope}%
\begin{pgfscope}%
\pgfpathrectangle{\pgfqpoint{0.017500in}{0.017500in}}{\pgfqpoint{6.218282in}{3.580372in}} %
\pgfusepath{clip}%
\pgfsetbuttcap%
\pgfsetroundjoin%
\definecolor{currentfill}{rgb}{0.795938,0.241845,0.220830}%
\pgfsetfillcolor{currentfill}%
\pgfsetlinewidth{0.000000pt}%
\definecolor{currentstroke}{rgb}{0.000000,0.000000,0.000000}%
\pgfsetstrokecolor{currentstroke}%
\pgfsetdash{}{0pt}%
\pgfpathmoveto{\pgfqpoint{3.763681in}{2.314857in}}%
\pgfpathlineto{\pgfqpoint{4.018512in}{2.470906in}}%
\pgfpathlineto{\pgfqpoint{3.763681in}{2.314857in}}%
\pgfpathclose%
\pgfusepath{fill}%
\end{pgfscope}%
\begin{pgfscope}%
\pgfpathrectangle{\pgfqpoint{0.017500in}{0.017500in}}{\pgfqpoint{6.218282in}{3.580372in}} %
\pgfusepath{clip}%
\pgfsetbuttcap%
\pgfsetroundjoin%
\definecolor{currentfill}{rgb}{0.795938,0.241845,0.220830}%
\pgfsetfillcolor{currentfill}%
\pgfsetlinewidth{0.000000pt}%
\definecolor{currentstroke}{rgb}{0.000000,0.000000,0.000000}%
\pgfsetstrokecolor{currentstroke}%
\pgfsetdash{}{0pt}%
\pgfpathmoveto{\pgfqpoint{3.763681in}{2.314857in}}%
\pgfpathlineto{\pgfqpoint{4.018512in}{2.470906in}}%
\pgfpathlineto{\pgfqpoint{3.763681in}{2.314857in}}%
\pgfpathclose%
\pgfusepath{fill}%
\end{pgfscope}%
\begin{pgfscope}%
\pgfpathrectangle{\pgfqpoint{0.017500in}{0.017500in}}{\pgfqpoint{6.218282in}{3.580372in}} %
\pgfusepath{clip}%
\pgfsetbuttcap%
\pgfsetroundjoin%
\definecolor{currentfill}{rgb}{0.348323,0.465711,0.888346}%
\pgfsetfillcolor{currentfill}%
\pgfsetlinewidth{0.000000pt}%
\definecolor{currentstroke}{rgb}{0.000000,0.000000,0.000000}%
\pgfsetstrokecolor{currentstroke}%
\pgfsetdash{}{0pt}%
\pgfpathmoveto{\pgfqpoint{3.236097in}{0.721171in}}%
\pgfpathlineto{\pgfqpoint{3.688410in}{0.645663in}}%
\pgfpathlineto{\pgfqpoint{3.968799in}{2.026158in}}%
\pgfpathlineto{\pgfqpoint{3.502717in}{2.091450in}}%
\pgfpathlineto{\pgfqpoint{3.236097in}{0.721171in}}%
\pgfpathclose%
\pgfusepath{fill}%
\end{pgfscope}%
\begin{pgfscope}%
\pgfpathrectangle{\pgfqpoint{0.017500in}{0.017500in}}{\pgfqpoint{6.218282in}{3.580372in}} %
\pgfusepath{clip}%
\pgfsetbuttcap%
\pgfsetroundjoin%
\definecolor{currentfill}{rgb}{0.348323,0.465711,0.888346}%
\pgfsetfillcolor{currentfill}%
\pgfsetlinewidth{0.000000pt}%
\definecolor{currentstroke}{rgb}{0.000000,0.000000,0.000000}%
\pgfsetstrokecolor{currentstroke}%
\pgfsetdash{}{0pt}%
\pgfpathmoveto{\pgfqpoint{3.236097in}{0.721171in}}%
\pgfpathlineto{\pgfqpoint{3.688410in}{0.645663in}}%
\pgfpathlineto{\pgfqpoint{3.968799in}{2.026158in}}%
\pgfpathlineto{\pgfqpoint{3.502717in}{2.091450in}}%
\pgfpathlineto{\pgfqpoint{3.236097in}{0.721171in}}%
\pgfpathclose%
\pgfusepath{fill}%
\end{pgfscope}%
\begin{pgfscope}%
\pgfpathrectangle{\pgfqpoint{0.017500in}{0.017500in}}{\pgfqpoint{6.218282in}{3.580372in}} %
\pgfusepath{clip}%
\pgfsetbuttcap%
\pgfsetroundjoin%
\definecolor{currentfill}{rgb}{0.348323,0.465711,0.888346}%
\pgfsetfillcolor{currentfill}%
\pgfsetlinewidth{0.000000pt}%
\definecolor{currentstroke}{rgb}{0.000000,0.000000,0.000000}%
\pgfsetstrokecolor{currentstroke}%
\pgfsetdash{}{0pt}%
\pgfpathmoveto{\pgfqpoint{3.236097in}{0.721171in}}%
\pgfpathlineto{\pgfqpoint{3.688410in}{0.645663in}}%
\pgfpathlineto{\pgfqpoint{3.968799in}{2.026158in}}%
\pgfpathlineto{\pgfqpoint{3.502717in}{2.091450in}}%
\pgfpathlineto{\pgfqpoint{3.236097in}{0.721171in}}%
\pgfpathclose%
\pgfusepath{fill}%
\end{pgfscope}%
\begin{pgfscope}%
\pgfpathrectangle{\pgfqpoint{0.017500in}{0.017500in}}{\pgfqpoint{6.218282in}{3.580372in}} %
\pgfusepath{clip}%
\pgfsetbuttcap%
\pgfsetroundjoin%
\definecolor{currentfill}{rgb}{0.348323,0.465711,0.888346}%
\pgfsetfillcolor{currentfill}%
\pgfsetlinewidth{0.000000pt}%
\definecolor{currentstroke}{rgb}{0.000000,0.000000,0.000000}%
\pgfsetstrokecolor{currentstroke}%
\pgfsetdash{}{0pt}%
\pgfpathmoveto{\pgfqpoint{3.236097in}{0.721171in}}%
\pgfpathlineto{\pgfqpoint{3.688410in}{0.645663in}}%
\pgfpathlineto{\pgfqpoint{3.968799in}{2.026158in}}%
\pgfpathlineto{\pgfqpoint{3.502717in}{2.091450in}}%
\pgfpathlineto{\pgfqpoint{3.236097in}{0.721171in}}%
\pgfpathclose%
\pgfusepath{fill}%
\end{pgfscope}%
\begin{pgfscope}%
\pgfpathrectangle{\pgfqpoint{0.017500in}{0.017500in}}{\pgfqpoint{6.218282in}{3.580372in}} %
\pgfusepath{clip}%
\pgfsetbuttcap%
\pgfsetroundjoin%
\definecolor{currentfill}{rgb}{0.348323,0.465711,0.888346}%
\pgfsetfillcolor{currentfill}%
\pgfsetlinewidth{0.000000pt}%
\definecolor{currentstroke}{rgb}{0.000000,0.000000,0.000000}%
\pgfsetstrokecolor{currentstroke}%
\pgfsetdash{}{0pt}%
\pgfpathmoveto{\pgfqpoint{3.236097in}{0.721171in}}%
\pgfpathlineto{\pgfqpoint{3.688410in}{0.645663in}}%
\pgfpathlineto{\pgfqpoint{3.968799in}{2.026158in}}%
\pgfpathlineto{\pgfqpoint{3.502717in}{2.091450in}}%
\pgfpathlineto{\pgfqpoint{3.236097in}{0.721171in}}%
\pgfpathclose%
\pgfusepath{fill}%
\end{pgfscope}%
\begin{pgfscope}%
\pgfpathrectangle{\pgfqpoint{0.017500in}{0.017500in}}{\pgfqpoint{6.218282in}{3.580372in}} %
\pgfusepath{clip}%
\pgfsetbuttcap%
\pgfsetroundjoin%
\definecolor{currentfill}{rgb}{0.348323,0.465711,0.888346}%
\pgfsetfillcolor{currentfill}%
\pgfsetlinewidth{0.000000pt}%
\definecolor{currentstroke}{rgb}{0.000000,0.000000,0.000000}%
\pgfsetstrokecolor{currentstroke}%
\pgfsetdash{}{0pt}%
\pgfpathmoveto{\pgfqpoint{3.236097in}{0.721171in}}%
\pgfpathlineto{\pgfqpoint{3.688410in}{0.645663in}}%
\pgfpathlineto{\pgfqpoint{3.968799in}{2.026158in}}%
\pgfpathlineto{\pgfqpoint{3.502717in}{2.091450in}}%
\pgfpathlineto{\pgfqpoint{3.236097in}{0.721171in}}%
\pgfpathclose%
\pgfusepath{fill}%
\end{pgfscope}%
\begin{pgfscope}%
\pgfpathrectangle{\pgfqpoint{0.017500in}{0.017500in}}{\pgfqpoint{6.218282in}{3.580372in}} %
\pgfusepath{clip}%
\pgfsetbuttcap%
\pgfsetroundjoin%
\definecolor{currentfill}{rgb}{0.790562,0.231397,0.216242}%
\pgfsetfillcolor{currentfill}%
\pgfsetlinewidth{0.000000pt}%
\definecolor{currentstroke}{rgb}{0.000000,0.000000,0.000000}%
\pgfsetstrokecolor{currentstroke}%
\pgfsetdash{}{0pt}%
\pgfpathmoveto{\pgfqpoint{3.763681in}{2.314857in}}%
\pgfpathlineto{\pgfqpoint{4.227140in}{2.242353in}}%
\pgfpathlineto{\pgfqpoint{4.478797in}{2.402809in}}%
\pgfpathlineto{\pgfqpoint{4.018512in}{2.470906in}}%
\pgfpathlineto{\pgfqpoint{3.763681in}{2.314857in}}%
\pgfpathclose%
\pgfusepath{fill}%
\end{pgfscope}%
\begin{pgfscope}%
\pgfpathrectangle{\pgfqpoint{0.017500in}{0.017500in}}{\pgfqpoint{6.218282in}{3.580372in}} %
\pgfusepath{clip}%
\pgfsetbuttcap%
\pgfsetroundjoin%
\definecolor{currentfill}{rgb}{0.790562,0.231397,0.216242}%
\pgfsetfillcolor{currentfill}%
\pgfsetlinewidth{0.000000pt}%
\definecolor{currentstroke}{rgb}{0.000000,0.000000,0.000000}%
\pgfsetstrokecolor{currentstroke}%
\pgfsetdash{}{0pt}%
\pgfpathmoveto{\pgfqpoint{3.763681in}{2.314857in}}%
\pgfpathlineto{\pgfqpoint{4.227140in}{2.242353in}}%
\pgfpathlineto{\pgfqpoint{4.478797in}{2.402809in}}%
\pgfpathlineto{\pgfqpoint{4.018512in}{2.470906in}}%
\pgfpathlineto{\pgfqpoint{3.763681in}{2.314857in}}%
\pgfpathclose%
\pgfusepath{fill}%
\end{pgfscope}%
\begin{pgfscope}%
\pgfpathrectangle{\pgfqpoint{0.017500in}{0.017500in}}{\pgfqpoint{6.218282in}{3.580372in}} %
\pgfusepath{clip}%
\pgfsetbuttcap%
\pgfsetroundjoin%
\definecolor{currentfill}{rgb}{0.790562,0.231397,0.216242}%
\pgfsetfillcolor{currentfill}%
\pgfsetlinewidth{0.000000pt}%
\definecolor{currentstroke}{rgb}{0.000000,0.000000,0.000000}%
\pgfsetstrokecolor{currentstroke}%
\pgfsetdash{}{0pt}%
\pgfpathmoveto{\pgfqpoint{3.763681in}{2.314857in}}%
\pgfpathlineto{\pgfqpoint{4.227140in}{2.242353in}}%
\pgfpathlineto{\pgfqpoint{4.478797in}{2.402809in}}%
\pgfpathlineto{\pgfqpoint{4.018512in}{2.470906in}}%
\pgfpathlineto{\pgfqpoint{3.763681in}{2.314857in}}%
\pgfpathclose%
\pgfusepath{fill}%
\end{pgfscope}%
\begin{pgfscope}%
\pgfpathrectangle{\pgfqpoint{0.017500in}{0.017500in}}{\pgfqpoint{6.218282in}{3.580372in}} %
\pgfusepath{clip}%
\pgfsetbuttcap%
\pgfsetroundjoin%
\definecolor{currentfill}{rgb}{0.790562,0.231397,0.216242}%
\pgfsetfillcolor{currentfill}%
\pgfsetlinewidth{0.000000pt}%
\definecolor{currentstroke}{rgb}{0.000000,0.000000,0.000000}%
\pgfsetstrokecolor{currentstroke}%
\pgfsetdash{}{0pt}%
\pgfpathmoveto{\pgfqpoint{3.763681in}{2.314857in}}%
\pgfpathlineto{\pgfqpoint{4.227140in}{2.242353in}}%
\pgfpathlineto{\pgfqpoint{4.478797in}{2.402809in}}%
\pgfpathlineto{\pgfqpoint{4.018512in}{2.470906in}}%
\pgfpathlineto{\pgfqpoint{3.763681in}{2.314857in}}%
\pgfpathclose%
\pgfusepath{fill}%
\end{pgfscope}%
\begin{pgfscope}%
\pgfpathrectangle{\pgfqpoint{0.017500in}{0.017500in}}{\pgfqpoint{6.218282in}{3.580372in}} %
\pgfusepath{clip}%
\pgfsetbuttcap%
\pgfsetroundjoin%
\definecolor{currentfill}{rgb}{0.790562,0.231397,0.216242}%
\pgfsetfillcolor{currentfill}%
\pgfsetlinewidth{0.000000pt}%
\definecolor{currentstroke}{rgb}{0.000000,0.000000,0.000000}%
\pgfsetstrokecolor{currentstroke}%
\pgfsetdash{}{0pt}%
\pgfpathmoveto{\pgfqpoint{3.763681in}{2.314857in}}%
\pgfpathlineto{\pgfqpoint{4.227140in}{2.242353in}}%
\pgfpathlineto{\pgfqpoint{4.478797in}{2.402809in}}%
\pgfpathlineto{\pgfqpoint{4.018512in}{2.470906in}}%
\pgfpathlineto{\pgfqpoint{3.763681in}{2.314857in}}%
\pgfpathclose%
\pgfusepath{fill}%
\end{pgfscope}%
\begin{pgfscope}%
\pgfpathrectangle{\pgfqpoint{0.017500in}{0.017500in}}{\pgfqpoint{6.218282in}{3.580372in}} %
\pgfusepath{clip}%
\pgfsetbuttcap%
\pgfsetroundjoin%
\definecolor{currentfill}{rgb}{0.790562,0.231397,0.216242}%
\pgfsetfillcolor{currentfill}%
\pgfsetlinewidth{0.000000pt}%
\definecolor{currentstroke}{rgb}{0.000000,0.000000,0.000000}%
\pgfsetstrokecolor{currentstroke}%
\pgfsetdash{}{0pt}%
\pgfpathmoveto{\pgfqpoint{3.763681in}{2.314857in}}%
\pgfpathlineto{\pgfqpoint{4.227140in}{2.242353in}}%
\pgfpathlineto{\pgfqpoint{4.478797in}{2.402809in}}%
\pgfpathlineto{\pgfqpoint{4.018512in}{2.470906in}}%
\pgfpathlineto{\pgfqpoint{3.763681in}{2.314857in}}%
\pgfpathclose%
\pgfusepath{fill}%
\end{pgfscope}%
\begin{pgfscope}%
\pgfpathrectangle{\pgfqpoint{0.017500in}{0.017500in}}{\pgfqpoint{6.218282in}{3.580372in}} %
\pgfusepath{clip}%
\pgfsetbuttcap%
\pgfsetroundjoin%
\definecolor{currentfill}{rgb}{0.252663,0.332837,0.783665}%
\pgfsetfillcolor{currentfill}%
\pgfsetlinewidth{0.000000pt}%
\definecolor{currentstroke}{rgb}{0.000000,0.000000,0.000000}%
\pgfsetstrokecolor{currentstroke}%
\pgfsetdash{}{0pt}%
\pgfpathmoveto{\pgfqpoint{3.688410in}{0.645663in}}%
\pgfpathlineto{\pgfqpoint{3.968799in}{2.026158in}}%
\pgfpathlineto{\pgfqpoint{3.688410in}{0.645663in}}%
\pgfpathclose%
\pgfusepath{fill}%
\end{pgfscope}%
\begin{pgfscope}%
\pgfpathrectangle{\pgfqpoint{0.017500in}{0.017500in}}{\pgfqpoint{6.218282in}{3.580372in}} %
\pgfusepath{clip}%
\pgfsetbuttcap%
\pgfsetroundjoin%
\definecolor{currentfill}{rgb}{0.252663,0.332837,0.783665}%
\pgfsetfillcolor{currentfill}%
\pgfsetlinewidth{0.000000pt}%
\definecolor{currentstroke}{rgb}{0.000000,0.000000,0.000000}%
\pgfsetstrokecolor{currentstroke}%
\pgfsetdash{}{0pt}%
\pgfpathmoveto{\pgfqpoint{3.688410in}{0.645663in}}%
\pgfpathlineto{\pgfqpoint{3.968799in}{2.026158in}}%
\pgfpathlineto{\pgfqpoint{3.688410in}{0.645663in}}%
\pgfpathclose%
\pgfusepath{fill}%
\end{pgfscope}%
\begin{pgfscope}%
\pgfpathrectangle{\pgfqpoint{0.017500in}{0.017500in}}{\pgfqpoint{6.218282in}{3.580372in}} %
\pgfusepath{clip}%
\pgfsetbuttcap%
\pgfsetroundjoin%
\definecolor{currentfill}{rgb}{0.252663,0.332837,0.783665}%
\pgfsetfillcolor{currentfill}%
\pgfsetlinewidth{0.000000pt}%
\definecolor{currentstroke}{rgb}{0.000000,0.000000,0.000000}%
\pgfsetstrokecolor{currentstroke}%
\pgfsetdash{}{0pt}%
\pgfpathmoveto{\pgfqpoint{3.688410in}{0.645663in}}%
\pgfpathlineto{\pgfqpoint{3.968799in}{2.026158in}}%
\pgfpathlineto{\pgfqpoint{3.688410in}{0.645663in}}%
\pgfpathclose%
\pgfusepath{fill}%
\end{pgfscope}%
\begin{pgfscope}%
\pgfpathrectangle{\pgfqpoint{0.017500in}{0.017500in}}{\pgfqpoint{6.218282in}{3.580372in}} %
\pgfusepath{clip}%
\pgfsetbuttcap%
\pgfsetroundjoin%
\definecolor{currentfill}{rgb}{0.252663,0.332837,0.783665}%
\pgfsetfillcolor{currentfill}%
\pgfsetlinewidth{0.000000pt}%
\definecolor{currentstroke}{rgb}{0.000000,0.000000,0.000000}%
\pgfsetstrokecolor{currentstroke}%
\pgfsetdash{}{0pt}%
\pgfpathmoveto{\pgfqpoint{3.688410in}{0.645663in}}%
\pgfpathlineto{\pgfqpoint{3.968799in}{2.026158in}}%
\pgfpathlineto{\pgfqpoint{3.688410in}{0.645663in}}%
\pgfpathclose%
\pgfusepath{fill}%
\end{pgfscope}%
\begin{pgfscope}%
\pgfpathrectangle{\pgfqpoint{0.017500in}{0.017500in}}{\pgfqpoint{6.218282in}{3.580372in}} %
\pgfusepath{clip}%
\pgfsetbuttcap%
\pgfsetroundjoin%
\definecolor{currentfill}{rgb}{0.252663,0.332837,0.783665}%
\pgfsetfillcolor{currentfill}%
\pgfsetlinewidth{0.000000pt}%
\definecolor{currentstroke}{rgb}{0.000000,0.000000,0.000000}%
\pgfsetstrokecolor{currentstroke}%
\pgfsetdash{}{0pt}%
\pgfpathmoveto{\pgfqpoint{3.688410in}{0.645663in}}%
\pgfpathlineto{\pgfqpoint{3.968799in}{2.026158in}}%
\pgfpathlineto{\pgfqpoint{3.688410in}{0.645663in}}%
\pgfpathclose%
\pgfusepath{fill}%
\end{pgfscope}%
\begin{pgfscope}%
\pgfpathrectangle{\pgfqpoint{0.017500in}{0.017500in}}{\pgfqpoint{6.218282in}{3.580372in}} %
\pgfusepath{clip}%
\pgfsetbuttcap%
\pgfsetroundjoin%
\definecolor{currentfill}{rgb}{0.252663,0.332837,0.783665}%
\pgfsetfillcolor{currentfill}%
\pgfsetlinewidth{0.000000pt}%
\definecolor{currentstroke}{rgb}{0.000000,0.000000,0.000000}%
\pgfsetstrokecolor{currentstroke}%
\pgfsetdash{}{0pt}%
\pgfpathmoveto{\pgfqpoint{3.688410in}{0.645663in}}%
\pgfpathlineto{\pgfqpoint{3.968799in}{2.026158in}}%
\pgfpathlineto{\pgfqpoint{3.688410in}{0.645663in}}%
\pgfpathclose%
\pgfusepath{fill}%
\end{pgfscope}%
\begin{pgfscope}%
\pgfpathrectangle{\pgfqpoint{0.017500in}{0.017500in}}{\pgfqpoint{6.218282in}{3.580372in}} %
\pgfusepath{clip}%
\pgfsetbuttcap%
\pgfsetroundjoin%
\definecolor{currentfill}{rgb}{0.252663,0.332837,0.783665}%
\pgfsetfillcolor{currentfill}%
\pgfsetlinewidth{0.000000pt}%
\definecolor{currentstroke}{rgb}{0.000000,0.000000,0.000000}%
\pgfsetstrokecolor{currentstroke}%
\pgfsetdash{}{0pt}%
\pgfpathmoveto{\pgfqpoint{3.688410in}{0.645663in}}%
\pgfpathlineto{\pgfqpoint{3.968799in}{2.026158in}}%
\pgfpathlineto{\pgfqpoint{3.688410in}{0.645663in}}%
\pgfpathclose%
\pgfusepath{fill}%
\end{pgfscope}%
\begin{pgfscope}%
\pgfpathrectangle{\pgfqpoint{0.017500in}{0.017500in}}{\pgfqpoint{6.218282in}{3.580372in}} %
\pgfusepath{clip}%
\pgfsetbuttcap%
\pgfsetroundjoin%
\definecolor{currentfill}{rgb}{0.252663,0.332837,0.783665}%
\pgfsetfillcolor{currentfill}%
\pgfsetlinewidth{0.000000pt}%
\definecolor{currentstroke}{rgb}{0.000000,0.000000,0.000000}%
\pgfsetstrokecolor{currentstroke}%
\pgfsetdash{}{0pt}%
\pgfpathmoveto{\pgfqpoint{3.688410in}{0.645663in}}%
\pgfpathlineto{\pgfqpoint{3.968799in}{2.026158in}}%
\pgfpathlineto{\pgfqpoint{3.688410in}{0.645663in}}%
\pgfpathclose%
\pgfusepath{fill}%
\end{pgfscope}%
\begin{pgfscope}%
\pgfpathrectangle{\pgfqpoint{0.017500in}{0.017500in}}{\pgfqpoint{6.218282in}{3.580372in}} %
\pgfusepath{clip}%
\pgfsetbuttcap%
\pgfsetroundjoin%
\definecolor{currentfill}{rgb}{0.252663,0.332837,0.783665}%
\pgfsetfillcolor{currentfill}%
\pgfsetlinewidth{0.000000pt}%
\definecolor{currentstroke}{rgb}{0.000000,0.000000,0.000000}%
\pgfsetstrokecolor{currentstroke}%
\pgfsetdash{}{0pt}%
\pgfpathmoveto{\pgfqpoint{3.688410in}{0.645663in}}%
\pgfpathlineto{\pgfqpoint{3.968799in}{2.026158in}}%
\pgfpathlineto{\pgfqpoint{3.688410in}{0.645663in}}%
\pgfpathclose%
\pgfusepath{fill}%
\end{pgfscope}%
\begin{pgfscope}%
\pgfpathrectangle{\pgfqpoint{0.017500in}{0.017500in}}{\pgfqpoint{6.218282in}{3.580372in}} %
\pgfusepath{clip}%
\pgfsetbuttcap%
\pgfsetroundjoin%
\definecolor{currentfill}{rgb}{0.252663,0.332837,0.783665}%
\pgfsetfillcolor{currentfill}%
\pgfsetlinewidth{0.000000pt}%
\definecolor{currentstroke}{rgb}{0.000000,0.000000,0.000000}%
\pgfsetstrokecolor{currentstroke}%
\pgfsetdash{}{0pt}%
\pgfpathmoveto{\pgfqpoint{3.688410in}{0.645663in}}%
\pgfpathlineto{\pgfqpoint{3.968799in}{2.026158in}}%
\pgfpathlineto{\pgfqpoint{3.688410in}{0.645663in}}%
\pgfpathclose%
\pgfusepath{fill}%
\end{pgfscope}%
\begin{pgfscope}%
\pgfpathrectangle{\pgfqpoint{0.017500in}{0.017500in}}{\pgfqpoint{6.218282in}{3.580372in}} %
\pgfusepath{clip}%
\pgfsetbuttcap%
\pgfsetroundjoin%
\definecolor{currentfill}{rgb}{0.252663,0.332837,0.783665}%
\pgfsetfillcolor{currentfill}%
\pgfsetlinewidth{0.000000pt}%
\definecolor{currentstroke}{rgb}{0.000000,0.000000,0.000000}%
\pgfsetstrokecolor{currentstroke}%
\pgfsetdash{}{0pt}%
\pgfpathmoveto{\pgfqpoint{3.688410in}{0.645663in}}%
\pgfpathlineto{\pgfqpoint{3.968799in}{2.026158in}}%
\pgfpathlineto{\pgfqpoint{3.688410in}{0.645663in}}%
\pgfpathclose%
\pgfusepath{fill}%
\end{pgfscope}%
\begin{pgfscope}%
\pgfpathrectangle{\pgfqpoint{0.017500in}{0.017500in}}{\pgfqpoint{6.218282in}{3.580372in}} %
\pgfusepath{clip}%
\pgfsetbuttcap%
\pgfsetroundjoin%
\definecolor{currentfill}{rgb}{0.252663,0.332837,0.783665}%
\pgfsetfillcolor{currentfill}%
\pgfsetlinewidth{0.000000pt}%
\definecolor{currentstroke}{rgb}{0.000000,0.000000,0.000000}%
\pgfsetstrokecolor{currentstroke}%
\pgfsetdash{}{0pt}%
\pgfpathmoveto{\pgfqpoint{3.688410in}{0.645663in}}%
\pgfpathlineto{\pgfqpoint{3.968799in}{2.026158in}}%
\pgfpathlineto{\pgfqpoint{3.688410in}{0.645663in}}%
\pgfpathclose%
\pgfusepath{fill}%
\end{pgfscope}%
\begin{pgfscope}%
\pgfpathrectangle{\pgfqpoint{0.017500in}{0.017500in}}{\pgfqpoint{6.218282in}{3.580372in}} %
\pgfusepath{clip}%
\pgfsetbuttcap%
\pgfsetroundjoin%
\definecolor{currentfill}{rgb}{0.252663,0.332837,0.783665}%
\pgfsetfillcolor{currentfill}%
\pgfsetlinewidth{0.000000pt}%
\definecolor{currentstroke}{rgb}{0.000000,0.000000,0.000000}%
\pgfsetstrokecolor{currentstroke}%
\pgfsetdash{}{0pt}%
\pgfpathmoveto{\pgfqpoint{3.688410in}{0.645663in}}%
\pgfpathlineto{\pgfqpoint{3.968799in}{2.026158in}}%
\pgfpathlineto{\pgfqpoint{3.688410in}{0.645663in}}%
\pgfpathclose%
\pgfusepath{fill}%
\end{pgfscope}%
\begin{pgfscope}%
\pgfpathrectangle{\pgfqpoint{0.017500in}{0.017500in}}{\pgfqpoint{6.218282in}{3.580372in}} %
\pgfusepath{clip}%
\pgfsetbuttcap%
\pgfsetroundjoin%
\definecolor{currentfill}{rgb}{0.252663,0.332837,0.783665}%
\pgfsetfillcolor{currentfill}%
\pgfsetlinewidth{0.000000pt}%
\definecolor{currentstroke}{rgb}{0.000000,0.000000,0.000000}%
\pgfsetstrokecolor{currentstroke}%
\pgfsetdash{}{0pt}%
\pgfpathmoveto{\pgfqpoint{3.688410in}{0.645663in}}%
\pgfpathlineto{\pgfqpoint{3.968799in}{2.026158in}}%
\pgfpathlineto{\pgfqpoint{3.688410in}{0.645663in}}%
\pgfpathclose%
\pgfusepath{fill}%
\end{pgfscope}%
\begin{pgfscope}%
\pgfpathrectangle{\pgfqpoint{0.017500in}{0.017500in}}{\pgfqpoint{6.218282in}{3.580372in}} %
\pgfusepath{clip}%
\pgfsetbuttcap%
\pgfsetroundjoin%
\definecolor{currentfill}{rgb}{0.252663,0.332837,0.783665}%
\pgfsetfillcolor{currentfill}%
\pgfsetlinewidth{0.000000pt}%
\definecolor{currentstroke}{rgb}{0.000000,0.000000,0.000000}%
\pgfsetstrokecolor{currentstroke}%
\pgfsetdash{}{0pt}%
\pgfpathmoveto{\pgfqpoint{3.688410in}{0.645663in}}%
\pgfpathlineto{\pgfqpoint{3.968799in}{2.026158in}}%
\pgfpathlineto{\pgfqpoint{3.688410in}{0.645663in}}%
\pgfpathclose%
\pgfusepath{fill}%
\end{pgfscope}%
\begin{pgfscope}%
\pgfpathrectangle{\pgfqpoint{0.017500in}{0.017500in}}{\pgfqpoint{6.218282in}{3.580372in}} %
\pgfusepath{clip}%
\pgfsetbuttcap%
\pgfsetroundjoin%
\definecolor{currentfill}{rgb}{0.252663,0.332837,0.783665}%
\pgfsetfillcolor{currentfill}%
\pgfsetlinewidth{0.000000pt}%
\definecolor{currentstroke}{rgb}{0.000000,0.000000,0.000000}%
\pgfsetstrokecolor{currentstroke}%
\pgfsetdash{}{0pt}%
\pgfpathmoveto{\pgfqpoint{3.688410in}{0.645663in}}%
\pgfpathlineto{\pgfqpoint{3.968799in}{2.026158in}}%
\pgfpathlineto{\pgfqpoint{3.688410in}{0.645663in}}%
\pgfpathclose%
\pgfusepath{fill}%
\end{pgfscope}%
\begin{pgfscope}%
\pgfpathrectangle{\pgfqpoint{0.017500in}{0.017500in}}{\pgfqpoint{6.218282in}{3.580372in}} %
\pgfusepath{clip}%
\pgfsetbuttcap%
\pgfsetroundjoin%
\definecolor{currentfill}{rgb}{0.252663,0.332837,0.783665}%
\pgfsetfillcolor{currentfill}%
\pgfsetlinewidth{0.000000pt}%
\definecolor{currentstroke}{rgb}{0.000000,0.000000,0.000000}%
\pgfsetstrokecolor{currentstroke}%
\pgfsetdash{}{0pt}%
\pgfpathmoveto{\pgfqpoint{3.688410in}{0.645663in}}%
\pgfpathlineto{\pgfqpoint{3.968799in}{2.026158in}}%
\pgfpathlineto{\pgfqpoint{3.688410in}{0.645663in}}%
\pgfpathclose%
\pgfusepath{fill}%
\end{pgfscope}%
\begin{pgfscope}%
\pgfpathrectangle{\pgfqpoint{0.017500in}{0.017500in}}{\pgfqpoint{6.218282in}{3.580372in}} %
\pgfusepath{clip}%
\pgfsetbuttcap%
\pgfsetroundjoin%
\definecolor{currentfill}{rgb}{0.252663,0.332837,0.783665}%
\pgfsetfillcolor{currentfill}%
\pgfsetlinewidth{0.000000pt}%
\definecolor{currentstroke}{rgb}{0.000000,0.000000,0.000000}%
\pgfsetstrokecolor{currentstroke}%
\pgfsetdash{}{0pt}%
\pgfpathmoveto{\pgfqpoint{3.688410in}{0.645663in}}%
\pgfpathlineto{\pgfqpoint{3.968799in}{2.026158in}}%
\pgfpathlineto{\pgfqpoint{3.688410in}{0.645663in}}%
\pgfpathclose%
\pgfusepath{fill}%
\end{pgfscope}%
\begin{pgfscope}%
\pgfpathrectangle{\pgfqpoint{0.017500in}{0.017500in}}{\pgfqpoint{6.218282in}{3.580372in}} %
\pgfusepath{clip}%
\pgfsetbuttcap%
\pgfsetroundjoin%
\definecolor{currentfill}{rgb}{0.252663,0.332837,0.783665}%
\pgfsetfillcolor{currentfill}%
\pgfsetlinewidth{0.000000pt}%
\definecolor{currentstroke}{rgb}{0.000000,0.000000,0.000000}%
\pgfsetstrokecolor{currentstroke}%
\pgfsetdash{}{0pt}%
\pgfpathmoveto{\pgfqpoint{3.688410in}{0.645663in}}%
\pgfpathlineto{\pgfqpoint{3.968799in}{2.026158in}}%
\pgfpathlineto{\pgfqpoint{3.688410in}{0.645663in}}%
\pgfpathclose%
\pgfusepath{fill}%
\end{pgfscope}%
\begin{pgfscope}%
\pgfpathrectangle{\pgfqpoint{0.017500in}{0.017500in}}{\pgfqpoint{6.218282in}{3.580372in}} %
\pgfusepath{clip}%
\pgfsetbuttcap%
\pgfsetroundjoin%
\definecolor{currentfill}{rgb}{0.252663,0.332837,0.783665}%
\pgfsetfillcolor{currentfill}%
\pgfsetlinewidth{0.000000pt}%
\definecolor{currentstroke}{rgb}{0.000000,0.000000,0.000000}%
\pgfsetstrokecolor{currentstroke}%
\pgfsetdash{}{0pt}%
\pgfpathmoveto{\pgfqpoint{3.688410in}{0.645663in}}%
\pgfpathlineto{\pgfqpoint{3.968799in}{2.026158in}}%
\pgfpathlineto{\pgfqpoint{3.688410in}{0.645663in}}%
\pgfpathclose%
\pgfusepath{fill}%
\end{pgfscope}%
\begin{pgfscope}%
\pgfpathrectangle{\pgfqpoint{0.017500in}{0.017500in}}{\pgfqpoint{6.218282in}{3.580372in}} %
\pgfusepath{clip}%
\pgfsetbuttcap%
\pgfsetroundjoin%
\definecolor{currentfill}{rgb}{0.252663,0.332837,0.783665}%
\pgfsetfillcolor{currentfill}%
\pgfsetlinewidth{0.000000pt}%
\definecolor{currentstroke}{rgb}{0.000000,0.000000,0.000000}%
\pgfsetstrokecolor{currentstroke}%
\pgfsetdash{}{0pt}%
\pgfpathmoveto{\pgfqpoint{3.688410in}{0.645663in}}%
\pgfpathlineto{\pgfqpoint{3.968799in}{2.026158in}}%
\pgfpathlineto{\pgfqpoint{3.688410in}{0.645663in}}%
\pgfpathclose%
\pgfusepath{fill}%
\end{pgfscope}%
\begin{pgfscope}%
\pgfpathrectangle{\pgfqpoint{0.017500in}{0.017500in}}{\pgfqpoint{6.218282in}{3.580372in}} %
\pgfusepath{clip}%
\pgfsetbuttcap%
\pgfsetroundjoin%
\definecolor{currentfill}{rgb}{0.252663,0.332837,0.783665}%
\pgfsetfillcolor{currentfill}%
\pgfsetlinewidth{0.000000pt}%
\definecolor{currentstroke}{rgb}{0.000000,0.000000,0.000000}%
\pgfsetstrokecolor{currentstroke}%
\pgfsetdash{}{0pt}%
\pgfpathmoveto{\pgfqpoint{3.688410in}{0.645663in}}%
\pgfpathlineto{\pgfqpoint{3.968799in}{2.026158in}}%
\pgfpathlineto{\pgfqpoint{3.688410in}{0.645663in}}%
\pgfpathclose%
\pgfusepath{fill}%
\end{pgfscope}%
\begin{pgfscope}%
\pgfpathrectangle{\pgfqpoint{0.017500in}{0.017500in}}{\pgfqpoint{6.218282in}{3.580372in}} %
\pgfusepath{clip}%
\pgfsetbuttcap%
\pgfsetroundjoin%
\definecolor{currentfill}{rgb}{0.252663,0.332837,0.783665}%
\pgfsetfillcolor{currentfill}%
\pgfsetlinewidth{0.000000pt}%
\definecolor{currentstroke}{rgb}{0.000000,0.000000,0.000000}%
\pgfsetstrokecolor{currentstroke}%
\pgfsetdash{}{0pt}%
\pgfpathmoveto{\pgfqpoint{3.688410in}{0.645663in}}%
\pgfpathlineto{\pgfqpoint{3.968799in}{2.026158in}}%
\pgfpathlineto{\pgfqpoint{3.688410in}{0.645663in}}%
\pgfpathclose%
\pgfusepath{fill}%
\end{pgfscope}%
\begin{pgfscope}%
\pgfpathrectangle{\pgfqpoint{0.017500in}{0.017500in}}{\pgfqpoint{6.218282in}{3.580372in}} %
\pgfusepath{clip}%
\pgfsetbuttcap%
\pgfsetroundjoin%
\definecolor{currentfill}{rgb}{0.252663,0.332837,0.783665}%
\pgfsetfillcolor{currentfill}%
\pgfsetlinewidth{0.000000pt}%
\definecolor{currentstroke}{rgb}{0.000000,0.000000,0.000000}%
\pgfsetstrokecolor{currentstroke}%
\pgfsetdash{}{0pt}%
\pgfpathmoveto{\pgfqpoint{3.688410in}{0.645663in}}%
\pgfpathlineto{\pgfqpoint{3.968799in}{2.026158in}}%
\pgfpathlineto{\pgfqpoint{3.688410in}{0.645663in}}%
\pgfpathclose%
\pgfusepath{fill}%
\end{pgfscope}%
\begin{pgfscope}%
\pgfpathrectangle{\pgfqpoint{0.017500in}{0.017500in}}{\pgfqpoint{6.218282in}{3.580372in}} %
\pgfusepath{clip}%
\pgfsetbuttcap%
\pgfsetroundjoin%
\definecolor{currentfill}{rgb}{0.252663,0.332837,0.783665}%
\pgfsetfillcolor{currentfill}%
\pgfsetlinewidth{0.000000pt}%
\definecolor{currentstroke}{rgb}{0.000000,0.000000,0.000000}%
\pgfsetstrokecolor{currentstroke}%
\pgfsetdash{}{0pt}%
\pgfpathmoveto{\pgfqpoint{3.688410in}{0.645663in}}%
\pgfpathlineto{\pgfqpoint{3.968799in}{2.026158in}}%
\pgfpathlineto{\pgfqpoint{3.688410in}{0.645663in}}%
\pgfpathclose%
\pgfusepath{fill}%
\end{pgfscope}%
\begin{pgfscope}%
\pgfpathrectangle{\pgfqpoint{0.017500in}{0.017500in}}{\pgfqpoint{6.218282in}{3.580372in}} %
\pgfusepath{clip}%
\pgfsetbuttcap%
\pgfsetroundjoin%
\definecolor{currentfill}{rgb}{0.252663,0.332837,0.783665}%
\pgfsetfillcolor{currentfill}%
\pgfsetlinewidth{0.000000pt}%
\definecolor{currentstroke}{rgb}{0.000000,0.000000,0.000000}%
\pgfsetstrokecolor{currentstroke}%
\pgfsetdash{}{0pt}%
\pgfpathmoveto{\pgfqpoint{3.688410in}{0.645663in}}%
\pgfpathlineto{\pgfqpoint{3.968799in}{2.026158in}}%
\pgfpathlineto{\pgfqpoint{3.688410in}{0.645663in}}%
\pgfpathclose%
\pgfusepath{fill}%
\end{pgfscope}%
\begin{pgfscope}%
\pgfpathrectangle{\pgfqpoint{0.017500in}{0.017500in}}{\pgfqpoint{6.218282in}{3.580372in}} %
\pgfusepath{clip}%
\pgfsetbuttcap%
\pgfsetroundjoin%
\definecolor{currentfill}{rgb}{0.252663,0.332837,0.783665}%
\pgfsetfillcolor{currentfill}%
\pgfsetlinewidth{0.000000pt}%
\definecolor{currentstroke}{rgb}{0.000000,0.000000,0.000000}%
\pgfsetstrokecolor{currentstroke}%
\pgfsetdash{}{0pt}%
\pgfpathmoveto{\pgfqpoint{3.688410in}{0.645663in}}%
\pgfpathlineto{\pgfqpoint{3.968799in}{2.026158in}}%
\pgfpathlineto{\pgfqpoint{3.688410in}{0.645663in}}%
\pgfpathclose%
\pgfusepath{fill}%
\end{pgfscope}%
\begin{pgfscope}%
\pgfpathrectangle{\pgfqpoint{0.017500in}{0.017500in}}{\pgfqpoint{6.218282in}{3.580372in}} %
\pgfusepath{clip}%
\pgfsetbuttcap%
\pgfsetroundjoin%
\definecolor{currentfill}{rgb}{0.252663,0.332837,0.783665}%
\pgfsetfillcolor{currentfill}%
\pgfsetlinewidth{0.000000pt}%
\definecolor{currentstroke}{rgb}{0.000000,0.000000,0.000000}%
\pgfsetstrokecolor{currentstroke}%
\pgfsetdash{}{0pt}%
\pgfpathmoveto{\pgfqpoint{3.688410in}{0.645663in}}%
\pgfpathlineto{\pgfqpoint{3.968799in}{2.026158in}}%
\pgfpathlineto{\pgfqpoint{3.688410in}{0.645663in}}%
\pgfpathclose%
\pgfusepath{fill}%
\end{pgfscope}%
\begin{pgfscope}%
\pgfpathrectangle{\pgfqpoint{0.017500in}{0.017500in}}{\pgfqpoint{6.218282in}{3.580372in}} %
\pgfusepath{clip}%
\pgfsetbuttcap%
\pgfsetroundjoin%
\definecolor{currentfill}{rgb}{0.252663,0.332837,0.783665}%
\pgfsetfillcolor{currentfill}%
\pgfsetlinewidth{0.000000pt}%
\definecolor{currentstroke}{rgb}{0.000000,0.000000,0.000000}%
\pgfsetstrokecolor{currentstroke}%
\pgfsetdash{}{0pt}%
\pgfpathmoveto{\pgfqpoint{3.688410in}{0.645663in}}%
\pgfpathlineto{\pgfqpoint{3.968799in}{2.026158in}}%
\pgfpathlineto{\pgfqpoint{3.688410in}{0.645663in}}%
\pgfpathclose%
\pgfusepath{fill}%
\end{pgfscope}%
\begin{pgfscope}%
\pgfpathrectangle{\pgfqpoint{0.017500in}{0.017500in}}{\pgfqpoint{6.218282in}{3.580372in}} %
\pgfusepath{clip}%
\pgfsetbuttcap%
\pgfsetroundjoin%
\definecolor{currentfill}{rgb}{0.252663,0.332837,0.783665}%
\pgfsetfillcolor{currentfill}%
\pgfsetlinewidth{0.000000pt}%
\definecolor{currentstroke}{rgb}{0.000000,0.000000,0.000000}%
\pgfsetstrokecolor{currentstroke}%
\pgfsetdash{}{0pt}%
\pgfpathmoveto{\pgfqpoint{3.688410in}{0.645663in}}%
\pgfpathlineto{\pgfqpoint{3.968799in}{2.026158in}}%
\pgfpathlineto{\pgfqpoint{3.688410in}{0.645663in}}%
\pgfpathclose%
\pgfusepath{fill}%
\end{pgfscope}%
\begin{pgfscope}%
\pgfpathrectangle{\pgfqpoint{0.017500in}{0.017500in}}{\pgfqpoint{6.218282in}{3.580372in}} %
\pgfusepath{clip}%
\pgfsetbuttcap%
\pgfsetroundjoin%
\definecolor{currentfill}{rgb}{0.873402,0.386960,0.306332}%
\pgfsetfillcolor{currentfill}%
\pgfsetlinewidth{0.000000pt}%
\definecolor{currentstroke}{rgb}{0.000000,0.000000,0.000000}%
\pgfsetstrokecolor{currentstroke}%
\pgfsetdash{}{0pt}%
\pgfpathmoveto{\pgfqpoint{3.043201in}{2.166904in}}%
\pgfpathlineto{\pgfqpoint{3.306522in}{2.395850in}}%
\pgfpathlineto{\pgfqpoint{3.043201in}{2.166904in}}%
\pgfpathclose%
\pgfusepath{fill}%
\end{pgfscope}%
\begin{pgfscope}%
\pgfpathrectangle{\pgfqpoint{0.017500in}{0.017500in}}{\pgfqpoint{6.218282in}{3.580372in}} %
\pgfusepath{clip}%
\pgfsetbuttcap%
\pgfsetroundjoin%
\definecolor{currentfill}{rgb}{0.873402,0.386960,0.306332}%
\pgfsetfillcolor{currentfill}%
\pgfsetlinewidth{0.000000pt}%
\definecolor{currentstroke}{rgb}{0.000000,0.000000,0.000000}%
\pgfsetstrokecolor{currentstroke}%
\pgfsetdash{}{0pt}%
\pgfpathmoveto{\pgfqpoint{3.043201in}{2.166904in}}%
\pgfpathlineto{\pgfqpoint{3.306522in}{2.395850in}}%
\pgfpathlineto{\pgfqpoint{3.043201in}{2.166904in}}%
\pgfpathclose%
\pgfusepath{fill}%
\end{pgfscope}%
\begin{pgfscope}%
\pgfpathrectangle{\pgfqpoint{0.017500in}{0.017500in}}{\pgfqpoint{6.218282in}{3.580372in}} %
\pgfusepath{clip}%
\pgfsetbuttcap%
\pgfsetroundjoin%
\definecolor{currentfill}{rgb}{0.873402,0.386960,0.306332}%
\pgfsetfillcolor{currentfill}%
\pgfsetlinewidth{0.000000pt}%
\definecolor{currentstroke}{rgb}{0.000000,0.000000,0.000000}%
\pgfsetstrokecolor{currentstroke}%
\pgfsetdash{}{0pt}%
\pgfpathmoveto{\pgfqpoint{3.043201in}{2.166904in}}%
\pgfpathlineto{\pgfqpoint{3.306522in}{2.395850in}}%
\pgfpathlineto{\pgfqpoint{3.043201in}{2.166904in}}%
\pgfpathclose%
\pgfusepath{fill}%
\end{pgfscope}%
\begin{pgfscope}%
\pgfpathrectangle{\pgfqpoint{0.017500in}{0.017500in}}{\pgfqpoint{6.218282in}{3.580372in}} %
\pgfusepath{clip}%
\pgfsetbuttcap%
\pgfsetroundjoin%
\definecolor{currentfill}{rgb}{0.873402,0.386960,0.306332}%
\pgfsetfillcolor{currentfill}%
\pgfsetlinewidth{0.000000pt}%
\definecolor{currentstroke}{rgb}{0.000000,0.000000,0.000000}%
\pgfsetstrokecolor{currentstroke}%
\pgfsetdash{}{0pt}%
\pgfpathmoveto{\pgfqpoint{3.043201in}{2.166904in}}%
\pgfpathlineto{\pgfqpoint{3.306522in}{2.395850in}}%
\pgfpathlineto{\pgfqpoint{3.043201in}{2.166904in}}%
\pgfpathclose%
\pgfusepath{fill}%
\end{pgfscope}%
\begin{pgfscope}%
\pgfpathrectangle{\pgfqpoint{0.017500in}{0.017500in}}{\pgfqpoint{6.218282in}{3.580372in}} %
\pgfusepath{clip}%
\pgfsetbuttcap%
\pgfsetroundjoin%
\definecolor{currentfill}{rgb}{0.873402,0.386960,0.306332}%
\pgfsetfillcolor{currentfill}%
\pgfsetlinewidth{0.000000pt}%
\definecolor{currentstroke}{rgb}{0.000000,0.000000,0.000000}%
\pgfsetstrokecolor{currentstroke}%
\pgfsetdash{}{0pt}%
\pgfpathmoveto{\pgfqpoint{3.043201in}{2.166904in}}%
\pgfpathlineto{\pgfqpoint{3.306522in}{2.395850in}}%
\pgfpathlineto{\pgfqpoint{3.043201in}{2.166904in}}%
\pgfpathclose%
\pgfusepath{fill}%
\end{pgfscope}%
\begin{pgfscope}%
\pgfpathrectangle{\pgfqpoint{0.017500in}{0.017500in}}{\pgfqpoint{6.218282in}{3.580372in}} %
\pgfusepath{clip}%
\pgfsetbuttcap%
\pgfsetroundjoin%
\definecolor{currentfill}{rgb}{0.873402,0.386960,0.306332}%
\pgfsetfillcolor{currentfill}%
\pgfsetlinewidth{0.000000pt}%
\definecolor{currentstroke}{rgb}{0.000000,0.000000,0.000000}%
\pgfsetstrokecolor{currentstroke}%
\pgfsetdash{}{0pt}%
\pgfpathmoveto{\pgfqpoint{3.043201in}{2.166904in}}%
\pgfpathlineto{\pgfqpoint{3.306522in}{2.395850in}}%
\pgfpathlineto{\pgfqpoint{3.043201in}{2.166904in}}%
\pgfpathclose%
\pgfusepath{fill}%
\end{pgfscope}%
\begin{pgfscope}%
\pgfpathrectangle{\pgfqpoint{0.017500in}{0.017500in}}{\pgfqpoint{6.218282in}{3.580372in}} %
\pgfusepath{clip}%
\pgfsetbuttcap%
\pgfsetroundjoin%
\definecolor{currentfill}{rgb}{0.873402,0.386960,0.306332}%
\pgfsetfillcolor{currentfill}%
\pgfsetlinewidth{0.000000pt}%
\definecolor{currentstroke}{rgb}{0.000000,0.000000,0.000000}%
\pgfsetstrokecolor{currentstroke}%
\pgfsetdash{}{0pt}%
\pgfpathmoveto{\pgfqpoint{3.043201in}{2.166904in}}%
\pgfpathlineto{\pgfqpoint{3.306522in}{2.395850in}}%
\pgfpathlineto{\pgfqpoint{3.043201in}{2.166904in}}%
\pgfpathclose%
\pgfusepath{fill}%
\end{pgfscope}%
\begin{pgfscope}%
\pgfpathrectangle{\pgfqpoint{0.017500in}{0.017500in}}{\pgfqpoint{6.218282in}{3.580372in}} %
\pgfusepath{clip}%
\pgfsetbuttcap%
\pgfsetroundjoin%
\definecolor{currentfill}{rgb}{0.873402,0.386960,0.306332}%
\pgfsetfillcolor{currentfill}%
\pgfsetlinewidth{0.000000pt}%
\definecolor{currentstroke}{rgb}{0.000000,0.000000,0.000000}%
\pgfsetstrokecolor{currentstroke}%
\pgfsetdash{}{0pt}%
\pgfpathmoveto{\pgfqpoint{3.043201in}{2.166904in}}%
\pgfpathlineto{\pgfqpoint{3.306522in}{2.395850in}}%
\pgfpathlineto{\pgfqpoint{3.043201in}{2.166904in}}%
\pgfpathclose%
\pgfusepath{fill}%
\end{pgfscope}%
\begin{pgfscope}%
\pgfpathrectangle{\pgfqpoint{0.017500in}{0.017500in}}{\pgfqpoint{6.218282in}{3.580372in}} %
\pgfusepath{clip}%
\pgfsetbuttcap%
\pgfsetroundjoin%
\definecolor{currentfill}{rgb}{0.873402,0.386960,0.306332}%
\pgfsetfillcolor{currentfill}%
\pgfsetlinewidth{0.000000pt}%
\definecolor{currentstroke}{rgb}{0.000000,0.000000,0.000000}%
\pgfsetstrokecolor{currentstroke}%
\pgfsetdash{}{0pt}%
\pgfpathmoveto{\pgfqpoint{3.043201in}{2.166904in}}%
\pgfpathlineto{\pgfqpoint{3.306522in}{2.395850in}}%
\pgfpathlineto{\pgfqpoint{3.043201in}{2.166904in}}%
\pgfpathclose%
\pgfusepath{fill}%
\end{pgfscope}%
\begin{pgfscope}%
\pgfpathrectangle{\pgfqpoint{0.017500in}{0.017500in}}{\pgfqpoint{6.218282in}{3.580372in}} %
\pgfusepath{clip}%
\pgfsetbuttcap%
\pgfsetroundjoin%
\definecolor{currentfill}{rgb}{0.873402,0.386960,0.306332}%
\pgfsetfillcolor{currentfill}%
\pgfsetlinewidth{0.000000pt}%
\definecolor{currentstroke}{rgb}{0.000000,0.000000,0.000000}%
\pgfsetstrokecolor{currentstroke}%
\pgfsetdash{}{0pt}%
\pgfpathmoveto{\pgfqpoint{3.043201in}{2.166904in}}%
\pgfpathlineto{\pgfqpoint{3.306522in}{2.395850in}}%
\pgfpathlineto{\pgfqpoint{3.043201in}{2.166904in}}%
\pgfpathclose%
\pgfusepath{fill}%
\end{pgfscope}%
\begin{pgfscope}%
\pgfpathrectangle{\pgfqpoint{0.017500in}{0.017500in}}{\pgfqpoint{6.218282in}{3.580372in}} %
\pgfusepath{clip}%
\pgfsetbuttcap%
\pgfsetroundjoin%
\definecolor{currentfill}{rgb}{0.873402,0.386960,0.306332}%
\pgfsetfillcolor{currentfill}%
\pgfsetlinewidth{0.000000pt}%
\definecolor{currentstroke}{rgb}{0.000000,0.000000,0.000000}%
\pgfsetstrokecolor{currentstroke}%
\pgfsetdash{}{0pt}%
\pgfpathmoveto{\pgfqpoint{3.043201in}{2.166904in}}%
\pgfpathlineto{\pgfqpoint{3.306522in}{2.395850in}}%
\pgfpathlineto{\pgfqpoint{3.043201in}{2.166904in}}%
\pgfpathclose%
\pgfusepath{fill}%
\end{pgfscope}%
\begin{pgfscope}%
\pgfpathrectangle{\pgfqpoint{0.017500in}{0.017500in}}{\pgfqpoint{6.218282in}{3.580372in}} %
\pgfusepath{clip}%
\pgfsetbuttcap%
\pgfsetroundjoin%
\definecolor{currentfill}{rgb}{0.873402,0.386960,0.306332}%
\pgfsetfillcolor{currentfill}%
\pgfsetlinewidth{0.000000pt}%
\definecolor{currentstroke}{rgb}{0.000000,0.000000,0.000000}%
\pgfsetstrokecolor{currentstroke}%
\pgfsetdash{}{0pt}%
\pgfpathmoveto{\pgfqpoint{3.043201in}{2.166904in}}%
\pgfpathlineto{\pgfqpoint{3.306522in}{2.395850in}}%
\pgfpathlineto{\pgfqpoint{3.043201in}{2.166904in}}%
\pgfpathclose%
\pgfusepath{fill}%
\end{pgfscope}%
\begin{pgfscope}%
\pgfpathrectangle{\pgfqpoint{0.017500in}{0.017500in}}{\pgfqpoint{6.218282in}{3.580372in}} %
\pgfusepath{clip}%
\pgfsetbuttcap%
\pgfsetroundjoin%
\definecolor{currentfill}{rgb}{0.873402,0.386960,0.306332}%
\pgfsetfillcolor{currentfill}%
\pgfsetlinewidth{0.000000pt}%
\definecolor{currentstroke}{rgb}{0.000000,0.000000,0.000000}%
\pgfsetstrokecolor{currentstroke}%
\pgfsetdash{}{0pt}%
\pgfpathmoveto{\pgfqpoint{3.043201in}{2.166904in}}%
\pgfpathlineto{\pgfqpoint{3.306522in}{2.395850in}}%
\pgfpathlineto{\pgfqpoint{3.043201in}{2.166904in}}%
\pgfpathclose%
\pgfusepath{fill}%
\end{pgfscope}%
\begin{pgfscope}%
\pgfpathrectangle{\pgfqpoint{0.017500in}{0.017500in}}{\pgfqpoint{6.218282in}{3.580372in}} %
\pgfusepath{clip}%
\pgfsetbuttcap%
\pgfsetroundjoin%
\definecolor{currentfill}{rgb}{0.873402,0.386960,0.306332}%
\pgfsetfillcolor{currentfill}%
\pgfsetlinewidth{0.000000pt}%
\definecolor{currentstroke}{rgb}{0.000000,0.000000,0.000000}%
\pgfsetstrokecolor{currentstroke}%
\pgfsetdash{}{0pt}%
\pgfpathmoveto{\pgfqpoint{3.043201in}{2.166904in}}%
\pgfpathlineto{\pgfqpoint{3.306522in}{2.395850in}}%
\pgfpathlineto{\pgfqpoint{3.043201in}{2.166904in}}%
\pgfpathclose%
\pgfusepath{fill}%
\end{pgfscope}%
\begin{pgfscope}%
\pgfpathrectangle{\pgfqpoint{0.017500in}{0.017500in}}{\pgfqpoint{6.218282in}{3.580372in}} %
\pgfusepath{clip}%
\pgfsetbuttcap%
\pgfsetroundjoin%
\definecolor{currentfill}{rgb}{0.873402,0.386960,0.306332}%
\pgfsetfillcolor{currentfill}%
\pgfsetlinewidth{0.000000pt}%
\definecolor{currentstroke}{rgb}{0.000000,0.000000,0.000000}%
\pgfsetstrokecolor{currentstroke}%
\pgfsetdash{}{0pt}%
\pgfpathmoveto{\pgfqpoint{3.043201in}{2.166904in}}%
\pgfpathlineto{\pgfqpoint{3.306522in}{2.395850in}}%
\pgfpathlineto{\pgfqpoint{3.043201in}{2.166904in}}%
\pgfpathclose%
\pgfusepath{fill}%
\end{pgfscope}%
\begin{pgfscope}%
\pgfpathrectangle{\pgfqpoint{0.017500in}{0.017500in}}{\pgfqpoint{6.218282in}{3.580372in}} %
\pgfusepath{clip}%
\pgfsetbuttcap%
\pgfsetroundjoin%
\definecolor{currentfill}{rgb}{0.873402,0.386960,0.306332}%
\pgfsetfillcolor{currentfill}%
\pgfsetlinewidth{0.000000pt}%
\definecolor{currentstroke}{rgb}{0.000000,0.000000,0.000000}%
\pgfsetstrokecolor{currentstroke}%
\pgfsetdash{}{0pt}%
\pgfpathmoveto{\pgfqpoint{3.043201in}{2.166904in}}%
\pgfpathlineto{\pgfqpoint{3.306522in}{2.395850in}}%
\pgfpathlineto{\pgfqpoint{3.043201in}{2.166904in}}%
\pgfpathclose%
\pgfusepath{fill}%
\end{pgfscope}%
\begin{pgfscope}%
\pgfpathrectangle{\pgfqpoint{0.017500in}{0.017500in}}{\pgfqpoint{6.218282in}{3.580372in}} %
\pgfusepath{clip}%
\pgfsetbuttcap%
\pgfsetroundjoin%
\definecolor{currentfill}{rgb}{0.873402,0.386960,0.306332}%
\pgfsetfillcolor{currentfill}%
\pgfsetlinewidth{0.000000pt}%
\definecolor{currentstroke}{rgb}{0.000000,0.000000,0.000000}%
\pgfsetstrokecolor{currentstroke}%
\pgfsetdash{}{0pt}%
\pgfpathmoveto{\pgfqpoint{3.043201in}{2.166904in}}%
\pgfpathlineto{\pgfqpoint{3.306522in}{2.395850in}}%
\pgfpathlineto{\pgfqpoint{3.043201in}{2.166904in}}%
\pgfpathclose%
\pgfusepath{fill}%
\end{pgfscope}%
\begin{pgfscope}%
\pgfpathrectangle{\pgfqpoint{0.017500in}{0.017500in}}{\pgfqpoint{6.218282in}{3.580372in}} %
\pgfusepath{clip}%
\pgfsetbuttcap%
\pgfsetroundjoin%
\definecolor{currentfill}{rgb}{0.873402,0.386960,0.306332}%
\pgfsetfillcolor{currentfill}%
\pgfsetlinewidth{0.000000pt}%
\definecolor{currentstroke}{rgb}{0.000000,0.000000,0.000000}%
\pgfsetstrokecolor{currentstroke}%
\pgfsetdash{}{0pt}%
\pgfpathmoveto{\pgfqpoint{3.043201in}{2.166904in}}%
\pgfpathlineto{\pgfqpoint{3.306522in}{2.395850in}}%
\pgfpathlineto{\pgfqpoint{3.043201in}{2.166904in}}%
\pgfpathclose%
\pgfusepath{fill}%
\end{pgfscope}%
\begin{pgfscope}%
\pgfpathrectangle{\pgfqpoint{0.017500in}{0.017500in}}{\pgfqpoint{6.218282in}{3.580372in}} %
\pgfusepath{clip}%
\pgfsetbuttcap%
\pgfsetroundjoin%
\definecolor{currentfill}{rgb}{0.873402,0.386960,0.306332}%
\pgfsetfillcolor{currentfill}%
\pgfsetlinewidth{0.000000pt}%
\definecolor{currentstroke}{rgb}{0.000000,0.000000,0.000000}%
\pgfsetstrokecolor{currentstroke}%
\pgfsetdash{}{0pt}%
\pgfpathmoveto{\pgfqpoint{3.043201in}{2.166904in}}%
\pgfpathlineto{\pgfqpoint{3.306522in}{2.395850in}}%
\pgfpathlineto{\pgfqpoint{3.043201in}{2.166904in}}%
\pgfpathclose%
\pgfusepath{fill}%
\end{pgfscope}%
\begin{pgfscope}%
\pgfpathrectangle{\pgfqpoint{0.017500in}{0.017500in}}{\pgfqpoint{6.218282in}{3.580372in}} %
\pgfusepath{clip}%
\pgfsetbuttcap%
\pgfsetroundjoin%
\definecolor{currentfill}{rgb}{0.873402,0.386960,0.306332}%
\pgfsetfillcolor{currentfill}%
\pgfsetlinewidth{0.000000pt}%
\definecolor{currentstroke}{rgb}{0.000000,0.000000,0.000000}%
\pgfsetstrokecolor{currentstroke}%
\pgfsetdash{}{0pt}%
\pgfpathmoveto{\pgfqpoint{3.043201in}{2.166904in}}%
\pgfpathlineto{\pgfqpoint{3.306522in}{2.395850in}}%
\pgfpathlineto{\pgfqpoint{3.043201in}{2.166904in}}%
\pgfpathclose%
\pgfusepath{fill}%
\end{pgfscope}%
\begin{pgfscope}%
\pgfpathrectangle{\pgfqpoint{0.017500in}{0.017500in}}{\pgfqpoint{6.218282in}{3.580372in}} %
\pgfusepath{clip}%
\pgfsetbuttcap%
\pgfsetroundjoin%
\definecolor{currentfill}{rgb}{0.873402,0.386960,0.306332}%
\pgfsetfillcolor{currentfill}%
\pgfsetlinewidth{0.000000pt}%
\definecolor{currentstroke}{rgb}{0.000000,0.000000,0.000000}%
\pgfsetstrokecolor{currentstroke}%
\pgfsetdash{}{0pt}%
\pgfpathmoveto{\pgfqpoint{3.043201in}{2.166904in}}%
\pgfpathlineto{\pgfqpoint{3.306522in}{2.395850in}}%
\pgfpathlineto{\pgfqpoint{3.043201in}{2.166904in}}%
\pgfpathclose%
\pgfusepath{fill}%
\end{pgfscope}%
\begin{pgfscope}%
\pgfpathrectangle{\pgfqpoint{0.017500in}{0.017500in}}{\pgfqpoint{6.218282in}{3.580372in}} %
\pgfusepath{clip}%
\pgfsetbuttcap%
\pgfsetroundjoin%
\definecolor{currentfill}{rgb}{0.873402,0.386960,0.306332}%
\pgfsetfillcolor{currentfill}%
\pgfsetlinewidth{0.000000pt}%
\definecolor{currentstroke}{rgb}{0.000000,0.000000,0.000000}%
\pgfsetstrokecolor{currentstroke}%
\pgfsetdash{}{0pt}%
\pgfpathmoveto{\pgfqpoint{3.043201in}{2.166904in}}%
\pgfpathlineto{\pgfqpoint{3.306522in}{2.395850in}}%
\pgfpathlineto{\pgfqpoint{3.043201in}{2.166904in}}%
\pgfpathclose%
\pgfusepath{fill}%
\end{pgfscope}%
\begin{pgfscope}%
\pgfpathrectangle{\pgfqpoint{0.017500in}{0.017500in}}{\pgfqpoint{6.218282in}{3.580372in}} %
\pgfusepath{clip}%
\pgfsetbuttcap%
\pgfsetroundjoin%
\definecolor{currentfill}{rgb}{0.873402,0.386960,0.306332}%
\pgfsetfillcolor{currentfill}%
\pgfsetlinewidth{0.000000pt}%
\definecolor{currentstroke}{rgb}{0.000000,0.000000,0.000000}%
\pgfsetstrokecolor{currentstroke}%
\pgfsetdash{}{0pt}%
\pgfpathmoveto{\pgfqpoint{3.043201in}{2.166904in}}%
\pgfpathlineto{\pgfqpoint{3.306522in}{2.395850in}}%
\pgfpathlineto{\pgfqpoint{3.043201in}{2.166904in}}%
\pgfpathclose%
\pgfusepath{fill}%
\end{pgfscope}%
\begin{pgfscope}%
\pgfpathrectangle{\pgfqpoint{0.017500in}{0.017500in}}{\pgfqpoint{6.218282in}{3.580372in}} %
\pgfusepath{clip}%
\pgfsetbuttcap%
\pgfsetroundjoin%
\definecolor{currentfill}{rgb}{0.873402,0.386960,0.306332}%
\pgfsetfillcolor{currentfill}%
\pgfsetlinewidth{0.000000pt}%
\definecolor{currentstroke}{rgb}{0.000000,0.000000,0.000000}%
\pgfsetstrokecolor{currentstroke}%
\pgfsetdash{}{0pt}%
\pgfpathmoveto{\pgfqpoint{3.043201in}{2.166904in}}%
\pgfpathlineto{\pgfqpoint{3.306522in}{2.395850in}}%
\pgfpathlineto{\pgfqpoint{3.043201in}{2.166904in}}%
\pgfpathclose%
\pgfusepath{fill}%
\end{pgfscope}%
\begin{pgfscope}%
\pgfpathrectangle{\pgfqpoint{0.017500in}{0.017500in}}{\pgfqpoint{6.218282in}{3.580372in}} %
\pgfusepath{clip}%
\pgfsetbuttcap%
\pgfsetroundjoin%
\definecolor{currentfill}{rgb}{0.873402,0.386960,0.306332}%
\pgfsetfillcolor{currentfill}%
\pgfsetlinewidth{0.000000pt}%
\definecolor{currentstroke}{rgb}{0.000000,0.000000,0.000000}%
\pgfsetstrokecolor{currentstroke}%
\pgfsetdash{}{0pt}%
\pgfpathmoveto{\pgfqpoint{3.043201in}{2.166904in}}%
\pgfpathlineto{\pgfqpoint{3.306522in}{2.395850in}}%
\pgfpathlineto{\pgfqpoint{3.043201in}{2.166904in}}%
\pgfpathclose%
\pgfusepath{fill}%
\end{pgfscope}%
\begin{pgfscope}%
\pgfpathrectangle{\pgfqpoint{0.017500in}{0.017500in}}{\pgfqpoint{6.218282in}{3.580372in}} %
\pgfusepath{clip}%
\pgfsetbuttcap%
\pgfsetroundjoin%
\definecolor{currentfill}{rgb}{0.873402,0.386960,0.306332}%
\pgfsetfillcolor{currentfill}%
\pgfsetlinewidth{0.000000pt}%
\definecolor{currentstroke}{rgb}{0.000000,0.000000,0.000000}%
\pgfsetstrokecolor{currentstroke}%
\pgfsetdash{}{0pt}%
\pgfpathmoveto{\pgfqpoint{3.043201in}{2.166904in}}%
\pgfpathlineto{\pgfqpoint{3.306522in}{2.395850in}}%
\pgfpathlineto{\pgfqpoint{3.043201in}{2.166904in}}%
\pgfpathclose%
\pgfusepath{fill}%
\end{pgfscope}%
\begin{pgfscope}%
\pgfpathrectangle{\pgfqpoint{0.017500in}{0.017500in}}{\pgfqpoint{6.218282in}{3.580372in}} %
\pgfusepath{clip}%
\pgfsetbuttcap%
\pgfsetroundjoin%
\definecolor{currentfill}{rgb}{0.873402,0.386960,0.306332}%
\pgfsetfillcolor{currentfill}%
\pgfsetlinewidth{0.000000pt}%
\definecolor{currentstroke}{rgb}{0.000000,0.000000,0.000000}%
\pgfsetstrokecolor{currentstroke}%
\pgfsetdash{}{0pt}%
\pgfpathmoveto{\pgfqpoint{3.043201in}{2.166904in}}%
\pgfpathlineto{\pgfqpoint{3.306522in}{2.395850in}}%
\pgfpathlineto{\pgfqpoint{3.043201in}{2.166904in}}%
\pgfpathclose%
\pgfusepath{fill}%
\end{pgfscope}%
\begin{pgfscope}%
\pgfpathrectangle{\pgfqpoint{0.017500in}{0.017500in}}{\pgfqpoint{6.218282in}{3.580372in}} %
\pgfusepath{clip}%
\pgfsetbuttcap%
\pgfsetroundjoin%
\definecolor{currentfill}{rgb}{0.873402,0.386960,0.306332}%
\pgfsetfillcolor{currentfill}%
\pgfsetlinewidth{0.000000pt}%
\definecolor{currentstroke}{rgb}{0.000000,0.000000,0.000000}%
\pgfsetstrokecolor{currentstroke}%
\pgfsetdash{}{0pt}%
\pgfpathmoveto{\pgfqpoint{3.043201in}{2.166904in}}%
\pgfpathlineto{\pgfqpoint{3.306522in}{2.395850in}}%
\pgfpathlineto{\pgfqpoint{3.043201in}{2.166904in}}%
\pgfpathclose%
\pgfusepath{fill}%
\end{pgfscope}%
\begin{pgfscope}%
\pgfpathrectangle{\pgfqpoint{0.017500in}{0.017500in}}{\pgfqpoint{6.218282in}{3.580372in}} %
\pgfusepath{clip}%
\pgfsetbuttcap%
\pgfsetroundjoin%
\definecolor{currentfill}{rgb}{0.873402,0.386960,0.306332}%
\pgfsetfillcolor{currentfill}%
\pgfsetlinewidth{0.000000pt}%
\definecolor{currentstroke}{rgb}{0.000000,0.000000,0.000000}%
\pgfsetstrokecolor{currentstroke}%
\pgfsetdash{}{0pt}%
\pgfpathmoveto{\pgfqpoint{3.043201in}{2.166904in}}%
\pgfpathlineto{\pgfqpoint{3.306522in}{2.395850in}}%
\pgfpathlineto{\pgfqpoint{3.043201in}{2.166904in}}%
\pgfpathclose%
\pgfusepath{fill}%
\end{pgfscope}%
\begin{pgfscope}%
\pgfpathrectangle{\pgfqpoint{0.017500in}{0.017500in}}{\pgfqpoint{6.218282in}{3.580372in}} %
\pgfusepath{clip}%
\pgfsetbuttcap%
\pgfsetroundjoin%
\definecolor{currentfill}{rgb}{0.873402,0.386960,0.306332}%
\pgfsetfillcolor{currentfill}%
\pgfsetlinewidth{0.000000pt}%
\definecolor{currentstroke}{rgb}{0.000000,0.000000,0.000000}%
\pgfsetstrokecolor{currentstroke}%
\pgfsetdash{}{0pt}%
\pgfpathmoveto{\pgfqpoint{3.043201in}{2.166904in}}%
\pgfpathlineto{\pgfqpoint{3.306522in}{2.395850in}}%
\pgfpathlineto{\pgfqpoint{3.043201in}{2.166904in}}%
\pgfpathclose%
\pgfusepath{fill}%
\end{pgfscope}%
\begin{pgfscope}%
\pgfpathrectangle{\pgfqpoint{0.017500in}{0.017500in}}{\pgfqpoint{6.218282in}{3.580372in}} %
\pgfusepath{clip}%
\pgfsetbuttcap%
\pgfsetroundjoin%
\definecolor{currentfill}{rgb}{0.869655,0.379274,0.300941}%
\pgfsetfillcolor{currentfill}%
\pgfsetlinewidth{0.000000pt}%
\definecolor{currentstroke}{rgb}{0.000000,0.000000,0.000000}%
\pgfsetstrokecolor{currentstroke}%
\pgfsetdash{}{0pt}%
\pgfpathmoveto{\pgfqpoint{3.043201in}{2.166904in}}%
\pgfpathlineto{\pgfqpoint{3.502717in}{2.091450in}}%
\pgfpathlineto{\pgfqpoint{3.763681in}{2.314857in}}%
\pgfpathlineto{\pgfqpoint{3.306522in}{2.395850in}}%
\pgfpathlineto{\pgfqpoint{3.043201in}{2.166904in}}%
\pgfpathclose%
\pgfusepath{fill}%
\end{pgfscope}%
\begin{pgfscope}%
\pgfpathrectangle{\pgfqpoint{0.017500in}{0.017500in}}{\pgfqpoint{6.218282in}{3.580372in}} %
\pgfusepath{clip}%
\pgfsetbuttcap%
\pgfsetroundjoin%
\definecolor{currentfill}{rgb}{0.869655,0.379274,0.300941}%
\pgfsetfillcolor{currentfill}%
\pgfsetlinewidth{0.000000pt}%
\definecolor{currentstroke}{rgb}{0.000000,0.000000,0.000000}%
\pgfsetstrokecolor{currentstroke}%
\pgfsetdash{}{0pt}%
\pgfpathmoveto{\pgfqpoint{3.043201in}{2.166904in}}%
\pgfpathlineto{\pgfqpoint{3.502717in}{2.091450in}}%
\pgfpathlineto{\pgfqpoint{3.763681in}{2.314857in}}%
\pgfpathlineto{\pgfqpoint{3.306522in}{2.395850in}}%
\pgfpathlineto{\pgfqpoint{3.043201in}{2.166904in}}%
\pgfpathclose%
\pgfusepath{fill}%
\end{pgfscope}%
\begin{pgfscope}%
\pgfpathrectangle{\pgfqpoint{0.017500in}{0.017500in}}{\pgfqpoint{6.218282in}{3.580372in}} %
\pgfusepath{clip}%
\pgfsetbuttcap%
\pgfsetroundjoin%
\definecolor{currentfill}{rgb}{0.869655,0.379274,0.300941}%
\pgfsetfillcolor{currentfill}%
\pgfsetlinewidth{0.000000pt}%
\definecolor{currentstroke}{rgb}{0.000000,0.000000,0.000000}%
\pgfsetstrokecolor{currentstroke}%
\pgfsetdash{}{0pt}%
\pgfpathmoveto{\pgfqpoint{3.043201in}{2.166904in}}%
\pgfpathlineto{\pgfqpoint{3.502717in}{2.091450in}}%
\pgfpathlineto{\pgfqpoint{3.763681in}{2.314857in}}%
\pgfpathlineto{\pgfqpoint{3.306522in}{2.395850in}}%
\pgfpathlineto{\pgfqpoint{3.043201in}{2.166904in}}%
\pgfpathclose%
\pgfusepath{fill}%
\end{pgfscope}%
\begin{pgfscope}%
\pgfpathrectangle{\pgfqpoint{0.017500in}{0.017500in}}{\pgfqpoint{6.218282in}{3.580372in}} %
\pgfusepath{clip}%
\pgfsetbuttcap%
\pgfsetroundjoin%
\definecolor{currentfill}{rgb}{0.869655,0.379274,0.300941}%
\pgfsetfillcolor{currentfill}%
\pgfsetlinewidth{0.000000pt}%
\definecolor{currentstroke}{rgb}{0.000000,0.000000,0.000000}%
\pgfsetstrokecolor{currentstroke}%
\pgfsetdash{}{0pt}%
\pgfpathmoveto{\pgfqpoint{3.043201in}{2.166904in}}%
\pgfpathlineto{\pgfqpoint{3.502717in}{2.091450in}}%
\pgfpathlineto{\pgfqpoint{3.763681in}{2.314857in}}%
\pgfpathlineto{\pgfqpoint{3.306522in}{2.395850in}}%
\pgfpathlineto{\pgfqpoint{3.043201in}{2.166904in}}%
\pgfpathclose%
\pgfusepath{fill}%
\end{pgfscope}%
\begin{pgfscope}%
\pgfpathrectangle{\pgfqpoint{0.017500in}{0.017500in}}{\pgfqpoint{6.218282in}{3.580372in}} %
\pgfusepath{clip}%
\pgfsetbuttcap%
\pgfsetroundjoin%
\definecolor{currentfill}{rgb}{0.869655,0.379274,0.300941}%
\pgfsetfillcolor{currentfill}%
\pgfsetlinewidth{0.000000pt}%
\definecolor{currentstroke}{rgb}{0.000000,0.000000,0.000000}%
\pgfsetstrokecolor{currentstroke}%
\pgfsetdash{}{0pt}%
\pgfpathmoveto{\pgfqpoint{3.043201in}{2.166904in}}%
\pgfpathlineto{\pgfqpoint{3.502717in}{2.091450in}}%
\pgfpathlineto{\pgfqpoint{3.763681in}{2.314857in}}%
\pgfpathlineto{\pgfqpoint{3.306522in}{2.395850in}}%
\pgfpathlineto{\pgfqpoint{3.043201in}{2.166904in}}%
\pgfpathclose%
\pgfusepath{fill}%
\end{pgfscope}%
\begin{pgfscope}%
\pgfpathrectangle{\pgfqpoint{0.017500in}{0.017500in}}{\pgfqpoint{6.218282in}{3.580372in}} %
\pgfusepath{clip}%
\pgfsetbuttcap%
\pgfsetroundjoin%
\definecolor{currentfill}{rgb}{0.869655,0.379274,0.300941}%
\pgfsetfillcolor{currentfill}%
\pgfsetlinewidth{0.000000pt}%
\definecolor{currentstroke}{rgb}{0.000000,0.000000,0.000000}%
\pgfsetstrokecolor{currentstroke}%
\pgfsetdash{}{0pt}%
\pgfpathmoveto{\pgfqpoint{3.043201in}{2.166904in}}%
\pgfpathlineto{\pgfqpoint{3.502717in}{2.091450in}}%
\pgfpathlineto{\pgfqpoint{3.763681in}{2.314857in}}%
\pgfpathlineto{\pgfqpoint{3.306522in}{2.395850in}}%
\pgfpathlineto{\pgfqpoint{3.043201in}{2.166904in}}%
\pgfpathclose%
\pgfusepath{fill}%
\end{pgfscope}%
\begin{pgfscope}%
\pgfpathrectangle{\pgfqpoint{0.017500in}{0.017500in}}{\pgfqpoint{6.218282in}{3.580372in}} %
\pgfusepath{clip}%
\pgfsetbuttcap%
\pgfsetroundjoin%
\definecolor{currentfill}{rgb}{0.790562,0.231397,0.216242}%
\pgfsetfillcolor{currentfill}%
\pgfsetlinewidth{0.000000pt}%
\definecolor{currentstroke}{rgb}{0.000000,0.000000,0.000000}%
\pgfsetstrokecolor{currentstroke}%
\pgfsetdash{}{0pt}%
\pgfpathmoveto{\pgfqpoint{4.227140in}{2.242353in}}%
\pgfpathlineto{\pgfqpoint{4.478797in}{2.402809in}}%
\pgfpathlineto{\pgfqpoint{4.227140in}{2.242353in}}%
\pgfpathclose%
\pgfusepath{fill}%
\end{pgfscope}%
\begin{pgfscope}%
\pgfpathrectangle{\pgfqpoint{0.017500in}{0.017500in}}{\pgfqpoint{6.218282in}{3.580372in}} %
\pgfusepath{clip}%
\pgfsetbuttcap%
\pgfsetroundjoin%
\definecolor{currentfill}{rgb}{0.790562,0.231397,0.216242}%
\pgfsetfillcolor{currentfill}%
\pgfsetlinewidth{0.000000pt}%
\definecolor{currentstroke}{rgb}{0.000000,0.000000,0.000000}%
\pgfsetstrokecolor{currentstroke}%
\pgfsetdash{}{0pt}%
\pgfpathmoveto{\pgfqpoint{4.227140in}{2.242353in}}%
\pgfpathlineto{\pgfqpoint{4.478797in}{2.402809in}}%
\pgfpathlineto{\pgfqpoint{4.227140in}{2.242353in}}%
\pgfpathclose%
\pgfusepath{fill}%
\end{pgfscope}%
\begin{pgfscope}%
\pgfpathrectangle{\pgfqpoint{0.017500in}{0.017500in}}{\pgfqpoint{6.218282in}{3.580372in}} %
\pgfusepath{clip}%
\pgfsetbuttcap%
\pgfsetroundjoin%
\definecolor{currentfill}{rgb}{0.790562,0.231397,0.216242}%
\pgfsetfillcolor{currentfill}%
\pgfsetlinewidth{0.000000pt}%
\definecolor{currentstroke}{rgb}{0.000000,0.000000,0.000000}%
\pgfsetstrokecolor{currentstroke}%
\pgfsetdash{}{0pt}%
\pgfpathmoveto{\pgfqpoint{4.227140in}{2.242353in}}%
\pgfpathlineto{\pgfqpoint{4.478797in}{2.402809in}}%
\pgfpathlineto{\pgfqpoint{4.227140in}{2.242353in}}%
\pgfpathclose%
\pgfusepath{fill}%
\end{pgfscope}%
\begin{pgfscope}%
\pgfpathrectangle{\pgfqpoint{0.017500in}{0.017500in}}{\pgfqpoint{6.218282in}{3.580372in}} %
\pgfusepath{clip}%
\pgfsetbuttcap%
\pgfsetroundjoin%
\definecolor{currentfill}{rgb}{0.790562,0.231397,0.216242}%
\pgfsetfillcolor{currentfill}%
\pgfsetlinewidth{0.000000pt}%
\definecolor{currentstroke}{rgb}{0.000000,0.000000,0.000000}%
\pgfsetstrokecolor{currentstroke}%
\pgfsetdash{}{0pt}%
\pgfpathmoveto{\pgfqpoint{4.227140in}{2.242353in}}%
\pgfpathlineto{\pgfqpoint{4.478797in}{2.402809in}}%
\pgfpathlineto{\pgfqpoint{4.227140in}{2.242353in}}%
\pgfpathclose%
\pgfusepath{fill}%
\end{pgfscope}%
\begin{pgfscope}%
\pgfpathrectangle{\pgfqpoint{0.017500in}{0.017500in}}{\pgfqpoint{6.218282in}{3.580372in}} %
\pgfusepath{clip}%
\pgfsetbuttcap%
\pgfsetroundjoin%
\definecolor{currentfill}{rgb}{0.790562,0.231397,0.216242}%
\pgfsetfillcolor{currentfill}%
\pgfsetlinewidth{0.000000pt}%
\definecolor{currentstroke}{rgb}{0.000000,0.000000,0.000000}%
\pgfsetstrokecolor{currentstroke}%
\pgfsetdash{}{0pt}%
\pgfpathmoveto{\pgfqpoint{4.227140in}{2.242353in}}%
\pgfpathlineto{\pgfqpoint{4.478797in}{2.402809in}}%
\pgfpathlineto{\pgfqpoint{4.227140in}{2.242353in}}%
\pgfpathclose%
\pgfusepath{fill}%
\end{pgfscope}%
\begin{pgfscope}%
\pgfpathrectangle{\pgfqpoint{0.017500in}{0.017500in}}{\pgfqpoint{6.218282in}{3.580372in}} %
\pgfusepath{clip}%
\pgfsetbuttcap%
\pgfsetroundjoin%
\definecolor{currentfill}{rgb}{0.790562,0.231397,0.216242}%
\pgfsetfillcolor{currentfill}%
\pgfsetlinewidth{0.000000pt}%
\definecolor{currentstroke}{rgb}{0.000000,0.000000,0.000000}%
\pgfsetstrokecolor{currentstroke}%
\pgfsetdash{}{0pt}%
\pgfpathmoveto{\pgfqpoint{4.227140in}{2.242353in}}%
\pgfpathlineto{\pgfqpoint{4.478797in}{2.402809in}}%
\pgfpathlineto{\pgfqpoint{4.227140in}{2.242353in}}%
\pgfpathclose%
\pgfusepath{fill}%
\end{pgfscope}%
\begin{pgfscope}%
\pgfpathrectangle{\pgfqpoint{0.017500in}{0.017500in}}{\pgfqpoint{6.218282in}{3.580372in}} %
\pgfusepath{clip}%
\pgfsetbuttcap%
\pgfsetroundjoin%
\definecolor{currentfill}{rgb}{0.790562,0.231397,0.216242}%
\pgfsetfillcolor{currentfill}%
\pgfsetlinewidth{0.000000pt}%
\definecolor{currentstroke}{rgb}{0.000000,0.000000,0.000000}%
\pgfsetstrokecolor{currentstroke}%
\pgfsetdash{}{0pt}%
\pgfpathmoveto{\pgfqpoint{4.227140in}{2.242353in}}%
\pgfpathlineto{\pgfqpoint{4.478797in}{2.402809in}}%
\pgfpathlineto{\pgfqpoint{4.227140in}{2.242353in}}%
\pgfpathclose%
\pgfusepath{fill}%
\end{pgfscope}%
\begin{pgfscope}%
\pgfpathrectangle{\pgfqpoint{0.017500in}{0.017500in}}{\pgfqpoint{6.218282in}{3.580372in}} %
\pgfusepath{clip}%
\pgfsetbuttcap%
\pgfsetroundjoin%
\definecolor{currentfill}{rgb}{0.790562,0.231397,0.216242}%
\pgfsetfillcolor{currentfill}%
\pgfsetlinewidth{0.000000pt}%
\definecolor{currentstroke}{rgb}{0.000000,0.000000,0.000000}%
\pgfsetstrokecolor{currentstroke}%
\pgfsetdash{}{0pt}%
\pgfpathmoveto{\pgfqpoint{4.227140in}{2.242353in}}%
\pgfpathlineto{\pgfqpoint{4.478797in}{2.402809in}}%
\pgfpathlineto{\pgfqpoint{4.227140in}{2.242353in}}%
\pgfpathclose%
\pgfusepath{fill}%
\end{pgfscope}%
\begin{pgfscope}%
\pgfpathrectangle{\pgfqpoint{0.017500in}{0.017500in}}{\pgfqpoint{6.218282in}{3.580372in}} %
\pgfusepath{clip}%
\pgfsetbuttcap%
\pgfsetroundjoin%
\definecolor{currentfill}{rgb}{0.790562,0.231397,0.216242}%
\pgfsetfillcolor{currentfill}%
\pgfsetlinewidth{0.000000pt}%
\definecolor{currentstroke}{rgb}{0.000000,0.000000,0.000000}%
\pgfsetstrokecolor{currentstroke}%
\pgfsetdash{}{0pt}%
\pgfpathmoveto{\pgfqpoint{4.227140in}{2.242353in}}%
\pgfpathlineto{\pgfqpoint{4.478797in}{2.402809in}}%
\pgfpathlineto{\pgfqpoint{4.227140in}{2.242353in}}%
\pgfpathclose%
\pgfusepath{fill}%
\end{pgfscope}%
\begin{pgfscope}%
\pgfpathrectangle{\pgfqpoint{0.017500in}{0.017500in}}{\pgfqpoint{6.218282in}{3.580372in}} %
\pgfusepath{clip}%
\pgfsetbuttcap%
\pgfsetroundjoin%
\definecolor{currentfill}{rgb}{0.790562,0.231397,0.216242}%
\pgfsetfillcolor{currentfill}%
\pgfsetlinewidth{0.000000pt}%
\definecolor{currentstroke}{rgb}{0.000000,0.000000,0.000000}%
\pgfsetstrokecolor{currentstroke}%
\pgfsetdash{}{0pt}%
\pgfpathmoveto{\pgfqpoint{4.227140in}{2.242353in}}%
\pgfpathlineto{\pgfqpoint{4.478797in}{2.402809in}}%
\pgfpathlineto{\pgfqpoint{4.227140in}{2.242353in}}%
\pgfpathclose%
\pgfusepath{fill}%
\end{pgfscope}%
\begin{pgfscope}%
\pgfpathrectangle{\pgfqpoint{0.017500in}{0.017500in}}{\pgfqpoint{6.218282in}{3.580372in}} %
\pgfusepath{clip}%
\pgfsetbuttcap%
\pgfsetroundjoin%
\definecolor{currentfill}{rgb}{0.790562,0.231397,0.216242}%
\pgfsetfillcolor{currentfill}%
\pgfsetlinewidth{0.000000pt}%
\definecolor{currentstroke}{rgb}{0.000000,0.000000,0.000000}%
\pgfsetstrokecolor{currentstroke}%
\pgfsetdash{}{0pt}%
\pgfpathmoveto{\pgfqpoint{4.227140in}{2.242353in}}%
\pgfpathlineto{\pgfqpoint{4.478797in}{2.402809in}}%
\pgfpathlineto{\pgfqpoint{4.227140in}{2.242353in}}%
\pgfpathclose%
\pgfusepath{fill}%
\end{pgfscope}%
\begin{pgfscope}%
\pgfpathrectangle{\pgfqpoint{0.017500in}{0.017500in}}{\pgfqpoint{6.218282in}{3.580372in}} %
\pgfusepath{clip}%
\pgfsetbuttcap%
\pgfsetroundjoin%
\definecolor{currentfill}{rgb}{0.790562,0.231397,0.216242}%
\pgfsetfillcolor{currentfill}%
\pgfsetlinewidth{0.000000pt}%
\definecolor{currentstroke}{rgb}{0.000000,0.000000,0.000000}%
\pgfsetstrokecolor{currentstroke}%
\pgfsetdash{}{0pt}%
\pgfpathmoveto{\pgfqpoint{4.227140in}{2.242353in}}%
\pgfpathlineto{\pgfqpoint{4.478797in}{2.402809in}}%
\pgfpathlineto{\pgfqpoint{4.227140in}{2.242353in}}%
\pgfpathclose%
\pgfusepath{fill}%
\end{pgfscope}%
\begin{pgfscope}%
\pgfpathrectangle{\pgfqpoint{0.017500in}{0.017500in}}{\pgfqpoint{6.218282in}{3.580372in}} %
\pgfusepath{clip}%
\pgfsetbuttcap%
\pgfsetroundjoin%
\definecolor{currentfill}{rgb}{0.790562,0.231397,0.216242}%
\pgfsetfillcolor{currentfill}%
\pgfsetlinewidth{0.000000pt}%
\definecolor{currentstroke}{rgb}{0.000000,0.000000,0.000000}%
\pgfsetstrokecolor{currentstroke}%
\pgfsetdash{}{0pt}%
\pgfpathmoveto{\pgfqpoint{4.227140in}{2.242353in}}%
\pgfpathlineto{\pgfqpoint{4.478797in}{2.402809in}}%
\pgfpathlineto{\pgfqpoint{4.227140in}{2.242353in}}%
\pgfpathclose%
\pgfusepath{fill}%
\end{pgfscope}%
\begin{pgfscope}%
\pgfpathrectangle{\pgfqpoint{0.017500in}{0.017500in}}{\pgfqpoint{6.218282in}{3.580372in}} %
\pgfusepath{clip}%
\pgfsetbuttcap%
\pgfsetroundjoin%
\definecolor{currentfill}{rgb}{0.790562,0.231397,0.216242}%
\pgfsetfillcolor{currentfill}%
\pgfsetlinewidth{0.000000pt}%
\definecolor{currentstroke}{rgb}{0.000000,0.000000,0.000000}%
\pgfsetstrokecolor{currentstroke}%
\pgfsetdash{}{0pt}%
\pgfpathmoveto{\pgfqpoint{4.227140in}{2.242353in}}%
\pgfpathlineto{\pgfqpoint{4.478797in}{2.402809in}}%
\pgfpathlineto{\pgfqpoint{4.227140in}{2.242353in}}%
\pgfpathclose%
\pgfusepath{fill}%
\end{pgfscope}%
\begin{pgfscope}%
\pgfpathrectangle{\pgfqpoint{0.017500in}{0.017500in}}{\pgfqpoint{6.218282in}{3.580372in}} %
\pgfusepath{clip}%
\pgfsetbuttcap%
\pgfsetroundjoin%
\definecolor{currentfill}{rgb}{0.790562,0.231397,0.216242}%
\pgfsetfillcolor{currentfill}%
\pgfsetlinewidth{0.000000pt}%
\definecolor{currentstroke}{rgb}{0.000000,0.000000,0.000000}%
\pgfsetstrokecolor{currentstroke}%
\pgfsetdash{}{0pt}%
\pgfpathmoveto{\pgfqpoint{4.227140in}{2.242353in}}%
\pgfpathlineto{\pgfqpoint{4.478797in}{2.402809in}}%
\pgfpathlineto{\pgfqpoint{4.227140in}{2.242353in}}%
\pgfpathclose%
\pgfusepath{fill}%
\end{pgfscope}%
\begin{pgfscope}%
\pgfpathrectangle{\pgfqpoint{0.017500in}{0.017500in}}{\pgfqpoint{6.218282in}{3.580372in}} %
\pgfusepath{clip}%
\pgfsetbuttcap%
\pgfsetroundjoin%
\definecolor{currentfill}{rgb}{0.790562,0.231397,0.216242}%
\pgfsetfillcolor{currentfill}%
\pgfsetlinewidth{0.000000pt}%
\definecolor{currentstroke}{rgb}{0.000000,0.000000,0.000000}%
\pgfsetstrokecolor{currentstroke}%
\pgfsetdash{}{0pt}%
\pgfpathmoveto{\pgfqpoint{4.227140in}{2.242353in}}%
\pgfpathlineto{\pgfqpoint{4.478797in}{2.402809in}}%
\pgfpathlineto{\pgfqpoint{4.227140in}{2.242353in}}%
\pgfpathclose%
\pgfusepath{fill}%
\end{pgfscope}%
\begin{pgfscope}%
\pgfpathrectangle{\pgfqpoint{0.017500in}{0.017500in}}{\pgfqpoint{6.218282in}{3.580372in}} %
\pgfusepath{clip}%
\pgfsetbuttcap%
\pgfsetroundjoin%
\definecolor{currentfill}{rgb}{0.790562,0.231397,0.216242}%
\pgfsetfillcolor{currentfill}%
\pgfsetlinewidth{0.000000pt}%
\definecolor{currentstroke}{rgb}{0.000000,0.000000,0.000000}%
\pgfsetstrokecolor{currentstroke}%
\pgfsetdash{}{0pt}%
\pgfpathmoveto{\pgfqpoint{4.227140in}{2.242353in}}%
\pgfpathlineto{\pgfqpoint{4.478797in}{2.402809in}}%
\pgfpathlineto{\pgfqpoint{4.227140in}{2.242353in}}%
\pgfpathclose%
\pgfusepath{fill}%
\end{pgfscope}%
\begin{pgfscope}%
\pgfpathrectangle{\pgfqpoint{0.017500in}{0.017500in}}{\pgfqpoint{6.218282in}{3.580372in}} %
\pgfusepath{clip}%
\pgfsetbuttcap%
\pgfsetroundjoin%
\definecolor{currentfill}{rgb}{0.790562,0.231397,0.216242}%
\pgfsetfillcolor{currentfill}%
\pgfsetlinewidth{0.000000pt}%
\definecolor{currentstroke}{rgb}{0.000000,0.000000,0.000000}%
\pgfsetstrokecolor{currentstroke}%
\pgfsetdash{}{0pt}%
\pgfpathmoveto{\pgfqpoint{4.227140in}{2.242353in}}%
\pgfpathlineto{\pgfqpoint{4.478797in}{2.402809in}}%
\pgfpathlineto{\pgfqpoint{4.227140in}{2.242353in}}%
\pgfpathclose%
\pgfusepath{fill}%
\end{pgfscope}%
\begin{pgfscope}%
\pgfpathrectangle{\pgfqpoint{0.017500in}{0.017500in}}{\pgfqpoint{6.218282in}{3.580372in}} %
\pgfusepath{clip}%
\pgfsetbuttcap%
\pgfsetroundjoin%
\definecolor{currentfill}{rgb}{0.790562,0.231397,0.216242}%
\pgfsetfillcolor{currentfill}%
\pgfsetlinewidth{0.000000pt}%
\definecolor{currentstroke}{rgb}{0.000000,0.000000,0.000000}%
\pgfsetstrokecolor{currentstroke}%
\pgfsetdash{}{0pt}%
\pgfpathmoveto{\pgfqpoint{4.227140in}{2.242353in}}%
\pgfpathlineto{\pgfqpoint{4.478797in}{2.402809in}}%
\pgfpathlineto{\pgfqpoint{4.227140in}{2.242353in}}%
\pgfpathclose%
\pgfusepath{fill}%
\end{pgfscope}%
\begin{pgfscope}%
\pgfpathrectangle{\pgfqpoint{0.017500in}{0.017500in}}{\pgfqpoint{6.218282in}{3.580372in}} %
\pgfusepath{clip}%
\pgfsetbuttcap%
\pgfsetroundjoin%
\definecolor{currentfill}{rgb}{0.790562,0.231397,0.216242}%
\pgfsetfillcolor{currentfill}%
\pgfsetlinewidth{0.000000pt}%
\definecolor{currentstroke}{rgb}{0.000000,0.000000,0.000000}%
\pgfsetstrokecolor{currentstroke}%
\pgfsetdash{}{0pt}%
\pgfpathmoveto{\pgfqpoint{4.227140in}{2.242353in}}%
\pgfpathlineto{\pgfqpoint{4.478797in}{2.402809in}}%
\pgfpathlineto{\pgfqpoint{4.227140in}{2.242353in}}%
\pgfpathclose%
\pgfusepath{fill}%
\end{pgfscope}%
\begin{pgfscope}%
\pgfpathrectangle{\pgfqpoint{0.017500in}{0.017500in}}{\pgfqpoint{6.218282in}{3.580372in}} %
\pgfusepath{clip}%
\pgfsetbuttcap%
\pgfsetroundjoin%
\definecolor{currentfill}{rgb}{0.790562,0.231397,0.216242}%
\pgfsetfillcolor{currentfill}%
\pgfsetlinewidth{0.000000pt}%
\definecolor{currentstroke}{rgb}{0.000000,0.000000,0.000000}%
\pgfsetstrokecolor{currentstroke}%
\pgfsetdash{}{0pt}%
\pgfpathmoveto{\pgfqpoint{4.227140in}{2.242353in}}%
\pgfpathlineto{\pgfqpoint{4.478797in}{2.402809in}}%
\pgfpathlineto{\pgfqpoint{4.227140in}{2.242353in}}%
\pgfpathclose%
\pgfusepath{fill}%
\end{pgfscope}%
\begin{pgfscope}%
\pgfpathrectangle{\pgfqpoint{0.017500in}{0.017500in}}{\pgfqpoint{6.218282in}{3.580372in}} %
\pgfusepath{clip}%
\pgfsetbuttcap%
\pgfsetroundjoin%
\definecolor{currentfill}{rgb}{0.790562,0.231397,0.216242}%
\pgfsetfillcolor{currentfill}%
\pgfsetlinewidth{0.000000pt}%
\definecolor{currentstroke}{rgb}{0.000000,0.000000,0.000000}%
\pgfsetstrokecolor{currentstroke}%
\pgfsetdash{}{0pt}%
\pgfpathmoveto{\pgfqpoint{4.227140in}{2.242353in}}%
\pgfpathlineto{\pgfqpoint{4.478797in}{2.402809in}}%
\pgfpathlineto{\pgfqpoint{4.227140in}{2.242353in}}%
\pgfpathclose%
\pgfusepath{fill}%
\end{pgfscope}%
\begin{pgfscope}%
\pgfpathrectangle{\pgfqpoint{0.017500in}{0.017500in}}{\pgfqpoint{6.218282in}{3.580372in}} %
\pgfusepath{clip}%
\pgfsetbuttcap%
\pgfsetroundjoin%
\definecolor{currentfill}{rgb}{0.790562,0.231397,0.216242}%
\pgfsetfillcolor{currentfill}%
\pgfsetlinewidth{0.000000pt}%
\definecolor{currentstroke}{rgb}{0.000000,0.000000,0.000000}%
\pgfsetstrokecolor{currentstroke}%
\pgfsetdash{}{0pt}%
\pgfpathmoveto{\pgfqpoint{4.227140in}{2.242353in}}%
\pgfpathlineto{\pgfqpoint{4.478797in}{2.402809in}}%
\pgfpathlineto{\pgfqpoint{4.227140in}{2.242353in}}%
\pgfpathclose%
\pgfusepath{fill}%
\end{pgfscope}%
\begin{pgfscope}%
\pgfpathrectangle{\pgfqpoint{0.017500in}{0.017500in}}{\pgfqpoint{6.218282in}{3.580372in}} %
\pgfusepath{clip}%
\pgfsetbuttcap%
\pgfsetroundjoin%
\definecolor{currentfill}{rgb}{0.790562,0.231397,0.216242}%
\pgfsetfillcolor{currentfill}%
\pgfsetlinewidth{0.000000pt}%
\definecolor{currentstroke}{rgb}{0.000000,0.000000,0.000000}%
\pgfsetstrokecolor{currentstroke}%
\pgfsetdash{}{0pt}%
\pgfpathmoveto{\pgfqpoint{4.227140in}{2.242353in}}%
\pgfpathlineto{\pgfqpoint{4.478797in}{2.402809in}}%
\pgfpathlineto{\pgfqpoint{4.227140in}{2.242353in}}%
\pgfpathclose%
\pgfusepath{fill}%
\end{pgfscope}%
\begin{pgfscope}%
\pgfpathrectangle{\pgfqpoint{0.017500in}{0.017500in}}{\pgfqpoint{6.218282in}{3.580372in}} %
\pgfusepath{clip}%
\pgfsetbuttcap%
\pgfsetroundjoin%
\definecolor{currentfill}{rgb}{0.790562,0.231397,0.216242}%
\pgfsetfillcolor{currentfill}%
\pgfsetlinewidth{0.000000pt}%
\definecolor{currentstroke}{rgb}{0.000000,0.000000,0.000000}%
\pgfsetstrokecolor{currentstroke}%
\pgfsetdash{}{0pt}%
\pgfpathmoveto{\pgfqpoint{4.227140in}{2.242353in}}%
\pgfpathlineto{\pgfqpoint{4.478797in}{2.402809in}}%
\pgfpathlineto{\pgfqpoint{4.227140in}{2.242353in}}%
\pgfpathclose%
\pgfusepath{fill}%
\end{pgfscope}%
\begin{pgfscope}%
\pgfpathrectangle{\pgfqpoint{0.017500in}{0.017500in}}{\pgfqpoint{6.218282in}{3.580372in}} %
\pgfusepath{clip}%
\pgfsetbuttcap%
\pgfsetroundjoin%
\definecolor{currentfill}{rgb}{0.790562,0.231397,0.216242}%
\pgfsetfillcolor{currentfill}%
\pgfsetlinewidth{0.000000pt}%
\definecolor{currentstroke}{rgb}{0.000000,0.000000,0.000000}%
\pgfsetstrokecolor{currentstroke}%
\pgfsetdash{}{0pt}%
\pgfpathmoveto{\pgfqpoint{4.227140in}{2.242353in}}%
\pgfpathlineto{\pgfqpoint{4.478797in}{2.402809in}}%
\pgfpathlineto{\pgfqpoint{4.227140in}{2.242353in}}%
\pgfpathclose%
\pgfusepath{fill}%
\end{pgfscope}%
\begin{pgfscope}%
\pgfpathrectangle{\pgfqpoint{0.017500in}{0.017500in}}{\pgfqpoint{6.218282in}{3.580372in}} %
\pgfusepath{clip}%
\pgfsetbuttcap%
\pgfsetroundjoin%
\definecolor{currentfill}{rgb}{0.790562,0.231397,0.216242}%
\pgfsetfillcolor{currentfill}%
\pgfsetlinewidth{0.000000pt}%
\definecolor{currentstroke}{rgb}{0.000000,0.000000,0.000000}%
\pgfsetstrokecolor{currentstroke}%
\pgfsetdash{}{0pt}%
\pgfpathmoveto{\pgfqpoint{4.227140in}{2.242353in}}%
\pgfpathlineto{\pgfqpoint{4.478797in}{2.402809in}}%
\pgfpathlineto{\pgfqpoint{4.227140in}{2.242353in}}%
\pgfpathclose%
\pgfusepath{fill}%
\end{pgfscope}%
\begin{pgfscope}%
\pgfpathrectangle{\pgfqpoint{0.017500in}{0.017500in}}{\pgfqpoint{6.218282in}{3.580372in}} %
\pgfusepath{clip}%
\pgfsetbuttcap%
\pgfsetroundjoin%
\definecolor{currentfill}{rgb}{0.790562,0.231397,0.216242}%
\pgfsetfillcolor{currentfill}%
\pgfsetlinewidth{0.000000pt}%
\definecolor{currentstroke}{rgb}{0.000000,0.000000,0.000000}%
\pgfsetstrokecolor{currentstroke}%
\pgfsetdash{}{0pt}%
\pgfpathmoveto{\pgfqpoint{4.227140in}{2.242353in}}%
\pgfpathlineto{\pgfqpoint{4.478797in}{2.402809in}}%
\pgfpathlineto{\pgfqpoint{4.227140in}{2.242353in}}%
\pgfpathclose%
\pgfusepath{fill}%
\end{pgfscope}%
\begin{pgfscope}%
\pgfpathrectangle{\pgfqpoint{0.017500in}{0.017500in}}{\pgfqpoint{6.218282in}{3.580372in}} %
\pgfusepath{clip}%
\pgfsetbuttcap%
\pgfsetroundjoin%
\definecolor{currentfill}{rgb}{0.790562,0.231397,0.216242}%
\pgfsetfillcolor{currentfill}%
\pgfsetlinewidth{0.000000pt}%
\definecolor{currentstroke}{rgb}{0.000000,0.000000,0.000000}%
\pgfsetstrokecolor{currentstroke}%
\pgfsetdash{}{0pt}%
\pgfpathmoveto{\pgfqpoint{4.227140in}{2.242353in}}%
\pgfpathlineto{\pgfqpoint{4.478797in}{2.402809in}}%
\pgfpathlineto{\pgfqpoint{4.227140in}{2.242353in}}%
\pgfpathclose%
\pgfusepath{fill}%
\end{pgfscope}%
\begin{pgfscope}%
\pgfpathrectangle{\pgfqpoint{0.017500in}{0.017500in}}{\pgfqpoint{6.218282in}{3.580372in}} %
\pgfusepath{clip}%
\pgfsetbuttcap%
\pgfsetroundjoin%
\definecolor{currentfill}{rgb}{0.790562,0.231397,0.216242}%
\pgfsetfillcolor{currentfill}%
\pgfsetlinewidth{0.000000pt}%
\definecolor{currentstroke}{rgb}{0.000000,0.000000,0.000000}%
\pgfsetstrokecolor{currentstroke}%
\pgfsetdash{}{0pt}%
\pgfpathmoveto{\pgfqpoint{4.227140in}{2.242353in}}%
\pgfpathlineto{\pgfqpoint{4.478797in}{2.402809in}}%
\pgfpathlineto{\pgfqpoint{4.227140in}{2.242353in}}%
\pgfpathclose%
\pgfusepath{fill}%
\end{pgfscope}%
\begin{pgfscope}%
\pgfpathrectangle{\pgfqpoint{0.017500in}{0.017500in}}{\pgfqpoint{6.218282in}{3.580372in}} %
\pgfusepath{clip}%
\pgfsetbuttcap%
\pgfsetroundjoin%
\definecolor{currentfill}{rgb}{0.877149,0.394645,0.311724}%
\pgfsetfillcolor{currentfill}%
\pgfsetlinewidth{0.000000pt}%
\definecolor{currentstroke}{rgb}{0.000000,0.000000,0.000000}%
\pgfsetstrokecolor{currentstroke}%
\pgfsetdash{}{0pt}%
\pgfpathmoveto{\pgfqpoint{3.502717in}{2.091450in}}%
\pgfpathlineto{\pgfqpoint{3.763681in}{2.314857in}}%
\pgfpathlineto{\pgfqpoint{3.502717in}{2.091450in}}%
\pgfpathclose%
\pgfusepath{fill}%
\end{pgfscope}%
\begin{pgfscope}%
\pgfpathrectangle{\pgfqpoint{0.017500in}{0.017500in}}{\pgfqpoint{6.218282in}{3.580372in}} %
\pgfusepath{clip}%
\pgfsetbuttcap%
\pgfsetroundjoin%
\definecolor{currentfill}{rgb}{0.877149,0.394645,0.311724}%
\pgfsetfillcolor{currentfill}%
\pgfsetlinewidth{0.000000pt}%
\definecolor{currentstroke}{rgb}{0.000000,0.000000,0.000000}%
\pgfsetstrokecolor{currentstroke}%
\pgfsetdash{}{0pt}%
\pgfpathmoveto{\pgfqpoint{3.502717in}{2.091450in}}%
\pgfpathlineto{\pgfqpoint{3.763681in}{2.314857in}}%
\pgfpathlineto{\pgfqpoint{3.502717in}{2.091450in}}%
\pgfpathclose%
\pgfusepath{fill}%
\end{pgfscope}%
\begin{pgfscope}%
\pgfpathrectangle{\pgfqpoint{0.017500in}{0.017500in}}{\pgfqpoint{6.218282in}{3.580372in}} %
\pgfusepath{clip}%
\pgfsetbuttcap%
\pgfsetroundjoin%
\definecolor{currentfill}{rgb}{0.877149,0.394645,0.311724}%
\pgfsetfillcolor{currentfill}%
\pgfsetlinewidth{0.000000pt}%
\definecolor{currentstroke}{rgb}{0.000000,0.000000,0.000000}%
\pgfsetstrokecolor{currentstroke}%
\pgfsetdash{}{0pt}%
\pgfpathmoveto{\pgfqpoint{3.502717in}{2.091450in}}%
\pgfpathlineto{\pgfqpoint{3.763681in}{2.314857in}}%
\pgfpathlineto{\pgfqpoint{3.502717in}{2.091450in}}%
\pgfpathclose%
\pgfusepath{fill}%
\end{pgfscope}%
\begin{pgfscope}%
\pgfpathrectangle{\pgfqpoint{0.017500in}{0.017500in}}{\pgfqpoint{6.218282in}{3.580372in}} %
\pgfusepath{clip}%
\pgfsetbuttcap%
\pgfsetroundjoin%
\definecolor{currentfill}{rgb}{0.877149,0.394645,0.311724}%
\pgfsetfillcolor{currentfill}%
\pgfsetlinewidth{0.000000pt}%
\definecolor{currentstroke}{rgb}{0.000000,0.000000,0.000000}%
\pgfsetstrokecolor{currentstroke}%
\pgfsetdash{}{0pt}%
\pgfpathmoveto{\pgfqpoint{3.502717in}{2.091450in}}%
\pgfpathlineto{\pgfqpoint{3.763681in}{2.314857in}}%
\pgfpathlineto{\pgfqpoint{3.502717in}{2.091450in}}%
\pgfpathclose%
\pgfusepath{fill}%
\end{pgfscope}%
\begin{pgfscope}%
\pgfpathrectangle{\pgfqpoint{0.017500in}{0.017500in}}{\pgfqpoint{6.218282in}{3.580372in}} %
\pgfusepath{clip}%
\pgfsetbuttcap%
\pgfsetroundjoin%
\definecolor{currentfill}{rgb}{0.877149,0.394645,0.311724}%
\pgfsetfillcolor{currentfill}%
\pgfsetlinewidth{0.000000pt}%
\definecolor{currentstroke}{rgb}{0.000000,0.000000,0.000000}%
\pgfsetstrokecolor{currentstroke}%
\pgfsetdash{}{0pt}%
\pgfpathmoveto{\pgfqpoint{3.502717in}{2.091450in}}%
\pgfpathlineto{\pgfqpoint{3.763681in}{2.314857in}}%
\pgfpathlineto{\pgfqpoint{3.502717in}{2.091450in}}%
\pgfpathclose%
\pgfusepath{fill}%
\end{pgfscope}%
\begin{pgfscope}%
\pgfpathrectangle{\pgfqpoint{0.017500in}{0.017500in}}{\pgfqpoint{6.218282in}{3.580372in}} %
\pgfusepath{clip}%
\pgfsetbuttcap%
\pgfsetroundjoin%
\definecolor{currentfill}{rgb}{0.877149,0.394645,0.311724}%
\pgfsetfillcolor{currentfill}%
\pgfsetlinewidth{0.000000pt}%
\definecolor{currentstroke}{rgb}{0.000000,0.000000,0.000000}%
\pgfsetstrokecolor{currentstroke}%
\pgfsetdash{}{0pt}%
\pgfpathmoveto{\pgfqpoint{3.502717in}{2.091450in}}%
\pgfpathlineto{\pgfqpoint{3.763681in}{2.314857in}}%
\pgfpathlineto{\pgfqpoint{3.502717in}{2.091450in}}%
\pgfpathclose%
\pgfusepath{fill}%
\end{pgfscope}%
\begin{pgfscope}%
\pgfpathrectangle{\pgfqpoint{0.017500in}{0.017500in}}{\pgfqpoint{6.218282in}{3.580372in}} %
\pgfusepath{clip}%
\pgfsetbuttcap%
\pgfsetroundjoin%
\definecolor{currentfill}{rgb}{0.877149,0.394645,0.311724}%
\pgfsetfillcolor{currentfill}%
\pgfsetlinewidth{0.000000pt}%
\definecolor{currentstroke}{rgb}{0.000000,0.000000,0.000000}%
\pgfsetstrokecolor{currentstroke}%
\pgfsetdash{}{0pt}%
\pgfpathmoveto{\pgfqpoint{3.502717in}{2.091450in}}%
\pgfpathlineto{\pgfqpoint{3.763681in}{2.314857in}}%
\pgfpathlineto{\pgfqpoint{3.502717in}{2.091450in}}%
\pgfpathclose%
\pgfusepath{fill}%
\end{pgfscope}%
\begin{pgfscope}%
\pgfpathrectangle{\pgfqpoint{0.017500in}{0.017500in}}{\pgfqpoint{6.218282in}{3.580372in}} %
\pgfusepath{clip}%
\pgfsetbuttcap%
\pgfsetroundjoin%
\definecolor{currentfill}{rgb}{0.877149,0.394645,0.311724}%
\pgfsetfillcolor{currentfill}%
\pgfsetlinewidth{0.000000pt}%
\definecolor{currentstroke}{rgb}{0.000000,0.000000,0.000000}%
\pgfsetstrokecolor{currentstroke}%
\pgfsetdash{}{0pt}%
\pgfpathmoveto{\pgfqpoint{3.502717in}{2.091450in}}%
\pgfpathlineto{\pgfqpoint{3.763681in}{2.314857in}}%
\pgfpathlineto{\pgfqpoint{3.502717in}{2.091450in}}%
\pgfpathclose%
\pgfusepath{fill}%
\end{pgfscope}%
\begin{pgfscope}%
\pgfpathrectangle{\pgfqpoint{0.017500in}{0.017500in}}{\pgfqpoint{6.218282in}{3.580372in}} %
\pgfusepath{clip}%
\pgfsetbuttcap%
\pgfsetroundjoin%
\definecolor{currentfill}{rgb}{0.877149,0.394645,0.311724}%
\pgfsetfillcolor{currentfill}%
\pgfsetlinewidth{0.000000pt}%
\definecolor{currentstroke}{rgb}{0.000000,0.000000,0.000000}%
\pgfsetstrokecolor{currentstroke}%
\pgfsetdash{}{0pt}%
\pgfpathmoveto{\pgfqpoint{3.502717in}{2.091450in}}%
\pgfpathlineto{\pgfqpoint{3.763681in}{2.314857in}}%
\pgfpathlineto{\pgfqpoint{3.502717in}{2.091450in}}%
\pgfpathclose%
\pgfusepath{fill}%
\end{pgfscope}%
\begin{pgfscope}%
\pgfpathrectangle{\pgfqpoint{0.017500in}{0.017500in}}{\pgfqpoint{6.218282in}{3.580372in}} %
\pgfusepath{clip}%
\pgfsetbuttcap%
\pgfsetroundjoin%
\definecolor{currentfill}{rgb}{0.877149,0.394645,0.311724}%
\pgfsetfillcolor{currentfill}%
\pgfsetlinewidth{0.000000pt}%
\definecolor{currentstroke}{rgb}{0.000000,0.000000,0.000000}%
\pgfsetstrokecolor{currentstroke}%
\pgfsetdash{}{0pt}%
\pgfpathmoveto{\pgfqpoint{3.502717in}{2.091450in}}%
\pgfpathlineto{\pgfqpoint{3.763681in}{2.314857in}}%
\pgfpathlineto{\pgfqpoint{3.502717in}{2.091450in}}%
\pgfpathclose%
\pgfusepath{fill}%
\end{pgfscope}%
\begin{pgfscope}%
\pgfpathrectangle{\pgfqpoint{0.017500in}{0.017500in}}{\pgfqpoint{6.218282in}{3.580372in}} %
\pgfusepath{clip}%
\pgfsetbuttcap%
\pgfsetroundjoin%
\definecolor{currentfill}{rgb}{0.877149,0.394645,0.311724}%
\pgfsetfillcolor{currentfill}%
\pgfsetlinewidth{0.000000pt}%
\definecolor{currentstroke}{rgb}{0.000000,0.000000,0.000000}%
\pgfsetstrokecolor{currentstroke}%
\pgfsetdash{}{0pt}%
\pgfpathmoveto{\pgfqpoint{3.502717in}{2.091450in}}%
\pgfpathlineto{\pgfqpoint{3.763681in}{2.314857in}}%
\pgfpathlineto{\pgfqpoint{3.502717in}{2.091450in}}%
\pgfpathclose%
\pgfusepath{fill}%
\end{pgfscope}%
\begin{pgfscope}%
\pgfpathrectangle{\pgfqpoint{0.017500in}{0.017500in}}{\pgfqpoint{6.218282in}{3.580372in}} %
\pgfusepath{clip}%
\pgfsetbuttcap%
\pgfsetroundjoin%
\definecolor{currentfill}{rgb}{0.877149,0.394645,0.311724}%
\pgfsetfillcolor{currentfill}%
\pgfsetlinewidth{0.000000pt}%
\definecolor{currentstroke}{rgb}{0.000000,0.000000,0.000000}%
\pgfsetstrokecolor{currentstroke}%
\pgfsetdash{}{0pt}%
\pgfpathmoveto{\pgfqpoint{3.502717in}{2.091450in}}%
\pgfpathlineto{\pgfqpoint{3.763681in}{2.314857in}}%
\pgfpathlineto{\pgfqpoint{3.502717in}{2.091450in}}%
\pgfpathclose%
\pgfusepath{fill}%
\end{pgfscope}%
\begin{pgfscope}%
\pgfpathrectangle{\pgfqpoint{0.017500in}{0.017500in}}{\pgfqpoint{6.218282in}{3.580372in}} %
\pgfusepath{clip}%
\pgfsetbuttcap%
\pgfsetroundjoin%
\definecolor{currentfill}{rgb}{0.877149,0.394645,0.311724}%
\pgfsetfillcolor{currentfill}%
\pgfsetlinewidth{0.000000pt}%
\definecolor{currentstroke}{rgb}{0.000000,0.000000,0.000000}%
\pgfsetstrokecolor{currentstroke}%
\pgfsetdash{}{0pt}%
\pgfpathmoveto{\pgfqpoint{3.502717in}{2.091450in}}%
\pgfpathlineto{\pgfqpoint{3.763681in}{2.314857in}}%
\pgfpathlineto{\pgfqpoint{3.502717in}{2.091450in}}%
\pgfpathclose%
\pgfusepath{fill}%
\end{pgfscope}%
\begin{pgfscope}%
\pgfpathrectangle{\pgfqpoint{0.017500in}{0.017500in}}{\pgfqpoint{6.218282in}{3.580372in}} %
\pgfusepath{clip}%
\pgfsetbuttcap%
\pgfsetroundjoin%
\definecolor{currentfill}{rgb}{0.877149,0.394645,0.311724}%
\pgfsetfillcolor{currentfill}%
\pgfsetlinewidth{0.000000pt}%
\definecolor{currentstroke}{rgb}{0.000000,0.000000,0.000000}%
\pgfsetstrokecolor{currentstroke}%
\pgfsetdash{}{0pt}%
\pgfpathmoveto{\pgfqpoint{3.502717in}{2.091450in}}%
\pgfpathlineto{\pgfqpoint{3.763681in}{2.314857in}}%
\pgfpathlineto{\pgfqpoint{3.502717in}{2.091450in}}%
\pgfpathclose%
\pgfusepath{fill}%
\end{pgfscope}%
\begin{pgfscope}%
\pgfpathrectangle{\pgfqpoint{0.017500in}{0.017500in}}{\pgfqpoint{6.218282in}{3.580372in}} %
\pgfusepath{clip}%
\pgfsetbuttcap%
\pgfsetroundjoin%
\definecolor{currentfill}{rgb}{0.877149,0.394645,0.311724}%
\pgfsetfillcolor{currentfill}%
\pgfsetlinewidth{0.000000pt}%
\definecolor{currentstroke}{rgb}{0.000000,0.000000,0.000000}%
\pgfsetstrokecolor{currentstroke}%
\pgfsetdash{}{0pt}%
\pgfpathmoveto{\pgfqpoint{3.502717in}{2.091450in}}%
\pgfpathlineto{\pgfqpoint{3.763681in}{2.314857in}}%
\pgfpathlineto{\pgfqpoint{3.502717in}{2.091450in}}%
\pgfpathclose%
\pgfusepath{fill}%
\end{pgfscope}%
\begin{pgfscope}%
\pgfpathrectangle{\pgfqpoint{0.017500in}{0.017500in}}{\pgfqpoint{6.218282in}{3.580372in}} %
\pgfusepath{clip}%
\pgfsetbuttcap%
\pgfsetroundjoin%
\definecolor{currentfill}{rgb}{0.877149,0.394645,0.311724}%
\pgfsetfillcolor{currentfill}%
\pgfsetlinewidth{0.000000pt}%
\definecolor{currentstroke}{rgb}{0.000000,0.000000,0.000000}%
\pgfsetstrokecolor{currentstroke}%
\pgfsetdash{}{0pt}%
\pgfpathmoveto{\pgfqpoint{3.502717in}{2.091450in}}%
\pgfpathlineto{\pgfqpoint{3.763681in}{2.314857in}}%
\pgfpathlineto{\pgfqpoint{3.502717in}{2.091450in}}%
\pgfpathclose%
\pgfusepath{fill}%
\end{pgfscope}%
\begin{pgfscope}%
\pgfpathrectangle{\pgfqpoint{0.017500in}{0.017500in}}{\pgfqpoint{6.218282in}{3.580372in}} %
\pgfusepath{clip}%
\pgfsetbuttcap%
\pgfsetroundjoin%
\definecolor{currentfill}{rgb}{0.877149,0.394645,0.311724}%
\pgfsetfillcolor{currentfill}%
\pgfsetlinewidth{0.000000pt}%
\definecolor{currentstroke}{rgb}{0.000000,0.000000,0.000000}%
\pgfsetstrokecolor{currentstroke}%
\pgfsetdash{}{0pt}%
\pgfpathmoveto{\pgfqpoint{3.502717in}{2.091450in}}%
\pgfpathlineto{\pgfqpoint{3.763681in}{2.314857in}}%
\pgfpathlineto{\pgfqpoint{3.502717in}{2.091450in}}%
\pgfpathclose%
\pgfusepath{fill}%
\end{pgfscope}%
\begin{pgfscope}%
\pgfpathrectangle{\pgfqpoint{0.017500in}{0.017500in}}{\pgfqpoint{6.218282in}{3.580372in}} %
\pgfusepath{clip}%
\pgfsetbuttcap%
\pgfsetroundjoin%
\definecolor{currentfill}{rgb}{0.877149,0.394645,0.311724}%
\pgfsetfillcolor{currentfill}%
\pgfsetlinewidth{0.000000pt}%
\definecolor{currentstroke}{rgb}{0.000000,0.000000,0.000000}%
\pgfsetstrokecolor{currentstroke}%
\pgfsetdash{}{0pt}%
\pgfpathmoveto{\pgfqpoint{3.502717in}{2.091450in}}%
\pgfpathlineto{\pgfqpoint{3.763681in}{2.314857in}}%
\pgfpathlineto{\pgfqpoint{3.502717in}{2.091450in}}%
\pgfpathclose%
\pgfusepath{fill}%
\end{pgfscope}%
\begin{pgfscope}%
\pgfpathrectangle{\pgfqpoint{0.017500in}{0.017500in}}{\pgfqpoint{6.218282in}{3.580372in}} %
\pgfusepath{clip}%
\pgfsetbuttcap%
\pgfsetroundjoin%
\definecolor{currentfill}{rgb}{0.877149,0.394645,0.311724}%
\pgfsetfillcolor{currentfill}%
\pgfsetlinewidth{0.000000pt}%
\definecolor{currentstroke}{rgb}{0.000000,0.000000,0.000000}%
\pgfsetstrokecolor{currentstroke}%
\pgfsetdash{}{0pt}%
\pgfpathmoveto{\pgfqpoint{3.502717in}{2.091450in}}%
\pgfpathlineto{\pgfqpoint{3.763681in}{2.314857in}}%
\pgfpathlineto{\pgfqpoint{3.502717in}{2.091450in}}%
\pgfpathclose%
\pgfusepath{fill}%
\end{pgfscope}%
\begin{pgfscope}%
\pgfpathrectangle{\pgfqpoint{0.017500in}{0.017500in}}{\pgfqpoint{6.218282in}{3.580372in}} %
\pgfusepath{clip}%
\pgfsetbuttcap%
\pgfsetroundjoin%
\definecolor{currentfill}{rgb}{0.877149,0.394645,0.311724}%
\pgfsetfillcolor{currentfill}%
\pgfsetlinewidth{0.000000pt}%
\definecolor{currentstroke}{rgb}{0.000000,0.000000,0.000000}%
\pgfsetstrokecolor{currentstroke}%
\pgfsetdash{}{0pt}%
\pgfpathmoveto{\pgfqpoint{3.502717in}{2.091450in}}%
\pgfpathlineto{\pgfqpoint{3.763681in}{2.314857in}}%
\pgfpathlineto{\pgfqpoint{3.502717in}{2.091450in}}%
\pgfpathclose%
\pgfusepath{fill}%
\end{pgfscope}%
\begin{pgfscope}%
\pgfpathrectangle{\pgfqpoint{0.017500in}{0.017500in}}{\pgfqpoint{6.218282in}{3.580372in}} %
\pgfusepath{clip}%
\pgfsetbuttcap%
\pgfsetroundjoin%
\definecolor{currentfill}{rgb}{0.877149,0.394645,0.311724}%
\pgfsetfillcolor{currentfill}%
\pgfsetlinewidth{0.000000pt}%
\definecolor{currentstroke}{rgb}{0.000000,0.000000,0.000000}%
\pgfsetstrokecolor{currentstroke}%
\pgfsetdash{}{0pt}%
\pgfpathmoveto{\pgfqpoint{3.502717in}{2.091450in}}%
\pgfpathlineto{\pgfqpoint{3.763681in}{2.314857in}}%
\pgfpathlineto{\pgfqpoint{3.502717in}{2.091450in}}%
\pgfpathclose%
\pgfusepath{fill}%
\end{pgfscope}%
\begin{pgfscope}%
\pgfpathrectangle{\pgfqpoint{0.017500in}{0.017500in}}{\pgfqpoint{6.218282in}{3.580372in}} %
\pgfusepath{clip}%
\pgfsetbuttcap%
\pgfsetroundjoin%
\definecolor{currentfill}{rgb}{0.877149,0.394645,0.311724}%
\pgfsetfillcolor{currentfill}%
\pgfsetlinewidth{0.000000pt}%
\definecolor{currentstroke}{rgb}{0.000000,0.000000,0.000000}%
\pgfsetstrokecolor{currentstroke}%
\pgfsetdash{}{0pt}%
\pgfpathmoveto{\pgfqpoint{3.502717in}{2.091450in}}%
\pgfpathlineto{\pgfqpoint{3.763681in}{2.314857in}}%
\pgfpathlineto{\pgfqpoint{3.502717in}{2.091450in}}%
\pgfpathclose%
\pgfusepath{fill}%
\end{pgfscope}%
\begin{pgfscope}%
\pgfpathrectangle{\pgfqpoint{0.017500in}{0.017500in}}{\pgfqpoint{6.218282in}{3.580372in}} %
\pgfusepath{clip}%
\pgfsetbuttcap%
\pgfsetroundjoin%
\definecolor{currentfill}{rgb}{0.877149,0.394645,0.311724}%
\pgfsetfillcolor{currentfill}%
\pgfsetlinewidth{0.000000pt}%
\definecolor{currentstroke}{rgb}{0.000000,0.000000,0.000000}%
\pgfsetstrokecolor{currentstroke}%
\pgfsetdash{}{0pt}%
\pgfpathmoveto{\pgfqpoint{3.502717in}{2.091450in}}%
\pgfpathlineto{\pgfqpoint{3.763681in}{2.314857in}}%
\pgfpathlineto{\pgfqpoint{3.502717in}{2.091450in}}%
\pgfpathclose%
\pgfusepath{fill}%
\end{pgfscope}%
\begin{pgfscope}%
\pgfpathrectangle{\pgfqpoint{0.017500in}{0.017500in}}{\pgfqpoint{6.218282in}{3.580372in}} %
\pgfusepath{clip}%
\pgfsetbuttcap%
\pgfsetroundjoin%
\definecolor{currentfill}{rgb}{0.877149,0.394645,0.311724}%
\pgfsetfillcolor{currentfill}%
\pgfsetlinewidth{0.000000pt}%
\definecolor{currentstroke}{rgb}{0.000000,0.000000,0.000000}%
\pgfsetstrokecolor{currentstroke}%
\pgfsetdash{}{0pt}%
\pgfpathmoveto{\pgfqpoint{3.502717in}{2.091450in}}%
\pgfpathlineto{\pgfqpoint{3.763681in}{2.314857in}}%
\pgfpathlineto{\pgfqpoint{3.502717in}{2.091450in}}%
\pgfpathclose%
\pgfusepath{fill}%
\end{pgfscope}%
\begin{pgfscope}%
\pgfpathrectangle{\pgfqpoint{0.017500in}{0.017500in}}{\pgfqpoint{6.218282in}{3.580372in}} %
\pgfusepath{clip}%
\pgfsetbuttcap%
\pgfsetroundjoin%
\definecolor{currentfill}{rgb}{0.877149,0.394645,0.311724}%
\pgfsetfillcolor{currentfill}%
\pgfsetlinewidth{0.000000pt}%
\definecolor{currentstroke}{rgb}{0.000000,0.000000,0.000000}%
\pgfsetstrokecolor{currentstroke}%
\pgfsetdash{}{0pt}%
\pgfpathmoveto{\pgfqpoint{3.502717in}{2.091450in}}%
\pgfpathlineto{\pgfqpoint{3.763681in}{2.314857in}}%
\pgfpathlineto{\pgfqpoint{3.502717in}{2.091450in}}%
\pgfpathclose%
\pgfusepath{fill}%
\end{pgfscope}%
\begin{pgfscope}%
\pgfpathrectangle{\pgfqpoint{0.017500in}{0.017500in}}{\pgfqpoint{6.218282in}{3.580372in}} %
\pgfusepath{clip}%
\pgfsetbuttcap%
\pgfsetroundjoin%
\definecolor{currentfill}{rgb}{0.877149,0.394645,0.311724}%
\pgfsetfillcolor{currentfill}%
\pgfsetlinewidth{0.000000pt}%
\definecolor{currentstroke}{rgb}{0.000000,0.000000,0.000000}%
\pgfsetstrokecolor{currentstroke}%
\pgfsetdash{}{0pt}%
\pgfpathmoveto{\pgfqpoint{3.502717in}{2.091450in}}%
\pgfpathlineto{\pgfqpoint{3.763681in}{2.314857in}}%
\pgfpathlineto{\pgfqpoint{3.502717in}{2.091450in}}%
\pgfpathclose%
\pgfusepath{fill}%
\end{pgfscope}%
\begin{pgfscope}%
\pgfpathrectangle{\pgfqpoint{0.017500in}{0.017500in}}{\pgfqpoint{6.218282in}{3.580372in}} %
\pgfusepath{clip}%
\pgfsetbuttcap%
\pgfsetroundjoin%
\definecolor{currentfill}{rgb}{0.877149,0.394645,0.311724}%
\pgfsetfillcolor{currentfill}%
\pgfsetlinewidth{0.000000pt}%
\definecolor{currentstroke}{rgb}{0.000000,0.000000,0.000000}%
\pgfsetstrokecolor{currentstroke}%
\pgfsetdash{}{0pt}%
\pgfpathmoveto{\pgfqpoint{3.502717in}{2.091450in}}%
\pgfpathlineto{\pgfqpoint{3.763681in}{2.314857in}}%
\pgfpathlineto{\pgfqpoint{3.502717in}{2.091450in}}%
\pgfpathclose%
\pgfusepath{fill}%
\end{pgfscope}%
\begin{pgfscope}%
\pgfpathrectangle{\pgfqpoint{0.017500in}{0.017500in}}{\pgfqpoint{6.218282in}{3.580372in}} %
\pgfusepath{clip}%
\pgfsetbuttcap%
\pgfsetroundjoin%
\definecolor{currentfill}{rgb}{0.877149,0.394645,0.311724}%
\pgfsetfillcolor{currentfill}%
\pgfsetlinewidth{0.000000pt}%
\definecolor{currentstroke}{rgb}{0.000000,0.000000,0.000000}%
\pgfsetstrokecolor{currentstroke}%
\pgfsetdash{}{0pt}%
\pgfpathmoveto{\pgfqpoint{3.502717in}{2.091450in}}%
\pgfpathlineto{\pgfqpoint{3.763681in}{2.314857in}}%
\pgfpathlineto{\pgfqpoint{3.502717in}{2.091450in}}%
\pgfpathclose%
\pgfusepath{fill}%
\end{pgfscope}%
\begin{pgfscope}%
\pgfpathrectangle{\pgfqpoint{0.017500in}{0.017500in}}{\pgfqpoint{6.218282in}{3.580372in}} %
\pgfusepath{clip}%
\pgfsetbuttcap%
\pgfsetroundjoin%
\definecolor{currentfill}{rgb}{0.877149,0.394645,0.311724}%
\pgfsetfillcolor{currentfill}%
\pgfsetlinewidth{0.000000pt}%
\definecolor{currentstroke}{rgb}{0.000000,0.000000,0.000000}%
\pgfsetstrokecolor{currentstroke}%
\pgfsetdash{}{0pt}%
\pgfpathmoveto{\pgfqpoint{3.502717in}{2.091450in}}%
\pgfpathlineto{\pgfqpoint{3.763681in}{2.314857in}}%
\pgfpathlineto{\pgfqpoint{3.502717in}{2.091450in}}%
\pgfpathclose%
\pgfusepath{fill}%
\end{pgfscope}%
\begin{pgfscope}%
\pgfpathrectangle{\pgfqpoint{0.017500in}{0.017500in}}{\pgfqpoint{6.218282in}{3.580372in}} %
\pgfusepath{clip}%
\pgfsetbuttcap%
\pgfsetroundjoin%
\definecolor{currentfill}{rgb}{0.877149,0.394645,0.311724}%
\pgfsetfillcolor{currentfill}%
\pgfsetlinewidth{0.000000pt}%
\definecolor{currentstroke}{rgb}{0.000000,0.000000,0.000000}%
\pgfsetstrokecolor{currentstroke}%
\pgfsetdash{}{0pt}%
\pgfpathmoveto{\pgfqpoint{3.502717in}{2.091450in}}%
\pgfpathlineto{\pgfqpoint{3.763681in}{2.314857in}}%
\pgfpathlineto{\pgfqpoint{3.502717in}{2.091450in}}%
\pgfpathclose%
\pgfusepath{fill}%
\end{pgfscope}%
\begin{pgfscope}%
\pgfpathrectangle{\pgfqpoint{0.017500in}{0.017500in}}{\pgfqpoint{6.218282in}{3.580372in}} %
\pgfusepath{clip}%
\pgfsetbuttcap%
\pgfsetroundjoin%
\definecolor{currentfill}{rgb}{0.869655,0.379274,0.300941}%
\pgfsetfillcolor{currentfill}%
\pgfsetlinewidth{0.000000pt}%
\definecolor{currentstroke}{rgb}{0.000000,0.000000,0.000000}%
\pgfsetstrokecolor{currentstroke}%
\pgfsetdash{}{0pt}%
\pgfpathmoveto{\pgfqpoint{3.502717in}{2.091450in}}%
\pgfpathlineto{\pgfqpoint{3.968799in}{2.026158in}}%
\pgfpathlineto{\pgfqpoint{4.227140in}{2.242353in}}%
\pgfpathlineto{\pgfqpoint{3.763681in}{2.314857in}}%
\pgfpathlineto{\pgfqpoint{3.502717in}{2.091450in}}%
\pgfpathclose%
\pgfusepath{fill}%
\end{pgfscope}%
\begin{pgfscope}%
\pgfpathrectangle{\pgfqpoint{0.017500in}{0.017500in}}{\pgfqpoint{6.218282in}{3.580372in}} %
\pgfusepath{clip}%
\pgfsetbuttcap%
\pgfsetroundjoin%
\definecolor{currentfill}{rgb}{0.869655,0.379274,0.300941}%
\pgfsetfillcolor{currentfill}%
\pgfsetlinewidth{0.000000pt}%
\definecolor{currentstroke}{rgb}{0.000000,0.000000,0.000000}%
\pgfsetstrokecolor{currentstroke}%
\pgfsetdash{}{0pt}%
\pgfpathmoveto{\pgfqpoint{3.502717in}{2.091450in}}%
\pgfpathlineto{\pgfqpoint{3.968799in}{2.026158in}}%
\pgfpathlineto{\pgfqpoint{4.227140in}{2.242353in}}%
\pgfpathlineto{\pgfqpoint{3.763681in}{2.314857in}}%
\pgfpathlineto{\pgfqpoint{3.502717in}{2.091450in}}%
\pgfpathclose%
\pgfusepath{fill}%
\end{pgfscope}%
\begin{pgfscope}%
\pgfpathrectangle{\pgfqpoint{0.017500in}{0.017500in}}{\pgfqpoint{6.218282in}{3.580372in}} %
\pgfusepath{clip}%
\pgfsetbuttcap%
\pgfsetroundjoin%
\definecolor{currentfill}{rgb}{0.869655,0.379274,0.300941}%
\pgfsetfillcolor{currentfill}%
\pgfsetlinewidth{0.000000pt}%
\definecolor{currentstroke}{rgb}{0.000000,0.000000,0.000000}%
\pgfsetstrokecolor{currentstroke}%
\pgfsetdash{}{0pt}%
\pgfpathmoveto{\pgfqpoint{3.502717in}{2.091450in}}%
\pgfpathlineto{\pgfqpoint{3.968799in}{2.026158in}}%
\pgfpathlineto{\pgfqpoint{4.227140in}{2.242353in}}%
\pgfpathlineto{\pgfqpoint{3.763681in}{2.314857in}}%
\pgfpathlineto{\pgfqpoint{3.502717in}{2.091450in}}%
\pgfpathclose%
\pgfusepath{fill}%
\end{pgfscope}%
\begin{pgfscope}%
\pgfpathrectangle{\pgfqpoint{0.017500in}{0.017500in}}{\pgfqpoint{6.218282in}{3.580372in}} %
\pgfusepath{clip}%
\pgfsetbuttcap%
\pgfsetroundjoin%
\definecolor{currentfill}{rgb}{0.869655,0.379274,0.300941}%
\pgfsetfillcolor{currentfill}%
\pgfsetlinewidth{0.000000pt}%
\definecolor{currentstroke}{rgb}{0.000000,0.000000,0.000000}%
\pgfsetstrokecolor{currentstroke}%
\pgfsetdash{}{0pt}%
\pgfpathmoveto{\pgfqpoint{3.502717in}{2.091450in}}%
\pgfpathlineto{\pgfqpoint{3.968799in}{2.026158in}}%
\pgfpathlineto{\pgfqpoint{4.227140in}{2.242353in}}%
\pgfpathlineto{\pgfqpoint{3.763681in}{2.314857in}}%
\pgfpathlineto{\pgfqpoint{3.502717in}{2.091450in}}%
\pgfpathclose%
\pgfusepath{fill}%
\end{pgfscope}%
\begin{pgfscope}%
\pgfpathrectangle{\pgfqpoint{0.017500in}{0.017500in}}{\pgfqpoint{6.218282in}{3.580372in}} %
\pgfusepath{clip}%
\pgfsetbuttcap%
\pgfsetroundjoin%
\definecolor{currentfill}{rgb}{0.869655,0.379274,0.300941}%
\pgfsetfillcolor{currentfill}%
\pgfsetlinewidth{0.000000pt}%
\definecolor{currentstroke}{rgb}{0.000000,0.000000,0.000000}%
\pgfsetstrokecolor{currentstroke}%
\pgfsetdash{}{0pt}%
\pgfpathmoveto{\pgfqpoint{3.502717in}{2.091450in}}%
\pgfpathlineto{\pgfqpoint{3.968799in}{2.026158in}}%
\pgfpathlineto{\pgfqpoint{4.227140in}{2.242353in}}%
\pgfpathlineto{\pgfqpoint{3.763681in}{2.314857in}}%
\pgfpathlineto{\pgfqpoint{3.502717in}{2.091450in}}%
\pgfpathclose%
\pgfusepath{fill}%
\end{pgfscope}%
\begin{pgfscope}%
\pgfpathrectangle{\pgfqpoint{0.017500in}{0.017500in}}{\pgfqpoint{6.218282in}{3.580372in}} %
\pgfusepath{clip}%
\pgfsetbuttcap%
\pgfsetroundjoin%
\definecolor{currentfill}{rgb}{0.869655,0.379274,0.300941}%
\pgfsetfillcolor{currentfill}%
\pgfsetlinewidth{0.000000pt}%
\definecolor{currentstroke}{rgb}{0.000000,0.000000,0.000000}%
\pgfsetstrokecolor{currentstroke}%
\pgfsetdash{}{0pt}%
\pgfpathmoveto{\pgfqpoint{3.502717in}{2.091450in}}%
\pgfpathlineto{\pgfqpoint{3.968799in}{2.026158in}}%
\pgfpathlineto{\pgfqpoint{4.227140in}{2.242353in}}%
\pgfpathlineto{\pgfqpoint{3.763681in}{2.314857in}}%
\pgfpathlineto{\pgfqpoint{3.502717in}{2.091450in}}%
\pgfpathclose%
\pgfusepath{fill}%
\end{pgfscope}%
\begin{pgfscope}%
\pgfpathrectangle{\pgfqpoint{0.017500in}{0.017500in}}{\pgfqpoint{6.218282in}{3.580372in}} %
\pgfusepath{clip}%
\pgfsetbuttcap%
\pgfsetroundjoin%
\definecolor{currentfill}{rgb}{0.869655,0.379274,0.300941}%
\pgfsetfillcolor{currentfill}%
\pgfsetlinewidth{0.000000pt}%
\definecolor{currentstroke}{rgb}{0.000000,0.000000,0.000000}%
\pgfsetstrokecolor{currentstroke}%
\pgfsetdash{}{0pt}%
\pgfpathmoveto{\pgfqpoint{3.968799in}{2.026158in}}%
\pgfpathlineto{\pgfqpoint{4.227140in}{2.242353in}}%
\pgfpathlineto{\pgfqpoint{3.968799in}{2.026158in}}%
\pgfpathclose%
\pgfusepath{fill}%
\end{pgfscope}%
\begin{pgfscope}%
\pgfpathrectangle{\pgfqpoint{0.017500in}{0.017500in}}{\pgfqpoint{6.218282in}{3.580372in}} %
\pgfusepath{clip}%
\pgfsetbuttcap%
\pgfsetroundjoin%
\definecolor{currentfill}{rgb}{0.869655,0.379274,0.300941}%
\pgfsetfillcolor{currentfill}%
\pgfsetlinewidth{0.000000pt}%
\definecolor{currentstroke}{rgb}{0.000000,0.000000,0.000000}%
\pgfsetstrokecolor{currentstroke}%
\pgfsetdash{}{0pt}%
\pgfpathmoveto{\pgfqpoint{3.968799in}{2.026158in}}%
\pgfpathlineto{\pgfqpoint{4.227140in}{2.242353in}}%
\pgfpathlineto{\pgfqpoint{3.968799in}{2.026158in}}%
\pgfpathclose%
\pgfusepath{fill}%
\end{pgfscope}%
\begin{pgfscope}%
\pgfpathrectangle{\pgfqpoint{0.017500in}{0.017500in}}{\pgfqpoint{6.218282in}{3.580372in}} %
\pgfusepath{clip}%
\pgfsetbuttcap%
\pgfsetroundjoin%
\definecolor{currentfill}{rgb}{0.869655,0.379274,0.300941}%
\pgfsetfillcolor{currentfill}%
\pgfsetlinewidth{0.000000pt}%
\definecolor{currentstroke}{rgb}{0.000000,0.000000,0.000000}%
\pgfsetstrokecolor{currentstroke}%
\pgfsetdash{}{0pt}%
\pgfpathmoveto{\pgfqpoint{3.968799in}{2.026158in}}%
\pgfpathlineto{\pgfqpoint{4.227140in}{2.242353in}}%
\pgfpathlineto{\pgfqpoint{3.968799in}{2.026158in}}%
\pgfpathclose%
\pgfusepath{fill}%
\end{pgfscope}%
\begin{pgfscope}%
\pgfpathrectangle{\pgfqpoint{0.017500in}{0.017500in}}{\pgfqpoint{6.218282in}{3.580372in}} %
\pgfusepath{clip}%
\pgfsetbuttcap%
\pgfsetroundjoin%
\definecolor{currentfill}{rgb}{0.869655,0.379274,0.300941}%
\pgfsetfillcolor{currentfill}%
\pgfsetlinewidth{0.000000pt}%
\definecolor{currentstroke}{rgb}{0.000000,0.000000,0.000000}%
\pgfsetstrokecolor{currentstroke}%
\pgfsetdash{}{0pt}%
\pgfpathmoveto{\pgfqpoint{3.968799in}{2.026158in}}%
\pgfpathlineto{\pgfqpoint{4.227140in}{2.242353in}}%
\pgfpathlineto{\pgfqpoint{3.968799in}{2.026158in}}%
\pgfpathclose%
\pgfusepath{fill}%
\end{pgfscope}%
\begin{pgfscope}%
\pgfpathrectangle{\pgfqpoint{0.017500in}{0.017500in}}{\pgfqpoint{6.218282in}{3.580372in}} %
\pgfusepath{clip}%
\pgfsetbuttcap%
\pgfsetroundjoin%
\definecolor{currentfill}{rgb}{0.869655,0.379274,0.300941}%
\pgfsetfillcolor{currentfill}%
\pgfsetlinewidth{0.000000pt}%
\definecolor{currentstroke}{rgb}{0.000000,0.000000,0.000000}%
\pgfsetstrokecolor{currentstroke}%
\pgfsetdash{}{0pt}%
\pgfpathmoveto{\pgfqpoint{3.968799in}{2.026158in}}%
\pgfpathlineto{\pgfqpoint{4.227140in}{2.242353in}}%
\pgfpathlineto{\pgfqpoint{3.968799in}{2.026158in}}%
\pgfpathclose%
\pgfusepath{fill}%
\end{pgfscope}%
\begin{pgfscope}%
\pgfpathrectangle{\pgfqpoint{0.017500in}{0.017500in}}{\pgfqpoint{6.218282in}{3.580372in}} %
\pgfusepath{clip}%
\pgfsetbuttcap%
\pgfsetroundjoin%
\definecolor{currentfill}{rgb}{0.869655,0.379274,0.300941}%
\pgfsetfillcolor{currentfill}%
\pgfsetlinewidth{0.000000pt}%
\definecolor{currentstroke}{rgb}{0.000000,0.000000,0.000000}%
\pgfsetstrokecolor{currentstroke}%
\pgfsetdash{}{0pt}%
\pgfpathmoveto{\pgfqpoint{3.968799in}{2.026158in}}%
\pgfpathlineto{\pgfqpoint{4.227140in}{2.242353in}}%
\pgfpathlineto{\pgfqpoint{3.968799in}{2.026158in}}%
\pgfpathclose%
\pgfusepath{fill}%
\end{pgfscope}%
\begin{pgfscope}%
\pgfpathrectangle{\pgfqpoint{0.017500in}{0.017500in}}{\pgfqpoint{6.218282in}{3.580372in}} %
\pgfusepath{clip}%
\pgfsetbuttcap%
\pgfsetroundjoin%
\definecolor{currentfill}{rgb}{0.869655,0.379274,0.300941}%
\pgfsetfillcolor{currentfill}%
\pgfsetlinewidth{0.000000pt}%
\definecolor{currentstroke}{rgb}{0.000000,0.000000,0.000000}%
\pgfsetstrokecolor{currentstroke}%
\pgfsetdash{}{0pt}%
\pgfpathmoveto{\pgfqpoint{3.968799in}{2.026158in}}%
\pgfpathlineto{\pgfqpoint{4.227140in}{2.242353in}}%
\pgfpathlineto{\pgfqpoint{3.968799in}{2.026158in}}%
\pgfpathclose%
\pgfusepath{fill}%
\end{pgfscope}%
\begin{pgfscope}%
\pgfpathrectangle{\pgfqpoint{0.017500in}{0.017500in}}{\pgfqpoint{6.218282in}{3.580372in}} %
\pgfusepath{clip}%
\pgfsetbuttcap%
\pgfsetroundjoin%
\definecolor{currentfill}{rgb}{0.869655,0.379274,0.300941}%
\pgfsetfillcolor{currentfill}%
\pgfsetlinewidth{0.000000pt}%
\definecolor{currentstroke}{rgb}{0.000000,0.000000,0.000000}%
\pgfsetstrokecolor{currentstroke}%
\pgfsetdash{}{0pt}%
\pgfpathmoveto{\pgfqpoint{3.968799in}{2.026158in}}%
\pgfpathlineto{\pgfqpoint{4.227140in}{2.242353in}}%
\pgfpathlineto{\pgfqpoint{3.968799in}{2.026158in}}%
\pgfpathclose%
\pgfusepath{fill}%
\end{pgfscope}%
\begin{pgfscope}%
\pgfpathrectangle{\pgfqpoint{0.017500in}{0.017500in}}{\pgfqpoint{6.218282in}{3.580372in}} %
\pgfusepath{clip}%
\pgfsetbuttcap%
\pgfsetroundjoin%
\definecolor{currentfill}{rgb}{0.869655,0.379274,0.300941}%
\pgfsetfillcolor{currentfill}%
\pgfsetlinewidth{0.000000pt}%
\definecolor{currentstroke}{rgb}{0.000000,0.000000,0.000000}%
\pgfsetstrokecolor{currentstroke}%
\pgfsetdash{}{0pt}%
\pgfpathmoveto{\pgfqpoint{3.968799in}{2.026158in}}%
\pgfpathlineto{\pgfqpoint{4.227140in}{2.242353in}}%
\pgfpathlineto{\pgfqpoint{3.968799in}{2.026158in}}%
\pgfpathclose%
\pgfusepath{fill}%
\end{pgfscope}%
\begin{pgfscope}%
\pgfpathrectangle{\pgfqpoint{0.017500in}{0.017500in}}{\pgfqpoint{6.218282in}{3.580372in}} %
\pgfusepath{clip}%
\pgfsetbuttcap%
\pgfsetroundjoin%
\definecolor{currentfill}{rgb}{0.869655,0.379274,0.300941}%
\pgfsetfillcolor{currentfill}%
\pgfsetlinewidth{0.000000pt}%
\definecolor{currentstroke}{rgb}{0.000000,0.000000,0.000000}%
\pgfsetstrokecolor{currentstroke}%
\pgfsetdash{}{0pt}%
\pgfpathmoveto{\pgfqpoint{3.968799in}{2.026158in}}%
\pgfpathlineto{\pgfqpoint{4.227140in}{2.242353in}}%
\pgfpathlineto{\pgfqpoint{3.968799in}{2.026158in}}%
\pgfpathclose%
\pgfusepath{fill}%
\end{pgfscope}%
\begin{pgfscope}%
\pgfpathrectangle{\pgfqpoint{0.017500in}{0.017500in}}{\pgfqpoint{6.218282in}{3.580372in}} %
\pgfusepath{clip}%
\pgfsetbuttcap%
\pgfsetroundjoin%
\definecolor{currentfill}{rgb}{0.869655,0.379274,0.300941}%
\pgfsetfillcolor{currentfill}%
\pgfsetlinewidth{0.000000pt}%
\definecolor{currentstroke}{rgb}{0.000000,0.000000,0.000000}%
\pgfsetstrokecolor{currentstroke}%
\pgfsetdash{}{0pt}%
\pgfpathmoveto{\pgfqpoint{3.968799in}{2.026158in}}%
\pgfpathlineto{\pgfqpoint{4.227140in}{2.242353in}}%
\pgfpathlineto{\pgfqpoint{3.968799in}{2.026158in}}%
\pgfpathclose%
\pgfusepath{fill}%
\end{pgfscope}%
\begin{pgfscope}%
\pgfpathrectangle{\pgfqpoint{0.017500in}{0.017500in}}{\pgfqpoint{6.218282in}{3.580372in}} %
\pgfusepath{clip}%
\pgfsetbuttcap%
\pgfsetroundjoin%
\definecolor{currentfill}{rgb}{0.869655,0.379274,0.300941}%
\pgfsetfillcolor{currentfill}%
\pgfsetlinewidth{0.000000pt}%
\definecolor{currentstroke}{rgb}{0.000000,0.000000,0.000000}%
\pgfsetstrokecolor{currentstroke}%
\pgfsetdash{}{0pt}%
\pgfpathmoveto{\pgfqpoint{3.968799in}{2.026158in}}%
\pgfpathlineto{\pgfqpoint{4.227140in}{2.242353in}}%
\pgfpathlineto{\pgfqpoint{3.968799in}{2.026158in}}%
\pgfpathclose%
\pgfusepath{fill}%
\end{pgfscope}%
\begin{pgfscope}%
\pgfpathrectangle{\pgfqpoint{0.017500in}{0.017500in}}{\pgfqpoint{6.218282in}{3.580372in}} %
\pgfusepath{clip}%
\pgfsetbuttcap%
\pgfsetroundjoin%
\definecolor{currentfill}{rgb}{0.869655,0.379274,0.300941}%
\pgfsetfillcolor{currentfill}%
\pgfsetlinewidth{0.000000pt}%
\definecolor{currentstroke}{rgb}{0.000000,0.000000,0.000000}%
\pgfsetstrokecolor{currentstroke}%
\pgfsetdash{}{0pt}%
\pgfpathmoveto{\pgfqpoint{3.968799in}{2.026158in}}%
\pgfpathlineto{\pgfqpoint{4.227140in}{2.242353in}}%
\pgfpathlineto{\pgfqpoint{3.968799in}{2.026158in}}%
\pgfpathclose%
\pgfusepath{fill}%
\end{pgfscope}%
\begin{pgfscope}%
\pgfpathrectangle{\pgfqpoint{0.017500in}{0.017500in}}{\pgfqpoint{6.218282in}{3.580372in}} %
\pgfusepath{clip}%
\pgfsetbuttcap%
\pgfsetroundjoin%
\definecolor{currentfill}{rgb}{0.869655,0.379274,0.300941}%
\pgfsetfillcolor{currentfill}%
\pgfsetlinewidth{0.000000pt}%
\definecolor{currentstroke}{rgb}{0.000000,0.000000,0.000000}%
\pgfsetstrokecolor{currentstroke}%
\pgfsetdash{}{0pt}%
\pgfpathmoveto{\pgfqpoint{3.968799in}{2.026158in}}%
\pgfpathlineto{\pgfqpoint{4.227140in}{2.242353in}}%
\pgfpathlineto{\pgfqpoint{3.968799in}{2.026158in}}%
\pgfpathclose%
\pgfusepath{fill}%
\end{pgfscope}%
\begin{pgfscope}%
\pgfpathrectangle{\pgfqpoint{0.017500in}{0.017500in}}{\pgfqpoint{6.218282in}{3.580372in}} %
\pgfusepath{clip}%
\pgfsetbuttcap%
\pgfsetroundjoin%
\definecolor{currentfill}{rgb}{0.869655,0.379274,0.300941}%
\pgfsetfillcolor{currentfill}%
\pgfsetlinewidth{0.000000pt}%
\definecolor{currentstroke}{rgb}{0.000000,0.000000,0.000000}%
\pgfsetstrokecolor{currentstroke}%
\pgfsetdash{}{0pt}%
\pgfpathmoveto{\pgfqpoint{3.968799in}{2.026158in}}%
\pgfpathlineto{\pgfqpoint{4.227140in}{2.242353in}}%
\pgfpathlineto{\pgfqpoint{3.968799in}{2.026158in}}%
\pgfpathclose%
\pgfusepath{fill}%
\end{pgfscope}%
\begin{pgfscope}%
\pgfpathrectangle{\pgfqpoint{0.017500in}{0.017500in}}{\pgfqpoint{6.218282in}{3.580372in}} %
\pgfusepath{clip}%
\pgfsetbuttcap%
\pgfsetroundjoin%
\definecolor{currentfill}{rgb}{0.869655,0.379274,0.300941}%
\pgfsetfillcolor{currentfill}%
\pgfsetlinewidth{0.000000pt}%
\definecolor{currentstroke}{rgb}{0.000000,0.000000,0.000000}%
\pgfsetstrokecolor{currentstroke}%
\pgfsetdash{}{0pt}%
\pgfpathmoveto{\pgfqpoint{3.968799in}{2.026158in}}%
\pgfpathlineto{\pgfqpoint{4.227140in}{2.242353in}}%
\pgfpathlineto{\pgfqpoint{3.968799in}{2.026158in}}%
\pgfpathclose%
\pgfusepath{fill}%
\end{pgfscope}%
\begin{pgfscope}%
\pgfpathrectangle{\pgfqpoint{0.017500in}{0.017500in}}{\pgfqpoint{6.218282in}{3.580372in}} %
\pgfusepath{clip}%
\pgfsetbuttcap%
\pgfsetroundjoin%
\definecolor{currentfill}{rgb}{0.869655,0.379274,0.300941}%
\pgfsetfillcolor{currentfill}%
\pgfsetlinewidth{0.000000pt}%
\definecolor{currentstroke}{rgb}{0.000000,0.000000,0.000000}%
\pgfsetstrokecolor{currentstroke}%
\pgfsetdash{}{0pt}%
\pgfpathmoveto{\pgfqpoint{3.968799in}{2.026158in}}%
\pgfpathlineto{\pgfqpoint{4.227140in}{2.242353in}}%
\pgfpathlineto{\pgfqpoint{3.968799in}{2.026158in}}%
\pgfpathclose%
\pgfusepath{fill}%
\end{pgfscope}%
\begin{pgfscope}%
\pgfpathrectangle{\pgfqpoint{0.017500in}{0.017500in}}{\pgfqpoint{6.218282in}{3.580372in}} %
\pgfusepath{clip}%
\pgfsetbuttcap%
\pgfsetroundjoin%
\definecolor{currentfill}{rgb}{0.869655,0.379274,0.300941}%
\pgfsetfillcolor{currentfill}%
\pgfsetlinewidth{0.000000pt}%
\definecolor{currentstroke}{rgb}{0.000000,0.000000,0.000000}%
\pgfsetstrokecolor{currentstroke}%
\pgfsetdash{}{0pt}%
\pgfpathmoveto{\pgfqpoint{3.968799in}{2.026158in}}%
\pgfpathlineto{\pgfqpoint{4.227140in}{2.242353in}}%
\pgfpathlineto{\pgfqpoint{3.968799in}{2.026158in}}%
\pgfpathclose%
\pgfusepath{fill}%
\end{pgfscope}%
\begin{pgfscope}%
\pgfpathrectangle{\pgfqpoint{0.017500in}{0.017500in}}{\pgfqpoint{6.218282in}{3.580372in}} %
\pgfusepath{clip}%
\pgfsetbuttcap%
\pgfsetroundjoin%
\definecolor{currentfill}{rgb}{0.869655,0.379274,0.300941}%
\pgfsetfillcolor{currentfill}%
\pgfsetlinewidth{0.000000pt}%
\definecolor{currentstroke}{rgb}{0.000000,0.000000,0.000000}%
\pgfsetstrokecolor{currentstroke}%
\pgfsetdash{}{0pt}%
\pgfpathmoveto{\pgfqpoint{3.968799in}{2.026158in}}%
\pgfpathlineto{\pgfqpoint{4.227140in}{2.242353in}}%
\pgfpathlineto{\pgfqpoint{3.968799in}{2.026158in}}%
\pgfpathclose%
\pgfusepath{fill}%
\end{pgfscope}%
\begin{pgfscope}%
\pgfpathrectangle{\pgfqpoint{0.017500in}{0.017500in}}{\pgfqpoint{6.218282in}{3.580372in}} %
\pgfusepath{clip}%
\pgfsetbuttcap%
\pgfsetroundjoin%
\definecolor{currentfill}{rgb}{0.869655,0.379274,0.300941}%
\pgfsetfillcolor{currentfill}%
\pgfsetlinewidth{0.000000pt}%
\definecolor{currentstroke}{rgb}{0.000000,0.000000,0.000000}%
\pgfsetstrokecolor{currentstroke}%
\pgfsetdash{}{0pt}%
\pgfpathmoveto{\pgfqpoint{3.968799in}{2.026158in}}%
\pgfpathlineto{\pgfqpoint{4.227140in}{2.242353in}}%
\pgfpathlineto{\pgfqpoint{3.968799in}{2.026158in}}%
\pgfpathclose%
\pgfusepath{fill}%
\end{pgfscope}%
\begin{pgfscope}%
\pgfpathrectangle{\pgfqpoint{0.017500in}{0.017500in}}{\pgfqpoint{6.218282in}{3.580372in}} %
\pgfusepath{clip}%
\pgfsetbuttcap%
\pgfsetroundjoin%
\definecolor{currentfill}{rgb}{0.869655,0.379274,0.300941}%
\pgfsetfillcolor{currentfill}%
\pgfsetlinewidth{0.000000pt}%
\definecolor{currentstroke}{rgb}{0.000000,0.000000,0.000000}%
\pgfsetstrokecolor{currentstroke}%
\pgfsetdash{}{0pt}%
\pgfpathmoveto{\pgfqpoint{3.968799in}{2.026158in}}%
\pgfpathlineto{\pgfqpoint{4.227140in}{2.242353in}}%
\pgfpathlineto{\pgfqpoint{3.968799in}{2.026158in}}%
\pgfpathclose%
\pgfusepath{fill}%
\end{pgfscope}%
\begin{pgfscope}%
\pgfpathrectangle{\pgfqpoint{0.017500in}{0.017500in}}{\pgfqpoint{6.218282in}{3.580372in}} %
\pgfusepath{clip}%
\pgfsetbuttcap%
\pgfsetroundjoin%
\definecolor{currentfill}{rgb}{0.869655,0.379274,0.300941}%
\pgfsetfillcolor{currentfill}%
\pgfsetlinewidth{0.000000pt}%
\definecolor{currentstroke}{rgb}{0.000000,0.000000,0.000000}%
\pgfsetstrokecolor{currentstroke}%
\pgfsetdash{}{0pt}%
\pgfpathmoveto{\pgfqpoint{3.968799in}{2.026158in}}%
\pgfpathlineto{\pgfqpoint{4.227140in}{2.242353in}}%
\pgfpathlineto{\pgfqpoint{3.968799in}{2.026158in}}%
\pgfpathclose%
\pgfusepath{fill}%
\end{pgfscope}%
\begin{pgfscope}%
\pgfpathrectangle{\pgfqpoint{0.017500in}{0.017500in}}{\pgfqpoint{6.218282in}{3.580372in}} %
\pgfusepath{clip}%
\pgfsetbuttcap%
\pgfsetroundjoin%
\definecolor{currentfill}{rgb}{0.869655,0.379274,0.300941}%
\pgfsetfillcolor{currentfill}%
\pgfsetlinewidth{0.000000pt}%
\definecolor{currentstroke}{rgb}{0.000000,0.000000,0.000000}%
\pgfsetstrokecolor{currentstroke}%
\pgfsetdash{}{0pt}%
\pgfpathmoveto{\pgfqpoint{3.968799in}{2.026158in}}%
\pgfpathlineto{\pgfqpoint{4.227140in}{2.242353in}}%
\pgfpathlineto{\pgfqpoint{3.968799in}{2.026158in}}%
\pgfpathclose%
\pgfusepath{fill}%
\end{pgfscope}%
\begin{pgfscope}%
\pgfpathrectangle{\pgfqpoint{0.017500in}{0.017500in}}{\pgfqpoint{6.218282in}{3.580372in}} %
\pgfusepath{clip}%
\pgfsetbuttcap%
\pgfsetroundjoin%
\definecolor{currentfill}{rgb}{0.869655,0.379274,0.300941}%
\pgfsetfillcolor{currentfill}%
\pgfsetlinewidth{0.000000pt}%
\definecolor{currentstroke}{rgb}{0.000000,0.000000,0.000000}%
\pgfsetstrokecolor{currentstroke}%
\pgfsetdash{}{0pt}%
\pgfpathmoveto{\pgfqpoint{3.968799in}{2.026158in}}%
\pgfpathlineto{\pgfqpoint{4.227140in}{2.242353in}}%
\pgfpathlineto{\pgfqpoint{3.968799in}{2.026158in}}%
\pgfpathclose%
\pgfusepath{fill}%
\end{pgfscope}%
\begin{pgfscope}%
\pgfpathrectangle{\pgfqpoint{0.017500in}{0.017500in}}{\pgfqpoint{6.218282in}{3.580372in}} %
\pgfusepath{clip}%
\pgfsetbuttcap%
\pgfsetroundjoin%
\definecolor{currentfill}{rgb}{0.869655,0.379274,0.300941}%
\pgfsetfillcolor{currentfill}%
\pgfsetlinewidth{0.000000pt}%
\definecolor{currentstroke}{rgb}{0.000000,0.000000,0.000000}%
\pgfsetstrokecolor{currentstroke}%
\pgfsetdash{}{0pt}%
\pgfpathmoveto{\pgfqpoint{3.968799in}{2.026158in}}%
\pgfpathlineto{\pgfqpoint{4.227140in}{2.242353in}}%
\pgfpathlineto{\pgfqpoint{3.968799in}{2.026158in}}%
\pgfpathclose%
\pgfusepath{fill}%
\end{pgfscope}%
\begin{pgfscope}%
\pgfpathrectangle{\pgfqpoint{0.017500in}{0.017500in}}{\pgfqpoint{6.218282in}{3.580372in}} %
\pgfusepath{clip}%
\pgfsetbuttcap%
\pgfsetroundjoin%
\definecolor{currentfill}{rgb}{0.869655,0.379274,0.300941}%
\pgfsetfillcolor{currentfill}%
\pgfsetlinewidth{0.000000pt}%
\definecolor{currentstroke}{rgb}{0.000000,0.000000,0.000000}%
\pgfsetstrokecolor{currentstroke}%
\pgfsetdash{}{0pt}%
\pgfpathmoveto{\pgfqpoint{3.968799in}{2.026158in}}%
\pgfpathlineto{\pgfqpoint{4.227140in}{2.242353in}}%
\pgfpathlineto{\pgfqpoint{3.968799in}{2.026158in}}%
\pgfpathclose%
\pgfusepath{fill}%
\end{pgfscope}%
\begin{pgfscope}%
\pgfpathrectangle{\pgfqpoint{0.017500in}{0.017500in}}{\pgfqpoint{6.218282in}{3.580372in}} %
\pgfusepath{clip}%
\pgfsetbuttcap%
\pgfsetroundjoin%
\definecolor{currentfill}{rgb}{0.869655,0.379274,0.300941}%
\pgfsetfillcolor{currentfill}%
\pgfsetlinewidth{0.000000pt}%
\definecolor{currentstroke}{rgb}{0.000000,0.000000,0.000000}%
\pgfsetstrokecolor{currentstroke}%
\pgfsetdash{}{0pt}%
\pgfpathmoveto{\pgfqpoint{3.968799in}{2.026158in}}%
\pgfpathlineto{\pgfqpoint{4.227140in}{2.242353in}}%
\pgfpathlineto{\pgfqpoint{3.968799in}{2.026158in}}%
\pgfpathclose%
\pgfusepath{fill}%
\end{pgfscope}%
\begin{pgfscope}%
\pgfpathrectangle{\pgfqpoint{0.017500in}{0.017500in}}{\pgfqpoint{6.218282in}{3.580372in}} %
\pgfusepath{clip}%
\pgfsetbuttcap%
\pgfsetroundjoin%
\definecolor{currentfill}{rgb}{0.869655,0.379274,0.300941}%
\pgfsetfillcolor{currentfill}%
\pgfsetlinewidth{0.000000pt}%
\definecolor{currentstroke}{rgb}{0.000000,0.000000,0.000000}%
\pgfsetstrokecolor{currentstroke}%
\pgfsetdash{}{0pt}%
\pgfpathmoveto{\pgfqpoint{3.968799in}{2.026158in}}%
\pgfpathlineto{\pgfqpoint{4.227140in}{2.242353in}}%
\pgfpathlineto{\pgfqpoint{3.968799in}{2.026158in}}%
\pgfpathclose%
\pgfusepath{fill}%
\end{pgfscope}%
\begin{pgfscope}%
\pgfpathrectangle{\pgfqpoint{0.017500in}{0.017500in}}{\pgfqpoint{6.218282in}{3.580372in}} %
\pgfusepath{clip}%
\pgfsetbuttcap%
\pgfsetroundjoin%
\definecolor{currentfill}{rgb}{0.869655,0.379274,0.300941}%
\pgfsetfillcolor{currentfill}%
\pgfsetlinewidth{0.000000pt}%
\definecolor{currentstroke}{rgb}{0.000000,0.000000,0.000000}%
\pgfsetstrokecolor{currentstroke}%
\pgfsetdash{}{0pt}%
\pgfpathmoveto{\pgfqpoint{3.968799in}{2.026158in}}%
\pgfpathlineto{\pgfqpoint{4.227140in}{2.242353in}}%
\pgfpathlineto{\pgfqpoint{3.968799in}{2.026158in}}%
\pgfpathclose%
\pgfusepath{fill}%
\end{pgfscope}%
\begin{pgfscope}%
\pgfpathrectangle{\pgfqpoint{0.017500in}{0.017500in}}{\pgfqpoint{6.218282in}{3.580372in}} %
\pgfusepath{clip}%
\pgfsetbuttcap%
\pgfsetroundjoin%
\definecolor{currentfill}{rgb}{0.869655,0.379274,0.300941}%
\pgfsetfillcolor{currentfill}%
\pgfsetlinewidth{0.000000pt}%
\definecolor{currentstroke}{rgb}{0.000000,0.000000,0.000000}%
\pgfsetstrokecolor{currentstroke}%
\pgfsetdash{}{0pt}%
\pgfpathmoveto{\pgfqpoint{3.968799in}{2.026158in}}%
\pgfpathlineto{\pgfqpoint{4.227140in}{2.242353in}}%
\pgfpathlineto{\pgfqpoint{3.968799in}{2.026158in}}%
\pgfpathclose%
\pgfusepath{fill}%
\end{pgfscope}%
\end{pgfpicture}%
\makeatother%
\endgroup%

	\caption{Learning surface for SemEval and CONLL2000 Chunk.}
\end{figure}
\begin{figure}[h!]
	\hspace*{-2cm}%% Creator: Matplotlib, PGF backend
%%
%% To include the figure in your LaTeX document, write
%%   \input{<filename>.pgf}
%%
%% Make sure the required packages are loaded in your preamble
%%   \usepackage{pgf}
%%
%% Figures using additional raster images can only be included by \input if
%% they are in the same directory as the main LaTeX file. For loading figures
%% from other directories you can use the `import` package
%%   \usepackage{import}
%% and then include the figures with
%%   \import{<path to file>}{<filename>.pgf}
%%
%% Matplotlib used the following preamble
%%   \usepackage{fontspec}
%%   \setmainfont{Palatino}
%%   \setsansfont{Lucida Grande}
%%   \setmonofont{Andale Mono}
%%
\begingroup%
\makeatletter%
\begin{pgfpicture}%
\pgfpathrectangle{\pgfpointorigin}{\pgfqpoint{6.253282in}{3.615372in}}%
\pgfusepath{use as bounding box, clip}%
\begin{pgfscope}%
\pgfsetbuttcap%
\pgfsetmiterjoin%
\definecolor{currentfill}{rgb}{1.000000,1.000000,1.000000}%
\pgfsetfillcolor{currentfill}%
\pgfsetlinewidth{0.000000pt}%
\definecolor{currentstroke}{rgb}{1.000000,1.000000,1.000000}%
\pgfsetstrokecolor{currentstroke}%
\pgfsetdash{}{0pt}%
\pgfpathmoveto{\pgfqpoint{0.000000in}{0.000000in}}%
\pgfpathlineto{\pgfqpoint{6.253282in}{0.000000in}}%
\pgfpathlineto{\pgfqpoint{6.253282in}{3.615372in}}%
\pgfpathlineto{\pgfqpoint{0.000000in}{3.615372in}}%
\pgfpathclose%
\pgfusepath{fill}%
\end{pgfscope}%
\begin{pgfscope}%
\pgfsetbuttcap%
\pgfsetmiterjoin%
\definecolor{currentfill}{rgb}{1.000000,1.000000,1.000000}%
\pgfsetfillcolor{currentfill}%
\pgfsetlinewidth{0.000000pt}%
\definecolor{currentstroke}{rgb}{0.000000,0.000000,0.000000}%
\pgfsetstrokecolor{currentstroke}%
\pgfsetstrokeopacity{0.000000}%
\pgfsetdash{}{0pt}%
\pgfpathmoveto{\pgfqpoint{0.017500in}{0.017500in}}%
\pgfpathlineto{\pgfqpoint{6.235782in}{0.017500in}}%
\pgfpathlineto{\pgfqpoint{6.235782in}{3.597872in}}%
\pgfpathlineto{\pgfqpoint{0.017500in}{3.597872in}}%
\pgfpathclose%
\pgfusepath{fill}%
\end{pgfscope}%
\begin{pgfscope}%
\pgfsetbuttcap%
\pgfsetmiterjoin%
\definecolor{currentfill}{rgb}{0.950000,0.950000,0.950000}%
\pgfsetfillcolor{currentfill}%
\pgfsetfillopacity{0.500000}%
\pgfsetlinewidth{0.200750pt}%
\definecolor{currentstroke}{rgb}{0.950000,0.950000,0.950000}%
\pgfsetstrokecolor{currentstroke}%
\pgfsetstrokeopacity{0.500000}%
\pgfsetdash{}{0pt}%
\pgfpathmoveto{\pgfqpoint{1.233525in}{0.981833in}}%
\pgfpathlineto{\pgfqpoint{2.713039in}{1.712806in}}%
\pgfpathlineto{\pgfqpoint{2.692063in}{3.119783in}}%
\pgfpathlineto{\pgfqpoint{1.144557in}{2.450885in}}%
\pgfusepath{stroke,fill}%
\end{pgfscope}%
\begin{pgfscope}%
\pgfsetbuttcap%
\pgfsetmiterjoin%
\definecolor{currentfill}{rgb}{0.900000,0.900000,0.900000}%
\pgfsetfillcolor{currentfill}%
\pgfsetfillopacity{0.500000}%
\pgfsetlinewidth{0.200750pt}%
\definecolor{currentstroke}{rgb}{0.900000,0.900000,0.900000}%
\pgfsetstrokecolor{currentstroke}%
\pgfsetstrokeopacity{0.500000}%
\pgfsetdash{}{0pt}%
\pgfpathmoveto{\pgfqpoint{2.713039in}{1.712806in}}%
\pgfpathlineto{\pgfqpoint{5.135278in}{1.301993in}}%
\pgfpathlineto{\pgfqpoint{5.219480in}{2.744306in}}%
\pgfpathlineto{\pgfqpoint{2.692063in}{3.119783in}}%
\pgfusepath{stroke,fill}%
\end{pgfscope}%
\begin{pgfscope}%
\pgfsetbuttcap%
\pgfsetmiterjoin%
\definecolor{currentfill}{rgb}{0.925000,0.925000,0.925000}%
\pgfsetfillcolor{currentfill}%
\pgfsetfillopacity{0.500000}%
\pgfsetlinewidth{0.200750pt}%
\definecolor{currentstroke}{rgb}{0.925000,0.925000,0.925000}%
\pgfsetstrokecolor{currentstroke}%
\pgfsetstrokeopacity{0.500000}%
\pgfsetdash{}{0pt}%
\pgfpathmoveto{\pgfqpoint{1.233525in}{0.981833in}}%
\pgfpathlineto{\pgfqpoint{3.760968in}{0.515142in}}%
\pgfpathlineto{\pgfqpoint{5.135278in}{1.301993in}}%
\pgfpathlineto{\pgfqpoint{2.713039in}{1.712806in}}%
\pgfusepath{stroke,fill}%
\end{pgfscope}%
\begin{pgfscope}%
\pgfsetrectcap%
\pgfsetroundjoin%
\pgfsetlinewidth{0.501875pt}%
\definecolor{currentstroke}{rgb}{0.000000,0.000000,0.000000}%
\pgfsetstrokecolor{currentstroke}%
\pgfsetdash{}{0pt}%
\pgfpathmoveto{\pgfqpoint{1.233525in}{0.981833in}}%
\pgfpathlineto{\pgfqpoint{3.760968in}{0.515142in}}%
\pgfusepath{stroke}%
\end{pgfscope}%
\begin{pgfscope}%
\pgftext[x=2.292998in,y=0.574477in,left,base,rotate=349.538214]{\rmfamily\fontsize{10.000000}{12.000000}\selectfont }%
\end{pgfscope}%
\begin{pgfscope}%
\pgftext[x=1.953595in,y=0.474946in,left,base,rotate=349.538214]{\rmfamily\fontsize{10.000000}{12.000000}\selectfont fraction of}%
\end{pgfscope}%
\begin{pgfscope}%
\pgftext[x=1.482715in,y=0.399693in,left,base,rotate=349.538214]{\rmfamily\fontsize{10.000000}{12.000000}\selectfont CONLL2000 Chunk data}%
\end{pgfscope}%
\begin{pgfscope}%
\pgfsetbuttcap%
\pgfsetroundjoin%
\pgfsetlinewidth{0.803000pt}%
\definecolor{currentstroke}{rgb}{0.690196,0.690196,0.690196}%
\pgfsetstrokecolor{currentstroke}%
\pgfsetdash{}{0pt}%
\pgfpathmoveto{\pgfqpoint{1.388733in}{0.953174in}}%
\pgfpathlineto{\pgfqpoint{2.862115in}{1.687522in}}%
\pgfpathlineto{\pgfqpoint{2.847388in}{3.096708in}}%
\pgfusepath{stroke}%
\end{pgfscope}%
\begin{pgfscope}%
\pgfsetbuttcap%
\pgfsetroundjoin%
\pgfsetlinewidth{0.803000pt}%
\definecolor{currentstroke}{rgb}{0.690196,0.690196,0.690196}%
\pgfsetstrokecolor{currentstroke}%
\pgfsetdash{}{0pt}%
\pgfpathmoveto{\pgfqpoint{1.818260in}{0.873862in}}%
\pgfpathlineto{\pgfqpoint{3.274447in}{1.617591in}}%
\pgfpathlineto{\pgfqpoint{3.277159in}{3.032860in}}%
\pgfusepath{stroke}%
\end{pgfscope}%
\begin{pgfscope}%
\pgfsetbuttcap%
\pgfsetroundjoin%
\pgfsetlinewidth{0.803000pt}%
\definecolor{currentstroke}{rgb}{0.690196,0.690196,0.690196}%
\pgfsetstrokecolor{currentstroke}%
\pgfsetdash{}{0pt}%
\pgfpathmoveto{\pgfqpoint{2.253467in}{0.793501in}}%
\pgfpathlineto{\pgfqpoint{3.691895in}{1.546791in}}%
\pgfpathlineto{\pgfqpoint{3.712491in}{2.968187in}}%
\pgfusepath{stroke}%
\end{pgfscope}%
\begin{pgfscope}%
\pgfsetbuttcap%
\pgfsetroundjoin%
\pgfsetlinewidth{0.803000pt}%
\definecolor{currentstroke}{rgb}{0.690196,0.690196,0.690196}%
\pgfsetstrokecolor{currentstroke}%
\pgfsetdash{}{0pt}%
\pgfpathmoveto{\pgfqpoint{2.694467in}{0.712071in}}%
\pgfpathlineto{\pgfqpoint{4.114556in}{1.475108in}}%
\pgfpathlineto{\pgfqpoint{4.153492in}{2.902671in}}%
\pgfusepath{stroke}%
\end{pgfscope}%
\begin{pgfscope}%
\pgfsetbuttcap%
\pgfsetroundjoin%
\pgfsetlinewidth{0.803000pt}%
\definecolor{currentstroke}{rgb}{0.690196,0.690196,0.690196}%
\pgfsetstrokecolor{currentstroke}%
\pgfsetdash{}{0pt}%
\pgfpathmoveto{\pgfqpoint{3.141375in}{0.629550in}}%
\pgfpathlineto{\pgfqpoint{4.542526in}{1.402524in}}%
\pgfpathlineto{\pgfqpoint{4.600274in}{2.836296in}}%
\pgfusepath{stroke}%
\end{pgfscope}%
\begin{pgfscope}%
\pgfsetbuttcap%
\pgfsetroundjoin%
\pgfsetlinewidth{0.803000pt}%
\definecolor{currentstroke}{rgb}{0.690196,0.690196,0.690196}%
\pgfsetstrokecolor{currentstroke}%
\pgfsetdash{}{0pt}%
\pgfpathmoveto{\pgfqpoint{3.594311in}{0.545915in}}%
\pgfpathlineto{\pgfqpoint{4.975908in}{1.329022in}}%
\pgfpathlineto{\pgfqpoint{5.052952in}{2.769046in}}%
\pgfusepath{stroke}%
\end{pgfscope}%
\begin{pgfscope}%
\pgfsetrectcap%
\pgfsetroundjoin%
\pgfsetlinewidth{0.803000pt}%
\definecolor{currentstroke}{rgb}{0.000000,0.000000,0.000000}%
\pgfsetstrokecolor{currentstroke}%
\pgfsetdash{}{0pt}%
\pgfpathmoveto{\pgfqpoint{1.401276in}{0.959425in}}%
\pgfpathlineto{\pgfqpoint{1.363606in}{0.940650in}}%
\pgfusepath{stroke}%
\end{pgfscope}%
\begin{pgfscope}%
\pgftext[x=1.295334in,y=0.815921in,,top]{\rmfamily\fontsize{8.000000}{9.600000}\selectfont 0.0}%
\end{pgfscope}%
\begin{pgfscope}%
\pgfsetrectcap%
\pgfsetroundjoin%
\pgfsetlinewidth{0.803000pt}%
\definecolor{currentstroke}{rgb}{0.000000,0.000000,0.000000}%
\pgfsetstrokecolor{currentstroke}%
\pgfsetdash{}{0pt}%
\pgfpathmoveto{\pgfqpoint{1.830663in}{0.880196in}}%
\pgfpathlineto{\pgfqpoint{1.793417in}{0.861173in}}%
\pgfusepath{stroke}%
\end{pgfscope}%
\begin{pgfscope}%
\pgftext[x=1.724832in,y=0.735598in,,top]{\rmfamily\fontsize{8.000000}{9.600000}\selectfont 0.2}%
\end{pgfscope}%
\begin{pgfscope}%
\pgfsetrectcap%
\pgfsetroundjoin%
\pgfsetlinewidth{0.803000pt}%
\definecolor{currentstroke}{rgb}{0.000000,0.000000,0.000000}%
\pgfsetstrokecolor{currentstroke}%
\pgfsetdash{}{0pt}%
\pgfpathmoveto{\pgfqpoint{2.265723in}{0.799919in}}%
\pgfpathlineto{\pgfqpoint{2.228917in}{0.780644in}}%
\pgfusepath{stroke}%
\end{pgfscope}%
\begin{pgfscope}%
\pgftext[x=2.160017in,y=0.654212in,,top]{\rmfamily\fontsize{8.000000}{9.600000}\selectfont 0.4}%
\end{pgfscope}%
\begin{pgfscope}%
\pgfsetrectcap%
\pgfsetroundjoin%
\pgfsetlinewidth{0.803000pt}%
\definecolor{currentstroke}{rgb}{0.000000,0.000000,0.000000}%
\pgfsetstrokecolor{currentstroke}%
\pgfsetdash{}{0pt}%
\pgfpathmoveto{\pgfqpoint{2.706571in}{0.718575in}}%
\pgfpathlineto{\pgfqpoint{2.670219in}{0.699042in}}%
\pgfusepath{stroke}%
\end{pgfscope}%
\begin{pgfscope}%
\pgftext[x=2.601002in,y=0.571741in,,top]{\rmfamily\fontsize{8.000000}{9.600000}\selectfont 0.6}%
\end{pgfscope}%
\begin{pgfscope}%
\pgfsetrectcap%
\pgfsetroundjoin%
\pgfsetlinewidth{0.803000pt}%
\definecolor{currentstroke}{rgb}{0.000000,0.000000,0.000000}%
\pgfsetstrokecolor{currentstroke}%
\pgfsetdash{}{0pt}%
\pgfpathmoveto{\pgfqpoint{3.153323in}{0.636141in}}%
\pgfpathlineto{\pgfqpoint{3.117440in}{0.616346in}}%
\pgfusepath{stroke}%
\end{pgfscope}%
\begin{pgfscope}%
\pgftext[x=3.047904in,y=0.488164in,,top]{\rmfamily\fontsize{8.000000}{9.600000}\selectfont 0.8}%
\end{pgfscope}%
\begin{pgfscope}%
\pgfsetrectcap%
\pgfsetroundjoin%
\pgfsetlinewidth{0.803000pt}%
\definecolor{currentstroke}{rgb}{0.000000,0.000000,0.000000}%
\pgfsetstrokecolor{currentstroke}%
\pgfsetdash{}{0pt}%
\pgfpathmoveto{\pgfqpoint{3.606097in}{0.552596in}}%
\pgfpathlineto{\pgfqpoint{3.570701in}{0.532532in}}%
\pgfusepath{stroke}%
\end{pgfscope}%
\begin{pgfscope}%
\pgftext[x=3.500842in,y=0.403458in,,top]{\rmfamily\fontsize{8.000000}{9.600000}\selectfont 1.0}%
\end{pgfscope}%
\begin{pgfscope}%
\pgfsetrectcap%
\pgfsetroundjoin%
\pgfsetlinewidth{0.501875pt}%
\definecolor{currentstroke}{rgb}{0.000000,0.000000,0.000000}%
\pgfsetstrokecolor{currentstroke}%
\pgfsetdash{}{0pt}%
\pgfpathmoveto{\pgfqpoint{5.135278in}{1.301993in}}%
\pgfpathlineto{\pgfqpoint{3.760968in}{0.515142in}}%
\pgfusepath{stroke}%
\end{pgfscope}%
\begin{pgfscope}%
\pgftext[x=4.748349in,y=0.779064in,left,base,rotate=29.792981]{\rmfamily\fontsize{10.000000}{12.000000}\selectfont }%
\end{pgfscope}%
\begin{pgfscope}%
\pgftext[x=4.553640in,y=0.483787in,left,base,rotate=29.792981]{\rmfamily\fontsize{10.000000}{12.000000}\selectfont fraction of}%
\end{pgfscope}%
\begin{pgfscope}%
\pgftext[x=4.542818in,y=0.293793in,left,base,rotate=29.792981]{\rmfamily\fontsize{10.000000}{12.000000}\selectfont SemEval data}%
\end{pgfscope}%
\begin{pgfscope}%
\pgfsetbuttcap%
\pgfsetroundjoin%
\pgfsetlinewidth{0.803000pt}%
\definecolor{currentstroke}{rgb}{0.690196,0.690196,0.690196}%
\pgfsetstrokecolor{currentstroke}%
\pgfsetdash{}{0pt}%
\pgfpathmoveto{\pgfqpoint{1.249232in}{2.496130in}}%
\pgfpathlineto{\pgfqpoint{1.333347in}{1.031151in}}%
\pgfpathlineto{\pgfqpoint{3.853895in}{0.568347in}}%
\pgfusepath{stroke}%
\end{pgfscope}%
\begin{pgfscope}%
\pgfsetbuttcap%
\pgfsetroundjoin%
\pgfsetlinewidth{0.803000pt}%
\definecolor{currentstroke}{rgb}{0.690196,0.690196,0.690196}%
\pgfsetstrokecolor{currentstroke}%
\pgfsetdash{}{0pt}%
\pgfpathmoveto{\pgfqpoint{1.531798in}{2.618267in}}%
\pgfpathlineto{\pgfqpoint{1.602996in}{1.164374in}}%
\pgfpathlineto{\pgfqpoint{4.104771in}{0.711984in}}%
\pgfusepath{stroke}%
\end{pgfscope}%
\begin{pgfscope}%
\pgfsetbuttcap%
\pgfsetroundjoin%
\pgfsetlinewidth{0.803000pt}%
\definecolor{currentstroke}{rgb}{0.690196,0.690196,0.690196}%
\pgfsetstrokecolor{currentstroke}%
\pgfsetdash{}{0pt}%
\pgfpathmoveto{\pgfqpoint{1.807886in}{2.737604in}}%
\pgfpathlineto{\pgfqpoint{1.866722in}{1.294672in}}%
\pgfpathlineto{\pgfqpoint{4.349930in}{0.852348in}}%
\pgfusepath{stroke}%
\end{pgfscope}%
\begin{pgfscope}%
\pgfsetbuttcap%
\pgfsetroundjoin%
\pgfsetlinewidth{0.803000pt}%
\definecolor{currentstroke}{rgb}{0.690196,0.690196,0.690196}%
\pgfsetstrokecolor{currentstroke}%
\pgfsetdash{}{0pt}%
\pgfpathmoveto{\pgfqpoint{2.077717in}{2.854237in}}%
\pgfpathlineto{\pgfqpoint{2.124720in}{1.422139in}}%
\pgfpathlineto{\pgfqpoint{4.589565in}{0.989549in}}%
\pgfusepath{stroke}%
\end{pgfscope}%
\begin{pgfscope}%
\pgfsetbuttcap%
\pgfsetroundjoin%
\pgfsetlinewidth{0.803000pt}%
\definecolor{currentstroke}{rgb}{0.690196,0.690196,0.690196}%
\pgfsetstrokecolor{currentstroke}%
\pgfsetdash{}{0pt}%
\pgfpathmoveto{\pgfqpoint{2.341501in}{2.968255in}}%
\pgfpathlineto{\pgfqpoint{2.377173in}{1.546866in}}%
\pgfpathlineto{\pgfqpoint{4.823859in}{1.123693in}}%
\pgfusepath{stroke}%
\end{pgfscope}%
\begin{pgfscope}%
\pgfsetbuttcap%
\pgfsetroundjoin%
\pgfsetlinewidth{0.803000pt}%
\definecolor{currentstroke}{rgb}{0.690196,0.690196,0.690196}%
\pgfsetstrokecolor{currentstroke}%
\pgfsetdash{}{0pt}%
\pgfpathmoveto{\pgfqpoint{2.599439in}{3.079747in}}%
\pgfpathlineto{\pgfqpoint{2.624258in}{1.668942in}}%
\pgfpathlineto{\pgfqpoint{5.052991in}{1.254881in}}%
\pgfusepath{stroke}%
\end{pgfscope}%
\begin{pgfscope}%
\pgfsetrectcap%
\pgfsetroundjoin%
\pgfsetlinewidth{0.803000pt}%
\definecolor{currentstroke}{rgb}{0.000000,0.000000,0.000000}%
\pgfsetstrokecolor{currentstroke}%
\pgfsetdash{}{0pt}%
\pgfpathmoveto{\pgfqpoint{3.832959in}{0.572191in}}%
\pgfpathlineto{\pgfqpoint{3.895805in}{0.560652in}}%
\pgfusepath{stroke}%
\end{pgfscope}%
\begin{pgfscope}%
\pgftext[x=4.015897in,y=0.448445in,,top]{\rmfamily\fontsize{8.000000}{9.600000}\selectfont 0.0}%
\end{pgfscope}%
\begin{pgfscope}%
\pgfsetrectcap%
\pgfsetroundjoin%
\pgfsetlinewidth{0.803000pt}%
\definecolor{currentstroke}{rgb}{0.000000,0.000000,0.000000}%
\pgfsetstrokecolor{currentstroke}%
\pgfsetdash{}{0pt}%
\pgfpathmoveto{\pgfqpoint{4.084000in}{0.715740in}}%
\pgfpathlineto{\pgfqpoint{4.146351in}{0.704465in}}%
\pgfusepath{stroke}%
\end{pgfscope}%
\begin{pgfscope}%
\pgftext[x=4.264814in,y=0.593411in,,top]{\rmfamily\fontsize{8.000000}{9.600000}\selectfont 0.2}%
\end{pgfscope}%
\begin{pgfscope}%
\pgfsetrectcap%
\pgfsetroundjoin%
\pgfsetlinewidth{0.803000pt}%
\definecolor{currentstroke}{rgb}{0.000000,0.000000,0.000000}%
\pgfsetstrokecolor{currentstroke}%
\pgfsetdash{}{0pt}%
\pgfpathmoveto{\pgfqpoint{4.329322in}{0.856019in}}%
\pgfpathlineto{\pgfqpoint{4.391184in}{0.845000in}}%
\pgfusepath{stroke}%
\end{pgfscope}%
\begin{pgfscope}%
\pgftext[x=4.508060in,y=0.735074in,,top]{\rmfamily\fontsize{8.000000}{9.600000}\selectfont 0.4}%
\end{pgfscope}%
\begin{pgfscope}%
\pgfsetrectcap%
\pgfsetroundjoin%
\pgfsetlinewidth{0.803000pt}%
\definecolor{currentstroke}{rgb}{0.000000,0.000000,0.000000}%
\pgfsetstrokecolor{currentstroke}%
\pgfsetdash{}{0pt}%
\pgfpathmoveto{\pgfqpoint{4.569117in}{0.993138in}}%
\pgfpathlineto{\pgfqpoint{4.630496in}{0.982365in}}%
\pgfusepath{stroke}%
\end{pgfscope}%
\begin{pgfscope}%
\pgftext[x=4.745827in,y=0.873546in,,top]{\rmfamily\fontsize{8.000000}{9.600000}\selectfont 0.6}%
\end{pgfscope}%
\begin{pgfscope}%
\pgfsetrectcap%
\pgfsetroundjoin%
\pgfsetlinewidth{0.803000pt}%
\definecolor{currentstroke}{rgb}{0.000000,0.000000,0.000000}%
\pgfsetstrokecolor{currentstroke}%
\pgfsetdash{}{0pt}%
\pgfpathmoveto{\pgfqpoint{4.803571in}{1.127202in}}%
\pgfpathlineto{\pgfqpoint{4.864473in}{1.116668in}}%
\pgfusepath{stroke}%
\end{pgfscope}%
\begin{pgfscope}%
\pgftext[x=4.978298in,y=1.008934in,,top]{\rmfamily\fontsize{8.000000}{9.600000}\selectfont 0.8}%
\end{pgfscope}%
\begin{pgfscope}%
\pgfsetrectcap%
\pgfsetroundjoin%
\pgfsetlinewidth{0.803000pt}%
\definecolor{currentstroke}{rgb}{0.000000,0.000000,0.000000}%
\pgfsetstrokecolor{currentstroke}%
\pgfsetdash{}{0pt}%
\pgfpathmoveto{\pgfqpoint{5.032859in}{1.258313in}}%
\pgfpathlineto{\pgfqpoint{5.093291in}{1.248010in}}%
\pgfusepath{stroke}%
\end{pgfscope}%
\begin{pgfscope}%
\pgftext[x=5.205648in,y=1.141340in,,top]{\rmfamily\fontsize{8.000000}{9.600000}\selectfont 1.0}%
\end{pgfscope}%
\begin{pgfscope}%
\pgfsetrectcap%
\pgfsetroundjoin%
\pgfsetlinewidth{0.501875pt}%
\definecolor{currentstroke}{rgb}{0.000000,0.000000,0.000000}%
\pgfsetstrokecolor{currentstroke}%
\pgfsetdash{}{0pt}%
\pgfpathmoveto{\pgfqpoint{5.135278in}{1.301993in}}%
\pgfpathlineto{\pgfqpoint{5.219480in}{2.744306in}}%
\pgfusepath{stroke}%
\end{pgfscope}%
\begin{pgfscope}%
\pgftext[x=5.653991in,y=2.051089in,left,base,rotate=86.658871]{\rmfamily\fontsize{10.000000}{12.000000}\selectfont }%
\end{pgfscope}%
\begin{pgfscope}%
\pgftext[x=5.798047in,y=1.781819in,left,base,rotate=86.658871]{\rmfamily\fontsize{10.000000}{12.000000}\selectfont mean F1}%
\end{pgfscope}%
\begin{pgfscope}%
\pgfsetbuttcap%
\pgfsetroundjoin%
\pgfsetlinewidth{0.803000pt}%
\definecolor{currentstroke}{rgb}{0.690196,0.690196,0.690196}%
\pgfsetstrokecolor{currentstroke}%
\pgfsetdash{}{0pt}%
\pgfpathmoveto{\pgfqpoint{5.143397in}{1.441079in}}%
\pgfpathlineto{\pgfqpoint{2.711014in}{1.848672in}}%
\pgfpathlineto{\pgfqpoint{1.224955in}{1.123344in}}%
\pgfusepath{stroke}%
\end{pgfscope}%
\begin{pgfscope}%
\pgfsetbuttcap%
\pgfsetroundjoin%
\pgfsetlinewidth{0.803000pt}%
\definecolor{currentstroke}{rgb}{0.690196,0.690196,0.690196}%
\pgfsetstrokecolor{currentstroke}%
\pgfsetdash{}{0pt}%
\pgfpathmoveto{\pgfqpoint{5.155442in}{1.647401in}}%
\pgfpathlineto{\pgfqpoint{2.708010in}{2.050144in}}%
\pgfpathlineto{\pgfqpoint{1.212238in}{1.333324in}}%
\pgfusepath{stroke}%
\end{pgfscope}%
\begin{pgfscope}%
\pgfsetbuttcap%
\pgfsetroundjoin%
\pgfsetlinewidth{0.803000pt}%
\definecolor{currentstroke}{rgb}{0.690196,0.690196,0.690196}%
\pgfsetstrokecolor{currentstroke}%
\pgfsetdash{}{0pt}%
\pgfpathmoveto{\pgfqpoint{5.167639in}{1.856311in}}%
\pgfpathlineto{\pgfqpoint{2.704970in}{2.254053in}}%
\pgfpathlineto{\pgfqpoint{1.199357in}{1.546011in}}%
\pgfusepath{stroke}%
\end{pgfscope}%
\begin{pgfscope}%
\pgfsetbuttcap%
\pgfsetroundjoin%
\pgfsetlinewidth{0.803000pt}%
\definecolor{currentstroke}{rgb}{0.690196,0.690196,0.690196}%
\pgfsetstrokecolor{currentstroke}%
\pgfsetdash{}{0pt}%
\pgfpathmoveto{\pgfqpoint{5.179989in}{2.067857in}}%
\pgfpathlineto{\pgfqpoint{2.701893in}{2.460443in}}%
\pgfpathlineto{\pgfqpoint{1.186310in}{1.761458in}}%
\pgfusepath{stroke}%
\end{pgfscope}%
\begin{pgfscope}%
\pgfsetbuttcap%
\pgfsetroundjoin%
\pgfsetlinewidth{0.803000pt}%
\definecolor{currentstroke}{rgb}{0.690196,0.690196,0.690196}%
\pgfsetstrokecolor{currentstroke}%
\pgfsetdash{}{0pt}%
\pgfpathmoveto{\pgfqpoint{5.192496in}{2.282090in}}%
\pgfpathlineto{\pgfqpoint{2.698778in}{2.669360in}}%
\pgfpathlineto{\pgfqpoint{1.173091in}{1.979718in}}%
\pgfusepath{stroke}%
\end{pgfscope}%
\begin{pgfscope}%
\pgfsetbuttcap%
\pgfsetroundjoin%
\pgfsetlinewidth{0.803000pt}%
\definecolor{currentstroke}{rgb}{0.690196,0.690196,0.690196}%
\pgfsetstrokecolor{currentstroke}%
\pgfsetdash{}{0pt}%
\pgfpathmoveto{\pgfqpoint{5.205162in}{2.499062in}}%
\pgfpathlineto{\pgfqpoint{2.695625in}{2.880851in}}%
\pgfpathlineto{\pgfqpoint{1.159699in}{2.200846in}}%
\pgfusepath{stroke}%
\end{pgfscope}%
\begin{pgfscope}%
\pgfsetrectcap%
\pgfsetroundjoin%
\pgfsetlinewidth{0.803000pt}%
\definecolor{currentstroke}{rgb}{0.000000,0.000000,0.000000}%
\pgfsetstrokecolor{currentstroke}%
\pgfsetdash{}{0pt}%
\pgfpathmoveto{\pgfqpoint{5.123235in}{1.444458in}}%
\pgfpathlineto{\pgfqpoint{5.183758in}{1.434316in}}%
\pgfusepath{stroke}%
\end{pgfscope}%
\begin{pgfscope}%
\pgftext[x=5.381724in,y=1.460404in,,top]{\rmfamily\fontsize{8.000000}{9.600000}\selectfont 0.1}%
\end{pgfscope}%
\begin{pgfscope}%
\pgfsetrectcap%
\pgfsetroundjoin%
\pgfsetlinewidth{0.803000pt}%
\definecolor{currentstroke}{rgb}{0.000000,0.000000,0.000000}%
\pgfsetstrokecolor{currentstroke}%
\pgfsetdash{}{0pt}%
\pgfpathmoveto{\pgfqpoint{5.135151in}{1.650740in}}%
\pgfpathlineto{\pgfqpoint{5.196062in}{1.640717in}}%
\pgfusepath{stroke}%
\end{pgfscope}%
\begin{pgfscope}%
\pgftext[x=5.395231in,y=1.666500in,,top]{\rmfamily\fontsize{8.000000}{9.600000}\selectfont 0.2}%
\end{pgfscope}%
\begin{pgfscope}%
\pgfsetrectcap%
\pgfsetroundjoin%
\pgfsetlinewidth{0.803000pt}%
\definecolor{currentstroke}{rgb}{0.000000,0.000000,0.000000}%
\pgfsetstrokecolor{currentstroke}%
\pgfsetdash{}{0pt}%
\pgfpathmoveto{\pgfqpoint{5.147216in}{1.859609in}}%
\pgfpathlineto{\pgfqpoint{5.208520in}{1.849708in}}%
\pgfusepath{stroke}%
\end{pgfscope}%
\begin{pgfscope}%
\pgftext[x=5.408908in,y=1.875177in,,top]{\rmfamily\fontsize{8.000000}{9.600000}\selectfont 0.3}%
\end{pgfscope}%
\begin{pgfscope}%
\pgfsetrectcap%
\pgfsetroundjoin%
\pgfsetlinewidth{0.803000pt}%
\definecolor{currentstroke}{rgb}{0.000000,0.000000,0.000000}%
\pgfsetstrokecolor{currentstroke}%
\pgfsetdash{}{0pt}%
\pgfpathmoveto{\pgfqpoint{5.159434in}{2.071113in}}%
\pgfpathlineto{\pgfqpoint{5.221135in}{2.061338in}}%
\pgfusepath{stroke}%
\end{pgfscope}%
\begin{pgfscope}%
\pgftext[x=5.422757in,y=2.086482in,,top]{\rmfamily\fontsize{8.000000}{9.600000}\selectfont 0.4}%
\end{pgfscope}%
\begin{pgfscope}%
\pgfsetrectcap%
\pgfsetroundjoin%
\pgfsetlinewidth{0.803000pt}%
\definecolor{currentstroke}{rgb}{0.000000,0.000000,0.000000}%
\pgfsetstrokecolor{currentstroke}%
\pgfsetdash{}{0pt}%
\pgfpathmoveto{\pgfqpoint{5.171806in}{2.285303in}}%
\pgfpathlineto{\pgfqpoint{5.233911in}{2.275658in}}%
\pgfusepath{stroke}%
\end{pgfscope}%
\begin{pgfscope}%
\pgftext[x=5.436782in,y=2.300467in,,top]{\rmfamily\fontsize{8.000000}{9.600000}\selectfont 0.5}%
\end{pgfscope}%
\begin{pgfscope}%
\pgfsetrectcap%
\pgfsetroundjoin%
\pgfsetlinewidth{0.803000pt}%
\definecolor{currentstroke}{rgb}{0.000000,0.000000,0.000000}%
\pgfsetstrokecolor{currentstroke}%
\pgfsetdash{}{0pt}%
\pgfpathmoveto{\pgfqpoint{5.184337in}{2.502230in}}%
\pgfpathlineto{\pgfqpoint{5.246851in}{2.492720in}}%
\pgfusepath{stroke}%
\end{pgfscope}%
\begin{pgfscope}%
\pgftext[x=5.450985in,y=2.517183in,,top]{\rmfamily\fontsize{8.000000}{9.600000}\selectfont 0.6}%
\end{pgfscope}%
\begin{pgfscope}%
\pgfpathrectangle{\pgfqpoint{0.017500in}{0.017500in}}{\pgfqpoint{6.218282in}{3.580372in}} %
\pgfusepath{clip}%
\pgfsetbuttcap%
\pgfsetroundjoin%
\definecolor{currentfill}{rgb}{0.906154,0.842091,0.806151}%
\pgfsetfillcolor{currentfill}%
\pgfsetlinewidth{0.000000pt}%
\definecolor{currentstroke}{rgb}{0.000000,0.000000,0.000000}%
\pgfsetstrokecolor{currentstroke}%
\pgfsetdash{}{0pt}%
\pgfpathmoveto{\pgfqpoint{2.755563in}{3.027050in}}%
\pgfpathlineto{\pgfqpoint{1.486050in}{1.042638in}}%
\pgfpathlineto{\pgfqpoint{2.755563in}{3.027050in}}%
\pgfpathclose%
\pgfusepath{fill}%
\end{pgfscope}%
\begin{pgfscope}%
\pgfpathrectangle{\pgfqpoint{0.017500in}{0.017500in}}{\pgfqpoint{6.218282in}{3.580372in}} %
\pgfusepath{clip}%
\pgfsetbuttcap%
\pgfsetroundjoin%
\definecolor{currentfill}{rgb}{0.906154,0.842091,0.806151}%
\pgfsetfillcolor{currentfill}%
\pgfsetlinewidth{0.000000pt}%
\definecolor{currentstroke}{rgb}{0.000000,0.000000,0.000000}%
\pgfsetstrokecolor{currentstroke}%
\pgfsetdash{}{0pt}%
\pgfpathmoveto{\pgfqpoint{2.755563in}{3.027050in}}%
\pgfpathlineto{\pgfqpoint{1.486050in}{1.042638in}}%
\pgfpathlineto{\pgfqpoint{2.755563in}{3.027050in}}%
\pgfpathclose%
\pgfusepath{fill}%
\end{pgfscope}%
\begin{pgfscope}%
\pgfpathrectangle{\pgfqpoint{0.017500in}{0.017500in}}{\pgfqpoint{6.218282in}{3.580372in}} %
\pgfusepath{clip}%
\pgfsetbuttcap%
\pgfsetroundjoin%
\definecolor{currentfill}{rgb}{0.906154,0.842091,0.806151}%
\pgfsetfillcolor{currentfill}%
\pgfsetlinewidth{0.000000pt}%
\definecolor{currentstroke}{rgb}{0.000000,0.000000,0.000000}%
\pgfsetstrokecolor{currentstroke}%
\pgfsetdash{}{0pt}%
\pgfpathmoveto{\pgfqpoint{2.755563in}{3.027050in}}%
\pgfpathlineto{\pgfqpoint{1.486050in}{1.042638in}}%
\pgfpathlineto{\pgfqpoint{2.755563in}{3.027050in}}%
\pgfpathclose%
\pgfusepath{fill}%
\end{pgfscope}%
\begin{pgfscope}%
\pgfpathrectangle{\pgfqpoint{0.017500in}{0.017500in}}{\pgfqpoint{6.218282in}{3.580372in}} %
\pgfusepath{clip}%
\pgfsetbuttcap%
\pgfsetroundjoin%
\definecolor{currentfill}{rgb}{0.906154,0.842091,0.806151}%
\pgfsetfillcolor{currentfill}%
\pgfsetlinewidth{0.000000pt}%
\definecolor{currentstroke}{rgb}{0.000000,0.000000,0.000000}%
\pgfsetstrokecolor{currentstroke}%
\pgfsetdash{}{0pt}%
\pgfpathmoveto{\pgfqpoint{2.755563in}{3.027050in}}%
\pgfpathlineto{\pgfqpoint{1.486050in}{1.042638in}}%
\pgfpathlineto{\pgfqpoint{2.755563in}{3.027050in}}%
\pgfpathclose%
\pgfusepath{fill}%
\end{pgfscope}%
\begin{pgfscope}%
\pgfpathrectangle{\pgfqpoint{0.017500in}{0.017500in}}{\pgfqpoint{6.218282in}{3.580372in}} %
\pgfusepath{clip}%
\pgfsetbuttcap%
\pgfsetroundjoin%
\definecolor{currentfill}{rgb}{0.906154,0.842091,0.806151}%
\pgfsetfillcolor{currentfill}%
\pgfsetlinewidth{0.000000pt}%
\definecolor{currentstroke}{rgb}{0.000000,0.000000,0.000000}%
\pgfsetstrokecolor{currentstroke}%
\pgfsetdash{}{0pt}%
\pgfpathmoveto{\pgfqpoint{2.755563in}{3.027050in}}%
\pgfpathlineto{\pgfqpoint{1.486050in}{1.042638in}}%
\pgfpathlineto{\pgfqpoint{2.755563in}{3.027050in}}%
\pgfpathclose%
\pgfusepath{fill}%
\end{pgfscope}%
\begin{pgfscope}%
\pgfpathrectangle{\pgfqpoint{0.017500in}{0.017500in}}{\pgfqpoint{6.218282in}{3.580372in}} %
\pgfusepath{clip}%
\pgfsetbuttcap%
\pgfsetroundjoin%
\definecolor{currentfill}{rgb}{0.906154,0.842091,0.806151}%
\pgfsetfillcolor{currentfill}%
\pgfsetlinewidth{0.000000pt}%
\definecolor{currentstroke}{rgb}{0.000000,0.000000,0.000000}%
\pgfsetstrokecolor{currentstroke}%
\pgfsetdash{}{0pt}%
\pgfpathmoveto{\pgfqpoint{2.755563in}{3.027050in}}%
\pgfpathlineto{\pgfqpoint{1.486050in}{1.042638in}}%
\pgfpathlineto{\pgfqpoint{2.755563in}{3.027050in}}%
\pgfpathclose%
\pgfusepath{fill}%
\end{pgfscope}%
\begin{pgfscope}%
\pgfpathrectangle{\pgfqpoint{0.017500in}{0.017500in}}{\pgfqpoint{6.218282in}{3.580372in}} %
\pgfusepath{clip}%
\pgfsetbuttcap%
\pgfsetroundjoin%
\definecolor{currentfill}{rgb}{0.906154,0.842091,0.806151}%
\pgfsetfillcolor{currentfill}%
\pgfsetlinewidth{0.000000pt}%
\definecolor{currentstroke}{rgb}{0.000000,0.000000,0.000000}%
\pgfsetstrokecolor{currentstroke}%
\pgfsetdash{}{0pt}%
\pgfpathmoveto{\pgfqpoint{2.755563in}{3.027050in}}%
\pgfpathlineto{\pgfqpoint{1.486050in}{1.042638in}}%
\pgfpathlineto{\pgfqpoint{2.755563in}{3.027050in}}%
\pgfpathclose%
\pgfusepath{fill}%
\end{pgfscope}%
\begin{pgfscope}%
\pgfpathrectangle{\pgfqpoint{0.017500in}{0.017500in}}{\pgfqpoint{6.218282in}{3.580372in}} %
\pgfusepath{clip}%
\pgfsetbuttcap%
\pgfsetroundjoin%
\definecolor{currentfill}{rgb}{0.906154,0.842091,0.806151}%
\pgfsetfillcolor{currentfill}%
\pgfsetlinewidth{0.000000pt}%
\definecolor{currentstroke}{rgb}{0.000000,0.000000,0.000000}%
\pgfsetstrokecolor{currentstroke}%
\pgfsetdash{}{0pt}%
\pgfpathmoveto{\pgfqpoint{2.755563in}{3.027050in}}%
\pgfpathlineto{\pgfqpoint{1.486050in}{1.042638in}}%
\pgfpathlineto{\pgfqpoint{2.755563in}{3.027050in}}%
\pgfpathclose%
\pgfusepath{fill}%
\end{pgfscope}%
\begin{pgfscope}%
\pgfpathrectangle{\pgfqpoint{0.017500in}{0.017500in}}{\pgfqpoint{6.218282in}{3.580372in}} %
\pgfusepath{clip}%
\pgfsetbuttcap%
\pgfsetroundjoin%
\definecolor{currentfill}{rgb}{0.906154,0.842091,0.806151}%
\pgfsetfillcolor{currentfill}%
\pgfsetlinewidth{0.000000pt}%
\definecolor{currentstroke}{rgb}{0.000000,0.000000,0.000000}%
\pgfsetstrokecolor{currentstroke}%
\pgfsetdash{}{0pt}%
\pgfpathmoveto{\pgfqpoint{2.755563in}{3.027050in}}%
\pgfpathlineto{\pgfqpoint{1.486050in}{1.042638in}}%
\pgfpathlineto{\pgfqpoint{2.755563in}{3.027050in}}%
\pgfpathclose%
\pgfusepath{fill}%
\end{pgfscope}%
\begin{pgfscope}%
\pgfpathrectangle{\pgfqpoint{0.017500in}{0.017500in}}{\pgfqpoint{6.218282in}{3.580372in}} %
\pgfusepath{clip}%
\pgfsetbuttcap%
\pgfsetroundjoin%
\definecolor{currentfill}{rgb}{0.906154,0.842091,0.806151}%
\pgfsetfillcolor{currentfill}%
\pgfsetlinewidth{0.000000pt}%
\definecolor{currentstroke}{rgb}{0.000000,0.000000,0.000000}%
\pgfsetstrokecolor{currentstroke}%
\pgfsetdash{}{0pt}%
\pgfpathmoveto{\pgfqpoint{2.755563in}{3.027050in}}%
\pgfpathlineto{\pgfqpoint{1.486050in}{1.042638in}}%
\pgfpathlineto{\pgfqpoint{2.755563in}{3.027050in}}%
\pgfpathclose%
\pgfusepath{fill}%
\end{pgfscope}%
\begin{pgfscope}%
\pgfpathrectangle{\pgfqpoint{0.017500in}{0.017500in}}{\pgfqpoint{6.218282in}{3.580372in}} %
\pgfusepath{clip}%
\pgfsetbuttcap%
\pgfsetroundjoin%
\definecolor{currentfill}{rgb}{0.906154,0.842091,0.806151}%
\pgfsetfillcolor{currentfill}%
\pgfsetlinewidth{0.000000pt}%
\definecolor{currentstroke}{rgb}{0.000000,0.000000,0.000000}%
\pgfsetstrokecolor{currentstroke}%
\pgfsetdash{}{0pt}%
\pgfpathmoveto{\pgfqpoint{2.755563in}{3.027050in}}%
\pgfpathlineto{\pgfqpoint{1.486050in}{1.042638in}}%
\pgfpathlineto{\pgfqpoint{2.755563in}{3.027050in}}%
\pgfpathclose%
\pgfusepath{fill}%
\end{pgfscope}%
\begin{pgfscope}%
\pgfpathrectangle{\pgfqpoint{0.017500in}{0.017500in}}{\pgfqpoint{6.218282in}{3.580372in}} %
\pgfusepath{clip}%
\pgfsetbuttcap%
\pgfsetroundjoin%
\definecolor{currentfill}{rgb}{0.906154,0.842091,0.806151}%
\pgfsetfillcolor{currentfill}%
\pgfsetlinewidth{0.000000pt}%
\definecolor{currentstroke}{rgb}{0.000000,0.000000,0.000000}%
\pgfsetstrokecolor{currentstroke}%
\pgfsetdash{}{0pt}%
\pgfpathmoveto{\pgfqpoint{2.755563in}{3.027050in}}%
\pgfpathlineto{\pgfqpoint{1.486050in}{1.042638in}}%
\pgfpathlineto{\pgfqpoint{2.755563in}{3.027050in}}%
\pgfpathclose%
\pgfusepath{fill}%
\end{pgfscope}%
\begin{pgfscope}%
\pgfpathrectangle{\pgfqpoint{0.017500in}{0.017500in}}{\pgfqpoint{6.218282in}{3.580372in}} %
\pgfusepath{clip}%
\pgfsetbuttcap%
\pgfsetroundjoin%
\definecolor{currentfill}{rgb}{0.906154,0.842091,0.806151}%
\pgfsetfillcolor{currentfill}%
\pgfsetlinewidth{0.000000pt}%
\definecolor{currentstroke}{rgb}{0.000000,0.000000,0.000000}%
\pgfsetstrokecolor{currentstroke}%
\pgfsetdash{}{0pt}%
\pgfpathmoveto{\pgfqpoint{2.755563in}{3.027050in}}%
\pgfpathlineto{\pgfqpoint{1.486050in}{1.042638in}}%
\pgfpathlineto{\pgfqpoint{2.755563in}{3.027050in}}%
\pgfpathclose%
\pgfusepath{fill}%
\end{pgfscope}%
\begin{pgfscope}%
\pgfpathrectangle{\pgfqpoint{0.017500in}{0.017500in}}{\pgfqpoint{6.218282in}{3.580372in}} %
\pgfusepath{clip}%
\pgfsetbuttcap%
\pgfsetroundjoin%
\definecolor{currentfill}{rgb}{0.906154,0.842091,0.806151}%
\pgfsetfillcolor{currentfill}%
\pgfsetlinewidth{0.000000pt}%
\definecolor{currentstroke}{rgb}{0.000000,0.000000,0.000000}%
\pgfsetstrokecolor{currentstroke}%
\pgfsetdash{}{0pt}%
\pgfpathmoveto{\pgfqpoint{2.755563in}{3.027050in}}%
\pgfpathlineto{\pgfqpoint{1.486050in}{1.042638in}}%
\pgfpathlineto{\pgfqpoint{2.755563in}{3.027050in}}%
\pgfpathclose%
\pgfusepath{fill}%
\end{pgfscope}%
\begin{pgfscope}%
\pgfpathrectangle{\pgfqpoint{0.017500in}{0.017500in}}{\pgfqpoint{6.218282in}{3.580372in}} %
\pgfusepath{clip}%
\pgfsetbuttcap%
\pgfsetroundjoin%
\definecolor{currentfill}{rgb}{0.906154,0.842091,0.806151}%
\pgfsetfillcolor{currentfill}%
\pgfsetlinewidth{0.000000pt}%
\definecolor{currentstroke}{rgb}{0.000000,0.000000,0.000000}%
\pgfsetstrokecolor{currentstroke}%
\pgfsetdash{}{0pt}%
\pgfpathmoveto{\pgfqpoint{2.755563in}{3.027050in}}%
\pgfpathlineto{\pgfqpoint{1.486050in}{1.042638in}}%
\pgfpathlineto{\pgfqpoint{2.755563in}{3.027050in}}%
\pgfpathclose%
\pgfusepath{fill}%
\end{pgfscope}%
\begin{pgfscope}%
\pgfpathrectangle{\pgfqpoint{0.017500in}{0.017500in}}{\pgfqpoint{6.218282in}{3.580372in}} %
\pgfusepath{clip}%
\pgfsetbuttcap%
\pgfsetroundjoin%
\definecolor{currentfill}{rgb}{0.906154,0.842091,0.806151}%
\pgfsetfillcolor{currentfill}%
\pgfsetlinewidth{0.000000pt}%
\definecolor{currentstroke}{rgb}{0.000000,0.000000,0.000000}%
\pgfsetstrokecolor{currentstroke}%
\pgfsetdash{}{0pt}%
\pgfpathmoveto{\pgfqpoint{2.755563in}{3.027050in}}%
\pgfpathlineto{\pgfqpoint{1.486050in}{1.042638in}}%
\pgfpathlineto{\pgfqpoint{2.755563in}{3.027050in}}%
\pgfpathclose%
\pgfusepath{fill}%
\end{pgfscope}%
\begin{pgfscope}%
\pgfpathrectangle{\pgfqpoint{0.017500in}{0.017500in}}{\pgfqpoint{6.218282in}{3.580372in}} %
\pgfusepath{clip}%
\pgfsetbuttcap%
\pgfsetroundjoin%
\definecolor{currentfill}{rgb}{0.906154,0.842091,0.806151}%
\pgfsetfillcolor{currentfill}%
\pgfsetlinewidth{0.000000pt}%
\definecolor{currentstroke}{rgb}{0.000000,0.000000,0.000000}%
\pgfsetstrokecolor{currentstroke}%
\pgfsetdash{}{0pt}%
\pgfpathmoveto{\pgfqpoint{2.755563in}{3.027050in}}%
\pgfpathlineto{\pgfqpoint{1.486050in}{1.042638in}}%
\pgfpathlineto{\pgfqpoint{2.755563in}{3.027050in}}%
\pgfpathclose%
\pgfusepath{fill}%
\end{pgfscope}%
\begin{pgfscope}%
\pgfpathrectangle{\pgfqpoint{0.017500in}{0.017500in}}{\pgfqpoint{6.218282in}{3.580372in}} %
\pgfusepath{clip}%
\pgfsetbuttcap%
\pgfsetroundjoin%
\definecolor{currentfill}{rgb}{0.906154,0.842091,0.806151}%
\pgfsetfillcolor{currentfill}%
\pgfsetlinewidth{0.000000pt}%
\definecolor{currentstroke}{rgb}{0.000000,0.000000,0.000000}%
\pgfsetstrokecolor{currentstroke}%
\pgfsetdash{}{0pt}%
\pgfpathmoveto{\pgfqpoint{2.755563in}{3.027050in}}%
\pgfpathlineto{\pgfqpoint{1.486050in}{1.042638in}}%
\pgfpathlineto{\pgfqpoint{2.755563in}{3.027050in}}%
\pgfpathclose%
\pgfusepath{fill}%
\end{pgfscope}%
\begin{pgfscope}%
\pgfpathrectangle{\pgfqpoint{0.017500in}{0.017500in}}{\pgfqpoint{6.218282in}{3.580372in}} %
\pgfusepath{clip}%
\pgfsetbuttcap%
\pgfsetroundjoin%
\definecolor{currentfill}{rgb}{0.906154,0.842091,0.806151}%
\pgfsetfillcolor{currentfill}%
\pgfsetlinewidth{0.000000pt}%
\definecolor{currentstroke}{rgb}{0.000000,0.000000,0.000000}%
\pgfsetstrokecolor{currentstroke}%
\pgfsetdash{}{0pt}%
\pgfpathmoveto{\pgfqpoint{2.755563in}{3.027050in}}%
\pgfpathlineto{\pgfqpoint{1.486050in}{1.042638in}}%
\pgfpathlineto{\pgfqpoint{2.755563in}{3.027050in}}%
\pgfpathclose%
\pgfusepath{fill}%
\end{pgfscope}%
\begin{pgfscope}%
\pgfpathrectangle{\pgfqpoint{0.017500in}{0.017500in}}{\pgfqpoint{6.218282in}{3.580372in}} %
\pgfusepath{clip}%
\pgfsetbuttcap%
\pgfsetroundjoin%
\definecolor{currentfill}{rgb}{0.906154,0.842091,0.806151}%
\pgfsetfillcolor{currentfill}%
\pgfsetlinewidth{0.000000pt}%
\definecolor{currentstroke}{rgb}{0.000000,0.000000,0.000000}%
\pgfsetstrokecolor{currentstroke}%
\pgfsetdash{}{0pt}%
\pgfpathmoveto{\pgfqpoint{2.755563in}{3.027050in}}%
\pgfpathlineto{\pgfqpoint{1.486050in}{1.042638in}}%
\pgfpathlineto{\pgfqpoint{2.755563in}{3.027050in}}%
\pgfpathclose%
\pgfusepath{fill}%
\end{pgfscope}%
\begin{pgfscope}%
\pgfpathrectangle{\pgfqpoint{0.017500in}{0.017500in}}{\pgfqpoint{6.218282in}{3.580372in}} %
\pgfusepath{clip}%
\pgfsetbuttcap%
\pgfsetroundjoin%
\definecolor{currentfill}{rgb}{0.906154,0.842091,0.806151}%
\pgfsetfillcolor{currentfill}%
\pgfsetlinewidth{0.000000pt}%
\definecolor{currentstroke}{rgb}{0.000000,0.000000,0.000000}%
\pgfsetstrokecolor{currentstroke}%
\pgfsetdash{}{0pt}%
\pgfpathmoveto{\pgfqpoint{2.755563in}{3.027050in}}%
\pgfpathlineto{\pgfqpoint{1.486050in}{1.042638in}}%
\pgfpathlineto{\pgfqpoint{2.755563in}{3.027050in}}%
\pgfpathclose%
\pgfusepath{fill}%
\end{pgfscope}%
\begin{pgfscope}%
\pgfpathrectangle{\pgfqpoint{0.017500in}{0.017500in}}{\pgfqpoint{6.218282in}{3.580372in}} %
\pgfusepath{clip}%
\pgfsetbuttcap%
\pgfsetroundjoin%
\definecolor{currentfill}{rgb}{0.906154,0.842091,0.806151}%
\pgfsetfillcolor{currentfill}%
\pgfsetlinewidth{0.000000pt}%
\definecolor{currentstroke}{rgb}{0.000000,0.000000,0.000000}%
\pgfsetstrokecolor{currentstroke}%
\pgfsetdash{}{0pt}%
\pgfpathmoveto{\pgfqpoint{2.755563in}{3.027050in}}%
\pgfpathlineto{\pgfqpoint{1.486050in}{1.042638in}}%
\pgfpathlineto{\pgfqpoint{2.755563in}{3.027050in}}%
\pgfpathclose%
\pgfusepath{fill}%
\end{pgfscope}%
\begin{pgfscope}%
\pgfpathrectangle{\pgfqpoint{0.017500in}{0.017500in}}{\pgfqpoint{6.218282in}{3.580372in}} %
\pgfusepath{clip}%
\pgfsetbuttcap%
\pgfsetroundjoin%
\definecolor{currentfill}{rgb}{0.906154,0.842091,0.806151}%
\pgfsetfillcolor{currentfill}%
\pgfsetlinewidth{0.000000pt}%
\definecolor{currentstroke}{rgb}{0.000000,0.000000,0.000000}%
\pgfsetstrokecolor{currentstroke}%
\pgfsetdash{}{0pt}%
\pgfpathmoveto{\pgfqpoint{2.755563in}{3.027050in}}%
\pgfpathlineto{\pgfqpoint{1.486050in}{1.042638in}}%
\pgfpathlineto{\pgfqpoint{2.755563in}{3.027050in}}%
\pgfpathclose%
\pgfusepath{fill}%
\end{pgfscope}%
\begin{pgfscope}%
\pgfpathrectangle{\pgfqpoint{0.017500in}{0.017500in}}{\pgfqpoint{6.218282in}{3.580372in}} %
\pgfusepath{clip}%
\pgfsetbuttcap%
\pgfsetroundjoin%
\definecolor{currentfill}{rgb}{0.906154,0.842091,0.806151}%
\pgfsetfillcolor{currentfill}%
\pgfsetlinewidth{0.000000pt}%
\definecolor{currentstroke}{rgb}{0.000000,0.000000,0.000000}%
\pgfsetstrokecolor{currentstroke}%
\pgfsetdash{}{0pt}%
\pgfpathmoveto{\pgfqpoint{2.755563in}{3.027050in}}%
\pgfpathlineto{\pgfqpoint{1.486050in}{1.042638in}}%
\pgfpathlineto{\pgfqpoint{2.755563in}{3.027050in}}%
\pgfpathclose%
\pgfusepath{fill}%
\end{pgfscope}%
\begin{pgfscope}%
\pgfpathrectangle{\pgfqpoint{0.017500in}{0.017500in}}{\pgfqpoint{6.218282in}{3.580372in}} %
\pgfusepath{clip}%
\pgfsetbuttcap%
\pgfsetroundjoin%
\definecolor{currentfill}{rgb}{0.906154,0.842091,0.806151}%
\pgfsetfillcolor{currentfill}%
\pgfsetlinewidth{0.000000pt}%
\definecolor{currentstroke}{rgb}{0.000000,0.000000,0.000000}%
\pgfsetstrokecolor{currentstroke}%
\pgfsetdash{}{0pt}%
\pgfpathmoveto{\pgfqpoint{2.755563in}{3.027050in}}%
\pgfpathlineto{\pgfqpoint{1.486050in}{1.042638in}}%
\pgfpathlineto{\pgfqpoint{2.755563in}{3.027050in}}%
\pgfpathclose%
\pgfusepath{fill}%
\end{pgfscope}%
\begin{pgfscope}%
\pgfpathrectangle{\pgfqpoint{0.017500in}{0.017500in}}{\pgfqpoint{6.218282in}{3.580372in}} %
\pgfusepath{clip}%
\pgfsetbuttcap%
\pgfsetroundjoin%
\definecolor{currentfill}{rgb}{0.705673,0.015556,0.150233}%
\pgfsetfillcolor{currentfill}%
\pgfsetlinewidth{0.000000pt}%
\definecolor{currentstroke}{rgb}{0.000000,0.000000,0.000000}%
\pgfsetstrokecolor{currentstroke}%
\pgfsetdash{}{0pt}%
\pgfpathmoveto{\pgfqpoint{2.499217in}{2.903684in}}%
\pgfpathlineto{\pgfqpoint{2.755563in}{3.027050in}}%
\pgfpathlineto{\pgfqpoint{2.499217in}{2.903684in}}%
\pgfpathclose%
\pgfusepath{fill}%
\end{pgfscope}%
\begin{pgfscope}%
\pgfpathrectangle{\pgfqpoint{0.017500in}{0.017500in}}{\pgfqpoint{6.218282in}{3.580372in}} %
\pgfusepath{clip}%
\pgfsetbuttcap%
\pgfsetroundjoin%
\definecolor{currentfill}{rgb}{0.705673,0.015556,0.150233}%
\pgfsetfillcolor{currentfill}%
\pgfsetlinewidth{0.000000pt}%
\definecolor{currentstroke}{rgb}{0.000000,0.000000,0.000000}%
\pgfsetstrokecolor{currentstroke}%
\pgfsetdash{}{0pt}%
\pgfpathmoveto{\pgfqpoint{2.499217in}{2.903684in}}%
\pgfpathlineto{\pgfqpoint{2.755563in}{3.027050in}}%
\pgfpathlineto{\pgfqpoint{2.499217in}{2.903684in}}%
\pgfpathclose%
\pgfusepath{fill}%
\end{pgfscope}%
\begin{pgfscope}%
\pgfpathrectangle{\pgfqpoint{0.017500in}{0.017500in}}{\pgfqpoint{6.218282in}{3.580372in}} %
\pgfusepath{clip}%
\pgfsetbuttcap%
\pgfsetroundjoin%
\definecolor{currentfill}{rgb}{0.705673,0.015556,0.150233}%
\pgfsetfillcolor{currentfill}%
\pgfsetlinewidth{0.000000pt}%
\definecolor{currentstroke}{rgb}{0.000000,0.000000,0.000000}%
\pgfsetstrokecolor{currentstroke}%
\pgfsetdash{}{0pt}%
\pgfpathmoveto{\pgfqpoint{2.499217in}{2.903684in}}%
\pgfpathlineto{\pgfqpoint{2.755563in}{3.027050in}}%
\pgfpathlineto{\pgfqpoint{2.499217in}{2.903684in}}%
\pgfpathclose%
\pgfusepath{fill}%
\end{pgfscope}%
\begin{pgfscope}%
\pgfpathrectangle{\pgfqpoint{0.017500in}{0.017500in}}{\pgfqpoint{6.218282in}{3.580372in}} %
\pgfusepath{clip}%
\pgfsetbuttcap%
\pgfsetroundjoin%
\definecolor{currentfill}{rgb}{0.705673,0.015556,0.150233}%
\pgfsetfillcolor{currentfill}%
\pgfsetlinewidth{0.000000pt}%
\definecolor{currentstroke}{rgb}{0.000000,0.000000,0.000000}%
\pgfsetstrokecolor{currentstroke}%
\pgfsetdash{}{0pt}%
\pgfpathmoveto{\pgfqpoint{2.499217in}{2.903684in}}%
\pgfpathlineto{\pgfqpoint{2.755563in}{3.027050in}}%
\pgfpathlineto{\pgfqpoint{2.499217in}{2.903684in}}%
\pgfpathclose%
\pgfusepath{fill}%
\end{pgfscope}%
\begin{pgfscope}%
\pgfpathrectangle{\pgfqpoint{0.017500in}{0.017500in}}{\pgfqpoint{6.218282in}{3.580372in}} %
\pgfusepath{clip}%
\pgfsetbuttcap%
\pgfsetroundjoin%
\definecolor{currentfill}{rgb}{0.705673,0.015556,0.150233}%
\pgfsetfillcolor{currentfill}%
\pgfsetlinewidth{0.000000pt}%
\definecolor{currentstroke}{rgb}{0.000000,0.000000,0.000000}%
\pgfsetstrokecolor{currentstroke}%
\pgfsetdash{}{0pt}%
\pgfpathmoveto{\pgfqpoint{2.499217in}{2.903684in}}%
\pgfpathlineto{\pgfqpoint{2.755563in}{3.027050in}}%
\pgfpathlineto{\pgfqpoint{2.499217in}{2.903684in}}%
\pgfpathclose%
\pgfusepath{fill}%
\end{pgfscope}%
\begin{pgfscope}%
\pgfpathrectangle{\pgfqpoint{0.017500in}{0.017500in}}{\pgfqpoint{6.218282in}{3.580372in}} %
\pgfusepath{clip}%
\pgfsetbuttcap%
\pgfsetroundjoin%
\definecolor{currentfill}{rgb}{0.705673,0.015556,0.150233}%
\pgfsetfillcolor{currentfill}%
\pgfsetlinewidth{0.000000pt}%
\definecolor{currentstroke}{rgb}{0.000000,0.000000,0.000000}%
\pgfsetstrokecolor{currentstroke}%
\pgfsetdash{}{0pt}%
\pgfpathmoveto{\pgfqpoint{2.499217in}{2.903684in}}%
\pgfpathlineto{\pgfqpoint{2.755563in}{3.027050in}}%
\pgfpathlineto{\pgfqpoint{2.499217in}{2.903684in}}%
\pgfpathclose%
\pgfusepath{fill}%
\end{pgfscope}%
\begin{pgfscope}%
\pgfpathrectangle{\pgfqpoint{0.017500in}{0.017500in}}{\pgfqpoint{6.218282in}{3.580372in}} %
\pgfusepath{clip}%
\pgfsetbuttcap%
\pgfsetroundjoin%
\definecolor{currentfill}{rgb}{0.705673,0.015556,0.150233}%
\pgfsetfillcolor{currentfill}%
\pgfsetlinewidth{0.000000pt}%
\definecolor{currentstroke}{rgb}{0.000000,0.000000,0.000000}%
\pgfsetstrokecolor{currentstroke}%
\pgfsetdash{}{0pt}%
\pgfpathmoveto{\pgfqpoint{2.499217in}{2.903684in}}%
\pgfpathlineto{\pgfqpoint{2.755563in}{3.027050in}}%
\pgfpathlineto{\pgfqpoint{2.499217in}{2.903684in}}%
\pgfpathclose%
\pgfusepath{fill}%
\end{pgfscope}%
\begin{pgfscope}%
\pgfpathrectangle{\pgfqpoint{0.017500in}{0.017500in}}{\pgfqpoint{6.218282in}{3.580372in}} %
\pgfusepath{clip}%
\pgfsetbuttcap%
\pgfsetroundjoin%
\definecolor{currentfill}{rgb}{0.705673,0.015556,0.150233}%
\pgfsetfillcolor{currentfill}%
\pgfsetlinewidth{0.000000pt}%
\definecolor{currentstroke}{rgb}{0.000000,0.000000,0.000000}%
\pgfsetstrokecolor{currentstroke}%
\pgfsetdash{}{0pt}%
\pgfpathmoveto{\pgfqpoint{2.499217in}{2.903684in}}%
\pgfpathlineto{\pgfqpoint{2.755563in}{3.027050in}}%
\pgfpathlineto{\pgfqpoint{2.499217in}{2.903684in}}%
\pgfpathclose%
\pgfusepath{fill}%
\end{pgfscope}%
\begin{pgfscope}%
\pgfpathrectangle{\pgfqpoint{0.017500in}{0.017500in}}{\pgfqpoint{6.218282in}{3.580372in}} %
\pgfusepath{clip}%
\pgfsetbuttcap%
\pgfsetroundjoin%
\definecolor{currentfill}{rgb}{0.705673,0.015556,0.150233}%
\pgfsetfillcolor{currentfill}%
\pgfsetlinewidth{0.000000pt}%
\definecolor{currentstroke}{rgb}{0.000000,0.000000,0.000000}%
\pgfsetstrokecolor{currentstroke}%
\pgfsetdash{}{0pt}%
\pgfpathmoveto{\pgfqpoint{2.499217in}{2.903684in}}%
\pgfpathlineto{\pgfqpoint{2.755563in}{3.027050in}}%
\pgfpathlineto{\pgfqpoint{2.499217in}{2.903684in}}%
\pgfpathclose%
\pgfusepath{fill}%
\end{pgfscope}%
\begin{pgfscope}%
\pgfpathrectangle{\pgfqpoint{0.017500in}{0.017500in}}{\pgfqpoint{6.218282in}{3.580372in}} %
\pgfusepath{clip}%
\pgfsetbuttcap%
\pgfsetroundjoin%
\definecolor{currentfill}{rgb}{0.705673,0.015556,0.150233}%
\pgfsetfillcolor{currentfill}%
\pgfsetlinewidth{0.000000pt}%
\definecolor{currentstroke}{rgb}{0.000000,0.000000,0.000000}%
\pgfsetstrokecolor{currentstroke}%
\pgfsetdash{}{0pt}%
\pgfpathmoveto{\pgfqpoint{2.499217in}{2.903684in}}%
\pgfpathlineto{\pgfqpoint{2.755563in}{3.027050in}}%
\pgfpathlineto{\pgfqpoint{2.499217in}{2.903684in}}%
\pgfpathclose%
\pgfusepath{fill}%
\end{pgfscope}%
\begin{pgfscope}%
\pgfpathrectangle{\pgfqpoint{0.017500in}{0.017500in}}{\pgfqpoint{6.218282in}{3.580372in}} %
\pgfusepath{clip}%
\pgfsetbuttcap%
\pgfsetroundjoin%
\definecolor{currentfill}{rgb}{0.705673,0.015556,0.150233}%
\pgfsetfillcolor{currentfill}%
\pgfsetlinewidth{0.000000pt}%
\definecolor{currentstroke}{rgb}{0.000000,0.000000,0.000000}%
\pgfsetstrokecolor{currentstroke}%
\pgfsetdash{}{0pt}%
\pgfpathmoveto{\pgfqpoint{2.499217in}{2.903684in}}%
\pgfpathlineto{\pgfqpoint{2.755563in}{3.027050in}}%
\pgfpathlineto{\pgfqpoint{2.499217in}{2.903684in}}%
\pgfpathclose%
\pgfusepath{fill}%
\end{pgfscope}%
\begin{pgfscope}%
\pgfpathrectangle{\pgfqpoint{0.017500in}{0.017500in}}{\pgfqpoint{6.218282in}{3.580372in}} %
\pgfusepath{clip}%
\pgfsetbuttcap%
\pgfsetroundjoin%
\definecolor{currentfill}{rgb}{0.705673,0.015556,0.150233}%
\pgfsetfillcolor{currentfill}%
\pgfsetlinewidth{0.000000pt}%
\definecolor{currentstroke}{rgb}{0.000000,0.000000,0.000000}%
\pgfsetstrokecolor{currentstroke}%
\pgfsetdash{}{0pt}%
\pgfpathmoveto{\pgfqpoint{2.499217in}{2.903684in}}%
\pgfpathlineto{\pgfqpoint{2.755563in}{3.027050in}}%
\pgfpathlineto{\pgfqpoint{2.499217in}{2.903684in}}%
\pgfpathclose%
\pgfusepath{fill}%
\end{pgfscope}%
\begin{pgfscope}%
\pgfpathrectangle{\pgfqpoint{0.017500in}{0.017500in}}{\pgfqpoint{6.218282in}{3.580372in}} %
\pgfusepath{clip}%
\pgfsetbuttcap%
\pgfsetroundjoin%
\definecolor{currentfill}{rgb}{0.705673,0.015556,0.150233}%
\pgfsetfillcolor{currentfill}%
\pgfsetlinewidth{0.000000pt}%
\definecolor{currentstroke}{rgb}{0.000000,0.000000,0.000000}%
\pgfsetstrokecolor{currentstroke}%
\pgfsetdash{}{0pt}%
\pgfpathmoveto{\pgfqpoint{2.499217in}{2.903684in}}%
\pgfpathlineto{\pgfqpoint{2.755563in}{3.027050in}}%
\pgfpathlineto{\pgfqpoint{2.499217in}{2.903684in}}%
\pgfpathclose%
\pgfusepath{fill}%
\end{pgfscope}%
\begin{pgfscope}%
\pgfpathrectangle{\pgfqpoint{0.017500in}{0.017500in}}{\pgfqpoint{6.218282in}{3.580372in}} %
\pgfusepath{clip}%
\pgfsetbuttcap%
\pgfsetroundjoin%
\definecolor{currentfill}{rgb}{0.705673,0.015556,0.150233}%
\pgfsetfillcolor{currentfill}%
\pgfsetlinewidth{0.000000pt}%
\definecolor{currentstroke}{rgb}{0.000000,0.000000,0.000000}%
\pgfsetstrokecolor{currentstroke}%
\pgfsetdash{}{0pt}%
\pgfpathmoveto{\pgfqpoint{2.499217in}{2.903684in}}%
\pgfpathlineto{\pgfqpoint{2.755563in}{3.027050in}}%
\pgfpathlineto{\pgfqpoint{2.499217in}{2.903684in}}%
\pgfpathclose%
\pgfusepath{fill}%
\end{pgfscope}%
\begin{pgfscope}%
\pgfpathrectangle{\pgfqpoint{0.017500in}{0.017500in}}{\pgfqpoint{6.218282in}{3.580372in}} %
\pgfusepath{clip}%
\pgfsetbuttcap%
\pgfsetroundjoin%
\definecolor{currentfill}{rgb}{0.705673,0.015556,0.150233}%
\pgfsetfillcolor{currentfill}%
\pgfsetlinewidth{0.000000pt}%
\definecolor{currentstroke}{rgb}{0.000000,0.000000,0.000000}%
\pgfsetstrokecolor{currentstroke}%
\pgfsetdash{}{0pt}%
\pgfpathmoveto{\pgfqpoint{2.499217in}{2.903684in}}%
\pgfpathlineto{\pgfqpoint{2.755563in}{3.027050in}}%
\pgfpathlineto{\pgfqpoint{2.499217in}{2.903684in}}%
\pgfpathclose%
\pgfusepath{fill}%
\end{pgfscope}%
\begin{pgfscope}%
\pgfpathrectangle{\pgfqpoint{0.017500in}{0.017500in}}{\pgfqpoint{6.218282in}{3.580372in}} %
\pgfusepath{clip}%
\pgfsetbuttcap%
\pgfsetroundjoin%
\definecolor{currentfill}{rgb}{0.705673,0.015556,0.150233}%
\pgfsetfillcolor{currentfill}%
\pgfsetlinewidth{0.000000pt}%
\definecolor{currentstroke}{rgb}{0.000000,0.000000,0.000000}%
\pgfsetstrokecolor{currentstroke}%
\pgfsetdash{}{0pt}%
\pgfpathmoveto{\pgfqpoint{2.499217in}{2.903684in}}%
\pgfpathlineto{\pgfqpoint{2.755563in}{3.027050in}}%
\pgfpathlineto{\pgfqpoint{2.499217in}{2.903684in}}%
\pgfpathclose%
\pgfusepath{fill}%
\end{pgfscope}%
\begin{pgfscope}%
\pgfpathrectangle{\pgfqpoint{0.017500in}{0.017500in}}{\pgfqpoint{6.218282in}{3.580372in}} %
\pgfusepath{clip}%
\pgfsetbuttcap%
\pgfsetroundjoin%
\definecolor{currentfill}{rgb}{0.705673,0.015556,0.150233}%
\pgfsetfillcolor{currentfill}%
\pgfsetlinewidth{0.000000pt}%
\definecolor{currentstroke}{rgb}{0.000000,0.000000,0.000000}%
\pgfsetstrokecolor{currentstroke}%
\pgfsetdash{}{0pt}%
\pgfpathmoveto{\pgfqpoint{2.499217in}{2.903684in}}%
\pgfpathlineto{\pgfqpoint{2.755563in}{3.027050in}}%
\pgfpathlineto{\pgfqpoint{2.499217in}{2.903684in}}%
\pgfpathclose%
\pgfusepath{fill}%
\end{pgfscope}%
\begin{pgfscope}%
\pgfpathrectangle{\pgfqpoint{0.017500in}{0.017500in}}{\pgfqpoint{6.218282in}{3.580372in}} %
\pgfusepath{clip}%
\pgfsetbuttcap%
\pgfsetroundjoin%
\definecolor{currentfill}{rgb}{0.705673,0.015556,0.150233}%
\pgfsetfillcolor{currentfill}%
\pgfsetlinewidth{0.000000pt}%
\definecolor{currentstroke}{rgb}{0.000000,0.000000,0.000000}%
\pgfsetstrokecolor{currentstroke}%
\pgfsetdash{}{0pt}%
\pgfpathmoveto{\pgfqpoint{2.499217in}{2.903684in}}%
\pgfpathlineto{\pgfqpoint{2.755563in}{3.027050in}}%
\pgfpathlineto{\pgfqpoint{2.499217in}{2.903684in}}%
\pgfpathclose%
\pgfusepath{fill}%
\end{pgfscope}%
\begin{pgfscope}%
\pgfpathrectangle{\pgfqpoint{0.017500in}{0.017500in}}{\pgfqpoint{6.218282in}{3.580372in}} %
\pgfusepath{clip}%
\pgfsetbuttcap%
\pgfsetroundjoin%
\definecolor{currentfill}{rgb}{0.705673,0.015556,0.150233}%
\pgfsetfillcolor{currentfill}%
\pgfsetlinewidth{0.000000pt}%
\definecolor{currentstroke}{rgb}{0.000000,0.000000,0.000000}%
\pgfsetstrokecolor{currentstroke}%
\pgfsetdash{}{0pt}%
\pgfpathmoveto{\pgfqpoint{2.499217in}{2.903684in}}%
\pgfpathlineto{\pgfqpoint{2.755563in}{3.027050in}}%
\pgfpathlineto{\pgfqpoint{2.499217in}{2.903684in}}%
\pgfpathclose%
\pgfusepath{fill}%
\end{pgfscope}%
\begin{pgfscope}%
\pgfpathrectangle{\pgfqpoint{0.017500in}{0.017500in}}{\pgfqpoint{6.218282in}{3.580372in}} %
\pgfusepath{clip}%
\pgfsetbuttcap%
\pgfsetroundjoin%
\definecolor{currentfill}{rgb}{0.705673,0.015556,0.150233}%
\pgfsetfillcolor{currentfill}%
\pgfsetlinewidth{0.000000pt}%
\definecolor{currentstroke}{rgb}{0.000000,0.000000,0.000000}%
\pgfsetstrokecolor{currentstroke}%
\pgfsetdash{}{0pt}%
\pgfpathmoveto{\pgfqpoint{2.499217in}{2.903684in}}%
\pgfpathlineto{\pgfqpoint{2.755563in}{3.027050in}}%
\pgfpathlineto{\pgfqpoint{2.499217in}{2.903684in}}%
\pgfpathclose%
\pgfusepath{fill}%
\end{pgfscope}%
\begin{pgfscope}%
\pgfpathrectangle{\pgfqpoint{0.017500in}{0.017500in}}{\pgfqpoint{6.218282in}{3.580372in}} %
\pgfusepath{clip}%
\pgfsetbuttcap%
\pgfsetroundjoin%
\definecolor{currentfill}{rgb}{0.705673,0.015556,0.150233}%
\pgfsetfillcolor{currentfill}%
\pgfsetlinewidth{0.000000pt}%
\definecolor{currentstroke}{rgb}{0.000000,0.000000,0.000000}%
\pgfsetstrokecolor{currentstroke}%
\pgfsetdash{}{0pt}%
\pgfpathmoveto{\pgfqpoint{2.499217in}{2.903684in}}%
\pgfpathlineto{\pgfqpoint{2.755563in}{3.027050in}}%
\pgfpathlineto{\pgfqpoint{2.499217in}{2.903684in}}%
\pgfpathclose%
\pgfusepath{fill}%
\end{pgfscope}%
\begin{pgfscope}%
\pgfpathrectangle{\pgfqpoint{0.017500in}{0.017500in}}{\pgfqpoint{6.218282in}{3.580372in}} %
\pgfusepath{clip}%
\pgfsetbuttcap%
\pgfsetroundjoin%
\definecolor{currentfill}{rgb}{0.705673,0.015556,0.150233}%
\pgfsetfillcolor{currentfill}%
\pgfsetlinewidth{0.000000pt}%
\definecolor{currentstroke}{rgb}{0.000000,0.000000,0.000000}%
\pgfsetstrokecolor{currentstroke}%
\pgfsetdash{}{0pt}%
\pgfpathmoveto{\pgfqpoint{2.499217in}{2.903684in}}%
\pgfpathlineto{\pgfqpoint{2.755563in}{3.027050in}}%
\pgfpathlineto{\pgfqpoint{2.499217in}{2.903684in}}%
\pgfpathclose%
\pgfusepath{fill}%
\end{pgfscope}%
\begin{pgfscope}%
\pgfpathrectangle{\pgfqpoint{0.017500in}{0.017500in}}{\pgfqpoint{6.218282in}{3.580372in}} %
\pgfusepath{clip}%
\pgfsetbuttcap%
\pgfsetroundjoin%
\definecolor{currentfill}{rgb}{0.705673,0.015556,0.150233}%
\pgfsetfillcolor{currentfill}%
\pgfsetlinewidth{0.000000pt}%
\definecolor{currentstroke}{rgb}{0.000000,0.000000,0.000000}%
\pgfsetstrokecolor{currentstroke}%
\pgfsetdash{}{0pt}%
\pgfpathmoveto{\pgfqpoint{2.499217in}{2.903684in}}%
\pgfpathlineto{\pgfqpoint{2.755563in}{3.027050in}}%
\pgfpathlineto{\pgfqpoint{2.499217in}{2.903684in}}%
\pgfpathclose%
\pgfusepath{fill}%
\end{pgfscope}%
\begin{pgfscope}%
\pgfpathrectangle{\pgfqpoint{0.017500in}{0.017500in}}{\pgfqpoint{6.218282in}{3.580372in}} %
\pgfusepath{clip}%
\pgfsetbuttcap%
\pgfsetroundjoin%
\definecolor{currentfill}{rgb}{0.705673,0.015556,0.150233}%
\pgfsetfillcolor{currentfill}%
\pgfsetlinewidth{0.000000pt}%
\definecolor{currentstroke}{rgb}{0.000000,0.000000,0.000000}%
\pgfsetstrokecolor{currentstroke}%
\pgfsetdash{}{0pt}%
\pgfpathmoveto{\pgfqpoint{2.499217in}{2.903684in}}%
\pgfpathlineto{\pgfqpoint{2.755563in}{3.027050in}}%
\pgfpathlineto{\pgfqpoint{2.499217in}{2.903684in}}%
\pgfpathclose%
\pgfusepath{fill}%
\end{pgfscope}%
\begin{pgfscope}%
\pgfpathrectangle{\pgfqpoint{0.017500in}{0.017500in}}{\pgfqpoint{6.218282in}{3.580372in}} %
\pgfusepath{clip}%
\pgfsetbuttcap%
\pgfsetroundjoin%
\definecolor{currentfill}{rgb}{0.705673,0.015556,0.150233}%
\pgfsetfillcolor{currentfill}%
\pgfsetlinewidth{0.000000pt}%
\definecolor{currentstroke}{rgb}{0.000000,0.000000,0.000000}%
\pgfsetstrokecolor{currentstroke}%
\pgfsetdash{}{0pt}%
\pgfpathmoveto{\pgfqpoint{2.499217in}{2.903684in}}%
\pgfpathlineto{\pgfqpoint{2.755563in}{3.027050in}}%
\pgfpathlineto{\pgfqpoint{2.499217in}{2.903684in}}%
\pgfpathclose%
\pgfusepath{fill}%
\end{pgfscope}%
\begin{pgfscope}%
\pgfpathrectangle{\pgfqpoint{0.017500in}{0.017500in}}{\pgfqpoint{6.218282in}{3.580372in}} %
\pgfusepath{clip}%
\pgfsetbuttcap%
\pgfsetroundjoin%
\definecolor{currentfill}{rgb}{0.705673,0.015556,0.150233}%
\pgfsetfillcolor{currentfill}%
\pgfsetlinewidth{0.000000pt}%
\definecolor{currentstroke}{rgb}{0.000000,0.000000,0.000000}%
\pgfsetstrokecolor{currentstroke}%
\pgfsetdash{}{0pt}%
\pgfpathmoveto{\pgfqpoint{2.499217in}{2.903684in}}%
\pgfpathlineto{\pgfqpoint{2.755563in}{3.027050in}}%
\pgfpathlineto{\pgfqpoint{2.499217in}{2.903684in}}%
\pgfpathclose%
\pgfusepath{fill}%
\end{pgfscope}%
\begin{pgfscope}%
\pgfpathrectangle{\pgfqpoint{0.017500in}{0.017500in}}{\pgfqpoint{6.218282in}{3.580372in}} %
\pgfusepath{clip}%
\pgfsetbuttcap%
\pgfsetroundjoin%
\definecolor{currentfill}{rgb}{0.705673,0.015556,0.150233}%
\pgfsetfillcolor{currentfill}%
\pgfsetlinewidth{0.000000pt}%
\definecolor{currentstroke}{rgb}{0.000000,0.000000,0.000000}%
\pgfsetstrokecolor{currentstroke}%
\pgfsetdash{}{0pt}%
\pgfpathmoveto{\pgfqpoint{2.499217in}{2.903684in}}%
\pgfpathlineto{\pgfqpoint{2.755563in}{3.027050in}}%
\pgfpathlineto{\pgfqpoint{2.499217in}{2.903684in}}%
\pgfpathclose%
\pgfusepath{fill}%
\end{pgfscope}%
\begin{pgfscope}%
\pgfpathrectangle{\pgfqpoint{0.017500in}{0.017500in}}{\pgfqpoint{6.218282in}{3.580372in}} %
\pgfusepath{clip}%
\pgfsetbuttcap%
\pgfsetroundjoin%
\definecolor{currentfill}{rgb}{0.705673,0.015556,0.150233}%
\pgfsetfillcolor{currentfill}%
\pgfsetlinewidth{0.000000pt}%
\definecolor{currentstroke}{rgb}{0.000000,0.000000,0.000000}%
\pgfsetstrokecolor{currentstroke}%
\pgfsetdash{}{0pt}%
\pgfpathmoveto{\pgfqpoint{2.499217in}{2.903684in}}%
\pgfpathlineto{\pgfqpoint{2.755563in}{3.027050in}}%
\pgfpathlineto{\pgfqpoint{2.499217in}{2.903684in}}%
\pgfpathclose%
\pgfusepath{fill}%
\end{pgfscope}%
\begin{pgfscope}%
\pgfpathrectangle{\pgfqpoint{0.017500in}{0.017500in}}{\pgfqpoint{6.218282in}{3.580372in}} %
\pgfusepath{clip}%
\pgfsetbuttcap%
\pgfsetroundjoin%
\definecolor{currentfill}{rgb}{0.705673,0.015556,0.150233}%
\pgfsetfillcolor{currentfill}%
\pgfsetlinewidth{0.000000pt}%
\definecolor{currentstroke}{rgb}{0.000000,0.000000,0.000000}%
\pgfsetstrokecolor{currentstroke}%
\pgfsetdash{}{0pt}%
\pgfpathmoveto{\pgfqpoint{2.499217in}{2.903684in}}%
\pgfpathlineto{\pgfqpoint{2.755563in}{3.027050in}}%
\pgfpathlineto{\pgfqpoint{2.499217in}{2.903684in}}%
\pgfpathclose%
\pgfusepath{fill}%
\end{pgfscope}%
\begin{pgfscope}%
\pgfpathrectangle{\pgfqpoint{0.017500in}{0.017500in}}{\pgfqpoint{6.218282in}{3.580372in}} %
\pgfusepath{clip}%
\pgfsetbuttcap%
\pgfsetroundjoin%
\definecolor{currentfill}{rgb}{0.705673,0.015556,0.150233}%
\pgfsetfillcolor{currentfill}%
\pgfsetlinewidth{0.000000pt}%
\definecolor{currentstroke}{rgb}{0.000000,0.000000,0.000000}%
\pgfsetstrokecolor{currentstroke}%
\pgfsetdash{}{0pt}%
\pgfpathmoveto{\pgfqpoint{2.499217in}{2.903684in}}%
\pgfpathlineto{\pgfqpoint{2.755563in}{3.027050in}}%
\pgfpathlineto{\pgfqpoint{2.499217in}{2.903684in}}%
\pgfpathclose%
\pgfusepath{fill}%
\end{pgfscope}%
\begin{pgfscope}%
\pgfpathrectangle{\pgfqpoint{0.017500in}{0.017500in}}{\pgfqpoint{6.218282in}{3.580372in}} %
\pgfusepath{clip}%
\pgfsetbuttcap%
\pgfsetroundjoin%
\definecolor{currentfill}{rgb}{0.809329,0.852974,0.922323}%
\pgfsetfillcolor{currentfill}%
\pgfsetlinewidth{0.000000pt}%
\definecolor{currentstroke}{rgb}{0.000000,0.000000,0.000000}%
\pgfsetstrokecolor{currentstroke}%
\pgfsetdash{}{0pt}%
\pgfpathmoveto{\pgfqpoint{2.755563in}{3.027050in}}%
\pgfpathlineto{\pgfqpoint{3.186131in}{2.954270in}}%
\pgfpathlineto{\pgfqpoint{1.914830in}{0.967594in}}%
\pgfpathlineto{\pgfqpoint{1.486050in}{1.042638in}}%
\pgfpathlineto{\pgfqpoint{2.755563in}{3.027050in}}%
\pgfpathclose%
\pgfusepath{fill}%
\end{pgfscope}%
\begin{pgfscope}%
\pgfpathrectangle{\pgfqpoint{0.017500in}{0.017500in}}{\pgfqpoint{6.218282in}{3.580372in}} %
\pgfusepath{clip}%
\pgfsetbuttcap%
\pgfsetroundjoin%
\definecolor{currentfill}{rgb}{0.809329,0.852974,0.922323}%
\pgfsetfillcolor{currentfill}%
\pgfsetlinewidth{0.000000pt}%
\definecolor{currentstroke}{rgb}{0.000000,0.000000,0.000000}%
\pgfsetstrokecolor{currentstroke}%
\pgfsetdash{}{0pt}%
\pgfpathmoveto{\pgfqpoint{2.755563in}{3.027050in}}%
\pgfpathlineto{\pgfqpoint{3.186131in}{2.954270in}}%
\pgfpathlineto{\pgfqpoint{1.914830in}{0.967594in}}%
\pgfpathlineto{\pgfqpoint{1.486050in}{1.042638in}}%
\pgfpathlineto{\pgfqpoint{2.755563in}{3.027050in}}%
\pgfpathclose%
\pgfusepath{fill}%
\end{pgfscope}%
\begin{pgfscope}%
\pgfpathrectangle{\pgfqpoint{0.017500in}{0.017500in}}{\pgfqpoint{6.218282in}{3.580372in}} %
\pgfusepath{clip}%
\pgfsetbuttcap%
\pgfsetroundjoin%
\definecolor{currentfill}{rgb}{0.809329,0.852974,0.922323}%
\pgfsetfillcolor{currentfill}%
\pgfsetlinewidth{0.000000pt}%
\definecolor{currentstroke}{rgb}{0.000000,0.000000,0.000000}%
\pgfsetstrokecolor{currentstroke}%
\pgfsetdash{}{0pt}%
\pgfpathmoveto{\pgfqpoint{2.755563in}{3.027050in}}%
\pgfpathlineto{\pgfqpoint{3.186131in}{2.954270in}}%
\pgfpathlineto{\pgfqpoint{1.914830in}{0.967594in}}%
\pgfpathlineto{\pgfqpoint{1.486050in}{1.042638in}}%
\pgfpathlineto{\pgfqpoint{2.755563in}{3.027050in}}%
\pgfpathclose%
\pgfusepath{fill}%
\end{pgfscope}%
\begin{pgfscope}%
\pgfpathrectangle{\pgfqpoint{0.017500in}{0.017500in}}{\pgfqpoint{6.218282in}{3.580372in}} %
\pgfusepath{clip}%
\pgfsetbuttcap%
\pgfsetroundjoin%
\definecolor{currentfill}{rgb}{0.809329,0.852974,0.922323}%
\pgfsetfillcolor{currentfill}%
\pgfsetlinewidth{0.000000pt}%
\definecolor{currentstroke}{rgb}{0.000000,0.000000,0.000000}%
\pgfsetstrokecolor{currentstroke}%
\pgfsetdash{}{0pt}%
\pgfpathmoveto{\pgfqpoint{2.755563in}{3.027050in}}%
\pgfpathlineto{\pgfqpoint{3.186131in}{2.954270in}}%
\pgfpathlineto{\pgfqpoint{1.914830in}{0.967594in}}%
\pgfpathlineto{\pgfqpoint{1.486050in}{1.042638in}}%
\pgfpathlineto{\pgfqpoint{2.755563in}{3.027050in}}%
\pgfpathclose%
\pgfusepath{fill}%
\end{pgfscope}%
\begin{pgfscope}%
\pgfpathrectangle{\pgfqpoint{0.017500in}{0.017500in}}{\pgfqpoint{6.218282in}{3.580372in}} %
\pgfusepath{clip}%
\pgfsetbuttcap%
\pgfsetroundjoin%
\definecolor{currentfill}{rgb}{0.809329,0.852974,0.922323}%
\pgfsetfillcolor{currentfill}%
\pgfsetlinewidth{0.000000pt}%
\definecolor{currentstroke}{rgb}{0.000000,0.000000,0.000000}%
\pgfsetstrokecolor{currentstroke}%
\pgfsetdash{}{0pt}%
\pgfpathmoveto{\pgfqpoint{2.755563in}{3.027050in}}%
\pgfpathlineto{\pgfqpoint{3.186131in}{2.954270in}}%
\pgfpathlineto{\pgfqpoint{1.914830in}{0.967594in}}%
\pgfpathlineto{\pgfqpoint{1.486050in}{1.042638in}}%
\pgfpathlineto{\pgfqpoint{2.755563in}{3.027050in}}%
\pgfpathclose%
\pgfusepath{fill}%
\end{pgfscope}%
\begin{pgfscope}%
\pgfpathrectangle{\pgfqpoint{0.017500in}{0.017500in}}{\pgfqpoint{6.218282in}{3.580372in}} %
\pgfusepath{clip}%
\pgfsetbuttcap%
\pgfsetroundjoin%
\definecolor{currentfill}{rgb}{0.711554,0.033337,0.154485}%
\pgfsetfillcolor{currentfill}%
\pgfsetlinewidth{0.000000pt}%
\definecolor{currentstroke}{rgb}{0.000000,0.000000,0.000000}%
\pgfsetstrokecolor{currentstroke}%
\pgfsetdash{}{0pt}%
\pgfpathmoveto{\pgfqpoint{2.499217in}{2.903684in}}%
\pgfpathlineto{\pgfqpoint{2.932968in}{2.820668in}}%
\pgfpathlineto{\pgfqpoint{3.186131in}{2.954270in}}%
\pgfpathlineto{\pgfqpoint{2.755563in}{3.027050in}}%
\pgfpathlineto{\pgfqpoint{2.499217in}{2.903684in}}%
\pgfpathclose%
\pgfusepath{fill}%
\end{pgfscope}%
\begin{pgfscope}%
\pgfpathrectangle{\pgfqpoint{0.017500in}{0.017500in}}{\pgfqpoint{6.218282in}{3.580372in}} %
\pgfusepath{clip}%
\pgfsetbuttcap%
\pgfsetroundjoin%
\definecolor{currentfill}{rgb}{0.711554,0.033337,0.154485}%
\pgfsetfillcolor{currentfill}%
\pgfsetlinewidth{0.000000pt}%
\definecolor{currentstroke}{rgb}{0.000000,0.000000,0.000000}%
\pgfsetstrokecolor{currentstroke}%
\pgfsetdash{}{0pt}%
\pgfpathmoveto{\pgfqpoint{2.499217in}{2.903684in}}%
\pgfpathlineto{\pgfqpoint{2.932968in}{2.820668in}}%
\pgfpathlineto{\pgfqpoint{3.186131in}{2.954270in}}%
\pgfpathlineto{\pgfqpoint{2.755563in}{3.027050in}}%
\pgfpathlineto{\pgfqpoint{2.499217in}{2.903684in}}%
\pgfpathclose%
\pgfusepath{fill}%
\end{pgfscope}%
\begin{pgfscope}%
\pgfpathrectangle{\pgfqpoint{0.017500in}{0.017500in}}{\pgfqpoint{6.218282in}{3.580372in}} %
\pgfusepath{clip}%
\pgfsetbuttcap%
\pgfsetroundjoin%
\definecolor{currentfill}{rgb}{0.711554,0.033337,0.154485}%
\pgfsetfillcolor{currentfill}%
\pgfsetlinewidth{0.000000pt}%
\definecolor{currentstroke}{rgb}{0.000000,0.000000,0.000000}%
\pgfsetstrokecolor{currentstroke}%
\pgfsetdash{}{0pt}%
\pgfpathmoveto{\pgfqpoint{2.499217in}{2.903684in}}%
\pgfpathlineto{\pgfqpoint{2.932968in}{2.820668in}}%
\pgfpathlineto{\pgfqpoint{3.186131in}{2.954270in}}%
\pgfpathlineto{\pgfqpoint{2.755563in}{3.027050in}}%
\pgfpathlineto{\pgfqpoint{2.499217in}{2.903684in}}%
\pgfpathclose%
\pgfusepath{fill}%
\end{pgfscope}%
\begin{pgfscope}%
\pgfpathrectangle{\pgfqpoint{0.017500in}{0.017500in}}{\pgfqpoint{6.218282in}{3.580372in}} %
\pgfusepath{clip}%
\pgfsetbuttcap%
\pgfsetroundjoin%
\definecolor{currentfill}{rgb}{0.711554,0.033337,0.154485}%
\pgfsetfillcolor{currentfill}%
\pgfsetlinewidth{0.000000pt}%
\definecolor{currentstroke}{rgb}{0.000000,0.000000,0.000000}%
\pgfsetstrokecolor{currentstroke}%
\pgfsetdash{}{0pt}%
\pgfpathmoveto{\pgfqpoint{2.499217in}{2.903684in}}%
\pgfpathlineto{\pgfqpoint{2.932968in}{2.820668in}}%
\pgfpathlineto{\pgfqpoint{3.186131in}{2.954270in}}%
\pgfpathlineto{\pgfqpoint{2.755563in}{3.027050in}}%
\pgfpathlineto{\pgfqpoint{2.499217in}{2.903684in}}%
\pgfpathclose%
\pgfusepath{fill}%
\end{pgfscope}%
\begin{pgfscope}%
\pgfpathrectangle{\pgfqpoint{0.017500in}{0.017500in}}{\pgfqpoint{6.218282in}{3.580372in}} %
\pgfusepath{clip}%
\pgfsetbuttcap%
\pgfsetroundjoin%
\definecolor{currentfill}{rgb}{0.711554,0.033337,0.154485}%
\pgfsetfillcolor{currentfill}%
\pgfsetlinewidth{0.000000pt}%
\definecolor{currentstroke}{rgb}{0.000000,0.000000,0.000000}%
\pgfsetstrokecolor{currentstroke}%
\pgfsetdash{}{0pt}%
\pgfpathmoveto{\pgfqpoint{2.499217in}{2.903684in}}%
\pgfpathlineto{\pgfqpoint{2.932968in}{2.820668in}}%
\pgfpathlineto{\pgfqpoint{3.186131in}{2.954270in}}%
\pgfpathlineto{\pgfqpoint{2.755563in}{3.027050in}}%
\pgfpathlineto{\pgfqpoint{2.499217in}{2.903684in}}%
\pgfpathclose%
\pgfusepath{fill}%
\end{pgfscope}%
\begin{pgfscope}%
\pgfpathrectangle{\pgfqpoint{0.017500in}{0.017500in}}{\pgfqpoint{6.218282in}{3.580372in}} %
\pgfusepath{clip}%
\pgfsetbuttcap%
\pgfsetroundjoin%
\definecolor{currentfill}{rgb}{0.711554,0.033337,0.154485}%
\pgfsetfillcolor{currentfill}%
\pgfsetlinewidth{0.000000pt}%
\definecolor{currentstroke}{rgb}{0.000000,0.000000,0.000000}%
\pgfsetstrokecolor{currentstroke}%
\pgfsetdash{}{0pt}%
\pgfpathmoveto{\pgfqpoint{2.499217in}{2.903684in}}%
\pgfpathlineto{\pgfqpoint{2.932968in}{2.820668in}}%
\pgfpathlineto{\pgfqpoint{3.186131in}{2.954270in}}%
\pgfpathlineto{\pgfqpoint{2.755563in}{3.027050in}}%
\pgfpathlineto{\pgfqpoint{2.499217in}{2.903684in}}%
\pgfpathclose%
\pgfusepath{fill}%
\end{pgfscope}%
\begin{pgfscope}%
\pgfpathrectangle{\pgfqpoint{0.017500in}{0.017500in}}{\pgfqpoint{6.218282in}{3.580372in}} %
\pgfusepath{clip}%
\pgfsetbuttcap%
\pgfsetroundjoin%
\definecolor{currentfill}{rgb}{0.902849,0.844796,0.811970}%
\pgfsetfillcolor{currentfill}%
\pgfsetlinewidth{0.000000pt}%
\definecolor{currentstroke}{rgb}{0.000000,0.000000,0.000000}%
\pgfsetstrokecolor{currentstroke}%
\pgfsetdash{}{0pt}%
\pgfpathmoveto{\pgfqpoint{3.186131in}{2.954270in}}%
\pgfpathlineto{\pgfqpoint{1.914830in}{0.967594in}}%
\pgfpathlineto{\pgfqpoint{3.186131in}{2.954270in}}%
\pgfpathclose%
\pgfusepath{fill}%
\end{pgfscope}%
\begin{pgfscope}%
\pgfpathrectangle{\pgfqpoint{0.017500in}{0.017500in}}{\pgfqpoint{6.218282in}{3.580372in}} %
\pgfusepath{clip}%
\pgfsetbuttcap%
\pgfsetroundjoin%
\definecolor{currentfill}{rgb}{0.902849,0.844796,0.811970}%
\pgfsetfillcolor{currentfill}%
\pgfsetlinewidth{0.000000pt}%
\definecolor{currentstroke}{rgb}{0.000000,0.000000,0.000000}%
\pgfsetstrokecolor{currentstroke}%
\pgfsetdash{}{0pt}%
\pgfpathmoveto{\pgfqpoint{3.186131in}{2.954270in}}%
\pgfpathlineto{\pgfqpoint{1.914830in}{0.967594in}}%
\pgfpathlineto{\pgfqpoint{3.186131in}{2.954270in}}%
\pgfpathclose%
\pgfusepath{fill}%
\end{pgfscope}%
\begin{pgfscope}%
\pgfpathrectangle{\pgfqpoint{0.017500in}{0.017500in}}{\pgfqpoint{6.218282in}{3.580372in}} %
\pgfusepath{clip}%
\pgfsetbuttcap%
\pgfsetroundjoin%
\definecolor{currentfill}{rgb}{0.902849,0.844796,0.811970}%
\pgfsetfillcolor{currentfill}%
\pgfsetlinewidth{0.000000pt}%
\definecolor{currentstroke}{rgb}{0.000000,0.000000,0.000000}%
\pgfsetstrokecolor{currentstroke}%
\pgfsetdash{}{0pt}%
\pgfpathmoveto{\pgfqpoint{3.186131in}{2.954270in}}%
\pgfpathlineto{\pgfqpoint{1.914830in}{0.967594in}}%
\pgfpathlineto{\pgfqpoint{3.186131in}{2.954270in}}%
\pgfpathclose%
\pgfusepath{fill}%
\end{pgfscope}%
\begin{pgfscope}%
\pgfpathrectangle{\pgfqpoint{0.017500in}{0.017500in}}{\pgfqpoint{6.218282in}{3.580372in}} %
\pgfusepath{clip}%
\pgfsetbuttcap%
\pgfsetroundjoin%
\definecolor{currentfill}{rgb}{0.902849,0.844796,0.811970}%
\pgfsetfillcolor{currentfill}%
\pgfsetlinewidth{0.000000pt}%
\definecolor{currentstroke}{rgb}{0.000000,0.000000,0.000000}%
\pgfsetstrokecolor{currentstroke}%
\pgfsetdash{}{0pt}%
\pgfpathmoveto{\pgfqpoint{3.186131in}{2.954270in}}%
\pgfpathlineto{\pgfqpoint{1.914830in}{0.967594in}}%
\pgfpathlineto{\pgfqpoint{3.186131in}{2.954270in}}%
\pgfpathclose%
\pgfusepath{fill}%
\end{pgfscope}%
\begin{pgfscope}%
\pgfpathrectangle{\pgfqpoint{0.017500in}{0.017500in}}{\pgfqpoint{6.218282in}{3.580372in}} %
\pgfusepath{clip}%
\pgfsetbuttcap%
\pgfsetroundjoin%
\definecolor{currentfill}{rgb}{0.902849,0.844796,0.811970}%
\pgfsetfillcolor{currentfill}%
\pgfsetlinewidth{0.000000pt}%
\definecolor{currentstroke}{rgb}{0.000000,0.000000,0.000000}%
\pgfsetstrokecolor{currentstroke}%
\pgfsetdash{}{0pt}%
\pgfpathmoveto{\pgfqpoint{3.186131in}{2.954270in}}%
\pgfpathlineto{\pgfqpoint{1.914830in}{0.967594in}}%
\pgfpathlineto{\pgfqpoint{3.186131in}{2.954270in}}%
\pgfpathclose%
\pgfusepath{fill}%
\end{pgfscope}%
\begin{pgfscope}%
\pgfpathrectangle{\pgfqpoint{0.017500in}{0.017500in}}{\pgfqpoint{6.218282in}{3.580372in}} %
\pgfusepath{clip}%
\pgfsetbuttcap%
\pgfsetroundjoin%
\definecolor{currentfill}{rgb}{0.902849,0.844796,0.811970}%
\pgfsetfillcolor{currentfill}%
\pgfsetlinewidth{0.000000pt}%
\definecolor{currentstroke}{rgb}{0.000000,0.000000,0.000000}%
\pgfsetstrokecolor{currentstroke}%
\pgfsetdash{}{0pt}%
\pgfpathmoveto{\pgfqpoint{3.186131in}{2.954270in}}%
\pgfpathlineto{\pgfqpoint{1.914830in}{0.967594in}}%
\pgfpathlineto{\pgfqpoint{3.186131in}{2.954270in}}%
\pgfpathclose%
\pgfusepath{fill}%
\end{pgfscope}%
\begin{pgfscope}%
\pgfpathrectangle{\pgfqpoint{0.017500in}{0.017500in}}{\pgfqpoint{6.218282in}{3.580372in}} %
\pgfusepath{clip}%
\pgfsetbuttcap%
\pgfsetroundjoin%
\definecolor{currentfill}{rgb}{0.902849,0.844796,0.811970}%
\pgfsetfillcolor{currentfill}%
\pgfsetlinewidth{0.000000pt}%
\definecolor{currentstroke}{rgb}{0.000000,0.000000,0.000000}%
\pgfsetstrokecolor{currentstroke}%
\pgfsetdash{}{0pt}%
\pgfpathmoveto{\pgfqpoint{3.186131in}{2.954270in}}%
\pgfpathlineto{\pgfqpoint{1.914830in}{0.967594in}}%
\pgfpathlineto{\pgfqpoint{3.186131in}{2.954270in}}%
\pgfpathclose%
\pgfusepath{fill}%
\end{pgfscope}%
\begin{pgfscope}%
\pgfpathrectangle{\pgfqpoint{0.017500in}{0.017500in}}{\pgfqpoint{6.218282in}{3.580372in}} %
\pgfusepath{clip}%
\pgfsetbuttcap%
\pgfsetroundjoin%
\definecolor{currentfill}{rgb}{0.902849,0.844796,0.811970}%
\pgfsetfillcolor{currentfill}%
\pgfsetlinewidth{0.000000pt}%
\definecolor{currentstroke}{rgb}{0.000000,0.000000,0.000000}%
\pgfsetstrokecolor{currentstroke}%
\pgfsetdash{}{0pt}%
\pgfpathmoveto{\pgfqpoint{3.186131in}{2.954270in}}%
\pgfpathlineto{\pgfqpoint{1.914830in}{0.967594in}}%
\pgfpathlineto{\pgfqpoint{3.186131in}{2.954270in}}%
\pgfpathclose%
\pgfusepath{fill}%
\end{pgfscope}%
\begin{pgfscope}%
\pgfpathrectangle{\pgfqpoint{0.017500in}{0.017500in}}{\pgfqpoint{6.218282in}{3.580372in}} %
\pgfusepath{clip}%
\pgfsetbuttcap%
\pgfsetroundjoin%
\definecolor{currentfill}{rgb}{0.902849,0.844796,0.811970}%
\pgfsetfillcolor{currentfill}%
\pgfsetlinewidth{0.000000pt}%
\definecolor{currentstroke}{rgb}{0.000000,0.000000,0.000000}%
\pgfsetstrokecolor{currentstroke}%
\pgfsetdash{}{0pt}%
\pgfpathmoveto{\pgfqpoint{3.186131in}{2.954270in}}%
\pgfpathlineto{\pgfqpoint{1.914830in}{0.967594in}}%
\pgfpathlineto{\pgfqpoint{3.186131in}{2.954270in}}%
\pgfpathclose%
\pgfusepath{fill}%
\end{pgfscope}%
\begin{pgfscope}%
\pgfpathrectangle{\pgfqpoint{0.017500in}{0.017500in}}{\pgfqpoint{6.218282in}{3.580372in}} %
\pgfusepath{clip}%
\pgfsetbuttcap%
\pgfsetroundjoin%
\definecolor{currentfill}{rgb}{0.902849,0.844796,0.811970}%
\pgfsetfillcolor{currentfill}%
\pgfsetlinewidth{0.000000pt}%
\definecolor{currentstroke}{rgb}{0.000000,0.000000,0.000000}%
\pgfsetstrokecolor{currentstroke}%
\pgfsetdash{}{0pt}%
\pgfpathmoveto{\pgfqpoint{3.186131in}{2.954270in}}%
\pgfpathlineto{\pgfqpoint{1.914830in}{0.967594in}}%
\pgfpathlineto{\pgfqpoint{3.186131in}{2.954270in}}%
\pgfpathclose%
\pgfusepath{fill}%
\end{pgfscope}%
\begin{pgfscope}%
\pgfpathrectangle{\pgfqpoint{0.017500in}{0.017500in}}{\pgfqpoint{6.218282in}{3.580372in}} %
\pgfusepath{clip}%
\pgfsetbuttcap%
\pgfsetroundjoin%
\definecolor{currentfill}{rgb}{0.902849,0.844796,0.811970}%
\pgfsetfillcolor{currentfill}%
\pgfsetlinewidth{0.000000pt}%
\definecolor{currentstroke}{rgb}{0.000000,0.000000,0.000000}%
\pgfsetstrokecolor{currentstroke}%
\pgfsetdash{}{0pt}%
\pgfpathmoveto{\pgfqpoint{3.186131in}{2.954270in}}%
\pgfpathlineto{\pgfqpoint{1.914830in}{0.967594in}}%
\pgfpathlineto{\pgfqpoint{3.186131in}{2.954270in}}%
\pgfpathclose%
\pgfusepath{fill}%
\end{pgfscope}%
\begin{pgfscope}%
\pgfpathrectangle{\pgfqpoint{0.017500in}{0.017500in}}{\pgfqpoint{6.218282in}{3.580372in}} %
\pgfusepath{clip}%
\pgfsetbuttcap%
\pgfsetroundjoin%
\definecolor{currentfill}{rgb}{0.902849,0.844796,0.811970}%
\pgfsetfillcolor{currentfill}%
\pgfsetlinewidth{0.000000pt}%
\definecolor{currentstroke}{rgb}{0.000000,0.000000,0.000000}%
\pgfsetstrokecolor{currentstroke}%
\pgfsetdash{}{0pt}%
\pgfpathmoveto{\pgfqpoint{3.186131in}{2.954270in}}%
\pgfpathlineto{\pgfqpoint{1.914830in}{0.967594in}}%
\pgfpathlineto{\pgfqpoint{3.186131in}{2.954270in}}%
\pgfpathclose%
\pgfusepath{fill}%
\end{pgfscope}%
\begin{pgfscope}%
\pgfpathrectangle{\pgfqpoint{0.017500in}{0.017500in}}{\pgfqpoint{6.218282in}{3.580372in}} %
\pgfusepath{clip}%
\pgfsetbuttcap%
\pgfsetroundjoin%
\definecolor{currentfill}{rgb}{0.902849,0.844796,0.811970}%
\pgfsetfillcolor{currentfill}%
\pgfsetlinewidth{0.000000pt}%
\definecolor{currentstroke}{rgb}{0.000000,0.000000,0.000000}%
\pgfsetstrokecolor{currentstroke}%
\pgfsetdash{}{0pt}%
\pgfpathmoveto{\pgfqpoint{3.186131in}{2.954270in}}%
\pgfpathlineto{\pgfqpoint{1.914830in}{0.967594in}}%
\pgfpathlineto{\pgfqpoint{3.186131in}{2.954270in}}%
\pgfpathclose%
\pgfusepath{fill}%
\end{pgfscope}%
\begin{pgfscope}%
\pgfpathrectangle{\pgfqpoint{0.017500in}{0.017500in}}{\pgfqpoint{6.218282in}{3.580372in}} %
\pgfusepath{clip}%
\pgfsetbuttcap%
\pgfsetroundjoin%
\definecolor{currentfill}{rgb}{0.902849,0.844796,0.811970}%
\pgfsetfillcolor{currentfill}%
\pgfsetlinewidth{0.000000pt}%
\definecolor{currentstroke}{rgb}{0.000000,0.000000,0.000000}%
\pgfsetstrokecolor{currentstroke}%
\pgfsetdash{}{0pt}%
\pgfpathmoveto{\pgfqpoint{3.186131in}{2.954270in}}%
\pgfpathlineto{\pgfqpoint{1.914830in}{0.967594in}}%
\pgfpathlineto{\pgfqpoint{3.186131in}{2.954270in}}%
\pgfpathclose%
\pgfusepath{fill}%
\end{pgfscope}%
\begin{pgfscope}%
\pgfpathrectangle{\pgfqpoint{0.017500in}{0.017500in}}{\pgfqpoint{6.218282in}{3.580372in}} %
\pgfusepath{clip}%
\pgfsetbuttcap%
\pgfsetroundjoin%
\definecolor{currentfill}{rgb}{0.902849,0.844796,0.811970}%
\pgfsetfillcolor{currentfill}%
\pgfsetlinewidth{0.000000pt}%
\definecolor{currentstroke}{rgb}{0.000000,0.000000,0.000000}%
\pgfsetstrokecolor{currentstroke}%
\pgfsetdash{}{0pt}%
\pgfpathmoveto{\pgfqpoint{3.186131in}{2.954270in}}%
\pgfpathlineto{\pgfqpoint{1.914830in}{0.967594in}}%
\pgfpathlineto{\pgfqpoint{3.186131in}{2.954270in}}%
\pgfpathclose%
\pgfusepath{fill}%
\end{pgfscope}%
\begin{pgfscope}%
\pgfpathrectangle{\pgfqpoint{0.017500in}{0.017500in}}{\pgfqpoint{6.218282in}{3.580372in}} %
\pgfusepath{clip}%
\pgfsetbuttcap%
\pgfsetroundjoin%
\definecolor{currentfill}{rgb}{0.902849,0.844796,0.811970}%
\pgfsetfillcolor{currentfill}%
\pgfsetlinewidth{0.000000pt}%
\definecolor{currentstroke}{rgb}{0.000000,0.000000,0.000000}%
\pgfsetstrokecolor{currentstroke}%
\pgfsetdash{}{0pt}%
\pgfpathmoveto{\pgfqpoint{3.186131in}{2.954270in}}%
\pgfpathlineto{\pgfqpoint{1.914830in}{0.967594in}}%
\pgfpathlineto{\pgfqpoint{3.186131in}{2.954270in}}%
\pgfpathclose%
\pgfusepath{fill}%
\end{pgfscope}%
\begin{pgfscope}%
\pgfpathrectangle{\pgfqpoint{0.017500in}{0.017500in}}{\pgfqpoint{6.218282in}{3.580372in}} %
\pgfusepath{clip}%
\pgfsetbuttcap%
\pgfsetroundjoin%
\definecolor{currentfill}{rgb}{0.902849,0.844796,0.811970}%
\pgfsetfillcolor{currentfill}%
\pgfsetlinewidth{0.000000pt}%
\definecolor{currentstroke}{rgb}{0.000000,0.000000,0.000000}%
\pgfsetstrokecolor{currentstroke}%
\pgfsetdash{}{0pt}%
\pgfpathmoveto{\pgfqpoint{3.186131in}{2.954270in}}%
\pgfpathlineto{\pgfqpoint{1.914830in}{0.967594in}}%
\pgfpathlineto{\pgfqpoint{3.186131in}{2.954270in}}%
\pgfpathclose%
\pgfusepath{fill}%
\end{pgfscope}%
\begin{pgfscope}%
\pgfpathrectangle{\pgfqpoint{0.017500in}{0.017500in}}{\pgfqpoint{6.218282in}{3.580372in}} %
\pgfusepath{clip}%
\pgfsetbuttcap%
\pgfsetroundjoin%
\definecolor{currentfill}{rgb}{0.902849,0.844796,0.811970}%
\pgfsetfillcolor{currentfill}%
\pgfsetlinewidth{0.000000pt}%
\definecolor{currentstroke}{rgb}{0.000000,0.000000,0.000000}%
\pgfsetstrokecolor{currentstroke}%
\pgfsetdash{}{0pt}%
\pgfpathmoveto{\pgfqpoint{3.186131in}{2.954270in}}%
\pgfpathlineto{\pgfqpoint{1.914830in}{0.967594in}}%
\pgfpathlineto{\pgfqpoint{3.186131in}{2.954270in}}%
\pgfpathclose%
\pgfusepath{fill}%
\end{pgfscope}%
\begin{pgfscope}%
\pgfpathrectangle{\pgfqpoint{0.017500in}{0.017500in}}{\pgfqpoint{6.218282in}{3.580372in}} %
\pgfusepath{clip}%
\pgfsetbuttcap%
\pgfsetroundjoin%
\definecolor{currentfill}{rgb}{0.902849,0.844796,0.811970}%
\pgfsetfillcolor{currentfill}%
\pgfsetlinewidth{0.000000pt}%
\definecolor{currentstroke}{rgb}{0.000000,0.000000,0.000000}%
\pgfsetstrokecolor{currentstroke}%
\pgfsetdash{}{0pt}%
\pgfpathmoveto{\pgfqpoint{3.186131in}{2.954270in}}%
\pgfpathlineto{\pgfqpoint{1.914830in}{0.967594in}}%
\pgfpathlineto{\pgfqpoint{3.186131in}{2.954270in}}%
\pgfpathclose%
\pgfusepath{fill}%
\end{pgfscope}%
\begin{pgfscope}%
\pgfpathrectangle{\pgfqpoint{0.017500in}{0.017500in}}{\pgfqpoint{6.218282in}{3.580372in}} %
\pgfusepath{clip}%
\pgfsetbuttcap%
\pgfsetroundjoin%
\definecolor{currentfill}{rgb}{0.902849,0.844796,0.811970}%
\pgfsetfillcolor{currentfill}%
\pgfsetlinewidth{0.000000pt}%
\definecolor{currentstroke}{rgb}{0.000000,0.000000,0.000000}%
\pgfsetstrokecolor{currentstroke}%
\pgfsetdash{}{0pt}%
\pgfpathmoveto{\pgfqpoint{3.186131in}{2.954270in}}%
\pgfpathlineto{\pgfqpoint{1.914830in}{0.967594in}}%
\pgfpathlineto{\pgfqpoint{3.186131in}{2.954270in}}%
\pgfpathclose%
\pgfusepath{fill}%
\end{pgfscope}%
\begin{pgfscope}%
\pgfpathrectangle{\pgfqpoint{0.017500in}{0.017500in}}{\pgfqpoint{6.218282in}{3.580372in}} %
\pgfusepath{clip}%
\pgfsetbuttcap%
\pgfsetroundjoin%
\definecolor{currentfill}{rgb}{0.902849,0.844796,0.811970}%
\pgfsetfillcolor{currentfill}%
\pgfsetlinewidth{0.000000pt}%
\definecolor{currentstroke}{rgb}{0.000000,0.000000,0.000000}%
\pgfsetstrokecolor{currentstroke}%
\pgfsetdash{}{0pt}%
\pgfpathmoveto{\pgfqpoint{3.186131in}{2.954270in}}%
\pgfpathlineto{\pgfqpoint{1.914830in}{0.967594in}}%
\pgfpathlineto{\pgfqpoint{3.186131in}{2.954270in}}%
\pgfpathclose%
\pgfusepath{fill}%
\end{pgfscope}%
\begin{pgfscope}%
\pgfpathrectangle{\pgfqpoint{0.017500in}{0.017500in}}{\pgfqpoint{6.218282in}{3.580372in}} %
\pgfusepath{clip}%
\pgfsetbuttcap%
\pgfsetroundjoin%
\definecolor{currentfill}{rgb}{0.902849,0.844796,0.811970}%
\pgfsetfillcolor{currentfill}%
\pgfsetlinewidth{0.000000pt}%
\definecolor{currentstroke}{rgb}{0.000000,0.000000,0.000000}%
\pgfsetstrokecolor{currentstroke}%
\pgfsetdash{}{0pt}%
\pgfpathmoveto{\pgfqpoint{3.186131in}{2.954270in}}%
\pgfpathlineto{\pgfqpoint{1.914830in}{0.967594in}}%
\pgfpathlineto{\pgfqpoint{3.186131in}{2.954270in}}%
\pgfpathclose%
\pgfusepath{fill}%
\end{pgfscope}%
\begin{pgfscope}%
\pgfpathrectangle{\pgfqpoint{0.017500in}{0.017500in}}{\pgfqpoint{6.218282in}{3.580372in}} %
\pgfusepath{clip}%
\pgfsetbuttcap%
\pgfsetroundjoin%
\definecolor{currentfill}{rgb}{0.902849,0.844796,0.811970}%
\pgfsetfillcolor{currentfill}%
\pgfsetlinewidth{0.000000pt}%
\definecolor{currentstroke}{rgb}{0.000000,0.000000,0.000000}%
\pgfsetstrokecolor{currentstroke}%
\pgfsetdash{}{0pt}%
\pgfpathmoveto{\pgfqpoint{3.186131in}{2.954270in}}%
\pgfpathlineto{\pgfqpoint{1.914830in}{0.967594in}}%
\pgfpathlineto{\pgfqpoint{3.186131in}{2.954270in}}%
\pgfpathclose%
\pgfusepath{fill}%
\end{pgfscope}%
\begin{pgfscope}%
\pgfpathrectangle{\pgfqpoint{0.017500in}{0.017500in}}{\pgfqpoint{6.218282in}{3.580372in}} %
\pgfusepath{clip}%
\pgfsetbuttcap%
\pgfsetroundjoin%
\definecolor{currentfill}{rgb}{0.902849,0.844796,0.811970}%
\pgfsetfillcolor{currentfill}%
\pgfsetlinewidth{0.000000pt}%
\definecolor{currentstroke}{rgb}{0.000000,0.000000,0.000000}%
\pgfsetstrokecolor{currentstroke}%
\pgfsetdash{}{0pt}%
\pgfpathmoveto{\pgfqpoint{3.186131in}{2.954270in}}%
\pgfpathlineto{\pgfqpoint{1.914830in}{0.967594in}}%
\pgfpathlineto{\pgfqpoint{3.186131in}{2.954270in}}%
\pgfpathclose%
\pgfusepath{fill}%
\end{pgfscope}%
\begin{pgfscope}%
\pgfpathrectangle{\pgfqpoint{0.017500in}{0.017500in}}{\pgfqpoint{6.218282in}{3.580372in}} %
\pgfusepath{clip}%
\pgfsetbuttcap%
\pgfsetroundjoin%
\definecolor{currentfill}{rgb}{0.902849,0.844796,0.811970}%
\pgfsetfillcolor{currentfill}%
\pgfsetlinewidth{0.000000pt}%
\definecolor{currentstroke}{rgb}{0.000000,0.000000,0.000000}%
\pgfsetstrokecolor{currentstroke}%
\pgfsetdash{}{0pt}%
\pgfpathmoveto{\pgfqpoint{3.186131in}{2.954270in}}%
\pgfpathlineto{\pgfqpoint{1.914830in}{0.967594in}}%
\pgfpathlineto{\pgfqpoint{3.186131in}{2.954270in}}%
\pgfpathclose%
\pgfusepath{fill}%
\end{pgfscope}%
\begin{pgfscope}%
\pgfpathrectangle{\pgfqpoint{0.017500in}{0.017500in}}{\pgfqpoint{6.218282in}{3.580372in}} %
\pgfusepath{clip}%
\pgfsetbuttcap%
\pgfsetroundjoin%
\definecolor{currentfill}{rgb}{0.723315,0.068898,0.162989}%
\pgfsetfillcolor{currentfill}%
\pgfsetlinewidth{0.000000pt}%
\definecolor{currentstroke}{rgb}{0.000000,0.000000,0.000000}%
\pgfsetstrokecolor{currentstroke}%
\pgfsetdash{}{0pt}%
\pgfpathmoveto{\pgfqpoint{2.932968in}{2.820668in}}%
\pgfpathlineto{\pgfqpoint{3.186131in}{2.954270in}}%
\pgfpathlineto{\pgfqpoint{2.932968in}{2.820668in}}%
\pgfpathclose%
\pgfusepath{fill}%
\end{pgfscope}%
\begin{pgfscope}%
\pgfpathrectangle{\pgfqpoint{0.017500in}{0.017500in}}{\pgfqpoint{6.218282in}{3.580372in}} %
\pgfusepath{clip}%
\pgfsetbuttcap%
\pgfsetroundjoin%
\definecolor{currentfill}{rgb}{0.723315,0.068898,0.162989}%
\pgfsetfillcolor{currentfill}%
\pgfsetlinewidth{0.000000pt}%
\definecolor{currentstroke}{rgb}{0.000000,0.000000,0.000000}%
\pgfsetstrokecolor{currentstroke}%
\pgfsetdash{}{0pt}%
\pgfpathmoveto{\pgfqpoint{2.932968in}{2.820668in}}%
\pgfpathlineto{\pgfqpoint{3.186131in}{2.954270in}}%
\pgfpathlineto{\pgfqpoint{2.932968in}{2.820668in}}%
\pgfpathclose%
\pgfusepath{fill}%
\end{pgfscope}%
\begin{pgfscope}%
\pgfpathrectangle{\pgfqpoint{0.017500in}{0.017500in}}{\pgfqpoint{6.218282in}{3.580372in}} %
\pgfusepath{clip}%
\pgfsetbuttcap%
\pgfsetroundjoin%
\definecolor{currentfill}{rgb}{0.723315,0.068898,0.162989}%
\pgfsetfillcolor{currentfill}%
\pgfsetlinewidth{0.000000pt}%
\definecolor{currentstroke}{rgb}{0.000000,0.000000,0.000000}%
\pgfsetstrokecolor{currentstroke}%
\pgfsetdash{}{0pt}%
\pgfpathmoveto{\pgfqpoint{2.932968in}{2.820668in}}%
\pgfpathlineto{\pgfqpoint{3.186131in}{2.954270in}}%
\pgfpathlineto{\pgfqpoint{2.932968in}{2.820668in}}%
\pgfpathclose%
\pgfusepath{fill}%
\end{pgfscope}%
\begin{pgfscope}%
\pgfpathrectangle{\pgfqpoint{0.017500in}{0.017500in}}{\pgfqpoint{6.218282in}{3.580372in}} %
\pgfusepath{clip}%
\pgfsetbuttcap%
\pgfsetroundjoin%
\definecolor{currentfill}{rgb}{0.723315,0.068898,0.162989}%
\pgfsetfillcolor{currentfill}%
\pgfsetlinewidth{0.000000pt}%
\definecolor{currentstroke}{rgb}{0.000000,0.000000,0.000000}%
\pgfsetstrokecolor{currentstroke}%
\pgfsetdash{}{0pt}%
\pgfpathmoveto{\pgfqpoint{2.932968in}{2.820668in}}%
\pgfpathlineto{\pgfqpoint{3.186131in}{2.954270in}}%
\pgfpathlineto{\pgfqpoint{2.932968in}{2.820668in}}%
\pgfpathclose%
\pgfusepath{fill}%
\end{pgfscope}%
\begin{pgfscope}%
\pgfpathrectangle{\pgfqpoint{0.017500in}{0.017500in}}{\pgfqpoint{6.218282in}{3.580372in}} %
\pgfusepath{clip}%
\pgfsetbuttcap%
\pgfsetroundjoin%
\definecolor{currentfill}{rgb}{0.723315,0.068898,0.162989}%
\pgfsetfillcolor{currentfill}%
\pgfsetlinewidth{0.000000pt}%
\definecolor{currentstroke}{rgb}{0.000000,0.000000,0.000000}%
\pgfsetstrokecolor{currentstroke}%
\pgfsetdash{}{0pt}%
\pgfpathmoveto{\pgfqpoint{2.932968in}{2.820668in}}%
\pgfpathlineto{\pgfqpoint{3.186131in}{2.954270in}}%
\pgfpathlineto{\pgfqpoint{2.932968in}{2.820668in}}%
\pgfpathclose%
\pgfusepath{fill}%
\end{pgfscope}%
\begin{pgfscope}%
\pgfpathrectangle{\pgfqpoint{0.017500in}{0.017500in}}{\pgfqpoint{6.218282in}{3.580372in}} %
\pgfusepath{clip}%
\pgfsetbuttcap%
\pgfsetroundjoin%
\definecolor{currentfill}{rgb}{0.723315,0.068898,0.162989}%
\pgfsetfillcolor{currentfill}%
\pgfsetlinewidth{0.000000pt}%
\definecolor{currentstroke}{rgb}{0.000000,0.000000,0.000000}%
\pgfsetstrokecolor{currentstroke}%
\pgfsetdash{}{0pt}%
\pgfpathmoveto{\pgfqpoint{2.932968in}{2.820668in}}%
\pgfpathlineto{\pgfqpoint{3.186131in}{2.954270in}}%
\pgfpathlineto{\pgfqpoint{2.932968in}{2.820668in}}%
\pgfpathclose%
\pgfusepath{fill}%
\end{pgfscope}%
\begin{pgfscope}%
\pgfpathrectangle{\pgfqpoint{0.017500in}{0.017500in}}{\pgfqpoint{6.218282in}{3.580372in}} %
\pgfusepath{clip}%
\pgfsetbuttcap%
\pgfsetroundjoin%
\definecolor{currentfill}{rgb}{0.723315,0.068898,0.162989}%
\pgfsetfillcolor{currentfill}%
\pgfsetlinewidth{0.000000pt}%
\definecolor{currentstroke}{rgb}{0.000000,0.000000,0.000000}%
\pgfsetstrokecolor{currentstroke}%
\pgfsetdash{}{0pt}%
\pgfpathmoveto{\pgfqpoint{2.932968in}{2.820668in}}%
\pgfpathlineto{\pgfqpoint{3.186131in}{2.954270in}}%
\pgfpathlineto{\pgfqpoint{2.932968in}{2.820668in}}%
\pgfpathclose%
\pgfusepath{fill}%
\end{pgfscope}%
\begin{pgfscope}%
\pgfpathrectangle{\pgfqpoint{0.017500in}{0.017500in}}{\pgfqpoint{6.218282in}{3.580372in}} %
\pgfusepath{clip}%
\pgfsetbuttcap%
\pgfsetroundjoin%
\definecolor{currentfill}{rgb}{0.723315,0.068898,0.162989}%
\pgfsetfillcolor{currentfill}%
\pgfsetlinewidth{0.000000pt}%
\definecolor{currentstroke}{rgb}{0.000000,0.000000,0.000000}%
\pgfsetstrokecolor{currentstroke}%
\pgfsetdash{}{0pt}%
\pgfpathmoveto{\pgfqpoint{2.932968in}{2.820668in}}%
\pgfpathlineto{\pgfqpoint{3.186131in}{2.954270in}}%
\pgfpathlineto{\pgfqpoint{2.932968in}{2.820668in}}%
\pgfpathclose%
\pgfusepath{fill}%
\end{pgfscope}%
\begin{pgfscope}%
\pgfpathrectangle{\pgfqpoint{0.017500in}{0.017500in}}{\pgfqpoint{6.218282in}{3.580372in}} %
\pgfusepath{clip}%
\pgfsetbuttcap%
\pgfsetroundjoin%
\definecolor{currentfill}{rgb}{0.723315,0.068898,0.162989}%
\pgfsetfillcolor{currentfill}%
\pgfsetlinewidth{0.000000pt}%
\definecolor{currentstroke}{rgb}{0.000000,0.000000,0.000000}%
\pgfsetstrokecolor{currentstroke}%
\pgfsetdash{}{0pt}%
\pgfpathmoveto{\pgfqpoint{2.932968in}{2.820668in}}%
\pgfpathlineto{\pgfqpoint{3.186131in}{2.954270in}}%
\pgfpathlineto{\pgfqpoint{2.932968in}{2.820668in}}%
\pgfpathclose%
\pgfusepath{fill}%
\end{pgfscope}%
\begin{pgfscope}%
\pgfpathrectangle{\pgfqpoint{0.017500in}{0.017500in}}{\pgfqpoint{6.218282in}{3.580372in}} %
\pgfusepath{clip}%
\pgfsetbuttcap%
\pgfsetroundjoin%
\definecolor{currentfill}{rgb}{0.723315,0.068898,0.162989}%
\pgfsetfillcolor{currentfill}%
\pgfsetlinewidth{0.000000pt}%
\definecolor{currentstroke}{rgb}{0.000000,0.000000,0.000000}%
\pgfsetstrokecolor{currentstroke}%
\pgfsetdash{}{0pt}%
\pgfpathmoveto{\pgfqpoint{2.932968in}{2.820668in}}%
\pgfpathlineto{\pgfqpoint{3.186131in}{2.954270in}}%
\pgfpathlineto{\pgfqpoint{2.932968in}{2.820668in}}%
\pgfpathclose%
\pgfusepath{fill}%
\end{pgfscope}%
\begin{pgfscope}%
\pgfpathrectangle{\pgfqpoint{0.017500in}{0.017500in}}{\pgfqpoint{6.218282in}{3.580372in}} %
\pgfusepath{clip}%
\pgfsetbuttcap%
\pgfsetroundjoin%
\definecolor{currentfill}{rgb}{0.723315,0.068898,0.162989}%
\pgfsetfillcolor{currentfill}%
\pgfsetlinewidth{0.000000pt}%
\definecolor{currentstroke}{rgb}{0.000000,0.000000,0.000000}%
\pgfsetstrokecolor{currentstroke}%
\pgfsetdash{}{0pt}%
\pgfpathmoveto{\pgfqpoint{2.932968in}{2.820668in}}%
\pgfpathlineto{\pgfqpoint{3.186131in}{2.954270in}}%
\pgfpathlineto{\pgfqpoint{2.932968in}{2.820668in}}%
\pgfpathclose%
\pgfusepath{fill}%
\end{pgfscope}%
\begin{pgfscope}%
\pgfpathrectangle{\pgfqpoint{0.017500in}{0.017500in}}{\pgfqpoint{6.218282in}{3.580372in}} %
\pgfusepath{clip}%
\pgfsetbuttcap%
\pgfsetroundjoin%
\definecolor{currentfill}{rgb}{0.723315,0.068898,0.162989}%
\pgfsetfillcolor{currentfill}%
\pgfsetlinewidth{0.000000pt}%
\definecolor{currentstroke}{rgb}{0.000000,0.000000,0.000000}%
\pgfsetstrokecolor{currentstroke}%
\pgfsetdash{}{0pt}%
\pgfpathmoveto{\pgfqpoint{2.932968in}{2.820668in}}%
\pgfpathlineto{\pgfqpoint{3.186131in}{2.954270in}}%
\pgfpathlineto{\pgfqpoint{2.932968in}{2.820668in}}%
\pgfpathclose%
\pgfusepath{fill}%
\end{pgfscope}%
\begin{pgfscope}%
\pgfpathrectangle{\pgfqpoint{0.017500in}{0.017500in}}{\pgfqpoint{6.218282in}{3.580372in}} %
\pgfusepath{clip}%
\pgfsetbuttcap%
\pgfsetroundjoin%
\definecolor{currentfill}{rgb}{0.723315,0.068898,0.162989}%
\pgfsetfillcolor{currentfill}%
\pgfsetlinewidth{0.000000pt}%
\definecolor{currentstroke}{rgb}{0.000000,0.000000,0.000000}%
\pgfsetstrokecolor{currentstroke}%
\pgfsetdash{}{0pt}%
\pgfpathmoveto{\pgfqpoint{2.932968in}{2.820668in}}%
\pgfpathlineto{\pgfqpoint{3.186131in}{2.954270in}}%
\pgfpathlineto{\pgfqpoint{2.932968in}{2.820668in}}%
\pgfpathclose%
\pgfusepath{fill}%
\end{pgfscope}%
\begin{pgfscope}%
\pgfpathrectangle{\pgfqpoint{0.017500in}{0.017500in}}{\pgfqpoint{6.218282in}{3.580372in}} %
\pgfusepath{clip}%
\pgfsetbuttcap%
\pgfsetroundjoin%
\definecolor{currentfill}{rgb}{0.723315,0.068898,0.162989}%
\pgfsetfillcolor{currentfill}%
\pgfsetlinewidth{0.000000pt}%
\definecolor{currentstroke}{rgb}{0.000000,0.000000,0.000000}%
\pgfsetstrokecolor{currentstroke}%
\pgfsetdash{}{0pt}%
\pgfpathmoveto{\pgfqpoint{2.932968in}{2.820668in}}%
\pgfpathlineto{\pgfqpoint{3.186131in}{2.954270in}}%
\pgfpathlineto{\pgfqpoint{2.932968in}{2.820668in}}%
\pgfpathclose%
\pgfusepath{fill}%
\end{pgfscope}%
\begin{pgfscope}%
\pgfpathrectangle{\pgfqpoint{0.017500in}{0.017500in}}{\pgfqpoint{6.218282in}{3.580372in}} %
\pgfusepath{clip}%
\pgfsetbuttcap%
\pgfsetroundjoin%
\definecolor{currentfill}{rgb}{0.723315,0.068898,0.162989}%
\pgfsetfillcolor{currentfill}%
\pgfsetlinewidth{0.000000pt}%
\definecolor{currentstroke}{rgb}{0.000000,0.000000,0.000000}%
\pgfsetstrokecolor{currentstroke}%
\pgfsetdash{}{0pt}%
\pgfpathmoveto{\pgfqpoint{2.932968in}{2.820668in}}%
\pgfpathlineto{\pgfqpoint{3.186131in}{2.954270in}}%
\pgfpathlineto{\pgfqpoint{2.932968in}{2.820668in}}%
\pgfpathclose%
\pgfusepath{fill}%
\end{pgfscope}%
\begin{pgfscope}%
\pgfpathrectangle{\pgfqpoint{0.017500in}{0.017500in}}{\pgfqpoint{6.218282in}{3.580372in}} %
\pgfusepath{clip}%
\pgfsetbuttcap%
\pgfsetroundjoin%
\definecolor{currentfill}{rgb}{0.723315,0.068898,0.162989}%
\pgfsetfillcolor{currentfill}%
\pgfsetlinewidth{0.000000pt}%
\definecolor{currentstroke}{rgb}{0.000000,0.000000,0.000000}%
\pgfsetstrokecolor{currentstroke}%
\pgfsetdash{}{0pt}%
\pgfpathmoveto{\pgfqpoint{2.932968in}{2.820668in}}%
\pgfpathlineto{\pgfqpoint{3.186131in}{2.954270in}}%
\pgfpathlineto{\pgfqpoint{2.932968in}{2.820668in}}%
\pgfpathclose%
\pgfusepath{fill}%
\end{pgfscope}%
\begin{pgfscope}%
\pgfpathrectangle{\pgfqpoint{0.017500in}{0.017500in}}{\pgfqpoint{6.218282in}{3.580372in}} %
\pgfusepath{clip}%
\pgfsetbuttcap%
\pgfsetroundjoin%
\definecolor{currentfill}{rgb}{0.723315,0.068898,0.162989}%
\pgfsetfillcolor{currentfill}%
\pgfsetlinewidth{0.000000pt}%
\definecolor{currentstroke}{rgb}{0.000000,0.000000,0.000000}%
\pgfsetstrokecolor{currentstroke}%
\pgfsetdash{}{0pt}%
\pgfpathmoveto{\pgfqpoint{2.932968in}{2.820668in}}%
\pgfpathlineto{\pgfqpoint{3.186131in}{2.954270in}}%
\pgfpathlineto{\pgfqpoint{2.932968in}{2.820668in}}%
\pgfpathclose%
\pgfusepath{fill}%
\end{pgfscope}%
\begin{pgfscope}%
\pgfpathrectangle{\pgfqpoint{0.017500in}{0.017500in}}{\pgfqpoint{6.218282in}{3.580372in}} %
\pgfusepath{clip}%
\pgfsetbuttcap%
\pgfsetroundjoin%
\definecolor{currentfill}{rgb}{0.723315,0.068898,0.162989}%
\pgfsetfillcolor{currentfill}%
\pgfsetlinewidth{0.000000pt}%
\definecolor{currentstroke}{rgb}{0.000000,0.000000,0.000000}%
\pgfsetstrokecolor{currentstroke}%
\pgfsetdash{}{0pt}%
\pgfpathmoveto{\pgfqpoint{2.932968in}{2.820668in}}%
\pgfpathlineto{\pgfqpoint{3.186131in}{2.954270in}}%
\pgfpathlineto{\pgfqpoint{2.932968in}{2.820668in}}%
\pgfpathclose%
\pgfusepath{fill}%
\end{pgfscope}%
\begin{pgfscope}%
\pgfpathrectangle{\pgfqpoint{0.017500in}{0.017500in}}{\pgfqpoint{6.218282in}{3.580372in}} %
\pgfusepath{clip}%
\pgfsetbuttcap%
\pgfsetroundjoin%
\definecolor{currentfill}{rgb}{0.723315,0.068898,0.162989}%
\pgfsetfillcolor{currentfill}%
\pgfsetlinewidth{0.000000pt}%
\definecolor{currentstroke}{rgb}{0.000000,0.000000,0.000000}%
\pgfsetstrokecolor{currentstroke}%
\pgfsetdash{}{0pt}%
\pgfpathmoveto{\pgfqpoint{2.932968in}{2.820668in}}%
\pgfpathlineto{\pgfqpoint{3.186131in}{2.954270in}}%
\pgfpathlineto{\pgfqpoint{2.932968in}{2.820668in}}%
\pgfpathclose%
\pgfusepath{fill}%
\end{pgfscope}%
\begin{pgfscope}%
\pgfpathrectangle{\pgfqpoint{0.017500in}{0.017500in}}{\pgfqpoint{6.218282in}{3.580372in}} %
\pgfusepath{clip}%
\pgfsetbuttcap%
\pgfsetroundjoin%
\definecolor{currentfill}{rgb}{0.723315,0.068898,0.162989}%
\pgfsetfillcolor{currentfill}%
\pgfsetlinewidth{0.000000pt}%
\definecolor{currentstroke}{rgb}{0.000000,0.000000,0.000000}%
\pgfsetstrokecolor{currentstroke}%
\pgfsetdash{}{0pt}%
\pgfpathmoveto{\pgfqpoint{2.932968in}{2.820668in}}%
\pgfpathlineto{\pgfqpoint{3.186131in}{2.954270in}}%
\pgfpathlineto{\pgfqpoint{2.932968in}{2.820668in}}%
\pgfpathclose%
\pgfusepath{fill}%
\end{pgfscope}%
\begin{pgfscope}%
\pgfpathrectangle{\pgfqpoint{0.017500in}{0.017500in}}{\pgfqpoint{6.218282in}{3.580372in}} %
\pgfusepath{clip}%
\pgfsetbuttcap%
\pgfsetroundjoin%
\definecolor{currentfill}{rgb}{0.723315,0.068898,0.162989}%
\pgfsetfillcolor{currentfill}%
\pgfsetlinewidth{0.000000pt}%
\definecolor{currentstroke}{rgb}{0.000000,0.000000,0.000000}%
\pgfsetstrokecolor{currentstroke}%
\pgfsetdash{}{0pt}%
\pgfpathmoveto{\pgfqpoint{2.932968in}{2.820668in}}%
\pgfpathlineto{\pgfqpoint{3.186131in}{2.954270in}}%
\pgfpathlineto{\pgfqpoint{2.932968in}{2.820668in}}%
\pgfpathclose%
\pgfusepath{fill}%
\end{pgfscope}%
\begin{pgfscope}%
\pgfpathrectangle{\pgfqpoint{0.017500in}{0.017500in}}{\pgfqpoint{6.218282in}{3.580372in}} %
\pgfusepath{clip}%
\pgfsetbuttcap%
\pgfsetroundjoin%
\definecolor{currentfill}{rgb}{0.723315,0.068898,0.162989}%
\pgfsetfillcolor{currentfill}%
\pgfsetlinewidth{0.000000pt}%
\definecolor{currentstroke}{rgb}{0.000000,0.000000,0.000000}%
\pgfsetstrokecolor{currentstroke}%
\pgfsetdash{}{0pt}%
\pgfpathmoveto{\pgfqpoint{2.932968in}{2.820668in}}%
\pgfpathlineto{\pgfqpoint{3.186131in}{2.954270in}}%
\pgfpathlineto{\pgfqpoint{2.932968in}{2.820668in}}%
\pgfpathclose%
\pgfusepath{fill}%
\end{pgfscope}%
\begin{pgfscope}%
\pgfpathrectangle{\pgfqpoint{0.017500in}{0.017500in}}{\pgfqpoint{6.218282in}{3.580372in}} %
\pgfusepath{clip}%
\pgfsetbuttcap%
\pgfsetroundjoin%
\definecolor{currentfill}{rgb}{0.723315,0.068898,0.162989}%
\pgfsetfillcolor{currentfill}%
\pgfsetlinewidth{0.000000pt}%
\definecolor{currentstroke}{rgb}{0.000000,0.000000,0.000000}%
\pgfsetstrokecolor{currentstroke}%
\pgfsetdash{}{0pt}%
\pgfpathmoveto{\pgfqpoint{2.932968in}{2.820668in}}%
\pgfpathlineto{\pgfqpoint{3.186131in}{2.954270in}}%
\pgfpathlineto{\pgfqpoint{2.932968in}{2.820668in}}%
\pgfpathclose%
\pgfusepath{fill}%
\end{pgfscope}%
\begin{pgfscope}%
\pgfpathrectangle{\pgfqpoint{0.017500in}{0.017500in}}{\pgfqpoint{6.218282in}{3.580372in}} %
\pgfusepath{clip}%
\pgfsetbuttcap%
\pgfsetroundjoin%
\definecolor{currentfill}{rgb}{0.723315,0.068898,0.162989}%
\pgfsetfillcolor{currentfill}%
\pgfsetlinewidth{0.000000pt}%
\definecolor{currentstroke}{rgb}{0.000000,0.000000,0.000000}%
\pgfsetstrokecolor{currentstroke}%
\pgfsetdash{}{0pt}%
\pgfpathmoveto{\pgfqpoint{2.932968in}{2.820668in}}%
\pgfpathlineto{\pgfqpoint{3.186131in}{2.954270in}}%
\pgfpathlineto{\pgfqpoint{2.932968in}{2.820668in}}%
\pgfpathclose%
\pgfusepath{fill}%
\end{pgfscope}%
\begin{pgfscope}%
\pgfpathrectangle{\pgfqpoint{0.017500in}{0.017500in}}{\pgfqpoint{6.218282in}{3.580372in}} %
\pgfusepath{clip}%
\pgfsetbuttcap%
\pgfsetroundjoin%
\definecolor{currentfill}{rgb}{0.723315,0.068898,0.162989}%
\pgfsetfillcolor{currentfill}%
\pgfsetlinewidth{0.000000pt}%
\definecolor{currentstroke}{rgb}{0.000000,0.000000,0.000000}%
\pgfsetstrokecolor{currentstroke}%
\pgfsetdash{}{0pt}%
\pgfpathmoveto{\pgfqpoint{2.932968in}{2.820668in}}%
\pgfpathlineto{\pgfqpoint{3.186131in}{2.954270in}}%
\pgfpathlineto{\pgfqpoint{2.932968in}{2.820668in}}%
\pgfpathclose%
\pgfusepath{fill}%
\end{pgfscope}%
\begin{pgfscope}%
\pgfpathrectangle{\pgfqpoint{0.017500in}{0.017500in}}{\pgfqpoint{6.218282in}{3.580372in}} %
\pgfusepath{clip}%
\pgfsetbuttcap%
\pgfsetroundjoin%
\definecolor{currentfill}{rgb}{0.723315,0.068898,0.162989}%
\pgfsetfillcolor{currentfill}%
\pgfsetlinewidth{0.000000pt}%
\definecolor{currentstroke}{rgb}{0.000000,0.000000,0.000000}%
\pgfsetstrokecolor{currentstroke}%
\pgfsetdash{}{0pt}%
\pgfpathmoveto{\pgfqpoint{2.932968in}{2.820668in}}%
\pgfpathlineto{\pgfqpoint{3.186131in}{2.954270in}}%
\pgfpathlineto{\pgfqpoint{2.932968in}{2.820668in}}%
\pgfpathclose%
\pgfusepath{fill}%
\end{pgfscope}%
\begin{pgfscope}%
\pgfpathrectangle{\pgfqpoint{0.017500in}{0.017500in}}{\pgfqpoint{6.218282in}{3.580372in}} %
\pgfusepath{clip}%
\pgfsetbuttcap%
\pgfsetroundjoin%
\definecolor{currentfill}{rgb}{0.723315,0.068898,0.162989}%
\pgfsetfillcolor{currentfill}%
\pgfsetlinewidth{0.000000pt}%
\definecolor{currentstroke}{rgb}{0.000000,0.000000,0.000000}%
\pgfsetstrokecolor{currentstroke}%
\pgfsetdash{}{0pt}%
\pgfpathmoveto{\pgfqpoint{2.932968in}{2.820668in}}%
\pgfpathlineto{\pgfqpoint{3.186131in}{2.954270in}}%
\pgfpathlineto{\pgfqpoint{2.932968in}{2.820668in}}%
\pgfpathclose%
\pgfusepath{fill}%
\end{pgfscope}%
\begin{pgfscope}%
\pgfpathrectangle{\pgfqpoint{0.017500in}{0.017500in}}{\pgfqpoint{6.218282in}{3.580372in}} %
\pgfusepath{clip}%
\pgfsetbuttcap%
\pgfsetroundjoin%
\definecolor{currentfill}{rgb}{0.723315,0.068898,0.162989}%
\pgfsetfillcolor{currentfill}%
\pgfsetlinewidth{0.000000pt}%
\definecolor{currentstroke}{rgb}{0.000000,0.000000,0.000000}%
\pgfsetstrokecolor{currentstroke}%
\pgfsetdash{}{0pt}%
\pgfpathmoveto{\pgfqpoint{2.932968in}{2.820668in}}%
\pgfpathlineto{\pgfqpoint{3.186131in}{2.954270in}}%
\pgfpathlineto{\pgfqpoint{2.932968in}{2.820668in}}%
\pgfpathclose%
\pgfusepath{fill}%
\end{pgfscope}%
\begin{pgfscope}%
\pgfpathrectangle{\pgfqpoint{0.017500in}{0.017500in}}{\pgfqpoint{6.218282in}{3.580372in}} %
\pgfusepath{clip}%
\pgfsetbuttcap%
\pgfsetroundjoin%
\definecolor{currentfill}{rgb}{0.723315,0.068898,0.162989}%
\pgfsetfillcolor{currentfill}%
\pgfsetlinewidth{0.000000pt}%
\definecolor{currentstroke}{rgb}{0.000000,0.000000,0.000000}%
\pgfsetstrokecolor{currentstroke}%
\pgfsetdash{}{0pt}%
\pgfpathmoveto{\pgfqpoint{2.932968in}{2.820668in}}%
\pgfpathlineto{\pgfqpoint{3.186131in}{2.954270in}}%
\pgfpathlineto{\pgfqpoint{2.932968in}{2.820668in}}%
\pgfpathclose%
\pgfusepath{fill}%
\end{pgfscope}%
\begin{pgfscope}%
\pgfpathrectangle{\pgfqpoint{0.017500in}{0.017500in}}{\pgfqpoint{6.218282in}{3.580372in}} %
\pgfusepath{clip}%
\pgfsetbuttcap%
\pgfsetroundjoin%
\definecolor{currentfill}{rgb}{0.723315,0.068898,0.162989}%
\pgfsetfillcolor{currentfill}%
\pgfsetlinewidth{0.000000pt}%
\definecolor{currentstroke}{rgb}{0.000000,0.000000,0.000000}%
\pgfsetstrokecolor{currentstroke}%
\pgfsetdash{}{0pt}%
\pgfpathmoveto{\pgfqpoint{2.932968in}{2.820668in}}%
\pgfpathlineto{\pgfqpoint{3.186131in}{2.954270in}}%
\pgfpathlineto{\pgfqpoint{2.932968in}{2.820668in}}%
\pgfpathclose%
\pgfusepath{fill}%
\end{pgfscope}%
\begin{pgfscope}%
\pgfpathrectangle{\pgfqpoint{0.017500in}{0.017500in}}{\pgfqpoint{6.218282in}{3.580372in}} %
\pgfusepath{clip}%
\pgfsetbuttcap%
\pgfsetroundjoin%
\definecolor{currentfill}{rgb}{0.804965,0.851666,0.926165}%
\pgfsetfillcolor{currentfill}%
\pgfsetlinewidth{0.000000pt}%
\definecolor{currentstroke}{rgb}{0.000000,0.000000,0.000000}%
\pgfsetstrokecolor{currentstroke}%
\pgfsetdash{}{0pt}%
\pgfpathmoveto{\pgfqpoint{3.186131in}{2.954270in}}%
\pgfpathlineto{\pgfqpoint{3.622205in}{2.891176in}}%
\pgfpathlineto{\pgfqpoint{2.349853in}{0.872712in}}%
\pgfpathlineto{\pgfqpoint{1.914830in}{0.967594in}}%
\pgfpathlineto{\pgfqpoint{3.186131in}{2.954270in}}%
\pgfpathclose%
\pgfusepath{fill}%
\end{pgfscope}%
\begin{pgfscope}%
\pgfpathrectangle{\pgfqpoint{0.017500in}{0.017500in}}{\pgfqpoint{6.218282in}{3.580372in}} %
\pgfusepath{clip}%
\pgfsetbuttcap%
\pgfsetroundjoin%
\definecolor{currentfill}{rgb}{0.804965,0.851666,0.926165}%
\pgfsetfillcolor{currentfill}%
\pgfsetlinewidth{0.000000pt}%
\definecolor{currentstroke}{rgb}{0.000000,0.000000,0.000000}%
\pgfsetstrokecolor{currentstroke}%
\pgfsetdash{}{0pt}%
\pgfpathmoveto{\pgfqpoint{3.186131in}{2.954270in}}%
\pgfpathlineto{\pgfqpoint{3.622205in}{2.891176in}}%
\pgfpathlineto{\pgfqpoint{2.349853in}{0.872712in}}%
\pgfpathlineto{\pgfqpoint{1.914830in}{0.967594in}}%
\pgfpathlineto{\pgfqpoint{3.186131in}{2.954270in}}%
\pgfpathclose%
\pgfusepath{fill}%
\end{pgfscope}%
\begin{pgfscope}%
\pgfpathrectangle{\pgfqpoint{0.017500in}{0.017500in}}{\pgfqpoint{6.218282in}{3.580372in}} %
\pgfusepath{clip}%
\pgfsetbuttcap%
\pgfsetroundjoin%
\definecolor{currentfill}{rgb}{0.804965,0.851666,0.926165}%
\pgfsetfillcolor{currentfill}%
\pgfsetlinewidth{0.000000pt}%
\definecolor{currentstroke}{rgb}{0.000000,0.000000,0.000000}%
\pgfsetstrokecolor{currentstroke}%
\pgfsetdash{}{0pt}%
\pgfpathmoveto{\pgfqpoint{3.186131in}{2.954270in}}%
\pgfpathlineto{\pgfqpoint{3.622205in}{2.891176in}}%
\pgfpathlineto{\pgfqpoint{2.349853in}{0.872712in}}%
\pgfpathlineto{\pgfqpoint{1.914830in}{0.967594in}}%
\pgfpathlineto{\pgfqpoint{3.186131in}{2.954270in}}%
\pgfpathclose%
\pgfusepath{fill}%
\end{pgfscope}%
\begin{pgfscope}%
\pgfpathrectangle{\pgfqpoint{0.017500in}{0.017500in}}{\pgfqpoint{6.218282in}{3.580372in}} %
\pgfusepath{clip}%
\pgfsetbuttcap%
\pgfsetroundjoin%
\definecolor{currentfill}{rgb}{0.804965,0.851666,0.926165}%
\pgfsetfillcolor{currentfill}%
\pgfsetlinewidth{0.000000pt}%
\definecolor{currentstroke}{rgb}{0.000000,0.000000,0.000000}%
\pgfsetstrokecolor{currentstroke}%
\pgfsetdash{}{0pt}%
\pgfpathmoveto{\pgfqpoint{3.186131in}{2.954270in}}%
\pgfpathlineto{\pgfqpoint{3.622205in}{2.891176in}}%
\pgfpathlineto{\pgfqpoint{2.349853in}{0.872712in}}%
\pgfpathlineto{\pgfqpoint{1.914830in}{0.967594in}}%
\pgfpathlineto{\pgfqpoint{3.186131in}{2.954270in}}%
\pgfpathclose%
\pgfusepath{fill}%
\end{pgfscope}%
\begin{pgfscope}%
\pgfpathrectangle{\pgfqpoint{0.017500in}{0.017500in}}{\pgfqpoint{6.218282in}{3.580372in}} %
\pgfusepath{clip}%
\pgfsetbuttcap%
\pgfsetroundjoin%
\definecolor{currentfill}{rgb}{0.804965,0.851666,0.926165}%
\pgfsetfillcolor{currentfill}%
\pgfsetlinewidth{0.000000pt}%
\definecolor{currentstroke}{rgb}{0.000000,0.000000,0.000000}%
\pgfsetstrokecolor{currentstroke}%
\pgfsetdash{}{0pt}%
\pgfpathmoveto{\pgfqpoint{3.186131in}{2.954270in}}%
\pgfpathlineto{\pgfqpoint{3.622205in}{2.891176in}}%
\pgfpathlineto{\pgfqpoint{2.349853in}{0.872712in}}%
\pgfpathlineto{\pgfqpoint{1.914830in}{0.967594in}}%
\pgfpathlineto{\pgfqpoint{3.186131in}{2.954270in}}%
\pgfpathclose%
\pgfusepath{fill}%
\end{pgfscope}%
\begin{pgfscope}%
\pgfpathrectangle{\pgfqpoint{0.017500in}{0.017500in}}{\pgfqpoint{6.218282in}{3.580372in}} %
\pgfusepath{clip}%
\pgfsetbuttcap%
\pgfsetroundjoin%
\definecolor{currentfill}{rgb}{0.717435,0.051118,0.158737}%
\pgfsetfillcolor{currentfill}%
\pgfsetlinewidth{0.000000pt}%
\definecolor{currentstroke}{rgb}{0.000000,0.000000,0.000000}%
\pgfsetstrokecolor{currentstroke}%
\pgfsetdash{}{0pt}%
\pgfpathmoveto{\pgfqpoint{2.932968in}{2.820668in}}%
\pgfpathlineto{\pgfqpoint{3.372093in}{2.761754in}}%
\pgfpathlineto{\pgfqpoint{3.622205in}{2.891176in}}%
\pgfpathlineto{\pgfqpoint{3.186131in}{2.954270in}}%
\pgfpathlineto{\pgfqpoint{2.932968in}{2.820668in}}%
\pgfpathclose%
\pgfusepath{fill}%
\end{pgfscope}%
\begin{pgfscope}%
\pgfpathrectangle{\pgfqpoint{0.017500in}{0.017500in}}{\pgfqpoint{6.218282in}{3.580372in}} %
\pgfusepath{clip}%
\pgfsetbuttcap%
\pgfsetroundjoin%
\definecolor{currentfill}{rgb}{0.717435,0.051118,0.158737}%
\pgfsetfillcolor{currentfill}%
\pgfsetlinewidth{0.000000pt}%
\definecolor{currentstroke}{rgb}{0.000000,0.000000,0.000000}%
\pgfsetstrokecolor{currentstroke}%
\pgfsetdash{}{0pt}%
\pgfpathmoveto{\pgfqpoint{2.932968in}{2.820668in}}%
\pgfpathlineto{\pgfqpoint{3.372093in}{2.761754in}}%
\pgfpathlineto{\pgfqpoint{3.622205in}{2.891176in}}%
\pgfpathlineto{\pgfqpoint{3.186131in}{2.954270in}}%
\pgfpathlineto{\pgfqpoint{2.932968in}{2.820668in}}%
\pgfpathclose%
\pgfusepath{fill}%
\end{pgfscope}%
\begin{pgfscope}%
\pgfpathrectangle{\pgfqpoint{0.017500in}{0.017500in}}{\pgfqpoint{6.218282in}{3.580372in}} %
\pgfusepath{clip}%
\pgfsetbuttcap%
\pgfsetroundjoin%
\definecolor{currentfill}{rgb}{0.717435,0.051118,0.158737}%
\pgfsetfillcolor{currentfill}%
\pgfsetlinewidth{0.000000pt}%
\definecolor{currentstroke}{rgb}{0.000000,0.000000,0.000000}%
\pgfsetstrokecolor{currentstroke}%
\pgfsetdash{}{0pt}%
\pgfpathmoveto{\pgfqpoint{2.932968in}{2.820668in}}%
\pgfpathlineto{\pgfqpoint{3.372093in}{2.761754in}}%
\pgfpathlineto{\pgfqpoint{3.622205in}{2.891176in}}%
\pgfpathlineto{\pgfqpoint{3.186131in}{2.954270in}}%
\pgfpathlineto{\pgfqpoint{2.932968in}{2.820668in}}%
\pgfpathclose%
\pgfusepath{fill}%
\end{pgfscope}%
\begin{pgfscope}%
\pgfpathrectangle{\pgfqpoint{0.017500in}{0.017500in}}{\pgfqpoint{6.218282in}{3.580372in}} %
\pgfusepath{clip}%
\pgfsetbuttcap%
\pgfsetroundjoin%
\definecolor{currentfill}{rgb}{0.717435,0.051118,0.158737}%
\pgfsetfillcolor{currentfill}%
\pgfsetlinewidth{0.000000pt}%
\definecolor{currentstroke}{rgb}{0.000000,0.000000,0.000000}%
\pgfsetstrokecolor{currentstroke}%
\pgfsetdash{}{0pt}%
\pgfpathmoveto{\pgfqpoint{2.932968in}{2.820668in}}%
\pgfpathlineto{\pgfqpoint{3.372093in}{2.761754in}}%
\pgfpathlineto{\pgfqpoint{3.622205in}{2.891176in}}%
\pgfpathlineto{\pgfqpoint{3.186131in}{2.954270in}}%
\pgfpathlineto{\pgfqpoint{2.932968in}{2.820668in}}%
\pgfpathclose%
\pgfusepath{fill}%
\end{pgfscope}%
\begin{pgfscope}%
\pgfpathrectangle{\pgfqpoint{0.017500in}{0.017500in}}{\pgfqpoint{6.218282in}{3.580372in}} %
\pgfusepath{clip}%
\pgfsetbuttcap%
\pgfsetroundjoin%
\definecolor{currentfill}{rgb}{0.717435,0.051118,0.158737}%
\pgfsetfillcolor{currentfill}%
\pgfsetlinewidth{0.000000pt}%
\definecolor{currentstroke}{rgb}{0.000000,0.000000,0.000000}%
\pgfsetstrokecolor{currentstroke}%
\pgfsetdash{}{0pt}%
\pgfpathmoveto{\pgfqpoint{2.932968in}{2.820668in}}%
\pgfpathlineto{\pgfqpoint{3.372093in}{2.761754in}}%
\pgfpathlineto{\pgfqpoint{3.622205in}{2.891176in}}%
\pgfpathlineto{\pgfqpoint{3.186131in}{2.954270in}}%
\pgfpathlineto{\pgfqpoint{2.932968in}{2.820668in}}%
\pgfpathclose%
\pgfusepath{fill}%
\end{pgfscope}%
\begin{pgfscope}%
\pgfpathrectangle{\pgfqpoint{0.017500in}{0.017500in}}{\pgfqpoint{6.218282in}{3.580372in}} %
\pgfusepath{clip}%
\pgfsetbuttcap%
\pgfsetroundjoin%
\definecolor{currentfill}{rgb}{0.717435,0.051118,0.158737}%
\pgfsetfillcolor{currentfill}%
\pgfsetlinewidth{0.000000pt}%
\definecolor{currentstroke}{rgb}{0.000000,0.000000,0.000000}%
\pgfsetstrokecolor{currentstroke}%
\pgfsetdash{}{0pt}%
\pgfpathmoveto{\pgfqpoint{2.932968in}{2.820668in}}%
\pgfpathlineto{\pgfqpoint{3.372093in}{2.761754in}}%
\pgfpathlineto{\pgfqpoint{3.622205in}{2.891176in}}%
\pgfpathlineto{\pgfqpoint{3.186131in}{2.954270in}}%
\pgfpathlineto{\pgfqpoint{2.932968in}{2.820668in}}%
\pgfpathclose%
\pgfusepath{fill}%
\end{pgfscope}%
\begin{pgfscope}%
\pgfpathrectangle{\pgfqpoint{0.017500in}{0.017500in}}{\pgfqpoint{6.218282in}{3.580372in}} %
\pgfusepath{clip}%
\pgfsetbuttcap%
\pgfsetroundjoin%
\definecolor{currentfill}{rgb}{0.740957,0.122240,0.175744}%
\pgfsetfillcolor{currentfill}%
\pgfsetlinewidth{0.000000pt}%
\definecolor{currentstroke}{rgb}{0.000000,0.000000,0.000000}%
\pgfsetstrokecolor{currentstroke}%
\pgfsetdash{}{0pt}%
\pgfpathmoveto{\pgfqpoint{2.237924in}{2.752812in}}%
\pgfpathlineto{\pgfqpoint{2.499217in}{2.903684in}}%
\pgfpathlineto{\pgfqpoint{2.237924in}{2.752812in}}%
\pgfpathclose%
\pgfusepath{fill}%
\end{pgfscope}%
\begin{pgfscope}%
\pgfpathrectangle{\pgfqpoint{0.017500in}{0.017500in}}{\pgfqpoint{6.218282in}{3.580372in}} %
\pgfusepath{clip}%
\pgfsetbuttcap%
\pgfsetroundjoin%
\definecolor{currentfill}{rgb}{0.740957,0.122240,0.175744}%
\pgfsetfillcolor{currentfill}%
\pgfsetlinewidth{0.000000pt}%
\definecolor{currentstroke}{rgb}{0.000000,0.000000,0.000000}%
\pgfsetstrokecolor{currentstroke}%
\pgfsetdash{}{0pt}%
\pgfpathmoveto{\pgfqpoint{2.237924in}{2.752812in}}%
\pgfpathlineto{\pgfqpoint{2.499217in}{2.903684in}}%
\pgfpathlineto{\pgfqpoint{2.237924in}{2.752812in}}%
\pgfpathclose%
\pgfusepath{fill}%
\end{pgfscope}%
\begin{pgfscope}%
\pgfpathrectangle{\pgfqpoint{0.017500in}{0.017500in}}{\pgfqpoint{6.218282in}{3.580372in}} %
\pgfusepath{clip}%
\pgfsetbuttcap%
\pgfsetroundjoin%
\definecolor{currentfill}{rgb}{0.740957,0.122240,0.175744}%
\pgfsetfillcolor{currentfill}%
\pgfsetlinewidth{0.000000pt}%
\definecolor{currentstroke}{rgb}{0.000000,0.000000,0.000000}%
\pgfsetstrokecolor{currentstroke}%
\pgfsetdash{}{0pt}%
\pgfpathmoveto{\pgfqpoint{2.237924in}{2.752812in}}%
\pgfpathlineto{\pgfqpoint{2.499217in}{2.903684in}}%
\pgfpathlineto{\pgfqpoint{2.237924in}{2.752812in}}%
\pgfpathclose%
\pgfusepath{fill}%
\end{pgfscope}%
\begin{pgfscope}%
\pgfpathrectangle{\pgfqpoint{0.017500in}{0.017500in}}{\pgfqpoint{6.218282in}{3.580372in}} %
\pgfusepath{clip}%
\pgfsetbuttcap%
\pgfsetroundjoin%
\definecolor{currentfill}{rgb}{0.740957,0.122240,0.175744}%
\pgfsetfillcolor{currentfill}%
\pgfsetlinewidth{0.000000pt}%
\definecolor{currentstroke}{rgb}{0.000000,0.000000,0.000000}%
\pgfsetstrokecolor{currentstroke}%
\pgfsetdash{}{0pt}%
\pgfpathmoveto{\pgfqpoint{2.237924in}{2.752812in}}%
\pgfpathlineto{\pgfqpoint{2.499217in}{2.903684in}}%
\pgfpathlineto{\pgfqpoint{2.237924in}{2.752812in}}%
\pgfpathclose%
\pgfusepath{fill}%
\end{pgfscope}%
\begin{pgfscope}%
\pgfpathrectangle{\pgfqpoint{0.017500in}{0.017500in}}{\pgfqpoint{6.218282in}{3.580372in}} %
\pgfusepath{clip}%
\pgfsetbuttcap%
\pgfsetroundjoin%
\definecolor{currentfill}{rgb}{0.740957,0.122240,0.175744}%
\pgfsetfillcolor{currentfill}%
\pgfsetlinewidth{0.000000pt}%
\definecolor{currentstroke}{rgb}{0.000000,0.000000,0.000000}%
\pgfsetstrokecolor{currentstroke}%
\pgfsetdash{}{0pt}%
\pgfpathmoveto{\pgfqpoint{2.237924in}{2.752812in}}%
\pgfpathlineto{\pgfqpoint{2.499217in}{2.903684in}}%
\pgfpathlineto{\pgfqpoint{2.237924in}{2.752812in}}%
\pgfpathclose%
\pgfusepath{fill}%
\end{pgfscope}%
\begin{pgfscope}%
\pgfpathrectangle{\pgfqpoint{0.017500in}{0.017500in}}{\pgfqpoint{6.218282in}{3.580372in}} %
\pgfusepath{clip}%
\pgfsetbuttcap%
\pgfsetroundjoin%
\definecolor{currentfill}{rgb}{0.740957,0.122240,0.175744}%
\pgfsetfillcolor{currentfill}%
\pgfsetlinewidth{0.000000pt}%
\definecolor{currentstroke}{rgb}{0.000000,0.000000,0.000000}%
\pgfsetstrokecolor{currentstroke}%
\pgfsetdash{}{0pt}%
\pgfpathmoveto{\pgfqpoint{2.237924in}{2.752812in}}%
\pgfpathlineto{\pgfqpoint{2.499217in}{2.903684in}}%
\pgfpathlineto{\pgfqpoint{2.237924in}{2.752812in}}%
\pgfpathclose%
\pgfusepath{fill}%
\end{pgfscope}%
\begin{pgfscope}%
\pgfpathrectangle{\pgfqpoint{0.017500in}{0.017500in}}{\pgfqpoint{6.218282in}{3.580372in}} %
\pgfusepath{clip}%
\pgfsetbuttcap%
\pgfsetroundjoin%
\definecolor{currentfill}{rgb}{0.740957,0.122240,0.175744}%
\pgfsetfillcolor{currentfill}%
\pgfsetlinewidth{0.000000pt}%
\definecolor{currentstroke}{rgb}{0.000000,0.000000,0.000000}%
\pgfsetstrokecolor{currentstroke}%
\pgfsetdash{}{0pt}%
\pgfpathmoveto{\pgfqpoint{2.237924in}{2.752812in}}%
\pgfpathlineto{\pgfqpoint{2.499217in}{2.903684in}}%
\pgfpathlineto{\pgfqpoint{2.237924in}{2.752812in}}%
\pgfpathclose%
\pgfusepath{fill}%
\end{pgfscope}%
\begin{pgfscope}%
\pgfpathrectangle{\pgfqpoint{0.017500in}{0.017500in}}{\pgfqpoint{6.218282in}{3.580372in}} %
\pgfusepath{clip}%
\pgfsetbuttcap%
\pgfsetroundjoin%
\definecolor{currentfill}{rgb}{0.740957,0.122240,0.175744}%
\pgfsetfillcolor{currentfill}%
\pgfsetlinewidth{0.000000pt}%
\definecolor{currentstroke}{rgb}{0.000000,0.000000,0.000000}%
\pgfsetstrokecolor{currentstroke}%
\pgfsetdash{}{0pt}%
\pgfpathmoveto{\pgfqpoint{2.237924in}{2.752812in}}%
\pgfpathlineto{\pgfqpoint{2.499217in}{2.903684in}}%
\pgfpathlineto{\pgfqpoint{2.237924in}{2.752812in}}%
\pgfpathclose%
\pgfusepath{fill}%
\end{pgfscope}%
\begin{pgfscope}%
\pgfpathrectangle{\pgfqpoint{0.017500in}{0.017500in}}{\pgfqpoint{6.218282in}{3.580372in}} %
\pgfusepath{clip}%
\pgfsetbuttcap%
\pgfsetroundjoin%
\definecolor{currentfill}{rgb}{0.740957,0.122240,0.175744}%
\pgfsetfillcolor{currentfill}%
\pgfsetlinewidth{0.000000pt}%
\definecolor{currentstroke}{rgb}{0.000000,0.000000,0.000000}%
\pgfsetstrokecolor{currentstroke}%
\pgfsetdash{}{0pt}%
\pgfpathmoveto{\pgfqpoint{2.237924in}{2.752812in}}%
\pgfpathlineto{\pgfqpoint{2.499217in}{2.903684in}}%
\pgfpathlineto{\pgfqpoint{2.237924in}{2.752812in}}%
\pgfpathclose%
\pgfusepath{fill}%
\end{pgfscope}%
\begin{pgfscope}%
\pgfpathrectangle{\pgfqpoint{0.017500in}{0.017500in}}{\pgfqpoint{6.218282in}{3.580372in}} %
\pgfusepath{clip}%
\pgfsetbuttcap%
\pgfsetroundjoin%
\definecolor{currentfill}{rgb}{0.740957,0.122240,0.175744}%
\pgfsetfillcolor{currentfill}%
\pgfsetlinewidth{0.000000pt}%
\definecolor{currentstroke}{rgb}{0.000000,0.000000,0.000000}%
\pgfsetstrokecolor{currentstroke}%
\pgfsetdash{}{0pt}%
\pgfpathmoveto{\pgfqpoint{2.237924in}{2.752812in}}%
\pgfpathlineto{\pgfqpoint{2.499217in}{2.903684in}}%
\pgfpathlineto{\pgfqpoint{2.237924in}{2.752812in}}%
\pgfpathclose%
\pgfusepath{fill}%
\end{pgfscope}%
\begin{pgfscope}%
\pgfpathrectangle{\pgfqpoint{0.017500in}{0.017500in}}{\pgfqpoint{6.218282in}{3.580372in}} %
\pgfusepath{clip}%
\pgfsetbuttcap%
\pgfsetroundjoin%
\definecolor{currentfill}{rgb}{0.740957,0.122240,0.175744}%
\pgfsetfillcolor{currentfill}%
\pgfsetlinewidth{0.000000pt}%
\definecolor{currentstroke}{rgb}{0.000000,0.000000,0.000000}%
\pgfsetstrokecolor{currentstroke}%
\pgfsetdash{}{0pt}%
\pgfpathmoveto{\pgfqpoint{2.237924in}{2.752812in}}%
\pgfpathlineto{\pgfqpoint{2.499217in}{2.903684in}}%
\pgfpathlineto{\pgfqpoint{2.237924in}{2.752812in}}%
\pgfpathclose%
\pgfusepath{fill}%
\end{pgfscope}%
\begin{pgfscope}%
\pgfpathrectangle{\pgfqpoint{0.017500in}{0.017500in}}{\pgfqpoint{6.218282in}{3.580372in}} %
\pgfusepath{clip}%
\pgfsetbuttcap%
\pgfsetroundjoin%
\definecolor{currentfill}{rgb}{0.740957,0.122240,0.175744}%
\pgfsetfillcolor{currentfill}%
\pgfsetlinewidth{0.000000pt}%
\definecolor{currentstroke}{rgb}{0.000000,0.000000,0.000000}%
\pgfsetstrokecolor{currentstroke}%
\pgfsetdash{}{0pt}%
\pgfpathmoveto{\pgfqpoint{2.237924in}{2.752812in}}%
\pgfpathlineto{\pgfqpoint{2.499217in}{2.903684in}}%
\pgfpathlineto{\pgfqpoint{2.237924in}{2.752812in}}%
\pgfpathclose%
\pgfusepath{fill}%
\end{pgfscope}%
\begin{pgfscope}%
\pgfpathrectangle{\pgfqpoint{0.017500in}{0.017500in}}{\pgfqpoint{6.218282in}{3.580372in}} %
\pgfusepath{clip}%
\pgfsetbuttcap%
\pgfsetroundjoin%
\definecolor{currentfill}{rgb}{0.740957,0.122240,0.175744}%
\pgfsetfillcolor{currentfill}%
\pgfsetlinewidth{0.000000pt}%
\definecolor{currentstroke}{rgb}{0.000000,0.000000,0.000000}%
\pgfsetstrokecolor{currentstroke}%
\pgfsetdash{}{0pt}%
\pgfpathmoveto{\pgfqpoint{2.237924in}{2.752812in}}%
\pgfpathlineto{\pgfqpoint{2.499217in}{2.903684in}}%
\pgfpathlineto{\pgfqpoint{2.237924in}{2.752812in}}%
\pgfpathclose%
\pgfusepath{fill}%
\end{pgfscope}%
\begin{pgfscope}%
\pgfpathrectangle{\pgfqpoint{0.017500in}{0.017500in}}{\pgfqpoint{6.218282in}{3.580372in}} %
\pgfusepath{clip}%
\pgfsetbuttcap%
\pgfsetroundjoin%
\definecolor{currentfill}{rgb}{0.740957,0.122240,0.175744}%
\pgfsetfillcolor{currentfill}%
\pgfsetlinewidth{0.000000pt}%
\definecolor{currentstroke}{rgb}{0.000000,0.000000,0.000000}%
\pgfsetstrokecolor{currentstroke}%
\pgfsetdash{}{0pt}%
\pgfpathmoveto{\pgfqpoint{2.237924in}{2.752812in}}%
\pgfpathlineto{\pgfqpoint{2.499217in}{2.903684in}}%
\pgfpathlineto{\pgfqpoint{2.237924in}{2.752812in}}%
\pgfpathclose%
\pgfusepath{fill}%
\end{pgfscope}%
\begin{pgfscope}%
\pgfpathrectangle{\pgfqpoint{0.017500in}{0.017500in}}{\pgfqpoint{6.218282in}{3.580372in}} %
\pgfusepath{clip}%
\pgfsetbuttcap%
\pgfsetroundjoin%
\definecolor{currentfill}{rgb}{0.740957,0.122240,0.175744}%
\pgfsetfillcolor{currentfill}%
\pgfsetlinewidth{0.000000pt}%
\definecolor{currentstroke}{rgb}{0.000000,0.000000,0.000000}%
\pgfsetstrokecolor{currentstroke}%
\pgfsetdash{}{0pt}%
\pgfpathmoveto{\pgfqpoint{2.237924in}{2.752812in}}%
\pgfpathlineto{\pgfqpoint{2.499217in}{2.903684in}}%
\pgfpathlineto{\pgfqpoint{2.237924in}{2.752812in}}%
\pgfpathclose%
\pgfusepath{fill}%
\end{pgfscope}%
\begin{pgfscope}%
\pgfpathrectangle{\pgfqpoint{0.017500in}{0.017500in}}{\pgfqpoint{6.218282in}{3.580372in}} %
\pgfusepath{clip}%
\pgfsetbuttcap%
\pgfsetroundjoin%
\definecolor{currentfill}{rgb}{0.740957,0.122240,0.175744}%
\pgfsetfillcolor{currentfill}%
\pgfsetlinewidth{0.000000pt}%
\definecolor{currentstroke}{rgb}{0.000000,0.000000,0.000000}%
\pgfsetstrokecolor{currentstroke}%
\pgfsetdash{}{0pt}%
\pgfpathmoveto{\pgfqpoint{2.237924in}{2.752812in}}%
\pgfpathlineto{\pgfqpoint{2.499217in}{2.903684in}}%
\pgfpathlineto{\pgfqpoint{2.237924in}{2.752812in}}%
\pgfpathclose%
\pgfusepath{fill}%
\end{pgfscope}%
\begin{pgfscope}%
\pgfpathrectangle{\pgfqpoint{0.017500in}{0.017500in}}{\pgfqpoint{6.218282in}{3.580372in}} %
\pgfusepath{clip}%
\pgfsetbuttcap%
\pgfsetroundjoin%
\definecolor{currentfill}{rgb}{0.740957,0.122240,0.175744}%
\pgfsetfillcolor{currentfill}%
\pgfsetlinewidth{0.000000pt}%
\definecolor{currentstroke}{rgb}{0.000000,0.000000,0.000000}%
\pgfsetstrokecolor{currentstroke}%
\pgfsetdash{}{0pt}%
\pgfpathmoveto{\pgfqpoint{2.237924in}{2.752812in}}%
\pgfpathlineto{\pgfqpoint{2.499217in}{2.903684in}}%
\pgfpathlineto{\pgfqpoint{2.237924in}{2.752812in}}%
\pgfpathclose%
\pgfusepath{fill}%
\end{pgfscope}%
\begin{pgfscope}%
\pgfpathrectangle{\pgfqpoint{0.017500in}{0.017500in}}{\pgfqpoint{6.218282in}{3.580372in}} %
\pgfusepath{clip}%
\pgfsetbuttcap%
\pgfsetroundjoin%
\definecolor{currentfill}{rgb}{0.740957,0.122240,0.175744}%
\pgfsetfillcolor{currentfill}%
\pgfsetlinewidth{0.000000pt}%
\definecolor{currentstroke}{rgb}{0.000000,0.000000,0.000000}%
\pgfsetstrokecolor{currentstroke}%
\pgfsetdash{}{0pt}%
\pgfpathmoveto{\pgfqpoint{2.237924in}{2.752812in}}%
\pgfpathlineto{\pgfqpoint{2.499217in}{2.903684in}}%
\pgfpathlineto{\pgfqpoint{2.237924in}{2.752812in}}%
\pgfpathclose%
\pgfusepath{fill}%
\end{pgfscope}%
\begin{pgfscope}%
\pgfpathrectangle{\pgfqpoint{0.017500in}{0.017500in}}{\pgfqpoint{6.218282in}{3.580372in}} %
\pgfusepath{clip}%
\pgfsetbuttcap%
\pgfsetroundjoin%
\definecolor{currentfill}{rgb}{0.740957,0.122240,0.175744}%
\pgfsetfillcolor{currentfill}%
\pgfsetlinewidth{0.000000pt}%
\definecolor{currentstroke}{rgb}{0.000000,0.000000,0.000000}%
\pgfsetstrokecolor{currentstroke}%
\pgfsetdash{}{0pt}%
\pgfpathmoveto{\pgfqpoint{2.237924in}{2.752812in}}%
\pgfpathlineto{\pgfqpoint{2.499217in}{2.903684in}}%
\pgfpathlineto{\pgfqpoint{2.237924in}{2.752812in}}%
\pgfpathclose%
\pgfusepath{fill}%
\end{pgfscope}%
\begin{pgfscope}%
\pgfpathrectangle{\pgfqpoint{0.017500in}{0.017500in}}{\pgfqpoint{6.218282in}{3.580372in}} %
\pgfusepath{clip}%
\pgfsetbuttcap%
\pgfsetroundjoin%
\definecolor{currentfill}{rgb}{0.740957,0.122240,0.175744}%
\pgfsetfillcolor{currentfill}%
\pgfsetlinewidth{0.000000pt}%
\definecolor{currentstroke}{rgb}{0.000000,0.000000,0.000000}%
\pgfsetstrokecolor{currentstroke}%
\pgfsetdash{}{0pt}%
\pgfpathmoveto{\pgfqpoint{2.237924in}{2.752812in}}%
\pgfpathlineto{\pgfqpoint{2.499217in}{2.903684in}}%
\pgfpathlineto{\pgfqpoint{2.237924in}{2.752812in}}%
\pgfpathclose%
\pgfusepath{fill}%
\end{pgfscope}%
\begin{pgfscope}%
\pgfpathrectangle{\pgfqpoint{0.017500in}{0.017500in}}{\pgfqpoint{6.218282in}{3.580372in}} %
\pgfusepath{clip}%
\pgfsetbuttcap%
\pgfsetroundjoin%
\definecolor{currentfill}{rgb}{0.740957,0.122240,0.175744}%
\pgfsetfillcolor{currentfill}%
\pgfsetlinewidth{0.000000pt}%
\definecolor{currentstroke}{rgb}{0.000000,0.000000,0.000000}%
\pgfsetstrokecolor{currentstroke}%
\pgfsetdash{}{0pt}%
\pgfpathmoveto{\pgfqpoint{2.237924in}{2.752812in}}%
\pgfpathlineto{\pgfqpoint{2.499217in}{2.903684in}}%
\pgfpathlineto{\pgfqpoint{2.237924in}{2.752812in}}%
\pgfpathclose%
\pgfusepath{fill}%
\end{pgfscope}%
\begin{pgfscope}%
\pgfpathrectangle{\pgfqpoint{0.017500in}{0.017500in}}{\pgfqpoint{6.218282in}{3.580372in}} %
\pgfusepath{clip}%
\pgfsetbuttcap%
\pgfsetroundjoin%
\definecolor{currentfill}{rgb}{0.740957,0.122240,0.175744}%
\pgfsetfillcolor{currentfill}%
\pgfsetlinewidth{0.000000pt}%
\definecolor{currentstroke}{rgb}{0.000000,0.000000,0.000000}%
\pgfsetstrokecolor{currentstroke}%
\pgfsetdash{}{0pt}%
\pgfpathmoveto{\pgfqpoint{2.237924in}{2.752812in}}%
\pgfpathlineto{\pgfqpoint{2.499217in}{2.903684in}}%
\pgfpathlineto{\pgfqpoint{2.237924in}{2.752812in}}%
\pgfpathclose%
\pgfusepath{fill}%
\end{pgfscope}%
\begin{pgfscope}%
\pgfpathrectangle{\pgfqpoint{0.017500in}{0.017500in}}{\pgfqpoint{6.218282in}{3.580372in}} %
\pgfusepath{clip}%
\pgfsetbuttcap%
\pgfsetroundjoin%
\definecolor{currentfill}{rgb}{0.740957,0.122240,0.175744}%
\pgfsetfillcolor{currentfill}%
\pgfsetlinewidth{0.000000pt}%
\definecolor{currentstroke}{rgb}{0.000000,0.000000,0.000000}%
\pgfsetstrokecolor{currentstroke}%
\pgfsetdash{}{0pt}%
\pgfpathmoveto{\pgfqpoint{2.237924in}{2.752812in}}%
\pgfpathlineto{\pgfqpoint{2.499217in}{2.903684in}}%
\pgfpathlineto{\pgfqpoint{2.237924in}{2.752812in}}%
\pgfpathclose%
\pgfusepath{fill}%
\end{pgfscope}%
\begin{pgfscope}%
\pgfpathrectangle{\pgfqpoint{0.017500in}{0.017500in}}{\pgfqpoint{6.218282in}{3.580372in}} %
\pgfusepath{clip}%
\pgfsetbuttcap%
\pgfsetroundjoin%
\definecolor{currentfill}{rgb}{0.740957,0.122240,0.175744}%
\pgfsetfillcolor{currentfill}%
\pgfsetlinewidth{0.000000pt}%
\definecolor{currentstroke}{rgb}{0.000000,0.000000,0.000000}%
\pgfsetstrokecolor{currentstroke}%
\pgfsetdash{}{0pt}%
\pgfpathmoveto{\pgfqpoint{2.237924in}{2.752812in}}%
\pgfpathlineto{\pgfqpoint{2.499217in}{2.903684in}}%
\pgfpathlineto{\pgfqpoint{2.237924in}{2.752812in}}%
\pgfpathclose%
\pgfusepath{fill}%
\end{pgfscope}%
\begin{pgfscope}%
\pgfpathrectangle{\pgfqpoint{0.017500in}{0.017500in}}{\pgfqpoint{6.218282in}{3.580372in}} %
\pgfusepath{clip}%
\pgfsetbuttcap%
\pgfsetroundjoin%
\definecolor{currentfill}{rgb}{0.740957,0.122240,0.175744}%
\pgfsetfillcolor{currentfill}%
\pgfsetlinewidth{0.000000pt}%
\definecolor{currentstroke}{rgb}{0.000000,0.000000,0.000000}%
\pgfsetstrokecolor{currentstroke}%
\pgfsetdash{}{0pt}%
\pgfpathmoveto{\pgfqpoint{2.237924in}{2.752812in}}%
\pgfpathlineto{\pgfqpoint{2.499217in}{2.903684in}}%
\pgfpathlineto{\pgfqpoint{2.237924in}{2.752812in}}%
\pgfpathclose%
\pgfusepath{fill}%
\end{pgfscope}%
\begin{pgfscope}%
\pgfpathrectangle{\pgfqpoint{0.017500in}{0.017500in}}{\pgfqpoint{6.218282in}{3.580372in}} %
\pgfusepath{clip}%
\pgfsetbuttcap%
\pgfsetroundjoin%
\definecolor{currentfill}{rgb}{0.740957,0.122240,0.175744}%
\pgfsetfillcolor{currentfill}%
\pgfsetlinewidth{0.000000pt}%
\definecolor{currentstroke}{rgb}{0.000000,0.000000,0.000000}%
\pgfsetstrokecolor{currentstroke}%
\pgfsetdash{}{0pt}%
\pgfpathmoveto{\pgfqpoint{2.237924in}{2.752812in}}%
\pgfpathlineto{\pgfqpoint{2.499217in}{2.903684in}}%
\pgfpathlineto{\pgfqpoint{2.237924in}{2.752812in}}%
\pgfpathclose%
\pgfusepath{fill}%
\end{pgfscope}%
\begin{pgfscope}%
\pgfpathrectangle{\pgfqpoint{0.017500in}{0.017500in}}{\pgfqpoint{6.218282in}{3.580372in}} %
\pgfusepath{clip}%
\pgfsetbuttcap%
\pgfsetroundjoin%
\definecolor{currentfill}{rgb}{0.740957,0.122240,0.175744}%
\pgfsetfillcolor{currentfill}%
\pgfsetlinewidth{0.000000pt}%
\definecolor{currentstroke}{rgb}{0.000000,0.000000,0.000000}%
\pgfsetstrokecolor{currentstroke}%
\pgfsetdash{}{0pt}%
\pgfpathmoveto{\pgfqpoint{2.237924in}{2.752812in}}%
\pgfpathlineto{\pgfqpoint{2.499217in}{2.903684in}}%
\pgfpathlineto{\pgfqpoint{2.237924in}{2.752812in}}%
\pgfpathclose%
\pgfusepath{fill}%
\end{pgfscope}%
\begin{pgfscope}%
\pgfpathrectangle{\pgfqpoint{0.017500in}{0.017500in}}{\pgfqpoint{6.218282in}{3.580372in}} %
\pgfusepath{clip}%
\pgfsetbuttcap%
\pgfsetroundjoin%
\definecolor{currentfill}{rgb}{0.740957,0.122240,0.175744}%
\pgfsetfillcolor{currentfill}%
\pgfsetlinewidth{0.000000pt}%
\definecolor{currentstroke}{rgb}{0.000000,0.000000,0.000000}%
\pgfsetstrokecolor{currentstroke}%
\pgfsetdash{}{0pt}%
\pgfpathmoveto{\pgfqpoint{2.237924in}{2.752812in}}%
\pgfpathlineto{\pgfqpoint{2.499217in}{2.903684in}}%
\pgfpathlineto{\pgfqpoint{2.237924in}{2.752812in}}%
\pgfpathclose%
\pgfusepath{fill}%
\end{pgfscope}%
\begin{pgfscope}%
\pgfpathrectangle{\pgfqpoint{0.017500in}{0.017500in}}{\pgfqpoint{6.218282in}{3.580372in}} %
\pgfusepath{clip}%
\pgfsetbuttcap%
\pgfsetroundjoin%
\definecolor{currentfill}{rgb}{0.740957,0.122240,0.175744}%
\pgfsetfillcolor{currentfill}%
\pgfsetlinewidth{0.000000pt}%
\definecolor{currentstroke}{rgb}{0.000000,0.000000,0.000000}%
\pgfsetstrokecolor{currentstroke}%
\pgfsetdash{}{0pt}%
\pgfpathmoveto{\pgfqpoint{2.237924in}{2.752812in}}%
\pgfpathlineto{\pgfqpoint{2.499217in}{2.903684in}}%
\pgfpathlineto{\pgfqpoint{2.237924in}{2.752812in}}%
\pgfpathclose%
\pgfusepath{fill}%
\end{pgfscope}%
\begin{pgfscope}%
\pgfpathrectangle{\pgfqpoint{0.017500in}{0.017500in}}{\pgfqpoint{6.218282in}{3.580372in}} %
\pgfusepath{clip}%
\pgfsetbuttcap%
\pgfsetroundjoin%
\definecolor{currentfill}{rgb}{0.740957,0.122240,0.175744}%
\pgfsetfillcolor{currentfill}%
\pgfsetlinewidth{0.000000pt}%
\definecolor{currentstroke}{rgb}{0.000000,0.000000,0.000000}%
\pgfsetstrokecolor{currentstroke}%
\pgfsetdash{}{0pt}%
\pgfpathmoveto{\pgfqpoint{2.237924in}{2.752812in}}%
\pgfpathlineto{\pgfqpoint{2.499217in}{2.903684in}}%
\pgfpathlineto{\pgfqpoint{2.237924in}{2.752812in}}%
\pgfpathclose%
\pgfusepath{fill}%
\end{pgfscope}%
\begin{pgfscope}%
\pgfpathrectangle{\pgfqpoint{0.017500in}{0.017500in}}{\pgfqpoint{6.218282in}{3.580372in}} %
\pgfusepath{clip}%
\pgfsetbuttcap%
\pgfsetroundjoin%
\definecolor{currentfill}{rgb}{0.899543,0.847500,0.817789}%
\pgfsetfillcolor{currentfill}%
\pgfsetlinewidth{0.000000pt}%
\definecolor{currentstroke}{rgb}{0.000000,0.000000,0.000000}%
\pgfsetstrokecolor{currentstroke}%
\pgfsetdash{}{0pt}%
\pgfpathmoveto{\pgfqpoint{3.622205in}{2.891176in}}%
\pgfpathlineto{\pgfqpoint{2.349853in}{0.872712in}}%
\pgfpathlineto{\pgfqpoint{3.622205in}{2.891176in}}%
\pgfpathclose%
\pgfusepath{fill}%
\end{pgfscope}%
\begin{pgfscope}%
\pgfpathrectangle{\pgfqpoint{0.017500in}{0.017500in}}{\pgfqpoint{6.218282in}{3.580372in}} %
\pgfusepath{clip}%
\pgfsetbuttcap%
\pgfsetroundjoin%
\definecolor{currentfill}{rgb}{0.899543,0.847500,0.817789}%
\pgfsetfillcolor{currentfill}%
\pgfsetlinewidth{0.000000pt}%
\definecolor{currentstroke}{rgb}{0.000000,0.000000,0.000000}%
\pgfsetstrokecolor{currentstroke}%
\pgfsetdash{}{0pt}%
\pgfpathmoveto{\pgfqpoint{3.622205in}{2.891176in}}%
\pgfpathlineto{\pgfqpoint{2.349853in}{0.872712in}}%
\pgfpathlineto{\pgfqpoint{3.622205in}{2.891176in}}%
\pgfpathclose%
\pgfusepath{fill}%
\end{pgfscope}%
\begin{pgfscope}%
\pgfpathrectangle{\pgfqpoint{0.017500in}{0.017500in}}{\pgfqpoint{6.218282in}{3.580372in}} %
\pgfusepath{clip}%
\pgfsetbuttcap%
\pgfsetroundjoin%
\definecolor{currentfill}{rgb}{0.899543,0.847500,0.817789}%
\pgfsetfillcolor{currentfill}%
\pgfsetlinewidth{0.000000pt}%
\definecolor{currentstroke}{rgb}{0.000000,0.000000,0.000000}%
\pgfsetstrokecolor{currentstroke}%
\pgfsetdash{}{0pt}%
\pgfpathmoveto{\pgfqpoint{3.622205in}{2.891176in}}%
\pgfpathlineto{\pgfqpoint{2.349853in}{0.872712in}}%
\pgfpathlineto{\pgfqpoint{3.622205in}{2.891176in}}%
\pgfpathclose%
\pgfusepath{fill}%
\end{pgfscope}%
\begin{pgfscope}%
\pgfpathrectangle{\pgfqpoint{0.017500in}{0.017500in}}{\pgfqpoint{6.218282in}{3.580372in}} %
\pgfusepath{clip}%
\pgfsetbuttcap%
\pgfsetroundjoin%
\definecolor{currentfill}{rgb}{0.899543,0.847500,0.817789}%
\pgfsetfillcolor{currentfill}%
\pgfsetlinewidth{0.000000pt}%
\definecolor{currentstroke}{rgb}{0.000000,0.000000,0.000000}%
\pgfsetstrokecolor{currentstroke}%
\pgfsetdash{}{0pt}%
\pgfpathmoveto{\pgfqpoint{3.622205in}{2.891176in}}%
\pgfpathlineto{\pgfqpoint{2.349853in}{0.872712in}}%
\pgfpathlineto{\pgfqpoint{3.622205in}{2.891176in}}%
\pgfpathclose%
\pgfusepath{fill}%
\end{pgfscope}%
\begin{pgfscope}%
\pgfpathrectangle{\pgfqpoint{0.017500in}{0.017500in}}{\pgfqpoint{6.218282in}{3.580372in}} %
\pgfusepath{clip}%
\pgfsetbuttcap%
\pgfsetroundjoin%
\definecolor{currentfill}{rgb}{0.899543,0.847500,0.817789}%
\pgfsetfillcolor{currentfill}%
\pgfsetlinewidth{0.000000pt}%
\definecolor{currentstroke}{rgb}{0.000000,0.000000,0.000000}%
\pgfsetstrokecolor{currentstroke}%
\pgfsetdash{}{0pt}%
\pgfpathmoveto{\pgfqpoint{3.622205in}{2.891176in}}%
\pgfpathlineto{\pgfqpoint{2.349853in}{0.872712in}}%
\pgfpathlineto{\pgfqpoint{3.622205in}{2.891176in}}%
\pgfpathclose%
\pgfusepath{fill}%
\end{pgfscope}%
\begin{pgfscope}%
\pgfpathrectangle{\pgfqpoint{0.017500in}{0.017500in}}{\pgfqpoint{6.218282in}{3.580372in}} %
\pgfusepath{clip}%
\pgfsetbuttcap%
\pgfsetroundjoin%
\definecolor{currentfill}{rgb}{0.899543,0.847500,0.817789}%
\pgfsetfillcolor{currentfill}%
\pgfsetlinewidth{0.000000pt}%
\definecolor{currentstroke}{rgb}{0.000000,0.000000,0.000000}%
\pgfsetstrokecolor{currentstroke}%
\pgfsetdash{}{0pt}%
\pgfpathmoveto{\pgfqpoint{3.622205in}{2.891176in}}%
\pgfpathlineto{\pgfqpoint{2.349853in}{0.872712in}}%
\pgfpathlineto{\pgfqpoint{3.622205in}{2.891176in}}%
\pgfpathclose%
\pgfusepath{fill}%
\end{pgfscope}%
\begin{pgfscope}%
\pgfpathrectangle{\pgfqpoint{0.017500in}{0.017500in}}{\pgfqpoint{6.218282in}{3.580372in}} %
\pgfusepath{clip}%
\pgfsetbuttcap%
\pgfsetroundjoin%
\definecolor{currentfill}{rgb}{0.899543,0.847500,0.817789}%
\pgfsetfillcolor{currentfill}%
\pgfsetlinewidth{0.000000pt}%
\definecolor{currentstroke}{rgb}{0.000000,0.000000,0.000000}%
\pgfsetstrokecolor{currentstroke}%
\pgfsetdash{}{0pt}%
\pgfpathmoveto{\pgfqpoint{3.622205in}{2.891176in}}%
\pgfpathlineto{\pgfqpoint{2.349853in}{0.872712in}}%
\pgfpathlineto{\pgfqpoint{3.622205in}{2.891176in}}%
\pgfpathclose%
\pgfusepath{fill}%
\end{pgfscope}%
\begin{pgfscope}%
\pgfpathrectangle{\pgfqpoint{0.017500in}{0.017500in}}{\pgfqpoint{6.218282in}{3.580372in}} %
\pgfusepath{clip}%
\pgfsetbuttcap%
\pgfsetroundjoin%
\definecolor{currentfill}{rgb}{0.899543,0.847500,0.817789}%
\pgfsetfillcolor{currentfill}%
\pgfsetlinewidth{0.000000pt}%
\definecolor{currentstroke}{rgb}{0.000000,0.000000,0.000000}%
\pgfsetstrokecolor{currentstroke}%
\pgfsetdash{}{0pt}%
\pgfpathmoveto{\pgfqpoint{3.622205in}{2.891176in}}%
\pgfpathlineto{\pgfqpoint{2.349853in}{0.872712in}}%
\pgfpathlineto{\pgfqpoint{3.622205in}{2.891176in}}%
\pgfpathclose%
\pgfusepath{fill}%
\end{pgfscope}%
\begin{pgfscope}%
\pgfpathrectangle{\pgfqpoint{0.017500in}{0.017500in}}{\pgfqpoint{6.218282in}{3.580372in}} %
\pgfusepath{clip}%
\pgfsetbuttcap%
\pgfsetroundjoin%
\definecolor{currentfill}{rgb}{0.899543,0.847500,0.817789}%
\pgfsetfillcolor{currentfill}%
\pgfsetlinewidth{0.000000pt}%
\definecolor{currentstroke}{rgb}{0.000000,0.000000,0.000000}%
\pgfsetstrokecolor{currentstroke}%
\pgfsetdash{}{0pt}%
\pgfpathmoveto{\pgfqpoint{3.622205in}{2.891176in}}%
\pgfpathlineto{\pgfqpoint{2.349853in}{0.872712in}}%
\pgfpathlineto{\pgfqpoint{3.622205in}{2.891176in}}%
\pgfpathclose%
\pgfusepath{fill}%
\end{pgfscope}%
\begin{pgfscope}%
\pgfpathrectangle{\pgfqpoint{0.017500in}{0.017500in}}{\pgfqpoint{6.218282in}{3.580372in}} %
\pgfusepath{clip}%
\pgfsetbuttcap%
\pgfsetroundjoin%
\definecolor{currentfill}{rgb}{0.899543,0.847500,0.817789}%
\pgfsetfillcolor{currentfill}%
\pgfsetlinewidth{0.000000pt}%
\definecolor{currentstroke}{rgb}{0.000000,0.000000,0.000000}%
\pgfsetstrokecolor{currentstroke}%
\pgfsetdash{}{0pt}%
\pgfpathmoveto{\pgfqpoint{3.622205in}{2.891176in}}%
\pgfpathlineto{\pgfqpoint{2.349853in}{0.872712in}}%
\pgfpathlineto{\pgfqpoint{3.622205in}{2.891176in}}%
\pgfpathclose%
\pgfusepath{fill}%
\end{pgfscope}%
\begin{pgfscope}%
\pgfpathrectangle{\pgfqpoint{0.017500in}{0.017500in}}{\pgfqpoint{6.218282in}{3.580372in}} %
\pgfusepath{clip}%
\pgfsetbuttcap%
\pgfsetroundjoin%
\definecolor{currentfill}{rgb}{0.899543,0.847500,0.817789}%
\pgfsetfillcolor{currentfill}%
\pgfsetlinewidth{0.000000pt}%
\definecolor{currentstroke}{rgb}{0.000000,0.000000,0.000000}%
\pgfsetstrokecolor{currentstroke}%
\pgfsetdash{}{0pt}%
\pgfpathmoveto{\pgfqpoint{3.622205in}{2.891176in}}%
\pgfpathlineto{\pgfqpoint{2.349853in}{0.872712in}}%
\pgfpathlineto{\pgfqpoint{3.622205in}{2.891176in}}%
\pgfpathclose%
\pgfusepath{fill}%
\end{pgfscope}%
\begin{pgfscope}%
\pgfpathrectangle{\pgfqpoint{0.017500in}{0.017500in}}{\pgfqpoint{6.218282in}{3.580372in}} %
\pgfusepath{clip}%
\pgfsetbuttcap%
\pgfsetroundjoin%
\definecolor{currentfill}{rgb}{0.899543,0.847500,0.817789}%
\pgfsetfillcolor{currentfill}%
\pgfsetlinewidth{0.000000pt}%
\definecolor{currentstroke}{rgb}{0.000000,0.000000,0.000000}%
\pgfsetstrokecolor{currentstroke}%
\pgfsetdash{}{0pt}%
\pgfpathmoveto{\pgfqpoint{3.622205in}{2.891176in}}%
\pgfpathlineto{\pgfqpoint{2.349853in}{0.872712in}}%
\pgfpathlineto{\pgfqpoint{3.622205in}{2.891176in}}%
\pgfpathclose%
\pgfusepath{fill}%
\end{pgfscope}%
\begin{pgfscope}%
\pgfpathrectangle{\pgfqpoint{0.017500in}{0.017500in}}{\pgfqpoint{6.218282in}{3.580372in}} %
\pgfusepath{clip}%
\pgfsetbuttcap%
\pgfsetroundjoin%
\definecolor{currentfill}{rgb}{0.899543,0.847500,0.817789}%
\pgfsetfillcolor{currentfill}%
\pgfsetlinewidth{0.000000pt}%
\definecolor{currentstroke}{rgb}{0.000000,0.000000,0.000000}%
\pgfsetstrokecolor{currentstroke}%
\pgfsetdash{}{0pt}%
\pgfpathmoveto{\pgfqpoint{3.622205in}{2.891176in}}%
\pgfpathlineto{\pgfqpoint{2.349853in}{0.872712in}}%
\pgfpathlineto{\pgfqpoint{3.622205in}{2.891176in}}%
\pgfpathclose%
\pgfusepath{fill}%
\end{pgfscope}%
\begin{pgfscope}%
\pgfpathrectangle{\pgfqpoint{0.017500in}{0.017500in}}{\pgfqpoint{6.218282in}{3.580372in}} %
\pgfusepath{clip}%
\pgfsetbuttcap%
\pgfsetroundjoin%
\definecolor{currentfill}{rgb}{0.899543,0.847500,0.817789}%
\pgfsetfillcolor{currentfill}%
\pgfsetlinewidth{0.000000pt}%
\definecolor{currentstroke}{rgb}{0.000000,0.000000,0.000000}%
\pgfsetstrokecolor{currentstroke}%
\pgfsetdash{}{0pt}%
\pgfpathmoveto{\pgfqpoint{3.622205in}{2.891176in}}%
\pgfpathlineto{\pgfqpoint{2.349853in}{0.872712in}}%
\pgfpathlineto{\pgfqpoint{3.622205in}{2.891176in}}%
\pgfpathclose%
\pgfusepath{fill}%
\end{pgfscope}%
\begin{pgfscope}%
\pgfpathrectangle{\pgfqpoint{0.017500in}{0.017500in}}{\pgfqpoint{6.218282in}{3.580372in}} %
\pgfusepath{clip}%
\pgfsetbuttcap%
\pgfsetroundjoin%
\definecolor{currentfill}{rgb}{0.899543,0.847500,0.817789}%
\pgfsetfillcolor{currentfill}%
\pgfsetlinewidth{0.000000pt}%
\definecolor{currentstroke}{rgb}{0.000000,0.000000,0.000000}%
\pgfsetstrokecolor{currentstroke}%
\pgfsetdash{}{0pt}%
\pgfpathmoveto{\pgfqpoint{3.622205in}{2.891176in}}%
\pgfpathlineto{\pgfqpoint{2.349853in}{0.872712in}}%
\pgfpathlineto{\pgfqpoint{3.622205in}{2.891176in}}%
\pgfpathclose%
\pgfusepath{fill}%
\end{pgfscope}%
\begin{pgfscope}%
\pgfpathrectangle{\pgfqpoint{0.017500in}{0.017500in}}{\pgfqpoint{6.218282in}{3.580372in}} %
\pgfusepath{clip}%
\pgfsetbuttcap%
\pgfsetroundjoin%
\definecolor{currentfill}{rgb}{0.899543,0.847500,0.817789}%
\pgfsetfillcolor{currentfill}%
\pgfsetlinewidth{0.000000pt}%
\definecolor{currentstroke}{rgb}{0.000000,0.000000,0.000000}%
\pgfsetstrokecolor{currentstroke}%
\pgfsetdash{}{0pt}%
\pgfpathmoveto{\pgfqpoint{3.622205in}{2.891176in}}%
\pgfpathlineto{\pgfqpoint{2.349853in}{0.872712in}}%
\pgfpathlineto{\pgfqpoint{3.622205in}{2.891176in}}%
\pgfpathclose%
\pgfusepath{fill}%
\end{pgfscope}%
\begin{pgfscope}%
\pgfpathrectangle{\pgfqpoint{0.017500in}{0.017500in}}{\pgfqpoint{6.218282in}{3.580372in}} %
\pgfusepath{clip}%
\pgfsetbuttcap%
\pgfsetroundjoin%
\definecolor{currentfill}{rgb}{0.899543,0.847500,0.817789}%
\pgfsetfillcolor{currentfill}%
\pgfsetlinewidth{0.000000pt}%
\definecolor{currentstroke}{rgb}{0.000000,0.000000,0.000000}%
\pgfsetstrokecolor{currentstroke}%
\pgfsetdash{}{0pt}%
\pgfpathmoveto{\pgfqpoint{3.622205in}{2.891176in}}%
\pgfpathlineto{\pgfqpoint{2.349853in}{0.872712in}}%
\pgfpathlineto{\pgfqpoint{3.622205in}{2.891176in}}%
\pgfpathclose%
\pgfusepath{fill}%
\end{pgfscope}%
\begin{pgfscope}%
\pgfpathrectangle{\pgfqpoint{0.017500in}{0.017500in}}{\pgfqpoint{6.218282in}{3.580372in}} %
\pgfusepath{clip}%
\pgfsetbuttcap%
\pgfsetroundjoin%
\definecolor{currentfill}{rgb}{0.899543,0.847500,0.817789}%
\pgfsetfillcolor{currentfill}%
\pgfsetlinewidth{0.000000pt}%
\definecolor{currentstroke}{rgb}{0.000000,0.000000,0.000000}%
\pgfsetstrokecolor{currentstroke}%
\pgfsetdash{}{0pt}%
\pgfpathmoveto{\pgfqpoint{3.622205in}{2.891176in}}%
\pgfpathlineto{\pgfqpoint{2.349853in}{0.872712in}}%
\pgfpathlineto{\pgfqpoint{3.622205in}{2.891176in}}%
\pgfpathclose%
\pgfusepath{fill}%
\end{pgfscope}%
\begin{pgfscope}%
\pgfpathrectangle{\pgfqpoint{0.017500in}{0.017500in}}{\pgfqpoint{6.218282in}{3.580372in}} %
\pgfusepath{clip}%
\pgfsetbuttcap%
\pgfsetroundjoin%
\definecolor{currentfill}{rgb}{0.899543,0.847500,0.817789}%
\pgfsetfillcolor{currentfill}%
\pgfsetlinewidth{0.000000pt}%
\definecolor{currentstroke}{rgb}{0.000000,0.000000,0.000000}%
\pgfsetstrokecolor{currentstroke}%
\pgfsetdash{}{0pt}%
\pgfpathmoveto{\pgfqpoint{3.622205in}{2.891176in}}%
\pgfpathlineto{\pgfqpoint{2.349853in}{0.872712in}}%
\pgfpathlineto{\pgfqpoint{3.622205in}{2.891176in}}%
\pgfpathclose%
\pgfusepath{fill}%
\end{pgfscope}%
\begin{pgfscope}%
\pgfpathrectangle{\pgfqpoint{0.017500in}{0.017500in}}{\pgfqpoint{6.218282in}{3.580372in}} %
\pgfusepath{clip}%
\pgfsetbuttcap%
\pgfsetroundjoin%
\definecolor{currentfill}{rgb}{0.899543,0.847500,0.817789}%
\pgfsetfillcolor{currentfill}%
\pgfsetlinewidth{0.000000pt}%
\definecolor{currentstroke}{rgb}{0.000000,0.000000,0.000000}%
\pgfsetstrokecolor{currentstroke}%
\pgfsetdash{}{0pt}%
\pgfpathmoveto{\pgfqpoint{3.622205in}{2.891176in}}%
\pgfpathlineto{\pgfqpoint{2.349853in}{0.872712in}}%
\pgfpathlineto{\pgfqpoint{3.622205in}{2.891176in}}%
\pgfpathclose%
\pgfusepath{fill}%
\end{pgfscope}%
\begin{pgfscope}%
\pgfpathrectangle{\pgfqpoint{0.017500in}{0.017500in}}{\pgfqpoint{6.218282in}{3.580372in}} %
\pgfusepath{clip}%
\pgfsetbuttcap%
\pgfsetroundjoin%
\definecolor{currentfill}{rgb}{0.899543,0.847500,0.817789}%
\pgfsetfillcolor{currentfill}%
\pgfsetlinewidth{0.000000pt}%
\definecolor{currentstroke}{rgb}{0.000000,0.000000,0.000000}%
\pgfsetstrokecolor{currentstroke}%
\pgfsetdash{}{0pt}%
\pgfpathmoveto{\pgfqpoint{3.622205in}{2.891176in}}%
\pgfpathlineto{\pgfqpoint{2.349853in}{0.872712in}}%
\pgfpathlineto{\pgfqpoint{3.622205in}{2.891176in}}%
\pgfpathclose%
\pgfusepath{fill}%
\end{pgfscope}%
\begin{pgfscope}%
\pgfpathrectangle{\pgfqpoint{0.017500in}{0.017500in}}{\pgfqpoint{6.218282in}{3.580372in}} %
\pgfusepath{clip}%
\pgfsetbuttcap%
\pgfsetroundjoin%
\definecolor{currentfill}{rgb}{0.899543,0.847500,0.817789}%
\pgfsetfillcolor{currentfill}%
\pgfsetlinewidth{0.000000pt}%
\definecolor{currentstroke}{rgb}{0.000000,0.000000,0.000000}%
\pgfsetstrokecolor{currentstroke}%
\pgfsetdash{}{0pt}%
\pgfpathmoveto{\pgfqpoint{3.622205in}{2.891176in}}%
\pgfpathlineto{\pgfqpoint{2.349853in}{0.872712in}}%
\pgfpathlineto{\pgfqpoint{3.622205in}{2.891176in}}%
\pgfpathclose%
\pgfusepath{fill}%
\end{pgfscope}%
\begin{pgfscope}%
\pgfpathrectangle{\pgfqpoint{0.017500in}{0.017500in}}{\pgfqpoint{6.218282in}{3.580372in}} %
\pgfusepath{clip}%
\pgfsetbuttcap%
\pgfsetroundjoin%
\definecolor{currentfill}{rgb}{0.899543,0.847500,0.817789}%
\pgfsetfillcolor{currentfill}%
\pgfsetlinewidth{0.000000pt}%
\definecolor{currentstroke}{rgb}{0.000000,0.000000,0.000000}%
\pgfsetstrokecolor{currentstroke}%
\pgfsetdash{}{0pt}%
\pgfpathmoveto{\pgfqpoint{3.622205in}{2.891176in}}%
\pgfpathlineto{\pgfqpoint{2.349853in}{0.872712in}}%
\pgfpathlineto{\pgfqpoint{3.622205in}{2.891176in}}%
\pgfpathclose%
\pgfusepath{fill}%
\end{pgfscope}%
\begin{pgfscope}%
\pgfpathrectangle{\pgfqpoint{0.017500in}{0.017500in}}{\pgfqpoint{6.218282in}{3.580372in}} %
\pgfusepath{clip}%
\pgfsetbuttcap%
\pgfsetroundjoin%
\definecolor{currentfill}{rgb}{0.899543,0.847500,0.817789}%
\pgfsetfillcolor{currentfill}%
\pgfsetlinewidth{0.000000pt}%
\definecolor{currentstroke}{rgb}{0.000000,0.000000,0.000000}%
\pgfsetstrokecolor{currentstroke}%
\pgfsetdash{}{0pt}%
\pgfpathmoveto{\pgfqpoint{3.622205in}{2.891176in}}%
\pgfpathlineto{\pgfqpoint{2.349853in}{0.872712in}}%
\pgfpathlineto{\pgfqpoint{3.622205in}{2.891176in}}%
\pgfpathclose%
\pgfusepath{fill}%
\end{pgfscope}%
\begin{pgfscope}%
\pgfpathrectangle{\pgfqpoint{0.017500in}{0.017500in}}{\pgfqpoint{6.218282in}{3.580372in}} %
\pgfusepath{clip}%
\pgfsetbuttcap%
\pgfsetroundjoin%
\definecolor{currentfill}{rgb}{0.899543,0.847500,0.817789}%
\pgfsetfillcolor{currentfill}%
\pgfsetlinewidth{0.000000pt}%
\definecolor{currentstroke}{rgb}{0.000000,0.000000,0.000000}%
\pgfsetstrokecolor{currentstroke}%
\pgfsetdash{}{0pt}%
\pgfpathmoveto{\pgfqpoint{3.622205in}{2.891176in}}%
\pgfpathlineto{\pgfqpoint{2.349853in}{0.872712in}}%
\pgfpathlineto{\pgfqpoint{3.622205in}{2.891176in}}%
\pgfpathclose%
\pgfusepath{fill}%
\end{pgfscope}%
\begin{pgfscope}%
\pgfpathrectangle{\pgfqpoint{0.017500in}{0.017500in}}{\pgfqpoint{6.218282in}{3.580372in}} %
\pgfusepath{clip}%
\pgfsetbuttcap%
\pgfsetroundjoin%
\definecolor{currentfill}{rgb}{0.746838,0.140021,0.179996}%
\pgfsetfillcolor{currentfill}%
\pgfsetlinewidth{0.000000pt}%
\definecolor{currentstroke}{rgb}{0.000000,0.000000,0.000000}%
\pgfsetstrokecolor{currentstroke}%
\pgfsetdash{}{0pt}%
\pgfpathmoveto{\pgfqpoint{2.237924in}{2.752812in}}%
\pgfpathlineto{\pgfqpoint{2.674254in}{2.688576in}}%
\pgfpathlineto{\pgfqpoint{2.932968in}{2.820668in}}%
\pgfpathlineto{\pgfqpoint{2.499217in}{2.903684in}}%
\pgfpathlineto{\pgfqpoint{2.237924in}{2.752812in}}%
\pgfpathclose%
\pgfusepath{fill}%
\end{pgfscope}%
\begin{pgfscope}%
\pgfpathrectangle{\pgfqpoint{0.017500in}{0.017500in}}{\pgfqpoint{6.218282in}{3.580372in}} %
\pgfusepath{clip}%
\pgfsetbuttcap%
\pgfsetroundjoin%
\definecolor{currentfill}{rgb}{0.746838,0.140021,0.179996}%
\pgfsetfillcolor{currentfill}%
\pgfsetlinewidth{0.000000pt}%
\definecolor{currentstroke}{rgb}{0.000000,0.000000,0.000000}%
\pgfsetstrokecolor{currentstroke}%
\pgfsetdash{}{0pt}%
\pgfpathmoveto{\pgfqpoint{2.237924in}{2.752812in}}%
\pgfpathlineto{\pgfqpoint{2.674254in}{2.688576in}}%
\pgfpathlineto{\pgfqpoint{2.932968in}{2.820668in}}%
\pgfpathlineto{\pgfqpoint{2.499217in}{2.903684in}}%
\pgfpathlineto{\pgfqpoint{2.237924in}{2.752812in}}%
\pgfpathclose%
\pgfusepath{fill}%
\end{pgfscope}%
\begin{pgfscope}%
\pgfpathrectangle{\pgfqpoint{0.017500in}{0.017500in}}{\pgfqpoint{6.218282in}{3.580372in}} %
\pgfusepath{clip}%
\pgfsetbuttcap%
\pgfsetroundjoin%
\definecolor{currentfill}{rgb}{0.746838,0.140021,0.179996}%
\pgfsetfillcolor{currentfill}%
\pgfsetlinewidth{0.000000pt}%
\definecolor{currentstroke}{rgb}{0.000000,0.000000,0.000000}%
\pgfsetstrokecolor{currentstroke}%
\pgfsetdash{}{0pt}%
\pgfpathmoveto{\pgfqpoint{2.237924in}{2.752812in}}%
\pgfpathlineto{\pgfqpoint{2.674254in}{2.688576in}}%
\pgfpathlineto{\pgfqpoint{2.932968in}{2.820668in}}%
\pgfpathlineto{\pgfqpoint{2.499217in}{2.903684in}}%
\pgfpathlineto{\pgfqpoint{2.237924in}{2.752812in}}%
\pgfpathclose%
\pgfusepath{fill}%
\end{pgfscope}%
\begin{pgfscope}%
\pgfpathrectangle{\pgfqpoint{0.017500in}{0.017500in}}{\pgfqpoint{6.218282in}{3.580372in}} %
\pgfusepath{clip}%
\pgfsetbuttcap%
\pgfsetroundjoin%
\definecolor{currentfill}{rgb}{0.746838,0.140021,0.179996}%
\pgfsetfillcolor{currentfill}%
\pgfsetlinewidth{0.000000pt}%
\definecolor{currentstroke}{rgb}{0.000000,0.000000,0.000000}%
\pgfsetstrokecolor{currentstroke}%
\pgfsetdash{}{0pt}%
\pgfpathmoveto{\pgfqpoint{2.237924in}{2.752812in}}%
\pgfpathlineto{\pgfqpoint{2.674254in}{2.688576in}}%
\pgfpathlineto{\pgfqpoint{2.932968in}{2.820668in}}%
\pgfpathlineto{\pgfqpoint{2.499217in}{2.903684in}}%
\pgfpathlineto{\pgfqpoint{2.237924in}{2.752812in}}%
\pgfpathclose%
\pgfusepath{fill}%
\end{pgfscope}%
\begin{pgfscope}%
\pgfpathrectangle{\pgfqpoint{0.017500in}{0.017500in}}{\pgfqpoint{6.218282in}{3.580372in}} %
\pgfusepath{clip}%
\pgfsetbuttcap%
\pgfsetroundjoin%
\definecolor{currentfill}{rgb}{0.746838,0.140021,0.179996}%
\pgfsetfillcolor{currentfill}%
\pgfsetlinewidth{0.000000pt}%
\definecolor{currentstroke}{rgb}{0.000000,0.000000,0.000000}%
\pgfsetstrokecolor{currentstroke}%
\pgfsetdash{}{0pt}%
\pgfpathmoveto{\pgfqpoint{2.237924in}{2.752812in}}%
\pgfpathlineto{\pgfqpoint{2.674254in}{2.688576in}}%
\pgfpathlineto{\pgfqpoint{2.932968in}{2.820668in}}%
\pgfpathlineto{\pgfqpoint{2.499217in}{2.903684in}}%
\pgfpathlineto{\pgfqpoint{2.237924in}{2.752812in}}%
\pgfpathclose%
\pgfusepath{fill}%
\end{pgfscope}%
\begin{pgfscope}%
\pgfpathrectangle{\pgfqpoint{0.017500in}{0.017500in}}{\pgfqpoint{6.218282in}{3.580372in}} %
\pgfusepath{clip}%
\pgfsetbuttcap%
\pgfsetroundjoin%
\definecolor{currentfill}{rgb}{0.746838,0.140021,0.179996}%
\pgfsetfillcolor{currentfill}%
\pgfsetlinewidth{0.000000pt}%
\definecolor{currentstroke}{rgb}{0.000000,0.000000,0.000000}%
\pgfsetstrokecolor{currentstroke}%
\pgfsetdash{}{0pt}%
\pgfpathmoveto{\pgfqpoint{2.237924in}{2.752812in}}%
\pgfpathlineto{\pgfqpoint{2.674254in}{2.688576in}}%
\pgfpathlineto{\pgfqpoint{2.932968in}{2.820668in}}%
\pgfpathlineto{\pgfqpoint{2.499217in}{2.903684in}}%
\pgfpathlineto{\pgfqpoint{2.237924in}{2.752812in}}%
\pgfpathclose%
\pgfusepath{fill}%
\end{pgfscope}%
\begin{pgfscope}%
\pgfpathrectangle{\pgfqpoint{0.017500in}{0.017500in}}{\pgfqpoint{6.218282in}{3.580372in}} %
\pgfusepath{clip}%
\pgfsetbuttcap%
\pgfsetroundjoin%
\definecolor{currentfill}{rgb}{0.717435,0.051118,0.158737}%
\pgfsetfillcolor{currentfill}%
\pgfsetlinewidth{0.000000pt}%
\definecolor{currentstroke}{rgb}{0.000000,0.000000,0.000000}%
\pgfsetstrokecolor{currentstroke}%
\pgfsetdash{}{0pt}%
\pgfpathmoveto{\pgfqpoint{3.372093in}{2.761754in}}%
\pgfpathlineto{\pgfqpoint{3.622205in}{2.891176in}}%
\pgfpathlineto{\pgfqpoint{3.372093in}{2.761754in}}%
\pgfpathclose%
\pgfusepath{fill}%
\end{pgfscope}%
\begin{pgfscope}%
\pgfpathrectangle{\pgfqpoint{0.017500in}{0.017500in}}{\pgfqpoint{6.218282in}{3.580372in}} %
\pgfusepath{clip}%
\pgfsetbuttcap%
\pgfsetroundjoin%
\definecolor{currentfill}{rgb}{0.717435,0.051118,0.158737}%
\pgfsetfillcolor{currentfill}%
\pgfsetlinewidth{0.000000pt}%
\definecolor{currentstroke}{rgb}{0.000000,0.000000,0.000000}%
\pgfsetstrokecolor{currentstroke}%
\pgfsetdash{}{0pt}%
\pgfpathmoveto{\pgfqpoint{3.372093in}{2.761754in}}%
\pgfpathlineto{\pgfqpoint{3.622205in}{2.891176in}}%
\pgfpathlineto{\pgfqpoint{3.372093in}{2.761754in}}%
\pgfpathclose%
\pgfusepath{fill}%
\end{pgfscope}%
\begin{pgfscope}%
\pgfpathrectangle{\pgfqpoint{0.017500in}{0.017500in}}{\pgfqpoint{6.218282in}{3.580372in}} %
\pgfusepath{clip}%
\pgfsetbuttcap%
\pgfsetroundjoin%
\definecolor{currentfill}{rgb}{0.717435,0.051118,0.158737}%
\pgfsetfillcolor{currentfill}%
\pgfsetlinewidth{0.000000pt}%
\definecolor{currentstroke}{rgb}{0.000000,0.000000,0.000000}%
\pgfsetstrokecolor{currentstroke}%
\pgfsetdash{}{0pt}%
\pgfpathmoveto{\pgfqpoint{3.372093in}{2.761754in}}%
\pgfpathlineto{\pgfqpoint{3.622205in}{2.891176in}}%
\pgfpathlineto{\pgfqpoint{3.372093in}{2.761754in}}%
\pgfpathclose%
\pgfusepath{fill}%
\end{pgfscope}%
\begin{pgfscope}%
\pgfpathrectangle{\pgfqpoint{0.017500in}{0.017500in}}{\pgfqpoint{6.218282in}{3.580372in}} %
\pgfusepath{clip}%
\pgfsetbuttcap%
\pgfsetroundjoin%
\definecolor{currentfill}{rgb}{0.717435,0.051118,0.158737}%
\pgfsetfillcolor{currentfill}%
\pgfsetlinewidth{0.000000pt}%
\definecolor{currentstroke}{rgb}{0.000000,0.000000,0.000000}%
\pgfsetstrokecolor{currentstroke}%
\pgfsetdash{}{0pt}%
\pgfpathmoveto{\pgfqpoint{3.372093in}{2.761754in}}%
\pgfpathlineto{\pgfqpoint{3.622205in}{2.891176in}}%
\pgfpathlineto{\pgfqpoint{3.372093in}{2.761754in}}%
\pgfpathclose%
\pgfusepath{fill}%
\end{pgfscope}%
\begin{pgfscope}%
\pgfpathrectangle{\pgfqpoint{0.017500in}{0.017500in}}{\pgfqpoint{6.218282in}{3.580372in}} %
\pgfusepath{clip}%
\pgfsetbuttcap%
\pgfsetroundjoin%
\definecolor{currentfill}{rgb}{0.717435,0.051118,0.158737}%
\pgfsetfillcolor{currentfill}%
\pgfsetlinewidth{0.000000pt}%
\definecolor{currentstroke}{rgb}{0.000000,0.000000,0.000000}%
\pgfsetstrokecolor{currentstroke}%
\pgfsetdash{}{0pt}%
\pgfpathmoveto{\pgfqpoint{3.372093in}{2.761754in}}%
\pgfpathlineto{\pgfqpoint{3.622205in}{2.891176in}}%
\pgfpathlineto{\pgfqpoint{3.372093in}{2.761754in}}%
\pgfpathclose%
\pgfusepath{fill}%
\end{pgfscope}%
\begin{pgfscope}%
\pgfpathrectangle{\pgfqpoint{0.017500in}{0.017500in}}{\pgfqpoint{6.218282in}{3.580372in}} %
\pgfusepath{clip}%
\pgfsetbuttcap%
\pgfsetroundjoin%
\definecolor{currentfill}{rgb}{0.717435,0.051118,0.158737}%
\pgfsetfillcolor{currentfill}%
\pgfsetlinewidth{0.000000pt}%
\definecolor{currentstroke}{rgb}{0.000000,0.000000,0.000000}%
\pgfsetstrokecolor{currentstroke}%
\pgfsetdash{}{0pt}%
\pgfpathmoveto{\pgfqpoint{3.372093in}{2.761754in}}%
\pgfpathlineto{\pgfqpoint{3.622205in}{2.891176in}}%
\pgfpathlineto{\pgfqpoint{3.372093in}{2.761754in}}%
\pgfpathclose%
\pgfusepath{fill}%
\end{pgfscope}%
\begin{pgfscope}%
\pgfpathrectangle{\pgfqpoint{0.017500in}{0.017500in}}{\pgfqpoint{6.218282in}{3.580372in}} %
\pgfusepath{clip}%
\pgfsetbuttcap%
\pgfsetroundjoin%
\definecolor{currentfill}{rgb}{0.717435,0.051118,0.158737}%
\pgfsetfillcolor{currentfill}%
\pgfsetlinewidth{0.000000pt}%
\definecolor{currentstroke}{rgb}{0.000000,0.000000,0.000000}%
\pgfsetstrokecolor{currentstroke}%
\pgfsetdash{}{0pt}%
\pgfpathmoveto{\pgfqpoint{3.372093in}{2.761754in}}%
\pgfpathlineto{\pgfqpoint{3.622205in}{2.891176in}}%
\pgfpathlineto{\pgfqpoint{3.372093in}{2.761754in}}%
\pgfpathclose%
\pgfusepath{fill}%
\end{pgfscope}%
\begin{pgfscope}%
\pgfpathrectangle{\pgfqpoint{0.017500in}{0.017500in}}{\pgfqpoint{6.218282in}{3.580372in}} %
\pgfusepath{clip}%
\pgfsetbuttcap%
\pgfsetroundjoin%
\definecolor{currentfill}{rgb}{0.717435,0.051118,0.158737}%
\pgfsetfillcolor{currentfill}%
\pgfsetlinewidth{0.000000pt}%
\definecolor{currentstroke}{rgb}{0.000000,0.000000,0.000000}%
\pgfsetstrokecolor{currentstroke}%
\pgfsetdash{}{0pt}%
\pgfpathmoveto{\pgfqpoint{3.372093in}{2.761754in}}%
\pgfpathlineto{\pgfqpoint{3.622205in}{2.891176in}}%
\pgfpathlineto{\pgfqpoint{3.372093in}{2.761754in}}%
\pgfpathclose%
\pgfusepath{fill}%
\end{pgfscope}%
\begin{pgfscope}%
\pgfpathrectangle{\pgfqpoint{0.017500in}{0.017500in}}{\pgfqpoint{6.218282in}{3.580372in}} %
\pgfusepath{clip}%
\pgfsetbuttcap%
\pgfsetroundjoin%
\definecolor{currentfill}{rgb}{0.717435,0.051118,0.158737}%
\pgfsetfillcolor{currentfill}%
\pgfsetlinewidth{0.000000pt}%
\definecolor{currentstroke}{rgb}{0.000000,0.000000,0.000000}%
\pgfsetstrokecolor{currentstroke}%
\pgfsetdash{}{0pt}%
\pgfpathmoveto{\pgfqpoint{3.372093in}{2.761754in}}%
\pgfpathlineto{\pgfqpoint{3.622205in}{2.891176in}}%
\pgfpathlineto{\pgfqpoint{3.372093in}{2.761754in}}%
\pgfpathclose%
\pgfusepath{fill}%
\end{pgfscope}%
\begin{pgfscope}%
\pgfpathrectangle{\pgfqpoint{0.017500in}{0.017500in}}{\pgfqpoint{6.218282in}{3.580372in}} %
\pgfusepath{clip}%
\pgfsetbuttcap%
\pgfsetroundjoin%
\definecolor{currentfill}{rgb}{0.717435,0.051118,0.158737}%
\pgfsetfillcolor{currentfill}%
\pgfsetlinewidth{0.000000pt}%
\definecolor{currentstroke}{rgb}{0.000000,0.000000,0.000000}%
\pgfsetstrokecolor{currentstroke}%
\pgfsetdash{}{0pt}%
\pgfpathmoveto{\pgfqpoint{3.372093in}{2.761754in}}%
\pgfpathlineto{\pgfqpoint{3.622205in}{2.891176in}}%
\pgfpathlineto{\pgfqpoint{3.372093in}{2.761754in}}%
\pgfpathclose%
\pgfusepath{fill}%
\end{pgfscope}%
\begin{pgfscope}%
\pgfpathrectangle{\pgfqpoint{0.017500in}{0.017500in}}{\pgfqpoint{6.218282in}{3.580372in}} %
\pgfusepath{clip}%
\pgfsetbuttcap%
\pgfsetroundjoin%
\definecolor{currentfill}{rgb}{0.717435,0.051118,0.158737}%
\pgfsetfillcolor{currentfill}%
\pgfsetlinewidth{0.000000pt}%
\definecolor{currentstroke}{rgb}{0.000000,0.000000,0.000000}%
\pgfsetstrokecolor{currentstroke}%
\pgfsetdash{}{0pt}%
\pgfpathmoveto{\pgfqpoint{3.372093in}{2.761754in}}%
\pgfpathlineto{\pgfqpoint{3.622205in}{2.891176in}}%
\pgfpathlineto{\pgfqpoint{3.372093in}{2.761754in}}%
\pgfpathclose%
\pgfusepath{fill}%
\end{pgfscope}%
\begin{pgfscope}%
\pgfpathrectangle{\pgfqpoint{0.017500in}{0.017500in}}{\pgfqpoint{6.218282in}{3.580372in}} %
\pgfusepath{clip}%
\pgfsetbuttcap%
\pgfsetroundjoin%
\definecolor{currentfill}{rgb}{0.717435,0.051118,0.158737}%
\pgfsetfillcolor{currentfill}%
\pgfsetlinewidth{0.000000pt}%
\definecolor{currentstroke}{rgb}{0.000000,0.000000,0.000000}%
\pgfsetstrokecolor{currentstroke}%
\pgfsetdash{}{0pt}%
\pgfpathmoveto{\pgfqpoint{3.372093in}{2.761754in}}%
\pgfpathlineto{\pgfqpoint{3.622205in}{2.891176in}}%
\pgfpathlineto{\pgfqpoint{3.372093in}{2.761754in}}%
\pgfpathclose%
\pgfusepath{fill}%
\end{pgfscope}%
\begin{pgfscope}%
\pgfpathrectangle{\pgfqpoint{0.017500in}{0.017500in}}{\pgfqpoint{6.218282in}{3.580372in}} %
\pgfusepath{clip}%
\pgfsetbuttcap%
\pgfsetroundjoin%
\definecolor{currentfill}{rgb}{0.717435,0.051118,0.158737}%
\pgfsetfillcolor{currentfill}%
\pgfsetlinewidth{0.000000pt}%
\definecolor{currentstroke}{rgb}{0.000000,0.000000,0.000000}%
\pgfsetstrokecolor{currentstroke}%
\pgfsetdash{}{0pt}%
\pgfpathmoveto{\pgfqpoint{3.372093in}{2.761754in}}%
\pgfpathlineto{\pgfqpoint{3.622205in}{2.891176in}}%
\pgfpathlineto{\pgfqpoint{3.372093in}{2.761754in}}%
\pgfpathclose%
\pgfusepath{fill}%
\end{pgfscope}%
\begin{pgfscope}%
\pgfpathrectangle{\pgfqpoint{0.017500in}{0.017500in}}{\pgfqpoint{6.218282in}{3.580372in}} %
\pgfusepath{clip}%
\pgfsetbuttcap%
\pgfsetroundjoin%
\definecolor{currentfill}{rgb}{0.717435,0.051118,0.158737}%
\pgfsetfillcolor{currentfill}%
\pgfsetlinewidth{0.000000pt}%
\definecolor{currentstroke}{rgb}{0.000000,0.000000,0.000000}%
\pgfsetstrokecolor{currentstroke}%
\pgfsetdash{}{0pt}%
\pgfpathmoveto{\pgfqpoint{3.372093in}{2.761754in}}%
\pgfpathlineto{\pgfqpoint{3.622205in}{2.891176in}}%
\pgfpathlineto{\pgfqpoint{3.372093in}{2.761754in}}%
\pgfpathclose%
\pgfusepath{fill}%
\end{pgfscope}%
\begin{pgfscope}%
\pgfpathrectangle{\pgfqpoint{0.017500in}{0.017500in}}{\pgfqpoint{6.218282in}{3.580372in}} %
\pgfusepath{clip}%
\pgfsetbuttcap%
\pgfsetroundjoin%
\definecolor{currentfill}{rgb}{0.717435,0.051118,0.158737}%
\pgfsetfillcolor{currentfill}%
\pgfsetlinewidth{0.000000pt}%
\definecolor{currentstroke}{rgb}{0.000000,0.000000,0.000000}%
\pgfsetstrokecolor{currentstroke}%
\pgfsetdash{}{0pt}%
\pgfpathmoveto{\pgfqpoint{3.372093in}{2.761754in}}%
\pgfpathlineto{\pgfqpoint{3.622205in}{2.891176in}}%
\pgfpathlineto{\pgfqpoint{3.372093in}{2.761754in}}%
\pgfpathclose%
\pgfusepath{fill}%
\end{pgfscope}%
\begin{pgfscope}%
\pgfpathrectangle{\pgfqpoint{0.017500in}{0.017500in}}{\pgfqpoint{6.218282in}{3.580372in}} %
\pgfusepath{clip}%
\pgfsetbuttcap%
\pgfsetroundjoin%
\definecolor{currentfill}{rgb}{0.717435,0.051118,0.158737}%
\pgfsetfillcolor{currentfill}%
\pgfsetlinewidth{0.000000pt}%
\definecolor{currentstroke}{rgb}{0.000000,0.000000,0.000000}%
\pgfsetstrokecolor{currentstroke}%
\pgfsetdash{}{0pt}%
\pgfpathmoveto{\pgfqpoint{3.372093in}{2.761754in}}%
\pgfpathlineto{\pgfqpoint{3.622205in}{2.891176in}}%
\pgfpathlineto{\pgfqpoint{3.372093in}{2.761754in}}%
\pgfpathclose%
\pgfusepath{fill}%
\end{pgfscope}%
\begin{pgfscope}%
\pgfpathrectangle{\pgfqpoint{0.017500in}{0.017500in}}{\pgfqpoint{6.218282in}{3.580372in}} %
\pgfusepath{clip}%
\pgfsetbuttcap%
\pgfsetroundjoin%
\definecolor{currentfill}{rgb}{0.717435,0.051118,0.158737}%
\pgfsetfillcolor{currentfill}%
\pgfsetlinewidth{0.000000pt}%
\definecolor{currentstroke}{rgb}{0.000000,0.000000,0.000000}%
\pgfsetstrokecolor{currentstroke}%
\pgfsetdash{}{0pt}%
\pgfpathmoveto{\pgfqpoint{3.372093in}{2.761754in}}%
\pgfpathlineto{\pgfqpoint{3.622205in}{2.891176in}}%
\pgfpathlineto{\pgfqpoint{3.372093in}{2.761754in}}%
\pgfpathclose%
\pgfusepath{fill}%
\end{pgfscope}%
\begin{pgfscope}%
\pgfpathrectangle{\pgfqpoint{0.017500in}{0.017500in}}{\pgfqpoint{6.218282in}{3.580372in}} %
\pgfusepath{clip}%
\pgfsetbuttcap%
\pgfsetroundjoin%
\definecolor{currentfill}{rgb}{0.717435,0.051118,0.158737}%
\pgfsetfillcolor{currentfill}%
\pgfsetlinewidth{0.000000pt}%
\definecolor{currentstroke}{rgb}{0.000000,0.000000,0.000000}%
\pgfsetstrokecolor{currentstroke}%
\pgfsetdash{}{0pt}%
\pgfpathmoveto{\pgfqpoint{3.372093in}{2.761754in}}%
\pgfpathlineto{\pgfqpoint{3.622205in}{2.891176in}}%
\pgfpathlineto{\pgfqpoint{3.372093in}{2.761754in}}%
\pgfpathclose%
\pgfusepath{fill}%
\end{pgfscope}%
\begin{pgfscope}%
\pgfpathrectangle{\pgfqpoint{0.017500in}{0.017500in}}{\pgfqpoint{6.218282in}{3.580372in}} %
\pgfusepath{clip}%
\pgfsetbuttcap%
\pgfsetroundjoin%
\definecolor{currentfill}{rgb}{0.717435,0.051118,0.158737}%
\pgfsetfillcolor{currentfill}%
\pgfsetlinewidth{0.000000pt}%
\definecolor{currentstroke}{rgb}{0.000000,0.000000,0.000000}%
\pgfsetstrokecolor{currentstroke}%
\pgfsetdash{}{0pt}%
\pgfpathmoveto{\pgfqpoint{3.372093in}{2.761754in}}%
\pgfpathlineto{\pgfqpoint{3.622205in}{2.891176in}}%
\pgfpathlineto{\pgfqpoint{3.372093in}{2.761754in}}%
\pgfpathclose%
\pgfusepath{fill}%
\end{pgfscope}%
\begin{pgfscope}%
\pgfpathrectangle{\pgfqpoint{0.017500in}{0.017500in}}{\pgfqpoint{6.218282in}{3.580372in}} %
\pgfusepath{clip}%
\pgfsetbuttcap%
\pgfsetroundjoin%
\definecolor{currentfill}{rgb}{0.717435,0.051118,0.158737}%
\pgfsetfillcolor{currentfill}%
\pgfsetlinewidth{0.000000pt}%
\definecolor{currentstroke}{rgb}{0.000000,0.000000,0.000000}%
\pgfsetstrokecolor{currentstroke}%
\pgfsetdash{}{0pt}%
\pgfpathmoveto{\pgfqpoint{3.372093in}{2.761754in}}%
\pgfpathlineto{\pgfqpoint{3.622205in}{2.891176in}}%
\pgfpathlineto{\pgfqpoint{3.372093in}{2.761754in}}%
\pgfpathclose%
\pgfusepath{fill}%
\end{pgfscope}%
\begin{pgfscope}%
\pgfpathrectangle{\pgfqpoint{0.017500in}{0.017500in}}{\pgfqpoint{6.218282in}{3.580372in}} %
\pgfusepath{clip}%
\pgfsetbuttcap%
\pgfsetroundjoin%
\definecolor{currentfill}{rgb}{0.717435,0.051118,0.158737}%
\pgfsetfillcolor{currentfill}%
\pgfsetlinewidth{0.000000pt}%
\definecolor{currentstroke}{rgb}{0.000000,0.000000,0.000000}%
\pgfsetstrokecolor{currentstroke}%
\pgfsetdash{}{0pt}%
\pgfpathmoveto{\pgfqpoint{3.372093in}{2.761754in}}%
\pgfpathlineto{\pgfqpoint{3.622205in}{2.891176in}}%
\pgfpathlineto{\pgfqpoint{3.372093in}{2.761754in}}%
\pgfpathclose%
\pgfusepath{fill}%
\end{pgfscope}%
\begin{pgfscope}%
\pgfpathrectangle{\pgfqpoint{0.017500in}{0.017500in}}{\pgfqpoint{6.218282in}{3.580372in}} %
\pgfusepath{clip}%
\pgfsetbuttcap%
\pgfsetroundjoin%
\definecolor{currentfill}{rgb}{0.717435,0.051118,0.158737}%
\pgfsetfillcolor{currentfill}%
\pgfsetlinewidth{0.000000pt}%
\definecolor{currentstroke}{rgb}{0.000000,0.000000,0.000000}%
\pgfsetstrokecolor{currentstroke}%
\pgfsetdash{}{0pt}%
\pgfpathmoveto{\pgfqpoint{3.372093in}{2.761754in}}%
\pgfpathlineto{\pgfqpoint{3.622205in}{2.891176in}}%
\pgfpathlineto{\pgfqpoint{3.372093in}{2.761754in}}%
\pgfpathclose%
\pgfusepath{fill}%
\end{pgfscope}%
\begin{pgfscope}%
\pgfpathrectangle{\pgfqpoint{0.017500in}{0.017500in}}{\pgfqpoint{6.218282in}{3.580372in}} %
\pgfusepath{clip}%
\pgfsetbuttcap%
\pgfsetroundjoin%
\definecolor{currentfill}{rgb}{0.717435,0.051118,0.158737}%
\pgfsetfillcolor{currentfill}%
\pgfsetlinewidth{0.000000pt}%
\definecolor{currentstroke}{rgb}{0.000000,0.000000,0.000000}%
\pgfsetstrokecolor{currentstroke}%
\pgfsetdash{}{0pt}%
\pgfpathmoveto{\pgfqpoint{3.372093in}{2.761754in}}%
\pgfpathlineto{\pgfqpoint{3.622205in}{2.891176in}}%
\pgfpathlineto{\pgfqpoint{3.372093in}{2.761754in}}%
\pgfpathclose%
\pgfusepath{fill}%
\end{pgfscope}%
\begin{pgfscope}%
\pgfpathrectangle{\pgfqpoint{0.017500in}{0.017500in}}{\pgfqpoint{6.218282in}{3.580372in}} %
\pgfusepath{clip}%
\pgfsetbuttcap%
\pgfsetroundjoin%
\definecolor{currentfill}{rgb}{0.717435,0.051118,0.158737}%
\pgfsetfillcolor{currentfill}%
\pgfsetlinewidth{0.000000pt}%
\definecolor{currentstroke}{rgb}{0.000000,0.000000,0.000000}%
\pgfsetstrokecolor{currentstroke}%
\pgfsetdash{}{0pt}%
\pgfpathmoveto{\pgfqpoint{3.372093in}{2.761754in}}%
\pgfpathlineto{\pgfqpoint{3.622205in}{2.891176in}}%
\pgfpathlineto{\pgfqpoint{3.372093in}{2.761754in}}%
\pgfpathclose%
\pgfusepath{fill}%
\end{pgfscope}%
\begin{pgfscope}%
\pgfpathrectangle{\pgfqpoint{0.017500in}{0.017500in}}{\pgfqpoint{6.218282in}{3.580372in}} %
\pgfusepath{clip}%
\pgfsetbuttcap%
\pgfsetroundjoin%
\definecolor{currentfill}{rgb}{0.717435,0.051118,0.158737}%
\pgfsetfillcolor{currentfill}%
\pgfsetlinewidth{0.000000pt}%
\definecolor{currentstroke}{rgb}{0.000000,0.000000,0.000000}%
\pgfsetstrokecolor{currentstroke}%
\pgfsetdash{}{0pt}%
\pgfpathmoveto{\pgfqpoint{3.372093in}{2.761754in}}%
\pgfpathlineto{\pgfqpoint{3.622205in}{2.891176in}}%
\pgfpathlineto{\pgfqpoint{3.372093in}{2.761754in}}%
\pgfpathclose%
\pgfusepath{fill}%
\end{pgfscope}%
\begin{pgfscope}%
\pgfpathrectangle{\pgfqpoint{0.017500in}{0.017500in}}{\pgfqpoint{6.218282in}{3.580372in}} %
\pgfusepath{clip}%
\pgfsetbuttcap%
\pgfsetroundjoin%
\definecolor{currentfill}{rgb}{0.717435,0.051118,0.158737}%
\pgfsetfillcolor{currentfill}%
\pgfsetlinewidth{0.000000pt}%
\definecolor{currentstroke}{rgb}{0.000000,0.000000,0.000000}%
\pgfsetstrokecolor{currentstroke}%
\pgfsetdash{}{0pt}%
\pgfpathmoveto{\pgfqpoint{3.372093in}{2.761754in}}%
\pgfpathlineto{\pgfqpoint{3.622205in}{2.891176in}}%
\pgfpathlineto{\pgfqpoint{3.372093in}{2.761754in}}%
\pgfpathclose%
\pgfusepath{fill}%
\end{pgfscope}%
\begin{pgfscope}%
\pgfpathrectangle{\pgfqpoint{0.017500in}{0.017500in}}{\pgfqpoint{6.218282in}{3.580372in}} %
\pgfusepath{clip}%
\pgfsetbuttcap%
\pgfsetroundjoin%
\definecolor{currentfill}{rgb}{0.717435,0.051118,0.158737}%
\pgfsetfillcolor{currentfill}%
\pgfsetlinewidth{0.000000pt}%
\definecolor{currentstroke}{rgb}{0.000000,0.000000,0.000000}%
\pgfsetstrokecolor{currentstroke}%
\pgfsetdash{}{0pt}%
\pgfpathmoveto{\pgfqpoint{3.372093in}{2.761754in}}%
\pgfpathlineto{\pgfqpoint{3.622205in}{2.891176in}}%
\pgfpathlineto{\pgfqpoint{3.372093in}{2.761754in}}%
\pgfpathclose%
\pgfusepath{fill}%
\end{pgfscope}%
\begin{pgfscope}%
\pgfpathrectangle{\pgfqpoint{0.017500in}{0.017500in}}{\pgfqpoint{6.218282in}{3.580372in}} %
\pgfusepath{clip}%
\pgfsetbuttcap%
\pgfsetroundjoin%
\definecolor{currentfill}{rgb}{0.717435,0.051118,0.158737}%
\pgfsetfillcolor{currentfill}%
\pgfsetlinewidth{0.000000pt}%
\definecolor{currentstroke}{rgb}{0.000000,0.000000,0.000000}%
\pgfsetstrokecolor{currentstroke}%
\pgfsetdash{}{0pt}%
\pgfpathmoveto{\pgfqpoint{3.372093in}{2.761754in}}%
\pgfpathlineto{\pgfqpoint{3.622205in}{2.891176in}}%
\pgfpathlineto{\pgfqpoint{3.372093in}{2.761754in}}%
\pgfpathclose%
\pgfusepath{fill}%
\end{pgfscope}%
\begin{pgfscope}%
\pgfpathrectangle{\pgfqpoint{0.017500in}{0.017500in}}{\pgfqpoint{6.218282in}{3.580372in}} %
\pgfusepath{clip}%
\pgfsetbuttcap%
\pgfsetroundjoin%
\definecolor{currentfill}{rgb}{0.717435,0.051118,0.158737}%
\pgfsetfillcolor{currentfill}%
\pgfsetlinewidth{0.000000pt}%
\definecolor{currentstroke}{rgb}{0.000000,0.000000,0.000000}%
\pgfsetstrokecolor{currentstroke}%
\pgfsetdash{}{0pt}%
\pgfpathmoveto{\pgfqpoint{3.372093in}{2.761754in}}%
\pgfpathlineto{\pgfqpoint{3.622205in}{2.891176in}}%
\pgfpathlineto{\pgfqpoint{3.372093in}{2.761754in}}%
\pgfpathclose%
\pgfusepath{fill}%
\end{pgfscope}%
\begin{pgfscope}%
\pgfpathrectangle{\pgfqpoint{0.017500in}{0.017500in}}{\pgfqpoint{6.218282in}{3.580372in}} %
\pgfusepath{clip}%
\pgfsetbuttcap%
\pgfsetroundjoin%
\definecolor{currentfill}{rgb}{0.717435,0.051118,0.158737}%
\pgfsetfillcolor{currentfill}%
\pgfsetlinewidth{0.000000pt}%
\definecolor{currentstroke}{rgb}{0.000000,0.000000,0.000000}%
\pgfsetstrokecolor{currentstroke}%
\pgfsetdash{}{0pt}%
\pgfpathmoveto{\pgfqpoint{3.372093in}{2.761754in}}%
\pgfpathlineto{\pgfqpoint{3.622205in}{2.891176in}}%
\pgfpathlineto{\pgfqpoint{3.372093in}{2.761754in}}%
\pgfpathclose%
\pgfusepath{fill}%
\end{pgfscope}%
\begin{pgfscope}%
\pgfpathrectangle{\pgfqpoint{0.017500in}{0.017500in}}{\pgfqpoint{6.218282in}{3.580372in}} %
\pgfusepath{clip}%
\pgfsetbuttcap%
\pgfsetroundjoin%
\definecolor{currentfill}{rgb}{0.800601,0.850358,0.930008}%
\pgfsetfillcolor{currentfill}%
\pgfsetlinewidth{0.000000pt}%
\definecolor{currentstroke}{rgb}{0.000000,0.000000,0.000000}%
\pgfsetstrokecolor{currentstroke}%
\pgfsetdash{}{0pt}%
\pgfpathmoveto{\pgfqpoint{3.622205in}{2.891176in}}%
\pgfpathlineto{\pgfqpoint{4.063701in}{2.813820in}}%
\pgfpathlineto{\pgfqpoint{2.789950in}{0.798346in}}%
\pgfpathlineto{\pgfqpoint{2.349853in}{0.872712in}}%
\pgfpathlineto{\pgfqpoint{3.622205in}{2.891176in}}%
\pgfpathclose%
\pgfusepath{fill}%
\end{pgfscope}%
\begin{pgfscope}%
\pgfpathrectangle{\pgfqpoint{0.017500in}{0.017500in}}{\pgfqpoint{6.218282in}{3.580372in}} %
\pgfusepath{clip}%
\pgfsetbuttcap%
\pgfsetroundjoin%
\definecolor{currentfill}{rgb}{0.800601,0.850358,0.930008}%
\pgfsetfillcolor{currentfill}%
\pgfsetlinewidth{0.000000pt}%
\definecolor{currentstroke}{rgb}{0.000000,0.000000,0.000000}%
\pgfsetstrokecolor{currentstroke}%
\pgfsetdash{}{0pt}%
\pgfpathmoveto{\pgfqpoint{3.622205in}{2.891176in}}%
\pgfpathlineto{\pgfqpoint{4.063701in}{2.813820in}}%
\pgfpathlineto{\pgfqpoint{2.789950in}{0.798346in}}%
\pgfpathlineto{\pgfqpoint{2.349853in}{0.872712in}}%
\pgfpathlineto{\pgfqpoint{3.622205in}{2.891176in}}%
\pgfpathclose%
\pgfusepath{fill}%
\end{pgfscope}%
\begin{pgfscope}%
\pgfpathrectangle{\pgfqpoint{0.017500in}{0.017500in}}{\pgfqpoint{6.218282in}{3.580372in}} %
\pgfusepath{clip}%
\pgfsetbuttcap%
\pgfsetroundjoin%
\definecolor{currentfill}{rgb}{0.800601,0.850358,0.930008}%
\pgfsetfillcolor{currentfill}%
\pgfsetlinewidth{0.000000pt}%
\definecolor{currentstroke}{rgb}{0.000000,0.000000,0.000000}%
\pgfsetstrokecolor{currentstroke}%
\pgfsetdash{}{0pt}%
\pgfpathmoveto{\pgfqpoint{3.622205in}{2.891176in}}%
\pgfpathlineto{\pgfqpoint{4.063701in}{2.813820in}}%
\pgfpathlineto{\pgfqpoint{2.789950in}{0.798346in}}%
\pgfpathlineto{\pgfqpoint{2.349853in}{0.872712in}}%
\pgfpathlineto{\pgfqpoint{3.622205in}{2.891176in}}%
\pgfpathclose%
\pgfusepath{fill}%
\end{pgfscope}%
\begin{pgfscope}%
\pgfpathrectangle{\pgfqpoint{0.017500in}{0.017500in}}{\pgfqpoint{6.218282in}{3.580372in}} %
\pgfusepath{clip}%
\pgfsetbuttcap%
\pgfsetroundjoin%
\definecolor{currentfill}{rgb}{0.800601,0.850358,0.930008}%
\pgfsetfillcolor{currentfill}%
\pgfsetlinewidth{0.000000pt}%
\definecolor{currentstroke}{rgb}{0.000000,0.000000,0.000000}%
\pgfsetstrokecolor{currentstroke}%
\pgfsetdash{}{0pt}%
\pgfpathmoveto{\pgfqpoint{3.622205in}{2.891176in}}%
\pgfpathlineto{\pgfqpoint{4.063701in}{2.813820in}}%
\pgfpathlineto{\pgfqpoint{2.789950in}{0.798346in}}%
\pgfpathlineto{\pgfqpoint{2.349853in}{0.872712in}}%
\pgfpathlineto{\pgfqpoint{3.622205in}{2.891176in}}%
\pgfpathclose%
\pgfusepath{fill}%
\end{pgfscope}%
\begin{pgfscope}%
\pgfpathrectangle{\pgfqpoint{0.017500in}{0.017500in}}{\pgfqpoint{6.218282in}{3.580372in}} %
\pgfusepath{clip}%
\pgfsetbuttcap%
\pgfsetroundjoin%
\definecolor{currentfill}{rgb}{0.800601,0.850358,0.930008}%
\pgfsetfillcolor{currentfill}%
\pgfsetlinewidth{0.000000pt}%
\definecolor{currentstroke}{rgb}{0.000000,0.000000,0.000000}%
\pgfsetstrokecolor{currentstroke}%
\pgfsetdash{}{0pt}%
\pgfpathmoveto{\pgfqpoint{3.622205in}{2.891176in}}%
\pgfpathlineto{\pgfqpoint{4.063701in}{2.813820in}}%
\pgfpathlineto{\pgfqpoint{2.789950in}{0.798346in}}%
\pgfpathlineto{\pgfqpoint{2.349853in}{0.872712in}}%
\pgfpathlineto{\pgfqpoint{3.622205in}{2.891176in}}%
\pgfpathclose%
\pgfusepath{fill}%
\end{pgfscope}%
\begin{pgfscope}%
\pgfpathrectangle{\pgfqpoint{0.017500in}{0.017500in}}{\pgfqpoint{6.218282in}{3.580372in}} %
\pgfusepath{clip}%
\pgfsetbuttcap%
\pgfsetroundjoin%
\definecolor{currentfill}{rgb}{0.723315,0.068898,0.162989}%
\pgfsetfillcolor{currentfill}%
\pgfsetlinewidth{0.000000pt}%
\definecolor{currentstroke}{rgb}{0.000000,0.000000,0.000000}%
\pgfsetstrokecolor{currentstroke}%
\pgfsetdash{}{0pt}%
\pgfpathmoveto{\pgfqpoint{3.372093in}{2.761754in}}%
\pgfpathlineto{\pgfqpoint{3.816789in}{2.677291in}}%
\pgfpathlineto{\pgfqpoint{4.063701in}{2.813820in}}%
\pgfpathlineto{\pgfqpoint{3.622205in}{2.891176in}}%
\pgfpathlineto{\pgfqpoint{3.372093in}{2.761754in}}%
\pgfpathclose%
\pgfusepath{fill}%
\end{pgfscope}%
\begin{pgfscope}%
\pgfpathrectangle{\pgfqpoint{0.017500in}{0.017500in}}{\pgfqpoint{6.218282in}{3.580372in}} %
\pgfusepath{clip}%
\pgfsetbuttcap%
\pgfsetroundjoin%
\definecolor{currentfill}{rgb}{0.723315,0.068898,0.162989}%
\pgfsetfillcolor{currentfill}%
\pgfsetlinewidth{0.000000pt}%
\definecolor{currentstroke}{rgb}{0.000000,0.000000,0.000000}%
\pgfsetstrokecolor{currentstroke}%
\pgfsetdash{}{0pt}%
\pgfpathmoveto{\pgfqpoint{3.372093in}{2.761754in}}%
\pgfpathlineto{\pgfqpoint{3.816789in}{2.677291in}}%
\pgfpathlineto{\pgfqpoint{4.063701in}{2.813820in}}%
\pgfpathlineto{\pgfqpoint{3.622205in}{2.891176in}}%
\pgfpathlineto{\pgfqpoint{3.372093in}{2.761754in}}%
\pgfpathclose%
\pgfusepath{fill}%
\end{pgfscope}%
\begin{pgfscope}%
\pgfpathrectangle{\pgfqpoint{0.017500in}{0.017500in}}{\pgfqpoint{6.218282in}{3.580372in}} %
\pgfusepath{clip}%
\pgfsetbuttcap%
\pgfsetroundjoin%
\definecolor{currentfill}{rgb}{0.723315,0.068898,0.162989}%
\pgfsetfillcolor{currentfill}%
\pgfsetlinewidth{0.000000pt}%
\definecolor{currentstroke}{rgb}{0.000000,0.000000,0.000000}%
\pgfsetstrokecolor{currentstroke}%
\pgfsetdash{}{0pt}%
\pgfpathmoveto{\pgfqpoint{3.372093in}{2.761754in}}%
\pgfpathlineto{\pgfqpoint{3.816789in}{2.677291in}}%
\pgfpathlineto{\pgfqpoint{4.063701in}{2.813820in}}%
\pgfpathlineto{\pgfqpoint{3.622205in}{2.891176in}}%
\pgfpathlineto{\pgfqpoint{3.372093in}{2.761754in}}%
\pgfpathclose%
\pgfusepath{fill}%
\end{pgfscope}%
\begin{pgfscope}%
\pgfpathrectangle{\pgfqpoint{0.017500in}{0.017500in}}{\pgfqpoint{6.218282in}{3.580372in}} %
\pgfusepath{clip}%
\pgfsetbuttcap%
\pgfsetroundjoin%
\definecolor{currentfill}{rgb}{0.723315,0.068898,0.162989}%
\pgfsetfillcolor{currentfill}%
\pgfsetlinewidth{0.000000pt}%
\definecolor{currentstroke}{rgb}{0.000000,0.000000,0.000000}%
\pgfsetstrokecolor{currentstroke}%
\pgfsetdash{}{0pt}%
\pgfpathmoveto{\pgfqpoint{3.372093in}{2.761754in}}%
\pgfpathlineto{\pgfqpoint{3.816789in}{2.677291in}}%
\pgfpathlineto{\pgfqpoint{4.063701in}{2.813820in}}%
\pgfpathlineto{\pgfqpoint{3.622205in}{2.891176in}}%
\pgfpathlineto{\pgfqpoint{3.372093in}{2.761754in}}%
\pgfpathclose%
\pgfusepath{fill}%
\end{pgfscope}%
\begin{pgfscope}%
\pgfpathrectangle{\pgfqpoint{0.017500in}{0.017500in}}{\pgfqpoint{6.218282in}{3.580372in}} %
\pgfusepath{clip}%
\pgfsetbuttcap%
\pgfsetroundjoin%
\definecolor{currentfill}{rgb}{0.723315,0.068898,0.162989}%
\pgfsetfillcolor{currentfill}%
\pgfsetlinewidth{0.000000pt}%
\definecolor{currentstroke}{rgb}{0.000000,0.000000,0.000000}%
\pgfsetstrokecolor{currentstroke}%
\pgfsetdash{}{0pt}%
\pgfpathmoveto{\pgfqpoint{3.372093in}{2.761754in}}%
\pgfpathlineto{\pgfqpoint{3.816789in}{2.677291in}}%
\pgfpathlineto{\pgfqpoint{4.063701in}{2.813820in}}%
\pgfpathlineto{\pgfqpoint{3.622205in}{2.891176in}}%
\pgfpathlineto{\pgfqpoint{3.372093in}{2.761754in}}%
\pgfpathclose%
\pgfusepath{fill}%
\end{pgfscope}%
\begin{pgfscope}%
\pgfpathrectangle{\pgfqpoint{0.017500in}{0.017500in}}{\pgfqpoint{6.218282in}{3.580372in}} %
\pgfusepath{clip}%
\pgfsetbuttcap%
\pgfsetroundjoin%
\definecolor{currentfill}{rgb}{0.723315,0.068898,0.162989}%
\pgfsetfillcolor{currentfill}%
\pgfsetlinewidth{0.000000pt}%
\definecolor{currentstroke}{rgb}{0.000000,0.000000,0.000000}%
\pgfsetstrokecolor{currentstroke}%
\pgfsetdash{}{0pt}%
\pgfpathmoveto{\pgfqpoint{3.372093in}{2.761754in}}%
\pgfpathlineto{\pgfqpoint{3.816789in}{2.677291in}}%
\pgfpathlineto{\pgfqpoint{4.063701in}{2.813820in}}%
\pgfpathlineto{\pgfqpoint{3.622205in}{2.891176in}}%
\pgfpathlineto{\pgfqpoint{3.372093in}{2.761754in}}%
\pgfpathclose%
\pgfusepath{fill}%
\end{pgfscope}%
\begin{pgfscope}%
\pgfpathrectangle{\pgfqpoint{0.017500in}{0.017500in}}{\pgfqpoint{6.218282in}{3.580372in}} %
\pgfusepath{clip}%
\pgfsetbuttcap%
\pgfsetroundjoin%
\definecolor{currentfill}{rgb}{0.243520,0.319189,0.771672}%
\pgfsetfillcolor{currentfill}%
\pgfsetlinewidth{0.000000pt}%
\definecolor{currentstroke}{rgb}{0.000000,0.000000,0.000000}%
\pgfsetstrokecolor{currentstroke}%
\pgfsetdash{}{0pt}%
\pgfpathmoveto{\pgfqpoint{1.486050in}{1.042638in}}%
\pgfpathlineto{\pgfqpoint{1.701778in}{2.375902in}}%
\pgfpathlineto{\pgfqpoint{1.486050in}{1.042638in}}%
\pgfpathclose%
\pgfusepath{fill}%
\end{pgfscope}%
\begin{pgfscope}%
\pgfpathrectangle{\pgfqpoint{0.017500in}{0.017500in}}{\pgfqpoint{6.218282in}{3.580372in}} %
\pgfusepath{clip}%
\pgfsetbuttcap%
\pgfsetroundjoin%
\definecolor{currentfill}{rgb}{0.243520,0.319189,0.771672}%
\pgfsetfillcolor{currentfill}%
\pgfsetlinewidth{0.000000pt}%
\definecolor{currentstroke}{rgb}{0.000000,0.000000,0.000000}%
\pgfsetstrokecolor{currentstroke}%
\pgfsetdash{}{0pt}%
\pgfpathmoveto{\pgfqpoint{1.486050in}{1.042638in}}%
\pgfpathlineto{\pgfqpoint{1.701778in}{2.375902in}}%
\pgfpathlineto{\pgfqpoint{1.486050in}{1.042638in}}%
\pgfpathclose%
\pgfusepath{fill}%
\end{pgfscope}%
\begin{pgfscope}%
\pgfpathrectangle{\pgfqpoint{0.017500in}{0.017500in}}{\pgfqpoint{6.218282in}{3.580372in}} %
\pgfusepath{clip}%
\pgfsetbuttcap%
\pgfsetroundjoin%
\definecolor{currentfill}{rgb}{0.243520,0.319189,0.771672}%
\pgfsetfillcolor{currentfill}%
\pgfsetlinewidth{0.000000pt}%
\definecolor{currentstroke}{rgb}{0.000000,0.000000,0.000000}%
\pgfsetstrokecolor{currentstroke}%
\pgfsetdash{}{0pt}%
\pgfpathmoveto{\pgfqpoint{1.486050in}{1.042638in}}%
\pgfpathlineto{\pgfqpoint{1.701778in}{2.375902in}}%
\pgfpathlineto{\pgfqpoint{1.486050in}{1.042638in}}%
\pgfpathclose%
\pgfusepath{fill}%
\end{pgfscope}%
\begin{pgfscope}%
\pgfpathrectangle{\pgfqpoint{0.017500in}{0.017500in}}{\pgfqpoint{6.218282in}{3.580372in}} %
\pgfusepath{clip}%
\pgfsetbuttcap%
\pgfsetroundjoin%
\definecolor{currentfill}{rgb}{0.243520,0.319189,0.771672}%
\pgfsetfillcolor{currentfill}%
\pgfsetlinewidth{0.000000pt}%
\definecolor{currentstroke}{rgb}{0.000000,0.000000,0.000000}%
\pgfsetstrokecolor{currentstroke}%
\pgfsetdash{}{0pt}%
\pgfpathmoveto{\pgfqpoint{1.486050in}{1.042638in}}%
\pgfpathlineto{\pgfqpoint{1.701778in}{2.375902in}}%
\pgfpathlineto{\pgfqpoint{1.486050in}{1.042638in}}%
\pgfpathclose%
\pgfusepath{fill}%
\end{pgfscope}%
\begin{pgfscope}%
\pgfpathrectangle{\pgfqpoint{0.017500in}{0.017500in}}{\pgfqpoint{6.218282in}{3.580372in}} %
\pgfusepath{clip}%
\pgfsetbuttcap%
\pgfsetroundjoin%
\definecolor{currentfill}{rgb}{0.243520,0.319189,0.771672}%
\pgfsetfillcolor{currentfill}%
\pgfsetlinewidth{0.000000pt}%
\definecolor{currentstroke}{rgb}{0.000000,0.000000,0.000000}%
\pgfsetstrokecolor{currentstroke}%
\pgfsetdash{}{0pt}%
\pgfpathmoveto{\pgfqpoint{1.486050in}{1.042638in}}%
\pgfpathlineto{\pgfqpoint{1.701778in}{2.375902in}}%
\pgfpathlineto{\pgfqpoint{1.486050in}{1.042638in}}%
\pgfpathclose%
\pgfusepath{fill}%
\end{pgfscope}%
\begin{pgfscope}%
\pgfpathrectangle{\pgfqpoint{0.017500in}{0.017500in}}{\pgfqpoint{6.218282in}{3.580372in}} %
\pgfusepath{clip}%
\pgfsetbuttcap%
\pgfsetroundjoin%
\definecolor{currentfill}{rgb}{0.243520,0.319189,0.771672}%
\pgfsetfillcolor{currentfill}%
\pgfsetlinewidth{0.000000pt}%
\definecolor{currentstroke}{rgb}{0.000000,0.000000,0.000000}%
\pgfsetstrokecolor{currentstroke}%
\pgfsetdash{}{0pt}%
\pgfpathmoveto{\pgfqpoint{1.486050in}{1.042638in}}%
\pgfpathlineto{\pgfqpoint{1.701778in}{2.375902in}}%
\pgfpathlineto{\pgfqpoint{1.486050in}{1.042638in}}%
\pgfpathclose%
\pgfusepath{fill}%
\end{pgfscope}%
\begin{pgfscope}%
\pgfpathrectangle{\pgfqpoint{0.017500in}{0.017500in}}{\pgfqpoint{6.218282in}{3.580372in}} %
\pgfusepath{clip}%
\pgfsetbuttcap%
\pgfsetroundjoin%
\definecolor{currentfill}{rgb}{0.243520,0.319189,0.771672}%
\pgfsetfillcolor{currentfill}%
\pgfsetlinewidth{0.000000pt}%
\definecolor{currentstroke}{rgb}{0.000000,0.000000,0.000000}%
\pgfsetstrokecolor{currentstroke}%
\pgfsetdash{}{0pt}%
\pgfpathmoveto{\pgfqpoint{1.486050in}{1.042638in}}%
\pgfpathlineto{\pgfqpoint{1.701778in}{2.375902in}}%
\pgfpathlineto{\pgfqpoint{1.486050in}{1.042638in}}%
\pgfpathclose%
\pgfusepath{fill}%
\end{pgfscope}%
\begin{pgfscope}%
\pgfpathrectangle{\pgfqpoint{0.017500in}{0.017500in}}{\pgfqpoint{6.218282in}{3.580372in}} %
\pgfusepath{clip}%
\pgfsetbuttcap%
\pgfsetroundjoin%
\definecolor{currentfill}{rgb}{0.243520,0.319189,0.771672}%
\pgfsetfillcolor{currentfill}%
\pgfsetlinewidth{0.000000pt}%
\definecolor{currentstroke}{rgb}{0.000000,0.000000,0.000000}%
\pgfsetstrokecolor{currentstroke}%
\pgfsetdash{}{0pt}%
\pgfpathmoveto{\pgfqpoint{1.486050in}{1.042638in}}%
\pgfpathlineto{\pgfqpoint{1.701778in}{2.375902in}}%
\pgfpathlineto{\pgfqpoint{1.486050in}{1.042638in}}%
\pgfpathclose%
\pgfusepath{fill}%
\end{pgfscope}%
\begin{pgfscope}%
\pgfpathrectangle{\pgfqpoint{0.017500in}{0.017500in}}{\pgfqpoint{6.218282in}{3.580372in}} %
\pgfusepath{clip}%
\pgfsetbuttcap%
\pgfsetroundjoin%
\definecolor{currentfill}{rgb}{0.243520,0.319189,0.771672}%
\pgfsetfillcolor{currentfill}%
\pgfsetlinewidth{0.000000pt}%
\definecolor{currentstroke}{rgb}{0.000000,0.000000,0.000000}%
\pgfsetstrokecolor{currentstroke}%
\pgfsetdash{}{0pt}%
\pgfpathmoveto{\pgfqpoint{1.486050in}{1.042638in}}%
\pgfpathlineto{\pgfqpoint{1.701778in}{2.375902in}}%
\pgfpathlineto{\pgfqpoint{1.486050in}{1.042638in}}%
\pgfpathclose%
\pgfusepath{fill}%
\end{pgfscope}%
\begin{pgfscope}%
\pgfpathrectangle{\pgfqpoint{0.017500in}{0.017500in}}{\pgfqpoint{6.218282in}{3.580372in}} %
\pgfusepath{clip}%
\pgfsetbuttcap%
\pgfsetroundjoin%
\definecolor{currentfill}{rgb}{0.243520,0.319189,0.771672}%
\pgfsetfillcolor{currentfill}%
\pgfsetlinewidth{0.000000pt}%
\definecolor{currentstroke}{rgb}{0.000000,0.000000,0.000000}%
\pgfsetstrokecolor{currentstroke}%
\pgfsetdash{}{0pt}%
\pgfpathmoveto{\pgfqpoint{1.486050in}{1.042638in}}%
\pgfpathlineto{\pgfqpoint{1.701778in}{2.375902in}}%
\pgfpathlineto{\pgfqpoint{1.486050in}{1.042638in}}%
\pgfpathclose%
\pgfusepath{fill}%
\end{pgfscope}%
\begin{pgfscope}%
\pgfpathrectangle{\pgfqpoint{0.017500in}{0.017500in}}{\pgfqpoint{6.218282in}{3.580372in}} %
\pgfusepath{clip}%
\pgfsetbuttcap%
\pgfsetroundjoin%
\definecolor{currentfill}{rgb}{0.243520,0.319189,0.771672}%
\pgfsetfillcolor{currentfill}%
\pgfsetlinewidth{0.000000pt}%
\definecolor{currentstroke}{rgb}{0.000000,0.000000,0.000000}%
\pgfsetstrokecolor{currentstroke}%
\pgfsetdash{}{0pt}%
\pgfpathmoveto{\pgfqpoint{1.486050in}{1.042638in}}%
\pgfpathlineto{\pgfqpoint{1.701778in}{2.375902in}}%
\pgfpathlineto{\pgfqpoint{1.486050in}{1.042638in}}%
\pgfpathclose%
\pgfusepath{fill}%
\end{pgfscope}%
\begin{pgfscope}%
\pgfpathrectangle{\pgfqpoint{0.017500in}{0.017500in}}{\pgfqpoint{6.218282in}{3.580372in}} %
\pgfusepath{clip}%
\pgfsetbuttcap%
\pgfsetroundjoin%
\definecolor{currentfill}{rgb}{0.243520,0.319189,0.771672}%
\pgfsetfillcolor{currentfill}%
\pgfsetlinewidth{0.000000pt}%
\definecolor{currentstroke}{rgb}{0.000000,0.000000,0.000000}%
\pgfsetstrokecolor{currentstroke}%
\pgfsetdash{}{0pt}%
\pgfpathmoveto{\pgfqpoint{1.486050in}{1.042638in}}%
\pgfpathlineto{\pgfqpoint{1.701778in}{2.375902in}}%
\pgfpathlineto{\pgfqpoint{1.486050in}{1.042638in}}%
\pgfpathclose%
\pgfusepath{fill}%
\end{pgfscope}%
\begin{pgfscope}%
\pgfpathrectangle{\pgfqpoint{0.017500in}{0.017500in}}{\pgfqpoint{6.218282in}{3.580372in}} %
\pgfusepath{clip}%
\pgfsetbuttcap%
\pgfsetroundjoin%
\definecolor{currentfill}{rgb}{0.243520,0.319189,0.771672}%
\pgfsetfillcolor{currentfill}%
\pgfsetlinewidth{0.000000pt}%
\definecolor{currentstroke}{rgb}{0.000000,0.000000,0.000000}%
\pgfsetstrokecolor{currentstroke}%
\pgfsetdash{}{0pt}%
\pgfpathmoveto{\pgfqpoint{1.486050in}{1.042638in}}%
\pgfpathlineto{\pgfqpoint{1.701778in}{2.375902in}}%
\pgfpathlineto{\pgfqpoint{1.486050in}{1.042638in}}%
\pgfpathclose%
\pgfusepath{fill}%
\end{pgfscope}%
\begin{pgfscope}%
\pgfpathrectangle{\pgfqpoint{0.017500in}{0.017500in}}{\pgfqpoint{6.218282in}{3.580372in}} %
\pgfusepath{clip}%
\pgfsetbuttcap%
\pgfsetroundjoin%
\definecolor{currentfill}{rgb}{0.243520,0.319189,0.771672}%
\pgfsetfillcolor{currentfill}%
\pgfsetlinewidth{0.000000pt}%
\definecolor{currentstroke}{rgb}{0.000000,0.000000,0.000000}%
\pgfsetstrokecolor{currentstroke}%
\pgfsetdash{}{0pt}%
\pgfpathmoveto{\pgfqpoint{1.486050in}{1.042638in}}%
\pgfpathlineto{\pgfqpoint{1.701778in}{2.375902in}}%
\pgfpathlineto{\pgfqpoint{1.486050in}{1.042638in}}%
\pgfpathclose%
\pgfusepath{fill}%
\end{pgfscope}%
\begin{pgfscope}%
\pgfpathrectangle{\pgfqpoint{0.017500in}{0.017500in}}{\pgfqpoint{6.218282in}{3.580372in}} %
\pgfusepath{clip}%
\pgfsetbuttcap%
\pgfsetroundjoin%
\definecolor{currentfill}{rgb}{0.243520,0.319189,0.771672}%
\pgfsetfillcolor{currentfill}%
\pgfsetlinewidth{0.000000pt}%
\definecolor{currentstroke}{rgb}{0.000000,0.000000,0.000000}%
\pgfsetstrokecolor{currentstroke}%
\pgfsetdash{}{0pt}%
\pgfpathmoveto{\pgfqpoint{1.486050in}{1.042638in}}%
\pgfpathlineto{\pgfqpoint{1.701778in}{2.375902in}}%
\pgfpathlineto{\pgfqpoint{1.486050in}{1.042638in}}%
\pgfpathclose%
\pgfusepath{fill}%
\end{pgfscope}%
\begin{pgfscope}%
\pgfpathrectangle{\pgfqpoint{0.017500in}{0.017500in}}{\pgfqpoint{6.218282in}{3.580372in}} %
\pgfusepath{clip}%
\pgfsetbuttcap%
\pgfsetroundjoin%
\definecolor{currentfill}{rgb}{0.243520,0.319189,0.771672}%
\pgfsetfillcolor{currentfill}%
\pgfsetlinewidth{0.000000pt}%
\definecolor{currentstroke}{rgb}{0.000000,0.000000,0.000000}%
\pgfsetstrokecolor{currentstroke}%
\pgfsetdash{}{0pt}%
\pgfpathmoveto{\pgfqpoint{1.486050in}{1.042638in}}%
\pgfpathlineto{\pgfqpoint{1.701778in}{2.375902in}}%
\pgfpathlineto{\pgfqpoint{1.486050in}{1.042638in}}%
\pgfpathclose%
\pgfusepath{fill}%
\end{pgfscope}%
\begin{pgfscope}%
\pgfpathrectangle{\pgfqpoint{0.017500in}{0.017500in}}{\pgfqpoint{6.218282in}{3.580372in}} %
\pgfusepath{clip}%
\pgfsetbuttcap%
\pgfsetroundjoin%
\definecolor{currentfill}{rgb}{0.243520,0.319189,0.771672}%
\pgfsetfillcolor{currentfill}%
\pgfsetlinewidth{0.000000pt}%
\definecolor{currentstroke}{rgb}{0.000000,0.000000,0.000000}%
\pgfsetstrokecolor{currentstroke}%
\pgfsetdash{}{0pt}%
\pgfpathmoveto{\pgfqpoint{1.486050in}{1.042638in}}%
\pgfpathlineto{\pgfqpoint{1.701778in}{2.375902in}}%
\pgfpathlineto{\pgfqpoint{1.486050in}{1.042638in}}%
\pgfpathclose%
\pgfusepath{fill}%
\end{pgfscope}%
\begin{pgfscope}%
\pgfpathrectangle{\pgfqpoint{0.017500in}{0.017500in}}{\pgfqpoint{6.218282in}{3.580372in}} %
\pgfusepath{clip}%
\pgfsetbuttcap%
\pgfsetroundjoin%
\definecolor{currentfill}{rgb}{0.243520,0.319189,0.771672}%
\pgfsetfillcolor{currentfill}%
\pgfsetlinewidth{0.000000pt}%
\definecolor{currentstroke}{rgb}{0.000000,0.000000,0.000000}%
\pgfsetstrokecolor{currentstroke}%
\pgfsetdash{}{0pt}%
\pgfpathmoveto{\pgfqpoint{1.486050in}{1.042638in}}%
\pgfpathlineto{\pgfqpoint{1.701778in}{2.375902in}}%
\pgfpathlineto{\pgfqpoint{1.486050in}{1.042638in}}%
\pgfpathclose%
\pgfusepath{fill}%
\end{pgfscope}%
\begin{pgfscope}%
\pgfpathrectangle{\pgfqpoint{0.017500in}{0.017500in}}{\pgfqpoint{6.218282in}{3.580372in}} %
\pgfusepath{clip}%
\pgfsetbuttcap%
\pgfsetroundjoin%
\definecolor{currentfill}{rgb}{0.243520,0.319189,0.771672}%
\pgfsetfillcolor{currentfill}%
\pgfsetlinewidth{0.000000pt}%
\definecolor{currentstroke}{rgb}{0.000000,0.000000,0.000000}%
\pgfsetstrokecolor{currentstroke}%
\pgfsetdash{}{0pt}%
\pgfpathmoveto{\pgfqpoint{1.486050in}{1.042638in}}%
\pgfpathlineto{\pgfqpoint{1.701778in}{2.375902in}}%
\pgfpathlineto{\pgfqpoint{1.486050in}{1.042638in}}%
\pgfpathclose%
\pgfusepath{fill}%
\end{pgfscope}%
\begin{pgfscope}%
\pgfpathrectangle{\pgfqpoint{0.017500in}{0.017500in}}{\pgfqpoint{6.218282in}{3.580372in}} %
\pgfusepath{clip}%
\pgfsetbuttcap%
\pgfsetroundjoin%
\definecolor{currentfill}{rgb}{0.243520,0.319189,0.771672}%
\pgfsetfillcolor{currentfill}%
\pgfsetlinewidth{0.000000pt}%
\definecolor{currentstroke}{rgb}{0.000000,0.000000,0.000000}%
\pgfsetstrokecolor{currentstroke}%
\pgfsetdash{}{0pt}%
\pgfpathmoveto{\pgfqpoint{1.486050in}{1.042638in}}%
\pgfpathlineto{\pgfqpoint{1.701778in}{2.375902in}}%
\pgfpathlineto{\pgfqpoint{1.486050in}{1.042638in}}%
\pgfpathclose%
\pgfusepath{fill}%
\end{pgfscope}%
\begin{pgfscope}%
\pgfpathrectangle{\pgfqpoint{0.017500in}{0.017500in}}{\pgfqpoint{6.218282in}{3.580372in}} %
\pgfusepath{clip}%
\pgfsetbuttcap%
\pgfsetroundjoin%
\definecolor{currentfill}{rgb}{0.243520,0.319189,0.771672}%
\pgfsetfillcolor{currentfill}%
\pgfsetlinewidth{0.000000pt}%
\definecolor{currentstroke}{rgb}{0.000000,0.000000,0.000000}%
\pgfsetstrokecolor{currentstroke}%
\pgfsetdash{}{0pt}%
\pgfpathmoveto{\pgfqpoint{1.486050in}{1.042638in}}%
\pgfpathlineto{\pgfqpoint{1.701778in}{2.375902in}}%
\pgfpathlineto{\pgfqpoint{1.486050in}{1.042638in}}%
\pgfpathclose%
\pgfusepath{fill}%
\end{pgfscope}%
\begin{pgfscope}%
\pgfpathrectangle{\pgfqpoint{0.017500in}{0.017500in}}{\pgfqpoint{6.218282in}{3.580372in}} %
\pgfusepath{clip}%
\pgfsetbuttcap%
\pgfsetroundjoin%
\definecolor{currentfill}{rgb}{0.243520,0.319189,0.771672}%
\pgfsetfillcolor{currentfill}%
\pgfsetlinewidth{0.000000pt}%
\definecolor{currentstroke}{rgb}{0.000000,0.000000,0.000000}%
\pgfsetstrokecolor{currentstroke}%
\pgfsetdash{}{0pt}%
\pgfpathmoveto{\pgfqpoint{1.486050in}{1.042638in}}%
\pgfpathlineto{\pgfqpoint{1.701778in}{2.375902in}}%
\pgfpathlineto{\pgfqpoint{1.486050in}{1.042638in}}%
\pgfpathclose%
\pgfusepath{fill}%
\end{pgfscope}%
\begin{pgfscope}%
\pgfpathrectangle{\pgfqpoint{0.017500in}{0.017500in}}{\pgfqpoint{6.218282in}{3.580372in}} %
\pgfusepath{clip}%
\pgfsetbuttcap%
\pgfsetroundjoin%
\definecolor{currentfill}{rgb}{0.243520,0.319189,0.771672}%
\pgfsetfillcolor{currentfill}%
\pgfsetlinewidth{0.000000pt}%
\definecolor{currentstroke}{rgb}{0.000000,0.000000,0.000000}%
\pgfsetstrokecolor{currentstroke}%
\pgfsetdash{}{0pt}%
\pgfpathmoveto{\pgfqpoint{1.486050in}{1.042638in}}%
\pgfpathlineto{\pgfqpoint{1.701778in}{2.375902in}}%
\pgfpathlineto{\pgfqpoint{1.486050in}{1.042638in}}%
\pgfpathclose%
\pgfusepath{fill}%
\end{pgfscope}%
\begin{pgfscope}%
\pgfpathrectangle{\pgfqpoint{0.017500in}{0.017500in}}{\pgfqpoint{6.218282in}{3.580372in}} %
\pgfusepath{clip}%
\pgfsetbuttcap%
\pgfsetroundjoin%
\definecolor{currentfill}{rgb}{0.243520,0.319189,0.771672}%
\pgfsetfillcolor{currentfill}%
\pgfsetlinewidth{0.000000pt}%
\definecolor{currentstroke}{rgb}{0.000000,0.000000,0.000000}%
\pgfsetstrokecolor{currentstroke}%
\pgfsetdash{}{0pt}%
\pgfpathmoveto{\pgfqpoint{1.486050in}{1.042638in}}%
\pgfpathlineto{\pgfqpoint{1.701778in}{2.375902in}}%
\pgfpathlineto{\pgfqpoint{1.486050in}{1.042638in}}%
\pgfpathclose%
\pgfusepath{fill}%
\end{pgfscope}%
\begin{pgfscope}%
\pgfpathrectangle{\pgfqpoint{0.017500in}{0.017500in}}{\pgfqpoint{6.218282in}{3.580372in}} %
\pgfusepath{clip}%
\pgfsetbuttcap%
\pgfsetroundjoin%
\definecolor{currentfill}{rgb}{0.243520,0.319189,0.771672}%
\pgfsetfillcolor{currentfill}%
\pgfsetlinewidth{0.000000pt}%
\definecolor{currentstroke}{rgb}{0.000000,0.000000,0.000000}%
\pgfsetstrokecolor{currentstroke}%
\pgfsetdash{}{0pt}%
\pgfpathmoveto{\pgfqpoint{1.486050in}{1.042638in}}%
\pgfpathlineto{\pgfqpoint{1.701778in}{2.375902in}}%
\pgfpathlineto{\pgfqpoint{1.486050in}{1.042638in}}%
\pgfpathclose%
\pgfusepath{fill}%
\end{pgfscope}%
\begin{pgfscope}%
\pgfpathrectangle{\pgfqpoint{0.017500in}{0.017500in}}{\pgfqpoint{6.218282in}{3.580372in}} %
\pgfusepath{clip}%
\pgfsetbuttcap%
\pgfsetroundjoin%
\definecolor{currentfill}{rgb}{0.243520,0.319189,0.771672}%
\pgfsetfillcolor{currentfill}%
\pgfsetlinewidth{0.000000pt}%
\definecolor{currentstroke}{rgb}{0.000000,0.000000,0.000000}%
\pgfsetstrokecolor{currentstroke}%
\pgfsetdash{}{0pt}%
\pgfpathmoveto{\pgfqpoint{1.486050in}{1.042638in}}%
\pgfpathlineto{\pgfqpoint{1.701778in}{2.375902in}}%
\pgfpathlineto{\pgfqpoint{1.486050in}{1.042638in}}%
\pgfpathclose%
\pgfusepath{fill}%
\end{pgfscope}%
\begin{pgfscope}%
\pgfpathrectangle{\pgfqpoint{0.017500in}{0.017500in}}{\pgfqpoint{6.218282in}{3.580372in}} %
\pgfusepath{clip}%
\pgfsetbuttcap%
\pgfsetroundjoin%
\definecolor{currentfill}{rgb}{0.243520,0.319189,0.771672}%
\pgfsetfillcolor{currentfill}%
\pgfsetlinewidth{0.000000pt}%
\definecolor{currentstroke}{rgb}{0.000000,0.000000,0.000000}%
\pgfsetstrokecolor{currentstroke}%
\pgfsetdash{}{0pt}%
\pgfpathmoveto{\pgfqpoint{1.486050in}{1.042638in}}%
\pgfpathlineto{\pgfqpoint{1.701778in}{2.375902in}}%
\pgfpathlineto{\pgfqpoint{1.486050in}{1.042638in}}%
\pgfpathclose%
\pgfusepath{fill}%
\end{pgfscope}%
\begin{pgfscope}%
\pgfpathrectangle{\pgfqpoint{0.017500in}{0.017500in}}{\pgfqpoint{6.218282in}{3.580372in}} %
\pgfusepath{clip}%
\pgfsetbuttcap%
\pgfsetroundjoin%
\definecolor{currentfill}{rgb}{0.243520,0.319189,0.771672}%
\pgfsetfillcolor{currentfill}%
\pgfsetlinewidth{0.000000pt}%
\definecolor{currentstroke}{rgb}{0.000000,0.000000,0.000000}%
\pgfsetstrokecolor{currentstroke}%
\pgfsetdash{}{0pt}%
\pgfpathmoveto{\pgfqpoint{1.486050in}{1.042638in}}%
\pgfpathlineto{\pgfqpoint{1.701778in}{2.375902in}}%
\pgfpathlineto{\pgfqpoint{1.486050in}{1.042638in}}%
\pgfpathclose%
\pgfusepath{fill}%
\end{pgfscope}%
\begin{pgfscope}%
\pgfpathrectangle{\pgfqpoint{0.017500in}{0.017500in}}{\pgfqpoint{6.218282in}{3.580372in}} %
\pgfusepath{clip}%
\pgfsetbuttcap%
\pgfsetroundjoin%
\definecolor{currentfill}{rgb}{0.243520,0.319189,0.771672}%
\pgfsetfillcolor{currentfill}%
\pgfsetlinewidth{0.000000pt}%
\definecolor{currentstroke}{rgb}{0.000000,0.000000,0.000000}%
\pgfsetstrokecolor{currentstroke}%
\pgfsetdash{}{0pt}%
\pgfpathmoveto{\pgfqpoint{1.486050in}{1.042638in}}%
\pgfpathlineto{\pgfqpoint{1.701778in}{2.375902in}}%
\pgfpathlineto{\pgfqpoint{1.486050in}{1.042638in}}%
\pgfpathclose%
\pgfusepath{fill}%
\end{pgfscope}%
\begin{pgfscope}%
\pgfpathrectangle{\pgfqpoint{0.017500in}{0.017500in}}{\pgfqpoint{6.218282in}{3.580372in}} %
\pgfusepath{clip}%
\pgfsetbuttcap%
\pgfsetroundjoin%
\definecolor{currentfill}{rgb}{0.243520,0.319189,0.771672}%
\pgfsetfillcolor{currentfill}%
\pgfsetlinewidth{0.000000pt}%
\definecolor{currentstroke}{rgb}{0.000000,0.000000,0.000000}%
\pgfsetstrokecolor{currentstroke}%
\pgfsetdash{}{0pt}%
\pgfpathmoveto{\pgfqpoint{1.486050in}{1.042638in}}%
\pgfpathlineto{\pgfqpoint{1.701778in}{2.375902in}}%
\pgfpathlineto{\pgfqpoint{1.486050in}{1.042638in}}%
\pgfpathclose%
\pgfusepath{fill}%
\end{pgfscope}%
\begin{pgfscope}%
\pgfpathrectangle{\pgfqpoint{0.017500in}{0.017500in}}{\pgfqpoint{6.218282in}{3.580372in}} %
\pgfusepath{clip}%
\pgfsetbuttcap%
\pgfsetroundjoin%
\definecolor{currentfill}{rgb}{0.746838,0.140021,0.179996}%
\pgfsetfillcolor{currentfill}%
\pgfsetlinewidth{0.000000pt}%
\definecolor{currentstroke}{rgb}{0.000000,0.000000,0.000000}%
\pgfsetstrokecolor{currentstroke}%
\pgfsetdash{}{0pt}%
\pgfpathmoveto{\pgfqpoint{2.674254in}{2.688576in}}%
\pgfpathlineto{\pgfqpoint{2.932968in}{2.820668in}}%
\pgfpathlineto{\pgfqpoint{2.674254in}{2.688576in}}%
\pgfpathclose%
\pgfusepath{fill}%
\end{pgfscope}%
\begin{pgfscope}%
\pgfpathrectangle{\pgfqpoint{0.017500in}{0.017500in}}{\pgfqpoint{6.218282in}{3.580372in}} %
\pgfusepath{clip}%
\pgfsetbuttcap%
\pgfsetroundjoin%
\definecolor{currentfill}{rgb}{0.746838,0.140021,0.179996}%
\pgfsetfillcolor{currentfill}%
\pgfsetlinewidth{0.000000pt}%
\definecolor{currentstroke}{rgb}{0.000000,0.000000,0.000000}%
\pgfsetstrokecolor{currentstroke}%
\pgfsetdash{}{0pt}%
\pgfpathmoveto{\pgfqpoint{2.674254in}{2.688576in}}%
\pgfpathlineto{\pgfqpoint{2.932968in}{2.820668in}}%
\pgfpathlineto{\pgfqpoint{2.674254in}{2.688576in}}%
\pgfpathclose%
\pgfusepath{fill}%
\end{pgfscope}%
\begin{pgfscope}%
\pgfpathrectangle{\pgfqpoint{0.017500in}{0.017500in}}{\pgfqpoint{6.218282in}{3.580372in}} %
\pgfusepath{clip}%
\pgfsetbuttcap%
\pgfsetroundjoin%
\definecolor{currentfill}{rgb}{0.746838,0.140021,0.179996}%
\pgfsetfillcolor{currentfill}%
\pgfsetlinewidth{0.000000pt}%
\definecolor{currentstroke}{rgb}{0.000000,0.000000,0.000000}%
\pgfsetstrokecolor{currentstroke}%
\pgfsetdash{}{0pt}%
\pgfpathmoveto{\pgfqpoint{2.674254in}{2.688576in}}%
\pgfpathlineto{\pgfqpoint{2.932968in}{2.820668in}}%
\pgfpathlineto{\pgfqpoint{2.674254in}{2.688576in}}%
\pgfpathclose%
\pgfusepath{fill}%
\end{pgfscope}%
\begin{pgfscope}%
\pgfpathrectangle{\pgfqpoint{0.017500in}{0.017500in}}{\pgfqpoint{6.218282in}{3.580372in}} %
\pgfusepath{clip}%
\pgfsetbuttcap%
\pgfsetroundjoin%
\definecolor{currentfill}{rgb}{0.746838,0.140021,0.179996}%
\pgfsetfillcolor{currentfill}%
\pgfsetlinewidth{0.000000pt}%
\definecolor{currentstroke}{rgb}{0.000000,0.000000,0.000000}%
\pgfsetstrokecolor{currentstroke}%
\pgfsetdash{}{0pt}%
\pgfpathmoveto{\pgfqpoint{2.674254in}{2.688576in}}%
\pgfpathlineto{\pgfqpoint{2.932968in}{2.820668in}}%
\pgfpathlineto{\pgfqpoint{2.674254in}{2.688576in}}%
\pgfpathclose%
\pgfusepath{fill}%
\end{pgfscope}%
\begin{pgfscope}%
\pgfpathrectangle{\pgfqpoint{0.017500in}{0.017500in}}{\pgfqpoint{6.218282in}{3.580372in}} %
\pgfusepath{clip}%
\pgfsetbuttcap%
\pgfsetroundjoin%
\definecolor{currentfill}{rgb}{0.746838,0.140021,0.179996}%
\pgfsetfillcolor{currentfill}%
\pgfsetlinewidth{0.000000pt}%
\definecolor{currentstroke}{rgb}{0.000000,0.000000,0.000000}%
\pgfsetstrokecolor{currentstroke}%
\pgfsetdash{}{0pt}%
\pgfpathmoveto{\pgfqpoint{2.674254in}{2.688576in}}%
\pgfpathlineto{\pgfqpoint{2.932968in}{2.820668in}}%
\pgfpathlineto{\pgfqpoint{2.674254in}{2.688576in}}%
\pgfpathclose%
\pgfusepath{fill}%
\end{pgfscope}%
\begin{pgfscope}%
\pgfpathrectangle{\pgfqpoint{0.017500in}{0.017500in}}{\pgfqpoint{6.218282in}{3.580372in}} %
\pgfusepath{clip}%
\pgfsetbuttcap%
\pgfsetroundjoin%
\definecolor{currentfill}{rgb}{0.746838,0.140021,0.179996}%
\pgfsetfillcolor{currentfill}%
\pgfsetlinewidth{0.000000pt}%
\definecolor{currentstroke}{rgb}{0.000000,0.000000,0.000000}%
\pgfsetstrokecolor{currentstroke}%
\pgfsetdash{}{0pt}%
\pgfpathmoveto{\pgfqpoint{2.674254in}{2.688576in}}%
\pgfpathlineto{\pgfqpoint{2.932968in}{2.820668in}}%
\pgfpathlineto{\pgfqpoint{2.674254in}{2.688576in}}%
\pgfpathclose%
\pgfusepath{fill}%
\end{pgfscope}%
\begin{pgfscope}%
\pgfpathrectangle{\pgfqpoint{0.017500in}{0.017500in}}{\pgfqpoint{6.218282in}{3.580372in}} %
\pgfusepath{clip}%
\pgfsetbuttcap%
\pgfsetroundjoin%
\definecolor{currentfill}{rgb}{0.746838,0.140021,0.179996}%
\pgfsetfillcolor{currentfill}%
\pgfsetlinewidth{0.000000pt}%
\definecolor{currentstroke}{rgb}{0.000000,0.000000,0.000000}%
\pgfsetstrokecolor{currentstroke}%
\pgfsetdash{}{0pt}%
\pgfpathmoveto{\pgfqpoint{2.674254in}{2.688576in}}%
\pgfpathlineto{\pgfqpoint{2.932968in}{2.820668in}}%
\pgfpathlineto{\pgfqpoint{2.674254in}{2.688576in}}%
\pgfpathclose%
\pgfusepath{fill}%
\end{pgfscope}%
\begin{pgfscope}%
\pgfpathrectangle{\pgfqpoint{0.017500in}{0.017500in}}{\pgfqpoint{6.218282in}{3.580372in}} %
\pgfusepath{clip}%
\pgfsetbuttcap%
\pgfsetroundjoin%
\definecolor{currentfill}{rgb}{0.746838,0.140021,0.179996}%
\pgfsetfillcolor{currentfill}%
\pgfsetlinewidth{0.000000pt}%
\definecolor{currentstroke}{rgb}{0.000000,0.000000,0.000000}%
\pgfsetstrokecolor{currentstroke}%
\pgfsetdash{}{0pt}%
\pgfpathmoveto{\pgfqpoint{2.674254in}{2.688576in}}%
\pgfpathlineto{\pgfqpoint{2.932968in}{2.820668in}}%
\pgfpathlineto{\pgfqpoint{2.674254in}{2.688576in}}%
\pgfpathclose%
\pgfusepath{fill}%
\end{pgfscope}%
\begin{pgfscope}%
\pgfpathrectangle{\pgfqpoint{0.017500in}{0.017500in}}{\pgfqpoint{6.218282in}{3.580372in}} %
\pgfusepath{clip}%
\pgfsetbuttcap%
\pgfsetroundjoin%
\definecolor{currentfill}{rgb}{0.746838,0.140021,0.179996}%
\pgfsetfillcolor{currentfill}%
\pgfsetlinewidth{0.000000pt}%
\definecolor{currentstroke}{rgb}{0.000000,0.000000,0.000000}%
\pgfsetstrokecolor{currentstroke}%
\pgfsetdash{}{0pt}%
\pgfpathmoveto{\pgfqpoint{2.674254in}{2.688576in}}%
\pgfpathlineto{\pgfqpoint{2.932968in}{2.820668in}}%
\pgfpathlineto{\pgfqpoint{2.674254in}{2.688576in}}%
\pgfpathclose%
\pgfusepath{fill}%
\end{pgfscope}%
\begin{pgfscope}%
\pgfpathrectangle{\pgfqpoint{0.017500in}{0.017500in}}{\pgfqpoint{6.218282in}{3.580372in}} %
\pgfusepath{clip}%
\pgfsetbuttcap%
\pgfsetroundjoin%
\definecolor{currentfill}{rgb}{0.746838,0.140021,0.179996}%
\pgfsetfillcolor{currentfill}%
\pgfsetlinewidth{0.000000pt}%
\definecolor{currentstroke}{rgb}{0.000000,0.000000,0.000000}%
\pgfsetstrokecolor{currentstroke}%
\pgfsetdash{}{0pt}%
\pgfpathmoveto{\pgfqpoint{2.674254in}{2.688576in}}%
\pgfpathlineto{\pgfqpoint{2.932968in}{2.820668in}}%
\pgfpathlineto{\pgfqpoint{2.674254in}{2.688576in}}%
\pgfpathclose%
\pgfusepath{fill}%
\end{pgfscope}%
\begin{pgfscope}%
\pgfpathrectangle{\pgfqpoint{0.017500in}{0.017500in}}{\pgfqpoint{6.218282in}{3.580372in}} %
\pgfusepath{clip}%
\pgfsetbuttcap%
\pgfsetroundjoin%
\definecolor{currentfill}{rgb}{0.746838,0.140021,0.179996}%
\pgfsetfillcolor{currentfill}%
\pgfsetlinewidth{0.000000pt}%
\definecolor{currentstroke}{rgb}{0.000000,0.000000,0.000000}%
\pgfsetstrokecolor{currentstroke}%
\pgfsetdash{}{0pt}%
\pgfpathmoveto{\pgfqpoint{2.674254in}{2.688576in}}%
\pgfpathlineto{\pgfqpoint{2.932968in}{2.820668in}}%
\pgfpathlineto{\pgfqpoint{2.674254in}{2.688576in}}%
\pgfpathclose%
\pgfusepath{fill}%
\end{pgfscope}%
\begin{pgfscope}%
\pgfpathrectangle{\pgfqpoint{0.017500in}{0.017500in}}{\pgfqpoint{6.218282in}{3.580372in}} %
\pgfusepath{clip}%
\pgfsetbuttcap%
\pgfsetroundjoin%
\definecolor{currentfill}{rgb}{0.746838,0.140021,0.179996}%
\pgfsetfillcolor{currentfill}%
\pgfsetlinewidth{0.000000pt}%
\definecolor{currentstroke}{rgb}{0.000000,0.000000,0.000000}%
\pgfsetstrokecolor{currentstroke}%
\pgfsetdash{}{0pt}%
\pgfpathmoveto{\pgfqpoint{2.674254in}{2.688576in}}%
\pgfpathlineto{\pgfqpoint{2.932968in}{2.820668in}}%
\pgfpathlineto{\pgfqpoint{2.674254in}{2.688576in}}%
\pgfpathclose%
\pgfusepath{fill}%
\end{pgfscope}%
\begin{pgfscope}%
\pgfpathrectangle{\pgfqpoint{0.017500in}{0.017500in}}{\pgfqpoint{6.218282in}{3.580372in}} %
\pgfusepath{clip}%
\pgfsetbuttcap%
\pgfsetroundjoin%
\definecolor{currentfill}{rgb}{0.746838,0.140021,0.179996}%
\pgfsetfillcolor{currentfill}%
\pgfsetlinewidth{0.000000pt}%
\definecolor{currentstroke}{rgb}{0.000000,0.000000,0.000000}%
\pgfsetstrokecolor{currentstroke}%
\pgfsetdash{}{0pt}%
\pgfpathmoveto{\pgfqpoint{2.674254in}{2.688576in}}%
\pgfpathlineto{\pgfqpoint{2.932968in}{2.820668in}}%
\pgfpathlineto{\pgfqpoint{2.674254in}{2.688576in}}%
\pgfpathclose%
\pgfusepath{fill}%
\end{pgfscope}%
\begin{pgfscope}%
\pgfpathrectangle{\pgfqpoint{0.017500in}{0.017500in}}{\pgfqpoint{6.218282in}{3.580372in}} %
\pgfusepath{clip}%
\pgfsetbuttcap%
\pgfsetroundjoin%
\definecolor{currentfill}{rgb}{0.746838,0.140021,0.179996}%
\pgfsetfillcolor{currentfill}%
\pgfsetlinewidth{0.000000pt}%
\definecolor{currentstroke}{rgb}{0.000000,0.000000,0.000000}%
\pgfsetstrokecolor{currentstroke}%
\pgfsetdash{}{0pt}%
\pgfpathmoveto{\pgfqpoint{2.674254in}{2.688576in}}%
\pgfpathlineto{\pgfqpoint{2.932968in}{2.820668in}}%
\pgfpathlineto{\pgfqpoint{2.674254in}{2.688576in}}%
\pgfpathclose%
\pgfusepath{fill}%
\end{pgfscope}%
\begin{pgfscope}%
\pgfpathrectangle{\pgfqpoint{0.017500in}{0.017500in}}{\pgfqpoint{6.218282in}{3.580372in}} %
\pgfusepath{clip}%
\pgfsetbuttcap%
\pgfsetroundjoin%
\definecolor{currentfill}{rgb}{0.746838,0.140021,0.179996}%
\pgfsetfillcolor{currentfill}%
\pgfsetlinewidth{0.000000pt}%
\definecolor{currentstroke}{rgb}{0.000000,0.000000,0.000000}%
\pgfsetstrokecolor{currentstroke}%
\pgfsetdash{}{0pt}%
\pgfpathmoveto{\pgfqpoint{2.674254in}{2.688576in}}%
\pgfpathlineto{\pgfqpoint{2.932968in}{2.820668in}}%
\pgfpathlineto{\pgfqpoint{2.674254in}{2.688576in}}%
\pgfpathclose%
\pgfusepath{fill}%
\end{pgfscope}%
\begin{pgfscope}%
\pgfpathrectangle{\pgfqpoint{0.017500in}{0.017500in}}{\pgfqpoint{6.218282in}{3.580372in}} %
\pgfusepath{clip}%
\pgfsetbuttcap%
\pgfsetroundjoin%
\definecolor{currentfill}{rgb}{0.746838,0.140021,0.179996}%
\pgfsetfillcolor{currentfill}%
\pgfsetlinewidth{0.000000pt}%
\definecolor{currentstroke}{rgb}{0.000000,0.000000,0.000000}%
\pgfsetstrokecolor{currentstroke}%
\pgfsetdash{}{0pt}%
\pgfpathmoveto{\pgfqpoint{2.674254in}{2.688576in}}%
\pgfpathlineto{\pgfqpoint{2.932968in}{2.820668in}}%
\pgfpathlineto{\pgfqpoint{2.674254in}{2.688576in}}%
\pgfpathclose%
\pgfusepath{fill}%
\end{pgfscope}%
\begin{pgfscope}%
\pgfpathrectangle{\pgfqpoint{0.017500in}{0.017500in}}{\pgfqpoint{6.218282in}{3.580372in}} %
\pgfusepath{clip}%
\pgfsetbuttcap%
\pgfsetroundjoin%
\definecolor{currentfill}{rgb}{0.746838,0.140021,0.179996}%
\pgfsetfillcolor{currentfill}%
\pgfsetlinewidth{0.000000pt}%
\definecolor{currentstroke}{rgb}{0.000000,0.000000,0.000000}%
\pgfsetstrokecolor{currentstroke}%
\pgfsetdash{}{0pt}%
\pgfpathmoveto{\pgfqpoint{2.674254in}{2.688576in}}%
\pgfpathlineto{\pgfqpoint{2.932968in}{2.820668in}}%
\pgfpathlineto{\pgfqpoint{2.674254in}{2.688576in}}%
\pgfpathclose%
\pgfusepath{fill}%
\end{pgfscope}%
\begin{pgfscope}%
\pgfpathrectangle{\pgfqpoint{0.017500in}{0.017500in}}{\pgfqpoint{6.218282in}{3.580372in}} %
\pgfusepath{clip}%
\pgfsetbuttcap%
\pgfsetroundjoin%
\definecolor{currentfill}{rgb}{0.746838,0.140021,0.179996}%
\pgfsetfillcolor{currentfill}%
\pgfsetlinewidth{0.000000pt}%
\definecolor{currentstroke}{rgb}{0.000000,0.000000,0.000000}%
\pgfsetstrokecolor{currentstroke}%
\pgfsetdash{}{0pt}%
\pgfpathmoveto{\pgfqpoint{2.674254in}{2.688576in}}%
\pgfpathlineto{\pgfqpoint{2.932968in}{2.820668in}}%
\pgfpathlineto{\pgfqpoint{2.674254in}{2.688576in}}%
\pgfpathclose%
\pgfusepath{fill}%
\end{pgfscope}%
\begin{pgfscope}%
\pgfpathrectangle{\pgfqpoint{0.017500in}{0.017500in}}{\pgfqpoint{6.218282in}{3.580372in}} %
\pgfusepath{clip}%
\pgfsetbuttcap%
\pgfsetroundjoin%
\definecolor{currentfill}{rgb}{0.746838,0.140021,0.179996}%
\pgfsetfillcolor{currentfill}%
\pgfsetlinewidth{0.000000pt}%
\definecolor{currentstroke}{rgb}{0.000000,0.000000,0.000000}%
\pgfsetstrokecolor{currentstroke}%
\pgfsetdash{}{0pt}%
\pgfpathmoveto{\pgfqpoint{2.674254in}{2.688576in}}%
\pgfpathlineto{\pgfqpoint{2.932968in}{2.820668in}}%
\pgfpathlineto{\pgfqpoint{2.674254in}{2.688576in}}%
\pgfpathclose%
\pgfusepath{fill}%
\end{pgfscope}%
\begin{pgfscope}%
\pgfpathrectangle{\pgfqpoint{0.017500in}{0.017500in}}{\pgfqpoint{6.218282in}{3.580372in}} %
\pgfusepath{clip}%
\pgfsetbuttcap%
\pgfsetroundjoin%
\definecolor{currentfill}{rgb}{0.746838,0.140021,0.179996}%
\pgfsetfillcolor{currentfill}%
\pgfsetlinewidth{0.000000pt}%
\definecolor{currentstroke}{rgb}{0.000000,0.000000,0.000000}%
\pgfsetstrokecolor{currentstroke}%
\pgfsetdash{}{0pt}%
\pgfpathmoveto{\pgfqpoint{2.674254in}{2.688576in}}%
\pgfpathlineto{\pgfqpoint{2.932968in}{2.820668in}}%
\pgfpathlineto{\pgfqpoint{2.674254in}{2.688576in}}%
\pgfpathclose%
\pgfusepath{fill}%
\end{pgfscope}%
\begin{pgfscope}%
\pgfpathrectangle{\pgfqpoint{0.017500in}{0.017500in}}{\pgfqpoint{6.218282in}{3.580372in}} %
\pgfusepath{clip}%
\pgfsetbuttcap%
\pgfsetroundjoin%
\definecolor{currentfill}{rgb}{0.746838,0.140021,0.179996}%
\pgfsetfillcolor{currentfill}%
\pgfsetlinewidth{0.000000pt}%
\definecolor{currentstroke}{rgb}{0.000000,0.000000,0.000000}%
\pgfsetstrokecolor{currentstroke}%
\pgfsetdash{}{0pt}%
\pgfpathmoveto{\pgfqpoint{2.674254in}{2.688576in}}%
\pgfpathlineto{\pgfqpoint{2.932968in}{2.820668in}}%
\pgfpathlineto{\pgfqpoint{2.674254in}{2.688576in}}%
\pgfpathclose%
\pgfusepath{fill}%
\end{pgfscope}%
\begin{pgfscope}%
\pgfpathrectangle{\pgfqpoint{0.017500in}{0.017500in}}{\pgfqpoint{6.218282in}{3.580372in}} %
\pgfusepath{clip}%
\pgfsetbuttcap%
\pgfsetroundjoin%
\definecolor{currentfill}{rgb}{0.746838,0.140021,0.179996}%
\pgfsetfillcolor{currentfill}%
\pgfsetlinewidth{0.000000pt}%
\definecolor{currentstroke}{rgb}{0.000000,0.000000,0.000000}%
\pgfsetstrokecolor{currentstroke}%
\pgfsetdash{}{0pt}%
\pgfpathmoveto{\pgfqpoint{2.674254in}{2.688576in}}%
\pgfpathlineto{\pgfqpoint{2.932968in}{2.820668in}}%
\pgfpathlineto{\pgfqpoint{2.674254in}{2.688576in}}%
\pgfpathclose%
\pgfusepath{fill}%
\end{pgfscope}%
\begin{pgfscope}%
\pgfpathrectangle{\pgfqpoint{0.017500in}{0.017500in}}{\pgfqpoint{6.218282in}{3.580372in}} %
\pgfusepath{clip}%
\pgfsetbuttcap%
\pgfsetroundjoin%
\definecolor{currentfill}{rgb}{0.746838,0.140021,0.179996}%
\pgfsetfillcolor{currentfill}%
\pgfsetlinewidth{0.000000pt}%
\definecolor{currentstroke}{rgb}{0.000000,0.000000,0.000000}%
\pgfsetstrokecolor{currentstroke}%
\pgfsetdash{}{0pt}%
\pgfpathmoveto{\pgfqpoint{2.674254in}{2.688576in}}%
\pgfpathlineto{\pgfqpoint{2.932968in}{2.820668in}}%
\pgfpathlineto{\pgfqpoint{2.674254in}{2.688576in}}%
\pgfpathclose%
\pgfusepath{fill}%
\end{pgfscope}%
\begin{pgfscope}%
\pgfpathrectangle{\pgfqpoint{0.017500in}{0.017500in}}{\pgfqpoint{6.218282in}{3.580372in}} %
\pgfusepath{clip}%
\pgfsetbuttcap%
\pgfsetroundjoin%
\definecolor{currentfill}{rgb}{0.746838,0.140021,0.179996}%
\pgfsetfillcolor{currentfill}%
\pgfsetlinewidth{0.000000pt}%
\definecolor{currentstroke}{rgb}{0.000000,0.000000,0.000000}%
\pgfsetstrokecolor{currentstroke}%
\pgfsetdash{}{0pt}%
\pgfpathmoveto{\pgfqpoint{2.674254in}{2.688576in}}%
\pgfpathlineto{\pgfqpoint{2.932968in}{2.820668in}}%
\pgfpathlineto{\pgfqpoint{2.674254in}{2.688576in}}%
\pgfpathclose%
\pgfusepath{fill}%
\end{pgfscope}%
\begin{pgfscope}%
\pgfpathrectangle{\pgfqpoint{0.017500in}{0.017500in}}{\pgfqpoint{6.218282in}{3.580372in}} %
\pgfusepath{clip}%
\pgfsetbuttcap%
\pgfsetroundjoin%
\definecolor{currentfill}{rgb}{0.746838,0.140021,0.179996}%
\pgfsetfillcolor{currentfill}%
\pgfsetlinewidth{0.000000pt}%
\definecolor{currentstroke}{rgb}{0.000000,0.000000,0.000000}%
\pgfsetstrokecolor{currentstroke}%
\pgfsetdash{}{0pt}%
\pgfpathmoveto{\pgfqpoint{2.674254in}{2.688576in}}%
\pgfpathlineto{\pgfqpoint{2.932968in}{2.820668in}}%
\pgfpathlineto{\pgfqpoint{2.674254in}{2.688576in}}%
\pgfpathclose%
\pgfusepath{fill}%
\end{pgfscope}%
\begin{pgfscope}%
\pgfpathrectangle{\pgfqpoint{0.017500in}{0.017500in}}{\pgfqpoint{6.218282in}{3.580372in}} %
\pgfusepath{clip}%
\pgfsetbuttcap%
\pgfsetroundjoin%
\definecolor{currentfill}{rgb}{0.746838,0.140021,0.179996}%
\pgfsetfillcolor{currentfill}%
\pgfsetlinewidth{0.000000pt}%
\definecolor{currentstroke}{rgb}{0.000000,0.000000,0.000000}%
\pgfsetstrokecolor{currentstroke}%
\pgfsetdash{}{0pt}%
\pgfpathmoveto{\pgfqpoint{2.674254in}{2.688576in}}%
\pgfpathlineto{\pgfqpoint{2.932968in}{2.820668in}}%
\pgfpathlineto{\pgfqpoint{2.674254in}{2.688576in}}%
\pgfpathclose%
\pgfusepath{fill}%
\end{pgfscope}%
\begin{pgfscope}%
\pgfpathrectangle{\pgfqpoint{0.017500in}{0.017500in}}{\pgfqpoint{6.218282in}{3.580372in}} %
\pgfusepath{clip}%
\pgfsetbuttcap%
\pgfsetroundjoin%
\definecolor{currentfill}{rgb}{0.746838,0.140021,0.179996}%
\pgfsetfillcolor{currentfill}%
\pgfsetlinewidth{0.000000pt}%
\definecolor{currentstroke}{rgb}{0.000000,0.000000,0.000000}%
\pgfsetstrokecolor{currentstroke}%
\pgfsetdash{}{0pt}%
\pgfpathmoveto{\pgfqpoint{2.674254in}{2.688576in}}%
\pgfpathlineto{\pgfqpoint{2.932968in}{2.820668in}}%
\pgfpathlineto{\pgfqpoint{2.674254in}{2.688576in}}%
\pgfpathclose%
\pgfusepath{fill}%
\end{pgfscope}%
\begin{pgfscope}%
\pgfpathrectangle{\pgfqpoint{0.017500in}{0.017500in}}{\pgfqpoint{6.218282in}{3.580372in}} %
\pgfusepath{clip}%
\pgfsetbuttcap%
\pgfsetroundjoin%
\definecolor{currentfill}{rgb}{0.746838,0.140021,0.179996}%
\pgfsetfillcolor{currentfill}%
\pgfsetlinewidth{0.000000pt}%
\definecolor{currentstroke}{rgb}{0.000000,0.000000,0.000000}%
\pgfsetstrokecolor{currentstroke}%
\pgfsetdash{}{0pt}%
\pgfpathmoveto{\pgfqpoint{2.674254in}{2.688576in}}%
\pgfpathlineto{\pgfqpoint{2.932968in}{2.820668in}}%
\pgfpathlineto{\pgfqpoint{2.674254in}{2.688576in}}%
\pgfpathclose%
\pgfusepath{fill}%
\end{pgfscope}%
\begin{pgfscope}%
\pgfpathrectangle{\pgfqpoint{0.017500in}{0.017500in}}{\pgfqpoint{6.218282in}{3.580372in}} %
\pgfusepath{clip}%
\pgfsetbuttcap%
\pgfsetroundjoin%
\definecolor{currentfill}{rgb}{0.746838,0.140021,0.179996}%
\pgfsetfillcolor{currentfill}%
\pgfsetlinewidth{0.000000pt}%
\definecolor{currentstroke}{rgb}{0.000000,0.000000,0.000000}%
\pgfsetstrokecolor{currentstroke}%
\pgfsetdash{}{0pt}%
\pgfpathmoveto{\pgfqpoint{2.674254in}{2.688576in}}%
\pgfpathlineto{\pgfqpoint{2.932968in}{2.820668in}}%
\pgfpathlineto{\pgfqpoint{2.674254in}{2.688576in}}%
\pgfpathclose%
\pgfusepath{fill}%
\end{pgfscope}%
\begin{pgfscope}%
\pgfpathrectangle{\pgfqpoint{0.017500in}{0.017500in}}{\pgfqpoint{6.218282in}{3.580372in}} %
\pgfusepath{clip}%
\pgfsetbuttcap%
\pgfsetroundjoin%
\definecolor{currentfill}{rgb}{0.746838,0.140021,0.179996}%
\pgfsetfillcolor{currentfill}%
\pgfsetlinewidth{0.000000pt}%
\definecolor{currentstroke}{rgb}{0.000000,0.000000,0.000000}%
\pgfsetstrokecolor{currentstroke}%
\pgfsetdash{}{0pt}%
\pgfpathmoveto{\pgfqpoint{2.674254in}{2.688576in}}%
\pgfpathlineto{\pgfqpoint{2.932968in}{2.820668in}}%
\pgfpathlineto{\pgfqpoint{2.674254in}{2.688576in}}%
\pgfpathclose%
\pgfusepath{fill}%
\end{pgfscope}%
\begin{pgfscope}%
\pgfpathrectangle{\pgfqpoint{0.017500in}{0.017500in}}{\pgfqpoint{6.218282in}{3.580372in}} %
\pgfusepath{clip}%
\pgfsetbuttcap%
\pgfsetroundjoin%
\definecolor{currentfill}{rgb}{0.895882,0.849906,0.823499}%
\pgfsetfillcolor{currentfill}%
\pgfsetlinewidth{0.000000pt}%
\definecolor{currentstroke}{rgb}{0.000000,0.000000,0.000000}%
\pgfsetstrokecolor{currentstroke}%
\pgfsetdash{}{0pt}%
\pgfpathmoveto{\pgfqpoint{4.063701in}{2.813820in}}%
\pgfpathlineto{\pgfqpoint{2.789950in}{0.798346in}}%
\pgfpathlineto{\pgfqpoint{4.063701in}{2.813820in}}%
\pgfpathclose%
\pgfusepath{fill}%
\end{pgfscope}%
\begin{pgfscope}%
\pgfpathrectangle{\pgfqpoint{0.017500in}{0.017500in}}{\pgfqpoint{6.218282in}{3.580372in}} %
\pgfusepath{clip}%
\pgfsetbuttcap%
\pgfsetroundjoin%
\definecolor{currentfill}{rgb}{0.895882,0.849906,0.823499}%
\pgfsetfillcolor{currentfill}%
\pgfsetlinewidth{0.000000pt}%
\definecolor{currentstroke}{rgb}{0.000000,0.000000,0.000000}%
\pgfsetstrokecolor{currentstroke}%
\pgfsetdash{}{0pt}%
\pgfpathmoveto{\pgfqpoint{4.063701in}{2.813820in}}%
\pgfpathlineto{\pgfqpoint{2.789950in}{0.798346in}}%
\pgfpathlineto{\pgfqpoint{4.063701in}{2.813820in}}%
\pgfpathclose%
\pgfusepath{fill}%
\end{pgfscope}%
\begin{pgfscope}%
\pgfpathrectangle{\pgfqpoint{0.017500in}{0.017500in}}{\pgfqpoint{6.218282in}{3.580372in}} %
\pgfusepath{clip}%
\pgfsetbuttcap%
\pgfsetroundjoin%
\definecolor{currentfill}{rgb}{0.895882,0.849906,0.823499}%
\pgfsetfillcolor{currentfill}%
\pgfsetlinewidth{0.000000pt}%
\definecolor{currentstroke}{rgb}{0.000000,0.000000,0.000000}%
\pgfsetstrokecolor{currentstroke}%
\pgfsetdash{}{0pt}%
\pgfpathmoveto{\pgfqpoint{4.063701in}{2.813820in}}%
\pgfpathlineto{\pgfqpoint{2.789950in}{0.798346in}}%
\pgfpathlineto{\pgfqpoint{4.063701in}{2.813820in}}%
\pgfpathclose%
\pgfusepath{fill}%
\end{pgfscope}%
\begin{pgfscope}%
\pgfpathrectangle{\pgfqpoint{0.017500in}{0.017500in}}{\pgfqpoint{6.218282in}{3.580372in}} %
\pgfusepath{clip}%
\pgfsetbuttcap%
\pgfsetroundjoin%
\definecolor{currentfill}{rgb}{0.895882,0.849906,0.823499}%
\pgfsetfillcolor{currentfill}%
\pgfsetlinewidth{0.000000pt}%
\definecolor{currentstroke}{rgb}{0.000000,0.000000,0.000000}%
\pgfsetstrokecolor{currentstroke}%
\pgfsetdash{}{0pt}%
\pgfpathmoveto{\pgfqpoint{4.063701in}{2.813820in}}%
\pgfpathlineto{\pgfqpoint{2.789950in}{0.798346in}}%
\pgfpathlineto{\pgfqpoint{4.063701in}{2.813820in}}%
\pgfpathclose%
\pgfusepath{fill}%
\end{pgfscope}%
\begin{pgfscope}%
\pgfpathrectangle{\pgfqpoint{0.017500in}{0.017500in}}{\pgfqpoint{6.218282in}{3.580372in}} %
\pgfusepath{clip}%
\pgfsetbuttcap%
\pgfsetroundjoin%
\definecolor{currentfill}{rgb}{0.895882,0.849906,0.823499}%
\pgfsetfillcolor{currentfill}%
\pgfsetlinewidth{0.000000pt}%
\definecolor{currentstroke}{rgb}{0.000000,0.000000,0.000000}%
\pgfsetstrokecolor{currentstroke}%
\pgfsetdash{}{0pt}%
\pgfpathmoveto{\pgfqpoint{4.063701in}{2.813820in}}%
\pgfpathlineto{\pgfqpoint{2.789950in}{0.798346in}}%
\pgfpathlineto{\pgfqpoint{4.063701in}{2.813820in}}%
\pgfpathclose%
\pgfusepath{fill}%
\end{pgfscope}%
\begin{pgfscope}%
\pgfpathrectangle{\pgfqpoint{0.017500in}{0.017500in}}{\pgfqpoint{6.218282in}{3.580372in}} %
\pgfusepath{clip}%
\pgfsetbuttcap%
\pgfsetroundjoin%
\definecolor{currentfill}{rgb}{0.895882,0.849906,0.823499}%
\pgfsetfillcolor{currentfill}%
\pgfsetlinewidth{0.000000pt}%
\definecolor{currentstroke}{rgb}{0.000000,0.000000,0.000000}%
\pgfsetstrokecolor{currentstroke}%
\pgfsetdash{}{0pt}%
\pgfpathmoveto{\pgfqpoint{4.063701in}{2.813820in}}%
\pgfpathlineto{\pgfqpoint{2.789950in}{0.798346in}}%
\pgfpathlineto{\pgfqpoint{4.063701in}{2.813820in}}%
\pgfpathclose%
\pgfusepath{fill}%
\end{pgfscope}%
\begin{pgfscope}%
\pgfpathrectangle{\pgfqpoint{0.017500in}{0.017500in}}{\pgfqpoint{6.218282in}{3.580372in}} %
\pgfusepath{clip}%
\pgfsetbuttcap%
\pgfsetroundjoin%
\definecolor{currentfill}{rgb}{0.895882,0.849906,0.823499}%
\pgfsetfillcolor{currentfill}%
\pgfsetlinewidth{0.000000pt}%
\definecolor{currentstroke}{rgb}{0.000000,0.000000,0.000000}%
\pgfsetstrokecolor{currentstroke}%
\pgfsetdash{}{0pt}%
\pgfpathmoveto{\pgfqpoint{4.063701in}{2.813820in}}%
\pgfpathlineto{\pgfqpoint{2.789950in}{0.798346in}}%
\pgfpathlineto{\pgfqpoint{4.063701in}{2.813820in}}%
\pgfpathclose%
\pgfusepath{fill}%
\end{pgfscope}%
\begin{pgfscope}%
\pgfpathrectangle{\pgfqpoint{0.017500in}{0.017500in}}{\pgfqpoint{6.218282in}{3.580372in}} %
\pgfusepath{clip}%
\pgfsetbuttcap%
\pgfsetroundjoin%
\definecolor{currentfill}{rgb}{0.895882,0.849906,0.823499}%
\pgfsetfillcolor{currentfill}%
\pgfsetlinewidth{0.000000pt}%
\definecolor{currentstroke}{rgb}{0.000000,0.000000,0.000000}%
\pgfsetstrokecolor{currentstroke}%
\pgfsetdash{}{0pt}%
\pgfpathmoveto{\pgfqpoint{4.063701in}{2.813820in}}%
\pgfpathlineto{\pgfqpoint{2.789950in}{0.798346in}}%
\pgfpathlineto{\pgfqpoint{4.063701in}{2.813820in}}%
\pgfpathclose%
\pgfusepath{fill}%
\end{pgfscope}%
\begin{pgfscope}%
\pgfpathrectangle{\pgfqpoint{0.017500in}{0.017500in}}{\pgfqpoint{6.218282in}{3.580372in}} %
\pgfusepath{clip}%
\pgfsetbuttcap%
\pgfsetroundjoin%
\definecolor{currentfill}{rgb}{0.895882,0.849906,0.823499}%
\pgfsetfillcolor{currentfill}%
\pgfsetlinewidth{0.000000pt}%
\definecolor{currentstroke}{rgb}{0.000000,0.000000,0.000000}%
\pgfsetstrokecolor{currentstroke}%
\pgfsetdash{}{0pt}%
\pgfpathmoveto{\pgfqpoint{4.063701in}{2.813820in}}%
\pgfpathlineto{\pgfqpoint{2.789950in}{0.798346in}}%
\pgfpathlineto{\pgfqpoint{4.063701in}{2.813820in}}%
\pgfpathclose%
\pgfusepath{fill}%
\end{pgfscope}%
\begin{pgfscope}%
\pgfpathrectangle{\pgfqpoint{0.017500in}{0.017500in}}{\pgfqpoint{6.218282in}{3.580372in}} %
\pgfusepath{clip}%
\pgfsetbuttcap%
\pgfsetroundjoin%
\definecolor{currentfill}{rgb}{0.895882,0.849906,0.823499}%
\pgfsetfillcolor{currentfill}%
\pgfsetlinewidth{0.000000pt}%
\definecolor{currentstroke}{rgb}{0.000000,0.000000,0.000000}%
\pgfsetstrokecolor{currentstroke}%
\pgfsetdash{}{0pt}%
\pgfpathmoveto{\pgfqpoint{4.063701in}{2.813820in}}%
\pgfpathlineto{\pgfqpoint{2.789950in}{0.798346in}}%
\pgfpathlineto{\pgfqpoint{4.063701in}{2.813820in}}%
\pgfpathclose%
\pgfusepath{fill}%
\end{pgfscope}%
\begin{pgfscope}%
\pgfpathrectangle{\pgfqpoint{0.017500in}{0.017500in}}{\pgfqpoint{6.218282in}{3.580372in}} %
\pgfusepath{clip}%
\pgfsetbuttcap%
\pgfsetroundjoin%
\definecolor{currentfill}{rgb}{0.895882,0.849906,0.823499}%
\pgfsetfillcolor{currentfill}%
\pgfsetlinewidth{0.000000pt}%
\definecolor{currentstroke}{rgb}{0.000000,0.000000,0.000000}%
\pgfsetstrokecolor{currentstroke}%
\pgfsetdash{}{0pt}%
\pgfpathmoveto{\pgfqpoint{4.063701in}{2.813820in}}%
\pgfpathlineto{\pgfqpoint{2.789950in}{0.798346in}}%
\pgfpathlineto{\pgfqpoint{4.063701in}{2.813820in}}%
\pgfpathclose%
\pgfusepath{fill}%
\end{pgfscope}%
\begin{pgfscope}%
\pgfpathrectangle{\pgfqpoint{0.017500in}{0.017500in}}{\pgfqpoint{6.218282in}{3.580372in}} %
\pgfusepath{clip}%
\pgfsetbuttcap%
\pgfsetroundjoin%
\definecolor{currentfill}{rgb}{0.895882,0.849906,0.823499}%
\pgfsetfillcolor{currentfill}%
\pgfsetlinewidth{0.000000pt}%
\definecolor{currentstroke}{rgb}{0.000000,0.000000,0.000000}%
\pgfsetstrokecolor{currentstroke}%
\pgfsetdash{}{0pt}%
\pgfpathmoveto{\pgfqpoint{4.063701in}{2.813820in}}%
\pgfpathlineto{\pgfqpoint{2.789950in}{0.798346in}}%
\pgfpathlineto{\pgfqpoint{4.063701in}{2.813820in}}%
\pgfpathclose%
\pgfusepath{fill}%
\end{pgfscope}%
\begin{pgfscope}%
\pgfpathrectangle{\pgfqpoint{0.017500in}{0.017500in}}{\pgfqpoint{6.218282in}{3.580372in}} %
\pgfusepath{clip}%
\pgfsetbuttcap%
\pgfsetroundjoin%
\definecolor{currentfill}{rgb}{0.895882,0.849906,0.823499}%
\pgfsetfillcolor{currentfill}%
\pgfsetlinewidth{0.000000pt}%
\definecolor{currentstroke}{rgb}{0.000000,0.000000,0.000000}%
\pgfsetstrokecolor{currentstroke}%
\pgfsetdash{}{0pt}%
\pgfpathmoveto{\pgfqpoint{4.063701in}{2.813820in}}%
\pgfpathlineto{\pgfqpoint{2.789950in}{0.798346in}}%
\pgfpathlineto{\pgfqpoint{4.063701in}{2.813820in}}%
\pgfpathclose%
\pgfusepath{fill}%
\end{pgfscope}%
\begin{pgfscope}%
\pgfpathrectangle{\pgfqpoint{0.017500in}{0.017500in}}{\pgfqpoint{6.218282in}{3.580372in}} %
\pgfusepath{clip}%
\pgfsetbuttcap%
\pgfsetroundjoin%
\definecolor{currentfill}{rgb}{0.895882,0.849906,0.823499}%
\pgfsetfillcolor{currentfill}%
\pgfsetlinewidth{0.000000pt}%
\definecolor{currentstroke}{rgb}{0.000000,0.000000,0.000000}%
\pgfsetstrokecolor{currentstroke}%
\pgfsetdash{}{0pt}%
\pgfpathmoveto{\pgfqpoint{4.063701in}{2.813820in}}%
\pgfpathlineto{\pgfqpoint{2.789950in}{0.798346in}}%
\pgfpathlineto{\pgfqpoint{4.063701in}{2.813820in}}%
\pgfpathclose%
\pgfusepath{fill}%
\end{pgfscope}%
\begin{pgfscope}%
\pgfpathrectangle{\pgfqpoint{0.017500in}{0.017500in}}{\pgfqpoint{6.218282in}{3.580372in}} %
\pgfusepath{clip}%
\pgfsetbuttcap%
\pgfsetroundjoin%
\definecolor{currentfill}{rgb}{0.895882,0.849906,0.823499}%
\pgfsetfillcolor{currentfill}%
\pgfsetlinewidth{0.000000pt}%
\definecolor{currentstroke}{rgb}{0.000000,0.000000,0.000000}%
\pgfsetstrokecolor{currentstroke}%
\pgfsetdash{}{0pt}%
\pgfpathmoveto{\pgfqpoint{4.063701in}{2.813820in}}%
\pgfpathlineto{\pgfqpoint{2.789950in}{0.798346in}}%
\pgfpathlineto{\pgfqpoint{4.063701in}{2.813820in}}%
\pgfpathclose%
\pgfusepath{fill}%
\end{pgfscope}%
\begin{pgfscope}%
\pgfpathrectangle{\pgfqpoint{0.017500in}{0.017500in}}{\pgfqpoint{6.218282in}{3.580372in}} %
\pgfusepath{clip}%
\pgfsetbuttcap%
\pgfsetroundjoin%
\definecolor{currentfill}{rgb}{0.895882,0.849906,0.823499}%
\pgfsetfillcolor{currentfill}%
\pgfsetlinewidth{0.000000pt}%
\definecolor{currentstroke}{rgb}{0.000000,0.000000,0.000000}%
\pgfsetstrokecolor{currentstroke}%
\pgfsetdash{}{0pt}%
\pgfpathmoveto{\pgfqpoint{4.063701in}{2.813820in}}%
\pgfpathlineto{\pgfqpoint{2.789950in}{0.798346in}}%
\pgfpathlineto{\pgfqpoint{4.063701in}{2.813820in}}%
\pgfpathclose%
\pgfusepath{fill}%
\end{pgfscope}%
\begin{pgfscope}%
\pgfpathrectangle{\pgfqpoint{0.017500in}{0.017500in}}{\pgfqpoint{6.218282in}{3.580372in}} %
\pgfusepath{clip}%
\pgfsetbuttcap%
\pgfsetroundjoin%
\definecolor{currentfill}{rgb}{0.895882,0.849906,0.823499}%
\pgfsetfillcolor{currentfill}%
\pgfsetlinewidth{0.000000pt}%
\definecolor{currentstroke}{rgb}{0.000000,0.000000,0.000000}%
\pgfsetstrokecolor{currentstroke}%
\pgfsetdash{}{0pt}%
\pgfpathmoveto{\pgfqpoint{4.063701in}{2.813820in}}%
\pgfpathlineto{\pgfqpoint{2.789950in}{0.798346in}}%
\pgfpathlineto{\pgfqpoint{4.063701in}{2.813820in}}%
\pgfpathclose%
\pgfusepath{fill}%
\end{pgfscope}%
\begin{pgfscope}%
\pgfpathrectangle{\pgfqpoint{0.017500in}{0.017500in}}{\pgfqpoint{6.218282in}{3.580372in}} %
\pgfusepath{clip}%
\pgfsetbuttcap%
\pgfsetroundjoin%
\definecolor{currentfill}{rgb}{0.895882,0.849906,0.823499}%
\pgfsetfillcolor{currentfill}%
\pgfsetlinewidth{0.000000pt}%
\definecolor{currentstroke}{rgb}{0.000000,0.000000,0.000000}%
\pgfsetstrokecolor{currentstroke}%
\pgfsetdash{}{0pt}%
\pgfpathmoveto{\pgfqpoint{4.063701in}{2.813820in}}%
\pgfpathlineto{\pgfqpoint{2.789950in}{0.798346in}}%
\pgfpathlineto{\pgfqpoint{4.063701in}{2.813820in}}%
\pgfpathclose%
\pgfusepath{fill}%
\end{pgfscope}%
\begin{pgfscope}%
\pgfpathrectangle{\pgfqpoint{0.017500in}{0.017500in}}{\pgfqpoint{6.218282in}{3.580372in}} %
\pgfusepath{clip}%
\pgfsetbuttcap%
\pgfsetroundjoin%
\definecolor{currentfill}{rgb}{0.895882,0.849906,0.823499}%
\pgfsetfillcolor{currentfill}%
\pgfsetlinewidth{0.000000pt}%
\definecolor{currentstroke}{rgb}{0.000000,0.000000,0.000000}%
\pgfsetstrokecolor{currentstroke}%
\pgfsetdash{}{0pt}%
\pgfpathmoveto{\pgfqpoint{4.063701in}{2.813820in}}%
\pgfpathlineto{\pgfqpoint{2.789950in}{0.798346in}}%
\pgfpathlineto{\pgfqpoint{4.063701in}{2.813820in}}%
\pgfpathclose%
\pgfusepath{fill}%
\end{pgfscope}%
\begin{pgfscope}%
\pgfpathrectangle{\pgfqpoint{0.017500in}{0.017500in}}{\pgfqpoint{6.218282in}{3.580372in}} %
\pgfusepath{clip}%
\pgfsetbuttcap%
\pgfsetroundjoin%
\definecolor{currentfill}{rgb}{0.895882,0.849906,0.823499}%
\pgfsetfillcolor{currentfill}%
\pgfsetlinewidth{0.000000pt}%
\definecolor{currentstroke}{rgb}{0.000000,0.000000,0.000000}%
\pgfsetstrokecolor{currentstroke}%
\pgfsetdash{}{0pt}%
\pgfpathmoveto{\pgfqpoint{4.063701in}{2.813820in}}%
\pgfpathlineto{\pgfqpoint{2.789950in}{0.798346in}}%
\pgfpathlineto{\pgfqpoint{4.063701in}{2.813820in}}%
\pgfpathclose%
\pgfusepath{fill}%
\end{pgfscope}%
\begin{pgfscope}%
\pgfpathrectangle{\pgfqpoint{0.017500in}{0.017500in}}{\pgfqpoint{6.218282in}{3.580372in}} %
\pgfusepath{clip}%
\pgfsetbuttcap%
\pgfsetroundjoin%
\definecolor{currentfill}{rgb}{0.895882,0.849906,0.823499}%
\pgfsetfillcolor{currentfill}%
\pgfsetlinewidth{0.000000pt}%
\definecolor{currentstroke}{rgb}{0.000000,0.000000,0.000000}%
\pgfsetstrokecolor{currentstroke}%
\pgfsetdash{}{0pt}%
\pgfpathmoveto{\pgfqpoint{4.063701in}{2.813820in}}%
\pgfpathlineto{\pgfqpoint{2.789950in}{0.798346in}}%
\pgfpathlineto{\pgfqpoint{4.063701in}{2.813820in}}%
\pgfpathclose%
\pgfusepath{fill}%
\end{pgfscope}%
\begin{pgfscope}%
\pgfpathrectangle{\pgfqpoint{0.017500in}{0.017500in}}{\pgfqpoint{6.218282in}{3.580372in}} %
\pgfusepath{clip}%
\pgfsetbuttcap%
\pgfsetroundjoin%
\definecolor{currentfill}{rgb}{0.895882,0.849906,0.823499}%
\pgfsetfillcolor{currentfill}%
\pgfsetlinewidth{0.000000pt}%
\definecolor{currentstroke}{rgb}{0.000000,0.000000,0.000000}%
\pgfsetstrokecolor{currentstroke}%
\pgfsetdash{}{0pt}%
\pgfpathmoveto{\pgfqpoint{4.063701in}{2.813820in}}%
\pgfpathlineto{\pgfqpoint{2.789950in}{0.798346in}}%
\pgfpathlineto{\pgfqpoint{4.063701in}{2.813820in}}%
\pgfpathclose%
\pgfusepath{fill}%
\end{pgfscope}%
\begin{pgfscope}%
\pgfpathrectangle{\pgfqpoint{0.017500in}{0.017500in}}{\pgfqpoint{6.218282in}{3.580372in}} %
\pgfusepath{clip}%
\pgfsetbuttcap%
\pgfsetroundjoin%
\definecolor{currentfill}{rgb}{0.895882,0.849906,0.823499}%
\pgfsetfillcolor{currentfill}%
\pgfsetlinewidth{0.000000pt}%
\definecolor{currentstroke}{rgb}{0.000000,0.000000,0.000000}%
\pgfsetstrokecolor{currentstroke}%
\pgfsetdash{}{0pt}%
\pgfpathmoveto{\pgfqpoint{4.063701in}{2.813820in}}%
\pgfpathlineto{\pgfqpoint{2.789950in}{0.798346in}}%
\pgfpathlineto{\pgfqpoint{4.063701in}{2.813820in}}%
\pgfpathclose%
\pgfusepath{fill}%
\end{pgfscope}%
\begin{pgfscope}%
\pgfpathrectangle{\pgfqpoint{0.017500in}{0.017500in}}{\pgfqpoint{6.218282in}{3.580372in}} %
\pgfusepath{clip}%
\pgfsetbuttcap%
\pgfsetroundjoin%
\definecolor{currentfill}{rgb}{0.895882,0.849906,0.823499}%
\pgfsetfillcolor{currentfill}%
\pgfsetlinewidth{0.000000pt}%
\definecolor{currentstroke}{rgb}{0.000000,0.000000,0.000000}%
\pgfsetstrokecolor{currentstroke}%
\pgfsetdash{}{0pt}%
\pgfpathmoveto{\pgfqpoint{4.063701in}{2.813820in}}%
\pgfpathlineto{\pgfqpoint{2.789950in}{0.798346in}}%
\pgfpathlineto{\pgfqpoint{4.063701in}{2.813820in}}%
\pgfpathclose%
\pgfusepath{fill}%
\end{pgfscope}%
\begin{pgfscope}%
\pgfpathrectangle{\pgfqpoint{0.017500in}{0.017500in}}{\pgfqpoint{6.218282in}{3.580372in}} %
\pgfusepath{clip}%
\pgfsetbuttcap%
\pgfsetroundjoin%
\definecolor{currentfill}{rgb}{0.895882,0.849906,0.823499}%
\pgfsetfillcolor{currentfill}%
\pgfsetlinewidth{0.000000pt}%
\definecolor{currentstroke}{rgb}{0.000000,0.000000,0.000000}%
\pgfsetstrokecolor{currentstroke}%
\pgfsetdash{}{0pt}%
\pgfpathmoveto{\pgfqpoint{4.063701in}{2.813820in}}%
\pgfpathlineto{\pgfqpoint{2.789950in}{0.798346in}}%
\pgfpathlineto{\pgfqpoint{4.063701in}{2.813820in}}%
\pgfpathclose%
\pgfusepath{fill}%
\end{pgfscope}%
\begin{pgfscope}%
\pgfpathrectangle{\pgfqpoint{0.017500in}{0.017500in}}{\pgfqpoint{6.218282in}{3.580372in}} %
\pgfusepath{clip}%
\pgfsetbuttcap%
\pgfsetroundjoin%
\definecolor{currentfill}{rgb}{0.746838,0.140021,0.179996}%
\pgfsetfillcolor{currentfill}%
\pgfsetlinewidth{0.000000pt}%
\definecolor{currentstroke}{rgb}{0.000000,0.000000,0.000000}%
\pgfsetstrokecolor{currentstroke}%
\pgfsetdash{}{0pt}%
\pgfpathmoveto{\pgfqpoint{2.674254in}{2.688576in}}%
\pgfpathlineto{\pgfqpoint{3.116513in}{2.603812in}}%
\pgfpathlineto{\pgfqpoint{3.372093in}{2.761754in}}%
\pgfpathlineto{\pgfqpoint{2.932968in}{2.820668in}}%
\pgfpathlineto{\pgfqpoint{2.674254in}{2.688576in}}%
\pgfpathclose%
\pgfusepath{fill}%
\end{pgfscope}%
\begin{pgfscope}%
\pgfpathrectangle{\pgfqpoint{0.017500in}{0.017500in}}{\pgfqpoint{6.218282in}{3.580372in}} %
\pgfusepath{clip}%
\pgfsetbuttcap%
\pgfsetroundjoin%
\definecolor{currentfill}{rgb}{0.746838,0.140021,0.179996}%
\pgfsetfillcolor{currentfill}%
\pgfsetlinewidth{0.000000pt}%
\definecolor{currentstroke}{rgb}{0.000000,0.000000,0.000000}%
\pgfsetstrokecolor{currentstroke}%
\pgfsetdash{}{0pt}%
\pgfpathmoveto{\pgfqpoint{2.674254in}{2.688576in}}%
\pgfpathlineto{\pgfqpoint{3.116513in}{2.603812in}}%
\pgfpathlineto{\pgfqpoint{3.372093in}{2.761754in}}%
\pgfpathlineto{\pgfqpoint{2.932968in}{2.820668in}}%
\pgfpathlineto{\pgfqpoint{2.674254in}{2.688576in}}%
\pgfpathclose%
\pgfusepath{fill}%
\end{pgfscope}%
\begin{pgfscope}%
\pgfpathrectangle{\pgfqpoint{0.017500in}{0.017500in}}{\pgfqpoint{6.218282in}{3.580372in}} %
\pgfusepath{clip}%
\pgfsetbuttcap%
\pgfsetroundjoin%
\definecolor{currentfill}{rgb}{0.746838,0.140021,0.179996}%
\pgfsetfillcolor{currentfill}%
\pgfsetlinewidth{0.000000pt}%
\definecolor{currentstroke}{rgb}{0.000000,0.000000,0.000000}%
\pgfsetstrokecolor{currentstroke}%
\pgfsetdash{}{0pt}%
\pgfpathmoveto{\pgfqpoint{2.674254in}{2.688576in}}%
\pgfpathlineto{\pgfqpoint{3.116513in}{2.603812in}}%
\pgfpathlineto{\pgfqpoint{3.372093in}{2.761754in}}%
\pgfpathlineto{\pgfqpoint{2.932968in}{2.820668in}}%
\pgfpathlineto{\pgfqpoint{2.674254in}{2.688576in}}%
\pgfpathclose%
\pgfusepath{fill}%
\end{pgfscope}%
\begin{pgfscope}%
\pgfpathrectangle{\pgfqpoint{0.017500in}{0.017500in}}{\pgfqpoint{6.218282in}{3.580372in}} %
\pgfusepath{clip}%
\pgfsetbuttcap%
\pgfsetroundjoin%
\definecolor{currentfill}{rgb}{0.746838,0.140021,0.179996}%
\pgfsetfillcolor{currentfill}%
\pgfsetlinewidth{0.000000pt}%
\definecolor{currentstroke}{rgb}{0.000000,0.000000,0.000000}%
\pgfsetstrokecolor{currentstroke}%
\pgfsetdash{}{0pt}%
\pgfpathmoveto{\pgfqpoint{2.674254in}{2.688576in}}%
\pgfpathlineto{\pgfqpoint{3.116513in}{2.603812in}}%
\pgfpathlineto{\pgfqpoint{3.372093in}{2.761754in}}%
\pgfpathlineto{\pgfqpoint{2.932968in}{2.820668in}}%
\pgfpathlineto{\pgfqpoint{2.674254in}{2.688576in}}%
\pgfpathclose%
\pgfusepath{fill}%
\end{pgfscope}%
\begin{pgfscope}%
\pgfpathrectangle{\pgfqpoint{0.017500in}{0.017500in}}{\pgfqpoint{6.218282in}{3.580372in}} %
\pgfusepath{clip}%
\pgfsetbuttcap%
\pgfsetroundjoin%
\definecolor{currentfill}{rgb}{0.746838,0.140021,0.179996}%
\pgfsetfillcolor{currentfill}%
\pgfsetlinewidth{0.000000pt}%
\definecolor{currentstroke}{rgb}{0.000000,0.000000,0.000000}%
\pgfsetstrokecolor{currentstroke}%
\pgfsetdash{}{0pt}%
\pgfpathmoveto{\pgfqpoint{2.674254in}{2.688576in}}%
\pgfpathlineto{\pgfqpoint{3.116513in}{2.603812in}}%
\pgfpathlineto{\pgfqpoint{3.372093in}{2.761754in}}%
\pgfpathlineto{\pgfqpoint{2.932968in}{2.820668in}}%
\pgfpathlineto{\pgfqpoint{2.674254in}{2.688576in}}%
\pgfpathclose%
\pgfusepath{fill}%
\end{pgfscope}%
\begin{pgfscope}%
\pgfpathrectangle{\pgfqpoint{0.017500in}{0.017500in}}{\pgfqpoint{6.218282in}{3.580372in}} %
\pgfusepath{clip}%
\pgfsetbuttcap%
\pgfsetroundjoin%
\definecolor{currentfill}{rgb}{0.746838,0.140021,0.179996}%
\pgfsetfillcolor{currentfill}%
\pgfsetlinewidth{0.000000pt}%
\definecolor{currentstroke}{rgb}{0.000000,0.000000,0.000000}%
\pgfsetstrokecolor{currentstroke}%
\pgfsetdash{}{0pt}%
\pgfpathmoveto{\pgfqpoint{2.674254in}{2.688576in}}%
\pgfpathlineto{\pgfqpoint{3.116513in}{2.603812in}}%
\pgfpathlineto{\pgfqpoint{3.372093in}{2.761754in}}%
\pgfpathlineto{\pgfqpoint{2.932968in}{2.820668in}}%
\pgfpathlineto{\pgfqpoint{2.674254in}{2.688576in}}%
\pgfpathclose%
\pgfusepath{fill}%
\end{pgfscope}%
\begin{pgfscope}%
\pgfpathrectangle{\pgfqpoint{0.017500in}{0.017500in}}{\pgfqpoint{6.218282in}{3.580372in}} %
\pgfusepath{clip}%
\pgfsetbuttcap%
\pgfsetroundjoin%
\definecolor{currentfill}{rgb}{0.735077,0.104460,0.171492}%
\pgfsetfillcolor{currentfill}%
\pgfsetlinewidth{0.000000pt}%
\definecolor{currentstroke}{rgb}{0.000000,0.000000,0.000000}%
\pgfsetstrokecolor{currentstroke}%
\pgfsetdash{}{0pt}%
\pgfpathmoveto{\pgfqpoint{3.816789in}{2.677291in}}%
\pgfpathlineto{\pgfqpoint{4.063701in}{2.813820in}}%
\pgfpathlineto{\pgfqpoint{3.816789in}{2.677291in}}%
\pgfpathclose%
\pgfusepath{fill}%
\end{pgfscope}%
\begin{pgfscope}%
\pgfpathrectangle{\pgfqpoint{0.017500in}{0.017500in}}{\pgfqpoint{6.218282in}{3.580372in}} %
\pgfusepath{clip}%
\pgfsetbuttcap%
\pgfsetroundjoin%
\definecolor{currentfill}{rgb}{0.735077,0.104460,0.171492}%
\pgfsetfillcolor{currentfill}%
\pgfsetlinewidth{0.000000pt}%
\definecolor{currentstroke}{rgb}{0.000000,0.000000,0.000000}%
\pgfsetstrokecolor{currentstroke}%
\pgfsetdash{}{0pt}%
\pgfpathmoveto{\pgfqpoint{3.816789in}{2.677291in}}%
\pgfpathlineto{\pgfqpoint{4.063701in}{2.813820in}}%
\pgfpathlineto{\pgfqpoint{3.816789in}{2.677291in}}%
\pgfpathclose%
\pgfusepath{fill}%
\end{pgfscope}%
\begin{pgfscope}%
\pgfpathrectangle{\pgfqpoint{0.017500in}{0.017500in}}{\pgfqpoint{6.218282in}{3.580372in}} %
\pgfusepath{clip}%
\pgfsetbuttcap%
\pgfsetroundjoin%
\definecolor{currentfill}{rgb}{0.735077,0.104460,0.171492}%
\pgfsetfillcolor{currentfill}%
\pgfsetlinewidth{0.000000pt}%
\definecolor{currentstroke}{rgb}{0.000000,0.000000,0.000000}%
\pgfsetstrokecolor{currentstroke}%
\pgfsetdash{}{0pt}%
\pgfpathmoveto{\pgfqpoint{3.816789in}{2.677291in}}%
\pgfpathlineto{\pgfqpoint{4.063701in}{2.813820in}}%
\pgfpathlineto{\pgfqpoint{3.816789in}{2.677291in}}%
\pgfpathclose%
\pgfusepath{fill}%
\end{pgfscope}%
\begin{pgfscope}%
\pgfpathrectangle{\pgfqpoint{0.017500in}{0.017500in}}{\pgfqpoint{6.218282in}{3.580372in}} %
\pgfusepath{clip}%
\pgfsetbuttcap%
\pgfsetroundjoin%
\definecolor{currentfill}{rgb}{0.735077,0.104460,0.171492}%
\pgfsetfillcolor{currentfill}%
\pgfsetlinewidth{0.000000pt}%
\definecolor{currentstroke}{rgb}{0.000000,0.000000,0.000000}%
\pgfsetstrokecolor{currentstroke}%
\pgfsetdash{}{0pt}%
\pgfpathmoveto{\pgfqpoint{3.816789in}{2.677291in}}%
\pgfpathlineto{\pgfqpoint{4.063701in}{2.813820in}}%
\pgfpathlineto{\pgfqpoint{3.816789in}{2.677291in}}%
\pgfpathclose%
\pgfusepath{fill}%
\end{pgfscope}%
\begin{pgfscope}%
\pgfpathrectangle{\pgfqpoint{0.017500in}{0.017500in}}{\pgfqpoint{6.218282in}{3.580372in}} %
\pgfusepath{clip}%
\pgfsetbuttcap%
\pgfsetroundjoin%
\definecolor{currentfill}{rgb}{0.735077,0.104460,0.171492}%
\pgfsetfillcolor{currentfill}%
\pgfsetlinewidth{0.000000pt}%
\definecolor{currentstroke}{rgb}{0.000000,0.000000,0.000000}%
\pgfsetstrokecolor{currentstroke}%
\pgfsetdash{}{0pt}%
\pgfpathmoveto{\pgfqpoint{3.816789in}{2.677291in}}%
\pgfpathlineto{\pgfqpoint{4.063701in}{2.813820in}}%
\pgfpathlineto{\pgfqpoint{3.816789in}{2.677291in}}%
\pgfpathclose%
\pgfusepath{fill}%
\end{pgfscope}%
\begin{pgfscope}%
\pgfpathrectangle{\pgfqpoint{0.017500in}{0.017500in}}{\pgfqpoint{6.218282in}{3.580372in}} %
\pgfusepath{clip}%
\pgfsetbuttcap%
\pgfsetroundjoin%
\definecolor{currentfill}{rgb}{0.735077,0.104460,0.171492}%
\pgfsetfillcolor{currentfill}%
\pgfsetlinewidth{0.000000pt}%
\definecolor{currentstroke}{rgb}{0.000000,0.000000,0.000000}%
\pgfsetstrokecolor{currentstroke}%
\pgfsetdash{}{0pt}%
\pgfpathmoveto{\pgfqpoint{3.816789in}{2.677291in}}%
\pgfpathlineto{\pgfqpoint{4.063701in}{2.813820in}}%
\pgfpathlineto{\pgfqpoint{3.816789in}{2.677291in}}%
\pgfpathclose%
\pgfusepath{fill}%
\end{pgfscope}%
\begin{pgfscope}%
\pgfpathrectangle{\pgfqpoint{0.017500in}{0.017500in}}{\pgfqpoint{6.218282in}{3.580372in}} %
\pgfusepath{clip}%
\pgfsetbuttcap%
\pgfsetroundjoin%
\definecolor{currentfill}{rgb}{0.735077,0.104460,0.171492}%
\pgfsetfillcolor{currentfill}%
\pgfsetlinewidth{0.000000pt}%
\definecolor{currentstroke}{rgb}{0.000000,0.000000,0.000000}%
\pgfsetstrokecolor{currentstroke}%
\pgfsetdash{}{0pt}%
\pgfpathmoveto{\pgfqpoint{3.816789in}{2.677291in}}%
\pgfpathlineto{\pgfqpoint{4.063701in}{2.813820in}}%
\pgfpathlineto{\pgfqpoint{3.816789in}{2.677291in}}%
\pgfpathclose%
\pgfusepath{fill}%
\end{pgfscope}%
\begin{pgfscope}%
\pgfpathrectangle{\pgfqpoint{0.017500in}{0.017500in}}{\pgfqpoint{6.218282in}{3.580372in}} %
\pgfusepath{clip}%
\pgfsetbuttcap%
\pgfsetroundjoin%
\definecolor{currentfill}{rgb}{0.735077,0.104460,0.171492}%
\pgfsetfillcolor{currentfill}%
\pgfsetlinewidth{0.000000pt}%
\definecolor{currentstroke}{rgb}{0.000000,0.000000,0.000000}%
\pgfsetstrokecolor{currentstroke}%
\pgfsetdash{}{0pt}%
\pgfpathmoveto{\pgfqpoint{3.816789in}{2.677291in}}%
\pgfpathlineto{\pgfqpoint{4.063701in}{2.813820in}}%
\pgfpathlineto{\pgfqpoint{3.816789in}{2.677291in}}%
\pgfpathclose%
\pgfusepath{fill}%
\end{pgfscope}%
\begin{pgfscope}%
\pgfpathrectangle{\pgfqpoint{0.017500in}{0.017500in}}{\pgfqpoint{6.218282in}{3.580372in}} %
\pgfusepath{clip}%
\pgfsetbuttcap%
\pgfsetroundjoin%
\definecolor{currentfill}{rgb}{0.735077,0.104460,0.171492}%
\pgfsetfillcolor{currentfill}%
\pgfsetlinewidth{0.000000pt}%
\definecolor{currentstroke}{rgb}{0.000000,0.000000,0.000000}%
\pgfsetstrokecolor{currentstroke}%
\pgfsetdash{}{0pt}%
\pgfpathmoveto{\pgfqpoint{3.816789in}{2.677291in}}%
\pgfpathlineto{\pgfqpoint{4.063701in}{2.813820in}}%
\pgfpathlineto{\pgfqpoint{3.816789in}{2.677291in}}%
\pgfpathclose%
\pgfusepath{fill}%
\end{pgfscope}%
\begin{pgfscope}%
\pgfpathrectangle{\pgfqpoint{0.017500in}{0.017500in}}{\pgfqpoint{6.218282in}{3.580372in}} %
\pgfusepath{clip}%
\pgfsetbuttcap%
\pgfsetroundjoin%
\definecolor{currentfill}{rgb}{0.735077,0.104460,0.171492}%
\pgfsetfillcolor{currentfill}%
\pgfsetlinewidth{0.000000pt}%
\definecolor{currentstroke}{rgb}{0.000000,0.000000,0.000000}%
\pgfsetstrokecolor{currentstroke}%
\pgfsetdash{}{0pt}%
\pgfpathmoveto{\pgfqpoint{3.816789in}{2.677291in}}%
\pgfpathlineto{\pgfqpoint{4.063701in}{2.813820in}}%
\pgfpathlineto{\pgfqpoint{3.816789in}{2.677291in}}%
\pgfpathclose%
\pgfusepath{fill}%
\end{pgfscope}%
\begin{pgfscope}%
\pgfpathrectangle{\pgfqpoint{0.017500in}{0.017500in}}{\pgfqpoint{6.218282in}{3.580372in}} %
\pgfusepath{clip}%
\pgfsetbuttcap%
\pgfsetroundjoin%
\definecolor{currentfill}{rgb}{0.735077,0.104460,0.171492}%
\pgfsetfillcolor{currentfill}%
\pgfsetlinewidth{0.000000pt}%
\definecolor{currentstroke}{rgb}{0.000000,0.000000,0.000000}%
\pgfsetstrokecolor{currentstroke}%
\pgfsetdash{}{0pt}%
\pgfpathmoveto{\pgfqpoint{3.816789in}{2.677291in}}%
\pgfpathlineto{\pgfqpoint{4.063701in}{2.813820in}}%
\pgfpathlineto{\pgfqpoint{3.816789in}{2.677291in}}%
\pgfpathclose%
\pgfusepath{fill}%
\end{pgfscope}%
\begin{pgfscope}%
\pgfpathrectangle{\pgfqpoint{0.017500in}{0.017500in}}{\pgfqpoint{6.218282in}{3.580372in}} %
\pgfusepath{clip}%
\pgfsetbuttcap%
\pgfsetroundjoin%
\definecolor{currentfill}{rgb}{0.735077,0.104460,0.171492}%
\pgfsetfillcolor{currentfill}%
\pgfsetlinewidth{0.000000pt}%
\definecolor{currentstroke}{rgb}{0.000000,0.000000,0.000000}%
\pgfsetstrokecolor{currentstroke}%
\pgfsetdash{}{0pt}%
\pgfpathmoveto{\pgfqpoint{3.816789in}{2.677291in}}%
\pgfpathlineto{\pgfqpoint{4.063701in}{2.813820in}}%
\pgfpathlineto{\pgfqpoint{3.816789in}{2.677291in}}%
\pgfpathclose%
\pgfusepath{fill}%
\end{pgfscope}%
\begin{pgfscope}%
\pgfpathrectangle{\pgfqpoint{0.017500in}{0.017500in}}{\pgfqpoint{6.218282in}{3.580372in}} %
\pgfusepath{clip}%
\pgfsetbuttcap%
\pgfsetroundjoin%
\definecolor{currentfill}{rgb}{0.735077,0.104460,0.171492}%
\pgfsetfillcolor{currentfill}%
\pgfsetlinewidth{0.000000pt}%
\definecolor{currentstroke}{rgb}{0.000000,0.000000,0.000000}%
\pgfsetstrokecolor{currentstroke}%
\pgfsetdash{}{0pt}%
\pgfpathmoveto{\pgfqpoint{3.816789in}{2.677291in}}%
\pgfpathlineto{\pgfqpoint{4.063701in}{2.813820in}}%
\pgfpathlineto{\pgfqpoint{3.816789in}{2.677291in}}%
\pgfpathclose%
\pgfusepath{fill}%
\end{pgfscope}%
\begin{pgfscope}%
\pgfpathrectangle{\pgfqpoint{0.017500in}{0.017500in}}{\pgfqpoint{6.218282in}{3.580372in}} %
\pgfusepath{clip}%
\pgfsetbuttcap%
\pgfsetroundjoin%
\definecolor{currentfill}{rgb}{0.735077,0.104460,0.171492}%
\pgfsetfillcolor{currentfill}%
\pgfsetlinewidth{0.000000pt}%
\definecolor{currentstroke}{rgb}{0.000000,0.000000,0.000000}%
\pgfsetstrokecolor{currentstroke}%
\pgfsetdash{}{0pt}%
\pgfpathmoveto{\pgfqpoint{3.816789in}{2.677291in}}%
\pgfpathlineto{\pgfqpoint{4.063701in}{2.813820in}}%
\pgfpathlineto{\pgfqpoint{3.816789in}{2.677291in}}%
\pgfpathclose%
\pgfusepath{fill}%
\end{pgfscope}%
\begin{pgfscope}%
\pgfpathrectangle{\pgfqpoint{0.017500in}{0.017500in}}{\pgfqpoint{6.218282in}{3.580372in}} %
\pgfusepath{clip}%
\pgfsetbuttcap%
\pgfsetroundjoin%
\definecolor{currentfill}{rgb}{0.735077,0.104460,0.171492}%
\pgfsetfillcolor{currentfill}%
\pgfsetlinewidth{0.000000pt}%
\definecolor{currentstroke}{rgb}{0.000000,0.000000,0.000000}%
\pgfsetstrokecolor{currentstroke}%
\pgfsetdash{}{0pt}%
\pgfpathmoveto{\pgfqpoint{3.816789in}{2.677291in}}%
\pgfpathlineto{\pgfqpoint{4.063701in}{2.813820in}}%
\pgfpathlineto{\pgfqpoint{3.816789in}{2.677291in}}%
\pgfpathclose%
\pgfusepath{fill}%
\end{pgfscope}%
\begin{pgfscope}%
\pgfpathrectangle{\pgfqpoint{0.017500in}{0.017500in}}{\pgfqpoint{6.218282in}{3.580372in}} %
\pgfusepath{clip}%
\pgfsetbuttcap%
\pgfsetroundjoin%
\definecolor{currentfill}{rgb}{0.735077,0.104460,0.171492}%
\pgfsetfillcolor{currentfill}%
\pgfsetlinewidth{0.000000pt}%
\definecolor{currentstroke}{rgb}{0.000000,0.000000,0.000000}%
\pgfsetstrokecolor{currentstroke}%
\pgfsetdash{}{0pt}%
\pgfpathmoveto{\pgfqpoint{3.816789in}{2.677291in}}%
\pgfpathlineto{\pgfqpoint{4.063701in}{2.813820in}}%
\pgfpathlineto{\pgfqpoint{3.816789in}{2.677291in}}%
\pgfpathclose%
\pgfusepath{fill}%
\end{pgfscope}%
\begin{pgfscope}%
\pgfpathrectangle{\pgfqpoint{0.017500in}{0.017500in}}{\pgfqpoint{6.218282in}{3.580372in}} %
\pgfusepath{clip}%
\pgfsetbuttcap%
\pgfsetroundjoin%
\definecolor{currentfill}{rgb}{0.735077,0.104460,0.171492}%
\pgfsetfillcolor{currentfill}%
\pgfsetlinewidth{0.000000pt}%
\definecolor{currentstroke}{rgb}{0.000000,0.000000,0.000000}%
\pgfsetstrokecolor{currentstroke}%
\pgfsetdash{}{0pt}%
\pgfpathmoveto{\pgfqpoint{3.816789in}{2.677291in}}%
\pgfpathlineto{\pgfqpoint{4.063701in}{2.813820in}}%
\pgfpathlineto{\pgfqpoint{3.816789in}{2.677291in}}%
\pgfpathclose%
\pgfusepath{fill}%
\end{pgfscope}%
\begin{pgfscope}%
\pgfpathrectangle{\pgfqpoint{0.017500in}{0.017500in}}{\pgfqpoint{6.218282in}{3.580372in}} %
\pgfusepath{clip}%
\pgfsetbuttcap%
\pgfsetroundjoin%
\definecolor{currentfill}{rgb}{0.735077,0.104460,0.171492}%
\pgfsetfillcolor{currentfill}%
\pgfsetlinewidth{0.000000pt}%
\definecolor{currentstroke}{rgb}{0.000000,0.000000,0.000000}%
\pgfsetstrokecolor{currentstroke}%
\pgfsetdash{}{0pt}%
\pgfpathmoveto{\pgfqpoint{3.816789in}{2.677291in}}%
\pgfpathlineto{\pgfqpoint{4.063701in}{2.813820in}}%
\pgfpathlineto{\pgfqpoint{3.816789in}{2.677291in}}%
\pgfpathclose%
\pgfusepath{fill}%
\end{pgfscope}%
\begin{pgfscope}%
\pgfpathrectangle{\pgfqpoint{0.017500in}{0.017500in}}{\pgfqpoint{6.218282in}{3.580372in}} %
\pgfusepath{clip}%
\pgfsetbuttcap%
\pgfsetroundjoin%
\definecolor{currentfill}{rgb}{0.735077,0.104460,0.171492}%
\pgfsetfillcolor{currentfill}%
\pgfsetlinewidth{0.000000pt}%
\definecolor{currentstroke}{rgb}{0.000000,0.000000,0.000000}%
\pgfsetstrokecolor{currentstroke}%
\pgfsetdash{}{0pt}%
\pgfpathmoveto{\pgfqpoint{3.816789in}{2.677291in}}%
\pgfpathlineto{\pgfqpoint{4.063701in}{2.813820in}}%
\pgfpathlineto{\pgfqpoint{3.816789in}{2.677291in}}%
\pgfpathclose%
\pgfusepath{fill}%
\end{pgfscope}%
\begin{pgfscope}%
\pgfpathrectangle{\pgfqpoint{0.017500in}{0.017500in}}{\pgfqpoint{6.218282in}{3.580372in}} %
\pgfusepath{clip}%
\pgfsetbuttcap%
\pgfsetroundjoin%
\definecolor{currentfill}{rgb}{0.735077,0.104460,0.171492}%
\pgfsetfillcolor{currentfill}%
\pgfsetlinewidth{0.000000pt}%
\definecolor{currentstroke}{rgb}{0.000000,0.000000,0.000000}%
\pgfsetstrokecolor{currentstroke}%
\pgfsetdash{}{0pt}%
\pgfpathmoveto{\pgfqpoint{3.816789in}{2.677291in}}%
\pgfpathlineto{\pgfqpoint{4.063701in}{2.813820in}}%
\pgfpathlineto{\pgfqpoint{3.816789in}{2.677291in}}%
\pgfpathclose%
\pgfusepath{fill}%
\end{pgfscope}%
\begin{pgfscope}%
\pgfpathrectangle{\pgfqpoint{0.017500in}{0.017500in}}{\pgfqpoint{6.218282in}{3.580372in}} %
\pgfusepath{clip}%
\pgfsetbuttcap%
\pgfsetroundjoin%
\definecolor{currentfill}{rgb}{0.735077,0.104460,0.171492}%
\pgfsetfillcolor{currentfill}%
\pgfsetlinewidth{0.000000pt}%
\definecolor{currentstroke}{rgb}{0.000000,0.000000,0.000000}%
\pgfsetstrokecolor{currentstroke}%
\pgfsetdash{}{0pt}%
\pgfpathmoveto{\pgfqpoint{3.816789in}{2.677291in}}%
\pgfpathlineto{\pgfqpoint{4.063701in}{2.813820in}}%
\pgfpathlineto{\pgfqpoint{3.816789in}{2.677291in}}%
\pgfpathclose%
\pgfusepath{fill}%
\end{pgfscope}%
\begin{pgfscope}%
\pgfpathrectangle{\pgfqpoint{0.017500in}{0.017500in}}{\pgfqpoint{6.218282in}{3.580372in}} %
\pgfusepath{clip}%
\pgfsetbuttcap%
\pgfsetroundjoin%
\definecolor{currentfill}{rgb}{0.735077,0.104460,0.171492}%
\pgfsetfillcolor{currentfill}%
\pgfsetlinewidth{0.000000pt}%
\definecolor{currentstroke}{rgb}{0.000000,0.000000,0.000000}%
\pgfsetstrokecolor{currentstroke}%
\pgfsetdash{}{0pt}%
\pgfpathmoveto{\pgfqpoint{3.816789in}{2.677291in}}%
\pgfpathlineto{\pgfqpoint{4.063701in}{2.813820in}}%
\pgfpathlineto{\pgfqpoint{3.816789in}{2.677291in}}%
\pgfpathclose%
\pgfusepath{fill}%
\end{pgfscope}%
\begin{pgfscope}%
\pgfpathrectangle{\pgfqpoint{0.017500in}{0.017500in}}{\pgfqpoint{6.218282in}{3.580372in}} %
\pgfusepath{clip}%
\pgfsetbuttcap%
\pgfsetroundjoin%
\definecolor{currentfill}{rgb}{0.735077,0.104460,0.171492}%
\pgfsetfillcolor{currentfill}%
\pgfsetlinewidth{0.000000pt}%
\definecolor{currentstroke}{rgb}{0.000000,0.000000,0.000000}%
\pgfsetstrokecolor{currentstroke}%
\pgfsetdash{}{0pt}%
\pgfpathmoveto{\pgfqpoint{3.816789in}{2.677291in}}%
\pgfpathlineto{\pgfqpoint{4.063701in}{2.813820in}}%
\pgfpathlineto{\pgfqpoint{3.816789in}{2.677291in}}%
\pgfpathclose%
\pgfusepath{fill}%
\end{pgfscope}%
\begin{pgfscope}%
\pgfpathrectangle{\pgfqpoint{0.017500in}{0.017500in}}{\pgfqpoint{6.218282in}{3.580372in}} %
\pgfusepath{clip}%
\pgfsetbuttcap%
\pgfsetroundjoin%
\definecolor{currentfill}{rgb}{0.735077,0.104460,0.171492}%
\pgfsetfillcolor{currentfill}%
\pgfsetlinewidth{0.000000pt}%
\definecolor{currentstroke}{rgb}{0.000000,0.000000,0.000000}%
\pgfsetstrokecolor{currentstroke}%
\pgfsetdash{}{0pt}%
\pgfpathmoveto{\pgfqpoint{3.816789in}{2.677291in}}%
\pgfpathlineto{\pgfqpoint{4.063701in}{2.813820in}}%
\pgfpathlineto{\pgfqpoint{3.816789in}{2.677291in}}%
\pgfpathclose%
\pgfusepath{fill}%
\end{pgfscope}%
\begin{pgfscope}%
\pgfpathrectangle{\pgfqpoint{0.017500in}{0.017500in}}{\pgfqpoint{6.218282in}{3.580372in}} %
\pgfusepath{clip}%
\pgfsetbuttcap%
\pgfsetroundjoin%
\definecolor{currentfill}{rgb}{0.735077,0.104460,0.171492}%
\pgfsetfillcolor{currentfill}%
\pgfsetlinewidth{0.000000pt}%
\definecolor{currentstroke}{rgb}{0.000000,0.000000,0.000000}%
\pgfsetstrokecolor{currentstroke}%
\pgfsetdash{}{0pt}%
\pgfpathmoveto{\pgfqpoint{3.816789in}{2.677291in}}%
\pgfpathlineto{\pgfqpoint{4.063701in}{2.813820in}}%
\pgfpathlineto{\pgfqpoint{3.816789in}{2.677291in}}%
\pgfpathclose%
\pgfusepath{fill}%
\end{pgfscope}%
\begin{pgfscope}%
\pgfpathrectangle{\pgfqpoint{0.017500in}{0.017500in}}{\pgfqpoint{6.218282in}{3.580372in}} %
\pgfusepath{clip}%
\pgfsetbuttcap%
\pgfsetroundjoin%
\definecolor{currentfill}{rgb}{0.735077,0.104460,0.171492}%
\pgfsetfillcolor{currentfill}%
\pgfsetlinewidth{0.000000pt}%
\definecolor{currentstroke}{rgb}{0.000000,0.000000,0.000000}%
\pgfsetstrokecolor{currentstroke}%
\pgfsetdash{}{0pt}%
\pgfpathmoveto{\pgfqpoint{3.816789in}{2.677291in}}%
\pgfpathlineto{\pgfqpoint{4.063701in}{2.813820in}}%
\pgfpathlineto{\pgfqpoint{3.816789in}{2.677291in}}%
\pgfpathclose%
\pgfusepath{fill}%
\end{pgfscope}%
\begin{pgfscope}%
\pgfpathrectangle{\pgfqpoint{0.017500in}{0.017500in}}{\pgfqpoint{6.218282in}{3.580372in}} %
\pgfusepath{clip}%
\pgfsetbuttcap%
\pgfsetroundjoin%
\definecolor{currentfill}{rgb}{0.735077,0.104460,0.171492}%
\pgfsetfillcolor{currentfill}%
\pgfsetlinewidth{0.000000pt}%
\definecolor{currentstroke}{rgb}{0.000000,0.000000,0.000000}%
\pgfsetstrokecolor{currentstroke}%
\pgfsetdash{}{0pt}%
\pgfpathmoveto{\pgfqpoint{3.816789in}{2.677291in}}%
\pgfpathlineto{\pgfqpoint{4.063701in}{2.813820in}}%
\pgfpathlineto{\pgfqpoint{3.816789in}{2.677291in}}%
\pgfpathclose%
\pgfusepath{fill}%
\end{pgfscope}%
\begin{pgfscope}%
\pgfpathrectangle{\pgfqpoint{0.017500in}{0.017500in}}{\pgfqpoint{6.218282in}{3.580372in}} %
\pgfusepath{clip}%
\pgfsetbuttcap%
\pgfsetroundjoin%
\definecolor{currentfill}{rgb}{0.735077,0.104460,0.171492}%
\pgfsetfillcolor{currentfill}%
\pgfsetlinewidth{0.000000pt}%
\definecolor{currentstroke}{rgb}{0.000000,0.000000,0.000000}%
\pgfsetstrokecolor{currentstroke}%
\pgfsetdash{}{0pt}%
\pgfpathmoveto{\pgfqpoint{3.816789in}{2.677291in}}%
\pgfpathlineto{\pgfqpoint{4.063701in}{2.813820in}}%
\pgfpathlineto{\pgfqpoint{3.816789in}{2.677291in}}%
\pgfpathclose%
\pgfusepath{fill}%
\end{pgfscope}%
\begin{pgfscope}%
\pgfpathrectangle{\pgfqpoint{0.017500in}{0.017500in}}{\pgfqpoint{6.218282in}{3.580372in}} %
\pgfusepath{clip}%
\pgfsetbuttcap%
\pgfsetroundjoin%
\definecolor{currentfill}{rgb}{0.735077,0.104460,0.171492}%
\pgfsetfillcolor{currentfill}%
\pgfsetlinewidth{0.000000pt}%
\definecolor{currentstroke}{rgb}{0.000000,0.000000,0.000000}%
\pgfsetstrokecolor{currentstroke}%
\pgfsetdash{}{0pt}%
\pgfpathmoveto{\pgfqpoint{3.816789in}{2.677291in}}%
\pgfpathlineto{\pgfqpoint{4.063701in}{2.813820in}}%
\pgfpathlineto{\pgfqpoint{3.816789in}{2.677291in}}%
\pgfpathclose%
\pgfusepath{fill}%
\end{pgfscope}%
\begin{pgfscope}%
\pgfpathrectangle{\pgfqpoint{0.017500in}{0.017500in}}{\pgfqpoint{6.218282in}{3.580372in}} %
\pgfusepath{clip}%
\pgfsetbuttcap%
\pgfsetroundjoin%
\definecolor{currentfill}{rgb}{0.735077,0.104460,0.171492}%
\pgfsetfillcolor{currentfill}%
\pgfsetlinewidth{0.000000pt}%
\definecolor{currentstroke}{rgb}{0.000000,0.000000,0.000000}%
\pgfsetstrokecolor{currentstroke}%
\pgfsetdash{}{0pt}%
\pgfpathmoveto{\pgfqpoint{3.816789in}{2.677291in}}%
\pgfpathlineto{\pgfqpoint{4.063701in}{2.813820in}}%
\pgfpathlineto{\pgfqpoint{3.816789in}{2.677291in}}%
\pgfpathclose%
\pgfusepath{fill}%
\end{pgfscope}%
\begin{pgfscope}%
\pgfpathrectangle{\pgfqpoint{0.017500in}{0.017500in}}{\pgfqpoint{6.218282in}{3.580372in}} %
\pgfusepath{clip}%
\pgfsetbuttcap%
\pgfsetroundjoin%
\definecolor{currentfill}{rgb}{0.805723,0.259813,0.230562}%
\pgfsetfillcolor{currentfill}%
\pgfsetlinewidth{0.000000pt}%
\definecolor{currentstroke}{rgb}{0.000000,0.000000,0.000000}%
\pgfsetstrokecolor{currentstroke}%
\pgfsetdash{}{0pt}%
\pgfpathmoveto{\pgfqpoint{1.971633in}{2.586781in}}%
\pgfpathlineto{\pgfqpoint{2.237924in}{2.752812in}}%
\pgfpathlineto{\pgfqpoint{1.971633in}{2.586781in}}%
\pgfpathclose%
\pgfusepath{fill}%
\end{pgfscope}%
\begin{pgfscope}%
\pgfpathrectangle{\pgfqpoint{0.017500in}{0.017500in}}{\pgfqpoint{6.218282in}{3.580372in}} %
\pgfusepath{clip}%
\pgfsetbuttcap%
\pgfsetroundjoin%
\definecolor{currentfill}{rgb}{0.805723,0.259813,0.230562}%
\pgfsetfillcolor{currentfill}%
\pgfsetlinewidth{0.000000pt}%
\definecolor{currentstroke}{rgb}{0.000000,0.000000,0.000000}%
\pgfsetstrokecolor{currentstroke}%
\pgfsetdash{}{0pt}%
\pgfpathmoveto{\pgfqpoint{1.971633in}{2.586781in}}%
\pgfpathlineto{\pgfqpoint{2.237924in}{2.752812in}}%
\pgfpathlineto{\pgfqpoint{1.971633in}{2.586781in}}%
\pgfpathclose%
\pgfusepath{fill}%
\end{pgfscope}%
\begin{pgfscope}%
\pgfpathrectangle{\pgfqpoint{0.017500in}{0.017500in}}{\pgfqpoint{6.218282in}{3.580372in}} %
\pgfusepath{clip}%
\pgfsetbuttcap%
\pgfsetroundjoin%
\definecolor{currentfill}{rgb}{0.805723,0.259813,0.230562}%
\pgfsetfillcolor{currentfill}%
\pgfsetlinewidth{0.000000pt}%
\definecolor{currentstroke}{rgb}{0.000000,0.000000,0.000000}%
\pgfsetstrokecolor{currentstroke}%
\pgfsetdash{}{0pt}%
\pgfpathmoveto{\pgfqpoint{1.971633in}{2.586781in}}%
\pgfpathlineto{\pgfqpoint{2.237924in}{2.752812in}}%
\pgfpathlineto{\pgfqpoint{1.971633in}{2.586781in}}%
\pgfpathclose%
\pgfusepath{fill}%
\end{pgfscope}%
\begin{pgfscope}%
\pgfpathrectangle{\pgfqpoint{0.017500in}{0.017500in}}{\pgfqpoint{6.218282in}{3.580372in}} %
\pgfusepath{clip}%
\pgfsetbuttcap%
\pgfsetroundjoin%
\definecolor{currentfill}{rgb}{0.805723,0.259813,0.230562}%
\pgfsetfillcolor{currentfill}%
\pgfsetlinewidth{0.000000pt}%
\definecolor{currentstroke}{rgb}{0.000000,0.000000,0.000000}%
\pgfsetstrokecolor{currentstroke}%
\pgfsetdash{}{0pt}%
\pgfpathmoveto{\pgfqpoint{1.971633in}{2.586781in}}%
\pgfpathlineto{\pgfqpoint{2.237924in}{2.752812in}}%
\pgfpathlineto{\pgfqpoint{1.971633in}{2.586781in}}%
\pgfpathclose%
\pgfusepath{fill}%
\end{pgfscope}%
\begin{pgfscope}%
\pgfpathrectangle{\pgfqpoint{0.017500in}{0.017500in}}{\pgfqpoint{6.218282in}{3.580372in}} %
\pgfusepath{clip}%
\pgfsetbuttcap%
\pgfsetroundjoin%
\definecolor{currentfill}{rgb}{0.805723,0.259813,0.230562}%
\pgfsetfillcolor{currentfill}%
\pgfsetlinewidth{0.000000pt}%
\definecolor{currentstroke}{rgb}{0.000000,0.000000,0.000000}%
\pgfsetstrokecolor{currentstroke}%
\pgfsetdash{}{0pt}%
\pgfpathmoveto{\pgfqpoint{1.971633in}{2.586781in}}%
\pgfpathlineto{\pgfqpoint{2.237924in}{2.752812in}}%
\pgfpathlineto{\pgfqpoint{1.971633in}{2.586781in}}%
\pgfpathclose%
\pgfusepath{fill}%
\end{pgfscope}%
\begin{pgfscope}%
\pgfpathrectangle{\pgfqpoint{0.017500in}{0.017500in}}{\pgfqpoint{6.218282in}{3.580372in}} %
\pgfusepath{clip}%
\pgfsetbuttcap%
\pgfsetroundjoin%
\definecolor{currentfill}{rgb}{0.805723,0.259813,0.230562}%
\pgfsetfillcolor{currentfill}%
\pgfsetlinewidth{0.000000pt}%
\definecolor{currentstroke}{rgb}{0.000000,0.000000,0.000000}%
\pgfsetstrokecolor{currentstroke}%
\pgfsetdash{}{0pt}%
\pgfpathmoveto{\pgfqpoint{1.971633in}{2.586781in}}%
\pgfpathlineto{\pgfqpoint{2.237924in}{2.752812in}}%
\pgfpathlineto{\pgfqpoint{1.971633in}{2.586781in}}%
\pgfpathclose%
\pgfusepath{fill}%
\end{pgfscope}%
\begin{pgfscope}%
\pgfpathrectangle{\pgfqpoint{0.017500in}{0.017500in}}{\pgfqpoint{6.218282in}{3.580372in}} %
\pgfusepath{clip}%
\pgfsetbuttcap%
\pgfsetroundjoin%
\definecolor{currentfill}{rgb}{0.805723,0.259813,0.230562}%
\pgfsetfillcolor{currentfill}%
\pgfsetlinewidth{0.000000pt}%
\definecolor{currentstroke}{rgb}{0.000000,0.000000,0.000000}%
\pgfsetstrokecolor{currentstroke}%
\pgfsetdash{}{0pt}%
\pgfpathmoveto{\pgfqpoint{1.971633in}{2.586781in}}%
\pgfpathlineto{\pgfqpoint{2.237924in}{2.752812in}}%
\pgfpathlineto{\pgfqpoint{1.971633in}{2.586781in}}%
\pgfpathclose%
\pgfusepath{fill}%
\end{pgfscope}%
\begin{pgfscope}%
\pgfpathrectangle{\pgfqpoint{0.017500in}{0.017500in}}{\pgfqpoint{6.218282in}{3.580372in}} %
\pgfusepath{clip}%
\pgfsetbuttcap%
\pgfsetroundjoin%
\definecolor{currentfill}{rgb}{0.805723,0.259813,0.230562}%
\pgfsetfillcolor{currentfill}%
\pgfsetlinewidth{0.000000pt}%
\definecolor{currentstroke}{rgb}{0.000000,0.000000,0.000000}%
\pgfsetstrokecolor{currentstroke}%
\pgfsetdash{}{0pt}%
\pgfpathmoveto{\pgfqpoint{1.971633in}{2.586781in}}%
\pgfpathlineto{\pgfqpoint{2.237924in}{2.752812in}}%
\pgfpathlineto{\pgfqpoint{1.971633in}{2.586781in}}%
\pgfpathclose%
\pgfusepath{fill}%
\end{pgfscope}%
\begin{pgfscope}%
\pgfpathrectangle{\pgfqpoint{0.017500in}{0.017500in}}{\pgfqpoint{6.218282in}{3.580372in}} %
\pgfusepath{clip}%
\pgfsetbuttcap%
\pgfsetroundjoin%
\definecolor{currentfill}{rgb}{0.805723,0.259813,0.230562}%
\pgfsetfillcolor{currentfill}%
\pgfsetlinewidth{0.000000pt}%
\definecolor{currentstroke}{rgb}{0.000000,0.000000,0.000000}%
\pgfsetstrokecolor{currentstroke}%
\pgfsetdash{}{0pt}%
\pgfpathmoveto{\pgfqpoint{1.971633in}{2.586781in}}%
\pgfpathlineto{\pgfqpoint{2.237924in}{2.752812in}}%
\pgfpathlineto{\pgfqpoint{1.971633in}{2.586781in}}%
\pgfpathclose%
\pgfusepath{fill}%
\end{pgfscope}%
\begin{pgfscope}%
\pgfpathrectangle{\pgfqpoint{0.017500in}{0.017500in}}{\pgfqpoint{6.218282in}{3.580372in}} %
\pgfusepath{clip}%
\pgfsetbuttcap%
\pgfsetroundjoin%
\definecolor{currentfill}{rgb}{0.805723,0.259813,0.230562}%
\pgfsetfillcolor{currentfill}%
\pgfsetlinewidth{0.000000pt}%
\definecolor{currentstroke}{rgb}{0.000000,0.000000,0.000000}%
\pgfsetstrokecolor{currentstroke}%
\pgfsetdash{}{0pt}%
\pgfpathmoveto{\pgfqpoint{1.971633in}{2.586781in}}%
\pgfpathlineto{\pgfqpoint{2.237924in}{2.752812in}}%
\pgfpathlineto{\pgfqpoint{1.971633in}{2.586781in}}%
\pgfpathclose%
\pgfusepath{fill}%
\end{pgfscope}%
\begin{pgfscope}%
\pgfpathrectangle{\pgfqpoint{0.017500in}{0.017500in}}{\pgfqpoint{6.218282in}{3.580372in}} %
\pgfusepath{clip}%
\pgfsetbuttcap%
\pgfsetroundjoin%
\definecolor{currentfill}{rgb}{0.805723,0.259813,0.230562}%
\pgfsetfillcolor{currentfill}%
\pgfsetlinewidth{0.000000pt}%
\definecolor{currentstroke}{rgb}{0.000000,0.000000,0.000000}%
\pgfsetstrokecolor{currentstroke}%
\pgfsetdash{}{0pt}%
\pgfpathmoveto{\pgfqpoint{1.971633in}{2.586781in}}%
\pgfpathlineto{\pgfqpoint{2.237924in}{2.752812in}}%
\pgfpathlineto{\pgfqpoint{1.971633in}{2.586781in}}%
\pgfpathclose%
\pgfusepath{fill}%
\end{pgfscope}%
\begin{pgfscope}%
\pgfpathrectangle{\pgfqpoint{0.017500in}{0.017500in}}{\pgfqpoint{6.218282in}{3.580372in}} %
\pgfusepath{clip}%
\pgfsetbuttcap%
\pgfsetroundjoin%
\definecolor{currentfill}{rgb}{0.805723,0.259813,0.230562}%
\pgfsetfillcolor{currentfill}%
\pgfsetlinewidth{0.000000pt}%
\definecolor{currentstroke}{rgb}{0.000000,0.000000,0.000000}%
\pgfsetstrokecolor{currentstroke}%
\pgfsetdash{}{0pt}%
\pgfpathmoveto{\pgfqpoint{1.971633in}{2.586781in}}%
\pgfpathlineto{\pgfqpoint{2.237924in}{2.752812in}}%
\pgfpathlineto{\pgfqpoint{1.971633in}{2.586781in}}%
\pgfpathclose%
\pgfusepath{fill}%
\end{pgfscope}%
\begin{pgfscope}%
\pgfpathrectangle{\pgfqpoint{0.017500in}{0.017500in}}{\pgfqpoint{6.218282in}{3.580372in}} %
\pgfusepath{clip}%
\pgfsetbuttcap%
\pgfsetroundjoin%
\definecolor{currentfill}{rgb}{0.805723,0.259813,0.230562}%
\pgfsetfillcolor{currentfill}%
\pgfsetlinewidth{0.000000pt}%
\definecolor{currentstroke}{rgb}{0.000000,0.000000,0.000000}%
\pgfsetstrokecolor{currentstroke}%
\pgfsetdash{}{0pt}%
\pgfpathmoveto{\pgfqpoint{1.971633in}{2.586781in}}%
\pgfpathlineto{\pgfqpoint{2.237924in}{2.752812in}}%
\pgfpathlineto{\pgfqpoint{1.971633in}{2.586781in}}%
\pgfpathclose%
\pgfusepath{fill}%
\end{pgfscope}%
\begin{pgfscope}%
\pgfpathrectangle{\pgfqpoint{0.017500in}{0.017500in}}{\pgfqpoint{6.218282in}{3.580372in}} %
\pgfusepath{clip}%
\pgfsetbuttcap%
\pgfsetroundjoin%
\definecolor{currentfill}{rgb}{0.805723,0.259813,0.230562}%
\pgfsetfillcolor{currentfill}%
\pgfsetlinewidth{0.000000pt}%
\definecolor{currentstroke}{rgb}{0.000000,0.000000,0.000000}%
\pgfsetstrokecolor{currentstroke}%
\pgfsetdash{}{0pt}%
\pgfpathmoveto{\pgfqpoint{1.971633in}{2.586781in}}%
\pgfpathlineto{\pgfqpoint{2.237924in}{2.752812in}}%
\pgfpathlineto{\pgfqpoint{1.971633in}{2.586781in}}%
\pgfpathclose%
\pgfusepath{fill}%
\end{pgfscope}%
\begin{pgfscope}%
\pgfpathrectangle{\pgfqpoint{0.017500in}{0.017500in}}{\pgfqpoint{6.218282in}{3.580372in}} %
\pgfusepath{clip}%
\pgfsetbuttcap%
\pgfsetroundjoin%
\definecolor{currentfill}{rgb}{0.805723,0.259813,0.230562}%
\pgfsetfillcolor{currentfill}%
\pgfsetlinewidth{0.000000pt}%
\definecolor{currentstroke}{rgb}{0.000000,0.000000,0.000000}%
\pgfsetstrokecolor{currentstroke}%
\pgfsetdash{}{0pt}%
\pgfpathmoveto{\pgfqpoint{1.971633in}{2.586781in}}%
\pgfpathlineto{\pgfqpoint{2.237924in}{2.752812in}}%
\pgfpathlineto{\pgfqpoint{1.971633in}{2.586781in}}%
\pgfpathclose%
\pgfusepath{fill}%
\end{pgfscope}%
\begin{pgfscope}%
\pgfpathrectangle{\pgfqpoint{0.017500in}{0.017500in}}{\pgfqpoint{6.218282in}{3.580372in}} %
\pgfusepath{clip}%
\pgfsetbuttcap%
\pgfsetroundjoin%
\definecolor{currentfill}{rgb}{0.805723,0.259813,0.230562}%
\pgfsetfillcolor{currentfill}%
\pgfsetlinewidth{0.000000pt}%
\definecolor{currentstroke}{rgb}{0.000000,0.000000,0.000000}%
\pgfsetstrokecolor{currentstroke}%
\pgfsetdash{}{0pt}%
\pgfpathmoveto{\pgfqpoint{1.971633in}{2.586781in}}%
\pgfpathlineto{\pgfqpoint{2.237924in}{2.752812in}}%
\pgfpathlineto{\pgfqpoint{1.971633in}{2.586781in}}%
\pgfpathclose%
\pgfusepath{fill}%
\end{pgfscope}%
\begin{pgfscope}%
\pgfpathrectangle{\pgfqpoint{0.017500in}{0.017500in}}{\pgfqpoint{6.218282in}{3.580372in}} %
\pgfusepath{clip}%
\pgfsetbuttcap%
\pgfsetroundjoin%
\definecolor{currentfill}{rgb}{0.805723,0.259813,0.230562}%
\pgfsetfillcolor{currentfill}%
\pgfsetlinewidth{0.000000pt}%
\definecolor{currentstroke}{rgb}{0.000000,0.000000,0.000000}%
\pgfsetstrokecolor{currentstroke}%
\pgfsetdash{}{0pt}%
\pgfpathmoveto{\pgfqpoint{1.971633in}{2.586781in}}%
\pgfpathlineto{\pgfqpoint{2.237924in}{2.752812in}}%
\pgfpathlineto{\pgfqpoint{1.971633in}{2.586781in}}%
\pgfpathclose%
\pgfusepath{fill}%
\end{pgfscope}%
\begin{pgfscope}%
\pgfpathrectangle{\pgfqpoint{0.017500in}{0.017500in}}{\pgfqpoint{6.218282in}{3.580372in}} %
\pgfusepath{clip}%
\pgfsetbuttcap%
\pgfsetroundjoin%
\definecolor{currentfill}{rgb}{0.805723,0.259813,0.230562}%
\pgfsetfillcolor{currentfill}%
\pgfsetlinewidth{0.000000pt}%
\definecolor{currentstroke}{rgb}{0.000000,0.000000,0.000000}%
\pgfsetstrokecolor{currentstroke}%
\pgfsetdash{}{0pt}%
\pgfpathmoveto{\pgfqpoint{1.971633in}{2.586781in}}%
\pgfpathlineto{\pgfqpoint{2.237924in}{2.752812in}}%
\pgfpathlineto{\pgfqpoint{1.971633in}{2.586781in}}%
\pgfpathclose%
\pgfusepath{fill}%
\end{pgfscope}%
\begin{pgfscope}%
\pgfpathrectangle{\pgfqpoint{0.017500in}{0.017500in}}{\pgfqpoint{6.218282in}{3.580372in}} %
\pgfusepath{clip}%
\pgfsetbuttcap%
\pgfsetroundjoin%
\definecolor{currentfill}{rgb}{0.805723,0.259813,0.230562}%
\pgfsetfillcolor{currentfill}%
\pgfsetlinewidth{0.000000pt}%
\definecolor{currentstroke}{rgb}{0.000000,0.000000,0.000000}%
\pgfsetstrokecolor{currentstroke}%
\pgfsetdash{}{0pt}%
\pgfpathmoveto{\pgfqpoint{1.971633in}{2.586781in}}%
\pgfpathlineto{\pgfqpoint{2.237924in}{2.752812in}}%
\pgfpathlineto{\pgfqpoint{1.971633in}{2.586781in}}%
\pgfpathclose%
\pgfusepath{fill}%
\end{pgfscope}%
\begin{pgfscope}%
\pgfpathrectangle{\pgfqpoint{0.017500in}{0.017500in}}{\pgfqpoint{6.218282in}{3.580372in}} %
\pgfusepath{clip}%
\pgfsetbuttcap%
\pgfsetroundjoin%
\definecolor{currentfill}{rgb}{0.805723,0.259813,0.230562}%
\pgfsetfillcolor{currentfill}%
\pgfsetlinewidth{0.000000pt}%
\definecolor{currentstroke}{rgb}{0.000000,0.000000,0.000000}%
\pgfsetstrokecolor{currentstroke}%
\pgfsetdash{}{0pt}%
\pgfpathmoveto{\pgfqpoint{1.971633in}{2.586781in}}%
\pgfpathlineto{\pgfqpoint{2.237924in}{2.752812in}}%
\pgfpathlineto{\pgfqpoint{1.971633in}{2.586781in}}%
\pgfpathclose%
\pgfusepath{fill}%
\end{pgfscope}%
\begin{pgfscope}%
\pgfpathrectangle{\pgfqpoint{0.017500in}{0.017500in}}{\pgfqpoint{6.218282in}{3.580372in}} %
\pgfusepath{clip}%
\pgfsetbuttcap%
\pgfsetroundjoin%
\definecolor{currentfill}{rgb}{0.805723,0.259813,0.230562}%
\pgfsetfillcolor{currentfill}%
\pgfsetlinewidth{0.000000pt}%
\definecolor{currentstroke}{rgb}{0.000000,0.000000,0.000000}%
\pgfsetstrokecolor{currentstroke}%
\pgfsetdash{}{0pt}%
\pgfpathmoveto{\pgfqpoint{1.971633in}{2.586781in}}%
\pgfpathlineto{\pgfqpoint{2.237924in}{2.752812in}}%
\pgfpathlineto{\pgfqpoint{1.971633in}{2.586781in}}%
\pgfpathclose%
\pgfusepath{fill}%
\end{pgfscope}%
\begin{pgfscope}%
\pgfpathrectangle{\pgfqpoint{0.017500in}{0.017500in}}{\pgfqpoint{6.218282in}{3.580372in}} %
\pgfusepath{clip}%
\pgfsetbuttcap%
\pgfsetroundjoin%
\definecolor{currentfill}{rgb}{0.805723,0.259813,0.230562}%
\pgfsetfillcolor{currentfill}%
\pgfsetlinewidth{0.000000pt}%
\definecolor{currentstroke}{rgb}{0.000000,0.000000,0.000000}%
\pgfsetstrokecolor{currentstroke}%
\pgfsetdash{}{0pt}%
\pgfpathmoveto{\pgfqpoint{1.971633in}{2.586781in}}%
\pgfpathlineto{\pgfqpoint{2.237924in}{2.752812in}}%
\pgfpathlineto{\pgfqpoint{1.971633in}{2.586781in}}%
\pgfpathclose%
\pgfusepath{fill}%
\end{pgfscope}%
\begin{pgfscope}%
\pgfpathrectangle{\pgfqpoint{0.017500in}{0.017500in}}{\pgfqpoint{6.218282in}{3.580372in}} %
\pgfusepath{clip}%
\pgfsetbuttcap%
\pgfsetroundjoin%
\definecolor{currentfill}{rgb}{0.805723,0.259813,0.230562}%
\pgfsetfillcolor{currentfill}%
\pgfsetlinewidth{0.000000pt}%
\definecolor{currentstroke}{rgb}{0.000000,0.000000,0.000000}%
\pgfsetstrokecolor{currentstroke}%
\pgfsetdash{}{0pt}%
\pgfpathmoveto{\pgfqpoint{1.971633in}{2.586781in}}%
\pgfpathlineto{\pgfqpoint{2.237924in}{2.752812in}}%
\pgfpathlineto{\pgfqpoint{1.971633in}{2.586781in}}%
\pgfpathclose%
\pgfusepath{fill}%
\end{pgfscope}%
\begin{pgfscope}%
\pgfpathrectangle{\pgfqpoint{0.017500in}{0.017500in}}{\pgfqpoint{6.218282in}{3.580372in}} %
\pgfusepath{clip}%
\pgfsetbuttcap%
\pgfsetroundjoin%
\definecolor{currentfill}{rgb}{0.805723,0.259813,0.230562}%
\pgfsetfillcolor{currentfill}%
\pgfsetlinewidth{0.000000pt}%
\definecolor{currentstroke}{rgb}{0.000000,0.000000,0.000000}%
\pgfsetstrokecolor{currentstroke}%
\pgfsetdash{}{0pt}%
\pgfpathmoveto{\pgfqpoint{1.971633in}{2.586781in}}%
\pgfpathlineto{\pgfqpoint{2.237924in}{2.752812in}}%
\pgfpathlineto{\pgfqpoint{1.971633in}{2.586781in}}%
\pgfpathclose%
\pgfusepath{fill}%
\end{pgfscope}%
\begin{pgfscope}%
\pgfpathrectangle{\pgfqpoint{0.017500in}{0.017500in}}{\pgfqpoint{6.218282in}{3.580372in}} %
\pgfusepath{clip}%
\pgfsetbuttcap%
\pgfsetroundjoin%
\definecolor{currentfill}{rgb}{0.805723,0.259813,0.230562}%
\pgfsetfillcolor{currentfill}%
\pgfsetlinewidth{0.000000pt}%
\definecolor{currentstroke}{rgb}{0.000000,0.000000,0.000000}%
\pgfsetstrokecolor{currentstroke}%
\pgfsetdash{}{0pt}%
\pgfpathmoveto{\pgfqpoint{1.971633in}{2.586781in}}%
\pgfpathlineto{\pgfqpoint{2.237924in}{2.752812in}}%
\pgfpathlineto{\pgfqpoint{1.971633in}{2.586781in}}%
\pgfpathclose%
\pgfusepath{fill}%
\end{pgfscope}%
\begin{pgfscope}%
\pgfpathrectangle{\pgfqpoint{0.017500in}{0.017500in}}{\pgfqpoint{6.218282in}{3.580372in}} %
\pgfusepath{clip}%
\pgfsetbuttcap%
\pgfsetroundjoin%
\definecolor{currentfill}{rgb}{0.805723,0.259813,0.230562}%
\pgfsetfillcolor{currentfill}%
\pgfsetlinewidth{0.000000pt}%
\definecolor{currentstroke}{rgb}{0.000000,0.000000,0.000000}%
\pgfsetstrokecolor{currentstroke}%
\pgfsetdash{}{0pt}%
\pgfpathmoveto{\pgfqpoint{1.971633in}{2.586781in}}%
\pgfpathlineto{\pgfqpoint{2.237924in}{2.752812in}}%
\pgfpathlineto{\pgfqpoint{1.971633in}{2.586781in}}%
\pgfpathclose%
\pgfusepath{fill}%
\end{pgfscope}%
\begin{pgfscope}%
\pgfpathrectangle{\pgfqpoint{0.017500in}{0.017500in}}{\pgfqpoint{6.218282in}{3.580372in}} %
\pgfusepath{clip}%
\pgfsetbuttcap%
\pgfsetroundjoin%
\definecolor{currentfill}{rgb}{0.805723,0.259813,0.230562}%
\pgfsetfillcolor{currentfill}%
\pgfsetlinewidth{0.000000pt}%
\definecolor{currentstroke}{rgb}{0.000000,0.000000,0.000000}%
\pgfsetstrokecolor{currentstroke}%
\pgfsetdash{}{0pt}%
\pgfpathmoveto{\pgfqpoint{1.971633in}{2.586781in}}%
\pgfpathlineto{\pgfqpoint{2.237924in}{2.752812in}}%
\pgfpathlineto{\pgfqpoint{1.971633in}{2.586781in}}%
\pgfpathclose%
\pgfusepath{fill}%
\end{pgfscope}%
\begin{pgfscope}%
\pgfpathrectangle{\pgfqpoint{0.017500in}{0.017500in}}{\pgfqpoint{6.218282in}{3.580372in}} %
\pgfusepath{clip}%
\pgfsetbuttcap%
\pgfsetroundjoin%
\definecolor{currentfill}{rgb}{0.805723,0.259813,0.230562}%
\pgfsetfillcolor{currentfill}%
\pgfsetlinewidth{0.000000pt}%
\definecolor{currentstroke}{rgb}{0.000000,0.000000,0.000000}%
\pgfsetstrokecolor{currentstroke}%
\pgfsetdash{}{0pt}%
\pgfpathmoveto{\pgfqpoint{1.971633in}{2.586781in}}%
\pgfpathlineto{\pgfqpoint{2.237924in}{2.752812in}}%
\pgfpathlineto{\pgfqpoint{1.971633in}{2.586781in}}%
\pgfpathclose%
\pgfusepath{fill}%
\end{pgfscope}%
\begin{pgfscope}%
\pgfpathrectangle{\pgfqpoint{0.017500in}{0.017500in}}{\pgfqpoint{6.218282in}{3.580372in}} %
\pgfusepath{clip}%
\pgfsetbuttcap%
\pgfsetroundjoin%
\definecolor{currentfill}{rgb}{0.805723,0.259813,0.230562}%
\pgfsetfillcolor{currentfill}%
\pgfsetlinewidth{0.000000pt}%
\definecolor{currentstroke}{rgb}{0.000000,0.000000,0.000000}%
\pgfsetstrokecolor{currentstroke}%
\pgfsetdash{}{0pt}%
\pgfpathmoveto{\pgfqpoint{1.971633in}{2.586781in}}%
\pgfpathlineto{\pgfqpoint{2.237924in}{2.752812in}}%
\pgfpathlineto{\pgfqpoint{1.971633in}{2.586781in}}%
\pgfpathclose%
\pgfusepath{fill}%
\end{pgfscope}%
\begin{pgfscope}%
\pgfpathrectangle{\pgfqpoint{0.017500in}{0.017500in}}{\pgfqpoint{6.218282in}{3.580372in}} %
\pgfusepath{clip}%
\pgfsetbuttcap%
\pgfsetroundjoin%
\definecolor{currentfill}{rgb}{0.805723,0.259813,0.230562}%
\pgfsetfillcolor{currentfill}%
\pgfsetlinewidth{0.000000pt}%
\definecolor{currentstroke}{rgb}{0.000000,0.000000,0.000000}%
\pgfsetstrokecolor{currentstroke}%
\pgfsetdash{}{0pt}%
\pgfpathmoveto{\pgfqpoint{1.971633in}{2.586781in}}%
\pgfpathlineto{\pgfqpoint{2.237924in}{2.752812in}}%
\pgfpathlineto{\pgfqpoint{1.971633in}{2.586781in}}%
\pgfpathclose%
\pgfusepath{fill}%
\end{pgfscope}%
\begin{pgfscope}%
\pgfpathrectangle{\pgfqpoint{0.017500in}{0.017500in}}{\pgfqpoint{6.218282in}{3.580372in}} %
\pgfusepath{clip}%
\pgfsetbuttcap%
\pgfsetroundjoin%
\definecolor{currentfill}{rgb}{0.343278,0.459354,0.884122}%
\pgfsetfillcolor{currentfill}%
\pgfsetlinewidth{0.000000pt}%
\definecolor{currentstroke}{rgb}{0.000000,0.000000,0.000000}%
\pgfsetstrokecolor{currentstroke}%
\pgfsetdash{}{0pt}%
\pgfpathmoveto{\pgfqpoint{1.486050in}{1.042638in}}%
\pgfpathlineto{\pgfqpoint{1.914830in}{0.967594in}}%
\pgfpathlineto{\pgfqpoint{2.143341in}{2.291288in}}%
\pgfpathlineto{\pgfqpoint{1.701778in}{2.375902in}}%
\pgfpathlineto{\pgfqpoint{1.486050in}{1.042638in}}%
\pgfpathclose%
\pgfusepath{fill}%
\end{pgfscope}%
\begin{pgfscope}%
\pgfpathrectangle{\pgfqpoint{0.017500in}{0.017500in}}{\pgfqpoint{6.218282in}{3.580372in}} %
\pgfusepath{clip}%
\pgfsetbuttcap%
\pgfsetroundjoin%
\definecolor{currentfill}{rgb}{0.343278,0.459354,0.884122}%
\pgfsetfillcolor{currentfill}%
\pgfsetlinewidth{0.000000pt}%
\definecolor{currentstroke}{rgb}{0.000000,0.000000,0.000000}%
\pgfsetstrokecolor{currentstroke}%
\pgfsetdash{}{0pt}%
\pgfpathmoveto{\pgfqpoint{1.486050in}{1.042638in}}%
\pgfpathlineto{\pgfqpoint{1.914830in}{0.967594in}}%
\pgfpathlineto{\pgfqpoint{2.143341in}{2.291288in}}%
\pgfpathlineto{\pgfqpoint{1.701778in}{2.375902in}}%
\pgfpathlineto{\pgfqpoint{1.486050in}{1.042638in}}%
\pgfpathclose%
\pgfusepath{fill}%
\end{pgfscope}%
\begin{pgfscope}%
\pgfpathrectangle{\pgfqpoint{0.017500in}{0.017500in}}{\pgfqpoint{6.218282in}{3.580372in}} %
\pgfusepath{clip}%
\pgfsetbuttcap%
\pgfsetroundjoin%
\definecolor{currentfill}{rgb}{0.343278,0.459354,0.884122}%
\pgfsetfillcolor{currentfill}%
\pgfsetlinewidth{0.000000pt}%
\definecolor{currentstroke}{rgb}{0.000000,0.000000,0.000000}%
\pgfsetstrokecolor{currentstroke}%
\pgfsetdash{}{0pt}%
\pgfpathmoveto{\pgfqpoint{1.486050in}{1.042638in}}%
\pgfpathlineto{\pgfqpoint{1.914830in}{0.967594in}}%
\pgfpathlineto{\pgfqpoint{2.143341in}{2.291288in}}%
\pgfpathlineto{\pgfqpoint{1.701778in}{2.375902in}}%
\pgfpathlineto{\pgfqpoint{1.486050in}{1.042638in}}%
\pgfpathclose%
\pgfusepath{fill}%
\end{pgfscope}%
\begin{pgfscope}%
\pgfpathrectangle{\pgfqpoint{0.017500in}{0.017500in}}{\pgfqpoint{6.218282in}{3.580372in}} %
\pgfusepath{clip}%
\pgfsetbuttcap%
\pgfsetroundjoin%
\definecolor{currentfill}{rgb}{0.343278,0.459354,0.884122}%
\pgfsetfillcolor{currentfill}%
\pgfsetlinewidth{0.000000pt}%
\definecolor{currentstroke}{rgb}{0.000000,0.000000,0.000000}%
\pgfsetstrokecolor{currentstroke}%
\pgfsetdash{}{0pt}%
\pgfpathmoveto{\pgfqpoint{1.486050in}{1.042638in}}%
\pgfpathlineto{\pgfqpoint{1.914830in}{0.967594in}}%
\pgfpathlineto{\pgfqpoint{2.143341in}{2.291288in}}%
\pgfpathlineto{\pgfqpoint{1.701778in}{2.375902in}}%
\pgfpathlineto{\pgfqpoint{1.486050in}{1.042638in}}%
\pgfpathclose%
\pgfusepath{fill}%
\end{pgfscope}%
\begin{pgfscope}%
\pgfpathrectangle{\pgfqpoint{0.017500in}{0.017500in}}{\pgfqpoint{6.218282in}{3.580372in}} %
\pgfusepath{clip}%
\pgfsetbuttcap%
\pgfsetroundjoin%
\definecolor{currentfill}{rgb}{0.343278,0.459354,0.884122}%
\pgfsetfillcolor{currentfill}%
\pgfsetlinewidth{0.000000pt}%
\definecolor{currentstroke}{rgb}{0.000000,0.000000,0.000000}%
\pgfsetstrokecolor{currentstroke}%
\pgfsetdash{}{0pt}%
\pgfpathmoveto{\pgfqpoint{1.486050in}{1.042638in}}%
\pgfpathlineto{\pgfqpoint{1.914830in}{0.967594in}}%
\pgfpathlineto{\pgfqpoint{2.143341in}{2.291288in}}%
\pgfpathlineto{\pgfqpoint{1.701778in}{2.375902in}}%
\pgfpathlineto{\pgfqpoint{1.486050in}{1.042638in}}%
\pgfpathclose%
\pgfusepath{fill}%
\end{pgfscope}%
\begin{pgfscope}%
\pgfpathrectangle{\pgfqpoint{0.017500in}{0.017500in}}{\pgfqpoint{6.218282in}{3.580372in}} %
\pgfusepath{clip}%
\pgfsetbuttcap%
\pgfsetroundjoin%
\definecolor{currentfill}{rgb}{0.343278,0.459354,0.884122}%
\pgfsetfillcolor{currentfill}%
\pgfsetlinewidth{0.000000pt}%
\definecolor{currentstroke}{rgb}{0.000000,0.000000,0.000000}%
\pgfsetstrokecolor{currentstroke}%
\pgfsetdash{}{0pt}%
\pgfpathmoveto{\pgfqpoint{1.486050in}{1.042638in}}%
\pgfpathlineto{\pgfqpoint{1.914830in}{0.967594in}}%
\pgfpathlineto{\pgfqpoint{2.143341in}{2.291288in}}%
\pgfpathlineto{\pgfqpoint{1.701778in}{2.375902in}}%
\pgfpathlineto{\pgfqpoint{1.486050in}{1.042638in}}%
\pgfpathclose%
\pgfusepath{fill}%
\end{pgfscope}%
\begin{pgfscope}%
\pgfpathrectangle{\pgfqpoint{0.017500in}{0.017500in}}{\pgfqpoint{6.218282in}{3.580372in}} %
\pgfusepath{clip}%
\pgfsetbuttcap%
\pgfsetroundjoin%
\definecolor{currentfill}{rgb}{0.800601,0.850358,0.930008}%
\pgfsetfillcolor{currentfill}%
\pgfsetlinewidth{0.000000pt}%
\definecolor{currentstroke}{rgb}{0.000000,0.000000,0.000000}%
\pgfsetstrokecolor{currentstroke}%
\pgfsetdash{}{0pt}%
\pgfpathmoveto{\pgfqpoint{4.063701in}{2.813820in}}%
\pgfpathlineto{\pgfqpoint{4.511410in}{2.753953in}}%
\pgfpathlineto{\pgfqpoint{3.236099in}{0.723497in}}%
\pgfpathlineto{\pgfqpoint{2.789950in}{0.798346in}}%
\pgfpathlineto{\pgfqpoint{4.063701in}{2.813820in}}%
\pgfpathclose%
\pgfusepath{fill}%
\end{pgfscope}%
\begin{pgfscope}%
\pgfpathrectangle{\pgfqpoint{0.017500in}{0.017500in}}{\pgfqpoint{6.218282in}{3.580372in}} %
\pgfusepath{clip}%
\pgfsetbuttcap%
\pgfsetroundjoin%
\definecolor{currentfill}{rgb}{0.800601,0.850358,0.930008}%
\pgfsetfillcolor{currentfill}%
\pgfsetlinewidth{0.000000pt}%
\definecolor{currentstroke}{rgb}{0.000000,0.000000,0.000000}%
\pgfsetstrokecolor{currentstroke}%
\pgfsetdash{}{0pt}%
\pgfpathmoveto{\pgfqpoint{4.063701in}{2.813820in}}%
\pgfpathlineto{\pgfqpoint{4.511410in}{2.753953in}}%
\pgfpathlineto{\pgfqpoint{3.236099in}{0.723497in}}%
\pgfpathlineto{\pgfqpoint{2.789950in}{0.798346in}}%
\pgfpathlineto{\pgfqpoint{4.063701in}{2.813820in}}%
\pgfpathclose%
\pgfusepath{fill}%
\end{pgfscope}%
\begin{pgfscope}%
\pgfpathrectangle{\pgfqpoint{0.017500in}{0.017500in}}{\pgfqpoint{6.218282in}{3.580372in}} %
\pgfusepath{clip}%
\pgfsetbuttcap%
\pgfsetroundjoin%
\definecolor{currentfill}{rgb}{0.800601,0.850358,0.930008}%
\pgfsetfillcolor{currentfill}%
\pgfsetlinewidth{0.000000pt}%
\definecolor{currentstroke}{rgb}{0.000000,0.000000,0.000000}%
\pgfsetstrokecolor{currentstroke}%
\pgfsetdash{}{0pt}%
\pgfpathmoveto{\pgfqpoint{4.063701in}{2.813820in}}%
\pgfpathlineto{\pgfqpoint{4.511410in}{2.753953in}}%
\pgfpathlineto{\pgfqpoint{3.236099in}{0.723497in}}%
\pgfpathlineto{\pgfqpoint{2.789950in}{0.798346in}}%
\pgfpathlineto{\pgfqpoint{4.063701in}{2.813820in}}%
\pgfpathclose%
\pgfusepath{fill}%
\end{pgfscope}%
\begin{pgfscope}%
\pgfpathrectangle{\pgfqpoint{0.017500in}{0.017500in}}{\pgfqpoint{6.218282in}{3.580372in}} %
\pgfusepath{clip}%
\pgfsetbuttcap%
\pgfsetroundjoin%
\definecolor{currentfill}{rgb}{0.800601,0.850358,0.930008}%
\pgfsetfillcolor{currentfill}%
\pgfsetlinewidth{0.000000pt}%
\definecolor{currentstroke}{rgb}{0.000000,0.000000,0.000000}%
\pgfsetstrokecolor{currentstroke}%
\pgfsetdash{}{0pt}%
\pgfpathmoveto{\pgfqpoint{4.063701in}{2.813820in}}%
\pgfpathlineto{\pgfqpoint{4.511410in}{2.753953in}}%
\pgfpathlineto{\pgfqpoint{3.236099in}{0.723497in}}%
\pgfpathlineto{\pgfqpoint{2.789950in}{0.798346in}}%
\pgfpathlineto{\pgfqpoint{4.063701in}{2.813820in}}%
\pgfpathclose%
\pgfusepath{fill}%
\end{pgfscope}%
\begin{pgfscope}%
\pgfpathrectangle{\pgfqpoint{0.017500in}{0.017500in}}{\pgfqpoint{6.218282in}{3.580372in}} %
\pgfusepath{clip}%
\pgfsetbuttcap%
\pgfsetroundjoin%
\definecolor{currentfill}{rgb}{0.800601,0.850358,0.930008}%
\pgfsetfillcolor{currentfill}%
\pgfsetlinewidth{0.000000pt}%
\definecolor{currentstroke}{rgb}{0.000000,0.000000,0.000000}%
\pgfsetstrokecolor{currentstroke}%
\pgfsetdash{}{0pt}%
\pgfpathmoveto{\pgfqpoint{4.063701in}{2.813820in}}%
\pgfpathlineto{\pgfqpoint{4.511410in}{2.753953in}}%
\pgfpathlineto{\pgfqpoint{3.236099in}{0.723497in}}%
\pgfpathlineto{\pgfqpoint{2.789950in}{0.798346in}}%
\pgfpathlineto{\pgfqpoint{4.063701in}{2.813820in}}%
\pgfpathclose%
\pgfusepath{fill}%
\end{pgfscope}%
\begin{pgfscope}%
\pgfpathrectangle{\pgfqpoint{0.017500in}{0.017500in}}{\pgfqpoint{6.218282in}{3.580372in}} %
\pgfusepath{clip}%
\pgfsetbuttcap%
\pgfsetroundjoin%
\definecolor{currentfill}{rgb}{0.729196,0.086679,0.167240}%
\pgfsetfillcolor{currentfill}%
\pgfsetlinewidth{0.000000pt}%
\definecolor{currentstroke}{rgb}{0.000000,0.000000,0.000000}%
\pgfsetstrokecolor{currentstroke}%
\pgfsetdash{}{0pt}%
\pgfpathmoveto{\pgfqpoint{3.816789in}{2.677291in}}%
\pgfpathlineto{\pgfqpoint{4.267628in}{2.613520in}}%
\pgfpathlineto{\pgfqpoint{4.511410in}{2.753953in}}%
\pgfpathlineto{\pgfqpoint{4.063701in}{2.813820in}}%
\pgfpathlineto{\pgfqpoint{3.816789in}{2.677291in}}%
\pgfpathclose%
\pgfusepath{fill}%
\end{pgfscope}%
\begin{pgfscope}%
\pgfpathrectangle{\pgfqpoint{0.017500in}{0.017500in}}{\pgfqpoint{6.218282in}{3.580372in}} %
\pgfusepath{clip}%
\pgfsetbuttcap%
\pgfsetroundjoin%
\definecolor{currentfill}{rgb}{0.729196,0.086679,0.167240}%
\pgfsetfillcolor{currentfill}%
\pgfsetlinewidth{0.000000pt}%
\definecolor{currentstroke}{rgb}{0.000000,0.000000,0.000000}%
\pgfsetstrokecolor{currentstroke}%
\pgfsetdash{}{0pt}%
\pgfpathmoveto{\pgfqpoint{3.816789in}{2.677291in}}%
\pgfpathlineto{\pgfqpoint{4.267628in}{2.613520in}}%
\pgfpathlineto{\pgfqpoint{4.511410in}{2.753953in}}%
\pgfpathlineto{\pgfqpoint{4.063701in}{2.813820in}}%
\pgfpathlineto{\pgfqpoint{3.816789in}{2.677291in}}%
\pgfpathclose%
\pgfusepath{fill}%
\end{pgfscope}%
\begin{pgfscope}%
\pgfpathrectangle{\pgfqpoint{0.017500in}{0.017500in}}{\pgfqpoint{6.218282in}{3.580372in}} %
\pgfusepath{clip}%
\pgfsetbuttcap%
\pgfsetroundjoin%
\definecolor{currentfill}{rgb}{0.729196,0.086679,0.167240}%
\pgfsetfillcolor{currentfill}%
\pgfsetlinewidth{0.000000pt}%
\definecolor{currentstroke}{rgb}{0.000000,0.000000,0.000000}%
\pgfsetstrokecolor{currentstroke}%
\pgfsetdash{}{0pt}%
\pgfpathmoveto{\pgfqpoint{3.816789in}{2.677291in}}%
\pgfpathlineto{\pgfqpoint{4.267628in}{2.613520in}}%
\pgfpathlineto{\pgfqpoint{4.511410in}{2.753953in}}%
\pgfpathlineto{\pgfqpoint{4.063701in}{2.813820in}}%
\pgfpathlineto{\pgfqpoint{3.816789in}{2.677291in}}%
\pgfpathclose%
\pgfusepath{fill}%
\end{pgfscope}%
\begin{pgfscope}%
\pgfpathrectangle{\pgfqpoint{0.017500in}{0.017500in}}{\pgfqpoint{6.218282in}{3.580372in}} %
\pgfusepath{clip}%
\pgfsetbuttcap%
\pgfsetroundjoin%
\definecolor{currentfill}{rgb}{0.729196,0.086679,0.167240}%
\pgfsetfillcolor{currentfill}%
\pgfsetlinewidth{0.000000pt}%
\definecolor{currentstroke}{rgb}{0.000000,0.000000,0.000000}%
\pgfsetstrokecolor{currentstroke}%
\pgfsetdash{}{0pt}%
\pgfpathmoveto{\pgfqpoint{3.816789in}{2.677291in}}%
\pgfpathlineto{\pgfqpoint{4.267628in}{2.613520in}}%
\pgfpathlineto{\pgfqpoint{4.511410in}{2.753953in}}%
\pgfpathlineto{\pgfqpoint{4.063701in}{2.813820in}}%
\pgfpathlineto{\pgfqpoint{3.816789in}{2.677291in}}%
\pgfpathclose%
\pgfusepath{fill}%
\end{pgfscope}%
\begin{pgfscope}%
\pgfpathrectangle{\pgfqpoint{0.017500in}{0.017500in}}{\pgfqpoint{6.218282in}{3.580372in}} %
\pgfusepath{clip}%
\pgfsetbuttcap%
\pgfsetroundjoin%
\definecolor{currentfill}{rgb}{0.729196,0.086679,0.167240}%
\pgfsetfillcolor{currentfill}%
\pgfsetlinewidth{0.000000pt}%
\definecolor{currentstroke}{rgb}{0.000000,0.000000,0.000000}%
\pgfsetstrokecolor{currentstroke}%
\pgfsetdash{}{0pt}%
\pgfpathmoveto{\pgfqpoint{3.816789in}{2.677291in}}%
\pgfpathlineto{\pgfqpoint{4.267628in}{2.613520in}}%
\pgfpathlineto{\pgfqpoint{4.511410in}{2.753953in}}%
\pgfpathlineto{\pgfqpoint{4.063701in}{2.813820in}}%
\pgfpathlineto{\pgfqpoint{3.816789in}{2.677291in}}%
\pgfpathclose%
\pgfusepath{fill}%
\end{pgfscope}%
\begin{pgfscope}%
\pgfpathrectangle{\pgfqpoint{0.017500in}{0.017500in}}{\pgfqpoint{6.218282in}{3.580372in}} %
\pgfusepath{clip}%
\pgfsetbuttcap%
\pgfsetroundjoin%
\definecolor{currentfill}{rgb}{0.729196,0.086679,0.167240}%
\pgfsetfillcolor{currentfill}%
\pgfsetlinewidth{0.000000pt}%
\definecolor{currentstroke}{rgb}{0.000000,0.000000,0.000000}%
\pgfsetstrokecolor{currentstroke}%
\pgfsetdash{}{0pt}%
\pgfpathmoveto{\pgfqpoint{3.816789in}{2.677291in}}%
\pgfpathlineto{\pgfqpoint{4.267628in}{2.613520in}}%
\pgfpathlineto{\pgfqpoint{4.511410in}{2.753953in}}%
\pgfpathlineto{\pgfqpoint{4.063701in}{2.813820in}}%
\pgfpathlineto{\pgfqpoint{3.816789in}{2.677291in}}%
\pgfpathclose%
\pgfusepath{fill}%
\end{pgfscope}%
\begin{pgfscope}%
\pgfpathrectangle{\pgfqpoint{0.017500in}{0.017500in}}{\pgfqpoint{6.218282in}{3.580372in}} %
\pgfusepath{clip}%
\pgfsetbuttcap%
\pgfsetroundjoin%
\definecolor{currentfill}{rgb}{0.800830,0.250829,0.225696}%
\pgfsetfillcolor{currentfill}%
\pgfsetlinewidth{0.000000pt}%
\definecolor{currentstroke}{rgb}{0.000000,0.000000,0.000000}%
\pgfsetstrokecolor{currentstroke}%
\pgfsetdash{}{0pt}%
\pgfpathmoveto{\pgfqpoint{1.971633in}{2.586781in}}%
\pgfpathlineto{\pgfqpoint{2.410872in}{2.508083in}}%
\pgfpathlineto{\pgfqpoint{2.674254in}{2.688576in}}%
\pgfpathlineto{\pgfqpoint{2.237924in}{2.752812in}}%
\pgfpathlineto{\pgfqpoint{1.971633in}{2.586781in}}%
\pgfpathclose%
\pgfusepath{fill}%
\end{pgfscope}%
\begin{pgfscope}%
\pgfpathrectangle{\pgfqpoint{0.017500in}{0.017500in}}{\pgfqpoint{6.218282in}{3.580372in}} %
\pgfusepath{clip}%
\pgfsetbuttcap%
\pgfsetroundjoin%
\definecolor{currentfill}{rgb}{0.800830,0.250829,0.225696}%
\pgfsetfillcolor{currentfill}%
\pgfsetlinewidth{0.000000pt}%
\definecolor{currentstroke}{rgb}{0.000000,0.000000,0.000000}%
\pgfsetstrokecolor{currentstroke}%
\pgfsetdash{}{0pt}%
\pgfpathmoveto{\pgfqpoint{1.971633in}{2.586781in}}%
\pgfpathlineto{\pgfqpoint{2.410872in}{2.508083in}}%
\pgfpathlineto{\pgfqpoint{2.674254in}{2.688576in}}%
\pgfpathlineto{\pgfqpoint{2.237924in}{2.752812in}}%
\pgfpathlineto{\pgfqpoint{1.971633in}{2.586781in}}%
\pgfpathclose%
\pgfusepath{fill}%
\end{pgfscope}%
\begin{pgfscope}%
\pgfpathrectangle{\pgfqpoint{0.017500in}{0.017500in}}{\pgfqpoint{6.218282in}{3.580372in}} %
\pgfusepath{clip}%
\pgfsetbuttcap%
\pgfsetroundjoin%
\definecolor{currentfill}{rgb}{0.800830,0.250829,0.225696}%
\pgfsetfillcolor{currentfill}%
\pgfsetlinewidth{0.000000pt}%
\definecolor{currentstroke}{rgb}{0.000000,0.000000,0.000000}%
\pgfsetstrokecolor{currentstroke}%
\pgfsetdash{}{0pt}%
\pgfpathmoveto{\pgfqpoint{1.971633in}{2.586781in}}%
\pgfpathlineto{\pgfqpoint{2.410872in}{2.508083in}}%
\pgfpathlineto{\pgfqpoint{2.674254in}{2.688576in}}%
\pgfpathlineto{\pgfqpoint{2.237924in}{2.752812in}}%
\pgfpathlineto{\pgfqpoint{1.971633in}{2.586781in}}%
\pgfpathclose%
\pgfusepath{fill}%
\end{pgfscope}%
\begin{pgfscope}%
\pgfpathrectangle{\pgfqpoint{0.017500in}{0.017500in}}{\pgfqpoint{6.218282in}{3.580372in}} %
\pgfusepath{clip}%
\pgfsetbuttcap%
\pgfsetroundjoin%
\definecolor{currentfill}{rgb}{0.800830,0.250829,0.225696}%
\pgfsetfillcolor{currentfill}%
\pgfsetlinewidth{0.000000pt}%
\definecolor{currentstroke}{rgb}{0.000000,0.000000,0.000000}%
\pgfsetstrokecolor{currentstroke}%
\pgfsetdash{}{0pt}%
\pgfpathmoveto{\pgfqpoint{1.971633in}{2.586781in}}%
\pgfpathlineto{\pgfqpoint{2.410872in}{2.508083in}}%
\pgfpathlineto{\pgfqpoint{2.674254in}{2.688576in}}%
\pgfpathlineto{\pgfqpoint{2.237924in}{2.752812in}}%
\pgfpathlineto{\pgfqpoint{1.971633in}{2.586781in}}%
\pgfpathclose%
\pgfusepath{fill}%
\end{pgfscope}%
\begin{pgfscope}%
\pgfpathrectangle{\pgfqpoint{0.017500in}{0.017500in}}{\pgfqpoint{6.218282in}{3.580372in}} %
\pgfusepath{clip}%
\pgfsetbuttcap%
\pgfsetroundjoin%
\definecolor{currentfill}{rgb}{0.800830,0.250829,0.225696}%
\pgfsetfillcolor{currentfill}%
\pgfsetlinewidth{0.000000pt}%
\definecolor{currentstroke}{rgb}{0.000000,0.000000,0.000000}%
\pgfsetstrokecolor{currentstroke}%
\pgfsetdash{}{0pt}%
\pgfpathmoveto{\pgfqpoint{1.971633in}{2.586781in}}%
\pgfpathlineto{\pgfqpoint{2.410872in}{2.508083in}}%
\pgfpathlineto{\pgfqpoint{2.674254in}{2.688576in}}%
\pgfpathlineto{\pgfqpoint{2.237924in}{2.752812in}}%
\pgfpathlineto{\pgfqpoint{1.971633in}{2.586781in}}%
\pgfpathclose%
\pgfusepath{fill}%
\end{pgfscope}%
\begin{pgfscope}%
\pgfpathrectangle{\pgfqpoint{0.017500in}{0.017500in}}{\pgfqpoint{6.218282in}{3.580372in}} %
\pgfusepath{clip}%
\pgfsetbuttcap%
\pgfsetroundjoin%
\definecolor{currentfill}{rgb}{0.800830,0.250829,0.225696}%
\pgfsetfillcolor{currentfill}%
\pgfsetlinewidth{0.000000pt}%
\definecolor{currentstroke}{rgb}{0.000000,0.000000,0.000000}%
\pgfsetstrokecolor{currentstroke}%
\pgfsetdash{}{0pt}%
\pgfpathmoveto{\pgfqpoint{1.971633in}{2.586781in}}%
\pgfpathlineto{\pgfqpoint{2.410872in}{2.508083in}}%
\pgfpathlineto{\pgfqpoint{2.674254in}{2.688576in}}%
\pgfpathlineto{\pgfqpoint{2.237924in}{2.752812in}}%
\pgfpathlineto{\pgfqpoint{1.971633in}{2.586781in}}%
\pgfpathclose%
\pgfusepath{fill}%
\end{pgfscope}%
\begin{pgfscope}%
\pgfpathrectangle{\pgfqpoint{0.017500in}{0.017500in}}{\pgfqpoint{6.218282in}{3.580372in}} %
\pgfusepath{clip}%
\pgfsetbuttcap%
\pgfsetroundjoin%
\definecolor{currentfill}{rgb}{0.243520,0.319189,0.771672}%
\pgfsetfillcolor{currentfill}%
\pgfsetlinewidth{0.000000pt}%
\definecolor{currentstroke}{rgb}{0.000000,0.000000,0.000000}%
\pgfsetstrokecolor{currentstroke}%
\pgfsetdash{}{0pt}%
\pgfpathmoveto{\pgfqpoint{1.914830in}{0.967594in}}%
\pgfpathlineto{\pgfqpoint{2.143341in}{2.291288in}}%
\pgfpathlineto{\pgfqpoint{1.914830in}{0.967594in}}%
\pgfpathclose%
\pgfusepath{fill}%
\end{pgfscope}%
\begin{pgfscope}%
\pgfpathrectangle{\pgfqpoint{0.017500in}{0.017500in}}{\pgfqpoint{6.218282in}{3.580372in}} %
\pgfusepath{clip}%
\pgfsetbuttcap%
\pgfsetroundjoin%
\definecolor{currentfill}{rgb}{0.243520,0.319189,0.771672}%
\pgfsetfillcolor{currentfill}%
\pgfsetlinewidth{0.000000pt}%
\definecolor{currentstroke}{rgb}{0.000000,0.000000,0.000000}%
\pgfsetstrokecolor{currentstroke}%
\pgfsetdash{}{0pt}%
\pgfpathmoveto{\pgfqpoint{1.914830in}{0.967594in}}%
\pgfpathlineto{\pgfqpoint{2.143341in}{2.291288in}}%
\pgfpathlineto{\pgfqpoint{1.914830in}{0.967594in}}%
\pgfpathclose%
\pgfusepath{fill}%
\end{pgfscope}%
\begin{pgfscope}%
\pgfpathrectangle{\pgfqpoint{0.017500in}{0.017500in}}{\pgfqpoint{6.218282in}{3.580372in}} %
\pgfusepath{clip}%
\pgfsetbuttcap%
\pgfsetroundjoin%
\definecolor{currentfill}{rgb}{0.243520,0.319189,0.771672}%
\pgfsetfillcolor{currentfill}%
\pgfsetlinewidth{0.000000pt}%
\definecolor{currentstroke}{rgb}{0.000000,0.000000,0.000000}%
\pgfsetstrokecolor{currentstroke}%
\pgfsetdash{}{0pt}%
\pgfpathmoveto{\pgfqpoint{1.914830in}{0.967594in}}%
\pgfpathlineto{\pgfqpoint{2.143341in}{2.291288in}}%
\pgfpathlineto{\pgfqpoint{1.914830in}{0.967594in}}%
\pgfpathclose%
\pgfusepath{fill}%
\end{pgfscope}%
\begin{pgfscope}%
\pgfpathrectangle{\pgfqpoint{0.017500in}{0.017500in}}{\pgfqpoint{6.218282in}{3.580372in}} %
\pgfusepath{clip}%
\pgfsetbuttcap%
\pgfsetroundjoin%
\definecolor{currentfill}{rgb}{0.243520,0.319189,0.771672}%
\pgfsetfillcolor{currentfill}%
\pgfsetlinewidth{0.000000pt}%
\definecolor{currentstroke}{rgb}{0.000000,0.000000,0.000000}%
\pgfsetstrokecolor{currentstroke}%
\pgfsetdash{}{0pt}%
\pgfpathmoveto{\pgfqpoint{1.914830in}{0.967594in}}%
\pgfpathlineto{\pgfqpoint{2.143341in}{2.291288in}}%
\pgfpathlineto{\pgfqpoint{1.914830in}{0.967594in}}%
\pgfpathclose%
\pgfusepath{fill}%
\end{pgfscope}%
\begin{pgfscope}%
\pgfpathrectangle{\pgfqpoint{0.017500in}{0.017500in}}{\pgfqpoint{6.218282in}{3.580372in}} %
\pgfusepath{clip}%
\pgfsetbuttcap%
\pgfsetroundjoin%
\definecolor{currentfill}{rgb}{0.243520,0.319189,0.771672}%
\pgfsetfillcolor{currentfill}%
\pgfsetlinewidth{0.000000pt}%
\definecolor{currentstroke}{rgb}{0.000000,0.000000,0.000000}%
\pgfsetstrokecolor{currentstroke}%
\pgfsetdash{}{0pt}%
\pgfpathmoveto{\pgfqpoint{1.914830in}{0.967594in}}%
\pgfpathlineto{\pgfqpoint{2.143341in}{2.291288in}}%
\pgfpathlineto{\pgfqpoint{1.914830in}{0.967594in}}%
\pgfpathclose%
\pgfusepath{fill}%
\end{pgfscope}%
\begin{pgfscope}%
\pgfpathrectangle{\pgfqpoint{0.017500in}{0.017500in}}{\pgfqpoint{6.218282in}{3.580372in}} %
\pgfusepath{clip}%
\pgfsetbuttcap%
\pgfsetroundjoin%
\definecolor{currentfill}{rgb}{0.243520,0.319189,0.771672}%
\pgfsetfillcolor{currentfill}%
\pgfsetlinewidth{0.000000pt}%
\definecolor{currentstroke}{rgb}{0.000000,0.000000,0.000000}%
\pgfsetstrokecolor{currentstroke}%
\pgfsetdash{}{0pt}%
\pgfpathmoveto{\pgfqpoint{1.914830in}{0.967594in}}%
\pgfpathlineto{\pgfqpoint{2.143341in}{2.291288in}}%
\pgfpathlineto{\pgfqpoint{1.914830in}{0.967594in}}%
\pgfpathclose%
\pgfusepath{fill}%
\end{pgfscope}%
\begin{pgfscope}%
\pgfpathrectangle{\pgfqpoint{0.017500in}{0.017500in}}{\pgfqpoint{6.218282in}{3.580372in}} %
\pgfusepath{clip}%
\pgfsetbuttcap%
\pgfsetroundjoin%
\definecolor{currentfill}{rgb}{0.243520,0.319189,0.771672}%
\pgfsetfillcolor{currentfill}%
\pgfsetlinewidth{0.000000pt}%
\definecolor{currentstroke}{rgb}{0.000000,0.000000,0.000000}%
\pgfsetstrokecolor{currentstroke}%
\pgfsetdash{}{0pt}%
\pgfpathmoveto{\pgfqpoint{1.914830in}{0.967594in}}%
\pgfpathlineto{\pgfqpoint{2.143341in}{2.291288in}}%
\pgfpathlineto{\pgfqpoint{1.914830in}{0.967594in}}%
\pgfpathclose%
\pgfusepath{fill}%
\end{pgfscope}%
\begin{pgfscope}%
\pgfpathrectangle{\pgfqpoint{0.017500in}{0.017500in}}{\pgfqpoint{6.218282in}{3.580372in}} %
\pgfusepath{clip}%
\pgfsetbuttcap%
\pgfsetroundjoin%
\definecolor{currentfill}{rgb}{0.243520,0.319189,0.771672}%
\pgfsetfillcolor{currentfill}%
\pgfsetlinewidth{0.000000pt}%
\definecolor{currentstroke}{rgb}{0.000000,0.000000,0.000000}%
\pgfsetstrokecolor{currentstroke}%
\pgfsetdash{}{0pt}%
\pgfpathmoveto{\pgfqpoint{1.914830in}{0.967594in}}%
\pgfpathlineto{\pgfqpoint{2.143341in}{2.291288in}}%
\pgfpathlineto{\pgfqpoint{1.914830in}{0.967594in}}%
\pgfpathclose%
\pgfusepath{fill}%
\end{pgfscope}%
\begin{pgfscope}%
\pgfpathrectangle{\pgfqpoint{0.017500in}{0.017500in}}{\pgfqpoint{6.218282in}{3.580372in}} %
\pgfusepath{clip}%
\pgfsetbuttcap%
\pgfsetroundjoin%
\definecolor{currentfill}{rgb}{0.243520,0.319189,0.771672}%
\pgfsetfillcolor{currentfill}%
\pgfsetlinewidth{0.000000pt}%
\definecolor{currentstroke}{rgb}{0.000000,0.000000,0.000000}%
\pgfsetstrokecolor{currentstroke}%
\pgfsetdash{}{0pt}%
\pgfpathmoveto{\pgfqpoint{1.914830in}{0.967594in}}%
\pgfpathlineto{\pgfqpoint{2.143341in}{2.291288in}}%
\pgfpathlineto{\pgfqpoint{1.914830in}{0.967594in}}%
\pgfpathclose%
\pgfusepath{fill}%
\end{pgfscope}%
\begin{pgfscope}%
\pgfpathrectangle{\pgfqpoint{0.017500in}{0.017500in}}{\pgfqpoint{6.218282in}{3.580372in}} %
\pgfusepath{clip}%
\pgfsetbuttcap%
\pgfsetroundjoin%
\definecolor{currentfill}{rgb}{0.243520,0.319189,0.771672}%
\pgfsetfillcolor{currentfill}%
\pgfsetlinewidth{0.000000pt}%
\definecolor{currentstroke}{rgb}{0.000000,0.000000,0.000000}%
\pgfsetstrokecolor{currentstroke}%
\pgfsetdash{}{0pt}%
\pgfpathmoveto{\pgfqpoint{1.914830in}{0.967594in}}%
\pgfpathlineto{\pgfqpoint{2.143341in}{2.291288in}}%
\pgfpathlineto{\pgfqpoint{1.914830in}{0.967594in}}%
\pgfpathclose%
\pgfusepath{fill}%
\end{pgfscope}%
\begin{pgfscope}%
\pgfpathrectangle{\pgfqpoint{0.017500in}{0.017500in}}{\pgfqpoint{6.218282in}{3.580372in}} %
\pgfusepath{clip}%
\pgfsetbuttcap%
\pgfsetroundjoin%
\definecolor{currentfill}{rgb}{0.243520,0.319189,0.771672}%
\pgfsetfillcolor{currentfill}%
\pgfsetlinewidth{0.000000pt}%
\definecolor{currentstroke}{rgb}{0.000000,0.000000,0.000000}%
\pgfsetstrokecolor{currentstroke}%
\pgfsetdash{}{0pt}%
\pgfpathmoveto{\pgfqpoint{1.914830in}{0.967594in}}%
\pgfpathlineto{\pgfqpoint{2.143341in}{2.291288in}}%
\pgfpathlineto{\pgfqpoint{1.914830in}{0.967594in}}%
\pgfpathclose%
\pgfusepath{fill}%
\end{pgfscope}%
\begin{pgfscope}%
\pgfpathrectangle{\pgfqpoint{0.017500in}{0.017500in}}{\pgfqpoint{6.218282in}{3.580372in}} %
\pgfusepath{clip}%
\pgfsetbuttcap%
\pgfsetroundjoin%
\definecolor{currentfill}{rgb}{0.243520,0.319189,0.771672}%
\pgfsetfillcolor{currentfill}%
\pgfsetlinewidth{0.000000pt}%
\definecolor{currentstroke}{rgb}{0.000000,0.000000,0.000000}%
\pgfsetstrokecolor{currentstroke}%
\pgfsetdash{}{0pt}%
\pgfpathmoveto{\pgfqpoint{1.914830in}{0.967594in}}%
\pgfpathlineto{\pgfqpoint{2.143341in}{2.291288in}}%
\pgfpathlineto{\pgfqpoint{1.914830in}{0.967594in}}%
\pgfpathclose%
\pgfusepath{fill}%
\end{pgfscope}%
\begin{pgfscope}%
\pgfpathrectangle{\pgfqpoint{0.017500in}{0.017500in}}{\pgfqpoint{6.218282in}{3.580372in}} %
\pgfusepath{clip}%
\pgfsetbuttcap%
\pgfsetroundjoin%
\definecolor{currentfill}{rgb}{0.243520,0.319189,0.771672}%
\pgfsetfillcolor{currentfill}%
\pgfsetlinewidth{0.000000pt}%
\definecolor{currentstroke}{rgb}{0.000000,0.000000,0.000000}%
\pgfsetstrokecolor{currentstroke}%
\pgfsetdash{}{0pt}%
\pgfpathmoveto{\pgfqpoint{1.914830in}{0.967594in}}%
\pgfpathlineto{\pgfqpoint{2.143341in}{2.291288in}}%
\pgfpathlineto{\pgfqpoint{1.914830in}{0.967594in}}%
\pgfpathclose%
\pgfusepath{fill}%
\end{pgfscope}%
\begin{pgfscope}%
\pgfpathrectangle{\pgfqpoint{0.017500in}{0.017500in}}{\pgfqpoint{6.218282in}{3.580372in}} %
\pgfusepath{clip}%
\pgfsetbuttcap%
\pgfsetroundjoin%
\definecolor{currentfill}{rgb}{0.243520,0.319189,0.771672}%
\pgfsetfillcolor{currentfill}%
\pgfsetlinewidth{0.000000pt}%
\definecolor{currentstroke}{rgb}{0.000000,0.000000,0.000000}%
\pgfsetstrokecolor{currentstroke}%
\pgfsetdash{}{0pt}%
\pgfpathmoveto{\pgfqpoint{1.914830in}{0.967594in}}%
\pgfpathlineto{\pgfqpoint{2.143341in}{2.291288in}}%
\pgfpathlineto{\pgfqpoint{1.914830in}{0.967594in}}%
\pgfpathclose%
\pgfusepath{fill}%
\end{pgfscope}%
\begin{pgfscope}%
\pgfpathrectangle{\pgfqpoint{0.017500in}{0.017500in}}{\pgfqpoint{6.218282in}{3.580372in}} %
\pgfusepath{clip}%
\pgfsetbuttcap%
\pgfsetroundjoin%
\definecolor{currentfill}{rgb}{0.243520,0.319189,0.771672}%
\pgfsetfillcolor{currentfill}%
\pgfsetlinewidth{0.000000pt}%
\definecolor{currentstroke}{rgb}{0.000000,0.000000,0.000000}%
\pgfsetstrokecolor{currentstroke}%
\pgfsetdash{}{0pt}%
\pgfpathmoveto{\pgfqpoint{1.914830in}{0.967594in}}%
\pgfpathlineto{\pgfqpoint{2.143341in}{2.291288in}}%
\pgfpathlineto{\pgfqpoint{1.914830in}{0.967594in}}%
\pgfpathclose%
\pgfusepath{fill}%
\end{pgfscope}%
\begin{pgfscope}%
\pgfpathrectangle{\pgfqpoint{0.017500in}{0.017500in}}{\pgfqpoint{6.218282in}{3.580372in}} %
\pgfusepath{clip}%
\pgfsetbuttcap%
\pgfsetroundjoin%
\definecolor{currentfill}{rgb}{0.243520,0.319189,0.771672}%
\pgfsetfillcolor{currentfill}%
\pgfsetlinewidth{0.000000pt}%
\definecolor{currentstroke}{rgb}{0.000000,0.000000,0.000000}%
\pgfsetstrokecolor{currentstroke}%
\pgfsetdash{}{0pt}%
\pgfpathmoveto{\pgfqpoint{1.914830in}{0.967594in}}%
\pgfpathlineto{\pgfqpoint{2.143341in}{2.291288in}}%
\pgfpathlineto{\pgfqpoint{1.914830in}{0.967594in}}%
\pgfpathclose%
\pgfusepath{fill}%
\end{pgfscope}%
\begin{pgfscope}%
\pgfpathrectangle{\pgfqpoint{0.017500in}{0.017500in}}{\pgfqpoint{6.218282in}{3.580372in}} %
\pgfusepath{clip}%
\pgfsetbuttcap%
\pgfsetroundjoin%
\definecolor{currentfill}{rgb}{0.243520,0.319189,0.771672}%
\pgfsetfillcolor{currentfill}%
\pgfsetlinewidth{0.000000pt}%
\definecolor{currentstroke}{rgb}{0.000000,0.000000,0.000000}%
\pgfsetstrokecolor{currentstroke}%
\pgfsetdash{}{0pt}%
\pgfpathmoveto{\pgfqpoint{1.914830in}{0.967594in}}%
\pgfpathlineto{\pgfqpoint{2.143341in}{2.291288in}}%
\pgfpathlineto{\pgfqpoint{1.914830in}{0.967594in}}%
\pgfpathclose%
\pgfusepath{fill}%
\end{pgfscope}%
\begin{pgfscope}%
\pgfpathrectangle{\pgfqpoint{0.017500in}{0.017500in}}{\pgfqpoint{6.218282in}{3.580372in}} %
\pgfusepath{clip}%
\pgfsetbuttcap%
\pgfsetroundjoin%
\definecolor{currentfill}{rgb}{0.243520,0.319189,0.771672}%
\pgfsetfillcolor{currentfill}%
\pgfsetlinewidth{0.000000pt}%
\definecolor{currentstroke}{rgb}{0.000000,0.000000,0.000000}%
\pgfsetstrokecolor{currentstroke}%
\pgfsetdash{}{0pt}%
\pgfpathmoveto{\pgfqpoint{1.914830in}{0.967594in}}%
\pgfpathlineto{\pgfqpoint{2.143341in}{2.291288in}}%
\pgfpathlineto{\pgfqpoint{1.914830in}{0.967594in}}%
\pgfpathclose%
\pgfusepath{fill}%
\end{pgfscope}%
\begin{pgfscope}%
\pgfpathrectangle{\pgfqpoint{0.017500in}{0.017500in}}{\pgfqpoint{6.218282in}{3.580372in}} %
\pgfusepath{clip}%
\pgfsetbuttcap%
\pgfsetroundjoin%
\definecolor{currentfill}{rgb}{0.243520,0.319189,0.771672}%
\pgfsetfillcolor{currentfill}%
\pgfsetlinewidth{0.000000pt}%
\definecolor{currentstroke}{rgb}{0.000000,0.000000,0.000000}%
\pgfsetstrokecolor{currentstroke}%
\pgfsetdash{}{0pt}%
\pgfpathmoveto{\pgfqpoint{1.914830in}{0.967594in}}%
\pgfpathlineto{\pgfqpoint{2.143341in}{2.291288in}}%
\pgfpathlineto{\pgfqpoint{1.914830in}{0.967594in}}%
\pgfpathclose%
\pgfusepath{fill}%
\end{pgfscope}%
\begin{pgfscope}%
\pgfpathrectangle{\pgfqpoint{0.017500in}{0.017500in}}{\pgfqpoint{6.218282in}{3.580372in}} %
\pgfusepath{clip}%
\pgfsetbuttcap%
\pgfsetroundjoin%
\definecolor{currentfill}{rgb}{0.243520,0.319189,0.771672}%
\pgfsetfillcolor{currentfill}%
\pgfsetlinewidth{0.000000pt}%
\definecolor{currentstroke}{rgb}{0.000000,0.000000,0.000000}%
\pgfsetstrokecolor{currentstroke}%
\pgfsetdash{}{0pt}%
\pgfpathmoveto{\pgfqpoint{1.914830in}{0.967594in}}%
\pgfpathlineto{\pgfqpoint{2.143341in}{2.291288in}}%
\pgfpathlineto{\pgfqpoint{1.914830in}{0.967594in}}%
\pgfpathclose%
\pgfusepath{fill}%
\end{pgfscope}%
\begin{pgfscope}%
\pgfpathrectangle{\pgfqpoint{0.017500in}{0.017500in}}{\pgfqpoint{6.218282in}{3.580372in}} %
\pgfusepath{clip}%
\pgfsetbuttcap%
\pgfsetroundjoin%
\definecolor{currentfill}{rgb}{0.243520,0.319189,0.771672}%
\pgfsetfillcolor{currentfill}%
\pgfsetlinewidth{0.000000pt}%
\definecolor{currentstroke}{rgb}{0.000000,0.000000,0.000000}%
\pgfsetstrokecolor{currentstroke}%
\pgfsetdash{}{0pt}%
\pgfpathmoveto{\pgfqpoint{1.914830in}{0.967594in}}%
\pgfpathlineto{\pgfqpoint{2.143341in}{2.291288in}}%
\pgfpathlineto{\pgfqpoint{1.914830in}{0.967594in}}%
\pgfpathclose%
\pgfusepath{fill}%
\end{pgfscope}%
\begin{pgfscope}%
\pgfpathrectangle{\pgfqpoint{0.017500in}{0.017500in}}{\pgfqpoint{6.218282in}{3.580372in}} %
\pgfusepath{clip}%
\pgfsetbuttcap%
\pgfsetroundjoin%
\definecolor{currentfill}{rgb}{0.243520,0.319189,0.771672}%
\pgfsetfillcolor{currentfill}%
\pgfsetlinewidth{0.000000pt}%
\definecolor{currentstroke}{rgb}{0.000000,0.000000,0.000000}%
\pgfsetstrokecolor{currentstroke}%
\pgfsetdash{}{0pt}%
\pgfpathmoveto{\pgfqpoint{1.914830in}{0.967594in}}%
\pgfpathlineto{\pgfqpoint{2.143341in}{2.291288in}}%
\pgfpathlineto{\pgfqpoint{1.914830in}{0.967594in}}%
\pgfpathclose%
\pgfusepath{fill}%
\end{pgfscope}%
\begin{pgfscope}%
\pgfpathrectangle{\pgfqpoint{0.017500in}{0.017500in}}{\pgfqpoint{6.218282in}{3.580372in}} %
\pgfusepath{clip}%
\pgfsetbuttcap%
\pgfsetroundjoin%
\definecolor{currentfill}{rgb}{0.243520,0.319189,0.771672}%
\pgfsetfillcolor{currentfill}%
\pgfsetlinewidth{0.000000pt}%
\definecolor{currentstroke}{rgb}{0.000000,0.000000,0.000000}%
\pgfsetstrokecolor{currentstroke}%
\pgfsetdash{}{0pt}%
\pgfpathmoveto{\pgfqpoint{1.914830in}{0.967594in}}%
\pgfpathlineto{\pgfqpoint{2.143341in}{2.291288in}}%
\pgfpathlineto{\pgfqpoint{1.914830in}{0.967594in}}%
\pgfpathclose%
\pgfusepath{fill}%
\end{pgfscope}%
\begin{pgfscope}%
\pgfpathrectangle{\pgfqpoint{0.017500in}{0.017500in}}{\pgfqpoint{6.218282in}{3.580372in}} %
\pgfusepath{clip}%
\pgfsetbuttcap%
\pgfsetroundjoin%
\definecolor{currentfill}{rgb}{0.243520,0.319189,0.771672}%
\pgfsetfillcolor{currentfill}%
\pgfsetlinewidth{0.000000pt}%
\definecolor{currentstroke}{rgb}{0.000000,0.000000,0.000000}%
\pgfsetstrokecolor{currentstroke}%
\pgfsetdash{}{0pt}%
\pgfpathmoveto{\pgfqpoint{1.914830in}{0.967594in}}%
\pgfpathlineto{\pgfqpoint{2.143341in}{2.291288in}}%
\pgfpathlineto{\pgfqpoint{1.914830in}{0.967594in}}%
\pgfpathclose%
\pgfusepath{fill}%
\end{pgfscope}%
\begin{pgfscope}%
\pgfpathrectangle{\pgfqpoint{0.017500in}{0.017500in}}{\pgfqpoint{6.218282in}{3.580372in}} %
\pgfusepath{clip}%
\pgfsetbuttcap%
\pgfsetroundjoin%
\definecolor{currentfill}{rgb}{0.243520,0.319189,0.771672}%
\pgfsetfillcolor{currentfill}%
\pgfsetlinewidth{0.000000pt}%
\definecolor{currentstroke}{rgb}{0.000000,0.000000,0.000000}%
\pgfsetstrokecolor{currentstroke}%
\pgfsetdash{}{0pt}%
\pgfpathmoveto{\pgfqpoint{1.914830in}{0.967594in}}%
\pgfpathlineto{\pgfqpoint{2.143341in}{2.291288in}}%
\pgfpathlineto{\pgfqpoint{1.914830in}{0.967594in}}%
\pgfpathclose%
\pgfusepath{fill}%
\end{pgfscope}%
\begin{pgfscope}%
\pgfpathrectangle{\pgfqpoint{0.017500in}{0.017500in}}{\pgfqpoint{6.218282in}{3.580372in}} %
\pgfusepath{clip}%
\pgfsetbuttcap%
\pgfsetroundjoin%
\definecolor{currentfill}{rgb}{0.243520,0.319189,0.771672}%
\pgfsetfillcolor{currentfill}%
\pgfsetlinewidth{0.000000pt}%
\definecolor{currentstroke}{rgb}{0.000000,0.000000,0.000000}%
\pgfsetstrokecolor{currentstroke}%
\pgfsetdash{}{0pt}%
\pgfpathmoveto{\pgfqpoint{1.914830in}{0.967594in}}%
\pgfpathlineto{\pgfqpoint{2.143341in}{2.291288in}}%
\pgfpathlineto{\pgfqpoint{1.914830in}{0.967594in}}%
\pgfpathclose%
\pgfusepath{fill}%
\end{pgfscope}%
\begin{pgfscope}%
\pgfpathrectangle{\pgfqpoint{0.017500in}{0.017500in}}{\pgfqpoint{6.218282in}{3.580372in}} %
\pgfusepath{clip}%
\pgfsetbuttcap%
\pgfsetroundjoin%
\definecolor{currentfill}{rgb}{0.243520,0.319189,0.771672}%
\pgfsetfillcolor{currentfill}%
\pgfsetlinewidth{0.000000pt}%
\definecolor{currentstroke}{rgb}{0.000000,0.000000,0.000000}%
\pgfsetstrokecolor{currentstroke}%
\pgfsetdash{}{0pt}%
\pgfpathmoveto{\pgfqpoint{1.914830in}{0.967594in}}%
\pgfpathlineto{\pgfqpoint{2.143341in}{2.291288in}}%
\pgfpathlineto{\pgfqpoint{1.914830in}{0.967594in}}%
\pgfpathclose%
\pgfusepath{fill}%
\end{pgfscope}%
\begin{pgfscope}%
\pgfpathrectangle{\pgfqpoint{0.017500in}{0.017500in}}{\pgfqpoint{6.218282in}{3.580372in}} %
\pgfusepath{clip}%
\pgfsetbuttcap%
\pgfsetroundjoin%
\definecolor{currentfill}{rgb}{0.243520,0.319189,0.771672}%
\pgfsetfillcolor{currentfill}%
\pgfsetlinewidth{0.000000pt}%
\definecolor{currentstroke}{rgb}{0.000000,0.000000,0.000000}%
\pgfsetstrokecolor{currentstroke}%
\pgfsetdash{}{0pt}%
\pgfpathmoveto{\pgfqpoint{1.914830in}{0.967594in}}%
\pgfpathlineto{\pgfqpoint{2.143341in}{2.291288in}}%
\pgfpathlineto{\pgfqpoint{1.914830in}{0.967594in}}%
\pgfpathclose%
\pgfusepath{fill}%
\end{pgfscope}%
\begin{pgfscope}%
\pgfpathrectangle{\pgfqpoint{0.017500in}{0.017500in}}{\pgfqpoint{6.218282in}{3.580372in}} %
\pgfusepath{clip}%
\pgfsetbuttcap%
\pgfsetroundjoin%
\definecolor{currentfill}{rgb}{0.243520,0.319189,0.771672}%
\pgfsetfillcolor{currentfill}%
\pgfsetlinewidth{0.000000pt}%
\definecolor{currentstroke}{rgb}{0.000000,0.000000,0.000000}%
\pgfsetstrokecolor{currentstroke}%
\pgfsetdash{}{0pt}%
\pgfpathmoveto{\pgfqpoint{1.914830in}{0.967594in}}%
\pgfpathlineto{\pgfqpoint{2.143341in}{2.291288in}}%
\pgfpathlineto{\pgfqpoint{1.914830in}{0.967594in}}%
\pgfpathclose%
\pgfusepath{fill}%
\end{pgfscope}%
\begin{pgfscope}%
\pgfpathrectangle{\pgfqpoint{0.017500in}{0.017500in}}{\pgfqpoint{6.218282in}{3.580372in}} %
\pgfusepath{clip}%
\pgfsetbuttcap%
\pgfsetroundjoin%
\definecolor{currentfill}{rgb}{0.243520,0.319189,0.771672}%
\pgfsetfillcolor{currentfill}%
\pgfsetlinewidth{0.000000pt}%
\definecolor{currentstroke}{rgb}{0.000000,0.000000,0.000000}%
\pgfsetstrokecolor{currentstroke}%
\pgfsetdash{}{0pt}%
\pgfpathmoveto{\pgfqpoint{1.914830in}{0.967594in}}%
\pgfpathlineto{\pgfqpoint{2.143341in}{2.291288in}}%
\pgfpathlineto{\pgfqpoint{1.914830in}{0.967594in}}%
\pgfpathclose%
\pgfusepath{fill}%
\end{pgfscope}%
\begin{pgfscope}%
\pgfpathrectangle{\pgfqpoint{0.017500in}{0.017500in}}{\pgfqpoint{6.218282in}{3.580372in}} %
\pgfusepath{clip}%
\pgfsetbuttcap%
\pgfsetroundjoin%
\definecolor{currentfill}{rgb}{0.758112,0.168122,0.188827}%
\pgfsetfillcolor{currentfill}%
\pgfsetlinewidth{0.000000pt}%
\definecolor{currentstroke}{rgb}{0.000000,0.000000,0.000000}%
\pgfsetstrokecolor{currentstroke}%
\pgfsetdash{}{0pt}%
\pgfpathmoveto{\pgfqpoint{3.116513in}{2.603812in}}%
\pgfpathlineto{\pgfqpoint{3.372093in}{2.761754in}}%
\pgfpathlineto{\pgfqpoint{3.116513in}{2.603812in}}%
\pgfpathclose%
\pgfusepath{fill}%
\end{pgfscope}%
\begin{pgfscope}%
\pgfpathrectangle{\pgfqpoint{0.017500in}{0.017500in}}{\pgfqpoint{6.218282in}{3.580372in}} %
\pgfusepath{clip}%
\pgfsetbuttcap%
\pgfsetroundjoin%
\definecolor{currentfill}{rgb}{0.758112,0.168122,0.188827}%
\pgfsetfillcolor{currentfill}%
\pgfsetlinewidth{0.000000pt}%
\definecolor{currentstroke}{rgb}{0.000000,0.000000,0.000000}%
\pgfsetstrokecolor{currentstroke}%
\pgfsetdash{}{0pt}%
\pgfpathmoveto{\pgfqpoint{3.116513in}{2.603812in}}%
\pgfpathlineto{\pgfqpoint{3.372093in}{2.761754in}}%
\pgfpathlineto{\pgfqpoint{3.116513in}{2.603812in}}%
\pgfpathclose%
\pgfusepath{fill}%
\end{pgfscope}%
\begin{pgfscope}%
\pgfpathrectangle{\pgfqpoint{0.017500in}{0.017500in}}{\pgfqpoint{6.218282in}{3.580372in}} %
\pgfusepath{clip}%
\pgfsetbuttcap%
\pgfsetroundjoin%
\definecolor{currentfill}{rgb}{0.758112,0.168122,0.188827}%
\pgfsetfillcolor{currentfill}%
\pgfsetlinewidth{0.000000pt}%
\definecolor{currentstroke}{rgb}{0.000000,0.000000,0.000000}%
\pgfsetstrokecolor{currentstroke}%
\pgfsetdash{}{0pt}%
\pgfpathmoveto{\pgfqpoint{3.116513in}{2.603812in}}%
\pgfpathlineto{\pgfqpoint{3.372093in}{2.761754in}}%
\pgfpathlineto{\pgfqpoint{3.116513in}{2.603812in}}%
\pgfpathclose%
\pgfusepath{fill}%
\end{pgfscope}%
\begin{pgfscope}%
\pgfpathrectangle{\pgfqpoint{0.017500in}{0.017500in}}{\pgfqpoint{6.218282in}{3.580372in}} %
\pgfusepath{clip}%
\pgfsetbuttcap%
\pgfsetroundjoin%
\definecolor{currentfill}{rgb}{0.758112,0.168122,0.188827}%
\pgfsetfillcolor{currentfill}%
\pgfsetlinewidth{0.000000pt}%
\definecolor{currentstroke}{rgb}{0.000000,0.000000,0.000000}%
\pgfsetstrokecolor{currentstroke}%
\pgfsetdash{}{0pt}%
\pgfpathmoveto{\pgfqpoint{3.116513in}{2.603812in}}%
\pgfpathlineto{\pgfqpoint{3.372093in}{2.761754in}}%
\pgfpathlineto{\pgfqpoint{3.116513in}{2.603812in}}%
\pgfpathclose%
\pgfusepath{fill}%
\end{pgfscope}%
\begin{pgfscope}%
\pgfpathrectangle{\pgfqpoint{0.017500in}{0.017500in}}{\pgfqpoint{6.218282in}{3.580372in}} %
\pgfusepath{clip}%
\pgfsetbuttcap%
\pgfsetroundjoin%
\definecolor{currentfill}{rgb}{0.758112,0.168122,0.188827}%
\pgfsetfillcolor{currentfill}%
\pgfsetlinewidth{0.000000pt}%
\definecolor{currentstroke}{rgb}{0.000000,0.000000,0.000000}%
\pgfsetstrokecolor{currentstroke}%
\pgfsetdash{}{0pt}%
\pgfpathmoveto{\pgfqpoint{3.116513in}{2.603812in}}%
\pgfpathlineto{\pgfqpoint{3.372093in}{2.761754in}}%
\pgfpathlineto{\pgfqpoint{3.116513in}{2.603812in}}%
\pgfpathclose%
\pgfusepath{fill}%
\end{pgfscope}%
\begin{pgfscope}%
\pgfpathrectangle{\pgfqpoint{0.017500in}{0.017500in}}{\pgfqpoint{6.218282in}{3.580372in}} %
\pgfusepath{clip}%
\pgfsetbuttcap%
\pgfsetroundjoin%
\definecolor{currentfill}{rgb}{0.758112,0.168122,0.188827}%
\pgfsetfillcolor{currentfill}%
\pgfsetlinewidth{0.000000pt}%
\definecolor{currentstroke}{rgb}{0.000000,0.000000,0.000000}%
\pgfsetstrokecolor{currentstroke}%
\pgfsetdash{}{0pt}%
\pgfpathmoveto{\pgfqpoint{3.116513in}{2.603812in}}%
\pgfpathlineto{\pgfqpoint{3.372093in}{2.761754in}}%
\pgfpathlineto{\pgfqpoint{3.116513in}{2.603812in}}%
\pgfpathclose%
\pgfusepath{fill}%
\end{pgfscope}%
\begin{pgfscope}%
\pgfpathrectangle{\pgfqpoint{0.017500in}{0.017500in}}{\pgfqpoint{6.218282in}{3.580372in}} %
\pgfusepath{clip}%
\pgfsetbuttcap%
\pgfsetroundjoin%
\definecolor{currentfill}{rgb}{0.758112,0.168122,0.188827}%
\pgfsetfillcolor{currentfill}%
\pgfsetlinewidth{0.000000pt}%
\definecolor{currentstroke}{rgb}{0.000000,0.000000,0.000000}%
\pgfsetstrokecolor{currentstroke}%
\pgfsetdash{}{0pt}%
\pgfpathmoveto{\pgfqpoint{3.116513in}{2.603812in}}%
\pgfpathlineto{\pgfqpoint{3.372093in}{2.761754in}}%
\pgfpathlineto{\pgfqpoint{3.116513in}{2.603812in}}%
\pgfpathclose%
\pgfusepath{fill}%
\end{pgfscope}%
\begin{pgfscope}%
\pgfpathrectangle{\pgfqpoint{0.017500in}{0.017500in}}{\pgfqpoint{6.218282in}{3.580372in}} %
\pgfusepath{clip}%
\pgfsetbuttcap%
\pgfsetroundjoin%
\definecolor{currentfill}{rgb}{0.758112,0.168122,0.188827}%
\pgfsetfillcolor{currentfill}%
\pgfsetlinewidth{0.000000pt}%
\definecolor{currentstroke}{rgb}{0.000000,0.000000,0.000000}%
\pgfsetstrokecolor{currentstroke}%
\pgfsetdash{}{0pt}%
\pgfpathmoveto{\pgfqpoint{3.116513in}{2.603812in}}%
\pgfpathlineto{\pgfqpoint{3.372093in}{2.761754in}}%
\pgfpathlineto{\pgfqpoint{3.116513in}{2.603812in}}%
\pgfpathclose%
\pgfusepath{fill}%
\end{pgfscope}%
\begin{pgfscope}%
\pgfpathrectangle{\pgfqpoint{0.017500in}{0.017500in}}{\pgfqpoint{6.218282in}{3.580372in}} %
\pgfusepath{clip}%
\pgfsetbuttcap%
\pgfsetroundjoin%
\definecolor{currentfill}{rgb}{0.758112,0.168122,0.188827}%
\pgfsetfillcolor{currentfill}%
\pgfsetlinewidth{0.000000pt}%
\definecolor{currentstroke}{rgb}{0.000000,0.000000,0.000000}%
\pgfsetstrokecolor{currentstroke}%
\pgfsetdash{}{0pt}%
\pgfpathmoveto{\pgfqpoint{3.116513in}{2.603812in}}%
\pgfpathlineto{\pgfqpoint{3.372093in}{2.761754in}}%
\pgfpathlineto{\pgfqpoint{3.116513in}{2.603812in}}%
\pgfpathclose%
\pgfusepath{fill}%
\end{pgfscope}%
\begin{pgfscope}%
\pgfpathrectangle{\pgfqpoint{0.017500in}{0.017500in}}{\pgfqpoint{6.218282in}{3.580372in}} %
\pgfusepath{clip}%
\pgfsetbuttcap%
\pgfsetroundjoin%
\definecolor{currentfill}{rgb}{0.758112,0.168122,0.188827}%
\pgfsetfillcolor{currentfill}%
\pgfsetlinewidth{0.000000pt}%
\definecolor{currentstroke}{rgb}{0.000000,0.000000,0.000000}%
\pgfsetstrokecolor{currentstroke}%
\pgfsetdash{}{0pt}%
\pgfpathmoveto{\pgfqpoint{3.116513in}{2.603812in}}%
\pgfpathlineto{\pgfqpoint{3.372093in}{2.761754in}}%
\pgfpathlineto{\pgfqpoint{3.116513in}{2.603812in}}%
\pgfpathclose%
\pgfusepath{fill}%
\end{pgfscope}%
\begin{pgfscope}%
\pgfpathrectangle{\pgfqpoint{0.017500in}{0.017500in}}{\pgfqpoint{6.218282in}{3.580372in}} %
\pgfusepath{clip}%
\pgfsetbuttcap%
\pgfsetroundjoin%
\definecolor{currentfill}{rgb}{0.758112,0.168122,0.188827}%
\pgfsetfillcolor{currentfill}%
\pgfsetlinewidth{0.000000pt}%
\definecolor{currentstroke}{rgb}{0.000000,0.000000,0.000000}%
\pgfsetstrokecolor{currentstroke}%
\pgfsetdash{}{0pt}%
\pgfpathmoveto{\pgfqpoint{3.116513in}{2.603812in}}%
\pgfpathlineto{\pgfqpoint{3.372093in}{2.761754in}}%
\pgfpathlineto{\pgfqpoint{3.116513in}{2.603812in}}%
\pgfpathclose%
\pgfusepath{fill}%
\end{pgfscope}%
\begin{pgfscope}%
\pgfpathrectangle{\pgfqpoint{0.017500in}{0.017500in}}{\pgfqpoint{6.218282in}{3.580372in}} %
\pgfusepath{clip}%
\pgfsetbuttcap%
\pgfsetroundjoin%
\definecolor{currentfill}{rgb}{0.758112,0.168122,0.188827}%
\pgfsetfillcolor{currentfill}%
\pgfsetlinewidth{0.000000pt}%
\definecolor{currentstroke}{rgb}{0.000000,0.000000,0.000000}%
\pgfsetstrokecolor{currentstroke}%
\pgfsetdash{}{0pt}%
\pgfpathmoveto{\pgfqpoint{3.116513in}{2.603812in}}%
\pgfpathlineto{\pgfqpoint{3.372093in}{2.761754in}}%
\pgfpathlineto{\pgfqpoint{3.116513in}{2.603812in}}%
\pgfpathclose%
\pgfusepath{fill}%
\end{pgfscope}%
\begin{pgfscope}%
\pgfpathrectangle{\pgfqpoint{0.017500in}{0.017500in}}{\pgfqpoint{6.218282in}{3.580372in}} %
\pgfusepath{clip}%
\pgfsetbuttcap%
\pgfsetroundjoin%
\definecolor{currentfill}{rgb}{0.758112,0.168122,0.188827}%
\pgfsetfillcolor{currentfill}%
\pgfsetlinewidth{0.000000pt}%
\definecolor{currentstroke}{rgb}{0.000000,0.000000,0.000000}%
\pgfsetstrokecolor{currentstroke}%
\pgfsetdash{}{0pt}%
\pgfpathmoveto{\pgfqpoint{3.116513in}{2.603812in}}%
\pgfpathlineto{\pgfqpoint{3.372093in}{2.761754in}}%
\pgfpathlineto{\pgfqpoint{3.116513in}{2.603812in}}%
\pgfpathclose%
\pgfusepath{fill}%
\end{pgfscope}%
\begin{pgfscope}%
\pgfpathrectangle{\pgfqpoint{0.017500in}{0.017500in}}{\pgfqpoint{6.218282in}{3.580372in}} %
\pgfusepath{clip}%
\pgfsetbuttcap%
\pgfsetroundjoin%
\definecolor{currentfill}{rgb}{0.758112,0.168122,0.188827}%
\pgfsetfillcolor{currentfill}%
\pgfsetlinewidth{0.000000pt}%
\definecolor{currentstroke}{rgb}{0.000000,0.000000,0.000000}%
\pgfsetstrokecolor{currentstroke}%
\pgfsetdash{}{0pt}%
\pgfpathmoveto{\pgfqpoint{3.116513in}{2.603812in}}%
\pgfpathlineto{\pgfqpoint{3.372093in}{2.761754in}}%
\pgfpathlineto{\pgfqpoint{3.116513in}{2.603812in}}%
\pgfpathclose%
\pgfusepath{fill}%
\end{pgfscope}%
\begin{pgfscope}%
\pgfpathrectangle{\pgfqpoint{0.017500in}{0.017500in}}{\pgfqpoint{6.218282in}{3.580372in}} %
\pgfusepath{clip}%
\pgfsetbuttcap%
\pgfsetroundjoin%
\definecolor{currentfill}{rgb}{0.758112,0.168122,0.188827}%
\pgfsetfillcolor{currentfill}%
\pgfsetlinewidth{0.000000pt}%
\definecolor{currentstroke}{rgb}{0.000000,0.000000,0.000000}%
\pgfsetstrokecolor{currentstroke}%
\pgfsetdash{}{0pt}%
\pgfpathmoveto{\pgfqpoint{3.116513in}{2.603812in}}%
\pgfpathlineto{\pgfqpoint{3.372093in}{2.761754in}}%
\pgfpathlineto{\pgfqpoint{3.116513in}{2.603812in}}%
\pgfpathclose%
\pgfusepath{fill}%
\end{pgfscope}%
\begin{pgfscope}%
\pgfpathrectangle{\pgfqpoint{0.017500in}{0.017500in}}{\pgfqpoint{6.218282in}{3.580372in}} %
\pgfusepath{clip}%
\pgfsetbuttcap%
\pgfsetroundjoin%
\definecolor{currentfill}{rgb}{0.758112,0.168122,0.188827}%
\pgfsetfillcolor{currentfill}%
\pgfsetlinewidth{0.000000pt}%
\definecolor{currentstroke}{rgb}{0.000000,0.000000,0.000000}%
\pgfsetstrokecolor{currentstroke}%
\pgfsetdash{}{0pt}%
\pgfpathmoveto{\pgfqpoint{3.116513in}{2.603812in}}%
\pgfpathlineto{\pgfqpoint{3.372093in}{2.761754in}}%
\pgfpathlineto{\pgfqpoint{3.116513in}{2.603812in}}%
\pgfpathclose%
\pgfusepath{fill}%
\end{pgfscope}%
\begin{pgfscope}%
\pgfpathrectangle{\pgfqpoint{0.017500in}{0.017500in}}{\pgfqpoint{6.218282in}{3.580372in}} %
\pgfusepath{clip}%
\pgfsetbuttcap%
\pgfsetroundjoin%
\definecolor{currentfill}{rgb}{0.758112,0.168122,0.188827}%
\pgfsetfillcolor{currentfill}%
\pgfsetlinewidth{0.000000pt}%
\definecolor{currentstroke}{rgb}{0.000000,0.000000,0.000000}%
\pgfsetstrokecolor{currentstroke}%
\pgfsetdash{}{0pt}%
\pgfpathmoveto{\pgfqpoint{3.116513in}{2.603812in}}%
\pgfpathlineto{\pgfqpoint{3.372093in}{2.761754in}}%
\pgfpathlineto{\pgfqpoint{3.116513in}{2.603812in}}%
\pgfpathclose%
\pgfusepath{fill}%
\end{pgfscope}%
\begin{pgfscope}%
\pgfpathrectangle{\pgfqpoint{0.017500in}{0.017500in}}{\pgfqpoint{6.218282in}{3.580372in}} %
\pgfusepath{clip}%
\pgfsetbuttcap%
\pgfsetroundjoin%
\definecolor{currentfill}{rgb}{0.758112,0.168122,0.188827}%
\pgfsetfillcolor{currentfill}%
\pgfsetlinewidth{0.000000pt}%
\definecolor{currentstroke}{rgb}{0.000000,0.000000,0.000000}%
\pgfsetstrokecolor{currentstroke}%
\pgfsetdash{}{0pt}%
\pgfpathmoveto{\pgfqpoint{3.116513in}{2.603812in}}%
\pgfpathlineto{\pgfqpoint{3.372093in}{2.761754in}}%
\pgfpathlineto{\pgfqpoint{3.116513in}{2.603812in}}%
\pgfpathclose%
\pgfusepath{fill}%
\end{pgfscope}%
\begin{pgfscope}%
\pgfpathrectangle{\pgfqpoint{0.017500in}{0.017500in}}{\pgfqpoint{6.218282in}{3.580372in}} %
\pgfusepath{clip}%
\pgfsetbuttcap%
\pgfsetroundjoin%
\definecolor{currentfill}{rgb}{0.758112,0.168122,0.188827}%
\pgfsetfillcolor{currentfill}%
\pgfsetlinewidth{0.000000pt}%
\definecolor{currentstroke}{rgb}{0.000000,0.000000,0.000000}%
\pgfsetstrokecolor{currentstroke}%
\pgfsetdash{}{0pt}%
\pgfpathmoveto{\pgfqpoint{3.116513in}{2.603812in}}%
\pgfpathlineto{\pgfqpoint{3.372093in}{2.761754in}}%
\pgfpathlineto{\pgfqpoint{3.116513in}{2.603812in}}%
\pgfpathclose%
\pgfusepath{fill}%
\end{pgfscope}%
\begin{pgfscope}%
\pgfpathrectangle{\pgfqpoint{0.017500in}{0.017500in}}{\pgfqpoint{6.218282in}{3.580372in}} %
\pgfusepath{clip}%
\pgfsetbuttcap%
\pgfsetroundjoin%
\definecolor{currentfill}{rgb}{0.758112,0.168122,0.188827}%
\pgfsetfillcolor{currentfill}%
\pgfsetlinewidth{0.000000pt}%
\definecolor{currentstroke}{rgb}{0.000000,0.000000,0.000000}%
\pgfsetstrokecolor{currentstroke}%
\pgfsetdash{}{0pt}%
\pgfpathmoveto{\pgfqpoint{3.116513in}{2.603812in}}%
\pgfpathlineto{\pgfqpoint{3.372093in}{2.761754in}}%
\pgfpathlineto{\pgfqpoint{3.116513in}{2.603812in}}%
\pgfpathclose%
\pgfusepath{fill}%
\end{pgfscope}%
\begin{pgfscope}%
\pgfpathrectangle{\pgfqpoint{0.017500in}{0.017500in}}{\pgfqpoint{6.218282in}{3.580372in}} %
\pgfusepath{clip}%
\pgfsetbuttcap%
\pgfsetroundjoin%
\definecolor{currentfill}{rgb}{0.758112,0.168122,0.188827}%
\pgfsetfillcolor{currentfill}%
\pgfsetlinewidth{0.000000pt}%
\definecolor{currentstroke}{rgb}{0.000000,0.000000,0.000000}%
\pgfsetstrokecolor{currentstroke}%
\pgfsetdash{}{0pt}%
\pgfpathmoveto{\pgfqpoint{3.116513in}{2.603812in}}%
\pgfpathlineto{\pgfqpoint{3.372093in}{2.761754in}}%
\pgfpathlineto{\pgfqpoint{3.116513in}{2.603812in}}%
\pgfpathclose%
\pgfusepath{fill}%
\end{pgfscope}%
\begin{pgfscope}%
\pgfpathrectangle{\pgfqpoint{0.017500in}{0.017500in}}{\pgfqpoint{6.218282in}{3.580372in}} %
\pgfusepath{clip}%
\pgfsetbuttcap%
\pgfsetroundjoin%
\definecolor{currentfill}{rgb}{0.758112,0.168122,0.188827}%
\pgfsetfillcolor{currentfill}%
\pgfsetlinewidth{0.000000pt}%
\definecolor{currentstroke}{rgb}{0.000000,0.000000,0.000000}%
\pgfsetstrokecolor{currentstroke}%
\pgfsetdash{}{0pt}%
\pgfpathmoveto{\pgfqpoint{3.116513in}{2.603812in}}%
\pgfpathlineto{\pgfqpoint{3.372093in}{2.761754in}}%
\pgfpathlineto{\pgfqpoint{3.116513in}{2.603812in}}%
\pgfpathclose%
\pgfusepath{fill}%
\end{pgfscope}%
\begin{pgfscope}%
\pgfpathrectangle{\pgfqpoint{0.017500in}{0.017500in}}{\pgfqpoint{6.218282in}{3.580372in}} %
\pgfusepath{clip}%
\pgfsetbuttcap%
\pgfsetroundjoin%
\definecolor{currentfill}{rgb}{0.758112,0.168122,0.188827}%
\pgfsetfillcolor{currentfill}%
\pgfsetlinewidth{0.000000pt}%
\definecolor{currentstroke}{rgb}{0.000000,0.000000,0.000000}%
\pgfsetstrokecolor{currentstroke}%
\pgfsetdash{}{0pt}%
\pgfpathmoveto{\pgfqpoint{3.116513in}{2.603812in}}%
\pgfpathlineto{\pgfqpoint{3.372093in}{2.761754in}}%
\pgfpathlineto{\pgfqpoint{3.116513in}{2.603812in}}%
\pgfpathclose%
\pgfusepath{fill}%
\end{pgfscope}%
\begin{pgfscope}%
\pgfpathrectangle{\pgfqpoint{0.017500in}{0.017500in}}{\pgfqpoint{6.218282in}{3.580372in}} %
\pgfusepath{clip}%
\pgfsetbuttcap%
\pgfsetroundjoin%
\definecolor{currentfill}{rgb}{0.758112,0.168122,0.188827}%
\pgfsetfillcolor{currentfill}%
\pgfsetlinewidth{0.000000pt}%
\definecolor{currentstroke}{rgb}{0.000000,0.000000,0.000000}%
\pgfsetstrokecolor{currentstroke}%
\pgfsetdash{}{0pt}%
\pgfpathmoveto{\pgfqpoint{3.116513in}{2.603812in}}%
\pgfpathlineto{\pgfqpoint{3.372093in}{2.761754in}}%
\pgfpathlineto{\pgfqpoint{3.116513in}{2.603812in}}%
\pgfpathclose%
\pgfusepath{fill}%
\end{pgfscope}%
\begin{pgfscope}%
\pgfpathrectangle{\pgfqpoint{0.017500in}{0.017500in}}{\pgfqpoint{6.218282in}{3.580372in}} %
\pgfusepath{clip}%
\pgfsetbuttcap%
\pgfsetroundjoin%
\definecolor{currentfill}{rgb}{0.758112,0.168122,0.188827}%
\pgfsetfillcolor{currentfill}%
\pgfsetlinewidth{0.000000pt}%
\definecolor{currentstroke}{rgb}{0.000000,0.000000,0.000000}%
\pgfsetstrokecolor{currentstroke}%
\pgfsetdash{}{0pt}%
\pgfpathmoveto{\pgfqpoint{3.116513in}{2.603812in}}%
\pgfpathlineto{\pgfqpoint{3.372093in}{2.761754in}}%
\pgfpathlineto{\pgfqpoint{3.116513in}{2.603812in}}%
\pgfpathclose%
\pgfusepath{fill}%
\end{pgfscope}%
\begin{pgfscope}%
\pgfpathrectangle{\pgfqpoint{0.017500in}{0.017500in}}{\pgfqpoint{6.218282in}{3.580372in}} %
\pgfusepath{clip}%
\pgfsetbuttcap%
\pgfsetroundjoin%
\definecolor{currentfill}{rgb}{0.758112,0.168122,0.188827}%
\pgfsetfillcolor{currentfill}%
\pgfsetlinewidth{0.000000pt}%
\definecolor{currentstroke}{rgb}{0.000000,0.000000,0.000000}%
\pgfsetstrokecolor{currentstroke}%
\pgfsetdash{}{0pt}%
\pgfpathmoveto{\pgfqpoint{3.116513in}{2.603812in}}%
\pgfpathlineto{\pgfqpoint{3.372093in}{2.761754in}}%
\pgfpathlineto{\pgfqpoint{3.116513in}{2.603812in}}%
\pgfpathclose%
\pgfusepath{fill}%
\end{pgfscope}%
\begin{pgfscope}%
\pgfpathrectangle{\pgfqpoint{0.017500in}{0.017500in}}{\pgfqpoint{6.218282in}{3.580372in}} %
\pgfusepath{clip}%
\pgfsetbuttcap%
\pgfsetroundjoin%
\definecolor{currentfill}{rgb}{0.758112,0.168122,0.188827}%
\pgfsetfillcolor{currentfill}%
\pgfsetlinewidth{0.000000pt}%
\definecolor{currentstroke}{rgb}{0.000000,0.000000,0.000000}%
\pgfsetstrokecolor{currentstroke}%
\pgfsetdash{}{0pt}%
\pgfpathmoveto{\pgfqpoint{3.116513in}{2.603812in}}%
\pgfpathlineto{\pgfqpoint{3.372093in}{2.761754in}}%
\pgfpathlineto{\pgfqpoint{3.116513in}{2.603812in}}%
\pgfpathclose%
\pgfusepath{fill}%
\end{pgfscope}%
\begin{pgfscope}%
\pgfpathrectangle{\pgfqpoint{0.017500in}{0.017500in}}{\pgfqpoint{6.218282in}{3.580372in}} %
\pgfusepath{clip}%
\pgfsetbuttcap%
\pgfsetroundjoin%
\definecolor{currentfill}{rgb}{0.758112,0.168122,0.188827}%
\pgfsetfillcolor{currentfill}%
\pgfsetlinewidth{0.000000pt}%
\definecolor{currentstroke}{rgb}{0.000000,0.000000,0.000000}%
\pgfsetstrokecolor{currentstroke}%
\pgfsetdash{}{0pt}%
\pgfpathmoveto{\pgfqpoint{3.116513in}{2.603812in}}%
\pgfpathlineto{\pgfqpoint{3.372093in}{2.761754in}}%
\pgfpathlineto{\pgfqpoint{3.116513in}{2.603812in}}%
\pgfpathclose%
\pgfusepath{fill}%
\end{pgfscope}%
\begin{pgfscope}%
\pgfpathrectangle{\pgfqpoint{0.017500in}{0.017500in}}{\pgfqpoint{6.218282in}{3.580372in}} %
\pgfusepath{clip}%
\pgfsetbuttcap%
\pgfsetroundjoin%
\definecolor{currentfill}{rgb}{0.758112,0.168122,0.188827}%
\pgfsetfillcolor{currentfill}%
\pgfsetlinewidth{0.000000pt}%
\definecolor{currentstroke}{rgb}{0.000000,0.000000,0.000000}%
\pgfsetstrokecolor{currentstroke}%
\pgfsetdash{}{0pt}%
\pgfpathmoveto{\pgfqpoint{3.116513in}{2.603812in}}%
\pgfpathlineto{\pgfqpoint{3.372093in}{2.761754in}}%
\pgfpathlineto{\pgfqpoint{3.116513in}{2.603812in}}%
\pgfpathclose%
\pgfusepath{fill}%
\end{pgfscope}%
\begin{pgfscope}%
\pgfpathrectangle{\pgfqpoint{0.017500in}{0.017500in}}{\pgfqpoint{6.218282in}{3.580372in}} %
\pgfusepath{clip}%
\pgfsetbuttcap%
\pgfsetroundjoin%
\definecolor{currentfill}{rgb}{0.758112,0.168122,0.188827}%
\pgfsetfillcolor{currentfill}%
\pgfsetlinewidth{0.000000pt}%
\definecolor{currentstroke}{rgb}{0.000000,0.000000,0.000000}%
\pgfsetstrokecolor{currentstroke}%
\pgfsetdash{}{0pt}%
\pgfpathmoveto{\pgfqpoint{3.116513in}{2.603812in}}%
\pgfpathlineto{\pgfqpoint{3.372093in}{2.761754in}}%
\pgfpathlineto{\pgfqpoint{3.116513in}{2.603812in}}%
\pgfpathclose%
\pgfusepath{fill}%
\end{pgfscope}%
\begin{pgfscope}%
\pgfpathrectangle{\pgfqpoint{0.017500in}{0.017500in}}{\pgfqpoint{6.218282in}{3.580372in}} %
\pgfusepath{clip}%
\pgfsetbuttcap%
\pgfsetroundjoin%
\definecolor{currentfill}{rgb}{0.902849,0.844796,0.811970}%
\pgfsetfillcolor{currentfill}%
\pgfsetlinewidth{0.000000pt}%
\definecolor{currentstroke}{rgb}{0.000000,0.000000,0.000000}%
\pgfsetstrokecolor{currentstroke}%
\pgfsetdash{}{0pt}%
\pgfpathmoveto{\pgfqpoint{4.511410in}{2.753953in}}%
\pgfpathlineto{\pgfqpoint{3.236099in}{0.723497in}}%
\pgfpathlineto{\pgfqpoint{4.511410in}{2.753953in}}%
\pgfpathclose%
\pgfusepath{fill}%
\end{pgfscope}%
\begin{pgfscope}%
\pgfpathrectangle{\pgfqpoint{0.017500in}{0.017500in}}{\pgfqpoint{6.218282in}{3.580372in}} %
\pgfusepath{clip}%
\pgfsetbuttcap%
\pgfsetroundjoin%
\definecolor{currentfill}{rgb}{0.902849,0.844796,0.811970}%
\pgfsetfillcolor{currentfill}%
\pgfsetlinewidth{0.000000pt}%
\definecolor{currentstroke}{rgb}{0.000000,0.000000,0.000000}%
\pgfsetstrokecolor{currentstroke}%
\pgfsetdash{}{0pt}%
\pgfpathmoveto{\pgfqpoint{4.511410in}{2.753953in}}%
\pgfpathlineto{\pgfqpoint{3.236099in}{0.723497in}}%
\pgfpathlineto{\pgfqpoint{4.511410in}{2.753953in}}%
\pgfpathclose%
\pgfusepath{fill}%
\end{pgfscope}%
\begin{pgfscope}%
\pgfpathrectangle{\pgfqpoint{0.017500in}{0.017500in}}{\pgfqpoint{6.218282in}{3.580372in}} %
\pgfusepath{clip}%
\pgfsetbuttcap%
\pgfsetroundjoin%
\definecolor{currentfill}{rgb}{0.902849,0.844796,0.811970}%
\pgfsetfillcolor{currentfill}%
\pgfsetlinewidth{0.000000pt}%
\definecolor{currentstroke}{rgb}{0.000000,0.000000,0.000000}%
\pgfsetstrokecolor{currentstroke}%
\pgfsetdash{}{0pt}%
\pgfpathmoveto{\pgfqpoint{4.511410in}{2.753953in}}%
\pgfpathlineto{\pgfqpoint{3.236099in}{0.723497in}}%
\pgfpathlineto{\pgfqpoint{4.511410in}{2.753953in}}%
\pgfpathclose%
\pgfusepath{fill}%
\end{pgfscope}%
\begin{pgfscope}%
\pgfpathrectangle{\pgfqpoint{0.017500in}{0.017500in}}{\pgfqpoint{6.218282in}{3.580372in}} %
\pgfusepath{clip}%
\pgfsetbuttcap%
\pgfsetroundjoin%
\definecolor{currentfill}{rgb}{0.902849,0.844796,0.811970}%
\pgfsetfillcolor{currentfill}%
\pgfsetlinewidth{0.000000pt}%
\definecolor{currentstroke}{rgb}{0.000000,0.000000,0.000000}%
\pgfsetstrokecolor{currentstroke}%
\pgfsetdash{}{0pt}%
\pgfpathmoveto{\pgfqpoint{4.511410in}{2.753953in}}%
\pgfpathlineto{\pgfqpoint{3.236099in}{0.723497in}}%
\pgfpathlineto{\pgfqpoint{4.511410in}{2.753953in}}%
\pgfpathclose%
\pgfusepath{fill}%
\end{pgfscope}%
\begin{pgfscope}%
\pgfpathrectangle{\pgfqpoint{0.017500in}{0.017500in}}{\pgfqpoint{6.218282in}{3.580372in}} %
\pgfusepath{clip}%
\pgfsetbuttcap%
\pgfsetroundjoin%
\definecolor{currentfill}{rgb}{0.902849,0.844796,0.811970}%
\pgfsetfillcolor{currentfill}%
\pgfsetlinewidth{0.000000pt}%
\definecolor{currentstroke}{rgb}{0.000000,0.000000,0.000000}%
\pgfsetstrokecolor{currentstroke}%
\pgfsetdash{}{0pt}%
\pgfpathmoveto{\pgfqpoint{4.511410in}{2.753953in}}%
\pgfpathlineto{\pgfqpoint{3.236099in}{0.723497in}}%
\pgfpathlineto{\pgfqpoint{4.511410in}{2.753953in}}%
\pgfpathclose%
\pgfusepath{fill}%
\end{pgfscope}%
\begin{pgfscope}%
\pgfpathrectangle{\pgfqpoint{0.017500in}{0.017500in}}{\pgfqpoint{6.218282in}{3.580372in}} %
\pgfusepath{clip}%
\pgfsetbuttcap%
\pgfsetroundjoin%
\definecolor{currentfill}{rgb}{0.902849,0.844796,0.811970}%
\pgfsetfillcolor{currentfill}%
\pgfsetlinewidth{0.000000pt}%
\definecolor{currentstroke}{rgb}{0.000000,0.000000,0.000000}%
\pgfsetstrokecolor{currentstroke}%
\pgfsetdash{}{0pt}%
\pgfpathmoveto{\pgfqpoint{4.511410in}{2.753953in}}%
\pgfpathlineto{\pgfqpoint{3.236099in}{0.723497in}}%
\pgfpathlineto{\pgfqpoint{4.511410in}{2.753953in}}%
\pgfpathclose%
\pgfusepath{fill}%
\end{pgfscope}%
\begin{pgfscope}%
\pgfpathrectangle{\pgfqpoint{0.017500in}{0.017500in}}{\pgfqpoint{6.218282in}{3.580372in}} %
\pgfusepath{clip}%
\pgfsetbuttcap%
\pgfsetroundjoin%
\definecolor{currentfill}{rgb}{0.902849,0.844796,0.811970}%
\pgfsetfillcolor{currentfill}%
\pgfsetlinewidth{0.000000pt}%
\definecolor{currentstroke}{rgb}{0.000000,0.000000,0.000000}%
\pgfsetstrokecolor{currentstroke}%
\pgfsetdash{}{0pt}%
\pgfpathmoveto{\pgfqpoint{4.511410in}{2.753953in}}%
\pgfpathlineto{\pgfqpoint{3.236099in}{0.723497in}}%
\pgfpathlineto{\pgfqpoint{4.511410in}{2.753953in}}%
\pgfpathclose%
\pgfusepath{fill}%
\end{pgfscope}%
\begin{pgfscope}%
\pgfpathrectangle{\pgfqpoint{0.017500in}{0.017500in}}{\pgfqpoint{6.218282in}{3.580372in}} %
\pgfusepath{clip}%
\pgfsetbuttcap%
\pgfsetroundjoin%
\definecolor{currentfill}{rgb}{0.902849,0.844796,0.811970}%
\pgfsetfillcolor{currentfill}%
\pgfsetlinewidth{0.000000pt}%
\definecolor{currentstroke}{rgb}{0.000000,0.000000,0.000000}%
\pgfsetstrokecolor{currentstroke}%
\pgfsetdash{}{0pt}%
\pgfpathmoveto{\pgfqpoint{4.511410in}{2.753953in}}%
\pgfpathlineto{\pgfqpoint{3.236099in}{0.723497in}}%
\pgfpathlineto{\pgfqpoint{4.511410in}{2.753953in}}%
\pgfpathclose%
\pgfusepath{fill}%
\end{pgfscope}%
\begin{pgfscope}%
\pgfpathrectangle{\pgfqpoint{0.017500in}{0.017500in}}{\pgfqpoint{6.218282in}{3.580372in}} %
\pgfusepath{clip}%
\pgfsetbuttcap%
\pgfsetroundjoin%
\definecolor{currentfill}{rgb}{0.902849,0.844796,0.811970}%
\pgfsetfillcolor{currentfill}%
\pgfsetlinewidth{0.000000pt}%
\definecolor{currentstroke}{rgb}{0.000000,0.000000,0.000000}%
\pgfsetstrokecolor{currentstroke}%
\pgfsetdash{}{0pt}%
\pgfpathmoveto{\pgfqpoint{4.511410in}{2.753953in}}%
\pgfpathlineto{\pgfqpoint{3.236099in}{0.723497in}}%
\pgfpathlineto{\pgfqpoint{4.511410in}{2.753953in}}%
\pgfpathclose%
\pgfusepath{fill}%
\end{pgfscope}%
\begin{pgfscope}%
\pgfpathrectangle{\pgfqpoint{0.017500in}{0.017500in}}{\pgfqpoint{6.218282in}{3.580372in}} %
\pgfusepath{clip}%
\pgfsetbuttcap%
\pgfsetroundjoin%
\definecolor{currentfill}{rgb}{0.902849,0.844796,0.811970}%
\pgfsetfillcolor{currentfill}%
\pgfsetlinewidth{0.000000pt}%
\definecolor{currentstroke}{rgb}{0.000000,0.000000,0.000000}%
\pgfsetstrokecolor{currentstroke}%
\pgfsetdash{}{0pt}%
\pgfpathmoveto{\pgfqpoint{4.511410in}{2.753953in}}%
\pgfpathlineto{\pgfqpoint{3.236099in}{0.723497in}}%
\pgfpathlineto{\pgfqpoint{4.511410in}{2.753953in}}%
\pgfpathclose%
\pgfusepath{fill}%
\end{pgfscope}%
\begin{pgfscope}%
\pgfpathrectangle{\pgfqpoint{0.017500in}{0.017500in}}{\pgfqpoint{6.218282in}{3.580372in}} %
\pgfusepath{clip}%
\pgfsetbuttcap%
\pgfsetroundjoin%
\definecolor{currentfill}{rgb}{0.902849,0.844796,0.811970}%
\pgfsetfillcolor{currentfill}%
\pgfsetlinewidth{0.000000pt}%
\definecolor{currentstroke}{rgb}{0.000000,0.000000,0.000000}%
\pgfsetstrokecolor{currentstroke}%
\pgfsetdash{}{0pt}%
\pgfpathmoveto{\pgfqpoint{4.511410in}{2.753953in}}%
\pgfpathlineto{\pgfqpoint{3.236099in}{0.723497in}}%
\pgfpathlineto{\pgfqpoint{4.511410in}{2.753953in}}%
\pgfpathclose%
\pgfusepath{fill}%
\end{pgfscope}%
\begin{pgfscope}%
\pgfpathrectangle{\pgfqpoint{0.017500in}{0.017500in}}{\pgfqpoint{6.218282in}{3.580372in}} %
\pgfusepath{clip}%
\pgfsetbuttcap%
\pgfsetroundjoin%
\definecolor{currentfill}{rgb}{0.902849,0.844796,0.811970}%
\pgfsetfillcolor{currentfill}%
\pgfsetlinewidth{0.000000pt}%
\definecolor{currentstroke}{rgb}{0.000000,0.000000,0.000000}%
\pgfsetstrokecolor{currentstroke}%
\pgfsetdash{}{0pt}%
\pgfpathmoveto{\pgfqpoint{4.511410in}{2.753953in}}%
\pgfpathlineto{\pgfqpoint{3.236099in}{0.723497in}}%
\pgfpathlineto{\pgfqpoint{4.511410in}{2.753953in}}%
\pgfpathclose%
\pgfusepath{fill}%
\end{pgfscope}%
\begin{pgfscope}%
\pgfpathrectangle{\pgfqpoint{0.017500in}{0.017500in}}{\pgfqpoint{6.218282in}{3.580372in}} %
\pgfusepath{clip}%
\pgfsetbuttcap%
\pgfsetroundjoin%
\definecolor{currentfill}{rgb}{0.902849,0.844796,0.811970}%
\pgfsetfillcolor{currentfill}%
\pgfsetlinewidth{0.000000pt}%
\definecolor{currentstroke}{rgb}{0.000000,0.000000,0.000000}%
\pgfsetstrokecolor{currentstroke}%
\pgfsetdash{}{0pt}%
\pgfpathmoveto{\pgfqpoint{4.511410in}{2.753953in}}%
\pgfpathlineto{\pgfqpoint{3.236099in}{0.723497in}}%
\pgfpathlineto{\pgfqpoint{4.511410in}{2.753953in}}%
\pgfpathclose%
\pgfusepath{fill}%
\end{pgfscope}%
\begin{pgfscope}%
\pgfpathrectangle{\pgfqpoint{0.017500in}{0.017500in}}{\pgfqpoint{6.218282in}{3.580372in}} %
\pgfusepath{clip}%
\pgfsetbuttcap%
\pgfsetroundjoin%
\definecolor{currentfill}{rgb}{0.902849,0.844796,0.811970}%
\pgfsetfillcolor{currentfill}%
\pgfsetlinewidth{0.000000pt}%
\definecolor{currentstroke}{rgb}{0.000000,0.000000,0.000000}%
\pgfsetstrokecolor{currentstroke}%
\pgfsetdash{}{0pt}%
\pgfpathmoveto{\pgfqpoint{4.511410in}{2.753953in}}%
\pgfpathlineto{\pgfqpoint{3.236099in}{0.723497in}}%
\pgfpathlineto{\pgfqpoint{4.511410in}{2.753953in}}%
\pgfpathclose%
\pgfusepath{fill}%
\end{pgfscope}%
\begin{pgfscope}%
\pgfpathrectangle{\pgfqpoint{0.017500in}{0.017500in}}{\pgfqpoint{6.218282in}{3.580372in}} %
\pgfusepath{clip}%
\pgfsetbuttcap%
\pgfsetroundjoin%
\definecolor{currentfill}{rgb}{0.902849,0.844796,0.811970}%
\pgfsetfillcolor{currentfill}%
\pgfsetlinewidth{0.000000pt}%
\definecolor{currentstroke}{rgb}{0.000000,0.000000,0.000000}%
\pgfsetstrokecolor{currentstroke}%
\pgfsetdash{}{0pt}%
\pgfpathmoveto{\pgfqpoint{4.511410in}{2.753953in}}%
\pgfpathlineto{\pgfqpoint{3.236099in}{0.723497in}}%
\pgfpathlineto{\pgfqpoint{4.511410in}{2.753953in}}%
\pgfpathclose%
\pgfusepath{fill}%
\end{pgfscope}%
\begin{pgfscope}%
\pgfpathrectangle{\pgfqpoint{0.017500in}{0.017500in}}{\pgfqpoint{6.218282in}{3.580372in}} %
\pgfusepath{clip}%
\pgfsetbuttcap%
\pgfsetroundjoin%
\definecolor{currentfill}{rgb}{0.902849,0.844796,0.811970}%
\pgfsetfillcolor{currentfill}%
\pgfsetlinewidth{0.000000pt}%
\definecolor{currentstroke}{rgb}{0.000000,0.000000,0.000000}%
\pgfsetstrokecolor{currentstroke}%
\pgfsetdash{}{0pt}%
\pgfpathmoveto{\pgfqpoint{4.511410in}{2.753953in}}%
\pgfpathlineto{\pgfqpoint{3.236099in}{0.723497in}}%
\pgfpathlineto{\pgfqpoint{4.511410in}{2.753953in}}%
\pgfpathclose%
\pgfusepath{fill}%
\end{pgfscope}%
\begin{pgfscope}%
\pgfpathrectangle{\pgfqpoint{0.017500in}{0.017500in}}{\pgfqpoint{6.218282in}{3.580372in}} %
\pgfusepath{clip}%
\pgfsetbuttcap%
\pgfsetroundjoin%
\definecolor{currentfill}{rgb}{0.902849,0.844796,0.811970}%
\pgfsetfillcolor{currentfill}%
\pgfsetlinewidth{0.000000pt}%
\definecolor{currentstroke}{rgb}{0.000000,0.000000,0.000000}%
\pgfsetstrokecolor{currentstroke}%
\pgfsetdash{}{0pt}%
\pgfpathmoveto{\pgfqpoint{4.511410in}{2.753953in}}%
\pgfpathlineto{\pgfqpoint{3.236099in}{0.723497in}}%
\pgfpathlineto{\pgfqpoint{4.511410in}{2.753953in}}%
\pgfpathclose%
\pgfusepath{fill}%
\end{pgfscope}%
\begin{pgfscope}%
\pgfpathrectangle{\pgfqpoint{0.017500in}{0.017500in}}{\pgfqpoint{6.218282in}{3.580372in}} %
\pgfusepath{clip}%
\pgfsetbuttcap%
\pgfsetroundjoin%
\definecolor{currentfill}{rgb}{0.902849,0.844796,0.811970}%
\pgfsetfillcolor{currentfill}%
\pgfsetlinewidth{0.000000pt}%
\definecolor{currentstroke}{rgb}{0.000000,0.000000,0.000000}%
\pgfsetstrokecolor{currentstroke}%
\pgfsetdash{}{0pt}%
\pgfpathmoveto{\pgfqpoint{4.511410in}{2.753953in}}%
\pgfpathlineto{\pgfqpoint{3.236099in}{0.723497in}}%
\pgfpathlineto{\pgfqpoint{4.511410in}{2.753953in}}%
\pgfpathclose%
\pgfusepath{fill}%
\end{pgfscope}%
\begin{pgfscope}%
\pgfpathrectangle{\pgfqpoint{0.017500in}{0.017500in}}{\pgfqpoint{6.218282in}{3.580372in}} %
\pgfusepath{clip}%
\pgfsetbuttcap%
\pgfsetroundjoin%
\definecolor{currentfill}{rgb}{0.902849,0.844796,0.811970}%
\pgfsetfillcolor{currentfill}%
\pgfsetlinewidth{0.000000pt}%
\definecolor{currentstroke}{rgb}{0.000000,0.000000,0.000000}%
\pgfsetstrokecolor{currentstroke}%
\pgfsetdash{}{0pt}%
\pgfpathmoveto{\pgfqpoint{4.511410in}{2.753953in}}%
\pgfpathlineto{\pgfqpoint{3.236099in}{0.723497in}}%
\pgfpathlineto{\pgfqpoint{4.511410in}{2.753953in}}%
\pgfpathclose%
\pgfusepath{fill}%
\end{pgfscope}%
\begin{pgfscope}%
\pgfpathrectangle{\pgfqpoint{0.017500in}{0.017500in}}{\pgfqpoint{6.218282in}{3.580372in}} %
\pgfusepath{clip}%
\pgfsetbuttcap%
\pgfsetroundjoin%
\definecolor{currentfill}{rgb}{0.902849,0.844796,0.811970}%
\pgfsetfillcolor{currentfill}%
\pgfsetlinewidth{0.000000pt}%
\definecolor{currentstroke}{rgb}{0.000000,0.000000,0.000000}%
\pgfsetstrokecolor{currentstroke}%
\pgfsetdash{}{0pt}%
\pgfpathmoveto{\pgfqpoint{4.511410in}{2.753953in}}%
\pgfpathlineto{\pgfqpoint{3.236099in}{0.723497in}}%
\pgfpathlineto{\pgfqpoint{4.511410in}{2.753953in}}%
\pgfpathclose%
\pgfusepath{fill}%
\end{pgfscope}%
\begin{pgfscope}%
\pgfpathrectangle{\pgfqpoint{0.017500in}{0.017500in}}{\pgfqpoint{6.218282in}{3.580372in}} %
\pgfusepath{clip}%
\pgfsetbuttcap%
\pgfsetroundjoin%
\definecolor{currentfill}{rgb}{0.902849,0.844796,0.811970}%
\pgfsetfillcolor{currentfill}%
\pgfsetlinewidth{0.000000pt}%
\definecolor{currentstroke}{rgb}{0.000000,0.000000,0.000000}%
\pgfsetstrokecolor{currentstroke}%
\pgfsetdash{}{0pt}%
\pgfpathmoveto{\pgfqpoint{4.511410in}{2.753953in}}%
\pgfpathlineto{\pgfqpoint{3.236099in}{0.723497in}}%
\pgfpathlineto{\pgfqpoint{4.511410in}{2.753953in}}%
\pgfpathclose%
\pgfusepath{fill}%
\end{pgfscope}%
\begin{pgfscope}%
\pgfpathrectangle{\pgfqpoint{0.017500in}{0.017500in}}{\pgfqpoint{6.218282in}{3.580372in}} %
\pgfusepath{clip}%
\pgfsetbuttcap%
\pgfsetroundjoin%
\definecolor{currentfill}{rgb}{0.902849,0.844796,0.811970}%
\pgfsetfillcolor{currentfill}%
\pgfsetlinewidth{0.000000pt}%
\definecolor{currentstroke}{rgb}{0.000000,0.000000,0.000000}%
\pgfsetstrokecolor{currentstroke}%
\pgfsetdash{}{0pt}%
\pgfpathmoveto{\pgfqpoint{4.511410in}{2.753953in}}%
\pgfpathlineto{\pgfqpoint{3.236099in}{0.723497in}}%
\pgfpathlineto{\pgfqpoint{4.511410in}{2.753953in}}%
\pgfpathclose%
\pgfusepath{fill}%
\end{pgfscope}%
\begin{pgfscope}%
\pgfpathrectangle{\pgfqpoint{0.017500in}{0.017500in}}{\pgfqpoint{6.218282in}{3.580372in}} %
\pgfusepath{clip}%
\pgfsetbuttcap%
\pgfsetroundjoin%
\definecolor{currentfill}{rgb}{0.902849,0.844796,0.811970}%
\pgfsetfillcolor{currentfill}%
\pgfsetlinewidth{0.000000pt}%
\definecolor{currentstroke}{rgb}{0.000000,0.000000,0.000000}%
\pgfsetstrokecolor{currentstroke}%
\pgfsetdash{}{0pt}%
\pgfpathmoveto{\pgfqpoint{4.511410in}{2.753953in}}%
\pgfpathlineto{\pgfqpoint{3.236099in}{0.723497in}}%
\pgfpathlineto{\pgfqpoint{4.511410in}{2.753953in}}%
\pgfpathclose%
\pgfusepath{fill}%
\end{pgfscope}%
\begin{pgfscope}%
\pgfpathrectangle{\pgfqpoint{0.017500in}{0.017500in}}{\pgfqpoint{6.218282in}{3.580372in}} %
\pgfusepath{clip}%
\pgfsetbuttcap%
\pgfsetroundjoin%
\definecolor{currentfill}{rgb}{0.902849,0.844796,0.811970}%
\pgfsetfillcolor{currentfill}%
\pgfsetlinewidth{0.000000pt}%
\definecolor{currentstroke}{rgb}{0.000000,0.000000,0.000000}%
\pgfsetstrokecolor{currentstroke}%
\pgfsetdash{}{0pt}%
\pgfpathmoveto{\pgfqpoint{4.511410in}{2.753953in}}%
\pgfpathlineto{\pgfqpoint{3.236099in}{0.723497in}}%
\pgfpathlineto{\pgfqpoint{4.511410in}{2.753953in}}%
\pgfpathclose%
\pgfusepath{fill}%
\end{pgfscope}%
\begin{pgfscope}%
\pgfpathrectangle{\pgfqpoint{0.017500in}{0.017500in}}{\pgfqpoint{6.218282in}{3.580372in}} %
\pgfusepath{clip}%
\pgfsetbuttcap%
\pgfsetroundjoin%
\definecolor{currentfill}{rgb}{0.902849,0.844796,0.811970}%
\pgfsetfillcolor{currentfill}%
\pgfsetlinewidth{0.000000pt}%
\definecolor{currentstroke}{rgb}{0.000000,0.000000,0.000000}%
\pgfsetstrokecolor{currentstroke}%
\pgfsetdash{}{0pt}%
\pgfpathmoveto{\pgfqpoint{4.511410in}{2.753953in}}%
\pgfpathlineto{\pgfqpoint{3.236099in}{0.723497in}}%
\pgfpathlineto{\pgfqpoint{4.511410in}{2.753953in}}%
\pgfpathclose%
\pgfusepath{fill}%
\end{pgfscope}%
\begin{pgfscope}%
\pgfpathrectangle{\pgfqpoint{0.017500in}{0.017500in}}{\pgfqpoint{6.218282in}{3.580372in}} %
\pgfusepath{clip}%
\pgfsetbuttcap%
\pgfsetroundjoin%
\definecolor{currentfill}{rgb}{0.758112,0.168122,0.188827}%
\pgfsetfillcolor{currentfill}%
\pgfsetlinewidth{0.000000pt}%
\definecolor{currentstroke}{rgb}{0.000000,0.000000,0.000000}%
\pgfsetstrokecolor{currentstroke}%
\pgfsetdash{}{0pt}%
\pgfpathmoveto{\pgfqpoint{3.116513in}{2.603812in}}%
\pgfpathlineto{\pgfqpoint{3.564492in}{2.541257in}}%
\pgfpathlineto{\pgfqpoint{3.816789in}{2.677291in}}%
\pgfpathlineto{\pgfqpoint{3.372093in}{2.761754in}}%
\pgfpathlineto{\pgfqpoint{3.116513in}{2.603812in}}%
\pgfpathclose%
\pgfusepath{fill}%
\end{pgfscope}%
\begin{pgfscope}%
\pgfpathrectangle{\pgfqpoint{0.017500in}{0.017500in}}{\pgfqpoint{6.218282in}{3.580372in}} %
\pgfusepath{clip}%
\pgfsetbuttcap%
\pgfsetroundjoin%
\definecolor{currentfill}{rgb}{0.758112,0.168122,0.188827}%
\pgfsetfillcolor{currentfill}%
\pgfsetlinewidth{0.000000pt}%
\definecolor{currentstroke}{rgb}{0.000000,0.000000,0.000000}%
\pgfsetstrokecolor{currentstroke}%
\pgfsetdash{}{0pt}%
\pgfpathmoveto{\pgfqpoint{3.116513in}{2.603812in}}%
\pgfpathlineto{\pgfqpoint{3.564492in}{2.541257in}}%
\pgfpathlineto{\pgfqpoint{3.816789in}{2.677291in}}%
\pgfpathlineto{\pgfqpoint{3.372093in}{2.761754in}}%
\pgfpathlineto{\pgfqpoint{3.116513in}{2.603812in}}%
\pgfpathclose%
\pgfusepath{fill}%
\end{pgfscope}%
\begin{pgfscope}%
\pgfpathrectangle{\pgfqpoint{0.017500in}{0.017500in}}{\pgfqpoint{6.218282in}{3.580372in}} %
\pgfusepath{clip}%
\pgfsetbuttcap%
\pgfsetroundjoin%
\definecolor{currentfill}{rgb}{0.758112,0.168122,0.188827}%
\pgfsetfillcolor{currentfill}%
\pgfsetlinewidth{0.000000pt}%
\definecolor{currentstroke}{rgb}{0.000000,0.000000,0.000000}%
\pgfsetstrokecolor{currentstroke}%
\pgfsetdash{}{0pt}%
\pgfpathmoveto{\pgfqpoint{3.116513in}{2.603812in}}%
\pgfpathlineto{\pgfqpoint{3.564492in}{2.541257in}}%
\pgfpathlineto{\pgfqpoint{3.816789in}{2.677291in}}%
\pgfpathlineto{\pgfqpoint{3.372093in}{2.761754in}}%
\pgfpathlineto{\pgfqpoint{3.116513in}{2.603812in}}%
\pgfpathclose%
\pgfusepath{fill}%
\end{pgfscope}%
\begin{pgfscope}%
\pgfpathrectangle{\pgfqpoint{0.017500in}{0.017500in}}{\pgfqpoint{6.218282in}{3.580372in}} %
\pgfusepath{clip}%
\pgfsetbuttcap%
\pgfsetroundjoin%
\definecolor{currentfill}{rgb}{0.758112,0.168122,0.188827}%
\pgfsetfillcolor{currentfill}%
\pgfsetlinewidth{0.000000pt}%
\definecolor{currentstroke}{rgb}{0.000000,0.000000,0.000000}%
\pgfsetstrokecolor{currentstroke}%
\pgfsetdash{}{0pt}%
\pgfpathmoveto{\pgfqpoint{3.116513in}{2.603812in}}%
\pgfpathlineto{\pgfqpoint{3.564492in}{2.541257in}}%
\pgfpathlineto{\pgfqpoint{3.816789in}{2.677291in}}%
\pgfpathlineto{\pgfqpoint{3.372093in}{2.761754in}}%
\pgfpathlineto{\pgfqpoint{3.116513in}{2.603812in}}%
\pgfpathclose%
\pgfusepath{fill}%
\end{pgfscope}%
\begin{pgfscope}%
\pgfpathrectangle{\pgfqpoint{0.017500in}{0.017500in}}{\pgfqpoint{6.218282in}{3.580372in}} %
\pgfusepath{clip}%
\pgfsetbuttcap%
\pgfsetroundjoin%
\definecolor{currentfill}{rgb}{0.758112,0.168122,0.188827}%
\pgfsetfillcolor{currentfill}%
\pgfsetlinewidth{0.000000pt}%
\definecolor{currentstroke}{rgb}{0.000000,0.000000,0.000000}%
\pgfsetstrokecolor{currentstroke}%
\pgfsetdash{}{0pt}%
\pgfpathmoveto{\pgfqpoint{3.116513in}{2.603812in}}%
\pgfpathlineto{\pgfqpoint{3.564492in}{2.541257in}}%
\pgfpathlineto{\pgfqpoint{3.816789in}{2.677291in}}%
\pgfpathlineto{\pgfqpoint{3.372093in}{2.761754in}}%
\pgfpathlineto{\pgfqpoint{3.116513in}{2.603812in}}%
\pgfpathclose%
\pgfusepath{fill}%
\end{pgfscope}%
\begin{pgfscope}%
\pgfpathrectangle{\pgfqpoint{0.017500in}{0.017500in}}{\pgfqpoint{6.218282in}{3.580372in}} %
\pgfusepath{clip}%
\pgfsetbuttcap%
\pgfsetroundjoin%
\definecolor{currentfill}{rgb}{0.758112,0.168122,0.188827}%
\pgfsetfillcolor{currentfill}%
\pgfsetlinewidth{0.000000pt}%
\definecolor{currentstroke}{rgb}{0.000000,0.000000,0.000000}%
\pgfsetstrokecolor{currentstroke}%
\pgfsetdash{}{0pt}%
\pgfpathmoveto{\pgfqpoint{3.116513in}{2.603812in}}%
\pgfpathlineto{\pgfqpoint{3.564492in}{2.541257in}}%
\pgfpathlineto{\pgfqpoint{3.816789in}{2.677291in}}%
\pgfpathlineto{\pgfqpoint{3.372093in}{2.761754in}}%
\pgfpathlineto{\pgfqpoint{3.116513in}{2.603812in}}%
\pgfpathclose%
\pgfusepath{fill}%
\end{pgfscope}%
\begin{pgfscope}%
\pgfpathrectangle{\pgfqpoint{0.017500in}{0.017500in}}{\pgfqpoint{6.218282in}{3.580372in}} %
\pgfusepath{clip}%
\pgfsetbuttcap%
\pgfsetroundjoin%
\definecolor{currentfill}{rgb}{0.729196,0.086679,0.167240}%
\pgfsetfillcolor{currentfill}%
\pgfsetlinewidth{0.000000pt}%
\definecolor{currentstroke}{rgb}{0.000000,0.000000,0.000000}%
\pgfsetstrokecolor{currentstroke}%
\pgfsetdash{}{0pt}%
\pgfpathmoveto{\pgfqpoint{4.267628in}{2.613520in}}%
\pgfpathlineto{\pgfqpoint{4.511410in}{2.753953in}}%
\pgfpathlineto{\pgfqpoint{4.267628in}{2.613520in}}%
\pgfpathclose%
\pgfusepath{fill}%
\end{pgfscope}%
\begin{pgfscope}%
\pgfpathrectangle{\pgfqpoint{0.017500in}{0.017500in}}{\pgfqpoint{6.218282in}{3.580372in}} %
\pgfusepath{clip}%
\pgfsetbuttcap%
\pgfsetroundjoin%
\definecolor{currentfill}{rgb}{0.729196,0.086679,0.167240}%
\pgfsetfillcolor{currentfill}%
\pgfsetlinewidth{0.000000pt}%
\definecolor{currentstroke}{rgb}{0.000000,0.000000,0.000000}%
\pgfsetstrokecolor{currentstroke}%
\pgfsetdash{}{0pt}%
\pgfpathmoveto{\pgfqpoint{4.267628in}{2.613520in}}%
\pgfpathlineto{\pgfqpoint{4.511410in}{2.753953in}}%
\pgfpathlineto{\pgfqpoint{4.267628in}{2.613520in}}%
\pgfpathclose%
\pgfusepath{fill}%
\end{pgfscope}%
\begin{pgfscope}%
\pgfpathrectangle{\pgfqpoint{0.017500in}{0.017500in}}{\pgfqpoint{6.218282in}{3.580372in}} %
\pgfusepath{clip}%
\pgfsetbuttcap%
\pgfsetroundjoin%
\definecolor{currentfill}{rgb}{0.729196,0.086679,0.167240}%
\pgfsetfillcolor{currentfill}%
\pgfsetlinewidth{0.000000pt}%
\definecolor{currentstroke}{rgb}{0.000000,0.000000,0.000000}%
\pgfsetstrokecolor{currentstroke}%
\pgfsetdash{}{0pt}%
\pgfpathmoveto{\pgfqpoint{4.267628in}{2.613520in}}%
\pgfpathlineto{\pgfqpoint{4.511410in}{2.753953in}}%
\pgfpathlineto{\pgfqpoint{4.267628in}{2.613520in}}%
\pgfpathclose%
\pgfusepath{fill}%
\end{pgfscope}%
\begin{pgfscope}%
\pgfpathrectangle{\pgfqpoint{0.017500in}{0.017500in}}{\pgfqpoint{6.218282in}{3.580372in}} %
\pgfusepath{clip}%
\pgfsetbuttcap%
\pgfsetroundjoin%
\definecolor{currentfill}{rgb}{0.729196,0.086679,0.167240}%
\pgfsetfillcolor{currentfill}%
\pgfsetlinewidth{0.000000pt}%
\definecolor{currentstroke}{rgb}{0.000000,0.000000,0.000000}%
\pgfsetstrokecolor{currentstroke}%
\pgfsetdash{}{0pt}%
\pgfpathmoveto{\pgfqpoint{4.267628in}{2.613520in}}%
\pgfpathlineto{\pgfqpoint{4.511410in}{2.753953in}}%
\pgfpathlineto{\pgfqpoint{4.267628in}{2.613520in}}%
\pgfpathclose%
\pgfusepath{fill}%
\end{pgfscope}%
\begin{pgfscope}%
\pgfpathrectangle{\pgfqpoint{0.017500in}{0.017500in}}{\pgfqpoint{6.218282in}{3.580372in}} %
\pgfusepath{clip}%
\pgfsetbuttcap%
\pgfsetroundjoin%
\definecolor{currentfill}{rgb}{0.729196,0.086679,0.167240}%
\pgfsetfillcolor{currentfill}%
\pgfsetlinewidth{0.000000pt}%
\definecolor{currentstroke}{rgb}{0.000000,0.000000,0.000000}%
\pgfsetstrokecolor{currentstroke}%
\pgfsetdash{}{0pt}%
\pgfpathmoveto{\pgfqpoint{4.267628in}{2.613520in}}%
\pgfpathlineto{\pgfqpoint{4.511410in}{2.753953in}}%
\pgfpathlineto{\pgfqpoint{4.267628in}{2.613520in}}%
\pgfpathclose%
\pgfusepath{fill}%
\end{pgfscope}%
\begin{pgfscope}%
\pgfpathrectangle{\pgfqpoint{0.017500in}{0.017500in}}{\pgfqpoint{6.218282in}{3.580372in}} %
\pgfusepath{clip}%
\pgfsetbuttcap%
\pgfsetroundjoin%
\definecolor{currentfill}{rgb}{0.729196,0.086679,0.167240}%
\pgfsetfillcolor{currentfill}%
\pgfsetlinewidth{0.000000pt}%
\definecolor{currentstroke}{rgb}{0.000000,0.000000,0.000000}%
\pgfsetstrokecolor{currentstroke}%
\pgfsetdash{}{0pt}%
\pgfpathmoveto{\pgfqpoint{4.267628in}{2.613520in}}%
\pgfpathlineto{\pgfqpoint{4.511410in}{2.753953in}}%
\pgfpathlineto{\pgfqpoint{4.267628in}{2.613520in}}%
\pgfpathclose%
\pgfusepath{fill}%
\end{pgfscope}%
\begin{pgfscope}%
\pgfpathrectangle{\pgfqpoint{0.017500in}{0.017500in}}{\pgfqpoint{6.218282in}{3.580372in}} %
\pgfusepath{clip}%
\pgfsetbuttcap%
\pgfsetroundjoin%
\definecolor{currentfill}{rgb}{0.729196,0.086679,0.167240}%
\pgfsetfillcolor{currentfill}%
\pgfsetlinewidth{0.000000pt}%
\definecolor{currentstroke}{rgb}{0.000000,0.000000,0.000000}%
\pgfsetstrokecolor{currentstroke}%
\pgfsetdash{}{0pt}%
\pgfpathmoveto{\pgfqpoint{4.267628in}{2.613520in}}%
\pgfpathlineto{\pgfqpoint{4.511410in}{2.753953in}}%
\pgfpathlineto{\pgfqpoint{4.267628in}{2.613520in}}%
\pgfpathclose%
\pgfusepath{fill}%
\end{pgfscope}%
\begin{pgfscope}%
\pgfpathrectangle{\pgfqpoint{0.017500in}{0.017500in}}{\pgfqpoint{6.218282in}{3.580372in}} %
\pgfusepath{clip}%
\pgfsetbuttcap%
\pgfsetroundjoin%
\definecolor{currentfill}{rgb}{0.729196,0.086679,0.167240}%
\pgfsetfillcolor{currentfill}%
\pgfsetlinewidth{0.000000pt}%
\definecolor{currentstroke}{rgb}{0.000000,0.000000,0.000000}%
\pgfsetstrokecolor{currentstroke}%
\pgfsetdash{}{0pt}%
\pgfpathmoveto{\pgfqpoint{4.267628in}{2.613520in}}%
\pgfpathlineto{\pgfqpoint{4.511410in}{2.753953in}}%
\pgfpathlineto{\pgfqpoint{4.267628in}{2.613520in}}%
\pgfpathclose%
\pgfusepath{fill}%
\end{pgfscope}%
\begin{pgfscope}%
\pgfpathrectangle{\pgfqpoint{0.017500in}{0.017500in}}{\pgfqpoint{6.218282in}{3.580372in}} %
\pgfusepath{clip}%
\pgfsetbuttcap%
\pgfsetroundjoin%
\definecolor{currentfill}{rgb}{0.729196,0.086679,0.167240}%
\pgfsetfillcolor{currentfill}%
\pgfsetlinewidth{0.000000pt}%
\definecolor{currentstroke}{rgb}{0.000000,0.000000,0.000000}%
\pgfsetstrokecolor{currentstroke}%
\pgfsetdash{}{0pt}%
\pgfpathmoveto{\pgfqpoint{4.267628in}{2.613520in}}%
\pgfpathlineto{\pgfqpoint{4.511410in}{2.753953in}}%
\pgfpathlineto{\pgfqpoint{4.267628in}{2.613520in}}%
\pgfpathclose%
\pgfusepath{fill}%
\end{pgfscope}%
\begin{pgfscope}%
\pgfpathrectangle{\pgfqpoint{0.017500in}{0.017500in}}{\pgfqpoint{6.218282in}{3.580372in}} %
\pgfusepath{clip}%
\pgfsetbuttcap%
\pgfsetroundjoin%
\definecolor{currentfill}{rgb}{0.729196,0.086679,0.167240}%
\pgfsetfillcolor{currentfill}%
\pgfsetlinewidth{0.000000pt}%
\definecolor{currentstroke}{rgb}{0.000000,0.000000,0.000000}%
\pgfsetstrokecolor{currentstroke}%
\pgfsetdash{}{0pt}%
\pgfpathmoveto{\pgfqpoint{4.267628in}{2.613520in}}%
\pgfpathlineto{\pgfqpoint{4.511410in}{2.753953in}}%
\pgfpathlineto{\pgfqpoint{4.267628in}{2.613520in}}%
\pgfpathclose%
\pgfusepath{fill}%
\end{pgfscope}%
\begin{pgfscope}%
\pgfpathrectangle{\pgfqpoint{0.017500in}{0.017500in}}{\pgfqpoint{6.218282in}{3.580372in}} %
\pgfusepath{clip}%
\pgfsetbuttcap%
\pgfsetroundjoin%
\definecolor{currentfill}{rgb}{0.729196,0.086679,0.167240}%
\pgfsetfillcolor{currentfill}%
\pgfsetlinewidth{0.000000pt}%
\definecolor{currentstroke}{rgb}{0.000000,0.000000,0.000000}%
\pgfsetstrokecolor{currentstroke}%
\pgfsetdash{}{0pt}%
\pgfpathmoveto{\pgfqpoint{4.267628in}{2.613520in}}%
\pgfpathlineto{\pgfqpoint{4.511410in}{2.753953in}}%
\pgfpathlineto{\pgfqpoint{4.267628in}{2.613520in}}%
\pgfpathclose%
\pgfusepath{fill}%
\end{pgfscope}%
\begin{pgfscope}%
\pgfpathrectangle{\pgfqpoint{0.017500in}{0.017500in}}{\pgfqpoint{6.218282in}{3.580372in}} %
\pgfusepath{clip}%
\pgfsetbuttcap%
\pgfsetroundjoin%
\definecolor{currentfill}{rgb}{0.729196,0.086679,0.167240}%
\pgfsetfillcolor{currentfill}%
\pgfsetlinewidth{0.000000pt}%
\definecolor{currentstroke}{rgb}{0.000000,0.000000,0.000000}%
\pgfsetstrokecolor{currentstroke}%
\pgfsetdash{}{0pt}%
\pgfpathmoveto{\pgfqpoint{4.267628in}{2.613520in}}%
\pgfpathlineto{\pgfqpoint{4.511410in}{2.753953in}}%
\pgfpathlineto{\pgfqpoint{4.267628in}{2.613520in}}%
\pgfpathclose%
\pgfusepath{fill}%
\end{pgfscope}%
\begin{pgfscope}%
\pgfpathrectangle{\pgfqpoint{0.017500in}{0.017500in}}{\pgfqpoint{6.218282in}{3.580372in}} %
\pgfusepath{clip}%
\pgfsetbuttcap%
\pgfsetroundjoin%
\definecolor{currentfill}{rgb}{0.729196,0.086679,0.167240}%
\pgfsetfillcolor{currentfill}%
\pgfsetlinewidth{0.000000pt}%
\definecolor{currentstroke}{rgb}{0.000000,0.000000,0.000000}%
\pgfsetstrokecolor{currentstroke}%
\pgfsetdash{}{0pt}%
\pgfpathmoveto{\pgfqpoint{4.267628in}{2.613520in}}%
\pgfpathlineto{\pgfqpoint{4.511410in}{2.753953in}}%
\pgfpathlineto{\pgfqpoint{4.267628in}{2.613520in}}%
\pgfpathclose%
\pgfusepath{fill}%
\end{pgfscope}%
\begin{pgfscope}%
\pgfpathrectangle{\pgfqpoint{0.017500in}{0.017500in}}{\pgfqpoint{6.218282in}{3.580372in}} %
\pgfusepath{clip}%
\pgfsetbuttcap%
\pgfsetroundjoin%
\definecolor{currentfill}{rgb}{0.729196,0.086679,0.167240}%
\pgfsetfillcolor{currentfill}%
\pgfsetlinewidth{0.000000pt}%
\definecolor{currentstroke}{rgb}{0.000000,0.000000,0.000000}%
\pgfsetstrokecolor{currentstroke}%
\pgfsetdash{}{0pt}%
\pgfpathmoveto{\pgfqpoint{4.267628in}{2.613520in}}%
\pgfpathlineto{\pgfqpoint{4.511410in}{2.753953in}}%
\pgfpathlineto{\pgfqpoint{4.267628in}{2.613520in}}%
\pgfpathclose%
\pgfusepath{fill}%
\end{pgfscope}%
\begin{pgfscope}%
\pgfpathrectangle{\pgfqpoint{0.017500in}{0.017500in}}{\pgfqpoint{6.218282in}{3.580372in}} %
\pgfusepath{clip}%
\pgfsetbuttcap%
\pgfsetroundjoin%
\definecolor{currentfill}{rgb}{0.729196,0.086679,0.167240}%
\pgfsetfillcolor{currentfill}%
\pgfsetlinewidth{0.000000pt}%
\definecolor{currentstroke}{rgb}{0.000000,0.000000,0.000000}%
\pgfsetstrokecolor{currentstroke}%
\pgfsetdash{}{0pt}%
\pgfpathmoveto{\pgfqpoint{4.267628in}{2.613520in}}%
\pgfpathlineto{\pgfqpoint{4.511410in}{2.753953in}}%
\pgfpathlineto{\pgfqpoint{4.267628in}{2.613520in}}%
\pgfpathclose%
\pgfusepath{fill}%
\end{pgfscope}%
\begin{pgfscope}%
\pgfpathrectangle{\pgfqpoint{0.017500in}{0.017500in}}{\pgfqpoint{6.218282in}{3.580372in}} %
\pgfusepath{clip}%
\pgfsetbuttcap%
\pgfsetroundjoin%
\definecolor{currentfill}{rgb}{0.729196,0.086679,0.167240}%
\pgfsetfillcolor{currentfill}%
\pgfsetlinewidth{0.000000pt}%
\definecolor{currentstroke}{rgb}{0.000000,0.000000,0.000000}%
\pgfsetstrokecolor{currentstroke}%
\pgfsetdash{}{0pt}%
\pgfpathmoveto{\pgfqpoint{4.267628in}{2.613520in}}%
\pgfpathlineto{\pgfqpoint{4.511410in}{2.753953in}}%
\pgfpathlineto{\pgfqpoint{4.267628in}{2.613520in}}%
\pgfpathclose%
\pgfusepath{fill}%
\end{pgfscope}%
\begin{pgfscope}%
\pgfpathrectangle{\pgfqpoint{0.017500in}{0.017500in}}{\pgfqpoint{6.218282in}{3.580372in}} %
\pgfusepath{clip}%
\pgfsetbuttcap%
\pgfsetroundjoin%
\definecolor{currentfill}{rgb}{0.729196,0.086679,0.167240}%
\pgfsetfillcolor{currentfill}%
\pgfsetlinewidth{0.000000pt}%
\definecolor{currentstroke}{rgb}{0.000000,0.000000,0.000000}%
\pgfsetstrokecolor{currentstroke}%
\pgfsetdash{}{0pt}%
\pgfpathmoveto{\pgfqpoint{4.267628in}{2.613520in}}%
\pgfpathlineto{\pgfqpoint{4.511410in}{2.753953in}}%
\pgfpathlineto{\pgfqpoint{4.267628in}{2.613520in}}%
\pgfpathclose%
\pgfusepath{fill}%
\end{pgfscope}%
\begin{pgfscope}%
\pgfpathrectangle{\pgfqpoint{0.017500in}{0.017500in}}{\pgfqpoint{6.218282in}{3.580372in}} %
\pgfusepath{clip}%
\pgfsetbuttcap%
\pgfsetroundjoin%
\definecolor{currentfill}{rgb}{0.729196,0.086679,0.167240}%
\pgfsetfillcolor{currentfill}%
\pgfsetlinewidth{0.000000pt}%
\definecolor{currentstroke}{rgb}{0.000000,0.000000,0.000000}%
\pgfsetstrokecolor{currentstroke}%
\pgfsetdash{}{0pt}%
\pgfpathmoveto{\pgfqpoint{4.267628in}{2.613520in}}%
\pgfpathlineto{\pgfqpoint{4.511410in}{2.753953in}}%
\pgfpathlineto{\pgfqpoint{4.267628in}{2.613520in}}%
\pgfpathclose%
\pgfusepath{fill}%
\end{pgfscope}%
\begin{pgfscope}%
\pgfpathrectangle{\pgfqpoint{0.017500in}{0.017500in}}{\pgfqpoint{6.218282in}{3.580372in}} %
\pgfusepath{clip}%
\pgfsetbuttcap%
\pgfsetroundjoin%
\definecolor{currentfill}{rgb}{0.729196,0.086679,0.167240}%
\pgfsetfillcolor{currentfill}%
\pgfsetlinewidth{0.000000pt}%
\definecolor{currentstroke}{rgb}{0.000000,0.000000,0.000000}%
\pgfsetstrokecolor{currentstroke}%
\pgfsetdash{}{0pt}%
\pgfpathmoveto{\pgfqpoint{4.267628in}{2.613520in}}%
\pgfpathlineto{\pgfqpoint{4.511410in}{2.753953in}}%
\pgfpathlineto{\pgfqpoint{4.267628in}{2.613520in}}%
\pgfpathclose%
\pgfusepath{fill}%
\end{pgfscope}%
\begin{pgfscope}%
\pgfpathrectangle{\pgfqpoint{0.017500in}{0.017500in}}{\pgfqpoint{6.218282in}{3.580372in}} %
\pgfusepath{clip}%
\pgfsetbuttcap%
\pgfsetroundjoin%
\definecolor{currentfill}{rgb}{0.729196,0.086679,0.167240}%
\pgfsetfillcolor{currentfill}%
\pgfsetlinewidth{0.000000pt}%
\definecolor{currentstroke}{rgb}{0.000000,0.000000,0.000000}%
\pgfsetstrokecolor{currentstroke}%
\pgfsetdash{}{0pt}%
\pgfpathmoveto{\pgfqpoint{4.267628in}{2.613520in}}%
\pgfpathlineto{\pgfqpoint{4.511410in}{2.753953in}}%
\pgfpathlineto{\pgfqpoint{4.267628in}{2.613520in}}%
\pgfpathclose%
\pgfusepath{fill}%
\end{pgfscope}%
\begin{pgfscope}%
\pgfpathrectangle{\pgfqpoint{0.017500in}{0.017500in}}{\pgfqpoint{6.218282in}{3.580372in}} %
\pgfusepath{clip}%
\pgfsetbuttcap%
\pgfsetroundjoin%
\definecolor{currentfill}{rgb}{0.729196,0.086679,0.167240}%
\pgfsetfillcolor{currentfill}%
\pgfsetlinewidth{0.000000pt}%
\definecolor{currentstroke}{rgb}{0.000000,0.000000,0.000000}%
\pgfsetstrokecolor{currentstroke}%
\pgfsetdash{}{0pt}%
\pgfpathmoveto{\pgfqpoint{4.267628in}{2.613520in}}%
\pgfpathlineto{\pgfqpoint{4.511410in}{2.753953in}}%
\pgfpathlineto{\pgfqpoint{4.267628in}{2.613520in}}%
\pgfpathclose%
\pgfusepath{fill}%
\end{pgfscope}%
\begin{pgfscope}%
\pgfpathrectangle{\pgfqpoint{0.017500in}{0.017500in}}{\pgfqpoint{6.218282in}{3.580372in}} %
\pgfusepath{clip}%
\pgfsetbuttcap%
\pgfsetroundjoin%
\definecolor{currentfill}{rgb}{0.729196,0.086679,0.167240}%
\pgfsetfillcolor{currentfill}%
\pgfsetlinewidth{0.000000pt}%
\definecolor{currentstroke}{rgb}{0.000000,0.000000,0.000000}%
\pgfsetstrokecolor{currentstroke}%
\pgfsetdash{}{0pt}%
\pgfpathmoveto{\pgfqpoint{4.267628in}{2.613520in}}%
\pgfpathlineto{\pgfqpoint{4.511410in}{2.753953in}}%
\pgfpathlineto{\pgfqpoint{4.267628in}{2.613520in}}%
\pgfpathclose%
\pgfusepath{fill}%
\end{pgfscope}%
\begin{pgfscope}%
\pgfpathrectangle{\pgfqpoint{0.017500in}{0.017500in}}{\pgfqpoint{6.218282in}{3.580372in}} %
\pgfusepath{clip}%
\pgfsetbuttcap%
\pgfsetroundjoin%
\definecolor{currentfill}{rgb}{0.729196,0.086679,0.167240}%
\pgfsetfillcolor{currentfill}%
\pgfsetlinewidth{0.000000pt}%
\definecolor{currentstroke}{rgb}{0.000000,0.000000,0.000000}%
\pgfsetstrokecolor{currentstroke}%
\pgfsetdash{}{0pt}%
\pgfpathmoveto{\pgfqpoint{4.267628in}{2.613520in}}%
\pgfpathlineto{\pgfqpoint{4.511410in}{2.753953in}}%
\pgfpathlineto{\pgfqpoint{4.267628in}{2.613520in}}%
\pgfpathclose%
\pgfusepath{fill}%
\end{pgfscope}%
\begin{pgfscope}%
\pgfpathrectangle{\pgfqpoint{0.017500in}{0.017500in}}{\pgfqpoint{6.218282in}{3.580372in}} %
\pgfusepath{clip}%
\pgfsetbuttcap%
\pgfsetroundjoin%
\definecolor{currentfill}{rgb}{0.729196,0.086679,0.167240}%
\pgfsetfillcolor{currentfill}%
\pgfsetlinewidth{0.000000pt}%
\definecolor{currentstroke}{rgb}{0.000000,0.000000,0.000000}%
\pgfsetstrokecolor{currentstroke}%
\pgfsetdash{}{0pt}%
\pgfpathmoveto{\pgfqpoint{4.267628in}{2.613520in}}%
\pgfpathlineto{\pgfqpoint{4.511410in}{2.753953in}}%
\pgfpathlineto{\pgfqpoint{4.267628in}{2.613520in}}%
\pgfpathclose%
\pgfusepath{fill}%
\end{pgfscope}%
\begin{pgfscope}%
\pgfpathrectangle{\pgfqpoint{0.017500in}{0.017500in}}{\pgfqpoint{6.218282in}{3.580372in}} %
\pgfusepath{clip}%
\pgfsetbuttcap%
\pgfsetroundjoin%
\definecolor{currentfill}{rgb}{0.729196,0.086679,0.167240}%
\pgfsetfillcolor{currentfill}%
\pgfsetlinewidth{0.000000pt}%
\definecolor{currentstroke}{rgb}{0.000000,0.000000,0.000000}%
\pgfsetstrokecolor{currentstroke}%
\pgfsetdash{}{0pt}%
\pgfpathmoveto{\pgfqpoint{4.267628in}{2.613520in}}%
\pgfpathlineto{\pgfqpoint{4.511410in}{2.753953in}}%
\pgfpathlineto{\pgfqpoint{4.267628in}{2.613520in}}%
\pgfpathclose%
\pgfusepath{fill}%
\end{pgfscope}%
\begin{pgfscope}%
\pgfpathrectangle{\pgfqpoint{0.017500in}{0.017500in}}{\pgfqpoint{6.218282in}{3.580372in}} %
\pgfusepath{clip}%
\pgfsetbuttcap%
\pgfsetroundjoin%
\definecolor{currentfill}{rgb}{0.729196,0.086679,0.167240}%
\pgfsetfillcolor{currentfill}%
\pgfsetlinewidth{0.000000pt}%
\definecolor{currentstroke}{rgb}{0.000000,0.000000,0.000000}%
\pgfsetstrokecolor{currentstroke}%
\pgfsetdash{}{0pt}%
\pgfpathmoveto{\pgfqpoint{4.267628in}{2.613520in}}%
\pgfpathlineto{\pgfqpoint{4.511410in}{2.753953in}}%
\pgfpathlineto{\pgfqpoint{4.267628in}{2.613520in}}%
\pgfpathclose%
\pgfusepath{fill}%
\end{pgfscope}%
\begin{pgfscope}%
\pgfpathrectangle{\pgfqpoint{0.017500in}{0.017500in}}{\pgfqpoint{6.218282in}{3.580372in}} %
\pgfusepath{clip}%
\pgfsetbuttcap%
\pgfsetroundjoin%
\definecolor{currentfill}{rgb}{0.729196,0.086679,0.167240}%
\pgfsetfillcolor{currentfill}%
\pgfsetlinewidth{0.000000pt}%
\definecolor{currentstroke}{rgb}{0.000000,0.000000,0.000000}%
\pgfsetstrokecolor{currentstroke}%
\pgfsetdash{}{0pt}%
\pgfpathmoveto{\pgfqpoint{4.267628in}{2.613520in}}%
\pgfpathlineto{\pgfqpoint{4.511410in}{2.753953in}}%
\pgfpathlineto{\pgfqpoint{4.267628in}{2.613520in}}%
\pgfpathclose%
\pgfusepath{fill}%
\end{pgfscope}%
\begin{pgfscope}%
\pgfpathrectangle{\pgfqpoint{0.017500in}{0.017500in}}{\pgfqpoint{6.218282in}{3.580372in}} %
\pgfusepath{clip}%
\pgfsetbuttcap%
\pgfsetroundjoin%
\definecolor{currentfill}{rgb}{0.729196,0.086679,0.167240}%
\pgfsetfillcolor{currentfill}%
\pgfsetlinewidth{0.000000pt}%
\definecolor{currentstroke}{rgb}{0.000000,0.000000,0.000000}%
\pgfsetstrokecolor{currentstroke}%
\pgfsetdash{}{0pt}%
\pgfpathmoveto{\pgfqpoint{4.267628in}{2.613520in}}%
\pgfpathlineto{\pgfqpoint{4.511410in}{2.753953in}}%
\pgfpathlineto{\pgfqpoint{4.267628in}{2.613520in}}%
\pgfpathclose%
\pgfusepath{fill}%
\end{pgfscope}%
\begin{pgfscope}%
\pgfpathrectangle{\pgfqpoint{0.017500in}{0.017500in}}{\pgfqpoint{6.218282in}{3.580372in}} %
\pgfusepath{clip}%
\pgfsetbuttcap%
\pgfsetroundjoin%
\definecolor{currentfill}{rgb}{0.729196,0.086679,0.167240}%
\pgfsetfillcolor{currentfill}%
\pgfsetlinewidth{0.000000pt}%
\definecolor{currentstroke}{rgb}{0.000000,0.000000,0.000000}%
\pgfsetstrokecolor{currentstroke}%
\pgfsetdash{}{0pt}%
\pgfpathmoveto{\pgfqpoint{4.267628in}{2.613520in}}%
\pgfpathlineto{\pgfqpoint{4.511410in}{2.753953in}}%
\pgfpathlineto{\pgfqpoint{4.267628in}{2.613520in}}%
\pgfpathclose%
\pgfusepath{fill}%
\end{pgfscope}%
\begin{pgfscope}%
\pgfpathrectangle{\pgfqpoint{0.017500in}{0.017500in}}{\pgfqpoint{6.218282in}{3.580372in}} %
\pgfusepath{clip}%
\pgfsetbuttcap%
\pgfsetroundjoin%
\definecolor{currentfill}{rgb}{0.729196,0.086679,0.167240}%
\pgfsetfillcolor{currentfill}%
\pgfsetlinewidth{0.000000pt}%
\definecolor{currentstroke}{rgb}{0.000000,0.000000,0.000000}%
\pgfsetstrokecolor{currentstroke}%
\pgfsetdash{}{0pt}%
\pgfpathmoveto{\pgfqpoint{4.267628in}{2.613520in}}%
\pgfpathlineto{\pgfqpoint{4.511410in}{2.753953in}}%
\pgfpathlineto{\pgfqpoint{4.267628in}{2.613520in}}%
\pgfpathclose%
\pgfusepath{fill}%
\end{pgfscope}%
\begin{pgfscope}%
\pgfpathrectangle{\pgfqpoint{0.017500in}{0.017500in}}{\pgfqpoint{6.218282in}{3.580372in}} %
\pgfusepath{clip}%
\pgfsetbuttcap%
\pgfsetroundjoin%
\definecolor{currentfill}{rgb}{0.810616,0.268797,0.235428}%
\pgfsetfillcolor{currentfill}%
\pgfsetlinewidth{0.000000pt}%
\definecolor{currentstroke}{rgb}{0.000000,0.000000,0.000000}%
\pgfsetstrokecolor{currentstroke}%
\pgfsetdash{}{0pt}%
\pgfpathmoveto{\pgfqpoint{2.410872in}{2.508083in}}%
\pgfpathlineto{\pgfqpoint{2.674254in}{2.688576in}}%
\pgfpathlineto{\pgfqpoint{2.410872in}{2.508083in}}%
\pgfpathclose%
\pgfusepath{fill}%
\end{pgfscope}%
\begin{pgfscope}%
\pgfpathrectangle{\pgfqpoint{0.017500in}{0.017500in}}{\pgfqpoint{6.218282in}{3.580372in}} %
\pgfusepath{clip}%
\pgfsetbuttcap%
\pgfsetroundjoin%
\definecolor{currentfill}{rgb}{0.810616,0.268797,0.235428}%
\pgfsetfillcolor{currentfill}%
\pgfsetlinewidth{0.000000pt}%
\definecolor{currentstroke}{rgb}{0.000000,0.000000,0.000000}%
\pgfsetstrokecolor{currentstroke}%
\pgfsetdash{}{0pt}%
\pgfpathmoveto{\pgfqpoint{2.410872in}{2.508083in}}%
\pgfpathlineto{\pgfqpoint{2.674254in}{2.688576in}}%
\pgfpathlineto{\pgfqpoint{2.410872in}{2.508083in}}%
\pgfpathclose%
\pgfusepath{fill}%
\end{pgfscope}%
\begin{pgfscope}%
\pgfpathrectangle{\pgfqpoint{0.017500in}{0.017500in}}{\pgfqpoint{6.218282in}{3.580372in}} %
\pgfusepath{clip}%
\pgfsetbuttcap%
\pgfsetroundjoin%
\definecolor{currentfill}{rgb}{0.810616,0.268797,0.235428}%
\pgfsetfillcolor{currentfill}%
\pgfsetlinewidth{0.000000pt}%
\definecolor{currentstroke}{rgb}{0.000000,0.000000,0.000000}%
\pgfsetstrokecolor{currentstroke}%
\pgfsetdash{}{0pt}%
\pgfpathmoveto{\pgfqpoint{2.410872in}{2.508083in}}%
\pgfpathlineto{\pgfqpoint{2.674254in}{2.688576in}}%
\pgfpathlineto{\pgfqpoint{2.410872in}{2.508083in}}%
\pgfpathclose%
\pgfusepath{fill}%
\end{pgfscope}%
\begin{pgfscope}%
\pgfpathrectangle{\pgfqpoint{0.017500in}{0.017500in}}{\pgfqpoint{6.218282in}{3.580372in}} %
\pgfusepath{clip}%
\pgfsetbuttcap%
\pgfsetroundjoin%
\definecolor{currentfill}{rgb}{0.810616,0.268797,0.235428}%
\pgfsetfillcolor{currentfill}%
\pgfsetlinewidth{0.000000pt}%
\definecolor{currentstroke}{rgb}{0.000000,0.000000,0.000000}%
\pgfsetstrokecolor{currentstroke}%
\pgfsetdash{}{0pt}%
\pgfpathmoveto{\pgfqpoint{2.410872in}{2.508083in}}%
\pgfpathlineto{\pgfqpoint{2.674254in}{2.688576in}}%
\pgfpathlineto{\pgfqpoint{2.410872in}{2.508083in}}%
\pgfpathclose%
\pgfusepath{fill}%
\end{pgfscope}%
\begin{pgfscope}%
\pgfpathrectangle{\pgfqpoint{0.017500in}{0.017500in}}{\pgfqpoint{6.218282in}{3.580372in}} %
\pgfusepath{clip}%
\pgfsetbuttcap%
\pgfsetroundjoin%
\definecolor{currentfill}{rgb}{0.810616,0.268797,0.235428}%
\pgfsetfillcolor{currentfill}%
\pgfsetlinewidth{0.000000pt}%
\definecolor{currentstroke}{rgb}{0.000000,0.000000,0.000000}%
\pgfsetstrokecolor{currentstroke}%
\pgfsetdash{}{0pt}%
\pgfpathmoveto{\pgfqpoint{2.410872in}{2.508083in}}%
\pgfpathlineto{\pgfqpoint{2.674254in}{2.688576in}}%
\pgfpathlineto{\pgfqpoint{2.410872in}{2.508083in}}%
\pgfpathclose%
\pgfusepath{fill}%
\end{pgfscope}%
\begin{pgfscope}%
\pgfpathrectangle{\pgfqpoint{0.017500in}{0.017500in}}{\pgfqpoint{6.218282in}{3.580372in}} %
\pgfusepath{clip}%
\pgfsetbuttcap%
\pgfsetroundjoin%
\definecolor{currentfill}{rgb}{0.810616,0.268797,0.235428}%
\pgfsetfillcolor{currentfill}%
\pgfsetlinewidth{0.000000pt}%
\definecolor{currentstroke}{rgb}{0.000000,0.000000,0.000000}%
\pgfsetstrokecolor{currentstroke}%
\pgfsetdash{}{0pt}%
\pgfpathmoveto{\pgfqpoint{2.410872in}{2.508083in}}%
\pgfpathlineto{\pgfqpoint{2.674254in}{2.688576in}}%
\pgfpathlineto{\pgfqpoint{2.410872in}{2.508083in}}%
\pgfpathclose%
\pgfusepath{fill}%
\end{pgfscope}%
\begin{pgfscope}%
\pgfpathrectangle{\pgfqpoint{0.017500in}{0.017500in}}{\pgfqpoint{6.218282in}{3.580372in}} %
\pgfusepath{clip}%
\pgfsetbuttcap%
\pgfsetroundjoin%
\definecolor{currentfill}{rgb}{0.810616,0.268797,0.235428}%
\pgfsetfillcolor{currentfill}%
\pgfsetlinewidth{0.000000pt}%
\definecolor{currentstroke}{rgb}{0.000000,0.000000,0.000000}%
\pgfsetstrokecolor{currentstroke}%
\pgfsetdash{}{0pt}%
\pgfpathmoveto{\pgfqpoint{2.410872in}{2.508083in}}%
\pgfpathlineto{\pgfqpoint{2.674254in}{2.688576in}}%
\pgfpathlineto{\pgfqpoint{2.410872in}{2.508083in}}%
\pgfpathclose%
\pgfusepath{fill}%
\end{pgfscope}%
\begin{pgfscope}%
\pgfpathrectangle{\pgfqpoint{0.017500in}{0.017500in}}{\pgfqpoint{6.218282in}{3.580372in}} %
\pgfusepath{clip}%
\pgfsetbuttcap%
\pgfsetroundjoin%
\definecolor{currentfill}{rgb}{0.810616,0.268797,0.235428}%
\pgfsetfillcolor{currentfill}%
\pgfsetlinewidth{0.000000pt}%
\definecolor{currentstroke}{rgb}{0.000000,0.000000,0.000000}%
\pgfsetstrokecolor{currentstroke}%
\pgfsetdash{}{0pt}%
\pgfpathmoveto{\pgfqpoint{2.410872in}{2.508083in}}%
\pgfpathlineto{\pgfqpoint{2.674254in}{2.688576in}}%
\pgfpathlineto{\pgfqpoint{2.410872in}{2.508083in}}%
\pgfpathclose%
\pgfusepath{fill}%
\end{pgfscope}%
\begin{pgfscope}%
\pgfpathrectangle{\pgfqpoint{0.017500in}{0.017500in}}{\pgfqpoint{6.218282in}{3.580372in}} %
\pgfusepath{clip}%
\pgfsetbuttcap%
\pgfsetroundjoin%
\definecolor{currentfill}{rgb}{0.810616,0.268797,0.235428}%
\pgfsetfillcolor{currentfill}%
\pgfsetlinewidth{0.000000pt}%
\definecolor{currentstroke}{rgb}{0.000000,0.000000,0.000000}%
\pgfsetstrokecolor{currentstroke}%
\pgfsetdash{}{0pt}%
\pgfpathmoveto{\pgfqpoint{2.410872in}{2.508083in}}%
\pgfpathlineto{\pgfqpoint{2.674254in}{2.688576in}}%
\pgfpathlineto{\pgfqpoint{2.410872in}{2.508083in}}%
\pgfpathclose%
\pgfusepath{fill}%
\end{pgfscope}%
\begin{pgfscope}%
\pgfpathrectangle{\pgfqpoint{0.017500in}{0.017500in}}{\pgfqpoint{6.218282in}{3.580372in}} %
\pgfusepath{clip}%
\pgfsetbuttcap%
\pgfsetroundjoin%
\definecolor{currentfill}{rgb}{0.810616,0.268797,0.235428}%
\pgfsetfillcolor{currentfill}%
\pgfsetlinewidth{0.000000pt}%
\definecolor{currentstroke}{rgb}{0.000000,0.000000,0.000000}%
\pgfsetstrokecolor{currentstroke}%
\pgfsetdash{}{0pt}%
\pgfpathmoveto{\pgfqpoint{2.410872in}{2.508083in}}%
\pgfpathlineto{\pgfqpoint{2.674254in}{2.688576in}}%
\pgfpathlineto{\pgfqpoint{2.410872in}{2.508083in}}%
\pgfpathclose%
\pgfusepath{fill}%
\end{pgfscope}%
\begin{pgfscope}%
\pgfpathrectangle{\pgfqpoint{0.017500in}{0.017500in}}{\pgfqpoint{6.218282in}{3.580372in}} %
\pgfusepath{clip}%
\pgfsetbuttcap%
\pgfsetroundjoin%
\definecolor{currentfill}{rgb}{0.810616,0.268797,0.235428}%
\pgfsetfillcolor{currentfill}%
\pgfsetlinewidth{0.000000pt}%
\definecolor{currentstroke}{rgb}{0.000000,0.000000,0.000000}%
\pgfsetstrokecolor{currentstroke}%
\pgfsetdash{}{0pt}%
\pgfpathmoveto{\pgfqpoint{2.410872in}{2.508083in}}%
\pgfpathlineto{\pgfqpoint{2.674254in}{2.688576in}}%
\pgfpathlineto{\pgfqpoint{2.410872in}{2.508083in}}%
\pgfpathclose%
\pgfusepath{fill}%
\end{pgfscope}%
\begin{pgfscope}%
\pgfpathrectangle{\pgfqpoint{0.017500in}{0.017500in}}{\pgfqpoint{6.218282in}{3.580372in}} %
\pgfusepath{clip}%
\pgfsetbuttcap%
\pgfsetroundjoin%
\definecolor{currentfill}{rgb}{0.810616,0.268797,0.235428}%
\pgfsetfillcolor{currentfill}%
\pgfsetlinewidth{0.000000pt}%
\definecolor{currentstroke}{rgb}{0.000000,0.000000,0.000000}%
\pgfsetstrokecolor{currentstroke}%
\pgfsetdash{}{0pt}%
\pgfpathmoveto{\pgfqpoint{2.410872in}{2.508083in}}%
\pgfpathlineto{\pgfqpoint{2.674254in}{2.688576in}}%
\pgfpathlineto{\pgfqpoint{2.410872in}{2.508083in}}%
\pgfpathclose%
\pgfusepath{fill}%
\end{pgfscope}%
\begin{pgfscope}%
\pgfpathrectangle{\pgfqpoint{0.017500in}{0.017500in}}{\pgfqpoint{6.218282in}{3.580372in}} %
\pgfusepath{clip}%
\pgfsetbuttcap%
\pgfsetroundjoin%
\definecolor{currentfill}{rgb}{0.810616,0.268797,0.235428}%
\pgfsetfillcolor{currentfill}%
\pgfsetlinewidth{0.000000pt}%
\definecolor{currentstroke}{rgb}{0.000000,0.000000,0.000000}%
\pgfsetstrokecolor{currentstroke}%
\pgfsetdash{}{0pt}%
\pgfpathmoveto{\pgfqpoint{2.410872in}{2.508083in}}%
\pgfpathlineto{\pgfqpoint{2.674254in}{2.688576in}}%
\pgfpathlineto{\pgfqpoint{2.410872in}{2.508083in}}%
\pgfpathclose%
\pgfusepath{fill}%
\end{pgfscope}%
\begin{pgfscope}%
\pgfpathrectangle{\pgfqpoint{0.017500in}{0.017500in}}{\pgfqpoint{6.218282in}{3.580372in}} %
\pgfusepath{clip}%
\pgfsetbuttcap%
\pgfsetroundjoin%
\definecolor{currentfill}{rgb}{0.810616,0.268797,0.235428}%
\pgfsetfillcolor{currentfill}%
\pgfsetlinewidth{0.000000pt}%
\definecolor{currentstroke}{rgb}{0.000000,0.000000,0.000000}%
\pgfsetstrokecolor{currentstroke}%
\pgfsetdash{}{0pt}%
\pgfpathmoveto{\pgfqpoint{2.410872in}{2.508083in}}%
\pgfpathlineto{\pgfqpoint{2.674254in}{2.688576in}}%
\pgfpathlineto{\pgfqpoint{2.410872in}{2.508083in}}%
\pgfpathclose%
\pgfusepath{fill}%
\end{pgfscope}%
\begin{pgfscope}%
\pgfpathrectangle{\pgfqpoint{0.017500in}{0.017500in}}{\pgfqpoint{6.218282in}{3.580372in}} %
\pgfusepath{clip}%
\pgfsetbuttcap%
\pgfsetroundjoin%
\definecolor{currentfill}{rgb}{0.810616,0.268797,0.235428}%
\pgfsetfillcolor{currentfill}%
\pgfsetlinewidth{0.000000pt}%
\definecolor{currentstroke}{rgb}{0.000000,0.000000,0.000000}%
\pgfsetstrokecolor{currentstroke}%
\pgfsetdash{}{0pt}%
\pgfpathmoveto{\pgfqpoint{2.410872in}{2.508083in}}%
\pgfpathlineto{\pgfqpoint{2.674254in}{2.688576in}}%
\pgfpathlineto{\pgfqpoint{2.410872in}{2.508083in}}%
\pgfpathclose%
\pgfusepath{fill}%
\end{pgfscope}%
\begin{pgfscope}%
\pgfpathrectangle{\pgfqpoint{0.017500in}{0.017500in}}{\pgfqpoint{6.218282in}{3.580372in}} %
\pgfusepath{clip}%
\pgfsetbuttcap%
\pgfsetroundjoin%
\definecolor{currentfill}{rgb}{0.810616,0.268797,0.235428}%
\pgfsetfillcolor{currentfill}%
\pgfsetlinewidth{0.000000pt}%
\definecolor{currentstroke}{rgb}{0.000000,0.000000,0.000000}%
\pgfsetstrokecolor{currentstroke}%
\pgfsetdash{}{0pt}%
\pgfpathmoveto{\pgfqpoint{2.410872in}{2.508083in}}%
\pgfpathlineto{\pgfqpoint{2.674254in}{2.688576in}}%
\pgfpathlineto{\pgfqpoint{2.410872in}{2.508083in}}%
\pgfpathclose%
\pgfusepath{fill}%
\end{pgfscope}%
\begin{pgfscope}%
\pgfpathrectangle{\pgfqpoint{0.017500in}{0.017500in}}{\pgfqpoint{6.218282in}{3.580372in}} %
\pgfusepath{clip}%
\pgfsetbuttcap%
\pgfsetroundjoin%
\definecolor{currentfill}{rgb}{0.810616,0.268797,0.235428}%
\pgfsetfillcolor{currentfill}%
\pgfsetlinewidth{0.000000pt}%
\definecolor{currentstroke}{rgb}{0.000000,0.000000,0.000000}%
\pgfsetstrokecolor{currentstroke}%
\pgfsetdash{}{0pt}%
\pgfpathmoveto{\pgfqpoint{2.410872in}{2.508083in}}%
\pgfpathlineto{\pgfqpoint{2.674254in}{2.688576in}}%
\pgfpathlineto{\pgfqpoint{2.410872in}{2.508083in}}%
\pgfpathclose%
\pgfusepath{fill}%
\end{pgfscope}%
\begin{pgfscope}%
\pgfpathrectangle{\pgfqpoint{0.017500in}{0.017500in}}{\pgfqpoint{6.218282in}{3.580372in}} %
\pgfusepath{clip}%
\pgfsetbuttcap%
\pgfsetroundjoin%
\definecolor{currentfill}{rgb}{0.810616,0.268797,0.235428}%
\pgfsetfillcolor{currentfill}%
\pgfsetlinewidth{0.000000pt}%
\definecolor{currentstroke}{rgb}{0.000000,0.000000,0.000000}%
\pgfsetstrokecolor{currentstroke}%
\pgfsetdash{}{0pt}%
\pgfpathmoveto{\pgfqpoint{2.410872in}{2.508083in}}%
\pgfpathlineto{\pgfqpoint{2.674254in}{2.688576in}}%
\pgfpathlineto{\pgfqpoint{2.410872in}{2.508083in}}%
\pgfpathclose%
\pgfusepath{fill}%
\end{pgfscope}%
\begin{pgfscope}%
\pgfpathrectangle{\pgfqpoint{0.017500in}{0.017500in}}{\pgfqpoint{6.218282in}{3.580372in}} %
\pgfusepath{clip}%
\pgfsetbuttcap%
\pgfsetroundjoin%
\definecolor{currentfill}{rgb}{0.810616,0.268797,0.235428}%
\pgfsetfillcolor{currentfill}%
\pgfsetlinewidth{0.000000pt}%
\definecolor{currentstroke}{rgb}{0.000000,0.000000,0.000000}%
\pgfsetstrokecolor{currentstroke}%
\pgfsetdash{}{0pt}%
\pgfpathmoveto{\pgfqpoint{2.410872in}{2.508083in}}%
\pgfpathlineto{\pgfqpoint{2.674254in}{2.688576in}}%
\pgfpathlineto{\pgfqpoint{2.410872in}{2.508083in}}%
\pgfpathclose%
\pgfusepath{fill}%
\end{pgfscope}%
\begin{pgfscope}%
\pgfpathrectangle{\pgfqpoint{0.017500in}{0.017500in}}{\pgfqpoint{6.218282in}{3.580372in}} %
\pgfusepath{clip}%
\pgfsetbuttcap%
\pgfsetroundjoin%
\definecolor{currentfill}{rgb}{0.810616,0.268797,0.235428}%
\pgfsetfillcolor{currentfill}%
\pgfsetlinewidth{0.000000pt}%
\definecolor{currentstroke}{rgb}{0.000000,0.000000,0.000000}%
\pgfsetstrokecolor{currentstroke}%
\pgfsetdash{}{0pt}%
\pgfpathmoveto{\pgfqpoint{2.410872in}{2.508083in}}%
\pgfpathlineto{\pgfqpoint{2.674254in}{2.688576in}}%
\pgfpathlineto{\pgfqpoint{2.410872in}{2.508083in}}%
\pgfpathclose%
\pgfusepath{fill}%
\end{pgfscope}%
\begin{pgfscope}%
\pgfpathrectangle{\pgfqpoint{0.017500in}{0.017500in}}{\pgfqpoint{6.218282in}{3.580372in}} %
\pgfusepath{clip}%
\pgfsetbuttcap%
\pgfsetroundjoin%
\definecolor{currentfill}{rgb}{0.810616,0.268797,0.235428}%
\pgfsetfillcolor{currentfill}%
\pgfsetlinewidth{0.000000pt}%
\definecolor{currentstroke}{rgb}{0.000000,0.000000,0.000000}%
\pgfsetstrokecolor{currentstroke}%
\pgfsetdash{}{0pt}%
\pgfpathmoveto{\pgfqpoint{2.410872in}{2.508083in}}%
\pgfpathlineto{\pgfqpoint{2.674254in}{2.688576in}}%
\pgfpathlineto{\pgfqpoint{2.410872in}{2.508083in}}%
\pgfpathclose%
\pgfusepath{fill}%
\end{pgfscope}%
\begin{pgfscope}%
\pgfpathrectangle{\pgfqpoint{0.017500in}{0.017500in}}{\pgfqpoint{6.218282in}{3.580372in}} %
\pgfusepath{clip}%
\pgfsetbuttcap%
\pgfsetroundjoin%
\definecolor{currentfill}{rgb}{0.810616,0.268797,0.235428}%
\pgfsetfillcolor{currentfill}%
\pgfsetlinewidth{0.000000pt}%
\definecolor{currentstroke}{rgb}{0.000000,0.000000,0.000000}%
\pgfsetstrokecolor{currentstroke}%
\pgfsetdash{}{0pt}%
\pgfpathmoveto{\pgfqpoint{2.410872in}{2.508083in}}%
\pgfpathlineto{\pgfqpoint{2.674254in}{2.688576in}}%
\pgfpathlineto{\pgfqpoint{2.410872in}{2.508083in}}%
\pgfpathclose%
\pgfusepath{fill}%
\end{pgfscope}%
\begin{pgfscope}%
\pgfpathrectangle{\pgfqpoint{0.017500in}{0.017500in}}{\pgfqpoint{6.218282in}{3.580372in}} %
\pgfusepath{clip}%
\pgfsetbuttcap%
\pgfsetroundjoin%
\definecolor{currentfill}{rgb}{0.810616,0.268797,0.235428}%
\pgfsetfillcolor{currentfill}%
\pgfsetlinewidth{0.000000pt}%
\definecolor{currentstroke}{rgb}{0.000000,0.000000,0.000000}%
\pgfsetstrokecolor{currentstroke}%
\pgfsetdash{}{0pt}%
\pgfpathmoveto{\pgfqpoint{2.410872in}{2.508083in}}%
\pgfpathlineto{\pgfqpoint{2.674254in}{2.688576in}}%
\pgfpathlineto{\pgfqpoint{2.410872in}{2.508083in}}%
\pgfpathclose%
\pgfusepath{fill}%
\end{pgfscope}%
\begin{pgfscope}%
\pgfpathrectangle{\pgfqpoint{0.017500in}{0.017500in}}{\pgfqpoint{6.218282in}{3.580372in}} %
\pgfusepath{clip}%
\pgfsetbuttcap%
\pgfsetroundjoin%
\definecolor{currentfill}{rgb}{0.810616,0.268797,0.235428}%
\pgfsetfillcolor{currentfill}%
\pgfsetlinewidth{0.000000pt}%
\definecolor{currentstroke}{rgb}{0.000000,0.000000,0.000000}%
\pgfsetstrokecolor{currentstroke}%
\pgfsetdash{}{0pt}%
\pgfpathmoveto{\pgfqpoint{2.410872in}{2.508083in}}%
\pgfpathlineto{\pgfqpoint{2.674254in}{2.688576in}}%
\pgfpathlineto{\pgfqpoint{2.410872in}{2.508083in}}%
\pgfpathclose%
\pgfusepath{fill}%
\end{pgfscope}%
\begin{pgfscope}%
\pgfpathrectangle{\pgfqpoint{0.017500in}{0.017500in}}{\pgfqpoint{6.218282in}{3.580372in}} %
\pgfusepath{clip}%
\pgfsetbuttcap%
\pgfsetroundjoin%
\definecolor{currentfill}{rgb}{0.810616,0.268797,0.235428}%
\pgfsetfillcolor{currentfill}%
\pgfsetlinewidth{0.000000pt}%
\definecolor{currentstroke}{rgb}{0.000000,0.000000,0.000000}%
\pgfsetstrokecolor{currentstroke}%
\pgfsetdash{}{0pt}%
\pgfpathmoveto{\pgfqpoint{2.410872in}{2.508083in}}%
\pgfpathlineto{\pgfqpoint{2.674254in}{2.688576in}}%
\pgfpathlineto{\pgfqpoint{2.410872in}{2.508083in}}%
\pgfpathclose%
\pgfusepath{fill}%
\end{pgfscope}%
\begin{pgfscope}%
\pgfpathrectangle{\pgfqpoint{0.017500in}{0.017500in}}{\pgfqpoint{6.218282in}{3.580372in}} %
\pgfusepath{clip}%
\pgfsetbuttcap%
\pgfsetroundjoin%
\definecolor{currentfill}{rgb}{0.810616,0.268797,0.235428}%
\pgfsetfillcolor{currentfill}%
\pgfsetlinewidth{0.000000pt}%
\definecolor{currentstroke}{rgb}{0.000000,0.000000,0.000000}%
\pgfsetstrokecolor{currentstroke}%
\pgfsetdash{}{0pt}%
\pgfpathmoveto{\pgfqpoint{2.410872in}{2.508083in}}%
\pgfpathlineto{\pgfqpoint{2.674254in}{2.688576in}}%
\pgfpathlineto{\pgfqpoint{2.410872in}{2.508083in}}%
\pgfpathclose%
\pgfusepath{fill}%
\end{pgfscope}%
\begin{pgfscope}%
\pgfpathrectangle{\pgfqpoint{0.017500in}{0.017500in}}{\pgfqpoint{6.218282in}{3.580372in}} %
\pgfusepath{clip}%
\pgfsetbuttcap%
\pgfsetroundjoin%
\definecolor{currentfill}{rgb}{0.810616,0.268797,0.235428}%
\pgfsetfillcolor{currentfill}%
\pgfsetlinewidth{0.000000pt}%
\definecolor{currentstroke}{rgb}{0.000000,0.000000,0.000000}%
\pgfsetstrokecolor{currentstroke}%
\pgfsetdash{}{0pt}%
\pgfpathmoveto{\pgfqpoint{2.410872in}{2.508083in}}%
\pgfpathlineto{\pgfqpoint{2.674254in}{2.688576in}}%
\pgfpathlineto{\pgfqpoint{2.410872in}{2.508083in}}%
\pgfpathclose%
\pgfusepath{fill}%
\end{pgfscope}%
\begin{pgfscope}%
\pgfpathrectangle{\pgfqpoint{0.017500in}{0.017500in}}{\pgfqpoint{6.218282in}{3.580372in}} %
\pgfusepath{clip}%
\pgfsetbuttcap%
\pgfsetroundjoin%
\definecolor{currentfill}{rgb}{0.810616,0.268797,0.235428}%
\pgfsetfillcolor{currentfill}%
\pgfsetlinewidth{0.000000pt}%
\definecolor{currentstroke}{rgb}{0.000000,0.000000,0.000000}%
\pgfsetstrokecolor{currentstroke}%
\pgfsetdash{}{0pt}%
\pgfpathmoveto{\pgfqpoint{2.410872in}{2.508083in}}%
\pgfpathlineto{\pgfqpoint{2.674254in}{2.688576in}}%
\pgfpathlineto{\pgfqpoint{2.410872in}{2.508083in}}%
\pgfpathclose%
\pgfusepath{fill}%
\end{pgfscope}%
\begin{pgfscope}%
\pgfpathrectangle{\pgfqpoint{0.017500in}{0.017500in}}{\pgfqpoint{6.218282in}{3.580372in}} %
\pgfusepath{clip}%
\pgfsetbuttcap%
\pgfsetroundjoin%
\definecolor{currentfill}{rgb}{0.810616,0.268797,0.235428}%
\pgfsetfillcolor{currentfill}%
\pgfsetlinewidth{0.000000pt}%
\definecolor{currentstroke}{rgb}{0.000000,0.000000,0.000000}%
\pgfsetstrokecolor{currentstroke}%
\pgfsetdash{}{0pt}%
\pgfpathmoveto{\pgfqpoint{2.410872in}{2.508083in}}%
\pgfpathlineto{\pgfqpoint{2.674254in}{2.688576in}}%
\pgfpathlineto{\pgfqpoint{2.410872in}{2.508083in}}%
\pgfpathclose%
\pgfusepath{fill}%
\end{pgfscope}%
\begin{pgfscope}%
\pgfpathrectangle{\pgfqpoint{0.017500in}{0.017500in}}{\pgfqpoint{6.218282in}{3.580372in}} %
\pgfusepath{clip}%
\pgfsetbuttcap%
\pgfsetroundjoin%
\definecolor{currentfill}{rgb}{0.810616,0.268797,0.235428}%
\pgfsetfillcolor{currentfill}%
\pgfsetlinewidth{0.000000pt}%
\definecolor{currentstroke}{rgb}{0.000000,0.000000,0.000000}%
\pgfsetstrokecolor{currentstroke}%
\pgfsetdash{}{0pt}%
\pgfpathmoveto{\pgfqpoint{2.410872in}{2.508083in}}%
\pgfpathlineto{\pgfqpoint{2.674254in}{2.688576in}}%
\pgfpathlineto{\pgfqpoint{2.410872in}{2.508083in}}%
\pgfpathclose%
\pgfusepath{fill}%
\end{pgfscope}%
\begin{pgfscope}%
\pgfpathrectangle{\pgfqpoint{0.017500in}{0.017500in}}{\pgfqpoint{6.218282in}{3.580372in}} %
\pgfusepath{clip}%
\pgfsetbuttcap%
\pgfsetroundjoin%
\definecolor{currentfill}{rgb}{0.333490,0.446265,0.874452}%
\pgfsetfillcolor{currentfill}%
\pgfsetlinewidth{0.000000pt}%
\definecolor{currentstroke}{rgb}{0.000000,0.000000,0.000000}%
\pgfsetstrokecolor{currentstroke}%
\pgfsetdash{}{0pt}%
\pgfpathmoveto{\pgfqpoint{1.914830in}{0.967594in}}%
\pgfpathlineto{\pgfqpoint{2.349853in}{0.872712in}}%
\pgfpathlineto{\pgfqpoint{2.590432in}{2.210709in}}%
\pgfpathlineto{\pgfqpoint{2.143341in}{2.291288in}}%
\pgfpathlineto{\pgfqpoint{1.914830in}{0.967594in}}%
\pgfpathclose%
\pgfusepath{fill}%
\end{pgfscope}%
\begin{pgfscope}%
\pgfpathrectangle{\pgfqpoint{0.017500in}{0.017500in}}{\pgfqpoint{6.218282in}{3.580372in}} %
\pgfusepath{clip}%
\pgfsetbuttcap%
\pgfsetroundjoin%
\definecolor{currentfill}{rgb}{0.333490,0.446265,0.874452}%
\pgfsetfillcolor{currentfill}%
\pgfsetlinewidth{0.000000pt}%
\definecolor{currentstroke}{rgb}{0.000000,0.000000,0.000000}%
\pgfsetstrokecolor{currentstroke}%
\pgfsetdash{}{0pt}%
\pgfpathmoveto{\pgfqpoint{1.914830in}{0.967594in}}%
\pgfpathlineto{\pgfqpoint{2.349853in}{0.872712in}}%
\pgfpathlineto{\pgfqpoint{2.590432in}{2.210709in}}%
\pgfpathlineto{\pgfqpoint{2.143341in}{2.291288in}}%
\pgfpathlineto{\pgfqpoint{1.914830in}{0.967594in}}%
\pgfpathclose%
\pgfusepath{fill}%
\end{pgfscope}%
\begin{pgfscope}%
\pgfpathrectangle{\pgfqpoint{0.017500in}{0.017500in}}{\pgfqpoint{6.218282in}{3.580372in}} %
\pgfusepath{clip}%
\pgfsetbuttcap%
\pgfsetroundjoin%
\definecolor{currentfill}{rgb}{0.333490,0.446265,0.874452}%
\pgfsetfillcolor{currentfill}%
\pgfsetlinewidth{0.000000pt}%
\definecolor{currentstroke}{rgb}{0.000000,0.000000,0.000000}%
\pgfsetstrokecolor{currentstroke}%
\pgfsetdash{}{0pt}%
\pgfpathmoveto{\pgfqpoint{1.914830in}{0.967594in}}%
\pgfpathlineto{\pgfqpoint{2.349853in}{0.872712in}}%
\pgfpathlineto{\pgfqpoint{2.590432in}{2.210709in}}%
\pgfpathlineto{\pgfqpoint{2.143341in}{2.291288in}}%
\pgfpathlineto{\pgfqpoint{1.914830in}{0.967594in}}%
\pgfpathclose%
\pgfusepath{fill}%
\end{pgfscope}%
\begin{pgfscope}%
\pgfpathrectangle{\pgfqpoint{0.017500in}{0.017500in}}{\pgfqpoint{6.218282in}{3.580372in}} %
\pgfusepath{clip}%
\pgfsetbuttcap%
\pgfsetroundjoin%
\definecolor{currentfill}{rgb}{0.333490,0.446265,0.874452}%
\pgfsetfillcolor{currentfill}%
\pgfsetlinewidth{0.000000pt}%
\definecolor{currentstroke}{rgb}{0.000000,0.000000,0.000000}%
\pgfsetstrokecolor{currentstroke}%
\pgfsetdash{}{0pt}%
\pgfpathmoveto{\pgfqpoint{1.914830in}{0.967594in}}%
\pgfpathlineto{\pgfqpoint{2.349853in}{0.872712in}}%
\pgfpathlineto{\pgfqpoint{2.590432in}{2.210709in}}%
\pgfpathlineto{\pgfqpoint{2.143341in}{2.291288in}}%
\pgfpathlineto{\pgfqpoint{1.914830in}{0.967594in}}%
\pgfpathclose%
\pgfusepath{fill}%
\end{pgfscope}%
\begin{pgfscope}%
\pgfpathrectangle{\pgfqpoint{0.017500in}{0.017500in}}{\pgfqpoint{6.218282in}{3.580372in}} %
\pgfusepath{clip}%
\pgfsetbuttcap%
\pgfsetroundjoin%
\definecolor{currentfill}{rgb}{0.333490,0.446265,0.874452}%
\pgfsetfillcolor{currentfill}%
\pgfsetlinewidth{0.000000pt}%
\definecolor{currentstroke}{rgb}{0.000000,0.000000,0.000000}%
\pgfsetstrokecolor{currentstroke}%
\pgfsetdash{}{0pt}%
\pgfpathmoveto{\pgfqpoint{1.914830in}{0.967594in}}%
\pgfpathlineto{\pgfqpoint{2.349853in}{0.872712in}}%
\pgfpathlineto{\pgfqpoint{2.590432in}{2.210709in}}%
\pgfpathlineto{\pgfqpoint{2.143341in}{2.291288in}}%
\pgfpathlineto{\pgfqpoint{1.914830in}{0.967594in}}%
\pgfpathclose%
\pgfusepath{fill}%
\end{pgfscope}%
\begin{pgfscope}%
\pgfpathrectangle{\pgfqpoint{0.017500in}{0.017500in}}{\pgfqpoint{6.218282in}{3.580372in}} %
\pgfusepath{clip}%
\pgfsetbuttcap%
\pgfsetroundjoin%
\definecolor{currentfill}{rgb}{0.333490,0.446265,0.874452}%
\pgfsetfillcolor{currentfill}%
\pgfsetlinewidth{0.000000pt}%
\definecolor{currentstroke}{rgb}{0.000000,0.000000,0.000000}%
\pgfsetstrokecolor{currentstroke}%
\pgfsetdash{}{0pt}%
\pgfpathmoveto{\pgfqpoint{1.914830in}{0.967594in}}%
\pgfpathlineto{\pgfqpoint{2.349853in}{0.872712in}}%
\pgfpathlineto{\pgfqpoint{2.590432in}{2.210709in}}%
\pgfpathlineto{\pgfqpoint{2.143341in}{2.291288in}}%
\pgfpathlineto{\pgfqpoint{1.914830in}{0.967594in}}%
\pgfpathclose%
\pgfusepath{fill}%
\end{pgfscope}%
\begin{pgfscope}%
\pgfpathrectangle{\pgfqpoint{0.017500in}{0.017500in}}{\pgfqpoint{6.218282in}{3.580372in}} %
\pgfusepath{clip}%
\pgfsetbuttcap%
\pgfsetroundjoin%
\definecolor{currentfill}{rgb}{0.804965,0.851666,0.926165}%
\pgfsetfillcolor{currentfill}%
\pgfsetlinewidth{0.000000pt}%
\definecolor{currentstroke}{rgb}{0.000000,0.000000,0.000000}%
\pgfsetstrokecolor{currentstroke}%
\pgfsetdash{}{0pt}%
\pgfpathmoveto{\pgfqpoint{4.511410in}{2.753953in}}%
\pgfpathlineto{\pgfqpoint{4.964811in}{2.684961in}}%
\pgfpathlineto{\pgfqpoint{3.688351in}{0.641693in}}%
\pgfpathlineto{\pgfqpoint{3.236099in}{0.723497in}}%
\pgfpathlineto{\pgfqpoint{4.511410in}{2.753953in}}%
\pgfpathclose%
\pgfusepath{fill}%
\end{pgfscope}%
\begin{pgfscope}%
\pgfpathrectangle{\pgfqpoint{0.017500in}{0.017500in}}{\pgfqpoint{6.218282in}{3.580372in}} %
\pgfusepath{clip}%
\pgfsetbuttcap%
\pgfsetroundjoin%
\definecolor{currentfill}{rgb}{0.804965,0.851666,0.926165}%
\pgfsetfillcolor{currentfill}%
\pgfsetlinewidth{0.000000pt}%
\definecolor{currentstroke}{rgb}{0.000000,0.000000,0.000000}%
\pgfsetstrokecolor{currentstroke}%
\pgfsetdash{}{0pt}%
\pgfpathmoveto{\pgfqpoint{4.511410in}{2.753953in}}%
\pgfpathlineto{\pgfqpoint{4.964811in}{2.684961in}}%
\pgfpathlineto{\pgfqpoint{3.688351in}{0.641693in}}%
\pgfpathlineto{\pgfqpoint{3.236099in}{0.723497in}}%
\pgfpathlineto{\pgfqpoint{4.511410in}{2.753953in}}%
\pgfpathclose%
\pgfusepath{fill}%
\end{pgfscope}%
\begin{pgfscope}%
\pgfpathrectangle{\pgfqpoint{0.017500in}{0.017500in}}{\pgfqpoint{6.218282in}{3.580372in}} %
\pgfusepath{clip}%
\pgfsetbuttcap%
\pgfsetroundjoin%
\definecolor{currentfill}{rgb}{0.804965,0.851666,0.926165}%
\pgfsetfillcolor{currentfill}%
\pgfsetlinewidth{0.000000pt}%
\definecolor{currentstroke}{rgb}{0.000000,0.000000,0.000000}%
\pgfsetstrokecolor{currentstroke}%
\pgfsetdash{}{0pt}%
\pgfpathmoveto{\pgfqpoint{4.511410in}{2.753953in}}%
\pgfpathlineto{\pgfqpoint{4.964811in}{2.684961in}}%
\pgfpathlineto{\pgfqpoint{3.688351in}{0.641693in}}%
\pgfpathlineto{\pgfqpoint{3.236099in}{0.723497in}}%
\pgfpathlineto{\pgfqpoint{4.511410in}{2.753953in}}%
\pgfpathclose%
\pgfusepath{fill}%
\end{pgfscope}%
\begin{pgfscope}%
\pgfpathrectangle{\pgfqpoint{0.017500in}{0.017500in}}{\pgfqpoint{6.218282in}{3.580372in}} %
\pgfusepath{clip}%
\pgfsetbuttcap%
\pgfsetroundjoin%
\definecolor{currentfill}{rgb}{0.804965,0.851666,0.926165}%
\pgfsetfillcolor{currentfill}%
\pgfsetlinewidth{0.000000pt}%
\definecolor{currentstroke}{rgb}{0.000000,0.000000,0.000000}%
\pgfsetstrokecolor{currentstroke}%
\pgfsetdash{}{0pt}%
\pgfpathmoveto{\pgfqpoint{4.511410in}{2.753953in}}%
\pgfpathlineto{\pgfqpoint{4.964811in}{2.684961in}}%
\pgfpathlineto{\pgfqpoint{3.688351in}{0.641693in}}%
\pgfpathlineto{\pgfqpoint{3.236099in}{0.723497in}}%
\pgfpathlineto{\pgfqpoint{4.511410in}{2.753953in}}%
\pgfpathclose%
\pgfusepath{fill}%
\end{pgfscope}%
\begin{pgfscope}%
\pgfpathrectangle{\pgfqpoint{0.017500in}{0.017500in}}{\pgfqpoint{6.218282in}{3.580372in}} %
\pgfusepath{clip}%
\pgfsetbuttcap%
\pgfsetroundjoin%
\definecolor{currentfill}{rgb}{0.804965,0.851666,0.926165}%
\pgfsetfillcolor{currentfill}%
\pgfsetlinewidth{0.000000pt}%
\definecolor{currentstroke}{rgb}{0.000000,0.000000,0.000000}%
\pgfsetstrokecolor{currentstroke}%
\pgfsetdash{}{0pt}%
\pgfpathmoveto{\pgfqpoint{4.511410in}{2.753953in}}%
\pgfpathlineto{\pgfqpoint{4.964811in}{2.684961in}}%
\pgfpathlineto{\pgfqpoint{3.688351in}{0.641693in}}%
\pgfpathlineto{\pgfqpoint{3.236099in}{0.723497in}}%
\pgfpathlineto{\pgfqpoint{4.511410in}{2.753953in}}%
\pgfpathclose%
\pgfusepath{fill}%
\end{pgfscope}%
\begin{pgfscope}%
\pgfpathrectangle{\pgfqpoint{0.017500in}{0.017500in}}{\pgfqpoint{6.218282in}{3.580372in}} %
\pgfusepath{clip}%
\pgfsetbuttcap%
\pgfsetroundjoin%
\definecolor{currentfill}{rgb}{0.723315,0.068898,0.162989}%
\pgfsetfillcolor{currentfill}%
\pgfsetlinewidth{0.000000pt}%
\definecolor{currentstroke}{rgb}{0.000000,0.000000,0.000000}%
\pgfsetstrokecolor{currentstroke}%
\pgfsetdash{}{0pt}%
\pgfpathmoveto{\pgfqpoint{4.267628in}{2.613520in}}%
\pgfpathlineto{\pgfqpoint{4.724651in}{2.549289in}}%
\pgfpathlineto{\pgfqpoint{4.964811in}{2.684961in}}%
\pgfpathlineto{\pgfqpoint{4.511410in}{2.753953in}}%
\pgfpathlineto{\pgfqpoint{4.267628in}{2.613520in}}%
\pgfpathclose%
\pgfusepath{fill}%
\end{pgfscope}%
\begin{pgfscope}%
\pgfpathrectangle{\pgfqpoint{0.017500in}{0.017500in}}{\pgfqpoint{6.218282in}{3.580372in}} %
\pgfusepath{clip}%
\pgfsetbuttcap%
\pgfsetroundjoin%
\definecolor{currentfill}{rgb}{0.723315,0.068898,0.162989}%
\pgfsetfillcolor{currentfill}%
\pgfsetlinewidth{0.000000pt}%
\definecolor{currentstroke}{rgb}{0.000000,0.000000,0.000000}%
\pgfsetstrokecolor{currentstroke}%
\pgfsetdash{}{0pt}%
\pgfpathmoveto{\pgfqpoint{4.267628in}{2.613520in}}%
\pgfpathlineto{\pgfqpoint{4.724651in}{2.549289in}}%
\pgfpathlineto{\pgfqpoint{4.964811in}{2.684961in}}%
\pgfpathlineto{\pgfqpoint{4.511410in}{2.753953in}}%
\pgfpathlineto{\pgfqpoint{4.267628in}{2.613520in}}%
\pgfpathclose%
\pgfusepath{fill}%
\end{pgfscope}%
\begin{pgfscope}%
\pgfpathrectangle{\pgfqpoint{0.017500in}{0.017500in}}{\pgfqpoint{6.218282in}{3.580372in}} %
\pgfusepath{clip}%
\pgfsetbuttcap%
\pgfsetroundjoin%
\definecolor{currentfill}{rgb}{0.723315,0.068898,0.162989}%
\pgfsetfillcolor{currentfill}%
\pgfsetlinewidth{0.000000pt}%
\definecolor{currentstroke}{rgb}{0.000000,0.000000,0.000000}%
\pgfsetstrokecolor{currentstroke}%
\pgfsetdash{}{0pt}%
\pgfpathmoveto{\pgfqpoint{4.267628in}{2.613520in}}%
\pgfpathlineto{\pgfqpoint{4.724651in}{2.549289in}}%
\pgfpathlineto{\pgfqpoint{4.964811in}{2.684961in}}%
\pgfpathlineto{\pgfqpoint{4.511410in}{2.753953in}}%
\pgfpathlineto{\pgfqpoint{4.267628in}{2.613520in}}%
\pgfpathclose%
\pgfusepath{fill}%
\end{pgfscope}%
\begin{pgfscope}%
\pgfpathrectangle{\pgfqpoint{0.017500in}{0.017500in}}{\pgfqpoint{6.218282in}{3.580372in}} %
\pgfusepath{clip}%
\pgfsetbuttcap%
\pgfsetroundjoin%
\definecolor{currentfill}{rgb}{0.723315,0.068898,0.162989}%
\pgfsetfillcolor{currentfill}%
\pgfsetlinewidth{0.000000pt}%
\definecolor{currentstroke}{rgb}{0.000000,0.000000,0.000000}%
\pgfsetstrokecolor{currentstroke}%
\pgfsetdash{}{0pt}%
\pgfpathmoveto{\pgfqpoint{4.267628in}{2.613520in}}%
\pgfpathlineto{\pgfqpoint{4.724651in}{2.549289in}}%
\pgfpathlineto{\pgfqpoint{4.964811in}{2.684961in}}%
\pgfpathlineto{\pgfqpoint{4.511410in}{2.753953in}}%
\pgfpathlineto{\pgfqpoint{4.267628in}{2.613520in}}%
\pgfpathclose%
\pgfusepath{fill}%
\end{pgfscope}%
\begin{pgfscope}%
\pgfpathrectangle{\pgfqpoint{0.017500in}{0.017500in}}{\pgfqpoint{6.218282in}{3.580372in}} %
\pgfusepath{clip}%
\pgfsetbuttcap%
\pgfsetroundjoin%
\definecolor{currentfill}{rgb}{0.723315,0.068898,0.162989}%
\pgfsetfillcolor{currentfill}%
\pgfsetlinewidth{0.000000pt}%
\definecolor{currentstroke}{rgb}{0.000000,0.000000,0.000000}%
\pgfsetstrokecolor{currentstroke}%
\pgfsetdash{}{0pt}%
\pgfpathmoveto{\pgfqpoint{4.267628in}{2.613520in}}%
\pgfpathlineto{\pgfqpoint{4.724651in}{2.549289in}}%
\pgfpathlineto{\pgfqpoint{4.964811in}{2.684961in}}%
\pgfpathlineto{\pgfqpoint{4.511410in}{2.753953in}}%
\pgfpathlineto{\pgfqpoint{4.267628in}{2.613520in}}%
\pgfpathclose%
\pgfusepath{fill}%
\end{pgfscope}%
\begin{pgfscope}%
\pgfpathrectangle{\pgfqpoint{0.017500in}{0.017500in}}{\pgfqpoint{6.218282in}{3.580372in}} %
\pgfusepath{clip}%
\pgfsetbuttcap%
\pgfsetroundjoin%
\definecolor{currentfill}{rgb}{0.723315,0.068898,0.162989}%
\pgfsetfillcolor{currentfill}%
\pgfsetlinewidth{0.000000pt}%
\definecolor{currentstroke}{rgb}{0.000000,0.000000,0.000000}%
\pgfsetstrokecolor{currentstroke}%
\pgfsetdash{}{0pt}%
\pgfpathmoveto{\pgfqpoint{4.267628in}{2.613520in}}%
\pgfpathlineto{\pgfqpoint{4.724651in}{2.549289in}}%
\pgfpathlineto{\pgfqpoint{4.964811in}{2.684961in}}%
\pgfpathlineto{\pgfqpoint{4.511410in}{2.753953in}}%
\pgfpathlineto{\pgfqpoint{4.267628in}{2.613520in}}%
\pgfpathclose%
\pgfusepath{fill}%
\end{pgfscope}%
\begin{pgfscope}%
\pgfpathrectangle{\pgfqpoint{0.017500in}{0.017500in}}{\pgfqpoint{6.218282in}{3.580372in}} %
\pgfusepath{clip}%
\pgfsetbuttcap%
\pgfsetroundjoin%
\definecolor{currentfill}{rgb}{0.805723,0.259813,0.230562}%
\pgfsetfillcolor{currentfill}%
\pgfsetlinewidth{0.000000pt}%
\definecolor{currentstroke}{rgb}{0.000000,0.000000,0.000000}%
\pgfsetstrokecolor{currentstroke}%
\pgfsetdash{}{0pt}%
\pgfpathmoveto{\pgfqpoint{2.410872in}{2.508083in}}%
\pgfpathlineto{\pgfqpoint{2.855513in}{2.451418in}}%
\pgfpathlineto{\pgfqpoint{3.116513in}{2.603812in}}%
\pgfpathlineto{\pgfqpoint{2.674254in}{2.688576in}}%
\pgfpathlineto{\pgfqpoint{2.410872in}{2.508083in}}%
\pgfpathclose%
\pgfusepath{fill}%
\end{pgfscope}%
\begin{pgfscope}%
\pgfpathrectangle{\pgfqpoint{0.017500in}{0.017500in}}{\pgfqpoint{6.218282in}{3.580372in}} %
\pgfusepath{clip}%
\pgfsetbuttcap%
\pgfsetroundjoin%
\definecolor{currentfill}{rgb}{0.805723,0.259813,0.230562}%
\pgfsetfillcolor{currentfill}%
\pgfsetlinewidth{0.000000pt}%
\definecolor{currentstroke}{rgb}{0.000000,0.000000,0.000000}%
\pgfsetstrokecolor{currentstroke}%
\pgfsetdash{}{0pt}%
\pgfpathmoveto{\pgfqpoint{2.410872in}{2.508083in}}%
\pgfpathlineto{\pgfqpoint{2.855513in}{2.451418in}}%
\pgfpathlineto{\pgfqpoint{3.116513in}{2.603812in}}%
\pgfpathlineto{\pgfqpoint{2.674254in}{2.688576in}}%
\pgfpathlineto{\pgfqpoint{2.410872in}{2.508083in}}%
\pgfpathclose%
\pgfusepath{fill}%
\end{pgfscope}%
\begin{pgfscope}%
\pgfpathrectangle{\pgfqpoint{0.017500in}{0.017500in}}{\pgfqpoint{6.218282in}{3.580372in}} %
\pgfusepath{clip}%
\pgfsetbuttcap%
\pgfsetroundjoin%
\definecolor{currentfill}{rgb}{0.805723,0.259813,0.230562}%
\pgfsetfillcolor{currentfill}%
\pgfsetlinewidth{0.000000pt}%
\definecolor{currentstroke}{rgb}{0.000000,0.000000,0.000000}%
\pgfsetstrokecolor{currentstroke}%
\pgfsetdash{}{0pt}%
\pgfpathmoveto{\pgfqpoint{2.410872in}{2.508083in}}%
\pgfpathlineto{\pgfqpoint{2.855513in}{2.451418in}}%
\pgfpathlineto{\pgfqpoint{3.116513in}{2.603812in}}%
\pgfpathlineto{\pgfqpoint{2.674254in}{2.688576in}}%
\pgfpathlineto{\pgfqpoint{2.410872in}{2.508083in}}%
\pgfpathclose%
\pgfusepath{fill}%
\end{pgfscope}%
\begin{pgfscope}%
\pgfpathrectangle{\pgfqpoint{0.017500in}{0.017500in}}{\pgfqpoint{6.218282in}{3.580372in}} %
\pgfusepath{clip}%
\pgfsetbuttcap%
\pgfsetroundjoin%
\definecolor{currentfill}{rgb}{0.805723,0.259813,0.230562}%
\pgfsetfillcolor{currentfill}%
\pgfsetlinewidth{0.000000pt}%
\definecolor{currentstroke}{rgb}{0.000000,0.000000,0.000000}%
\pgfsetstrokecolor{currentstroke}%
\pgfsetdash{}{0pt}%
\pgfpathmoveto{\pgfqpoint{2.410872in}{2.508083in}}%
\pgfpathlineto{\pgfqpoint{2.855513in}{2.451418in}}%
\pgfpathlineto{\pgfqpoint{3.116513in}{2.603812in}}%
\pgfpathlineto{\pgfqpoint{2.674254in}{2.688576in}}%
\pgfpathlineto{\pgfqpoint{2.410872in}{2.508083in}}%
\pgfpathclose%
\pgfusepath{fill}%
\end{pgfscope}%
\begin{pgfscope}%
\pgfpathrectangle{\pgfqpoint{0.017500in}{0.017500in}}{\pgfqpoint{6.218282in}{3.580372in}} %
\pgfusepath{clip}%
\pgfsetbuttcap%
\pgfsetroundjoin%
\definecolor{currentfill}{rgb}{0.805723,0.259813,0.230562}%
\pgfsetfillcolor{currentfill}%
\pgfsetlinewidth{0.000000pt}%
\definecolor{currentstroke}{rgb}{0.000000,0.000000,0.000000}%
\pgfsetstrokecolor{currentstroke}%
\pgfsetdash{}{0pt}%
\pgfpathmoveto{\pgfqpoint{2.410872in}{2.508083in}}%
\pgfpathlineto{\pgfqpoint{2.855513in}{2.451418in}}%
\pgfpathlineto{\pgfqpoint{3.116513in}{2.603812in}}%
\pgfpathlineto{\pgfqpoint{2.674254in}{2.688576in}}%
\pgfpathlineto{\pgfqpoint{2.410872in}{2.508083in}}%
\pgfpathclose%
\pgfusepath{fill}%
\end{pgfscope}%
\begin{pgfscope}%
\pgfpathrectangle{\pgfqpoint{0.017500in}{0.017500in}}{\pgfqpoint{6.218282in}{3.580372in}} %
\pgfusepath{clip}%
\pgfsetbuttcap%
\pgfsetroundjoin%
\definecolor{currentfill}{rgb}{0.805723,0.259813,0.230562}%
\pgfsetfillcolor{currentfill}%
\pgfsetlinewidth{0.000000pt}%
\definecolor{currentstroke}{rgb}{0.000000,0.000000,0.000000}%
\pgfsetstrokecolor{currentstroke}%
\pgfsetdash{}{0pt}%
\pgfpathmoveto{\pgfqpoint{2.410872in}{2.508083in}}%
\pgfpathlineto{\pgfqpoint{2.855513in}{2.451418in}}%
\pgfpathlineto{\pgfqpoint{3.116513in}{2.603812in}}%
\pgfpathlineto{\pgfqpoint{2.674254in}{2.688576in}}%
\pgfpathlineto{\pgfqpoint{2.410872in}{2.508083in}}%
\pgfpathclose%
\pgfusepath{fill}%
\end{pgfscope}%
\begin{pgfscope}%
\pgfpathrectangle{\pgfqpoint{0.017500in}{0.017500in}}{\pgfqpoint{6.218282in}{3.580372in}} %
\pgfusepath{clip}%
\pgfsetbuttcap%
\pgfsetroundjoin%
\definecolor{currentfill}{rgb}{0.229806,0.298718,0.753683}%
\pgfsetfillcolor{currentfill}%
\pgfsetlinewidth{0.000000pt}%
\definecolor{currentstroke}{rgb}{0.000000,0.000000,0.000000}%
\pgfsetstrokecolor{currentstroke}%
\pgfsetdash{}{0pt}%
\pgfpathmoveto{\pgfqpoint{2.349853in}{0.872712in}}%
\pgfpathlineto{\pgfqpoint{2.590432in}{2.210709in}}%
\pgfpathlineto{\pgfqpoint{2.349853in}{0.872712in}}%
\pgfpathclose%
\pgfusepath{fill}%
\end{pgfscope}%
\begin{pgfscope}%
\pgfpathrectangle{\pgfqpoint{0.017500in}{0.017500in}}{\pgfqpoint{6.218282in}{3.580372in}} %
\pgfusepath{clip}%
\pgfsetbuttcap%
\pgfsetroundjoin%
\definecolor{currentfill}{rgb}{0.229806,0.298718,0.753683}%
\pgfsetfillcolor{currentfill}%
\pgfsetlinewidth{0.000000pt}%
\definecolor{currentstroke}{rgb}{0.000000,0.000000,0.000000}%
\pgfsetstrokecolor{currentstroke}%
\pgfsetdash{}{0pt}%
\pgfpathmoveto{\pgfqpoint{2.349853in}{0.872712in}}%
\pgfpathlineto{\pgfqpoint{2.590432in}{2.210709in}}%
\pgfpathlineto{\pgfqpoint{2.349853in}{0.872712in}}%
\pgfpathclose%
\pgfusepath{fill}%
\end{pgfscope}%
\begin{pgfscope}%
\pgfpathrectangle{\pgfqpoint{0.017500in}{0.017500in}}{\pgfqpoint{6.218282in}{3.580372in}} %
\pgfusepath{clip}%
\pgfsetbuttcap%
\pgfsetroundjoin%
\definecolor{currentfill}{rgb}{0.229806,0.298718,0.753683}%
\pgfsetfillcolor{currentfill}%
\pgfsetlinewidth{0.000000pt}%
\definecolor{currentstroke}{rgb}{0.000000,0.000000,0.000000}%
\pgfsetstrokecolor{currentstroke}%
\pgfsetdash{}{0pt}%
\pgfpathmoveto{\pgfqpoint{2.349853in}{0.872712in}}%
\pgfpathlineto{\pgfqpoint{2.590432in}{2.210709in}}%
\pgfpathlineto{\pgfqpoint{2.349853in}{0.872712in}}%
\pgfpathclose%
\pgfusepath{fill}%
\end{pgfscope}%
\begin{pgfscope}%
\pgfpathrectangle{\pgfqpoint{0.017500in}{0.017500in}}{\pgfqpoint{6.218282in}{3.580372in}} %
\pgfusepath{clip}%
\pgfsetbuttcap%
\pgfsetroundjoin%
\definecolor{currentfill}{rgb}{0.229806,0.298718,0.753683}%
\pgfsetfillcolor{currentfill}%
\pgfsetlinewidth{0.000000pt}%
\definecolor{currentstroke}{rgb}{0.000000,0.000000,0.000000}%
\pgfsetstrokecolor{currentstroke}%
\pgfsetdash{}{0pt}%
\pgfpathmoveto{\pgfqpoint{2.349853in}{0.872712in}}%
\pgfpathlineto{\pgfqpoint{2.590432in}{2.210709in}}%
\pgfpathlineto{\pgfqpoint{2.349853in}{0.872712in}}%
\pgfpathclose%
\pgfusepath{fill}%
\end{pgfscope}%
\begin{pgfscope}%
\pgfpathrectangle{\pgfqpoint{0.017500in}{0.017500in}}{\pgfqpoint{6.218282in}{3.580372in}} %
\pgfusepath{clip}%
\pgfsetbuttcap%
\pgfsetroundjoin%
\definecolor{currentfill}{rgb}{0.229806,0.298718,0.753683}%
\pgfsetfillcolor{currentfill}%
\pgfsetlinewidth{0.000000pt}%
\definecolor{currentstroke}{rgb}{0.000000,0.000000,0.000000}%
\pgfsetstrokecolor{currentstroke}%
\pgfsetdash{}{0pt}%
\pgfpathmoveto{\pgfqpoint{2.349853in}{0.872712in}}%
\pgfpathlineto{\pgfqpoint{2.590432in}{2.210709in}}%
\pgfpathlineto{\pgfqpoint{2.349853in}{0.872712in}}%
\pgfpathclose%
\pgfusepath{fill}%
\end{pgfscope}%
\begin{pgfscope}%
\pgfpathrectangle{\pgfqpoint{0.017500in}{0.017500in}}{\pgfqpoint{6.218282in}{3.580372in}} %
\pgfusepath{clip}%
\pgfsetbuttcap%
\pgfsetroundjoin%
\definecolor{currentfill}{rgb}{0.229806,0.298718,0.753683}%
\pgfsetfillcolor{currentfill}%
\pgfsetlinewidth{0.000000pt}%
\definecolor{currentstroke}{rgb}{0.000000,0.000000,0.000000}%
\pgfsetstrokecolor{currentstroke}%
\pgfsetdash{}{0pt}%
\pgfpathmoveto{\pgfqpoint{2.349853in}{0.872712in}}%
\pgfpathlineto{\pgfqpoint{2.590432in}{2.210709in}}%
\pgfpathlineto{\pgfqpoint{2.349853in}{0.872712in}}%
\pgfpathclose%
\pgfusepath{fill}%
\end{pgfscope}%
\begin{pgfscope}%
\pgfpathrectangle{\pgfqpoint{0.017500in}{0.017500in}}{\pgfqpoint{6.218282in}{3.580372in}} %
\pgfusepath{clip}%
\pgfsetbuttcap%
\pgfsetroundjoin%
\definecolor{currentfill}{rgb}{0.229806,0.298718,0.753683}%
\pgfsetfillcolor{currentfill}%
\pgfsetlinewidth{0.000000pt}%
\definecolor{currentstroke}{rgb}{0.000000,0.000000,0.000000}%
\pgfsetstrokecolor{currentstroke}%
\pgfsetdash{}{0pt}%
\pgfpathmoveto{\pgfqpoint{2.349853in}{0.872712in}}%
\pgfpathlineto{\pgfqpoint{2.590432in}{2.210709in}}%
\pgfpathlineto{\pgfqpoint{2.349853in}{0.872712in}}%
\pgfpathclose%
\pgfusepath{fill}%
\end{pgfscope}%
\begin{pgfscope}%
\pgfpathrectangle{\pgfqpoint{0.017500in}{0.017500in}}{\pgfqpoint{6.218282in}{3.580372in}} %
\pgfusepath{clip}%
\pgfsetbuttcap%
\pgfsetroundjoin%
\definecolor{currentfill}{rgb}{0.229806,0.298718,0.753683}%
\pgfsetfillcolor{currentfill}%
\pgfsetlinewidth{0.000000pt}%
\definecolor{currentstroke}{rgb}{0.000000,0.000000,0.000000}%
\pgfsetstrokecolor{currentstroke}%
\pgfsetdash{}{0pt}%
\pgfpathmoveto{\pgfqpoint{2.349853in}{0.872712in}}%
\pgfpathlineto{\pgfqpoint{2.590432in}{2.210709in}}%
\pgfpathlineto{\pgfqpoint{2.349853in}{0.872712in}}%
\pgfpathclose%
\pgfusepath{fill}%
\end{pgfscope}%
\begin{pgfscope}%
\pgfpathrectangle{\pgfqpoint{0.017500in}{0.017500in}}{\pgfqpoint{6.218282in}{3.580372in}} %
\pgfusepath{clip}%
\pgfsetbuttcap%
\pgfsetroundjoin%
\definecolor{currentfill}{rgb}{0.229806,0.298718,0.753683}%
\pgfsetfillcolor{currentfill}%
\pgfsetlinewidth{0.000000pt}%
\definecolor{currentstroke}{rgb}{0.000000,0.000000,0.000000}%
\pgfsetstrokecolor{currentstroke}%
\pgfsetdash{}{0pt}%
\pgfpathmoveto{\pgfqpoint{2.349853in}{0.872712in}}%
\pgfpathlineto{\pgfqpoint{2.590432in}{2.210709in}}%
\pgfpathlineto{\pgfqpoint{2.349853in}{0.872712in}}%
\pgfpathclose%
\pgfusepath{fill}%
\end{pgfscope}%
\begin{pgfscope}%
\pgfpathrectangle{\pgfqpoint{0.017500in}{0.017500in}}{\pgfqpoint{6.218282in}{3.580372in}} %
\pgfusepath{clip}%
\pgfsetbuttcap%
\pgfsetroundjoin%
\definecolor{currentfill}{rgb}{0.229806,0.298718,0.753683}%
\pgfsetfillcolor{currentfill}%
\pgfsetlinewidth{0.000000pt}%
\definecolor{currentstroke}{rgb}{0.000000,0.000000,0.000000}%
\pgfsetstrokecolor{currentstroke}%
\pgfsetdash{}{0pt}%
\pgfpathmoveto{\pgfqpoint{2.349853in}{0.872712in}}%
\pgfpathlineto{\pgfqpoint{2.590432in}{2.210709in}}%
\pgfpathlineto{\pgfqpoint{2.349853in}{0.872712in}}%
\pgfpathclose%
\pgfusepath{fill}%
\end{pgfscope}%
\begin{pgfscope}%
\pgfpathrectangle{\pgfqpoint{0.017500in}{0.017500in}}{\pgfqpoint{6.218282in}{3.580372in}} %
\pgfusepath{clip}%
\pgfsetbuttcap%
\pgfsetroundjoin%
\definecolor{currentfill}{rgb}{0.229806,0.298718,0.753683}%
\pgfsetfillcolor{currentfill}%
\pgfsetlinewidth{0.000000pt}%
\definecolor{currentstroke}{rgb}{0.000000,0.000000,0.000000}%
\pgfsetstrokecolor{currentstroke}%
\pgfsetdash{}{0pt}%
\pgfpathmoveto{\pgfqpoint{2.349853in}{0.872712in}}%
\pgfpathlineto{\pgfqpoint{2.590432in}{2.210709in}}%
\pgfpathlineto{\pgfqpoint{2.349853in}{0.872712in}}%
\pgfpathclose%
\pgfusepath{fill}%
\end{pgfscope}%
\begin{pgfscope}%
\pgfpathrectangle{\pgfqpoint{0.017500in}{0.017500in}}{\pgfqpoint{6.218282in}{3.580372in}} %
\pgfusepath{clip}%
\pgfsetbuttcap%
\pgfsetroundjoin%
\definecolor{currentfill}{rgb}{0.229806,0.298718,0.753683}%
\pgfsetfillcolor{currentfill}%
\pgfsetlinewidth{0.000000pt}%
\definecolor{currentstroke}{rgb}{0.000000,0.000000,0.000000}%
\pgfsetstrokecolor{currentstroke}%
\pgfsetdash{}{0pt}%
\pgfpathmoveto{\pgfqpoint{2.349853in}{0.872712in}}%
\pgfpathlineto{\pgfqpoint{2.590432in}{2.210709in}}%
\pgfpathlineto{\pgfqpoint{2.349853in}{0.872712in}}%
\pgfpathclose%
\pgfusepath{fill}%
\end{pgfscope}%
\begin{pgfscope}%
\pgfpathrectangle{\pgfqpoint{0.017500in}{0.017500in}}{\pgfqpoint{6.218282in}{3.580372in}} %
\pgfusepath{clip}%
\pgfsetbuttcap%
\pgfsetroundjoin%
\definecolor{currentfill}{rgb}{0.229806,0.298718,0.753683}%
\pgfsetfillcolor{currentfill}%
\pgfsetlinewidth{0.000000pt}%
\definecolor{currentstroke}{rgb}{0.000000,0.000000,0.000000}%
\pgfsetstrokecolor{currentstroke}%
\pgfsetdash{}{0pt}%
\pgfpathmoveto{\pgfqpoint{2.349853in}{0.872712in}}%
\pgfpathlineto{\pgfqpoint{2.590432in}{2.210709in}}%
\pgfpathlineto{\pgfqpoint{2.349853in}{0.872712in}}%
\pgfpathclose%
\pgfusepath{fill}%
\end{pgfscope}%
\begin{pgfscope}%
\pgfpathrectangle{\pgfqpoint{0.017500in}{0.017500in}}{\pgfqpoint{6.218282in}{3.580372in}} %
\pgfusepath{clip}%
\pgfsetbuttcap%
\pgfsetroundjoin%
\definecolor{currentfill}{rgb}{0.229806,0.298718,0.753683}%
\pgfsetfillcolor{currentfill}%
\pgfsetlinewidth{0.000000pt}%
\definecolor{currentstroke}{rgb}{0.000000,0.000000,0.000000}%
\pgfsetstrokecolor{currentstroke}%
\pgfsetdash{}{0pt}%
\pgfpathmoveto{\pgfqpoint{2.349853in}{0.872712in}}%
\pgfpathlineto{\pgfqpoint{2.590432in}{2.210709in}}%
\pgfpathlineto{\pgfqpoint{2.349853in}{0.872712in}}%
\pgfpathclose%
\pgfusepath{fill}%
\end{pgfscope}%
\begin{pgfscope}%
\pgfpathrectangle{\pgfqpoint{0.017500in}{0.017500in}}{\pgfqpoint{6.218282in}{3.580372in}} %
\pgfusepath{clip}%
\pgfsetbuttcap%
\pgfsetroundjoin%
\definecolor{currentfill}{rgb}{0.229806,0.298718,0.753683}%
\pgfsetfillcolor{currentfill}%
\pgfsetlinewidth{0.000000pt}%
\definecolor{currentstroke}{rgb}{0.000000,0.000000,0.000000}%
\pgfsetstrokecolor{currentstroke}%
\pgfsetdash{}{0pt}%
\pgfpathmoveto{\pgfqpoint{2.349853in}{0.872712in}}%
\pgfpathlineto{\pgfqpoint{2.590432in}{2.210709in}}%
\pgfpathlineto{\pgfqpoint{2.349853in}{0.872712in}}%
\pgfpathclose%
\pgfusepath{fill}%
\end{pgfscope}%
\begin{pgfscope}%
\pgfpathrectangle{\pgfqpoint{0.017500in}{0.017500in}}{\pgfqpoint{6.218282in}{3.580372in}} %
\pgfusepath{clip}%
\pgfsetbuttcap%
\pgfsetroundjoin%
\definecolor{currentfill}{rgb}{0.229806,0.298718,0.753683}%
\pgfsetfillcolor{currentfill}%
\pgfsetlinewidth{0.000000pt}%
\definecolor{currentstroke}{rgb}{0.000000,0.000000,0.000000}%
\pgfsetstrokecolor{currentstroke}%
\pgfsetdash{}{0pt}%
\pgfpathmoveto{\pgfqpoint{2.349853in}{0.872712in}}%
\pgfpathlineto{\pgfqpoint{2.590432in}{2.210709in}}%
\pgfpathlineto{\pgfqpoint{2.349853in}{0.872712in}}%
\pgfpathclose%
\pgfusepath{fill}%
\end{pgfscope}%
\begin{pgfscope}%
\pgfpathrectangle{\pgfqpoint{0.017500in}{0.017500in}}{\pgfqpoint{6.218282in}{3.580372in}} %
\pgfusepath{clip}%
\pgfsetbuttcap%
\pgfsetroundjoin%
\definecolor{currentfill}{rgb}{0.229806,0.298718,0.753683}%
\pgfsetfillcolor{currentfill}%
\pgfsetlinewidth{0.000000pt}%
\definecolor{currentstroke}{rgb}{0.000000,0.000000,0.000000}%
\pgfsetstrokecolor{currentstroke}%
\pgfsetdash{}{0pt}%
\pgfpathmoveto{\pgfqpoint{2.349853in}{0.872712in}}%
\pgfpathlineto{\pgfqpoint{2.590432in}{2.210709in}}%
\pgfpathlineto{\pgfqpoint{2.349853in}{0.872712in}}%
\pgfpathclose%
\pgfusepath{fill}%
\end{pgfscope}%
\begin{pgfscope}%
\pgfpathrectangle{\pgfqpoint{0.017500in}{0.017500in}}{\pgfqpoint{6.218282in}{3.580372in}} %
\pgfusepath{clip}%
\pgfsetbuttcap%
\pgfsetroundjoin%
\definecolor{currentfill}{rgb}{0.229806,0.298718,0.753683}%
\pgfsetfillcolor{currentfill}%
\pgfsetlinewidth{0.000000pt}%
\definecolor{currentstroke}{rgb}{0.000000,0.000000,0.000000}%
\pgfsetstrokecolor{currentstroke}%
\pgfsetdash{}{0pt}%
\pgfpathmoveto{\pgfqpoint{2.349853in}{0.872712in}}%
\pgfpathlineto{\pgfqpoint{2.590432in}{2.210709in}}%
\pgfpathlineto{\pgfqpoint{2.349853in}{0.872712in}}%
\pgfpathclose%
\pgfusepath{fill}%
\end{pgfscope}%
\begin{pgfscope}%
\pgfpathrectangle{\pgfqpoint{0.017500in}{0.017500in}}{\pgfqpoint{6.218282in}{3.580372in}} %
\pgfusepath{clip}%
\pgfsetbuttcap%
\pgfsetroundjoin%
\definecolor{currentfill}{rgb}{0.229806,0.298718,0.753683}%
\pgfsetfillcolor{currentfill}%
\pgfsetlinewidth{0.000000pt}%
\definecolor{currentstroke}{rgb}{0.000000,0.000000,0.000000}%
\pgfsetstrokecolor{currentstroke}%
\pgfsetdash{}{0pt}%
\pgfpathmoveto{\pgfqpoint{2.349853in}{0.872712in}}%
\pgfpathlineto{\pgfqpoint{2.590432in}{2.210709in}}%
\pgfpathlineto{\pgfqpoint{2.349853in}{0.872712in}}%
\pgfpathclose%
\pgfusepath{fill}%
\end{pgfscope}%
\begin{pgfscope}%
\pgfpathrectangle{\pgfqpoint{0.017500in}{0.017500in}}{\pgfqpoint{6.218282in}{3.580372in}} %
\pgfusepath{clip}%
\pgfsetbuttcap%
\pgfsetroundjoin%
\definecolor{currentfill}{rgb}{0.229806,0.298718,0.753683}%
\pgfsetfillcolor{currentfill}%
\pgfsetlinewidth{0.000000pt}%
\definecolor{currentstroke}{rgb}{0.000000,0.000000,0.000000}%
\pgfsetstrokecolor{currentstroke}%
\pgfsetdash{}{0pt}%
\pgfpathmoveto{\pgfqpoint{2.349853in}{0.872712in}}%
\pgfpathlineto{\pgfqpoint{2.590432in}{2.210709in}}%
\pgfpathlineto{\pgfqpoint{2.349853in}{0.872712in}}%
\pgfpathclose%
\pgfusepath{fill}%
\end{pgfscope}%
\begin{pgfscope}%
\pgfpathrectangle{\pgfqpoint{0.017500in}{0.017500in}}{\pgfqpoint{6.218282in}{3.580372in}} %
\pgfusepath{clip}%
\pgfsetbuttcap%
\pgfsetroundjoin%
\definecolor{currentfill}{rgb}{0.229806,0.298718,0.753683}%
\pgfsetfillcolor{currentfill}%
\pgfsetlinewidth{0.000000pt}%
\definecolor{currentstroke}{rgb}{0.000000,0.000000,0.000000}%
\pgfsetstrokecolor{currentstroke}%
\pgfsetdash{}{0pt}%
\pgfpathmoveto{\pgfqpoint{2.349853in}{0.872712in}}%
\pgfpathlineto{\pgfqpoint{2.590432in}{2.210709in}}%
\pgfpathlineto{\pgfqpoint{2.349853in}{0.872712in}}%
\pgfpathclose%
\pgfusepath{fill}%
\end{pgfscope}%
\begin{pgfscope}%
\pgfpathrectangle{\pgfqpoint{0.017500in}{0.017500in}}{\pgfqpoint{6.218282in}{3.580372in}} %
\pgfusepath{clip}%
\pgfsetbuttcap%
\pgfsetroundjoin%
\definecolor{currentfill}{rgb}{0.229806,0.298718,0.753683}%
\pgfsetfillcolor{currentfill}%
\pgfsetlinewidth{0.000000pt}%
\definecolor{currentstroke}{rgb}{0.000000,0.000000,0.000000}%
\pgfsetstrokecolor{currentstroke}%
\pgfsetdash{}{0pt}%
\pgfpathmoveto{\pgfqpoint{2.349853in}{0.872712in}}%
\pgfpathlineto{\pgfqpoint{2.590432in}{2.210709in}}%
\pgfpathlineto{\pgfqpoint{2.349853in}{0.872712in}}%
\pgfpathclose%
\pgfusepath{fill}%
\end{pgfscope}%
\begin{pgfscope}%
\pgfpathrectangle{\pgfqpoint{0.017500in}{0.017500in}}{\pgfqpoint{6.218282in}{3.580372in}} %
\pgfusepath{clip}%
\pgfsetbuttcap%
\pgfsetroundjoin%
\definecolor{currentfill}{rgb}{0.229806,0.298718,0.753683}%
\pgfsetfillcolor{currentfill}%
\pgfsetlinewidth{0.000000pt}%
\definecolor{currentstroke}{rgb}{0.000000,0.000000,0.000000}%
\pgfsetstrokecolor{currentstroke}%
\pgfsetdash{}{0pt}%
\pgfpathmoveto{\pgfqpoint{2.349853in}{0.872712in}}%
\pgfpathlineto{\pgfqpoint{2.590432in}{2.210709in}}%
\pgfpathlineto{\pgfqpoint{2.349853in}{0.872712in}}%
\pgfpathclose%
\pgfusepath{fill}%
\end{pgfscope}%
\begin{pgfscope}%
\pgfpathrectangle{\pgfqpoint{0.017500in}{0.017500in}}{\pgfqpoint{6.218282in}{3.580372in}} %
\pgfusepath{clip}%
\pgfsetbuttcap%
\pgfsetroundjoin%
\definecolor{currentfill}{rgb}{0.229806,0.298718,0.753683}%
\pgfsetfillcolor{currentfill}%
\pgfsetlinewidth{0.000000pt}%
\definecolor{currentstroke}{rgb}{0.000000,0.000000,0.000000}%
\pgfsetstrokecolor{currentstroke}%
\pgfsetdash{}{0pt}%
\pgfpathmoveto{\pgfqpoint{2.349853in}{0.872712in}}%
\pgfpathlineto{\pgfqpoint{2.590432in}{2.210709in}}%
\pgfpathlineto{\pgfqpoint{2.349853in}{0.872712in}}%
\pgfpathclose%
\pgfusepath{fill}%
\end{pgfscope}%
\begin{pgfscope}%
\pgfpathrectangle{\pgfqpoint{0.017500in}{0.017500in}}{\pgfqpoint{6.218282in}{3.580372in}} %
\pgfusepath{clip}%
\pgfsetbuttcap%
\pgfsetroundjoin%
\definecolor{currentfill}{rgb}{0.229806,0.298718,0.753683}%
\pgfsetfillcolor{currentfill}%
\pgfsetlinewidth{0.000000pt}%
\definecolor{currentstroke}{rgb}{0.000000,0.000000,0.000000}%
\pgfsetstrokecolor{currentstroke}%
\pgfsetdash{}{0pt}%
\pgfpathmoveto{\pgfqpoint{2.349853in}{0.872712in}}%
\pgfpathlineto{\pgfqpoint{2.590432in}{2.210709in}}%
\pgfpathlineto{\pgfqpoint{2.349853in}{0.872712in}}%
\pgfpathclose%
\pgfusepath{fill}%
\end{pgfscope}%
\begin{pgfscope}%
\pgfpathrectangle{\pgfqpoint{0.017500in}{0.017500in}}{\pgfqpoint{6.218282in}{3.580372in}} %
\pgfusepath{clip}%
\pgfsetbuttcap%
\pgfsetroundjoin%
\definecolor{currentfill}{rgb}{0.229806,0.298718,0.753683}%
\pgfsetfillcolor{currentfill}%
\pgfsetlinewidth{0.000000pt}%
\definecolor{currentstroke}{rgb}{0.000000,0.000000,0.000000}%
\pgfsetstrokecolor{currentstroke}%
\pgfsetdash{}{0pt}%
\pgfpathmoveto{\pgfqpoint{2.349853in}{0.872712in}}%
\pgfpathlineto{\pgfqpoint{2.590432in}{2.210709in}}%
\pgfpathlineto{\pgfqpoint{2.349853in}{0.872712in}}%
\pgfpathclose%
\pgfusepath{fill}%
\end{pgfscope}%
\begin{pgfscope}%
\pgfpathrectangle{\pgfqpoint{0.017500in}{0.017500in}}{\pgfqpoint{6.218282in}{3.580372in}} %
\pgfusepath{clip}%
\pgfsetbuttcap%
\pgfsetroundjoin%
\definecolor{currentfill}{rgb}{0.229806,0.298718,0.753683}%
\pgfsetfillcolor{currentfill}%
\pgfsetlinewidth{0.000000pt}%
\definecolor{currentstroke}{rgb}{0.000000,0.000000,0.000000}%
\pgfsetstrokecolor{currentstroke}%
\pgfsetdash{}{0pt}%
\pgfpathmoveto{\pgfqpoint{2.349853in}{0.872712in}}%
\pgfpathlineto{\pgfqpoint{2.590432in}{2.210709in}}%
\pgfpathlineto{\pgfqpoint{2.349853in}{0.872712in}}%
\pgfpathclose%
\pgfusepath{fill}%
\end{pgfscope}%
\begin{pgfscope}%
\pgfpathrectangle{\pgfqpoint{0.017500in}{0.017500in}}{\pgfqpoint{6.218282in}{3.580372in}} %
\pgfusepath{clip}%
\pgfsetbuttcap%
\pgfsetroundjoin%
\definecolor{currentfill}{rgb}{0.229806,0.298718,0.753683}%
\pgfsetfillcolor{currentfill}%
\pgfsetlinewidth{0.000000pt}%
\definecolor{currentstroke}{rgb}{0.000000,0.000000,0.000000}%
\pgfsetstrokecolor{currentstroke}%
\pgfsetdash{}{0pt}%
\pgfpathmoveto{\pgfqpoint{2.349853in}{0.872712in}}%
\pgfpathlineto{\pgfqpoint{2.590432in}{2.210709in}}%
\pgfpathlineto{\pgfqpoint{2.349853in}{0.872712in}}%
\pgfpathclose%
\pgfusepath{fill}%
\end{pgfscope}%
\begin{pgfscope}%
\pgfpathrectangle{\pgfqpoint{0.017500in}{0.017500in}}{\pgfqpoint{6.218282in}{3.580372in}} %
\pgfusepath{clip}%
\pgfsetbuttcap%
\pgfsetroundjoin%
\definecolor{currentfill}{rgb}{0.229806,0.298718,0.753683}%
\pgfsetfillcolor{currentfill}%
\pgfsetlinewidth{0.000000pt}%
\definecolor{currentstroke}{rgb}{0.000000,0.000000,0.000000}%
\pgfsetstrokecolor{currentstroke}%
\pgfsetdash{}{0pt}%
\pgfpathmoveto{\pgfqpoint{2.349853in}{0.872712in}}%
\pgfpathlineto{\pgfqpoint{2.590432in}{2.210709in}}%
\pgfpathlineto{\pgfqpoint{2.349853in}{0.872712in}}%
\pgfpathclose%
\pgfusepath{fill}%
\end{pgfscope}%
\begin{pgfscope}%
\pgfpathrectangle{\pgfqpoint{0.017500in}{0.017500in}}{\pgfqpoint{6.218282in}{3.580372in}} %
\pgfusepath{clip}%
\pgfsetbuttcap%
\pgfsetroundjoin%
\definecolor{currentfill}{rgb}{0.229806,0.298718,0.753683}%
\pgfsetfillcolor{currentfill}%
\pgfsetlinewidth{0.000000pt}%
\definecolor{currentstroke}{rgb}{0.000000,0.000000,0.000000}%
\pgfsetstrokecolor{currentstroke}%
\pgfsetdash{}{0pt}%
\pgfpathmoveto{\pgfqpoint{2.349853in}{0.872712in}}%
\pgfpathlineto{\pgfqpoint{2.590432in}{2.210709in}}%
\pgfpathlineto{\pgfqpoint{2.349853in}{0.872712in}}%
\pgfpathclose%
\pgfusepath{fill}%
\end{pgfscope}%
\begin{pgfscope}%
\pgfpathrectangle{\pgfqpoint{0.017500in}{0.017500in}}{\pgfqpoint{6.218282in}{3.580372in}} %
\pgfusepath{clip}%
\pgfsetbuttcap%
\pgfsetroundjoin%
\definecolor{currentfill}{rgb}{0.888390,0.417703,0.327898}%
\pgfsetfillcolor{currentfill}%
\pgfsetlinewidth{0.000000pt}%
\definecolor{currentstroke}{rgb}{0.000000,0.000000,0.000000}%
\pgfsetstrokecolor{currentstroke}%
\pgfsetdash{}{0pt}%
\pgfpathmoveto{\pgfqpoint{1.701778in}{2.375902in}}%
\pgfpathlineto{\pgfqpoint{1.971633in}{2.586781in}}%
\pgfpathlineto{\pgfqpoint{1.701778in}{2.375902in}}%
\pgfpathclose%
\pgfusepath{fill}%
\end{pgfscope}%
\begin{pgfscope}%
\pgfpathrectangle{\pgfqpoint{0.017500in}{0.017500in}}{\pgfqpoint{6.218282in}{3.580372in}} %
\pgfusepath{clip}%
\pgfsetbuttcap%
\pgfsetroundjoin%
\definecolor{currentfill}{rgb}{0.888390,0.417703,0.327898}%
\pgfsetfillcolor{currentfill}%
\pgfsetlinewidth{0.000000pt}%
\definecolor{currentstroke}{rgb}{0.000000,0.000000,0.000000}%
\pgfsetstrokecolor{currentstroke}%
\pgfsetdash{}{0pt}%
\pgfpathmoveto{\pgfqpoint{1.701778in}{2.375902in}}%
\pgfpathlineto{\pgfqpoint{1.971633in}{2.586781in}}%
\pgfpathlineto{\pgfqpoint{1.701778in}{2.375902in}}%
\pgfpathclose%
\pgfusepath{fill}%
\end{pgfscope}%
\begin{pgfscope}%
\pgfpathrectangle{\pgfqpoint{0.017500in}{0.017500in}}{\pgfqpoint{6.218282in}{3.580372in}} %
\pgfusepath{clip}%
\pgfsetbuttcap%
\pgfsetroundjoin%
\definecolor{currentfill}{rgb}{0.888390,0.417703,0.327898}%
\pgfsetfillcolor{currentfill}%
\pgfsetlinewidth{0.000000pt}%
\definecolor{currentstroke}{rgb}{0.000000,0.000000,0.000000}%
\pgfsetstrokecolor{currentstroke}%
\pgfsetdash{}{0pt}%
\pgfpathmoveto{\pgfqpoint{1.701778in}{2.375902in}}%
\pgfpathlineto{\pgfqpoint{1.971633in}{2.586781in}}%
\pgfpathlineto{\pgfqpoint{1.701778in}{2.375902in}}%
\pgfpathclose%
\pgfusepath{fill}%
\end{pgfscope}%
\begin{pgfscope}%
\pgfpathrectangle{\pgfqpoint{0.017500in}{0.017500in}}{\pgfqpoint{6.218282in}{3.580372in}} %
\pgfusepath{clip}%
\pgfsetbuttcap%
\pgfsetroundjoin%
\definecolor{currentfill}{rgb}{0.888390,0.417703,0.327898}%
\pgfsetfillcolor{currentfill}%
\pgfsetlinewidth{0.000000pt}%
\definecolor{currentstroke}{rgb}{0.000000,0.000000,0.000000}%
\pgfsetstrokecolor{currentstroke}%
\pgfsetdash{}{0pt}%
\pgfpathmoveto{\pgfqpoint{1.701778in}{2.375902in}}%
\pgfpathlineto{\pgfqpoint{1.971633in}{2.586781in}}%
\pgfpathlineto{\pgfqpoint{1.701778in}{2.375902in}}%
\pgfpathclose%
\pgfusepath{fill}%
\end{pgfscope}%
\begin{pgfscope}%
\pgfpathrectangle{\pgfqpoint{0.017500in}{0.017500in}}{\pgfqpoint{6.218282in}{3.580372in}} %
\pgfusepath{clip}%
\pgfsetbuttcap%
\pgfsetroundjoin%
\definecolor{currentfill}{rgb}{0.888390,0.417703,0.327898}%
\pgfsetfillcolor{currentfill}%
\pgfsetlinewidth{0.000000pt}%
\definecolor{currentstroke}{rgb}{0.000000,0.000000,0.000000}%
\pgfsetstrokecolor{currentstroke}%
\pgfsetdash{}{0pt}%
\pgfpathmoveto{\pgfqpoint{1.701778in}{2.375902in}}%
\pgfpathlineto{\pgfqpoint{1.971633in}{2.586781in}}%
\pgfpathlineto{\pgfqpoint{1.701778in}{2.375902in}}%
\pgfpathclose%
\pgfusepath{fill}%
\end{pgfscope}%
\begin{pgfscope}%
\pgfpathrectangle{\pgfqpoint{0.017500in}{0.017500in}}{\pgfqpoint{6.218282in}{3.580372in}} %
\pgfusepath{clip}%
\pgfsetbuttcap%
\pgfsetroundjoin%
\definecolor{currentfill}{rgb}{0.888390,0.417703,0.327898}%
\pgfsetfillcolor{currentfill}%
\pgfsetlinewidth{0.000000pt}%
\definecolor{currentstroke}{rgb}{0.000000,0.000000,0.000000}%
\pgfsetstrokecolor{currentstroke}%
\pgfsetdash{}{0pt}%
\pgfpathmoveto{\pgfqpoint{1.701778in}{2.375902in}}%
\pgfpathlineto{\pgfqpoint{1.971633in}{2.586781in}}%
\pgfpathlineto{\pgfqpoint{1.701778in}{2.375902in}}%
\pgfpathclose%
\pgfusepath{fill}%
\end{pgfscope}%
\begin{pgfscope}%
\pgfpathrectangle{\pgfqpoint{0.017500in}{0.017500in}}{\pgfqpoint{6.218282in}{3.580372in}} %
\pgfusepath{clip}%
\pgfsetbuttcap%
\pgfsetroundjoin%
\definecolor{currentfill}{rgb}{0.888390,0.417703,0.327898}%
\pgfsetfillcolor{currentfill}%
\pgfsetlinewidth{0.000000pt}%
\definecolor{currentstroke}{rgb}{0.000000,0.000000,0.000000}%
\pgfsetstrokecolor{currentstroke}%
\pgfsetdash{}{0pt}%
\pgfpathmoveto{\pgfqpoint{1.701778in}{2.375902in}}%
\pgfpathlineto{\pgfqpoint{1.971633in}{2.586781in}}%
\pgfpathlineto{\pgfqpoint{1.701778in}{2.375902in}}%
\pgfpathclose%
\pgfusepath{fill}%
\end{pgfscope}%
\begin{pgfscope}%
\pgfpathrectangle{\pgfqpoint{0.017500in}{0.017500in}}{\pgfqpoint{6.218282in}{3.580372in}} %
\pgfusepath{clip}%
\pgfsetbuttcap%
\pgfsetroundjoin%
\definecolor{currentfill}{rgb}{0.888390,0.417703,0.327898}%
\pgfsetfillcolor{currentfill}%
\pgfsetlinewidth{0.000000pt}%
\definecolor{currentstroke}{rgb}{0.000000,0.000000,0.000000}%
\pgfsetstrokecolor{currentstroke}%
\pgfsetdash{}{0pt}%
\pgfpathmoveto{\pgfqpoint{1.701778in}{2.375902in}}%
\pgfpathlineto{\pgfqpoint{1.971633in}{2.586781in}}%
\pgfpathlineto{\pgfqpoint{1.701778in}{2.375902in}}%
\pgfpathclose%
\pgfusepath{fill}%
\end{pgfscope}%
\begin{pgfscope}%
\pgfpathrectangle{\pgfqpoint{0.017500in}{0.017500in}}{\pgfqpoint{6.218282in}{3.580372in}} %
\pgfusepath{clip}%
\pgfsetbuttcap%
\pgfsetroundjoin%
\definecolor{currentfill}{rgb}{0.888390,0.417703,0.327898}%
\pgfsetfillcolor{currentfill}%
\pgfsetlinewidth{0.000000pt}%
\definecolor{currentstroke}{rgb}{0.000000,0.000000,0.000000}%
\pgfsetstrokecolor{currentstroke}%
\pgfsetdash{}{0pt}%
\pgfpathmoveto{\pgfqpoint{1.701778in}{2.375902in}}%
\pgfpathlineto{\pgfqpoint{1.971633in}{2.586781in}}%
\pgfpathlineto{\pgfqpoint{1.701778in}{2.375902in}}%
\pgfpathclose%
\pgfusepath{fill}%
\end{pgfscope}%
\begin{pgfscope}%
\pgfpathrectangle{\pgfqpoint{0.017500in}{0.017500in}}{\pgfqpoint{6.218282in}{3.580372in}} %
\pgfusepath{clip}%
\pgfsetbuttcap%
\pgfsetroundjoin%
\definecolor{currentfill}{rgb}{0.888390,0.417703,0.327898}%
\pgfsetfillcolor{currentfill}%
\pgfsetlinewidth{0.000000pt}%
\definecolor{currentstroke}{rgb}{0.000000,0.000000,0.000000}%
\pgfsetstrokecolor{currentstroke}%
\pgfsetdash{}{0pt}%
\pgfpathmoveto{\pgfqpoint{1.701778in}{2.375902in}}%
\pgfpathlineto{\pgfqpoint{1.971633in}{2.586781in}}%
\pgfpathlineto{\pgfqpoint{1.701778in}{2.375902in}}%
\pgfpathclose%
\pgfusepath{fill}%
\end{pgfscope}%
\begin{pgfscope}%
\pgfpathrectangle{\pgfqpoint{0.017500in}{0.017500in}}{\pgfqpoint{6.218282in}{3.580372in}} %
\pgfusepath{clip}%
\pgfsetbuttcap%
\pgfsetroundjoin%
\definecolor{currentfill}{rgb}{0.888390,0.417703,0.327898}%
\pgfsetfillcolor{currentfill}%
\pgfsetlinewidth{0.000000pt}%
\definecolor{currentstroke}{rgb}{0.000000,0.000000,0.000000}%
\pgfsetstrokecolor{currentstroke}%
\pgfsetdash{}{0pt}%
\pgfpathmoveto{\pgfqpoint{1.701778in}{2.375902in}}%
\pgfpathlineto{\pgfqpoint{1.971633in}{2.586781in}}%
\pgfpathlineto{\pgfqpoint{1.701778in}{2.375902in}}%
\pgfpathclose%
\pgfusepath{fill}%
\end{pgfscope}%
\begin{pgfscope}%
\pgfpathrectangle{\pgfqpoint{0.017500in}{0.017500in}}{\pgfqpoint{6.218282in}{3.580372in}} %
\pgfusepath{clip}%
\pgfsetbuttcap%
\pgfsetroundjoin%
\definecolor{currentfill}{rgb}{0.888390,0.417703,0.327898}%
\pgfsetfillcolor{currentfill}%
\pgfsetlinewidth{0.000000pt}%
\definecolor{currentstroke}{rgb}{0.000000,0.000000,0.000000}%
\pgfsetstrokecolor{currentstroke}%
\pgfsetdash{}{0pt}%
\pgfpathmoveto{\pgfqpoint{1.701778in}{2.375902in}}%
\pgfpathlineto{\pgfqpoint{1.971633in}{2.586781in}}%
\pgfpathlineto{\pgfqpoint{1.701778in}{2.375902in}}%
\pgfpathclose%
\pgfusepath{fill}%
\end{pgfscope}%
\begin{pgfscope}%
\pgfpathrectangle{\pgfqpoint{0.017500in}{0.017500in}}{\pgfqpoint{6.218282in}{3.580372in}} %
\pgfusepath{clip}%
\pgfsetbuttcap%
\pgfsetroundjoin%
\definecolor{currentfill}{rgb}{0.888390,0.417703,0.327898}%
\pgfsetfillcolor{currentfill}%
\pgfsetlinewidth{0.000000pt}%
\definecolor{currentstroke}{rgb}{0.000000,0.000000,0.000000}%
\pgfsetstrokecolor{currentstroke}%
\pgfsetdash{}{0pt}%
\pgfpathmoveto{\pgfqpoint{1.701778in}{2.375902in}}%
\pgfpathlineto{\pgfqpoint{1.971633in}{2.586781in}}%
\pgfpathlineto{\pgfqpoint{1.701778in}{2.375902in}}%
\pgfpathclose%
\pgfusepath{fill}%
\end{pgfscope}%
\begin{pgfscope}%
\pgfpathrectangle{\pgfqpoint{0.017500in}{0.017500in}}{\pgfqpoint{6.218282in}{3.580372in}} %
\pgfusepath{clip}%
\pgfsetbuttcap%
\pgfsetroundjoin%
\definecolor{currentfill}{rgb}{0.888390,0.417703,0.327898}%
\pgfsetfillcolor{currentfill}%
\pgfsetlinewidth{0.000000pt}%
\definecolor{currentstroke}{rgb}{0.000000,0.000000,0.000000}%
\pgfsetstrokecolor{currentstroke}%
\pgfsetdash{}{0pt}%
\pgfpathmoveto{\pgfqpoint{1.701778in}{2.375902in}}%
\pgfpathlineto{\pgfqpoint{1.971633in}{2.586781in}}%
\pgfpathlineto{\pgfqpoint{1.701778in}{2.375902in}}%
\pgfpathclose%
\pgfusepath{fill}%
\end{pgfscope}%
\begin{pgfscope}%
\pgfpathrectangle{\pgfqpoint{0.017500in}{0.017500in}}{\pgfqpoint{6.218282in}{3.580372in}} %
\pgfusepath{clip}%
\pgfsetbuttcap%
\pgfsetroundjoin%
\definecolor{currentfill}{rgb}{0.888390,0.417703,0.327898}%
\pgfsetfillcolor{currentfill}%
\pgfsetlinewidth{0.000000pt}%
\definecolor{currentstroke}{rgb}{0.000000,0.000000,0.000000}%
\pgfsetstrokecolor{currentstroke}%
\pgfsetdash{}{0pt}%
\pgfpathmoveto{\pgfqpoint{1.701778in}{2.375902in}}%
\pgfpathlineto{\pgfqpoint{1.971633in}{2.586781in}}%
\pgfpathlineto{\pgfqpoint{1.701778in}{2.375902in}}%
\pgfpathclose%
\pgfusepath{fill}%
\end{pgfscope}%
\begin{pgfscope}%
\pgfpathrectangle{\pgfqpoint{0.017500in}{0.017500in}}{\pgfqpoint{6.218282in}{3.580372in}} %
\pgfusepath{clip}%
\pgfsetbuttcap%
\pgfsetroundjoin%
\definecolor{currentfill}{rgb}{0.888390,0.417703,0.327898}%
\pgfsetfillcolor{currentfill}%
\pgfsetlinewidth{0.000000pt}%
\definecolor{currentstroke}{rgb}{0.000000,0.000000,0.000000}%
\pgfsetstrokecolor{currentstroke}%
\pgfsetdash{}{0pt}%
\pgfpathmoveto{\pgfqpoint{1.701778in}{2.375902in}}%
\pgfpathlineto{\pgfqpoint{1.971633in}{2.586781in}}%
\pgfpathlineto{\pgfqpoint{1.701778in}{2.375902in}}%
\pgfpathclose%
\pgfusepath{fill}%
\end{pgfscope}%
\begin{pgfscope}%
\pgfpathrectangle{\pgfqpoint{0.017500in}{0.017500in}}{\pgfqpoint{6.218282in}{3.580372in}} %
\pgfusepath{clip}%
\pgfsetbuttcap%
\pgfsetroundjoin%
\definecolor{currentfill}{rgb}{0.888390,0.417703,0.327898}%
\pgfsetfillcolor{currentfill}%
\pgfsetlinewidth{0.000000pt}%
\definecolor{currentstroke}{rgb}{0.000000,0.000000,0.000000}%
\pgfsetstrokecolor{currentstroke}%
\pgfsetdash{}{0pt}%
\pgfpathmoveto{\pgfqpoint{1.701778in}{2.375902in}}%
\pgfpathlineto{\pgfqpoint{1.971633in}{2.586781in}}%
\pgfpathlineto{\pgfqpoint{1.701778in}{2.375902in}}%
\pgfpathclose%
\pgfusepath{fill}%
\end{pgfscope}%
\begin{pgfscope}%
\pgfpathrectangle{\pgfqpoint{0.017500in}{0.017500in}}{\pgfqpoint{6.218282in}{3.580372in}} %
\pgfusepath{clip}%
\pgfsetbuttcap%
\pgfsetroundjoin%
\definecolor{currentfill}{rgb}{0.888390,0.417703,0.327898}%
\pgfsetfillcolor{currentfill}%
\pgfsetlinewidth{0.000000pt}%
\definecolor{currentstroke}{rgb}{0.000000,0.000000,0.000000}%
\pgfsetstrokecolor{currentstroke}%
\pgfsetdash{}{0pt}%
\pgfpathmoveto{\pgfqpoint{1.701778in}{2.375902in}}%
\pgfpathlineto{\pgfqpoint{1.971633in}{2.586781in}}%
\pgfpathlineto{\pgfqpoint{1.701778in}{2.375902in}}%
\pgfpathclose%
\pgfusepath{fill}%
\end{pgfscope}%
\begin{pgfscope}%
\pgfpathrectangle{\pgfqpoint{0.017500in}{0.017500in}}{\pgfqpoint{6.218282in}{3.580372in}} %
\pgfusepath{clip}%
\pgfsetbuttcap%
\pgfsetroundjoin%
\definecolor{currentfill}{rgb}{0.888390,0.417703,0.327898}%
\pgfsetfillcolor{currentfill}%
\pgfsetlinewidth{0.000000pt}%
\definecolor{currentstroke}{rgb}{0.000000,0.000000,0.000000}%
\pgfsetstrokecolor{currentstroke}%
\pgfsetdash{}{0pt}%
\pgfpathmoveto{\pgfqpoint{1.701778in}{2.375902in}}%
\pgfpathlineto{\pgfqpoint{1.971633in}{2.586781in}}%
\pgfpathlineto{\pgfqpoint{1.701778in}{2.375902in}}%
\pgfpathclose%
\pgfusepath{fill}%
\end{pgfscope}%
\begin{pgfscope}%
\pgfpathrectangle{\pgfqpoint{0.017500in}{0.017500in}}{\pgfqpoint{6.218282in}{3.580372in}} %
\pgfusepath{clip}%
\pgfsetbuttcap%
\pgfsetroundjoin%
\definecolor{currentfill}{rgb}{0.888390,0.417703,0.327898}%
\pgfsetfillcolor{currentfill}%
\pgfsetlinewidth{0.000000pt}%
\definecolor{currentstroke}{rgb}{0.000000,0.000000,0.000000}%
\pgfsetstrokecolor{currentstroke}%
\pgfsetdash{}{0pt}%
\pgfpathmoveto{\pgfqpoint{1.701778in}{2.375902in}}%
\pgfpathlineto{\pgfqpoint{1.971633in}{2.586781in}}%
\pgfpathlineto{\pgfqpoint{1.701778in}{2.375902in}}%
\pgfpathclose%
\pgfusepath{fill}%
\end{pgfscope}%
\begin{pgfscope}%
\pgfpathrectangle{\pgfqpoint{0.017500in}{0.017500in}}{\pgfqpoint{6.218282in}{3.580372in}} %
\pgfusepath{clip}%
\pgfsetbuttcap%
\pgfsetroundjoin%
\definecolor{currentfill}{rgb}{0.888390,0.417703,0.327898}%
\pgfsetfillcolor{currentfill}%
\pgfsetlinewidth{0.000000pt}%
\definecolor{currentstroke}{rgb}{0.000000,0.000000,0.000000}%
\pgfsetstrokecolor{currentstroke}%
\pgfsetdash{}{0pt}%
\pgfpathmoveto{\pgfqpoint{1.701778in}{2.375902in}}%
\pgfpathlineto{\pgfqpoint{1.971633in}{2.586781in}}%
\pgfpathlineto{\pgfqpoint{1.701778in}{2.375902in}}%
\pgfpathclose%
\pgfusepath{fill}%
\end{pgfscope}%
\begin{pgfscope}%
\pgfpathrectangle{\pgfqpoint{0.017500in}{0.017500in}}{\pgfqpoint{6.218282in}{3.580372in}} %
\pgfusepath{clip}%
\pgfsetbuttcap%
\pgfsetroundjoin%
\definecolor{currentfill}{rgb}{0.888390,0.417703,0.327898}%
\pgfsetfillcolor{currentfill}%
\pgfsetlinewidth{0.000000pt}%
\definecolor{currentstroke}{rgb}{0.000000,0.000000,0.000000}%
\pgfsetstrokecolor{currentstroke}%
\pgfsetdash{}{0pt}%
\pgfpathmoveto{\pgfqpoint{1.701778in}{2.375902in}}%
\pgfpathlineto{\pgfqpoint{1.971633in}{2.586781in}}%
\pgfpathlineto{\pgfqpoint{1.701778in}{2.375902in}}%
\pgfpathclose%
\pgfusepath{fill}%
\end{pgfscope}%
\begin{pgfscope}%
\pgfpathrectangle{\pgfqpoint{0.017500in}{0.017500in}}{\pgfqpoint{6.218282in}{3.580372in}} %
\pgfusepath{clip}%
\pgfsetbuttcap%
\pgfsetroundjoin%
\definecolor{currentfill}{rgb}{0.888390,0.417703,0.327898}%
\pgfsetfillcolor{currentfill}%
\pgfsetlinewidth{0.000000pt}%
\definecolor{currentstroke}{rgb}{0.000000,0.000000,0.000000}%
\pgfsetstrokecolor{currentstroke}%
\pgfsetdash{}{0pt}%
\pgfpathmoveto{\pgfqpoint{1.701778in}{2.375902in}}%
\pgfpathlineto{\pgfqpoint{1.971633in}{2.586781in}}%
\pgfpathlineto{\pgfqpoint{1.701778in}{2.375902in}}%
\pgfpathclose%
\pgfusepath{fill}%
\end{pgfscope}%
\begin{pgfscope}%
\pgfpathrectangle{\pgfqpoint{0.017500in}{0.017500in}}{\pgfqpoint{6.218282in}{3.580372in}} %
\pgfusepath{clip}%
\pgfsetbuttcap%
\pgfsetroundjoin%
\definecolor{currentfill}{rgb}{0.888390,0.417703,0.327898}%
\pgfsetfillcolor{currentfill}%
\pgfsetlinewidth{0.000000pt}%
\definecolor{currentstroke}{rgb}{0.000000,0.000000,0.000000}%
\pgfsetstrokecolor{currentstroke}%
\pgfsetdash{}{0pt}%
\pgfpathmoveto{\pgfqpoint{1.701778in}{2.375902in}}%
\pgfpathlineto{\pgfqpoint{1.971633in}{2.586781in}}%
\pgfpathlineto{\pgfqpoint{1.701778in}{2.375902in}}%
\pgfpathclose%
\pgfusepath{fill}%
\end{pgfscope}%
\begin{pgfscope}%
\pgfpathrectangle{\pgfqpoint{0.017500in}{0.017500in}}{\pgfqpoint{6.218282in}{3.580372in}} %
\pgfusepath{clip}%
\pgfsetbuttcap%
\pgfsetroundjoin%
\definecolor{currentfill}{rgb}{0.888390,0.417703,0.327898}%
\pgfsetfillcolor{currentfill}%
\pgfsetlinewidth{0.000000pt}%
\definecolor{currentstroke}{rgb}{0.000000,0.000000,0.000000}%
\pgfsetstrokecolor{currentstroke}%
\pgfsetdash{}{0pt}%
\pgfpathmoveto{\pgfqpoint{1.701778in}{2.375902in}}%
\pgfpathlineto{\pgfqpoint{1.971633in}{2.586781in}}%
\pgfpathlineto{\pgfqpoint{1.701778in}{2.375902in}}%
\pgfpathclose%
\pgfusepath{fill}%
\end{pgfscope}%
\begin{pgfscope}%
\pgfpathrectangle{\pgfqpoint{0.017500in}{0.017500in}}{\pgfqpoint{6.218282in}{3.580372in}} %
\pgfusepath{clip}%
\pgfsetbuttcap%
\pgfsetroundjoin%
\definecolor{currentfill}{rgb}{0.888390,0.417703,0.327898}%
\pgfsetfillcolor{currentfill}%
\pgfsetlinewidth{0.000000pt}%
\definecolor{currentstroke}{rgb}{0.000000,0.000000,0.000000}%
\pgfsetstrokecolor{currentstroke}%
\pgfsetdash{}{0pt}%
\pgfpathmoveto{\pgfqpoint{1.701778in}{2.375902in}}%
\pgfpathlineto{\pgfqpoint{1.971633in}{2.586781in}}%
\pgfpathlineto{\pgfqpoint{1.701778in}{2.375902in}}%
\pgfpathclose%
\pgfusepath{fill}%
\end{pgfscope}%
\begin{pgfscope}%
\pgfpathrectangle{\pgfqpoint{0.017500in}{0.017500in}}{\pgfqpoint{6.218282in}{3.580372in}} %
\pgfusepath{clip}%
\pgfsetbuttcap%
\pgfsetroundjoin%
\definecolor{currentfill}{rgb}{0.888390,0.417703,0.327898}%
\pgfsetfillcolor{currentfill}%
\pgfsetlinewidth{0.000000pt}%
\definecolor{currentstroke}{rgb}{0.000000,0.000000,0.000000}%
\pgfsetstrokecolor{currentstroke}%
\pgfsetdash{}{0pt}%
\pgfpathmoveto{\pgfqpoint{1.701778in}{2.375902in}}%
\pgfpathlineto{\pgfqpoint{1.971633in}{2.586781in}}%
\pgfpathlineto{\pgfqpoint{1.701778in}{2.375902in}}%
\pgfpathclose%
\pgfusepath{fill}%
\end{pgfscope}%
\begin{pgfscope}%
\pgfpathrectangle{\pgfqpoint{0.017500in}{0.017500in}}{\pgfqpoint{6.218282in}{3.580372in}} %
\pgfusepath{clip}%
\pgfsetbuttcap%
\pgfsetroundjoin%
\definecolor{currentfill}{rgb}{0.888390,0.417703,0.327898}%
\pgfsetfillcolor{currentfill}%
\pgfsetlinewidth{0.000000pt}%
\definecolor{currentstroke}{rgb}{0.000000,0.000000,0.000000}%
\pgfsetstrokecolor{currentstroke}%
\pgfsetdash{}{0pt}%
\pgfpathmoveto{\pgfqpoint{1.701778in}{2.375902in}}%
\pgfpathlineto{\pgfqpoint{1.971633in}{2.586781in}}%
\pgfpathlineto{\pgfqpoint{1.701778in}{2.375902in}}%
\pgfpathclose%
\pgfusepath{fill}%
\end{pgfscope}%
\begin{pgfscope}%
\pgfpathrectangle{\pgfqpoint{0.017500in}{0.017500in}}{\pgfqpoint{6.218282in}{3.580372in}} %
\pgfusepath{clip}%
\pgfsetbuttcap%
\pgfsetroundjoin%
\definecolor{currentfill}{rgb}{0.888390,0.417703,0.327898}%
\pgfsetfillcolor{currentfill}%
\pgfsetlinewidth{0.000000pt}%
\definecolor{currentstroke}{rgb}{0.000000,0.000000,0.000000}%
\pgfsetstrokecolor{currentstroke}%
\pgfsetdash{}{0pt}%
\pgfpathmoveto{\pgfqpoint{1.701778in}{2.375902in}}%
\pgfpathlineto{\pgfqpoint{1.971633in}{2.586781in}}%
\pgfpathlineto{\pgfqpoint{1.701778in}{2.375902in}}%
\pgfpathclose%
\pgfusepath{fill}%
\end{pgfscope}%
\begin{pgfscope}%
\pgfpathrectangle{\pgfqpoint{0.017500in}{0.017500in}}{\pgfqpoint{6.218282in}{3.580372in}} %
\pgfusepath{clip}%
\pgfsetbuttcap%
\pgfsetroundjoin%
\definecolor{currentfill}{rgb}{0.888390,0.417703,0.327898}%
\pgfsetfillcolor{currentfill}%
\pgfsetlinewidth{0.000000pt}%
\definecolor{currentstroke}{rgb}{0.000000,0.000000,0.000000}%
\pgfsetstrokecolor{currentstroke}%
\pgfsetdash{}{0pt}%
\pgfpathmoveto{\pgfqpoint{1.701778in}{2.375902in}}%
\pgfpathlineto{\pgfqpoint{1.971633in}{2.586781in}}%
\pgfpathlineto{\pgfqpoint{1.701778in}{2.375902in}}%
\pgfpathclose%
\pgfusepath{fill}%
\end{pgfscope}%
\begin{pgfscope}%
\pgfpathrectangle{\pgfqpoint{0.017500in}{0.017500in}}{\pgfqpoint{6.218282in}{3.580372in}} %
\pgfusepath{clip}%
\pgfsetbuttcap%
\pgfsetroundjoin%
\definecolor{currentfill}{rgb}{0.763520,0.178667,0.193396}%
\pgfsetfillcolor{currentfill}%
\pgfsetlinewidth{0.000000pt}%
\definecolor{currentstroke}{rgb}{0.000000,0.000000,0.000000}%
\pgfsetstrokecolor{currentstroke}%
\pgfsetdash{}{0pt}%
\pgfpathmoveto{\pgfqpoint{3.564492in}{2.541257in}}%
\pgfpathlineto{\pgfqpoint{3.816789in}{2.677291in}}%
\pgfpathlineto{\pgfqpoint{3.564492in}{2.541257in}}%
\pgfpathclose%
\pgfusepath{fill}%
\end{pgfscope}%
\begin{pgfscope}%
\pgfpathrectangle{\pgfqpoint{0.017500in}{0.017500in}}{\pgfqpoint{6.218282in}{3.580372in}} %
\pgfusepath{clip}%
\pgfsetbuttcap%
\pgfsetroundjoin%
\definecolor{currentfill}{rgb}{0.763520,0.178667,0.193396}%
\pgfsetfillcolor{currentfill}%
\pgfsetlinewidth{0.000000pt}%
\definecolor{currentstroke}{rgb}{0.000000,0.000000,0.000000}%
\pgfsetstrokecolor{currentstroke}%
\pgfsetdash{}{0pt}%
\pgfpathmoveto{\pgfqpoint{3.564492in}{2.541257in}}%
\pgfpathlineto{\pgfqpoint{3.816789in}{2.677291in}}%
\pgfpathlineto{\pgfqpoint{3.564492in}{2.541257in}}%
\pgfpathclose%
\pgfusepath{fill}%
\end{pgfscope}%
\begin{pgfscope}%
\pgfpathrectangle{\pgfqpoint{0.017500in}{0.017500in}}{\pgfqpoint{6.218282in}{3.580372in}} %
\pgfusepath{clip}%
\pgfsetbuttcap%
\pgfsetroundjoin%
\definecolor{currentfill}{rgb}{0.763520,0.178667,0.193396}%
\pgfsetfillcolor{currentfill}%
\pgfsetlinewidth{0.000000pt}%
\definecolor{currentstroke}{rgb}{0.000000,0.000000,0.000000}%
\pgfsetstrokecolor{currentstroke}%
\pgfsetdash{}{0pt}%
\pgfpathmoveto{\pgfqpoint{3.564492in}{2.541257in}}%
\pgfpathlineto{\pgfqpoint{3.816789in}{2.677291in}}%
\pgfpathlineto{\pgfqpoint{3.564492in}{2.541257in}}%
\pgfpathclose%
\pgfusepath{fill}%
\end{pgfscope}%
\begin{pgfscope}%
\pgfpathrectangle{\pgfqpoint{0.017500in}{0.017500in}}{\pgfqpoint{6.218282in}{3.580372in}} %
\pgfusepath{clip}%
\pgfsetbuttcap%
\pgfsetroundjoin%
\definecolor{currentfill}{rgb}{0.763520,0.178667,0.193396}%
\pgfsetfillcolor{currentfill}%
\pgfsetlinewidth{0.000000pt}%
\definecolor{currentstroke}{rgb}{0.000000,0.000000,0.000000}%
\pgfsetstrokecolor{currentstroke}%
\pgfsetdash{}{0pt}%
\pgfpathmoveto{\pgfqpoint{3.564492in}{2.541257in}}%
\pgfpathlineto{\pgfqpoint{3.816789in}{2.677291in}}%
\pgfpathlineto{\pgfqpoint{3.564492in}{2.541257in}}%
\pgfpathclose%
\pgfusepath{fill}%
\end{pgfscope}%
\begin{pgfscope}%
\pgfpathrectangle{\pgfqpoint{0.017500in}{0.017500in}}{\pgfqpoint{6.218282in}{3.580372in}} %
\pgfusepath{clip}%
\pgfsetbuttcap%
\pgfsetroundjoin%
\definecolor{currentfill}{rgb}{0.763520,0.178667,0.193396}%
\pgfsetfillcolor{currentfill}%
\pgfsetlinewidth{0.000000pt}%
\definecolor{currentstroke}{rgb}{0.000000,0.000000,0.000000}%
\pgfsetstrokecolor{currentstroke}%
\pgfsetdash{}{0pt}%
\pgfpathmoveto{\pgfqpoint{3.564492in}{2.541257in}}%
\pgfpathlineto{\pgfqpoint{3.816789in}{2.677291in}}%
\pgfpathlineto{\pgfqpoint{3.564492in}{2.541257in}}%
\pgfpathclose%
\pgfusepath{fill}%
\end{pgfscope}%
\begin{pgfscope}%
\pgfpathrectangle{\pgfqpoint{0.017500in}{0.017500in}}{\pgfqpoint{6.218282in}{3.580372in}} %
\pgfusepath{clip}%
\pgfsetbuttcap%
\pgfsetroundjoin%
\definecolor{currentfill}{rgb}{0.763520,0.178667,0.193396}%
\pgfsetfillcolor{currentfill}%
\pgfsetlinewidth{0.000000pt}%
\definecolor{currentstroke}{rgb}{0.000000,0.000000,0.000000}%
\pgfsetstrokecolor{currentstroke}%
\pgfsetdash{}{0pt}%
\pgfpathmoveto{\pgfqpoint{3.564492in}{2.541257in}}%
\pgfpathlineto{\pgfqpoint{3.816789in}{2.677291in}}%
\pgfpathlineto{\pgfqpoint{3.564492in}{2.541257in}}%
\pgfpathclose%
\pgfusepath{fill}%
\end{pgfscope}%
\begin{pgfscope}%
\pgfpathrectangle{\pgfqpoint{0.017500in}{0.017500in}}{\pgfqpoint{6.218282in}{3.580372in}} %
\pgfusepath{clip}%
\pgfsetbuttcap%
\pgfsetroundjoin%
\definecolor{currentfill}{rgb}{0.763520,0.178667,0.193396}%
\pgfsetfillcolor{currentfill}%
\pgfsetlinewidth{0.000000pt}%
\definecolor{currentstroke}{rgb}{0.000000,0.000000,0.000000}%
\pgfsetstrokecolor{currentstroke}%
\pgfsetdash{}{0pt}%
\pgfpathmoveto{\pgfqpoint{3.564492in}{2.541257in}}%
\pgfpathlineto{\pgfqpoint{3.816789in}{2.677291in}}%
\pgfpathlineto{\pgfqpoint{3.564492in}{2.541257in}}%
\pgfpathclose%
\pgfusepath{fill}%
\end{pgfscope}%
\begin{pgfscope}%
\pgfpathrectangle{\pgfqpoint{0.017500in}{0.017500in}}{\pgfqpoint{6.218282in}{3.580372in}} %
\pgfusepath{clip}%
\pgfsetbuttcap%
\pgfsetroundjoin%
\definecolor{currentfill}{rgb}{0.763520,0.178667,0.193396}%
\pgfsetfillcolor{currentfill}%
\pgfsetlinewidth{0.000000pt}%
\definecolor{currentstroke}{rgb}{0.000000,0.000000,0.000000}%
\pgfsetstrokecolor{currentstroke}%
\pgfsetdash{}{0pt}%
\pgfpathmoveto{\pgfqpoint{3.564492in}{2.541257in}}%
\pgfpathlineto{\pgfqpoint{3.816789in}{2.677291in}}%
\pgfpathlineto{\pgfqpoint{3.564492in}{2.541257in}}%
\pgfpathclose%
\pgfusepath{fill}%
\end{pgfscope}%
\begin{pgfscope}%
\pgfpathrectangle{\pgfqpoint{0.017500in}{0.017500in}}{\pgfqpoint{6.218282in}{3.580372in}} %
\pgfusepath{clip}%
\pgfsetbuttcap%
\pgfsetroundjoin%
\definecolor{currentfill}{rgb}{0.763520,0.178667,0.193396}%
\pgfsetfillcolor{currentfill}%
\pgfsetlinewidth{0.000000pt}%
\definecolor{currentstroke}{rgb}{0.000000,0.000000,0.000000}%
\pgfsetstrokecolor{currentstroke}%
\pgfsetdash{}{0pt}%
\pgfpathmoveto{\pgfqpoint{3.564492in}{2.541257in}}%
\pgfpathlineto{\pgfqpoint{3.816789in}{2.677291in}}%
\pgfpathlineto{\pgfqpoint{3.564492in}{2.541257in}}%
\pgfpathclose%
\pgfusepath{fill}%
\end{pgfscope}%
\begin{pgfscope}%
\pgfpathrectangle{\pgfqpoint{0.017500in}{0.017500in}}{\pgfqpoint{6.218282in}{3.580372in}} %
\pgfusepath{clip}%
\pgfsetbuttcap%
\pgfsetroundjoin%
\definecolor{currentfill}{rgb}{0.763520,0.178667,0.193396}%
\pgfsetfillcolor{currentfill}%
\pgfsetlinewidth{0.000000pt}%
\definecolor{currentstroke}{rgb}{0.000000,0.000000,0.000000}%
\pgfsetstrokecolor{currentstroke}%
\pgfsetdash{}{0pt}%
\pgfpathmoveto{\pgfqpoint{3.564492in}{2.541257in}}%
\pgfpathlineto{\pgfqpoint{3.816789in}{2.677291in}}%
\pgfpathlineto{\pgfqpoint{3.564492in}{2.541257in}}%
\pgfpathclose%
\pgfusepath{fill}%
\end{pgfscope}%
\begin{pgfscope}%
\pgfpathrectangle{\pgfqpoint{0.017500in}{0.017500in}}{\pgfqpoint{6.218282in}{3.580372in}} %
\pgfusepath{clip}%
\pgfsetbuttcap%
\pgfsetroundjoin%
\definecolor{currentfill}{rgb}{0.763520,0.178667,0.193396}%
\pgfsetfillcolor{currentfill}%
\pgfsetlinewidth{0.000000pt}%
\definecolor{currentstroke}{rgb}{0.000000,0.000000,0.000000}%
\pgfsetstrokecolor{currentstroke}%
\pgfsetdash{}{0pt}%
\pgfpathmoveto{\pgfqpoint{3.564492in}{2.541257in}}%
\pgfpathlineto{\pgfqpoint{3.816789in}{2.677291in}}%
\pgfpathlineto{\pgfqpoint{3.564492in}{2.541257in}}%
\pgfpathclose%
\pgfusepath{fill}%
\end{pgfscope}%
\begin{pgfscope}%
\pgfpathrectangle{\pgfqpoint{0.017500in}{0.017500in}}{\pgfqpoint{6.218282in}{3.580372in}} %
\pgfusepath{clip}%
\pgfsetbuttcap%
\pgfsetroundjoin%
\definecolor{currentfill}{rgb}{0.763520,0.178667,0.193396}%
\pgfsetfillcolor{currentfill}%
\pgfsetlinewidth{0.000000pt}%
\definecolor{currentstroke}{rgb}{0.000000,0.000000,0.000000}%
\pgfsetstrokecolor{currentstroke}%
\pgfsetdash{}{0pt}%
\pgfpathmoveto{\pgfqpoint{3.564492in}{2.541257in}}%
\pgfpathlineto{\pgfqpoint{3.816789in}{2.677291in}}%
\pgfpathlineto{\pgfqpoint{3.564492in}{2.541257in}}%
\pgfpathclose%
\pgfusepath{fill}%
\end{pgfscope}%
\begin{pgfscope}%
\pgfpathrectangle{\pgfqpoint{0.017500in}{0.017500in}}{\pgfqpoint{6.218282in}{3.580372in}} %
\pgfusepath{clip}%
\pgfsetbuttcap%
\pgfsetroundjoin%
\definecolor{currentfill}{rgb}{0.763520,0.178667,0.193396}%
\pgfsetfillcolor{currentfill}%
\pgfsetlinewidth{0.000000pt}%
\definecolor{currentstroke}{rgb}{0.000000,0.000000,0.000000}%
\pgfsetstrokecolor{currentstroke}%
\pgfsetdash{}{0pt}%
\pgfpathmoveto{\pgfqpoint{3.564492in}{2.541257in}}%
\pgfpathlineto{\pgfqpoint{3.816789in}{2.677291in}}%
\pgfpathlineto{\pgfqpoint{3.564492in}{2.541257in}}%
\pgfpathclose%
\pgfusepath{fill}%
\end{pgfscope}%
\begin{pgfscope}%
\pgfpathrectangle{\pgfqpoint{0.017500in}{0.017500in}}{\pgfqpoint{6.218282in}{3.580372in}} %
\pgfusepath{clip}%
\pgfsetbuttcap%
\pgfsetroundjoin%
\definecolor{currentfill}{rgb}{0.763520,0.178667,0.193396}%
\pgfsetfillcolor{currentfill}%
\pgfsetlinewidth{0.000000pt}%
\definecolor{currentstroke}{rgb}{0.000000,0.000000,0.000000}%
\pgfsetstrokecolor{currentstroke}%
\pgfsetdash{}{0pt}%
\pgfpathmoveto{\pgfqpoint{3.564492in}{2.541257in}}%
\pgfpathlineto{\pgfqpoint{3.816789in}{2.677291in}}%
\pgfpathlineto{\pgfqpoint{3.564492in}{2.541257in}}%
\pgfpathclose%
\pgfusepath{fill}%
\end{pgfscope}%
\begin{pgfscope}%
\pgfpathrectangle{\pgfqpoint{0.017500in}{0.017500in}}{\pgfqpoint{6.218282in}{3.580372in}} %
\pgfusepath{clip}%
\pgfsetbuttcap%
\pgfsetroundjoin%
\definecolor{currentfill}{rgb}{0.763520,0.178667,0.193396}%
\pgfsetfillcolor{currentfill}%
\pgfsetlinewidth{0.000000pt}%
\definecolor{currentstroke}{rgb}{0.000000,0.000000,0.000000}%
\pgfsetstrokecolor{currentstroke}%
\pgfsetdash{}{0pt}%
\pgfpathmoveto{\pgfqpoint{3.564492in}{2.541257in}}%
\pgfpathlineto{\pgfqpoint{3.816789in}{2.677291in}}%
\pgfpathlineto{\pgfqpoint{3.564492in}{2.541257in}}%
\pgfpathclose%
\pgfusepath{fill}%
\end{pgfscope}%
\begin{pgfscope}%
\pgfpathrectangle{\pgfqpoint{0.017500in}{0.017500in}}{\pgfqpoint{6.218282in}{3.580372in}} %
\pgfusepath{clip}%
\pgfsetbuttcap%
\pgfsetroundjoin%
\definecolor{currentfill}{rgb}{0.763520,0.178667,0.193396}%
\pgfsetfillcolor{currentfill}%
\pgfsetlinewidth{0.000000pt}%
\definecolor{currentstroke}{rgb}{0.000000,0.000000,0.000000}%
\pgfsetstrokecolor{currentstroke}%
\pgfsetdash{}{0pt}%
\pgfpathmoveto{\pgfqpoint{3.564492in}{2.541257in}}%
\pgfpathlineto{\pgfqpoint{3.816789in}{2.677291in}}%
\pgfpathlineto{\pgfqpoint{3.564492in}{2.541257in}}%
\pgfpathclose%
\pgfusepath{fill}%
\end{pgfscope}%
\begin{pgfscope}%
\pgfpathrectangle{\pgfqpoint{0.017500in}{0.017500in}}{\pgfqpoint{6.218282in}{3.580372in}} %
\pgfusepath{clip}%
\pgfsetbuttcap%
\pgfsetroundjoin%
\definecolor{currentfill}{rgb}{0.763520,0.178667,0.193396}%
\pgfsetfillcolor{currentfill}%
\pgfsetlinewidth{0.000000pt}%
\definecolor{currentstroke}{rgb}{0.000000,0.000000,0.000000}%
\pgfsetstrokecolor{currentstroke}%
\pgfsetdash{}{0pt}%
\pgfpathmoveto{\pgfqpoint{3.564492in}{2.541257in}}%
\pgfpathlineto{\pgfqpoint{3.816789in}{2.677291in}}%
\pgfpathlineto{\pgfqpoint{3.564492in}{2.541257in}}%
\pgfpathclose%
\pgfusepath{fill}%
\end{pgfscope}%
\begin{pgfscope}%
\pgfpathrectangle{\pgfqpoint{0.017500in}{0.017500in}}{\pgfqpoint{6.218282in}{3.580372in}} %
\pgfusepath{clip}%
\pgfsetbuttcap%
\pgfsetroundjoin%
\definecolor{currentfill}{rgb}{0.763520,0.178667,0.193396}%
\pgfsetfillcolor{currentfill}%
\pgfsetlinewidth{0.000000pt}%
\definecolor{currentstroke}{rgb}{0.000000,0.000000,0.000000}%
\pgfsetstrokecolor{currentstroke}%
\pgfsetdash{}{0pt}%
\pgfpathmoveto{\pgfqpoint{3.564492in}{2.541257in}}%
\pgfpathlineto{\pgfqpoint{3.816789in}{2.677291in}}%
\pgfpathlineto{\pgfqpoint{3.564492in}{2.541257in}}%
\pgfpathclose%
\pgfusepath{fill}%
\end{pgfscope}%
\begin{pgfscope}%
\pgfpathrectangle{\pgfqpoint{0.017500in}{0.017500in}}{\pgfqpoint{6.218282in}{3.580372in}} %
\pgfusepath{clip}%
\pgfsetbuttcap%
\pgfsetroundjoin%
\definecolor{currentfill}{rgb}{0.763520,0.178667,0.193396}%
\pgfsetfillcolor{currentfill}%
\pgfsetlinewidth{0.000000pt}%
\definecolor{currentstroke}{rgb}{0.000000,0.000000,0.000000}%
\pgfsetstrokecolor{currentstroke}%
\pgfsetdash{}{0pt}%
\pgfpathmoveto{\pgfqpoint{3.564492in}{2.541257in}}%
\pgfpathlineto{\pgfqpoint{3.816789in}{2.677291in}}%
\pgfpathlineto{\pgfqpoint{3.564492in}{2.541257in}}%
\pgfpathclose%
\pgfusepath{fill}%
\end{pgfscope}%
\begin{pgfscope}%
\pgfpathrectangle{\pgfqpoint{0.017500in}{0.017500in}}{\pgfqpoint{6.218282in}{3.580372in}} %
\pgfusepath{clip}%
\pgfsetbuttcap%
\pgfsetroundjoin%
\definecolor{currentfill}{rgb}{0.763520,0.178667,0.193396}%
\pgfsetfillcolor{currentfill}%
\pgfsetlinewidth{0.000000pt}%
\definecolor{currentstroke}{rgb}{0.000000,0.000000,0.000000}%
\pgfsetstrokecolor{currentstroke}%
\pgfsetdash{}{0pt}%
\pgfpathmoveto{\pgfqpoint{3.564492in}{2.541257in}}%
\pgfpathlineto{\pgfqpoint{3.816789in}{2.677291in}}%
\pgfpathlineto{\pgfqpoint{3.564492in}{2.541257in}}%
\pgfpathclose%
\pgfusepath{fill}%
\end{pgfscope}%
\begin{pgfscope}%
\pgfpathrectangle{\pgfqpoint{0.017500in}{0.017500in}}{\pgfqpoint{6.218282in}{3.580372in}} %
\pgfusepath{clip}%
\pgfsetbuttcap%
\pgfsetroundjoin%
\definecolor{currentfill}{rgb}{0.763520,0.178667,0.193396}%
\pgfsetfillcolor{currentfill}%
\pgfsetlinewidth{0.000000pt}%
\definecolor{currentstroke}{rgb}{0.000000,0.000000,0.000000}%
\pgfsetstrokecolor{currentstroke}%
\pgfsetdash{}{0pt}%
\pgfpathmoveto{\pgfqpoint{3.564492in}{2.541257in}}%
\pgfpathlineto{\pgfqpoint{3.816789in}{2.677291in}}%
\pgfpathlineto{\pgfqpoint{3.564492in}{2.541257in}}%
\pgfpathclose%
\pgfusepath{fill}%
\end{pgfscope}%
\begin{pgfscope}%
\pgfpathrectangle{\pgfqpoint{0.017500in}{0.017500in}}{\pgfqpoint{6.218282in}{3.580372in}} %
\pgfusepath{clip}%
\pgfsetbuttcap%
\pgfsetroundjoin%
\definecolor{currentfill}{rgb}{0.763520,0.178667,0.193396}%
\pgfsetfillcolor{currentfill}%
\pgfsetlinewidth{0.000000pt}%
\definecolor{currentstroke}{rgb}{0.000000,0.000000,0.000000}%
\pgfsetstrokecolor{currentstroke}%
\pgfsetdash{}{0pt}%
\pgfpathmoveto{\pgfqpoint{3.564492in}{2.541257in}}%
\pgfpathlineto{\pgfqpoint{3.816789in}{2.677291in}}%
\pgfpathlineto{\pgfqpoint{3.564492in}{2.541257in}}%
\pgfpathclose%
\pgfusepath{fill}%
\end{pgfscope}%
\begin{pgfscope}%
\pgfpathrectangle{\pgfqpoint{0.017500in}{0.017500in}}{\pgfqpoint{6.218282in}{3.580372in}} %
\pgfusepath{clip}%
\pgfsetbuttcap%
\pgfsetroundjoin%
\definecolor{currentfill}{rgb}{0.763520,0.178667,0.193396}%
\pgfsetfillcolor{currentfill}%
\pgfsetlinewidth{0.000000pt}%
\definecolor{currentstroke}{rgb}{0.000000,0.000000,0.000000}%
\pgfsetstrokecolor{currentstroke}%
\pgfsetdash{}{0pt}%
\pgfpathmoveto{\pgfqpoint{3.564492in}{2.541257in}}%
\pgfpathlineto{\pgfqpoint{3.816789in}{2.677291in}}%
\pgfpathlineto{\pgfqpoint{3.564492in}{2.541257in}}%
\pgfpathclose%
\pgfusepath{fill}%
\end{pgfscope}%
\begin{pgfscope}%
\pgfpathrectangle{\pgfqpoint{0.017500in}{0.017500in}}{\pgfqpoint{6.218282in}{3.580372in}} %
\pgfusepath{clip}%
\pgfsetbuttcap%
\pgfsetroundjoin%
\definecolor{currentfill}{rgb}{0.763520,0.178667,0.193396}%
\pgfsetfillcolor{currentfill}%
\pgfsetlinewidth{0.000000pt}%
\definecolor{currentstroke}{rgb}{0.000000,0.000000,0.000000}%
\pgfsetstrokecolor{currentstroke}%
\pgfsetdash{}{0pt}%
\pgfpathmoveto{\pgfqpoint{3.564492in}{2.541257in}}%
\pgfpathlineto{\pgfqpoint{3.816789in}{2.677291in}}%
\pgfpathlineto{\pgfqpoint{3.564492in}{2.541257in}}%
\pgfpathclose%
\pgfusepath{fill}%
\end{pgfscope}%
\begin{pgfscope}%
\pgfpathrectangle{\pgfqpoint{0.017500in}{0.017500in}}{\pgfqpoint{6.218282in}{3.580372in}} %
\pgfusepath{clip}%
\pgfsetbuttcap%
\pgfsetroundjoin%
\definecolor{currentfill}{rgb}{0.763520,0.178667,0.193396}%
\pgfsetfillcolor{currentfill}%
\pgfsetlinewidth{0.000000pt}%
\definecolor{currentstroke}{rgb}{0.000000,0.000000,0.000000}%
\pgfsetstrokecolor{currentstroke}%
\pgfsetdash{}{0pt}%
\pgfpathmoveto{\pgfqpoint{3.564492in}{2.541257in}}%
\pgfpathlineto{\pgfqpoint{3.816789in}{2.677291in}}%
\pgfpathlineto{\pgfqpoint{3.564492in}{2.541257in}}%
\pgfpathclose%
\pgfusepath{fill}%
\end{pgfscope}%
\begin{pgfscope}%
\pgfpathrectangle{\pgfqpoint{0.017500in}{0.017500in}}{\pgfqpoint{6.218282in}{3.580372in}} %
\pgfusepath{clip}%
\pgfsetbuttcap%
\pgfsetroundjoin%
\definecolor{currentfill}{rgb}{0.763520,0.178667,0.193396}%
\pgfsetfillcolor{currentfill}%
\pgfsetlinewidth{0.000000pt}%
\definecolor{currentstroke}{rgb}{0.000000,0.000000,0.000000}%
\pgfsetstrokecolor{currentstroke}%
\pgfsetdash{}{0pt}%
\pgfpathmoveto{\pgfqpoint{3.564492in}{2.541257in}}%
\pgfpathlineto{\pgfqpoint{3.816789in}{2.677291in}}%
\pgfpathlineto{\pgfqpoint{3.564492in}{2.541257in}}%
\pgfpathclose%
\pgfusepath{fill}%
\end{pgfscope}%
\begin{pgfscope}%
\pgfpathrectangle{\pgfqpoint{0.017500in}{0.017500in}}{\pgfqpoint{6.218282in}{3.580372in}} %
\pgfusepath{clip}%
\pgfsetbuttcap%
\pgfsetroundjoin%
\definecolor{currentfill}{rgb}{0.763520,0.178667,0.193396}%
\pgfsetfillcolor{currentfill}%
\pgfsetlinewidth{0.000000pt}%
\definecolor{currentstroke}{rgb}{0.000000,0.000000,0.000000}%
\pgfsetstrokecolor{currentstroke}%
\pgfsetdash{}{0pt}%
\pgfpathmoveto{\pgfqpoint{3.564492in}{2.541257in}}%
\pgfpathlineto{\pgfqpoint{3.816789in}{2.677291in}}%
\pgfpathlineto{\pgfqpoint{3.564492in}{2.541257in}}%
\pgfpathclose%
\pgfusepath{fill}%
\end{pgfscope}%
\begin{pgfscope}%
\pgfpathrectangle{\pgfqpoint{0.017500in}{0.017500in}}{\pgfqpoint{6.218282in}{3.580372in}} %
\pgfusepath{clip}%
\pgfsetbuttcap%
\pgfsetroundjoin%
\definecolor{currentfill}{rgb}{0.763520,0.178667,0.193396}%
\pgfsetfillcolor{currentfill}%
\pgfsetlinewidth{0.000000pt}%
\definecolor{currentstroke}{rgb}{0.000000,0.000000,0.000000}%
\pgfsetstrokecolor{currentstroke}%
\pgfsetdash{}{0pt}%
\pgfpathmoveto{\pgfqpoint{3.564492in}{2.541257in}}%
\pgfpathlineto{\pgfqpoint{3.816789in}{2.677291in}}%
\pgfpathlineto{\pgfqpoint{3.564492in}{2.541257in}}%
\pgfpathclose%
\pgfusepath{fill}%
\end{pgfscope}%
\begin{pgfscope}%
\pgfpathrectangle{\pgfqpoint{0.017500in}{0.017500in}}{\pgfqpoint{6.218282in}{3.580372in}} %
\pgfusepath{clip}%
\pgfsetbuttcap%
\pgfsetroundjoin%
\definecolor{currentfill}{rgb}{0.763520,0.178667,0.193396}%
\pgfsetfillcolor{currentfill}%
\pgfsetlinewidth{0.000000pt}%
\definecolor{currentstroke}{rgb}{0.000000,0.000000,0.000000}%
\pgfsetstrokecolor{currentstroke}%
\pgfsetdash{}{0pt}%
\pgfpathmoveto{\pgfqpoint{3.564492in}{2.541257in}}%
\pgfpathlineto{\pgfqpoint{3.816789in}{2.677291in}}%
\pgfpathlineto{\pgfqpoint{3.564492in}{2.541257in}}%
\pgfpathclose%
\pgfusepath{fill}%
\end{pgfscope}%
\begin{pgfscope}%
\pgfpathrectangle{\pgfqpoint{0.017500in}{0.017500in}}{\pgfqpoint{6.218282in}{3.580372in}} %
\pgfusepath{clip}%
\pgfsetbuttcap%
\pgfsetroundjoin%
\definecolor{currentfill}{rgb}{0.763520,0.178667,0.193396}%
\pgfsetfillcolor{currentfill}%
\pgfsetlinewidth{0.000000pt}%
\definecolor{currentstroke}{rgb}{0.000000,0.000000,0.000000}%
\pgfsetstrokecolor{currentstroke}%
\pgfsetdash{}{0pt}%
\pgfpathmoveto{\pgfqpoint{3.564492in}{2.541257in}}%
\pgfpathlineto{\pgfqpoint{3.816789in}{2.677291in}}%
\pgfpathlineto{\pgfqpoint{3.564492in}{2.541257in}}%
\pgfpathclose%
\pgfusepath{fill}%
\end{pgfscope}%
\begin{pgfscope}%
\pgfpathrectangle{\pgfqpoint{0.017500in}{0.017500in}}{\pgfqpoint{6.218282in}{3.580372in}} %
\pgfusepath{clip}%
\pgfsetbuttcap%
\pgfsetroundjoin%
\definecolor{currentfill}{rgb}{0.899543,0.847500,0.817789}%
\pgfsetfillcolor{currentfill}%
\pgfsetlinewidth{0.000000pt}%
\definecolor{currentstroke}{rgb}{0.000000,0.000000,0.000000}%
\pgfsetstrokecolor{currentstroke}%
\pgfsetdash{}{0pt}%
\pgfpathmoveto{\pgfqpoint{4.964811in}{2.684961in}}%
\pgfpathlineto{\pgfqpoint{3.688351in}{0.641693in}}%
\pgfpathlineto{\pgfqpoint{4.964811in}{2.684961in}}%
\pgfpathclose%
\pgfusepath{fill}%
\end{pgfscope}%
\begin{pgfscope}%
\pgfpathrectangle{\pgfqpoint{0.017500in}{0.017500in}}{\pgfqpoint{6.218282in}{3.580372in}} %
\pgfusepath{clip}%
\pgfsetbuttcap%
\pgfsetroundjoin%
\definecolor{currentfill}{rgb}{0.899543,0.847500,0.817789}%
\pgfsetfillcolor{currentfill}%
\pgfsetlinewidth{0.000000pt}%
\definecolor{currentstroke}{rgb}{0.000000,0.000000,0.000000}%
\pgfsetstrokecolor{currentstroke}%
\pgfsetdash{}{0pt}%
\pgfpathmoveto{\pgfqpoint{4.964811in}{2.684961in}}%
\pgfpathlineto{\pgfqpoint{3.688351in}{0.641693in}}%
\pgfpathlineto{\pgfqpoint{4.964811in}{2.684961in}}%
\pgfpathclose%
\pgfusepath{fill}%
\end{pgfscope}%
\begin{pgfscope}%
\pgfpathrectangle{\pgfqpoint{0.017500in}{0.017500in}}{\pgfqpoint{6.218282in}{3.580372in}} %
\pgfusepath{clip}%
\pgfsetbuttcap%
\pgfsetroundjoin%
\definecolor{currentfill}{rgb}{0.899543,0.847500,0.817789}%
\pgfsetfillcolor{currentfill}%
\pgfsetlinewidth{0.000000pt}%
\definecolor{currentstroke}{rgb}{0.000000,0.000000,0.000000}%
\pgfsetstrokecolor{currentstroke}%
\pgfsetdash{}{0pt}%
\pgfpathmoveto{\pgfqpoint{4.964811in}{2.684961in}}%
\pgfpathlineto{\pgfqpoint{3.688351in}{0.641693in}}%
\pgfpathlineto{\pgfqpoint{4.964811in}{2.684961in}}%
\pgfpathclose%
\pgfusepath{fill}%
\end{pgfscope}%
\begin{pgfscope}%
\pgfpathrectangle{\pgfqpoint{0.017500in}{0.017500in}}{\pgfqpoint{6.218282in}{3.580372in}} %
\pgfusepath{clip}%
\pgfsetbuttcap%
\pgfsetroundjoin%
\definecolor{currentfill}{rgb}{0.899543,0.847500,0.817789}%
\pgfsetfillcolor{currentfill}%
\pgfsetlinewidth{0.000000pt}%
\definecolor{currentstroke}{rgb}{0.000000,0.000000,0.000000}%
\pgfsetstrokecolor{currentstroke}%
\pgfsetdash{}{0pt}%
\pgfpathmoveto{\pgfqpoint{4.964811in}{2.684961in}}%
\pgfpathlineto{\pgfqpoint{3.688351in}{0.641693in}}%
\pgfpathlineto{\pgfqpoint{4.964811in}{2.684961in}}%
\pgfpathclose%
\pgfusepath{fill}%
\end{pgfscope}%
\begin{pgfscope}%
\pgfpathrectangle{\pgfqpoint{0.017500in}{0.017500in}}{\pgfqpoint{6.218282in}{3.580372in}} %
\pgfusepath{clip}%
\pgfsetbuttcap%
\pgfsetroundjoin%
\definecolor{currentfill}{rgb}{0.899543,0.847500,0.817789}%
\pgfsetfillcolor{currentfill}%
\pgfsetlinewidth{0.000000pt}%
\definecolor{currentstroke}{rgb}{0.000000,0.000000,0.000000}%
\pgfsetstrokecolor{currentstroke}%
\pgfsetdash{}{0pt}%
\pgfpathmoveto{\pgfqpoint{4.964811in}{2.684961in}}%
\pgfpathlineto{\pgfqpoint{3.688351in}{0.641693in}}%
\pgfpathlineto{\pgfqpoint{4.964811in}{2.684961in}}%
\pgfpathclose%
\pgfusepath{fill}%
\end{pgfscope}%
\begin{pgfscope}%
\pgfpathrectangle{\pgfqpoint{0.017500in}{0.017500in}}{\pgfqpoint{6.218282in}{3.580372in}} %
\pgfusepath{clip}%
\pgfsetbuttcap%
\pgfsetroundjoin%
\definecolor{currentfill}{rgb}{0.899543,0.847500,0.817789}%
\pgfsetfillcolor{currentfill}%
\pgfsetlinewidth{0.000000pt}%
\definecolor{currentstroke}{rgb}{0.000000,0.000000,0.000000}%
\pgfsetstrokecolor{currentstroke}%
\pgfsetdash{}{0pt}%
\pgfpathmoveto{\pgfqpoint{4.964811in}{2.684961in}}%
\pgfpathlineto{\pgfqpoint{3.688351in}{0.641693in}}%
\pgfpathlineto{\pgfqpoint{4.964811in}{2.684961in}}%
\pgfpathclose%
\pgfusepath{fill}%
\end{pgfscope}%
\begin{pgfscope}%
\pgfpathrectangle{\pgfqpoint{0.017500in}{0.017500in}}{\pgfqpoint{6.218282in}{3.580372in}} %
\pgfusepath{clip}%
\pgfsetbuttcap%
\pgfsetroundjoin%
\definecolor{currentfill}{rgb}{0.899543,0.847500,0.817789}%
\pgfsetfillcolor{currentfill}%
\pgfsetlinewidth{0.000000pt}%
\definecolor{currentstroke}{rgb}{0.000000,0.000000,0.000000}%
\pgfsetstrokecolor{currentstroke}%
\pgfsetdash{}{0pt}%
\pgfpathmoveto{\pgfqpoint{4.964811in}{2.684961in}}%
\pgfpathlineto{\pgfqpoint{3.688351in}{0.641693in}}%
\pgfpathlineto{\pgfqpoint{4.964811in}{2.684961in}}%
\pgfpathclose%
\pgfusepath{fill}%
\end{pgfscope}%
\begin{pgfscope}%
\pgfpathrectangle{\pgfqpoint{0.017500in}{0.017500in}}{\pgfqpoint{6.218282in}{3.580372in}} %
\pgfusepath{clip}%
\pgfsetbuttcap%
\pgfsetroundjoin%
\definecolor{currentfill}{rgb}{0.899543,0.847500,0.817789}%
\pgfsetfillcolor{currentfill}%
\pgfsetlinewidth{0.000000pt}%
\definecolor{currentstroke}{rgb}{0.000000,0.000000,0.000000}%
\pgfsetstrokecolor{currentstroke}%
\pgfsetdash{}{0pt}%
\pgfpathmoveto{\pgfqpoint{4.964811in}{2.684961in}}%
\pgfpathlineto{\pgfqpoint{3.688351in}{0.641693in}}%
\pgfpathlineto{\pgfqpoint{4.964811in}{2.684961in}}%
\pgfpathclose%
\pgfusepath{fill}%
\end{pgfscope}%
\begin{pgfscope}%
\pgfpathrectangle{\pgfqpoint{0.017500in}{0.017500in}}{\pgfqpoint{6.218282in}{3.580372in}} %
\pgfusepath{clip}%
\pgfsetbuttcap%
\pgfsetroundjoin%
\definecolor{currentfill}{rgb}{0.899543,0.847500,0.817789}%
\pgfsetfillcolor{currentfill}%
\pgfsetlinewidth{0.000000pt}%
\definecolor{currentstroke}{rgb}{0.000000,0.000000,0.000000}%
\pgfsetstrokecolor{currentstroke}%
\pgfsetdash{}{0pt}%
\pgfpathmoveto{\pgfqpoint{4.964811in}{2.684961in}}%
\pgfpathlineto{\pgfqpoint{3.688351in}{0.641693in}}%
\pgfpathlineto{\pgfqpoint{4.964811in}{2.684961in}}%
\pgfpathclose%
\pgfusepath{fill}%
\end{pgfscope}%
\begin{pgfscope}%
\pgfpathrectangle{\pgfqpoint{0.017500in}{0.017500in}}{\pgfqpoint{6.218282in}{3.580372in}} %
\pgfusepath{clip}%
\pgfsetbuttcap%
\pgfsetroundjoin%
\definecolor{currentfill}{rgb}{0.899543,0.847500,0.817789}%
\pgfsetfillcolor{currentfill}%
\pgfsetlinewidth{0.000000pt}%
\definecolor{currentstroke}{rgb}{0.000000,0.000000,0.000000}%
\pgfsetstrokecolor{currentstroke}%
\pgfsetdash{}{0pt}%
\pgfpathmoveto{\pgfqpoint{4.964811in}{2.684961in}}%
\pgfpathlineto{\pgfqpoint{3.688351in}{0.641693in}}%
\pgfpathlineto{\pgfqpoint{4.964811in}{2.684961in}}%
\pgfpathclose%
\pgfusepath{fill}%
\end{pgfscope}%
\begin{pgfscope}%
\pgfpathrectangle{\pgfqpoint{0.017500in}{0.017500in}}{\pgfqpoint{6.218282in}{3.580372in}} %
\pgfusepath{clip}%
\pgfsetbuttcap%
\pgfsetroundjoin%
\definecolor{currentfill}{rgb}{0.899543,0.847500,0.817789}%
\pgfsetfillcolor{currentfill}%
\pgfsetlinewidth{0.000000pt}%
\definecolor{currentstroke}{rgb}{0.000000,0.000000,0.000000}%
\pgfsetstrokecolor{currentstroke}%
\pgfsetdash{}{0pt}%
\pgfpathmoveto{\pgfqpoint{4.964811in}{2.684961in}}%
\pgfpathlineto{\pgfqpoint{3.688351in}{0.641693in}}%
\pgfpathlineto{\pgfqpoint{4.964811in}{2.684961in}}%
\pgfpathclose%
\pgfusepath{fill}%
\end{pgfscope}%
\begin{pgfscope}%
\pgfpathrectangle{\pgfqpoint{0.017500in}{0.017500in}}{\pgfqpoint{6.218282in}{3.580372in}} %
\pgfusepath{clip}%
\pgfsetbuttcap%
\pgfsetroundjoin%
\definecolor{currentfill}{rgb}{0.899543,0.847500,0.817789}%
\pgfsetfillcolor{currentfill}%
\pgfsetlinewidth{0.000000pt}%
\definecolor{currentstroke}{rgb}{0.000000,0.000000,0.000000}%
\pgfsetstrokecolor{currentstroke}%
\pgfsetdash{}{0pt}%
\pgfpathmoveto{\pgfqpoint{4.964811in}{2.684961in}}%
\pgfpathlineto{\pgfqpoint{3.688351in}{0.641693in}}%
\pgfpathlineto{\pgfqpoint{4.964811in}{2.684961in}}%
\pgfpathclose%
\pgfusepath{fill}%
\end{pgfscope}%
\begin{pgfscope}%
\pgfpathrectangle{\pgfqpoint{0.017500in}{0.017500in}}{\pgfqpoint{6.218282in}{3.580372in}} %
\pgfusepath{clip}%
\pgfsetbuttcap%
\pgfsetroundjoin%
\definecolor{currentfill}{rgb}{0.899543,0.847500,0.817789}%
\pgfsetfillcolor{currentfill}%
\pgfsetlinewidth{0.000000pt}%
\definecolor{currentstroke}{rgb}{0.000000,0.000000,0.000000}%
\pgfsetstrokecolor{currentstroke}%
\pgfsetdash{}{0pt}%
\pgfpathmoveto{\pgfqpoint{4.964811in}{2.684961in}}%
\pgfpathlineto{\pgfqpoint{3.688351in}{0.641693in}}%
\pgfpathlineto{\pgfqpoint{4.964811in}{2.684961in}}%
\pgfpathclose%
\pgfusepath{fill}%
\end{pgfscope}%
\begin{pgfscope}%
\pgfpathrectangle{\pgfqpoint{0.017500in}{0.017500in}}{\pgfqpoint{6.218282in}{3.580372in}} %
\pgfusepath{clip}%
\pgfsetbuttcap%
\pgfsetroundjoin%
\definecolor{currentfill}{rgb}{0.899543,0.847500,0.817789}%
\pgfsetfillcolor{currentfill}%
\pgfsetlinewidth{0.000000pt}%
\definecolor{currentstroke}{rgb}{0.000000,0.000000,0.000000}%
\pgfsetstrokecolor{currentstroke}%
\pgfsetdash{}{0pt}%
\pgfpathmoveto{\pgfqpoint{4.964811in}{2.684961in}}%
\pgfpathlineto{\pgfqpoint{3.688351in}{0.641693in}}%
\pgfpathlineto{\pgfqpoint{4.964811in}{2.684961in}}%
\pgfpathclose%
\pgfusepath{fill}%
\end{pgfscope}%
\begin{pgfscope}%
\pgfpathrectangle{\pgfqpoint{0.017500in}{0.017500in}}{\pgfqpoint{6.218282in}{3.580372in}} %
\pgfusepath{clip}%
\pgfsetbuttcap%
\pgfsetroundjoin%
\definecolor{currentfill}{rgb}{0.899543,0.847500,0.817789}%
\pgfsetfillcolor{currentfill}%
\pgfsetlinewidth{0.000000pt}%
\definecolor{currentstroke}{rgb}{0.000000,0.000000,0.000000}%
\pgfsetstrokecolor{currentstroke}%
\pgfsetdash{}{0pt}%
\pgfpathmoveto{\pgfqpoint{4.964811in}{2.684961in}}%
\pgfpathlineto{\pgfqpoint{3.688351in}{0.641693in}}%
\pgfpathlineto{\pgfqpoint{4.964811in}{2.684961in}}%
\pgfpathclose%
\pgfusepath{fill}%
\end{pgfscope}%
\begin{pgfscope}%
\pgfpathrectangle{\pgfqpoint{0.017500in}{0.017500in}}{\pgfqpoint{6.218282in}{3.580372in}} %
\pgfusepath{clip}%
\pgfsetbuttcap%
\pgfsetroundjoin%
\definecolor{currentfill}{rgb}{0.899543,0.847500,0.817789}%
\pgfsetfillcolor{currentfill}%
\pgfsetlinewidth{0.000000pt}%
\definecolor{currentstroke}{rgb}{0.000000,0.000000,0.000000}%
\pgfsetstrokecolor{currentstroke}%
\pgfsetdash{}{0pt}%
\pgfpathmoveto{\pgfqpoint{4.964811in}{2.684961in}}%
\pgfpathlineto{\pgfqpoint{3.688351in}{0.641693in}}%
\pgfpathlineto{\pgfqpoint{4.964811in}{2.684961in}}%
\pgfpathclose%
\pgfusepath{fill}%
\end{pgfscope}%
\begin{pgfscope}%
\pgfpathrectangle{\pgfqpoint{0.017500in}{0.017500in}}{\pgfqpoint{6.218282in}{3.580372in}} %
\pgfusepath{clip}%
\pgfsetbuttcap%
\pgfsetroundjoin%
\definecolor{currentfill}{rgb}{0.899543,0.847500,0.817789}%
\pgfsetfillcolor{currentfill}%
\pgfsetlinewidth{0.000000pt}%
\definecolor{currentstroke}{rgb}{0.000000,0.000000,0.000000}%
\pgfsetstrokecolor{currentstroke}%
\pgfsetdash{}{0pt}%
\pgfpathmoveto{\pgfqpoint{4.964811in}{2.684961in}}%
\pgfpathlineto{\pgfqpoint{3.688351in}{0.641693in}}%
\pgfpathlineto{\pgfqpoint{4.964811in}{2.684961in}}%
\pgfpathclose%
\pgfusepath{fill}%
\end{pgfscope}%
\begin{pgfscope}%
\pgfpathrectangle{\pgfqpoint{0.017500in}{0.017500in}}{\pgfqpoint{6.218282in}{3.580372in}} %
\pgfusepath{clip}%
\pgfsetbuttcap%
\pgfsetroundjoin%
\definecolor{currentfill}{rgb}{0.899543,0.847500,0.817789}%
\pgfsetfillcolor{currentfill}%
\pgfsetlinewidth{0.000000pt}%
\definecolor{currentstroke}{rgb}{0.000000,0.000000,0.000000}%
\pgfsetstrokecolor{currentstroke}%
\pgfsetdash{}{0pt}%
\pgfpathmoveto{\pgfqpoint{4.964811in}{2.684961in}}%
\pgfpathlineto{\pgfqpoint{3.688351in}{0.641693in}}%
\pgfpathlineto{\pgfqpoint{4.964811in}{2.684961in}}%
\pgfpathclose%
\pgfusepath{fill}%
\end{pgfscope}%
\begin{pgfscope}%
\pgfpathrectangle{\pgfqpoint{0.017500in}{0.017500in}}{\pgfqpoint{6.218282in}{3.580372in}} %
\pgfusepath{clip}%
\pgfsetbuttcap%
\pgfsetroundjoin%
\definecolor{currentfill}{rgb}{0.899543,0.847500,0.817789}%
\pgfsetfillcolor{currentfill}%
\pgfsetlinewidth{0.000000pt}%
\definecolor{currentstroke}{rgb}{0.000000,0.000000,0.000000}%
\pgfsetstrokecolor{currentstroke}%
\pgfsetdash{}{0pt}%
\pgfpathmoveto{\pgfqpoint{4.964811in}{2.684961in}}%
\pgfpathlineto{\pgfqpoint{3.688351in}{0.641693in}}%
\pgfpathlineto{\pgfqpoint{4.964811in}{2.684961in}}%
\pgfpathclose%
\pgfusepath{fill}%
\end{pgfscope}%
\begin{pgfscope}%
\pgfpathrectangle{\pgfqpoint{0.017500in}{0.017500in}}{\pgfqpoint{6.218282in}{3.580372in}} %
\pgfusepath{clip}%
\pgfsetbuttcap%
\pgfsetroundjoin%
\definecolor{currentfill}{rgb}{0.899543,0.847500,0.817789}%
\pgfsetfillcolor{currentfill}%
\pgfsetlinewidth{0.000000pt}%
\definecolor{currentstroke}{rgb}{0.000000,0.000000,0.000000}%
\pgfsetstrokecolor{currentstroke}%
\pgfsetdash{}{0pt}%
\pgfpathmoveto{\pgfqpoint{4.964811in}{2.684961in}}%
\pgfpathlineto{\pgfqpoint{3.688351in}{0.641693in}}%
\pgfpathlineto{\pgfqpoint{4.964811in}{2.684961in}}%
\pgfpathclose%
\pgfusepath{fill}%
\end{pgfscope}%
\begin{pgfscope}%
\pgfpathrectangle{\pgfqpoint{0.017500in}{0.017500in}}{\pgfqpoint{6.218282in}{3.580372in}} %
\pgfusepath{clip}%
\pgfsetbuttcap%
\pgfsetroundjoin%
\definecolor{currentfill}{rgb}{0.899543,0.847500,0.817789}%
\pgfsetfillcolor{currentfill}%
\pgfsetlinewidth{0.000000pt}%
\definecolor{currentstroke}{rgb}{0.000000,0.000000,0.000000}%
\pgfsetstrokecolor{currentstroke}%
\pgfsetdash{}{0pt}%
\pgfpathmoveto{\pgfqpoint{4.964811in}{2.684961in}}%
\pgfpathlineto{\pgfqpoint{3.688351in}{0.641693in}}%
\pgfpathlineto{\pgfqpoint{4.964811in}{2.684961in}}%
\pgfpathclose%
\pgfusepath{fill}%
\end{pgfscope}%
\begin{pgfscope}%
\pgfpathrectangle{\pgfqpoint{0.017500in}{0.017500in}}{\pgfqpoint{6.218282in}{3.580372in}} %
\pgfusepath{clip}%
\pgfsetbuttcap%
\pgfsetroundjoin%
\definecolor{currentfill}{rgb}{0.899543,0.847500,0.817789}%
\pgfsetfillcolor{currentfill}%
\pgfsetlinewidth{0.000000pt}%
\definecolor{currentstroke}{rgb}{0.000000,0.000000,0.000000}%
\pgfsetstrokecolor{currentstroke}%
\pgfsetdash{}{0pt}%
\pgfpathmoveto{\pgfqpoint{4.964811in}{2.684961in}}%
\pgfpathlineto{\pgfqpoint{3.688351in}{0.641693in}}%
\pgfpathlineto{\pgfqpoint{4.964811in}{2.684961in}}%
\pgfpathclose%
\pgfusepath{fill}%
\end{pgfscope}%
\begin{pgfscope}%
\pgfpathrectangle{\pgfqpoint{0.017500in}{0.017500in}}{\pgfqpoint{6.218282in}{3.580372in}} %
\pgfusepath{clip}%
\pgfsetbuttcap%
\pgfsetroundjoin%
\definecolor{currentfill}{rgb}{0.899543,0.847500,0.817789}%
\pgfsetfillcolor{currentfill}%
\pgfsetlinewidth{0.000000pt}%
\definecolor{currentstroke}{rgb}{0.000000,0.000000,0.000000}%
\pgfsetstrokecolor{currentstroke}%
\pgfsetdash{}{0pt}%
\pgfpathmoveto{\pgfqpoint{4.964811in}{2.684961in}}%
\pgfpathlineto{\pgfqpoint{3.688351in}{0.641693in}}%
\pgfpathlineto{\pgfqpoint{4.964811in}{2.684961in}}%
\pgfpathclose%
\pgfusepath{fill}%
\end{pgfscope}%
\begin{pgfscope}%
\pgfpathrectangle{\pgfqpoint{0.017500in}{0.017500in}}{\pgfqpoint{6.218282in}{3.580372in}} %
\pgfusepath{clip}%
\pgfsetbuttcap%
\pgfsetroundjoin%
\definecolor{currentfill}{rgb}{0.899543,0.847500,0.817789}%
\pgfsetfillcolor{currentfill}%
\pgfsetlinewidth{0.000000pt}%
\definecolor{currentstroke}{rgb}{0.000000,0.000000,0.000000}%
\pgfsetstrokecolor{currentstroke}%
\pgfsetdash{}{0pt}%
\pgfpathmoveto{\pgfqpoint{4.964811in}{2.684961in}}%
\pgfpathlineto{\pgfqpoint{3.688351in}{0.641693in}}%
\pgfpathlineto{\pgfqpoint{4.964811in}{2.684961in}}%
\pgfpathclose%
\pgfusepath{fill}%
\end{pgfscope}%
\begin{pgfscope}%
\pgfpathrectangle{\pgfqpoint{0.017500in}{0.017500in}}{\pgfqpoint{6.218282in}{3.580372in}} %
\pgfusepath{clip}%
\pgfsetbuttcap%
\pgfsetroundjoin%
\definecolor{currentfill}{rgb}{0.899543,0.847500,0.817789}%
\pgfsetfillcolor{currentfill}%
\pgfsetlinewidth{0.000000pt}%
\definecolor{currentstroke}{rgb}{0.000000,0.000000,0.000000}%
\pgfsetstrokecolor{currentstroke}%
\pgfsetdash{}{0pt}%
\pgfpathmoveto{\pgfqpoint{4.964811in}{2.684961in}}%
\pgfpathlineto{\pgfqpoint{3.688351in}{0.641693in}}%
\pgfpathlineto{\pgfqpoint{4.964811in}{2.684961in}}%
\pgfpathclose%
\pgfusepath{fill}%
\end{pgfscope}%
\begin{pgfscope}%
\pgfpathrectangle{\pgfqpoint{0.017500in}{0.017500in}}{\pgfqpoint{6.218282in}{3.580372in}} %
\pgfusepath{clip}%
\pgfsetbuttcap%
\pgfsetroundjoin%
\definecolor{currentfill}{rgb}{0.884643,0.410017,0.322507}%
\pgfsetfillcolor{currentfill}%
\pgfsetlinewidth{0.000000pt}%
\definecolor{currentstroke}{rgb}{0.000000,0.000000,0.000000}%
\pgfsetstrokecolor{currentstroke}%
\pgfsetdash{}{0pt}%
\pgfpathmoveto{\pgfqpoint{1.701778in}{2.375902in}}%
\pgfpathlineto{\pgfqpoint{2.143341in}{2.291288in}}%
\pgfpathlineto{\pgfqpoint{2.410872in}{2.508083in}}%
\pgfpathlineto{\pgfqpoint{1.971633in}{2.586781in}}%
\pgfpathlineto{\pgfqpoint{1.701778in}{2.375902in}}%
\pgfpathclose%
\pgfusepath{fill}%
\end{pgfscope}%
\begin{pgfscope}%
\pgfpathrectangle{\pgfqpoint{0.017500in}{0.017500in}}{\pgfqpoint{6.218282in}{3.580372in}} %
\pgfusepath{clip}%
\pgfsetbuttcap%
\pgfsetroundjoin%
\definecolor{currentfill}{rgb}{0.884643,0.410017,0.322507}%
\pgfsetfillcolor{currentfill}%
\pgfsetlinewidth{0.000000pt}%
\definecolor{currentstroke}{rgb}{0.000000,0.000000,0.000000}%
\pgfsetstrokecolor{currentstroke}%
\pgfsetdash{}{0pt}%
\pgfpathmoveto{\pgfqpoint{1.701778in}{2.375902in}}%
\pgfpathlineto{\pgfqpoint{2.143341in}{2.291288in}}%
\pgfpathlineto{\pgfqpoint{2.410872in}{2.508083in}}%
\pgfpathlineto{\pgfqpoint{1.971633in}{2.586781in}}%
\pgfpathlineto{\pgfqpoint{1.701778in}{2.375902in}}%
\pgfpathclose%
\pgfusepath{fill}%
\end{pgfscope}%
\begin{pgfscope}%
\pgfpathrectangle{\pgfqpoint{0.017500in}{0.017500in}}{\pgfqpoint{6.218282in}{3.580372in}} %
\pgfusepath{clip}%
\pgfsetbuttcap%
\pgfsetroundjoin%
\definecolor{currentfill}{rgb}{0.884643,0.410017,0.322507}%
\pgfsetfillcolor{currentfill}%
\pgfsetlinewidth{0.000000pt}%
\definecolor{currentstroke}{rgb}{0.000000,0.000000,0.000000}%
\pgfsetstrokecolor{currentstroke}%
\pgfsetdash{}{0pt}%
\pgfpathmoveto{\pgfqpoint{1.701778in}{2.375902in}}%
\pgfpathlineto{\pgfqpoint{2.143341in}{2.291288in}}%
\pgfpathlineto{\pgfqpoint{2.410872in}{2.508083in}}%
\pgfpathlineto{\pgfqpoint{1.971633in}{2.586781in}}%
\pgfpathlineto{\pgfqpoint{1.701778in}{2.375902in}}%
\pgfpathclose%
\pgfusepath{fill}%
\end{pgfscope}%
\begin{pgfscope}%
\pgfpathrectangle{\pgfqpoint{0.017500in}{0.017500in}}{\pgfqpoint{6.218282in}{3.580372in}} %
\pgfusepath{clip}%
\pgfsetbuttcap%
\pgfsetroundjoin%
\definecolor{currentfill}{rgb}{0.884643,0.410017,0.322507}%
\pgfsetfillcolor{currentfill}%
\pgfsetlinewidth{0.000000pt}%
\definecolor{currentstroke}{rgb}{0.000000,0.000000,0.000000}%
\pgfsetstrokecolor{currentstroke}%
\pgfsetdash{}{0pt}%
\pgfpathmoveto{\pgfqpoint{1.701778in}{2.375902in}}%
\pgfpathlineto{\pgfqpoint{2.143341in}{2.291288in}}%
\pgfpathlineto{\pgfqpoint{2.410872in}{2.508083in}}%
\pgfpathlineto{\pgfqpoint{1.971633in}{2.586781in}}%
\pgfpathlineto{\pgfqpoint{1.701778in}{2.375902in}}%
\pgfpathclose%
\pgfusepath{fill}%
\end{pgfscope}%
\begin{pgfscope}%
\pgfpathrectangle{\pgfqpoint{0.017500in}{0.017500in}}{\pgfqpoint{6.218282in}{3.580372in}} %
\pgfusepath{clip}%
\pgfsetbuttcap%
\pgfsetroundjoin%
\definecolor{currentfill}{rgb}{0.884643,0.410017,0.322507}%
\pgfsetfillcolor{currentfill}%
\pgfsetlinewidth{0.000000pt}%
\definecolor{currentstroke}{rgb}{0.000000,0.000000,0.000000}%
\pgfsetstrokecolor{currentstroke}%
\pgfsetdash{}{0pt}%
\pgfpathmoveto{\pgfqpoint{1.701778in}{2.375902in}}%
\pgfpathlineto{\pgfqpoint{2.143341in}{2.291288in}}%
\pgfpathlineto{\pgfqpoint{2.410872in}{2.508083in}}%
\pgfpathlineto{\pgfqpoint{1.971633in}{2.586781in}}%
\pgfpathlineto{\pgfqpoint{1.701778in}{2.375902in}}%
\pgfpathclose%
\pgfusepath{fill}%
\end{pgfscope}%
\begin{pgfscope}%
\pgfpathrectangle{\pgfqpoint{0.017500in}{0.017500in}}{\pgfqpoint{6.218282in}{3.580372in}} %
\pgfusepath{clip}%
\pgfsetbuttcap%
\pgfsetroundjoin%
\definecolor{currentfill}{rgb}{0.884643,0.410017,0.322507}%
\pgfsetfillcolor{currentfill}%
\pgfsetlinewidth{0.000000pt}%
\definecolor{currentstroke}{rgb}{0.000000,0.000000,0.000000}%
\pgfsetstrokecolor{currentstroke}%
\pgfsetdash{}{0pt}%
\pgfpathmoveto{\pgfqpoint{1.701778in}{2.375902in}}%
\pgfpathlineto{\pgfqpoint{2.143341in}{2.291288in}}%
\pgfpathlineto{\pgfqpoint{2.410872in}{2.508083in}}%
\pgfpathlineto{\pgfqpoint{1.971633in}{2.586781in}}%
\pgfpathlineto{\pgfqpoint{1.701778in}{2.375902in}}%
\pgfpathclose%
\pgfusepath{fill}%
\end{pgfscope}%
\begin{pgfscope}%
\pgfpathrectangle{\pgfqpoint{0.017500in}{0.017500in}}{\pgfqpoint{6.218282in}{3.580372in}} %
\pgfusepath{clip}%
\pgfsetbuttcap%
\pgfsetroundjoin%
\definecolor{currentfill}{rgb}{0.758112,0.168122,0.188827}%
\pgfsetfillcolor{currentfill}%
\pgfsetlinewidth{0.000000pt}%
\definecolor{currentstroke}{rgb}{0.000000,0.000000,0.000000}%
\pgfsetstrokecolor{currentstroke}%
\pgfsetdash{}{0pt}%
\pgfpathmoveto{\pgfqpoint{3.564492in}{2.541257in}}%
\pgfpathlineto{\pgfqpoint{4.018447in}{2.468167in}}%
\pgfpathlineto{\pgfqpoint{4.267628in}{2.613520in}}%
\pgfpathlineto{\pgfqpoint{3.816789in}{2.677291in}}%
\pgfpathlineto{\pgfqpoint{3.564492in}{2.541257in}}%
\pgfpathclose%
\pgfusepath{fill}%
\end{pgfscope}%
\begin{pgfscope}%
\pgfpathrectangle{\pgfqpoint{0.017500in}{0.017500in}}{\pgfqpoint{6.218282in}{3.580372in}} %
\pgfusepath{clip}%
\pgfsetbuttcap%
\pgfsetroundjoin%
\definecolor{currentfill}{rgb}{0.758112,0.168122,0.188827}%
\pgfsetfillcolor{currentfill}%
\pgfsetlinewidth{0.000000pt}%
\definecolor{currentstroke}{rgb}{0.000000,0.000000,0.000000}%
\pgfsetstrokecolor{currentstroke}%
\pgfsetdash{}{0pt}%
\pgfpathmoveto{\pgfqpoint{3.564492in}{2.541257in}}%
\pgfpathlineto{\pgfqpoint{4.018447in}{2.468167in}}%
\pgfpathlineto{\pgfqpoint{4.267628in}{2.613520in}}%
\pgfpathlineto{\pgfqpoint{3.816789in}{2.677291in}}%
\pgfpathlineto{\pgfqpoint{3.564492in}{2.541257in}}%
\pgfpathclose%
\pgfusepath{fill}%
\end{pgfscope}%
\begin{pgfscope}%
\pgfpathrectangle{\pgfqpoint{0.017500in}{0.017500in}}{\pgfqpoint{6.218282in}{3.580372in}} %
\pgfusepath{clip}%
\pgfsetbuttcap%
\pgfsetroundjoin%
\definecolor{currentfill}{rgb}{0.758112,0.168122,0.188827}%
\pgfsetfillcolor{currentfill}%
\pgfsetlinewidth{0.000000pt}%
\definecolor{currentstroke}{rgb}{0.000000,0.000000,0.000000}%
\pgfsetstrokecolor{currentstroke}%
\pgfsetdash{}{0pt}%
\pgfpathmoveto{\pgfqpoint{3.564492in}{2.541257in}}%
\pgfpathlineto{\pgfqpoint{4.018447in}{2.468167in}}%
\pgfpathlineto{\pgfqpoint{4.267628in}{2.613520in}}%
\pgfpathlineto{\pgfqpoint{3.816789in}{2.677291in}}%
\pgfpathlineto{\pgfqpoint{3.564492in}{2.541257in}}%
\pgfpathclose%
\pgfusepath{fill}%
\end{pgfscope}%
\begin{pgfscope}%
\pgfpathrectangle{\pgfqpoint{0.017500in}{0.017500in}}{\pgfqpoint{6.218282in}{3.580372in}} %
\pgfusepath{clip}%
\pgfsetbuttcap%
\pgfsetroundjoin%
\definecolor{currentfill}{rgb}{0.758112,0.168122,0.188827}%
\pgfsetfillcolor{currentfill}%
\pgfsetlinewidth{0.000000pt}%
\definecolor{currentstroke}{rgb}{0.000000,0.000000,0.000000}%
\pgfsetstrokecolor{currentstroke}%
\pgfsetdash{}{0pt}%
\pgfpathmoveto{\pgfqpoint{3.564492in}{2.541257in}}%
\pgfpathlineto{\pgfqpoint{4.018447in}{2.468167in}}%
\pgfpathlineto{\pgfqpoint{4.267628in}{2.613520in}}%
\pgfpathlineto{\pgfqpoint{3.816789in}{2.677291in}}%
\pgfpathlineto{\pgfqpoint{3.564492in}{2.541257in}}%
\pgfpathclose%
\pgfusepath{fill}%
\end{pgfscope}%
\begin{pgfscope}%
\pgfpathrectangle{\pgfqpoint{0.017500in}{0.017500in}}{\pgfqpoint{6.218282in}{3.580372in}} %
\pgfusepath{clip}%
\pgfsetbuttcap%
\pgfsetroundjoin%
\definecolor{currentfill}{rgb}{0.758112,0.168122,0.188827}%
\pgfsetfillcolor{currentfill}%
\pgfsetlinewidth{0.000000pt}%
\definecolor{currentstroke}{rgb}{0.000000,0.000000,0.000000}%
\pgfsetstrokecolor{currentstroke}%
\pgfsetdash{}{0pt}%
\pgfpathmoveto{\pgfqpoint{3.564492in}{2.541257in}}%
\pgfpathlineto{\pgfqpoint{4.018447in}{2.468167in}}%
\pgfpathlineto{\pgfqpoint{4.267628in}{2.613520in}}%
\pgfpathlineto{\pgfqpoint{3.816789in}{2.677291in}}%
\pgfpathlineto{\pgfqpoint{3.564492in}{2.541257in}}%
\pgfpathclose%
\pgfusepath{fill}%
\end{pgfscope}%
\begin{pgfscope}%
\pgfpathrectangle{\pgfqpoint{0.017500in}{0.017500in}}{\pgfqpoint{6.218282in}{3.580372in}} %
\pgfusepath{clip}%
\pgfsetbuttcap%
\pgfsetroundjoin%
\definecolor{currentfill}{rgb}{0.758112,0.168122,0.188827}%
\pgfsetfillcolor{currentfill}%
\pgfsetlinewidth{0.000000pt}%
\definecolor{currentstroke}{rgb}{0.000000,0.000000,0.000000}%
\pgfsetstrokecolor{currentstroke}%
\pgfsetdash{}{0pt}%
\pgfpathmoveto{\pgfqpoint{3.564492in}{2.541257in}}%
\pgfpathlineto{\pgfqpoint{4.018447in}{2.468167in}}%
\pgfpathlineto{\pgfqpoint{4.267628in}{2.613520in}}%
\pgfpathlineto{\pgfqpoint{3.816789in}{2.677291in}}%
\pgfpathlineto{\pgfqpoint{3.564492in}{2.541257in}}%
\pgfpathclose%
\pgfusepath{fill}%
\end{pgfscope}%
\begin{pgfscope}%
\pgfpathrectangle{\pgfqpoint{0.017500in}{0.017500in}}{\pgfqpoint{6.218282in}{3.580372in}} %
\pgfusepath{clip}%
\pgfsetbuttcap%
\pgfsetroundjoin%
\definecolor{currentfill}{rgb}{0.723315,0.068898,0.162989}%
\pgfsetfillcolor{currentfill}%
\pgfsetlinewidth{0.000000pt}%
\definecolor{currentstroke}{rgb}{0.000000,0.000000,0.000000}%
\pgfsetstrokecolor{currentstroke}%
\pgfsetdash{}{0pt}%
\pgfpathmoveto{\pgfqpoint{4.724651in}{2.549289in}}%
\pgfpathlineto{\pgfqpoint{4.964811in}{2.684961in}}%
\pgfpathlineto{\pgfqpoint{4.724651in}{2.549289in}}%
\pgfpathclose%
\pgfusepath{fill}%
\end{pgfscope}%
\begin{pgfscope}%
\pgfpathrectangle{\pgfqpoint{0.017500in}{0.017500in}}{\pgfqpoint{6.218282in}{3.580372in}} %
\pgfusepath{clip}%
\pgfsetbuttcap%
\pgfsetroundjoin%
\definecolor{currentfill}{rgb}{0.723315,0.068898,0.162989}%
\pgfsetfillcolor{currentfill}%
\pgfsetlinewidth{0.000000pt}%
\definecolor{currentstroke}{rgb}{0.000000,0.000000,0.000000}%
\pgfsetstrokecolor{currentstroke}%
\pgfsetdash{}{0pt}%
\pgfpathmoveto{\pgfqpoint{4.724651in}{2.549289in}}%
\pgfpathlineto{\pgfqpoint{4.964811in}{2.684961in}}%
\pgfpathlineto{\pgfqpoint{4.724651in}{2.549289in}}%
\pgfpathclose%
\pgfusepath{fill}%
\end{pgfscope}%
\begin{pgfscope}%
\pgfpathrectangle{\pgfqpoint{0.017500in}{0.017500in}}{\pgfqpoint{6.218282in}{3.580372in}} %
\pgfusepath{clip}%
\pgfsetbuttcap%
\pgfsetroundjoin%
\definecolor{currentfill}{rgb}{0.723315,0.068898,0.162989}%
\pgfsetfillcolor{currentfill}%
\pgfsetlinewidth{0.000000pt}%
\definecolor{currentstroke}{rgb}{0.000000,0.000000,0.000000}%
\pgfsetstrokecolor{currentstroke}%
\pgfsetdash{}{0pt}%
\pgfpathmoveto{\pgfqpoint{4.724651in}{2.549289in}}%
\pgfpathlineto{\pgfqpoint{4.964811in}{2.684961in}}%
\pgfpathlineto{\pgfqpoint{4.724651in}{2.549289in}}%
\pgfpathclose%
\pgfusepath{fill}%
\end{pgfscope}%
\begin{pgfscope}%
\pgfpathrectangle{\pgfqpoint{0.017500in}{0.017500in}}{\pgfqpoint{6.218282in}{3.580372in}} %
\pgfusepath{clip}%
\pgfsetbuttcap%
\pgfsetroundjoin%
\definecolor{currentfill}{rgb}{0.723315,0.068898,0.162989}%
\pgfsetfillcolor{currentfill}%
\pgfsetlinewidth{0.000000pt}%
\definecolor{currentstroke}{rgb}{0.000000,0.000000,0.000000}%
\pgfsetstrokecolor{currentstroke}%
\pgfsetdash{}{0pt}%
\pgfpathmoveto{\pgfqpoint{4.724651in}{2.549289in}}%
\pgfpathlineto{\pgfqpoint{4.964811in}{2.684961in}}%
\pgfpathlineto{\pgfqpoint{4.724651in}{2.549289in}}%
\pgfpathclose%
\pgfusepath{fill}%
\end{pgfscope}%
\begin{pgfscope}%
\pgfpathrectangle{\pgfqpoint{0.017500in}{0.017500in}}{\pgfqpoint{6.218282in}{3.580372in}} %
\pgfusepath{clip}%
\pgfsetbuttcap%
\pgfsetroundjoin%
\definecolor{currentfill}{rgb}{0.723315,0.068898,0.162989}%
\pgfsetfillcolor{currentfill}%
\pgfsetlinewidth{0.000000pt}%
\definecolor{currentstroke}{rgb}{0.000000,0.000000,0.000000}%
\pgfsetstrokecolor{currentstroke}%
\pgfsetdash{}{0pt}%
\pgfpathmoveto{\pgfqpoint{4.724651in}{2.549289in}}%
\pgfpathlineto{\pgfqpoint{4.964811in}{2.684961in}}%
\pgfpathlineto{\pgfqpoint{4.724651in}{2.549289in}}%
\pgfpathclose%
\pgfusepath{fill}%
\end{pgfscope}%
\begin{pgfscope}%
\pgfpathrectangle{\pgfqpoint{0.017500in}{0.017500in}}{\pgfqpoint{6.218282in}{3.580372in}} %
\pgfusepath{clip}%
\pgfsetbuttcap%
\pgfsetroundjoin%
\definecolor{currentfill}{rgb}{0.723315,0.068898,0.162989}%
\pgfsetfillcolor{currentfill}%
\pgfsetlinewidth{0.000000pt}%
\definecolor{currentstroke}{rgb}{0.000000,0.000000,0.000000}%
\pgfsetstrokecolor{currentstroke}%
\pgfsetdash{}{0pt}%
\pgfpathmoveto{\pgfqpoint{4.724651in}{2.549289in}}%
\pgfpathlineto{\pgfqpoint{4.964811in}{2.684961in}}%
\pgfpathlineto{\pgfqpoint{4.724651in}{2.549289in}}%
\pgfpathclose%
\pgfusepath{fill}%
\end{pgfscope}%
\begin{pgfscope}%
\pgfpathrectangle{\pgfqpoint{0.017500in}{0.017500in}}{\pgfqpoint{6.218282in}{3.580372in}} %
\pgfusepath{clip}%
\pgfsetbuttcap%
\pgfsetroundjoin%
\definecolor{currentfill}{rgb}{0.723315,0.068898,0.162989}%
\pgfsetfillcolor{currentfill}%
\pgfsetlinewidth{0.000000pt}%
\definecolor{currentstroke}{rgb}{0.000000,0.000000,0.000000}%
\pgfsetstrokecolor{currentstroke}%
\pgfsetdash{}{0pt}%
\pgfpathmoveto{\pgfqpoint{4.724651in}{2.549289in}}%
\pgfpathlineto{\pgfqpoint{4.964811in}{2.684961in}}%
\pgfpathlineto{\pgfqpoint{4.724651in}{2.549289in}}%
\pgfpathclose%
\pgfusepath{fill}%
\end{pgfscope}%
\begin{pgfscope}%
\pgfpathrectangle{\pgfqpoint{0.017500in}{0.017500in}}{\pgfqpoint{6.218282in}{3.580372in}} %
\pgfusepath{clip}%
\pgfsetbuttcap%
\pgfsetroundjoin%
\definecolor{currentfill}{rgb}{0.723315,0.068898,0.162989}%
\pgfsetfillcolor{currentfill}%
\pgfsetlinewidth{0.000000pt}%
\definecolor{currentstroke}{rgb}{0.000000,0.000000,0.000000}%
\pgfsetstrokecolor{currentstroke}%
\pgfsetdash{}{0pt}%
\pgfpathmoveto{\pgfqpoint{4.724651in}{2.549289in}}%
\pgfpathlineto{\pgfqpoint{4.964811in}{2.684961in}}%
\pgfpathlineto{\pgfqpoint{4.724651in}{2.549289in}}%
\pgfpathclose%
\pgfusepath{fill}%
\end{pgfscope}%
\begin{pgfscope}%
\pgfpathrectangle{\pgfqpoint{0.017500in}{0.017500in}}{\pgfqpoint{6.218282in}{3.580372in}} %
\pgfusepath{clip}%
\pgfsetbuttcap%
\pgfsetroundjoin%
\definecolor{currentfill}{rgb}{0.723315,0.068898,0.162989}%
\pgfsetfillcolor{currentfill}%
\pgfsetlinewidth{0.000000pt}%
\definecolor{currentstroke}{rgb}{0.000000,0.000000,0.000000}%
\pgfsetstrokecolor{currentstroke}%
\pgfsetdash{}{0pt}%
\pgfpathmoveto{\pgfqpoint{4.724651in}{2.549289in}}%
\pgfpathlineto{\pgfqpoint{4.964811in}{2.684961in}}%
\pgfpathlineto{\pgfqpoint{4.724651in}{2.549289in}}%
\pgfpathclose%
\pgfusepath{fill}%
\end{pgfscope}%
\begin{pgfscope}%
\pgfpathrectangle{\pgfqpoint{0.017500in}{0.017500in}}{\pgfqpoint{6.218282in}{3.580372in}} %
\pgfusepath{clip}%
\pgfsetbuttcap%
\pgfsetroundjoin%
\definecolor{currentfill}{rgb}{0.723315,0.068898,0.162989}%
\pgfsetfillcolor{currentfill}%
\pgfsetlinewidth{0.000000pt}%
\definecolor{currentstroke}{rgb}{0.000000,0.000000,0.000000}%
\pgfsetstrokecolor{currentstroke}%
\pgfsetdash{}{0pt}%
\pgfpathmoveto{\pgfqpoint{4.724651in}{2.549289in}}%
\pgfpathlineto{\pgfqpoint{4.964811in}{2.684961in}}%
\pgfpathlineto{\pgfqpoint{4.724651in}{2.549289in}}%
\pgfpathclose%
\pgfusepath{fill}%
\end{pgfscope}%
\begin{pgfscope}%
\pgfpathrectangle{\pgfqpoint{0.017500in}{0.017500in}}{\pgfqpoint{6.218282in}{3.580372in}} %
\pgfusepath{clip}%
\pgfsetbuttcap%
\pgfsetroundjoin%
\definecolor{currentfill}{rgb}{0.723315,0.068898,0.162989}%
\pgfsetfillcolor{currentfill}%
\pgfsetlinewidth{0.000000pt}%
\definecolor{currentstroke}{rgb}{0.000000,0.000000,0.000000}%
\pgfsetstrokecolor{currentstroke}%
\pgfsetdash{}{0pt}%
\pgfpathmoveto{\pgfqpoint{4.724651in}{2.549289in}}%
\pgfpathlineto{\pgfqpoint{4.964811in}{2.684961in}}%
\pgfpathlineto{\pgfqpoint{4.724651in}{2.549289in}}%
\pgfpathclose%
\pgfusepath{fill}%
\end{pgfscope}%
\begin{pgfscope}%
\pgfpathrectangle{\pgfqpoint{0.017500in}{0.017500in}}{\pgfqpoint{6.218282in}{3.580372in}} %
\pgfusepath{clip}%
\pgfsetbuttcap%
\pgfsetroundjoin%
\definecolor{currentfill}{rgb}{0.723315,0.068898,0.162989}%
\pgfsetfillcolor{currentfill}%
\pgfsetlinewidth{0.000000pt}%
\definecolor{currentstroke}{rgb}{0.000000,0.000000,0.000000}%
\pgfsetstrokecolor{currentstroke}%
\pgfsetdash{}{0pt}%
\pgfpathmoveto{\pgfqpoint{4.724651in}{2.549289in}}%
\pgfpathlineto{\pgfqpoint{4.964811in}{2.684961in}}%
\pgfpathlineto{\pgfqpoint{4.724651in}{2.549289in}}%
\pgfpathclose%
\pgfusepath{fill}%
\end{pgfscope}%
\begin{pgfscope}%
\pgfpathrectangle{\pgfqpoint{0.017500in}{0.017500in}}{\pgfqpoint{6.218282in}{3.580372in}} %
\pgfusepath{clip}%
\pgfsetbuttcap%
\pgfsetroundjoin%
\definecolor{currentfill}{rgb}{0.723315,0.068898,0.162989}%
\pgfsetfillcolor{currentfill}%
\pgfsetlinewidth{0.000000pt}%
\definecolor{currentstroke}{rgb}{0.000000,0.000000,0.000000}%
\pgfsetstrokecolor{currentstroke}%
\pgfsetdash{}{0pt}%
\pgfpathmoveto{\pgfqpoint{4.724651in}{2.549289in}}%
\pgfpathlineto{\pgfqpoint{4.964811in}{2.684961in}}%
\pgfpathlineto{\pgfqpoint{4.724651in}{2.549289in}}%
\pgfpathclose%
\pgfusepath{fill}%
\end{pgfscope}%
\begin{pgfscope}%
\pgfpathrectangle{\pgfqpoint{0.017500in}{0.017500in}}{\pgfqpoint{6.218282in}{3.580372in}} %
\pgfusepath{clip}%
\pgfsetbuttcap%
\pgfsetroundjoin%
\definecolor{currentfill}{rgb}{0.723315,0.068898,0.162989}%
\pgfsetfillcolor{currentfill}%
\pgfsetlinewidth{0.000000pt}%
\definecolor{currentstroke}{rgb}{0.000000,0.000000,0.000000}%
\pgfsetstrokecolor{currentstroke}%
\pgfsetdash{}{0pt}%
\pgfpathmoveto{\pgfqpoint{4.724651in}{2.549289in}}%
\pgfpathlineto{\pgfqpoint{4.964811in}{2.684961in}}%
\pgfpathlineto{\pgfqpoint{4.724651in}{2.549289in}}%
\pgfpathclose%
\pgfusepath{fill}%
\end{pgfscope}%
\begin{pgfscope}%
\pgfpathrectangle{\pgfqpoint{0.017500in}{0.017500in}}{\pgfqpoint{6.218282in}{3.580372in}} %
\pgfusepath{clip}%
\pgfsetbuttcap%
\pgfsetroundjoin%
\definecolor{currentfill}{rgb}{0.723315,0.068898,0.162989}%
\pgfsetfillcolor{currentfill}%
\pgfsetlinewidth{0.000000pt}%
\definecolor{currentstroke}{rgb}{0.000000,0.000000,0.000000}%
\pgfsetstrokecolor{currentstroke}%
\pgfsetdash{}{0pt}%
\pgfpathmoveto{\pgfqpoint{4.724651in}{2.549289in}}%
\pgfpathlineto{\pgfqpoint{4.964811in}{2.684961in}}%
\pgfpathlineto{\pgfqpoint{4.724651in}{2.549289in}}%
\pgfpathclose%
\pgfusepath{fill}%
\end{pgfscope}%
\begin{pgfscope}%
\pgfpathrectangle{\pgfqpoint{0.017500in}{0.017500in}}{\pgfqpoint{6.218282in}{3.580372in}} %
\pgfusepath{clip}%
\pgfsetbuttcap%
\pgfsetroundjoin%
\definecolor{currentfill}{rgb}{0.723315,0.068898,0.162989}%
\pgfsetfillcolor{currentfill}%
\pgfsetlinewidth{0.000000pt}%
\definecolor{currentstroke}{rgb}{0.000000,0.000000,0.000000}%
\pgfsetstrokecolor{currentstroke}%
\pgfsetdash{}{0pt}%
\pgfpathmoveto{\pgfqpoint{4.724651in}{2.549289in}}%
\pgfpathlineto{\pgfqpoint{4.964811in}{2.684961in}}%
\pgfpathlineto{\pgfqpoint{4.724651in}{2.549289in}}%
\pgfpathclose%
\pgfusepath{fill}%
\end{pgfscope}%
\begin{pgfscope}%
\pgfpathrectangle{\pgfqpoint{0.017500in}{0.017500in}}{\pgfqpoint{6.218282in}{3.580372in}} %
\pgfusepath{clip}%
\pgfsetbuttcap%
\pgfsetroundjoin%
\definecolor{currentfill}{rgb}{0.723315,0.068898,0.162989}%
\pgfsetfillcolor{currentfill}%
\pgfsetlinewidth{0.000000pt}%
\definecolor{currentstroke}{rgb}{0.000000,0.000000,0.000000}%
\pgfsetstrokecolor{currentstroke}%
\pgfsetdash{}{0pt}%
\pgfpathmoveto{\pgfqpoint{4.724651in}{2.549289in}}%
\pgfpathlineto{\pgfqpoint{4.964811in}{2.684961in}}%
\pgfpathlineto{\pgfqpoint{4.724651in}{2.549289in}}%
\pgfpathclose%
\pgfusepath{fill}%
\end{pgfscope}%
\begin{pgfscope}%
\pgfpathrectangle{\pgfqpoint{0.017500in}{0.017500in}}{\pgfqpoint{6.218282in}{3.580372in}} %
\pgfusepath{clip}%
\pgfsetbuttcap%
\pgfsetroundjoin%
\definecolor{currentfill}{rgb}{0.723315,0.068898,0.162989}%
\pgfsetfillcolor{currentfill}%
\pgfsetlinewidth{0.000000pt}%
\definecolor{currentstroke}{rgb}{0.000000,0.000000,0.000000}%
\pgfsetstrokecolor{currentstroke}%
\pgfsetdash{}{0pt}%
\pgfpathmoveto{\pgfqpoint{4.724651in}{2.549289in}}%
\pgfpathlineto{\pgfqpoint{4.964811in}{2.684961in}}%
\pgfpathlineto{\pgfqpoint{4.724651in}{2.549289in}}%
\pgfpathclose%
\pgfusepath{fill}%
\end{pgfscope}%
\begin{pgfscope}%
\pgfpathrectangle{\pgfqpoint{0.017500in}{0.017500in}}{\pgfqpoint{6.218282in}{3.580372in}} %
\pgfusepath{clip}%
\pgfsetbuttcap%
\pgfsetroundjoin%
\definecolor{currentfill}{rgb}{0.723315,0.068898,0.162989}%
\pgfsetfillcolor{currentfill}%
\pgfsetlinewidth{0.000000pt}%
\definecolor{currentstroke}{rgb}{0.000000,0.000000,0.000000}%
\pgfsetstrokecolor{currentstroke}%
\pgfsetdash{}{0pt}%
\pgfpathmoveto{\pgfqpoint{4.724651in}{2.549289in}}%
\pgfpathlineto{\pgfqpoint{4.964811in}{2.684961in}}%
\pgfpathlineto{\pgfqpoint{4.724651in}{2.549289in}}%
\pgfpathclose%
\pgfusepath{fill}%
\end{pgfscope}%
\begin{pgfscope}%
\pgfpathrectangle{\pgfqpoint{0.017500in}{0.017500in}}{\pgfqpoint{6.218282in}{3.580372in}} %
\pgfusepath{clip}%
\pgfsetbuttcap%
\pgfsetroundjoin%
\definecolor{currentfill}{rgb}{0.723315,0.068898,0.162989}%
\pgfsetfillcolor{currentfill}%
\pgfsetlinewidth{0.000000pt}%
\definecolor{currentstroke}{rgb}{0.000000,0.000000,0.000000}%
\pgfsetstrokecolor{currentstroke}%
\pgfsetdash{}{0pt}%
\pgfpathmoveto{\pgfqpoint{4.724651in}{2.549289in}}%
\pgfpathlineto{\pgfqpoint{4.964811in}{2.684961in}}%
\pgfpathlineto{\pgfqpoint{4.724651in}{2.549289in}}%
\pgfpathclose%
\pgfusepath{fill}%
\end{pgfscope}%
\begin{pgfscope}%
\pgfpathrectangle{\pgfqpoint{0.017500in}{0.017500in}}{\pgfqpoint{6.218282in}{3.580372in}} %
\pgfusepath{clip}%
\pgfsetbuttcap%
\pgfsetroundjoin%
\definecolor{currentfill}{rgb}{0.723315,0.068898,0.162989}%
\pgfsetfillcolor{currentfill}%
\pgfsetlinewidth{0.000000pt}%
\definecolor{currentstroke}{rgb}{0.000000,0.000000,0.000000}%
\pgfsetstrokecolor{currentstroke}%
\pgfsetdash{}{0pt}%
\pgfpathmoveto{\pgfqpoint{4.724651in}{2.549289in}}%
\pgfpathlineto{\pgfqpoint{4.964811in}{2.684961in}}%
\pgfpathlineto{\pgfqpoint{4.724651in}{2.549289in}}%
\pgfpathclose%
\pgfusepath{fill}%
\end{pgfscope}%
\begin{pgfscope}%
\pgfpathrectangle{\pgfqpoint{0.017500in}{0.017500in}}{\pgfqpoint{6.218282in}{3.580372in}} %
\pgfusepath{clip}%
\pgfsetbuttcap%
\pgfsetroundjoin%
\definecolor{currentfill}{rgb}{0.723315,0.068898,0.162989}%
\pgfsetfillcolor{currentfill}%
\pgfsetlinewidth{0.000000pt}%
\definecolor{currentstroke}{rgb}{0.000000,0.000000,0.000000}%
\pgfsetstrokecolor{currentstroke}%
\pgfsetdash{}{0pt}%
\pgfpathmoveto{\pgfqpoint{4.724651in}{2.549289in}}%
\pgfpathlineto{\pgfqpoint{4.964811in}{2.684961in}}%
\pgfpathlineto{\pgfqpoint{4.724651in}{2.549289in}}%
\pgfpathclose%
\pgfusepath{fill}%
\end{pgfscope}%
\begin{pgfscope}%
\pgfpathrectangle{\pgfqpoint{0.017500in}{0.017500in}}{\pgfqpoint{6.218282in}{3.580372in}} %
\pgfusepath{clip}%
\pgfsetbuttcap%
\pgfsetroundjoin%
\definecolor{currentfill}{rgb}{0.723315,0.068898,0.162989}%
\pgfsetfillcolor{currentfill}%
\pgfsetlinewidth{0.000000pt}%
\definecolor{currentstroke}{rgb}{0.000000,0.000000,0.000000}%
\pgfsetstrokecolor{currentstroke}%
\pgfsetdash{}{0pt}%
\pgfpathmoveto{\pgfqpoint{4.724651in}{2.549289in}}%
\pgfpathlineto{\pgfqpoint{4.964811in}{2.684961in}}%
\pgfpathlineto{\pgfqpoint{4.724651in}{2.549289in}}%
\pgfpathclose%
\pgfusepath{fill}%
\end{pgfscope}%
\begin{pgfscope}%
\pgfpathrectangle{\pgfqpoint{0.017500in}{0.017500in}}{\pgfqpoint{6.218282in}{3.580372in}} %
\pgfusepath{clip}%
\pgfsetbuttcap%
\pgfsetroundjoin%
\definecolor{currentfill}{rgb}{0.723315,0.068898,0.162989}%
\pgfsetfillcolor{currentfill}%
\pgfsetlinewidth{0.000000pt}%
\definecolor{currentstroke}{rgb}{0.000000,0.000000,0.000000}%
\pgfsetstrokecolor{currentstroke}%
\pgfsetdash{}{0pt}%
\pgfpathmoveto{\pgfqpoint{4.724651in}{2.549289in}}%
\pgfpathlineto{\pgfqpoint{4.964811in}{2.684961in}}%
\pgfpathlineto{\pgfqpoint{4.724651in}{2.549289in}}%
\pgfpathclose%
\pgfusepath{fill}%
\end{pgfscope}%
\begin{pgfscope}%
\pgfpathrectangle{\pgfqpoint{0.017500in}{0.017500in}}{\pgfqpoint{6.218282in}{3.580372in}} %
\pgfusepath{clip}%
\pgfsetbuttcap%
\pgfsetroundjoin%
\definecolor{currentfill}{rgb}{0.723315,0.068898,0.162989}%
\pgfsetfillcolor{currentfill}%
\pgfsetlinewidth{0.000000pt}%
\definecolor{currentstroke}{rgb}{0.000000,0.000000,0.000000}%
\pgfsetstrokecolor{currentstroke}%
\pgfsetdash{}{0pt}%
\pgfpathmoveto{\pgfqpoint{4.724651in}{2.549289in}}%
\pgfpathlineto{\pgfqpoint{4.964811in}{2.684961in}}%
\pgfpathlineto{\pgfqpoint{4.724651in}{2.549289in}}%
\pgfpathclose%
\pgfusepath{fill}%
\end{pgfscope}%
\begin{pgfscope}%
\pgfpathrectangle{\pgfqpoint{0.017500in}{0.017500in}}{\pgfqpoint{6.218282in}{3.580372in}} %
\pgfusepath{clip}%
\pgfsetbuttcap%
\pgfsetroundjoin%
\definecolor{currentfill}{rgb}{0.723315,0.068898,0.162989}%
\pgfsetfillcolor{currentfill}%
\pgfsetlinewidth{0.000000pt}%
\definecolor{currentstroke}{rgb}{0.000000,0.000000,0.000000}%
\pgfsetstrokecolor{currentstroke}%
\pgfsetdash{}{0pt}%
\pgfpathmoveto{\pgfqpoint{4.724651in}{2.549289in}}%
\pgfpathlineto{\pgfqpoint{4.964811in}{2.684961in}}%
\pgfpathlineto{\pgfqpoint{4.724651in}{2.549289in}}%
\pgfpathclose%
\pgfusepath{fill}%
\end{pgfscope}%
\begin{pgfscope}%
\pgfpathrectangle{\pgfqpoint{0.017500in}{0.017500in}}{\pgfqpoint{6.218282in}{3.580372in}} %
\pgfusepath{clip}%
\pgfsetbuttcap%
\pgfsetroundjoin%
\definecolor{currentfill}{rgb}{0.723315,0.068898,0.162989}%
\pgfsetfillcolor{currentfill}%
\pgfsetlinewidth{0.000000pt}%
\definecolor{currentstroke}{rgb}{0.000000,0.000000,0.000000}%
\pgfsetstrokecolor{currentstroke}%
\pgfsetdash{}{0pt}%
\pgfpathmoveto{\pgfqpoint{4.724651in}{2.549289in}}%
\pgfpathlineto{\pgfqpoint{4.964811in}{2.684961in}}%
\pgfpathlineto{\pgfqpoint{4.724651in}{2.549289in}}%
\pgfpathclose%
\pgfusepath{fill}%
\end{pgfscope}%
\begin{pgfscope}%
\pgfpathrectangle{\pgfqpoint{0.017500in}{0.017500in}}{\pgfqpoint{6.218282in}{3.580372in}} %
\pgfusepath{clip}%
\pgfsetbuttcap%
\pgfsetroundjoin%
\definecolor{currentfill}{rgb}{0.723315,0.068898,0.162989}%
\pgfsetfillcolor{currentfill}%
\pgfsetlinewidth{0.000000pt}%
\definecolor{currentstroke}{rgb}{0.000000,0.000000,0.000000}%
\pgfsetstrokecolor{currentstroke}%
\pgfsetdash{}{0pt}%
\pgfpathmoveto{\pgfqpoint{4.724651in}{2.549289in}}%
\pgfpathlineto{\pgfqpoint{4.964811in}{2.684961in}}%
\pgfpathlineto{\pgfqpoint{4.724651in}{2.549289in}}%
\pgfpathclose%
\pgfusepath{fill}%
\end{pgfscope}%
\begin{pgfscope}%
\pgfpathrectangle{\pgfqpoint{0.017500in}{0.017500in}}{\pgfqpoint{6.218282in}{3.580372in}} %
\pgfusepath{clip}%
\pgfsetbuttcap%
\pgfsetroundjoin%
\definecolor{currentfill}{rgb}{0.723315,0.068898,0.162989}%
\pgfsetfillcolor{currentfill}%
\pgfsetlinewidth{0.000000pt}%
\definecolor{currentstroke}{rgb}{0.000000,0.000000,0.000000}%
\pgfsetstrokecolor{currentstroke}%
\pgfsetdash{}{0pt}%
\pgfpathmoveto{\pgfqpoint{4.724651in}{2.549289in}}%
\pgfpathlineto{\pgfqpoint{4.964811in}{2.684961in}}%
\pgfpathlineto{\pgfqpoint{4.724651in}{2.549289in}}%
\pgfpathclose%
\pgfusepath{fill}%
\end{pgfscope}%
\begin{pgfscope}%
\pgfpathrectangle{\pgfqpoint{0.017500in}{0.017500in}}{\pgfqpoint{6.218282in}{3.580372in}} %
\pgfusepath{clip}%
\pgfsetbuttcap%
\pgfsetroundjoin%
\definecolor{currentfill}{rgb}{0.723315,0.068898,0.162989}%
\pgfsetfillcolor{currentfill}%
\pgfsetlinewidth{0.000000pt}%
\definecolor{currentstroke}{rgb}{0.000000,0.000000,0.000000}%
\pgfsetstrokecolor{currentstroke}%
\pgfsetdash{}{0pt}%
\pgfpathmoveto{\pgfqpoint{4.724651in}{2.549289in}}%
\pgfpathlineto{\pgfqpoint{4.964811in}{2.684961in}}%
\pgfpathlineto{\pgfqpoint{4.724651in}{2.549289in}}%
\pgfpathclose%
\pgfusepath{fill}%
\end{pgfscope}%
\begin{pgfscope}%
\pgfpathrectangle{\pgfqpoint{0.017500in}{0.017500in}}{\pgfqpoint{6.218282in}{3.580372in}} %
\pgfusepath{clip}%
\pgfsetbuttcap%
\pgfsetroundjoin%
\definecolor{currentfill}{rgb}{0.328604,0.439712,0.869587}%
\pgfsetfillcolor{currentfill}%
\pgfsetlinewidth{0.000000pt}%
\definecolor{currentstroke}{rgb}{0.000000,0.000000,0.000000}%
\pgfsetstrokecolor{currentstroke}%
\pgfsetdash{}{0pt}%
\pgfpathmoveto{\pgfqpoint{2.349853in}{0.872712in}}%
\pgfpathlineto{\pgfqpoint{2.789950in}{0.798346in}}%
\pgfpathlineto{\pgfqpoint{3.043326in}{2.141782in}}%
\pgfpathlineto{\pgfqpoint{2.590432in}{2.210709in}}%
\pgfpathlineto{\pgfqpoint{2.349853in}{0.872712in}}%
\pgfpathclose%
\pgfusepath{fill}%
\end{pgfscope}%
\begin{pgfscope}%
\pgfpathrectangle{\pgfqpoint{0.017500in}{0.017500in}}{\pgfqpoint{6.218282in}{3.580372in}} %
\pgfusepath{clip}%
\pgfsetbuttcap%
\pgfsetroundjoin%
\definecolor{currentfill}{rgb}{0.328604,0.439712,0.869587}%
\pgfsetfillcolor{currentfill}%
\pgfsetlinewidth{0.000000pt}%
\definecolor{currentstroke}{rgb}{0.000000,0.000000,0.000000}%
\pgfsetstrokecolor{currentstroke}%
\pgfsetdash{}{0pt}%
\pgfpathmoveto{\pgfqpoint{2.349853in}{0.872712in}}%
\pgfpathlineto{\pgfqpoint{2.789950in}{0.798346in}}%
\pgfpathlineto{\pgfqpoint{3.043326in}{2.141782in}}%
\pgfpathlineto{\pgfqpoint{2.590432in}{2.210709in}}%
\pgfpathlineto{\pgfqpoint{2.349853in}{0.872712in}}%
\pgfpathclose%
\pgfusepath{fill}%
\end{pgfscope}%
\begin{pgfscope}%
\pgfpathrectangle{\pgfqpoint{0.017500in}{0.017500in}}{\pgfqpoint{6.218282in}{3.580372in}} %
\pgfusepath{clip}%
\pgfsetbuttcap%
\pgfsetroundjoin%
\definecolor{currentfill}{rgb}{0.328604,0.439712,0.869587}%
\pgfsetfillcolor{currentfill}%
\pgfsetlinewidth{0.000000pt}%
\definecolor{currentstroke}{rgb}{0.000000,0.000000,0.000000}%
\pgfsetstrokecolor{currentstroke}%
\pgfsetdash{}{0pt}%
\pgfpathmoveto{\pgfqpoint{2.349853in}{0.872712in}}%
\pgfpathlineto{\pgfqpoint{2.789950in}{0.798346in}}%
\pgfpathlineto{\pgfqpoint{3.043326in}{2.141782in}}%
\pgfpathlineto{\pgfqpoint{2.590432in}{2.210709in}}%
\pgfpathlineto{\pgfqpoint{2.349853in}{0.872712in}}%
\pgfpathclose%
\pgfusepath{fill}%
\end{pgfscope}%
\begin{pgfscope}%
\pgfpathrectangle{\pgfqpoint{0.017500in}{0.017500in}}{\pgfqpoint{6.218282in}{3.580372in}} %
\pgfusepath{clip}%
\pgfsetbuttcap%
\pgfsetroundjoin%
\definecolor{currentfill}{rgb}{0.328604,0.439712,0.869587}%
\pgfsetfillcolor{currentfill}%
\pgfsetlinewidth{0.000000pt}%
\definecolor{currentstroke}{rgb}{0.000000,0.000000,0.000000}%
\pgfsetstrokecolor{currentstroke}%
\pgfsetdash{}{0pt}%
\pgfpathmoveto{\pgfqpoint{2.349853in}{0.872712in}}%
\pgfpathlineto{\pgfqpoint{2.789950in}{0.798346in}}%
\pgfpathlineto{\pgfqpoint{3.043326in}{2.141782in}}%
\pgfpathlineto{\pgfqpoint{2.590432in}{2.210709in}}%
\pgfpathlineto{\pgfqpoint{2.349853in}{0.872712in}}%
\pgfpathclose%
\pgfusepath{fill}%
\end{pgfscope}%
\begin{pgfscope}%
\pgfpathrectangle{\pgfqpoint{0.017500in}{0.017500in}}{\pgfqpoint{6.218282in}{3.580372in}} %
\pgfusepath{clip}%
\pgfsetbuttcap%
\pgfsetroundjoin%
\definecolor{currentfill}{rgb}{0.328604,0.439712,0.869587}%
\pgfsetfillcolor{currentfill}%
\pgfsetlinewidth{0.000000pt}%
\definecolor{currentstroke}{rgb}{0.000000,0.000000,0.000000}%
\pgfsetstrokecolor{currentstroke}%
\pgfsetdash{}{0pt}%
\pgfpathmoveto{\pgfqpoint{2.349853in}{0.872712in}}%
\pgfpathlineto{\pgfqpoint{2.789950in}{0.798346in}}%
\pgfpathlineto{\pgfqpoint{3.043326in}{2.141782in}}%
\pgfpathlineto{\pgfqpoint{2.590432in}{2.210709in}}%
\pgfpathlineto{\pgfqpoint{2.349853in}{0.872712in}}%
\pgfpathclose%
\pgfusepath{fill}%
\end{pgfscope}%
\begin{pgfscope}%
\pgfpathrectangle{\pgfqpoint{0.017500in}{0.017500in}}{\pgfqpoint{6.218282in}{3.580372in}} %
\pgfusepath{clip}%
\pgfsetbuttcap%
\pgfsetroundjoin%
\definecolor{currentfill}{rgb}{0.328604,0.439712,0.869587}%
\pgfsetfillcolor{currentfill}%
\pgfsetlinewidth{0.000000pt}%
\definecolor{currentstroke}{rgb}{0.000000,0.000000,0.000000}%
\pgfsetstrokecolor{currentstroke}%
\pgfsetdash{}{0pt}%
\pgfpathmoveto{\pgfqpoint{2.349853in}{0.872712in}}%
\pgfpathlineto{\pgfqpoint{2.789950in}{0.798346in}}%
\pgfpathlineto{\pgfqpoint{3.043326in}{2.141782in}}%
\pgfpathlineto{\pgfqpoint{2.590432in}{2.210709in}}%
\pgfpathlineto{\pgfqpoint{2.349853in}{0.872712in}}%
\pgfpathclose%
\pgfusepath{fill}%
\end{pgfscope}%
\begin{pgfscope}%
\pgfpathrectangle{\pgfqpoint{0.017500in}{0.017500in}}{\pgfqpoint{6.218282in}{3.580372in}} %
\pgfusepath{clip}%
\pgfsetbuttcap%
\pgfsetroundjoin%
\definecolor{currentfill}{rgb}{0.805723,0.259813,0.230562}%
\pgfsetfillcolor{currentfill}%
\pgfsetlinewidth{0.000000pt}%
\definecolor{currentstroke}{rgb}{0.000000,0.000000,0.000000}%
\pgfsetstrokecolor{currentstroke}%
\pgfsetdash{}{0pt}%
\pgfpathmoveto{\pgfqpoint{2.855513in}{2.451418in}}%
\pgfpathlineto{\pgfqpoint{3.116513in}{2.603812in}}%
\pgfpathlineto{\pgfqpoint{2.855513in}{2.451418in}}%
\pgfpathclose%
\pgfusepath{fill}%
\end{pgfscope}%
\begin{pgfscope}%
\pgfpathrectangle{\pgfqpoint{0.017500in}{0.017500in}}{\pgfqpoint{6.218282in}{3.580372in}} %
\pgfusepath{clip}%
\pgfsetbuttcap%
\pgfsetroundjoin%
\definecolor{currentfill}{rgb}{0.805723,0.259813,0.230562}%
\pgfsetfillcolor{currentfill}%
\pgfsetlinewidth{0.000000pt}%
\definecolor{currentstroke}{rgb}{0.000000,0.000000,0.000000}%
\pgfsetstrokecolor{currentstroke}%
\pgfsetdash{}{0pt}%
\pgfpathmoveto{\pgfqpoint{2.855513in}{2.451418in}}%
\pgfpathlineto{\pgfqpoint{3.116513in}{2.603812in}}%
\pgfpathlineto{\pgfqpoint{2.855513in}{2.451418in}}%
\pgfpathclose%
\pgfusepath{fill}%
\end{pgfscope}%
\begin{pgfscope}%
\pgfpathrectangle{\pgfqpoint{0.017500in}{0.017500in}}{\pgfqpoint{6.218282in}{3.580372in}} %
\pgfusepath{clip}%
\pgfsetbuttcap%
\pgfsetroundjoin%
\definecolor{currentfill}{rgb}{0.805723,0.259813,0.230562}%
\pgfsetfillcolor{currentfill}%
\pgfsetlinewidth{0.000000pt}%
\definecolor{currentstroke}{rgb}{0.000000,0.000000,0.000000}%
\pgfsetstrokecolor{currentstroke}%
\pgfsetdash{}{0pt}%
\pgfpathmoveto{\pgfqpoint{2.855513in}{2.451418in}}%
\pgfpathlineto{\pgfqpoint{3.116513in}{2.603812in}}%
\pgfpathlineto{\pgfqpoint{2.855513in}{2.451418in}}%
\pgfpathclose%
\pgfusepath{fill}%
\end{pgfscope}%
\begin{pgfscope}%
\pgfpathrectangle{\pgfqpoint{0.017500in}{0.017500in}}{\pgfqpoint{6.218282in}{3.580372in}} %
\pgfusepath{clip}%
\pgfsetbuttcap%
\pgfsetroundjoin%
\definecolor{currentfill}{rgb}{0.805723,0.259813,0.230562}%
\pgfsetfillcolor{currentfill}%
\pgfsetlinewidth{0.000000pt}%
\definecolor{currentstroke}{rgb}{0.000000,0.000000,0.000000}%
\pgfsetstrokecolor{currentstroke}%
\pgfsetdash{}{0pt}%
\pgfpathmoveto{\pgfqpoint{2.855513in}{2.451418in}}%
\pgfpathlineto{\pgfqpoint{3.116513in}{2.603812in}}%
\pgfpathlineto{\pgfqpoint{2.855513in}{2.451418in}}%
\pgfpathclose%
\pgfusepath{fill}%
\end{pgfscope}%
\begin{pgfscope}%
\pgfpathrectangle{\pgfqpoint{0.017500in}{0.017500in}}{\pgfqpoint{6.218282in}{3.580372in}} %
\pgfusepath{clip}%
\pgfsetbuttcap%
\pgfsetroundjoin%
\definecolor{currentfill}{rgb}{0.805723,0.259813,0.230562}%
\pgfsetfillcolor{currentfill}%
\pgfsetlinewidth{0.000000pt}%
\definecolor{currentstroke}{rgb}{0.000000,0.000000,0.000000}%
\pgfsetstrokecolor{currentstroke}%
\pgfsetdash{}{0pt}%
\pgfpathmoveto{\pgfqpoint{2.855513in}{2.451418in}}%
\pgfpathlineto{\pgfqpoint{3.116513in}{2.603812in}}%
\pgfpathlineto{\pgfqpoint{2.855513in}{2.451418in}}%
\pgfpathclose%
\pgfusepath{fill}%
\end{pgfscope}%
\begin{pgfscope}%
\pgfpathrectangle{\pgfqpoint{0.017500in}{0.017500in}}{\pgfqpoint{6.218282in}{3.580372in}} %
\pgfusepath{clip}%
\pgfsetbuttcap%
\pgfsetroundjoin%
\definecolor{currentfill}{rgb}{0.805723,0.259813,0.230562}%
\pgfsetfillcolor{currentfill}%
\pgfsetlinewidth{0.000000pt}%
\definecolor{currentstroke}{rgb}{0.000000,0.000000,0.000000}%
\pgfsetstrokecolor{currentstroke}%
\pgfsetdash{}{0pt}%
\pgfpathmoveto{\pgfqpoint{2.855513in}{2.451418in}}%
\pgfpathlineto{\pgfqpoint{3.116513in}{2.603812in}}%
\pgfpathlineto{\pgfqpoint{2.855513in}{2.451418in}}%
\pgfpathclose%
\pgfusepath{fill}%
\end{pgfscope}%
\begin{pgfscope}%
\pgfpathrectangle{\pgfqpoint{0.017500in}{0.017500in}}{\pgfqpoint{6.218282in}{3.580372in}} %
\pgfusepath{clip}%
\pgfsetbuttcap%
\pgfsetroundjoin%
\definecolor{currentfill}{rgb}{0.805723,0.259813,0.230562}%
\pgfsetfillcolor{currentfill}%
\pgfsetlinewidth{0.000000pt}%
\definecolor{currentstroke}{rgb}{0.000000,0.000000,0.000000}%
\pgfsetstrokecolor{currentstroke}%
\pgfsetdash{}{0pt}%
\pgfpathmoveto{\pgfqpoint{2.855513in}{2.451418in}}%
\pgfpathlineto{\pgfqpoint{3.116513in}{2.603812in}}%
\pgfpathlineto{\pgfqpoint{2.855513in}{2.451418in}}%
\pgfpathclose%
\pgfusepath{fill}%
\end{pgfscope}%
\begin{pgfscope}%
\pgfpathrectangle{\pgfqpoint{0.017500in}{0.017500in}}{\pgfqpoint{6.218282in}{3.580372in}} %
\pgfusepath{clip}%
\pgfsetbuttcap%
\pgfsetroundjoin%
\definecolor{currentfill}{rgb}{0.805723,0.259813,0.230562}%
\pgfsetfillcolor{currentfill}%
\pgfsetlinewidth{0.000000pt}%
\definecolor{currentstroke}{rgb}{0.000000,0.000000,0.000000}%
\pgfsetstrokecolor{currentstroke}%
\pgfsetdash{}{0pt}%
\pgfpathmoveto{\pgfqpoint{2.855513in}{2.451418in}}%
\pgfpathlineto{\pgfqpoint{3.116513in}{2.603812in}}%
\pgfpathlineto{\pgfqpoint{2.855513in}{2.451418in}}%
\pgfpathclose%
\pgfusepath{fill}%
\end{pgfscope}%
\begin{pgfscope}%
\pgfpathrectangle{\pgfqpoint{0.017500in}{0.017500in}}{\pgfqpoint{6.218282in}{3.580372in}} %
\pgfusepath{clip}%
\pgfsetbuttcap%
\pgfsetroundjoin%
\definecolor{currentfill}{rgb}{0.805723,0.259813,0.230562}%
\pgfsetfillcolor{currentfill}%
\pgfsetlinewidth{0.000000pt}%
\definecolor{currentstroke}{rgb}{0.000000,0.000000,0.000000}%
\pgfsetstrokecolor{currentstroke}%
\pgfsetdash{}{0pt}%
\pgfpathmoveto{\pgfqpoint{2.855513in}{2.451418in}}%
\pgfpathlineto{\pgfqpoint{3.116513in}{2.603812in}}%
\pgfpathlineto{\pgfqpoint{2.855513in}{2.451418in}}%
\pgfpathclose%
\pgfusepath{fill}%
\end{pgfscope}%
\begin{pgfscope}%
\pgfpathrectangle{\pgfqpoint{0.017500in}{0.017500in}}{\pgfqpoint{6.218282in}{3.580372in}} %
\pgfusepath{clip}%
\pgfsetbuttcap%
\pgfsetroundjoin%
\definecolor{currentfill}{rgb}{0.805723,0.259813,0.230562}%
\pgfsetfillcolor{currentfill}%
\pgfsetlinewidth{0.000000pt}%
\definecolor{currentstroke}{rgb}{0.000000,0.000000,0.000000}%
\pgfsetstrokecolor{currentstroke}%
\pgfsetdash{}{0pt}%
\pgfpathmoveto{\pgfqpoint{2.855513in}{2.451418in}}%
\pgfpathlineto{\pgfqpoint{3.116513in}{2.603812in}}%
\pgfpathlineto{\pgfqpoint{2.855513in}{2.451418in}}%
\pgfpathclose%
\pgfusepath{fill}%
\end{pgfscope}%
\begin{pgfscope}%
\pgfpathrectangle{\pgfqpoint{0.017500in}{0.017500in}}{\pgfqpoint{6.218282in}{3.580372in}} %
\pgfusepath{clip}%
\pgfsetbuttcap%
\pgfsetroundjoin%
\definecolor{currentfill}{rgb}{0.805723,0.259813,0.230562}%
\pgfsetfillcolor{currentfill}%
\pgfsetlinewidth{0.000000pt}%
\definecolor{currentstroke}{rgb}{0.000000,0.000000,0.000000}%
\pgfsetstrokecolor{currentstroke}%
\pgfsetdash{}{0pt}%
\pgfpathmoveto{\pgfqpoint{2.855513in}{2.451418in}}%
\pgfpathlineto{\pgfqpoint{3.116513in}{2.603812in}}%
\pgfpathlineto{\pgfqpoint{2.855513in}{2.451418in}}%
\pgfpathclose%
\pgfusepath{fill}%
\end{pgfscope}%
\begin{pgfscope}%
\pgfpathrectangle{\pgfqpoint{0.017500in}{0.017500in}}{\pgfqpoint{6.218282in}{3.580372in}} %
\pgfusepath{clip}%
\pgfsetbuttcap%
\pgfsetroundjoin%
\definecolor{currentfill}{rgb}{0.805723,0.259813,0.230562}%
\pgfsetfillcolor{currentfill}%
\pgfsetlinewidth{0.000000pt}%
\definecolor{currentstroke}{rgb}{0.000000,0.000000,0.000000}%
\pgfsetstrokecolor{currentstroke}%
\pgfsetdash{}{0pt}%
\pgfpathmoveto{\pgfqpoint{2.855513in}{2.451418in}}%
\pgfpathlineto{\pgfqpoint{3.116513in}{2.603812in}}%
\pgfpathlineto{\pgfqpoint{2.855513in}{2.451418in}}%
\pgfpathclose%
\pgfusepath{fill}%
\end{pgfscope}%
\begin{pgfscope}%
\pgfpathrectangle{\pgfqpoint{0.017500in}{0.017500in}}{\pgfqpoint{6.218282in}{3.580372in}} %
\pgfusepath{clip}%
\pgfsetbuttcap%
\pgfsetroundjoin%
\definecolor{currentfill}{rgb}{0.805723,0.259813,0.230562}%
\pgfsetfillcolor{currentfill}%
\pgfsetlinewidth{0.000000pt}%
\definecolor{currentstroke}{rgb}{0.000000,0.000000,0.000000}%
\pgfsetstrokecolor{currentstroke}%
\pgfsetdash{}{0pt}%
\pgfpathmoveto{\pgfqpoint{2.855513in}{2.451418in}}%
\pgfpathlineto{\pgfqpoint{3.116513in}{2.603812in}}%
\pgfpathlineto{\pgfqpoint{2.855513in}{2.451418in}}%
\pgfpathclose%
\pgfusepath{fill}%
\end{pgfscope}%
\begin{pgfscope}%
\pgfpathrectangle{\pgfqpoint{0.017500in}{0.017500in}}{\pgfqpoint{6.218282in}{3.580372in}} %
\pgfusepath{clip}%
\pgfsetbuttcap%
\pgfsetroundjoin%
\definecolor{currentfill}{rgb}{0.805723,0.259813,0.230562}%
\pgfsetfillcolor{currentfill}%
\pgfsetlinewidth{0.000000pt}%
\definecolor{currentstroke}{rgb}{0.000000,0.000000,0.000000}%
\pgfsetstrokecolor{currentstroke}%
\pgfsetdash{}{0pt}%
\pgfpathmoveto{\pgfqpoint{2.855513in}{2.451418in}}%
\pgfpathlineto{\pgfqpoint{3.116513in}{2.603812in}}%
\pgfpathlineto{\pgfqpoint{2.855513in}{2.451418in}}%
\pgfpathclose%
\pgfusepath{fill}%
\end{pgfscope}%
\begin{pgfscope}%
\pgfpathrectangle{\pgfqpoint{0.017500in}{0.017500in}}{\pgfqpoint{6.218282in}{3.580372in}} %
\pgfusepath{clip}%
\pgfsetbuttcap%
\pgfsetroundjoin%
\definecolor{currentfill}{rgb}{0.805723,0.259813,0.230562}%
\pgfsetfillcolor{currentfill}%
\pgfsetlinewidth{0.000000pt}%
\definecolor{currentstroke}{rgb}{0.000000,0.000000,0.000000}%
\pgfsetstrokecolor{currentstroke}%
\pgfsetdash{}{0pt}%
\pgfpathmoveto{\pgfqpoint{2.855513in}{2.451418in}}%
\pgfpathlineto{\pgfqpoint{3.116513in}{2.603812in}}%
\pgfpathlineto{\pgfqpoint{2.855513in}{2.451418in}}%
\pgfpathclose%
\pgfusepath{fill}%
\end{pgfscope}%
\begin{pgfscope}%
\pgfpathrectangle{\pgfqpoint{0.017500in}{0.017500in}}{\pgfqpoint{6.218282in}{3.580372in}} %
\pgfusepath{clip}%
\pgfsetbuttcap%
\pgfsetroundjoin%
\definecolor{currentfill}{rgb}{0.805723,0.259813,0.230562}%
\pgfsetfillcolor{currentfill}%
\pgfsetlinewidth{0.000000pt}%
\definecolor{currentstroke}{rgb}{0.000000,0.000000,0.000000}%
\pgfsetstrokecolor{currentstroke}%
\pgfsetdash{}{0pt}%
\pgfpathmoveto{\pgfqpoint{2.855513in}{2.451418in}}%
\pgfpathlineto{\pgfqpoint{3.116513in}{2.603812in}}%
\pgfpathlineto{\pgfqpoint{2.855513in}{2.451418in}}%
\pgfpathclose%
\pgfusepath{fill}%
\end{pgfscope}%
\begin{pgfscope}%
\pgfpathrectangle{\pgfqpoint{0.017500in}{0.017500in}}{\pgfqpoint{6.218282in}{3.580372in}} %
\pgfusepath{clip}%
\pgfsetbuttcap%
\pgfsetroundjoin%
\definecolor{currentfill}{rgb}{0.805723,0.259813,0.230562}%
\pgfsetfillcolor{currentfill}%
\pgfsetlinewidth{0.000000pt}%
\definecolor{currentstroke}{rgb}{0.000000,0.000000,0.000000}%
\pgfsetstrokecolor{currentstroke}%
\pgfsetdash{}{0pt}%
\pgfpathmoveto{\pgfqpoint{2.855513in}{2.451418in}}%
\pgfpathlineto{\pgfqpoint{3.116513in}{2.603812in}}%
\pgfpathlineto{\pgfqpoint{2.855513in}{2.451418in}}%
\pgfpathclose%
\pgfusepath{fill}%
\end{pgfscope}%
\begin{pgfscope}%
\pgfpathrectangle{\pgfqpoint{0.017500in}{0.017500in}}{\pgfqpoint{6.218282in}{3.580372in}} %
\pgfusepath{clip}%
\pgfsetbuttcap%
\pgfsetroundjoin%
\definecolor{currentfill}{rgb}{0.805723,0.259813,0.230562}%
\pgfsetfillcolor{currentfill}%
\pgfsetlinewidth{0.000000pt}%
\definecolor{currentstroke}{rgb}{0.000000,0.000000,0.000000}%
\pgfsetstrokecolor{currentstroke}%
\pgfsetdash{}{0pt}%
\pgfpathmoveto{\pgfqpoint{2.855513in}{2.451418in}}%
\pgfpathlineto{\pgfqpoint{3.116513in}{2.603812in}}%
\pgfpathlineto{\pgfqpoint{2.855513in}{2.451418in}}%
\pgfpathclose%
\pgfusepath{fill}%
\end{pgfscope}%
\begin{pgfscope}%
\pgfpathrectangle{\pgfqpoint{0.017500in}{0.017500in}}{\pgfqpoint{6.218282in}{3.580372in}} %
\pgfusepath{clip}%
\pgfsetbuttcap%
\pgfsetroundjoin%
\definecolor{currentfill}{rgb}{0.805723,0.259813,0.230562}%
\pgfsetfillcolor{currentfill}%
\pgfsetlinewidth{0.000000pt}%
\definecolor{currentstroke}{rgb}{0.000000,0.000000,0.000000}%
\pgfsetstrokecolor{currentstroke}%
\pgfsetdash{}{0pt}%
\pgfpathmoveto{\pgfqpoint{2.855513in}{2.451418in}}%
\pgfpathlineto{\pgfqpoint{3.116513in}{2.603812in}}%
\pgfpathlineto{\pgfqpoint{2.855513in}{2.451418in}}%
\pgfpathclose%
\pgfusepath{fill}%
\end{pgfscope}%
\begin{pgfscope}%
\pgfpathrectangle{\pgfqpoint{0.017500in}{0.017500in}}{\pgfqpoint{6.218282in}{3.580372in}} %
\pgfusepath{clip}%
\pgfsetbuttcap%
\pgfsetroundjoin%
\definecolor{currentfill}{rgb}{0.805723,0.259813,0.230562}%
\pgfsetfillcolor{currentfill}%
\pgfsetlinewidth{0.000000pt}%
\definecolor{currentstroke}{rgb}{0.000000,0.000000,0.000000}%
\pgfsetstrokecolor{currentstroke}%
\pgfsetdash{}{0pt}%
\pgfpathmoveto{\pgfqpoint{2.855513in}{2.451418in}}%
\pgfpathlineto{\pgfqpoint{3.116513in}{2.603812in}}%
\pgfpathlineto{\pgfqpoint{2.855513in}{2.451418in}}%
\pgfpathclose%
\pgfusepath{fill}%
\end{pgfscope}%
\begin{pgfscope}%
\pgfpathrectangle{\pgfqpoint{0.017500in}{0.017500in}}{\pgfqpoint{6.218282in}{3.580372in}} %
\pgfusepath{clip}%
\pgfsetbuttcap%
\pgfsetroundjoin%
\definecolor{currentfill}{rgb}{0.805723,0.259813,0.230562}%
\pgfsetfillcolor{currentfill}%
\pgfsetlinewidth{0.000000pt}%
\definecolor{currentstroke}{rgb}{0.000000,0.000000,0.000000}%
\pgfsetstrokecolor{currentstroke}%
\pgfsetdash{}{0pt}%
\pgfpathmoveto{\pgfqpoint{2.855513in}{2.451418in}}%
\pgfpathlineto{\pgfqpoint{3.116513in}{2.603812in}}%
\pgfpathlineto{\pgfqpoint{2.855513in}{2.451418in}}%
\pgfpathclose%
\pgfusepath{fill}%
\end{pgfscope}%
\begin{pgfscope}%
\pgfpathrectangle{\pgfqpoint{0.017500in}{0.017500in}}{\pgfqpoint{6.218282in}{3.580372in}} %
\pgfusepath{clip}%
\pgfsetbuttcap%
\pgfsetroundjoin%
\definecolor{currentfill}{rgb}{0.805723,0.259813,0.230562}%
\pgfsetfillcolor{currentfill}%
\pgfsetlinewidth{0.000000pt}%
\definecolor{currentstroke}{rgb}{0.000000,0.000000,0.000000}%
\pgfsetstrokecolor{currentstroke}%
\pgfsetdash{}{0pt}%
\pgfpathmoveto{\pgfqpoint{2.855513in}{2.451418in}}%
\pgfpathlineto{\pgfqpoint{3.116513in}{2.603812in}}%
\pgfpathlineto{\pgfqpoint{2.855513in}{2.451418in}}%
\pgfpathclose%
\pgfusepath{fill}%
\end{pgfscope}%
\begin{pgfscope}%
\pgfpathrectangle{\pgfqpoint{0.017500in}{0.017500in}}{\pgfqpoint{6.218282in}{3.580372in}} %
\pgfusepath{clip}%
\pgfsetbuttcap%
\pgfsetroundjoin%
\definecolor{currentfill}{rgb}{0.805723,0.259813,0.230562}%
\pgfsetfillcolor{currentfill}%
\pgfsetlinewidth{0.000000pt}%
\definecolor{currentstroke}{rgb}{0.000000,0.000000,0.000000}%
\pgfsetstrokecolor{currentstroke}%
\pgfsetdash{}{0pt}%
\pgfpathmoveto{\pgfqpoint{2.855513in}{2.451418in}}%
\pgfpathlineto{\pgfqpoint{3.116513in}{2.603812in}}%
\pgfpathlineto{\pgfqpoint{2.855513in}{2.451418in}}%
\pgfpathclose%
\pgfusepath{fill}%
\end{pgfscope}%
\begin{pgfscope}%
\pgfpathrectangle{\pgfqpoint{0.017500in}{0.017500in}}{\pgfqpoint{6.218282in}{3.580372in}} %
\pgfusepath{clip}%
\pgfsetbuttcap%
\pgfsetroundjoin%
\definecolor{currentfill}{rgb}{0.805723,0.259813,0.230562}%
\pgfsetfillcolor{currentfill}%
\pgfsetlinewidth{0.000000pt}%
\definecolor{currentstroke}{rgb}{0.000000,0.000000,0.000000}%
\pgfsetstrokecolor{currentstroke}%
\pgfsetdash{}{0pt}%
\pgfpathmoveto{\pgfqpoint{2.855513in}{2.451418in}}%
\pgfpathlineto{\pgfqpoint{3.116513in}{2.603812in}}%
\pgfpathlineto{\pgfqpoint{2.855513in}{2.451418in}}%
\pgfpathclose%
\pgfusepath{fill}%
\end{pgfscope}%
\begin{pgfscope}%
\pgfpathrectangle{\pgfqpoint{0.017500in}{0.017500in}}{\pgfqpoint{6.218282in}{3.580372in}} %
\pgfusepath{clip}%
\pgfsetbuttcap%
\pgfsetroundjoin%
\definecolor{currentfill}{rgb}{0.805723,0.259813,0.230562}%
\pgfsetfillcolor{currentfill}%
\pgfsetlinewidth{0.000000pt}%
\definecolor{currentstroke}{rgb}{0.000000,0.000000,0.000000}%
\pgfsetstrokecolor{currentstroke}%
\pgfsetdash{}{0pt}%
\pgfpathmoveto{\pgfqpoint{2.855513in}{2.451418in}}%
\pgfpathlineto{\pgfqpoint{3.116513in}{2.603812in}}%
\pgfpathlineto{\pgfqpoint{2.855513in}{2.451418in}}%
\pgfpathclose%
\pgfusepath{fill}%
\end{pgfscope}%
\begin{pgfscope}%
\pgfpathrectangle{\pgfqpoint{0.017500in}{0.017500in}}{\pgfqpoint{6.218282in}{3.580372in}} %
\pgfusepath{clip}%
\pgfsetbuttcap%
\pgfsetroundjoin%
\definecolor{currentfill}{rgb}{0.805723,0.259813,0.230562}%
\pgfsetfillcolor{currentfill}%
\pgfsetlinewidth{0.000000pt}%
\definecolor{currentstroke}{rgb}{0.000000,0.000000,0.000000}%
\pgfsetstrokecolor{currentstroke}%
\pgfsetdash{}{0pt}%
\pgfpathmoveto{\pgfqpoint{2.855513in}{2.451418in}}%
\pgfpathlineto{\pgfqpoint{3.116513in}{2.603812in}}%
\pgfpathlineto{\pgfqpoint{2.855513in}{2.451418in}}%
\pgfpathclose%
\pgfusepath{fill}%
\end{pgfscope}%
\begin{pgfscope}%
\pgfpathrectangle{\pgfqpoint{0.017500in}{0.017500in}}{\pgfqpoint{6.218282in}{3.580372in}} %
\pgfusepath{clip}%
\pgfsetbuttcap%
\pgfsetroundjoin%
\definecolor{currentfill}{rgb}{0.805723,0.259813,0.230562}%
\pgfsetfillcolor{currentfill}%
\pgfsetlinewidth{0.000000pt}%
\definecolor{currentstroke}{rgb}{0.000000,0.000000,0.000000}%
\pgfsetstrokecolor{currentstroke}%
\pgfsetdash{}{0pt}%
\pgfpathmoveto{\pgfqpoint{2.855513in}{2.451418in}}%
\pgfpathlineto{\pgfqpoint{3.116513in}{2.603812in}}%
\pgfpathlineto{\pgfqpoint{2.855513in}{2.451418in}}%
\pgfpathclose%
\pgfusepath{fill}%
\end{pgfscope}%
\begin{pgfscope}%
\pgfpathrectangle{\pgfqpoint{0.017500in}{0.017500in}}{\pgfqpoint{6.218282in}{3.580372in}} %
\pgfusepath{clip}%
\pgfsetbuttcap%
\pgfsetroundjoin%
\definecolor{currentfill}{rgb}{0.805723,0.259813,0.230562}%
\pgfsetfillcolor{currentfill}%
\pgfsetlinewidth{0.000000pt}%
\definecolor{currentstroke}{rgb}{0.000000,0.000000,0.000000}%
\pgfsetstrokecolor{currentstroke}%
\pgfsetdash{}{0pt}%
\pgfpathmoveto{\pgfqpoint{2.855513in}{2.451418in}}%
\pgfpathlineto{\pgfqpoint{3.116513in}{2.603812in}}%
\pgfpathlineto{\pgfqpoint{2.855513in}{2.451418in}}%
\pgfpathclose%
\pgfusepath{fill}%
\end{pgfscope}%
\begin{pgfscope}%
\pgfpathrectangle{\pgfqpoint{0.017500in}{0.017500in}}{\pgfqpoint{6.218282in}{3.580372in}} %
\pgfusepath{clip}%
\pgfsetbuttcap%
\pgfsetroundjoin%
\definecolor{currentfill}{rgb}{0.805723,0.259813,0.230562}%
\pgfsetfillcolor{currentfill}%
\pgfsetlinewidth{0.000000pt}%
\definecolor{currentstroke}{rgb}{0.000000,0.000000,0.000000}%
\pgfsetstrokecolor{currentstroke}%
\pgfsetdash{}{0pt}%
\pgfpathmoveto{\pgfqpoint{2.855513in}{2.451418in}}%
\pgfpathlineto{\pgfqpoint{3.116513in}{2.603812in}}%
\pgfpathlineto{\pgfqpoint{2.855513in}{2.451418in}}%
\pgfpathclose%
\pgfusepath{fill}%
\end{pgfscope}%
\begin{pgfscope}%
\pgfpathrectangle{\pgfqpoint{0.017500in}{0.017500in}}{\pgfqpoint{6.218282in}{3.580372in}} %
\pgfusepath{clip}%
\pgfsetbuttcap%
\pgfsetroundjoin%
\definecolor{currentfill}{rgb}{0.805723,0.259813,0.230562}%
\pgfsetfillcolor{currentfill}%
\pgfsetlinewidth{0.000000pt}%
\definecolor{currentstroke}{rgb}{0.000000,0.000000,0.000000}%
\pgfsetstrokecolor{currentstroke}%
\pgfsetdash{}{0pt}%
\pgfpathmoveto{\pgfqpoint{2.855513in}{2.451418in}}%
\pgfpathlineto{\pgfqpoint{3.116513in}{2.603812in}}%
\pgfpathlineto{\pgfqpoint{2.855513in}{2.451418in}}%
\pgfpathclose%
\pgfusepath{fill}%
\end{pgfscope}%
\begin{pgfscope}%
\pgfpathrectangle{\pgfqpoint{0.017500in}{0.017500in}}{\pgfqpoint{6.218282in}{3.580372in}} %
\pgfusepath{clip}%
\pgfsetbuttcap%
\pgfsetroundjoin%
\definecolor{currentfill}{rgb}{0.800830,0.250829,0.225696}%
\pgfsetfillcolor{currentfill}%
\pgfsetlinewidth{0.000000pt}%
\definecolor{currentstroke}{rgb}{0.000000,0.000000,0.000000}%
\pgfsetstrokecolor{currentstroke}%
\pgfsetdash{}{0pt}%
\pgfpathmoveto{\pgfqpoint{2.855513in}{2.451418in}}%
\pgfpathlineto{\pgfqpoint{3.306485in}{2.382499in}}%
\pgfpathlineto{\pgfqpoint{3.564492in}{2.541257in}}%
\pgfpathlineto{\pgfqpoint{3.116513in}{2.603812in}}%
\pgfpathlineto{\pgfqpoint{2.855513in}{2.451418in}}%
\pgfpathclose%
\pgfusepath{fill}%
\end{pgfscope}%
\begin{pgfscope}%
\pgfpathrectangle{\pgfqpoint{0.017500in}{0.017500in}}{\pgfqpoint{6.218282in}{3.580372in}} %
\pgfusepath{clip}%
\pgfsetbuttcap%
\pgfsetroundjoin%
\definecolor{currentfill}{rgb}{0.800830,0.250829,0.225696}%
\pgfsetfillcolor{currentfill}%
\pgfsetlinewidth{0.000000pt}%
\definecolor{currentstroke}{rgb}{0.000000,0.000000,0.000000}%
\pgfsetstrokecolor{currentstroke}%
\pgfsetdash{}{0pt}%
\pgfpathmoveto{\pgfqpoint{2.855513in}{2.451418in}}%
\pgfpathlineto{\pgfqpoint{3.306485in}{2.382499in}}%
\pgfpathlineto{\pgfqpoint{3.564492in}{2.541257in}}%
\pgfpathlineto{\pgfqpoint{3.116513in}{2.603812in}}%
\pgfpathlineto{\pgfqpoint{2.855513in}{2.451418in}}%
\pgfpathclose%
\pgfusepath{fill}%
\end{pgfscope}%
\begin{pgfscope}%
\pgfpathrectangle{\pgfqpoint{0.017500in}{0.017500in}}{\pgfqpoint{6.218282in}{3.580372in}} %
\pgfusepath{clip}%
\pgfsetbuttcap%
\pgfsetroundjoin%
\definecolor{currentfill}{rgb}{0.800830,0.250829,0.225696}%
\pgfsetfillcolor{currentfill}%
\pgfsetlinewidth{0.000000pt}%
\definecolor{currentstroke}{rgb}{0.000000,0.000000,0.000000}%
\pgfsetstrokecolor{currentstroke}%
\pgfsetdash{}{0pt}%
\pgfpathmoveto{\pgfqpoint{2.855513in}{2.451418in}}%
\pgfpathlineto{\pgfqpoint{3.306485in}{2.382499in}}%
\pgfpathlineto{\pgfqpoint{3.564492in}{2.541257in}}%
\pgfpathlineto{\pgfqpoint{3.116513in}{2.603812in}}%
\pgfpathlineto{\pgfqpoint{2.855513in}{2.451418in}}%
\pgfpathclose%
\pgfusepath{fill}%
\end{pgfscope}%
\begin{pgfscope}%
\pgfpathrectangle{\pgfqpoint{0.017500in}{0.017500in}}{\pgfqpoint{6.218282in}{3.580372in}} %
\pgfusepath{clip}%
\pgfsetbuttcap%
\pgfsetroundjoin%
\definecolor{currentfill}{rgb}{0.800830,0.250829,0.225696}%
\pgfsetfillcolor{currentfill}%
\pgfsetlinewidth{0.000000pt}%
\definecolor{currentstroke}{rgb}{0.000000,0.000000,0.000000}%
\pgfsetstrokecolor{currentstroke}%
\pgfsetdash{}{0pt}%
\pgfpathmoveto{\pgfqpoint{2.855513in}{2.451418in}}%
\pgfpathlineto{\pgfqpoint{3.306485in}{2.382499in}}%
\pgfpathlineto{\pgfqpoint{3.564492in}{2.541257in}}%
\pgfpathlineto{\pgfqpoint{3.116513in}{2.603812in}}%
\pgfpathlineto{\pgfqpoint{2.855513in}{2.451418in}}%
\pgfpathclose%
\pgfusepath{fill}%
\end{pgfscope}%
\begin{pgfscope}%
\pgfpathrectangle{\pgfqpoint{0.017500in}{0.017500in}}{\pgfqpoint{6.218282in}{3.580372in}} %
\pgfusepath{clip}%
\pgfsetbuttcap%
\pgfsetroundjoin%
\definecolor{currentfill}{rgb}{0.800830,0.250829,0.225696}%
\pgfsetfillcolor{currentfill}%
\pgfsetlinewidth{0.000000pt}%
\definecolor{currentstroke}{rgb}{0.000000,0.000000,0.000000}%
\pgfsetstrokecolor{currentstroke}%
\pgfsetdash{}{0pt}%
\pgfpathmoveto{\pgfqpoint{2.855513in}{2.451418in}}%
\pgfpathlineto{\pgfqpoint{3.306485in}{2.382499in}}%
\pgfpathlineto{\pgfqpoint{3.564492in}{2.541257in}}%
\pgfpathlineto{\pgfqpoint{3.116513in}{2.603812in}}%
\pgfpathlineto{\pgfqpoint{2.855513in}{2.451418in}}%
\pgfpathclose%
\pgfusepath{fill}%
\end{pgfscope}%
\begin{pgfscope}%
\pgfpathrectangle{\pgfqpoint{0.017500in}{0.017500in}}{\pgfqpoint{6.218282in}{3.580372in}} %
\pgfusepath{clip}%
\pgfsetbuttcap%
\pgfsetroundjoin%
\definecolor{currentfill}{rgb}{0.800830,0.250829,0.225696}%
\pgfsetfillcolor{currentfill}%
\pgfsetlinewidth{0.000000pt}%
\definecolor{currentstroke}{rgb}{0.000000,0.000000,0.000000}%
\pgfsetstrokecolor{currentstroke}%
\pgfsetdash{}{0pt}%
\pgfpathmoveto{\pgfqpoint{2.855513in}{2.451418in}}%
\pgfpathlineto{\pgfqpoint{3.306485in}{2.382499in}}%
\pgfpathlineto{\pgfqpoint{3.564492in}{2.541257in}}%
\pgfpathlineto{\pgfqpoint{3.116513in}{2.603812in}}%
\pgfpathlineto{\pgfqpoint{2.855513in}{2.451418in}}%
\pgfpathclose%
\pgfusepath{fill}%
\end{pgfscope}%
\begin{pgfscope}%
\pgfpathrectangle{\pgfqpoint{0.017500in}{0.017500in}}{\pgfqpoint{6.218282in}{3.580372in}} %
\pgfusepath{clip}%
\pgfsetbuttcap%
\pgfsetroundjoin%
\definecolor{currentfill}{rgb}{0.234377,0.305542,0.759680}%
\pgfsetfillcolor{currentfill}%
\pgfsetlinewidth{0.000000pt}%
\definecolor{currentstroke}{rgb}{0.000000,0.000000,0.000000}%
\pgfsetstrokecolor{currentstroke}%
\pgfsetdash{}{0pt}%
\pgfpathmoveto{\pgfqpoint{2.789950in}{0.798346in}}%
\pgfpathlineto{\pgfqpoint{3.043326in}{2.141782in}}%
\pgfpathlineto{\pgfqpoint{2.789950in}{0.798346in}}%
\pgfpathclose%
\pgfusepath{fill}%
\end{pgfscope}%
\begin{pgfscope}%
\pgfpathrectangle{\pgfqpoint{0.017500in}{0.017500in}}{\pgfqpoint{6.218282in}{3.580372in}} %
\pgfusepath{clip}%
\pgfsetbuttcap%
\pgfsetroundjoin%
\definecolor{currentfill}{rgb}{0.234377,0.305542,0.759680}%
\pgfsetfillcolor{currentfill}%
\pgfsetlinewidth{0.000000pt}%
\definecolor{currentstroke}{rgb}{0.000000,0.000000,0.000000}%
\pgfsetstrokecolor{currentstroke}%
\pgfsetdash{}{0pt}%
\pgfpathmoveto{\pgfqpoint{2.789950in}{0.798346in}}%
\pgfpathlineto{\pgfqpoint{3.043326in}{2.141782in}}%
\pgfpathlineto{\pgfqpoint{2.789950in}{0.798346in}}%
\pgfpathclose%
\pgfusepath{fill}%
\end{pgfscope}%
\begin{pgfscope}%
\pgfpathrectangle{\pgfqpoint{0.017500in}{0.017500in}}{\pgfqpoint{6.218282in}{3.580372in}} %
\pgfusepath{clip}%
\pgfsetbuttcap%
\pgfsetroundjoin%
\definecolor{currentfill}{rgb}{0.234377,0.305542,0.759680}%
\pgfsetfillcolor{currentfill}%
\pgfsetlinewidth{0.000000pt}%
\definecolor{currentstroke}{rgb}{0.000000,0.000000,0.000000}%
\pgfsetstrokecolor{currentstroke}%
\pgfsetdash{}{0pt}%
\pgfpathmoveto{\pgfqpoint{2.789950in}{0.798346in}}%
\pgfpathlineto{\pgfqpoint{3.043326in}{2.141782in}}%
\pgfpathlineto{\pgfqpoint{2.789950in}{0.798346in}}%
\pgfpathclose%
\pgfusepath{fill}%
\end{pgfscope}%
\begin{pgfscope}%
\pgfpathrectangle{\pgfqpoint{0.017500in}{0.017500in}}{\pgfqpoint{6.218282in}{3.580372in}} %
\pgfusepath{clip}%
\pgfsetbuttcap%
\pgfsetroundjoin%
\definecolor{currentfill}{rgb}{0.234377,0.305542,0.759680}%
\pgfsetfillcolor{currentfill}%
\pgfsetlinewidth{0.000000pt}%
\definecolor{currentstroke}{rgb}{0.000000,0.000000,0.000000}%
\pgfsetstrokecolor{currentstroke}%
\pgfsetdash{}{0pt}%
\pgfpathmoveto{\pgfqpoint{2.789950in}{0.798346in}}%
\pgfpathlineto{\pgfqpoint{3.043326in}{2.141782in}}%
\pgfpathlineto{\pgfqpoint{2.789950in}{0.798346in}}%
\pgfpathclose%
\pgfusepath{fill}%
\end{pgfscope}%
\begin{pgfscope}%
\pgfpathrectangle{\pgfqpoint{0.017500in}{0.017500in}}{\pgfqpoint{6.218282in}{3.580372in}} %
\pgfusepath{clip}%
\pgfsetbuttcap%
\pgfsetroundjoin%
\definecolor{currentfill}{rgb}{0.234377,0.305542,0.759680}%
\pgfsetfillcolor{currentfill}%
\pgfsetlinewidth{0.000000pt}%
\definecolor{currentstroke}{rgb}{0.000000,0.000000,0.000000}%
\pgfsetstrokecolor{currentstroke}%
\pgfsetdash{}{0pt}%
\pgfpathmoveto{\pgfqpoint{2.789950in}{0.798346in}}%
\pgfpathlineto{\pgfqpoint{3.043326in}{2.141782in}}%
\pgfpathlineto{\pgfqpoint{2.789950in}{0.798346in}}%
\pgfpathclose%
\pgfusepath{fill}%
\end{pgfscope}%
\begin{pgfscope}%
\pgfpathrectangle{\pgfqpoint{0.017500in}{0.017500in}}{\pgfqpoint{6.218282in}{3.580372in}} %
\pgfusepath{clip}%
\pgfsetbuttcap%
\pgfsetroundjoin%
\definecolor{currentfill}{rgb}{0.234377,0.305542,0.759680}%
\pgfsetfillcolor{currentfill}%
\pgfsetlinewidth{0.000000pt}%
\definecolor{currentstroke}{rgb}{0.000000,0.000000,0.000000}%
\pgfsetstrokecolor{currentstroke}%
\pgfsetdash{}{0pt}%
\pgfpathmoveto{\pgfqpoint{2.789950in}{0.798346in}}%
\pgfpathlineto{\pgfqpoint{3.043326in}{2.141782in}}%
\pgfpathlineto{\pgfqpoint{2.789950in}{0.798346in}}%
\pgfpathclose%
\pgfusepath{fill}%
\end{pgfscope}%
\begin{pgfscope}%
\pgfpathrectangle{\pgfqpoint{0.017500in}{0.017500in}}{\pgfqpoint{6.218282in}{3.580372in}} %
\pgfusepath{clip}%
\pgfsetbuttcap%
\pgfsetroundjoin%
\definecolor{currentfill}{rgb}{0.234377,0.305542,0.759680}%
\pgfsetfillcolor{currentfill}%
\pgfsetlinewidth{0.000000pt}%
\definecolor{currentstroke}{rgb}{0.000000,0.000000,0.000000}%
\pgfsetstrokecolor{currentstroke}%
\pgfsetdash{}{0pt}%
\pgfpathmoveto{\pgfqpoint{2.789950in}{0.798346in}}%
\pgfpathlineto{\pgfqpoint{3.043326in}{2.141782in}}%
\pgfpathlineto{\pgfqpoint{2.789950in}{0.798346in}}%
\pgfpathclose%
\pgfusepath{fill}%
\end{pgfscope}%
\begin{pgfscope}%
\pgfpathrectangle{\pgfqpoint{0.017500in}{0.017500in}}{\pgfqpoint{6.218282in}{3.580372in}} %
\pgfusepath{clip}%
\pgfsetbuttcap%
\pgfsetroundjoin%
\definecolor{currentfill}{rgb}{0.234377,0.305542,0.759680}%
\pgfsetfillcolor{currentfill}%
\pgfsetlinewidth{0.000000pt}%
\definecolor{currentstroke}{rgb}{0.000000,0.000000,0.000000}%
\pgfsetstrokecolor{currentstroke}%
\pgfsetdash{}{0pt}%
\pgfpathmoveto{\pgfqpoint{2.789950in}{0.798346in}}%
\pgfpathlineto{\pgfqpoint{3.043326in}{2.141782in}}%
\pgfpathlineto{\pgfqpoint{2.789950in}{0.798346in}}%
\pgfpathclose%
\pgfusepath{fill}%
\end{pgfscope}%
\begin{pgfscope}%
\pgfpathrectangle{\pgfqpoint{0.017500in}{0.017500in}}{\pgfqpoint{6.218282in}{3.580372in}} %
\pgfusepath{clip}%
\pgfsetbuttcap%
\pgfsetroundjoin%
\definecolor{currentfill}{rgb}{0.234377,0.305542,0.759680}%
\pgfsetfillcolor{currentfill}%
\pgfsetlinewidth{0.000000pt}%
\definecolor{currentstroke}{rgb}{0.000000,0.000000,0.000000}%
\pgfsetstrokecolor{currentstroke}%
\pgfsetdash{}{0pt}%
\pgfpathmoveto{\pgfqpoint{2.789950in}{0.798346in}}%
\pgfpathlineto{\pgfqpoint{3.043326in}{2.141782in}}%
\pgfpathlineto{\pgfqpoint{2.789950in}{0.798346in}}%
\pgfpathclose%
\pgfusepath{fill}%
\end{pgfscope}%
\begin{pgfscope}%
\pgfpathrectangle{\pgfqpoint{0.017500in}{0.017500in}}{\pgfqpoint{6.218282in}{3.580372in}} %
\pgfusepath{clip}%
\pgfsetbuttcap%
\pgfsetroundjoin%
\definecolor{currentfill}{rgb}{0.234377,0.305542,0.759680}%
\pgfsetfillcolor{currentfill}%
\pgfsetlinewidth{0.000000pt}%
\definecolor{currentstroke}{rgb}{0.000000,0.000000,0.000000}%
\pgfsetstrokecolor{currentstroke}%
\pgfsetdash{}{0pt}%
\pgfpathmoveto{\pgfqpoint{2.789950in}{0.798346in}}%
\pgfpathlineto{\pgfqpoint{3.043326in}{2.141782in}}%
\pgfpathlineto{\pgfqpoint{2.789950in}{0.798346in}}%
\pgfpathclose%
\pgfusepath{fill}%
\end{pgfscope}%
\begin{pgfscope}%
\pgfpathrectangle{\pgfqpoint{0.017500in}{0.017500in}}{\pgfqpoint{6.218282in}{3.580372in}} %
\pgfusepath{clip}%
\pgfsetbuttcap%
\pgfsetroundjoin%
\definecolor{currentfill}{rgb}{0.234377,0.305542,0.759680}%
\pgfsetfillcolor{currentfill}%
\pgfsetlinewidth{0.000000pt}%
\definecolor{currentstroke}{rgb}{0.000000,0.000000,0.000000}%
\pgfsetstrokecolor{currentstroke}%
\pgfsetdash{}{0pt}%
\pgfpathmoveto{\pgfqpoint{2.789950in}{0.798346in}}%
\pgfpathlineto{\pgfqpoint{3.043326in}{2.141782in}}%
\pgfpathlineto{\pgfqpoint{2.789950in}{0.798346in}}%
\pgfpathclose%
\pgfusepath{fill}%
\end{pgfscope}%
\begin{pgfscope}%
\pgfpathrectangle{\pgfqpoint{0.017500in}{0.017500in}}{\pgfqpoint{6.218282in}{3.580372in}} %
\pgfusepath{clip}%
\pgfsetbuttcap%
\pgfsetroundjoin%
\definecolor{currentfill}{rgb}{0.234377,0.305542,0.759680}%
\pgfsetfillcolor{currentfill}%
\pgfsetlinewidth{0.000000pt}%
\definecolor{currentstroke}{rgb}{0.000000,0.000000,0.000000}%
\pgfsetstrokecolor{currentstroke}%
\pgfsetdash{}{0pt}%
\pgfpathmoveto{\pgfqpoint{2.789950in}{0.798346in}}%
\pgfpathlineto{\pgfqpoint{3.043326in}{2.141782in}}%
\pgfpathlineto{\pgfqpoint{2.789950in}{0.798346in}}%
\pgfpathclose%
\pgfusepath{fill}%
\end{pgfscope}%
\begin{pgfscope}%
\pgfpathrectangle{\pgfqpoint{0.017500in}{0.017500in}}{\pgfqpoint{6.218282in}{3.580372in}} %
\pgfusepath{clip}%
\pgfsetbuttcap%
\pgfsetroundjoin%
\definecolor{currentfill}{rgb}{0.234377,0.305542,0.759680}%
\pgfsetfillcolor{currentfill}%
\pgfsetlinewidth{0.000000pt}%
\definecolor{currentstroke}{rgb}{0.000000,0.000000,0.000000}%
\pgfsetstrokecolor{currentstroke}%
\pgfsetdash{}{0pt}%
\pgfpathmoveto{\pgfqpoint{2.789950in}{0.798346in}}%
\pgfpathlineto{\pgfqpoint{3.043326in}{2.141782in}}%
\pgfpathlineto{\pgfqpoint{2.789950in}{0.798346in}}%
\pgfpathclose%
\pgfusepath{fill}%
\end{pgfscope}%
\begin{pgfscope}%
\pgfpathrectangle{\pgfqpoint{0.017500in}{0.017500in}}{\pgfqpoint{6.218282in}{3.580372in}} %
\pgfusepath{clip}%
\pgfsetbuttcap%
\pgfsetroundjoin%
\definecolor{currentfill}{rgb}{0.234377,0.305542,0.759680}%
\pgfsetfillcolor{currentfill}%
\pgfsetlinewidth{0.000000pt}%
\definecolor{currentstroke}{rgb}{0.000000,0.000000,0.000000}%
\pgfsetstrokecolor{currentstroke}%
\pgfsetdash{}{0pt}%
\pgfpathmoveto{\pgfqpoint{2.789950in}{0.798346in}}%
\pgfpathlineto{\pgfqpoint{3.043326in}{2.141782in}}%
\pgfpathlineto{\pgfqpoint{2.789950in}{0.798346in}}%
\pgfpathclose%
\pgfusepath{fill}%
\end{pgfscope}%
\begin{pgfscope}%
\pgfpathrectangle{\pgfqpoint{0.017500in}{0.017500in}}{\pgfqpoint{6.218282in}{3.580372in}} %
\pgfusepath{clip}%
\pgfsetbuttcap%
\pgfsetroundjoin%
\definecolor{currentfill}{rgb}{0.234377,0.305542,0.759680}%
\pgfsetfillcolor{currentfill}%
\pgfsetlinewidth{0.000000pt}%
\definecolor{currentstroke}{rgb}{0.000000,0.000000,0.000000}%
\pgfsetstrokecolor{currentstroke}%
\pgfsetdash{}{0pt}%
\pgfpathmoveto{\pgfqpoint{2.789950in}{0.798346in}}%
\pgfpathlineto{\pgfqpoint{3.043326in}{2.141782in}}%
\pgfpathlineto{\pgfqpoint{2.789950in}{0.798346in}}%
\pgfpathclose%
\pgfusepath{fill}%
\end{pgfscope}%
\begin{pgfscope}%
\pgfpathrectangle{\pgfqpoint{0.017500in}{0.017500in}}{\pgfqpoint{6.218282in}{3.580372in}} %
\pgfusepath{clip}%
\pgfsetbuttcap%
\pgfsetroundjoin%
\definecolor{currentfill}{rgb}{0.234377,0.305542,0.759680}%
\pgfsetfillcolor{currentfill}%
\pgfsetlinewidth{0.000000pt}%
\definecolor{currentstroke}{rgb}{0.000000,0.000000,0.000000}%
\pgfsetstrokecolor{currentstroke}%
\pgfsetdash{}{0pt}%
\pgfpathmoveto{\pgfqpoint{2.789950in}{0.798346in}}%
\pgfpathlineto{\pgfqpoint{3.043326in}{2.141782in}}%
\pgfpathlineto{\pgfqpoint{2.789950in}{0.798346in}}%
\pgfpathclose%
\pgfusepath{fill}%
\end{pgfscope}%
\begin{pgfscope}%
\pgfpathrectangle{\pgfqpoint{0.017500in}{0.017500in}}{\pgfqpoint{6.218282in}{3.580372in}} %
\pgfusepath{clip}%
\pgfsetbuttcap%
\pgfsetroundjoin%
\definecolor{currentfill}{rgb}{0.234377,0.305542,0.759680}%
\pgfsetfillcolor{currentfill}%
\pgfsetlinewidth{0.000000pt}%
\definecolor{currentstroke}{rgb}{0.000000,0.000000,0.000000}%
\pgfsetstrokecolor{currentstroke}%
\pgfsetdash{}{0pt}%
\pgfpathmoveto{\pgfqpoint{2.789950in}{0.798346in}}%
\pgfpathlineto{\pgfqpoint{3.043326in}{2.141782in}}%
\pgfpathlineto{\pgfqpoint{2.789950in}{0.798346in}}%
\pgfpathclose%
\pgfusepath{fill}%
\end{pgfscope}%
\begin{pgfscope}%
\pgfpathrectangle{\pgfqpoint{0.017500in}{0.017500in}}{\pgfqpoint{6.218282in}{3.580372in}} %
\pgfusepath{clip}%
\pgfsetbuttcap%
\pgfsetroundjoin%
\definecolor{currentfill}{rgb}{0.234377,0.305542,0.759680}%
\pgfsetfillcolor{currentfill}%
\pgfsetlinewidth{0.000000pt}%
\definecolor{currentstroke}{rgb}{0.000000,0.000000,0.000000}%
\pgfsetstrokecolor{currentstroke}%
\pgfsetdash{}{0pt}%
\pgfpathmoveto{\pgfqpoint{2.789950in}{0.798346in}}%
\pgfpathlineto{\pgfqpoint{3.043326in}{2.141782in}}%
\pgfpathlineto{\pgfqpoint{2.789950in}{0.798346in}}%
\pgfpathclose%
\pgfusepath{fill}%
\end{pgfscope}%
\begin{pgfscope}%
\pgfpathrectangle{\pgfqpoint{0.017500in}{0.017500in}}{\pgfqpoint{6.218282in}{3.580372in}} %
\pgfusepath{clip}%
\pgfsetbuttcap%
\pgfsetroundjoin%
\definecolor{currentfill}{rgb}{0.234377,0.305542,0.759680}%
\pgfsetfillcolor{currentfill}%
\pgfsetlinewidth{0.000000pt}%
\definecolor{currentstroke}{rgb}{0.000000,0.000000,0.000000}%
\pgfsetstrokecolor{currentstroke}%
\pgfsetdash{}{0pt}%
\pgfpathmoveto{\pgfqpoint{2.789950in}{0.798346in}}%
\pgfpathlineto{\pgfqpoint{3.043326in}{2.141782in}}%
\pgfpathlineto{\pgfqpoint{2.789950in}{0.798346in}}%
\pgfpathclose%
\pgfusepath{fill}%
\end{pgfscope}%
\begin{pgfscope}%
\pgfpathrectangle{\pgfqpoint{0.017500in}{0.017500in}}{\pgfqpoint{6.218282in}{3.580372in}} %
\pgfusepath{clip}%
\pgfsetbuttcap%
\pgfsetroundjoin%
\definecolor{currentfill}{rgb}{0.234377,0.305542,0.759680}%
\pgfsetfillcolor{currentfill}%
\pgfsetlinewidth{0.000000pt}%
\definecolor{currentstroke}{rgb}{0.000000,0.000000,0.000000}%
\pgfsetstrokecolor{currentstroke}%
\pgfsetdash{}{0pt}%
\pgfpathmoveto{\pgfqpoint{2.789950in}{0.798346in}}%
\pgfpathlineto{\pgfqpoint{3.043326in}{2.141782in}}%
\pgfpathlineto{\pgfqpoint{2.789950in}{0.798346in}}%
\pgfpathclose%
\pgfusepath{fill}%
\end{pgfscope}%
\begin{pgfscope}%
\pgfpathrectangle{\pgfqpoint{0.017500in}{0.017500in}}{\pgfqpoint{6.218282in}{3.580372in}} %
\pgfusepath{clip}%
\pgfsetbuttcap%
\pgfsetroundjoin%
\definecolor{currentfill}{rgb}{0.234377,0.305542,0.759680}%
\pgfsetfillcolor{currentfill}%
\pgfsetlinewidth{0.000000pt}%
\definecolor{currentstroke}{rgb}{0.000000,0.000000,0.000000}%
\pgfsetstrokecolor{currentstroke}%
\pgfsetdash{}{0pt}%
\pgfpathmoveto{\pgfqpoint{2.789950in}{0.798346in}}%
\pgfpathlineto{\pgfqpoint{3.043326in}{2.141782in}}%
\pgfpathlineto{\pgfqpoint{2.789950in}{0.798346in}}%
\pgfpathclose%
\pgfusepath{fill}%
\end{pgfscope}%
\begin{pgfscope}%
\pgfpathrectangle{\pgfqpoint{0.017500in}{0.017500in}}{\pgfqpoint{6.218282in}{3.580372in}} %
\pgfusepath{clip}%
\pgfsetbuttcap%
\pgfsetroundjoin%
\definecolor{currentfill}{rgb}{0.234377,0.305542,0.759680}%
\pgfsetfillcolor{currentfill}%
\pgfsetlinewidth{0.000000pt}%
\definecolor{currentstroke}{rgb}{0.000000,0.000000,0.000000}%
\pgfsetstrokecolor{currentstroke}%
\pgfsetdash{}{0pt}%
\pgfpathmoveto{\pgfqpoint{2.789950in}{0.798346in}}%
\pgfpathlineto{\pgfqpoint{3.043326in}{2.141782in}}%
\pgfpathlineto{\pgfqpoint{2.789950in}{0.798346in}}%
\pgfpathclose%
\pgfusepath{fill}%
\end{pgfscope}%
\begin{pgfscope}%
\pgfpathrectangle{\pgfqpoint{0.017500in}{0.017500in}}{\pgfqpoint{6.218282in}{3.580372in}} %
\pgfusepath{clip}%
\pgfsetbuttcap%
\pgfsetroundjoin%
\definecolor{currentfill}{rgb}{0.234377,0.305542,0.759680}%
\pgfsetfillcolor{currentfill}%
\pgfsetlinewidth{0.000000pt}%
\definecolor{currentstroke}{rgb}{0.000000,0.000000,0.000000}%
\pgfsetstrokecolor{currentstroke}%
\pgfsetdash{}{0pt}%
\pgfpathmoveto{\pgfqpoint{2.789950in}{0.798346in}}%
\pgfpathlineto{\pgfqpoint{3.043326in}{2.141782in}}%
\pgfpathlineto{\pgfqpoint{2.789950in}{0.798346in}}%
\pgfpathclose%
\pgfusepath{fill}%
\end{pgfscope}%
\begin{pgfscope}%
\pgfpathrectangle{\pgfqpoint{0.017500in}{0.017500in}}{\pgfqpoint{6.218282in}{3.580372in}} %
\pgfusepath{clip}%
\pgfsetbuttcap%
\pgfsetroundjoin%
\definecolor{currentfill}{rgb}{0.234377,0.305542,0.759680}%
\pgfsetfillcolor{currentfill}%
\pgfsetlinewidth{0.000000pt}%
\definecolor{currentstroke}{rgb}{0.000000,0.000000,0.000000}%
\pgfsetstrokecolor{currentstroke}%
\pgfsetdash{}{0pt}%
\pgfpathmoveto{\pgfqpoint{2.789950in}{0.798346in}}%
\pgfpathlineto{\pgfqpoint{3.043326in}{2.141782in}}%
\pgfpathlineto{\pgfqpoint{2.789950in}{0.798346in}}%
\pgfpathclose%
\pgfusepath{fill}%
\end{pgfscope}%
\begin{pgfscope}%
\pgfpathrectangle{\pgfqpoint{0.017500in}{0.017500in}}{\pgfqpoint{6.218282in}{3.580372in}} %
\pgfusepath{clip}%
\pgfsetbuttcap%
\pgfsetroundjoin%
\definecolor{currentfill}{rgb}{0.234377,0.305542,0.759680}%
\pgfsetfillcolor{currentfill}%
\pgfsetlinewidth{0.000000pt}%
\definecolor{currentstroke}{rgb}{0.000000,0.000000,0.000000}%
\pgfsetstrokecolor{currentstroke}%
\pgfsetdash{}{0pt}%
\pgfpathmoveto{\pgfqpoint{2.789950in}{0.798346in}}%
\pgfpathlineto{\pgfqpoint{3.043326in}{2.141782in}}%
\pgfpathlineto{\pgfqpoint{2.789950in}{0.798346in}}%
\pgfpathclose%
\pgfusepath{fill}%
\end{pgfscope}%
\begin{pgfscope}%
\pgfpathrectangle{\pgfqpoint{0.017500in}{0.017500in}}{\pgfqpoint{6.218282in}{3.580372in}} %
\pgfusepath{clip}%
\pgfsetbuttcap%
\pgfsetroundjoin%
\definecolor{currentfill}{rgb}{0.234377,0.305542,0.759680}%
\pgfsetfillcolor{currentfill}%
\pgfsetlinewidth{0.000000pt}%
\definecolor{currentstroke}{rgb}{0.000000,0.000000,0.000000}%
\pgfsetstrokecolor{currentstroke}%
\pgfsetdash{}{0pt}%
\pgfpathmoveto{\pgfqpoint{2.789950in}{0.798346in}}%
\pgfpathlineto{\pgfqpoint{3.043326in}{2.141782in}}%
\pgfpathlineto{\pgfqpoint{2.789950in}{0.798346in}}%
\pgfpathclose%
\pgfusepath{fill}%
\end{pgfscope}%
\begin{pgfscope}%
\pgfpathrectangle{\pgfqpoint{0.017500in}{0.017500in}}{\pgfqpoint{6.218282in}{3.580372in}} %
\pgfusepath{clip}%
\pgfsetbuttcap%
\pgfsetroundjoin%
\definecolor{currentfill}{rgb}{0.234377,0.305542,0.759680}%
\pgfsetfillcolor{currentfill}%
\pgfsetlinewidth{0.000000pt}%
\definecolor{currentstroke}{rgb}{0.000000,0.000000,0.000000}%
\pgfsetstrokecolor{currentstroke}%
\pgfsetdash{}{0pt}%
\pgfpathmoveto{\pgfqpoint{2.789950in}{0.798346in}}%
\pgfpathlineto{\pgfqpoint{3.043326in}{2.141782in}}%
\pgfpathlineto{\pgfqpoint{2.789950in}{0.798346in}}%
\pgfpathclose%
\pgfusepath{fill}%
\end{pgfscope}%
\begin{pgfscope}%
\pgfpathrectangle{\pgfqpoint{0.017500in}{0.017500in}}{\pgfqpoint{6.218282in}{3.580372in}} %
\pgfusepath{clip}%
\pgfsetbuttcap%
\pgfsetroundjoin%
\definecolor{currentfill}{rgb}{0.234377,0.305542,0.759680}%
\pgfsetfillcolor{currentfill}%
\pgfsetlinewidth{0.000000pt}%
\definecolor{currentstroke}{rgb}{0.000000,0.000000,0.000000}%
\pgfsetstrokecolor{currentstroke}%
\pgfsetdash{}{0pt}%
\pgfpathmoveto{\pgfqpoint{2.789950in}{0.798346in}}%
\pgfpathlineto{\pgfqpoint{3.043326in}{2.141782in}}%
\pgfpathlineto{\pgfqpoint{2.789950in}{0.798346in}}%
\pgfpathclose%
\pgfusepath{fill}%
\end{pgfscope}%
\begin{pgfscope}%
\pgfpathrectangle{\pgfqpoint{0.017500in}{0.017500in}}{\pgfqpoint{6.218282in}{3.580372in}} %
\pgfusepath{clip}%
\pgfsetbuttcap%
\pgfsetroundjoin%
\definecolor{currentfill}{rgb}{0.234377,0.305542,0.759680}%
\pgfsetfillcolor{currentfill}%
\pgfsetlinewidth{0.000000pt}%
\definecolor{currentstroke}{rgb}{0.000000,0.000000,0.000000}%
\pgfsetstrokecolor{currentstroke}%
\pgfsetdash{}{0pt}%
\pgfpathmoveto{\pgfqpoint{2.789950in}{0.798346in}}%
\pgfpathlineto{\pgfqpoint{3.043326in}{2.141782in}}%
\pgfpathlineto{\pgfqpoint{2.789950in}{0.798346in}}%
\pgfpathclose%
\pgfusepath{fill}%
\end{pgfscope}%
\begin{pgfscope}%
\pgfpathrectangle{\pgfqpoint{0.017500in}{0.017500in}}{\pgfqpoint{6.218282in}{3.580372in}} %
\pgfusepath{clip}%
\pgfsetbuttcap%
\pgfsetroundjoin%
\definecolor{currentfill}{rgb}{0.234377,0.305542,0.759680}%
\pgfsetfillcolor{currentfill}%
\pgfsetlinewidth{0.000000pt}%
\definecolor{currentstroke}{rgb}{0.000000,0.000000,0.000000}%
\pgfsetstrokecolor{currentstroke}%
\pgfsetdash{}{0pt}%
\pgfpathmoveto{\pgfqpoint{2.789950in}{0.798346in}}%
\pgfpathlineto{\pgfqpoint{3.043326in}{2.141782in}}%
\pgfpathlineto{\pgfqpoint{2.789950in}{0.798346in}}%
\pgfpathclose%
\pgfusepath{fill}%
\end{pgfscope}%
\begin{pgfscope}%
\pgfpathrectangle{\pgfqpoint{0.017500in}{0.017500in}}{\pgfqpoint{6.218282in}{3.580372in}} %
\pgfusepath{clip}%
\pgfsetbuttcap%
\pgfsetroundjoin%
\definecolor{currentfill}{rgb}{0.899534,0.440692,0.344107}%
\pgfsetfillcolor{currentfill}%
\pgfsetlinewidth{0.000000pt}%
\definecolor{currentstroke}{rgb}{0.000000,0.000000,0.000000}%
\pgfsetstrokecolor{currentstroke}%
\pgfsetdash{}{0pt}%
\pgfpathmoveto{\pgfqpoint{2.143341in}{2.291288in}}%
\pgfpathlineto{\pgfqpoint{2.410872in}{2.508083in}}%
\pgfpathlineto{\pgfqpoint{2.143341in}{2.291288in}}%
\pgfpathclose%
\pgfusepath{fill}%
\end{pgfscope}%
\begin{pgfscope}%
\pgfpathrectangle{\pgfqpoint{0.017500in}{0.017500in}}{\pgfqpoint{6.218282in}{3.580372in}} %
\pgfusepath{clip}%
\pgfsetbuttcap%
\pgfsetroundjoin%
\definecolor{currentfill}{rgb}{0.899534,0.440692,0.344107}%
\pgfsetfillcolor{currentfill}%
\pgfsetlinewidth{0.000000pt}%
\definecolor{currentstroke}{rgb}{0.000000,0.000000,0.000000}%
\pgfsetstrokecolor{currentstroke}%
\pgfsetdash{}{0pt}%
\pgfpathmoveto{\pgfqpoint{2.143341in}{2.291288in}}%
\pgfpathlineto{\pgfqpoint{2.410872in}{2.508083in}}%
\pgfpathlineto{\pgfqpoint{2.143341in}{2.291288in}}%
\pgfpathclose%
\pgfusepath{fill}%
\end{pgfscope}%
\begin{pgfscope}%
\pgfpathrectangle{\pgfqpoint{0.017500in}{0.017500in}}{\pgfqpoint{6.218282in}{3.580372in}} %
\pgfusepath{clip}%
\pgfsetbuttcap%
\pgfsetroundjoin%
\definecolor{currentfill}{rgb}{0.899534,0.440692,0.344107}%
\pgfsetfillcolor{currentfill}%
\pgfsetlinewidth{0.000000pt}%
\definecolor{currentstroke}{rgb}{0.000000,0.000000,0.000000}%
\pgfsetstrokecolor{currentstroke}%
\pgfsetdash{}{0pt}%
\pgfpathmoveto{\pgfqpoint{2.143341in}{2.291288in}}%
\pgfpathlineto{\pgfqpoint{2.410872in}{2.508083in}}%
\pgfpathlineto{\pgfqpoint{2.143341in}{2.291288in}}%
\pgfpathclose%
\pgfusepath{fill}%
\end{pgfscope}%
\begin{pgfscope}%
\pgfpathrectangle{\pgfqpoint{0.017500in}{0.017500in}}{\pgfqpoint{6.218282in}{3.580372in}} %
\pgfusepath{clip}%
\pgfsetbuttcap%
\pgfsetroundjoin%
\definecolor{currentfill}{rgb}{0.899534,0.440692,0.344107}%
\pgfsetfillcolor{currentfill}%
\pgfsetlinewidth{0.000000pt}%
\definecolor{currentstroke}{rgb}{0.000000,0.000000,0.000000}%
\pgfsetstrokecolor{currentstroke}%
\pgfsetdash{}{0pt}%
\pgfpathmoveto{\pgfqpoint{2.143341in}{2.291288in}}%
\pgfpathlineto{\pgfqpoint{2.410872in}{2.508083in}}%
\pgfpathlineto{\pgfqpoint{2.143341in}{2.291288in}}%
\pgfpathclose%
\pgfusepath{fill}%
\end{pgfscope}%
\begin{pgfscope}%
\pgfpathrectangle{\pgfqpoint{0.017500in}{0.017500in}}{\pgfqpoint{6.218282in}{3.580372in}} %
\pgfusepath{clip}%
\pgfsetbuttcap%
\pgfsetroundjoin%
\definecolor{currentfill}{rgb}{0.899534,0.440692,0.344107}%
\pgfsetfillcolor{currentfill}%
\pgfsetlinewidth{0.000000pt}%
\definecolor{currentstroke}{rgb}{0.000000,0.000000,0.000000}%
\pgfsetstrokecolor{currentstroke}%
\pgfsetdash{}{0pt}%
\pgfpathmoveto{\pgfqpoint{2.143341in}{2.291288in}}%
\pgfpathlineto{\pgfqpoint{2.410872in}{2.508083in}}%
\pgfpathlineto{\pgfqpoint{2.143341in}{2.291288in}}%
\pgfpathclose%
\pgfusepath{fill}%
\end{pgfscope}%
\begin{pgfscope}%
\pgfpathrectangle{\pgfqpoint{0.017500in}{0.017500in}}{\pgfqpoint{6.218282in}{3.580372in}} %
\pgfusepath{clip}%
\pgfsetbuttcap%
\pgfsetroundjoin%
\definecolor{currentfill}{rgb}{0.899534,0.440692,0.344107}%
\pgfsetfillcolor{currentfill}%
\pgfsetlinewidth{0.000000pt}%
\definecolor{currentstroke}{rgb}{0.000000,0.000000,0.000000}%
\pgfsetstrokecolor{currentstroke}%
\pgfsetdash{}{0pt}%
\pgfpathmoveto{\pgfqpoint{2.143341in}{2.291288in}}%
\pgfpathlineto{\pgfqpoint{2.410872in}{2.508083in}}%
\pgfpathlineto{\pgfqpoint{2.143341in}{2.291288in}}%
\pgfpathclose%
\pgfusepath{fill}%
\end{pgfscope}%
\begin{pgfscope}%
\pgfpathrectangle{\pgfqpoint{0.017500in}{0.017500in}}{\pgfqpoint{6.218282in}{3.580372in}} %
\pgfusepath{clip}%
\pgfsetbuttcap%
\pgfsetroundjoin%
\definecolor{currentfill}{rgb}{0.899534,0.440692,0.344107}%
\pgfsetfillcolor{currentfill}%
\pgfsetlinewidth{0.000000pt}%
\definecolor{currentstroke}{rgb}{0.000000,0.000000,0.000000}%
\pgfsetstrokecolor{currentstroke}%
\pgfsetdash{}{0pt}%
\pgfpathmoveto{\pgfqpoint{2.143341in}{2.291288in}}%
\pgfpathlineto{\pgfqpoint{2.410872in}{2.508083in}}%
\pgfpathlineto{\pgfqpoint{2.143341in}{2.291288in}}%
\pgfpathclose%
\pgfusepath{fill}%
\end{pgfscope}%
\begin{pgfscope}%
\pgfpathrectangle{\pgfqpoint{0.017500in}{0.017500in}}{\pgfqpoint{6.218282in}{3.580372in}} %
\pgfusepath{clip}%
\pgfsetbuttcap%
\pgfsetroundjoin%
\definecolor{currentfill}{rgb}{0.899534,0.440692,0.344107}%
\pgfsetfillcolor{currentfill}%
\pgfsetlinewidth{0.000000pt}%
\definecolor{currentstroke}{rgb}{0.000000,0.000000,0.000000}%
\pgfsetstrokecolor{currentstroke}%
\pgfsetdash{}{0pt}%
\pgfpathmoveto{\pgfqpoint{2.143341in}{2.291288in}}%
\pgfpathlineto{\pgfqpoint{2.410872in}{2.508083in}}%
\pgfpathlineto{\pgfqpoint{2.143341in}{2.291288in}}%
\pgfpathclose%
\pgfusepath{fill}%
\end{pgfscope}%
\begin{pgfscope}%
\pgfpathrectangle{\pgfqpoint{0.017500in}{0.017500in}}{\pgfqpoint{6.218282in}{3.580372in}} %
\pgfusepath{clip}%
\pgfsetbuttcap%
\pgfsetroundjoin%
\definecolor{currentfill}{rgb}{0.899534,0.440692,0.344107}%
\pgfsetfillcolor{currentfill}%
\pgfsetlinewidth{0.000000pt}%
\definecolor{currentstroke}{rgb}{0.000000,0.000000,0.000000}%
\pgfsetstrokecolor{currentstroke}%
\pgfsetdash{}{0pt}%
\pgfpathmoveto{\pgfqpoint{2.143341in}{2.291288in}}%
\pgfpathlineto{\pgfqpoint{2.410872in}{2.508083in}}%
\pgfpathlineto{\pgfqpoint{2.143341in}{2.291288in}}%
\pgfpathclose%
\pgfusepath{fill}%
\end{pgfscope}%
\begin{pgfscope}%
\pgfpathrectangle{\pgfqpoint{0.017500in}{0.017500in}}{\pgfqpoint{6.218282in}{3.580372in}} %
\pgfusepath{clip}%
\pgfsetbuttcap%
\pgfsetroundjoin%
\definecolor{currentfill}{rgb}{0.899534,0.440692,0.344107}%
\pgfsetfillcolor{currentfill}%
\pgfsetlinewidth{0.000000pt}%
\definecolor{currentstroke}{rgb}{0.000000,0.000000,0.000000}%
\pgfsetstrokecolor{currentstroke}%
\pgfsetdash{}{0pt}%
\pgfpathmoveto{\pgfqpoint{2.143341in}{2.291288in}}%
\pgfpathlineto{\pgfqpoint{2.410872in}{2.508083in}}%
\pgfpathlineto{\pgfqpoint{2.143341in}{2.291288in}}%
\pgfpathclose%
\pgfusepath{fill}%
\end{pgfscope}%
\begin{pgfscope}%
\pgfpathrectangle{\pgfqpoint{0.017500in}{0.017500in}}{\pgfqpoint{6.218282in}{3.580372in}} %
\pgfusepath{clip}%
\pgfsetbuttcap%
\pgfsetroundjoin%
\definecolor{currentfill}{rgb}{0.899534,0.440692,0.344107}%
\pgfsetfillcolor{currentfill}%
\pgfsetlinewidth{0.000000pt}%
\definecolor{currentstroke}{rgb}{0.000000,0.000000,0.000000}%
\pgfsetstrokecolor{currentstroke}%
\pgfsetdash{}{0pt}%
\pgfpathmoveto{\pgfqpoint{2.143341in}{2.291288in}}%
\pgfpathlineto{\pgfqpoint{2.410872in}{2.508083in}}%
\pgfpathlineto{\pgfqpoint{2.143341in}{2.291288in}}%
\pgfpathclose%
\pgfusepath{fill}%
\end{pgfscope}%
\begin{pgfscope}%
\pgfpathrectangle{\pgfqpoint{0.017500in}{0.017500in}}{\pgfqpoint{6.218282in}{3.580372in}} %
\pgfusepath{clip}%
\pgfsetbuttcap%
\pgfsetroundjoin%
\definecolor{currentfill}{rgb}{0.899534,0.440692,0.344107}%
\pgfsetfillcolor{currentfill}%
\pgfsetlinewidth{0.000000pt}%
\definecolor{currentstroke}{rgb}{0.000000,0.000000,0.000000}%
\pgfsetstrokecolor{currentstroke}%
\pgfsetdash{}{0pt}%
\pgfpathmoveto{\pgfqpoint{2.143341in}{2.291288in}}%
\pgfpathlineto{\pgfqpoint{2.410872in}{2.508083in}}%
\pgfpathlineto{\pgfqpoint{2.143341in}{2.291288in}}%
\pgfpathclose%
\pgfusepath{fill}%
\end{pgfscope}%
\begin{pgfscope}%
\pgfpathrectangle{\pgfqpoint{0.017500in}{0.017500in}}{\pgfqpoint{6.218282in}{3.580372in}} %
\pgfusepath{clip}%
\pgfsetbuttcap%
\pgfsetroundjoin%
\definecolor{currentfill}{rgb}{0.899534,0.440692,0.344107}%
\pgfsetfillcolor{currentfill}%
\pgfsetlinewidth{0.000000pt}%
\definecolor{currentstroke}{rgb}{0.000000,0.000000,0.000000}%
\pgfsetstrokecolor{currentstroke}%
\pgfsetdash{}{0pt}%
\pgfpathmoveto{\pgfqpoint{2.143341in}{2.291288in}}%
\pgfpathlineto{\pgfqpoint{2.410872in}{2.508083in}}%
\pgfpathlineto{\pgfqpoint{2.143341in}{2.291288in}}%
\pgfpathclose%
\pgfusepath{fill}%
\end{pgfscope}%
\begin{pgfscope}%
\pgfpathrectangle{\pgfqpoint{0.017500in}{0.017500in}}{\pgfqpoint{6.218282in}{3.580372in}} %
\pgfusepath{clip}%
\pgfsetbuttcap%
\pgfsetroundjoin%
\definecolor{currentfill}{rgb}{0.899534,0.440692,0.344107}%
\pgfsetfillcolor{currentfill}%
\pgfsetlinewidth{0.000000pt}%
\definecolor{currentstroke}{rgb}{0.000000,0.000000,0.000000}%
\pgfsetstrokecolor{currentstroke}%
\pgfsetdash{}{0pt}%
\pgfpathmoveto{\pgfqpoint{2.143341in}{2.291288in}}%
\pgfpathlineto{\pgfqpoint{2.410872in}{2.508083in}}%
\pgfpathlineto{\pgfqpoint{2.143341in}{2.291288in}}%
\pgfpathclose%
\pgfusepath{fill}%
\end{pgfscope}%
\begin{pgfscope}%
\pgfpathrectangle{\pgfqpoint{0.017500in}{0.017500in}}{\pgfqpoint{6.218282in}{3.580372in}} %
\pgfusepath{clip}%
\pgfsetbuttcap%
\pgfsetroundjoin%
\definecolor{currentfill}{rgb}{0.899534,0.440692,0.344107}%
\pgfsetfillcolor{currentfill}%
\pgfsetlinewidth{0.000000pt}%
\definecolor{currentstroke}{rgb}{0.000000,0.000000,0.000000}%
\pgfsetstrokecolor{currentstroke}%
\pgfsetdash{}{0pt}%
\pgfpathmoveto{\pgfqpoint{2.143341in}{2.291288in}}%
\pgfpathlineto{\pgfqpoint{2.410872in}{2.508083in}}%
\pgfpathlineto{\pgfqpoint{2.143341in}{2.291288in}}%
\pgfpathclose%
\pgfusepath{fill}%
\end{pgfscope}%
\begin{pgfscope}%
\pgfpathrectangle{\pgfqpoint{0.017500in}{0.017500in}}{\pgfqpoint{6.218282in}{3.580372in}} %
\pgfusepath{clip}%
\pgfsetbuttcap%
\pgfsetroundjoin%
\definecolor{currentfill}{rgb}{0.899534,0.440692,0.344107}%
\pgfsetfillcolor{currentfill}%
\pgfsetlinewidth{0.000000pt}%
\definecolor{currentstroke}{rgb}{0.000000,0.000000,0.000000}%
\pgfsetstrokecolor{currentstroke}%
\pgfsetdash{}{0pt}%
\pgfpathmoveto{\pgfqpoint{2.143341in}{2.291288in}}%
\pgfpathlineto{\pgfqpoint{2.410872in}{2.508083in}}%
\pgfpathlineto{\pgfqpoint{2.143341in}{2.291288in}}%
\pgfpathclose%
\pgfusepath{fill}%
\end{pgfscope}%
\begin{pgfscope}%
\pgfpathrectangle{\pgfqpoint{0.017500in}{0.017500in}}{\pgfqpoint{6.218282in}{3.580372in}} %
\pgfusepath{clip}%
\pgfsetbuttcap%
\pgfsetroundjoin%
\definecolor{currentfill}{rgb}{0.899534,0.440692,0.344107}%
\pgfsetfillcolor{currentfill}%
\pgfsetlinewidth{0.000000pt}%
\definecolor{currentstroke}{rgb}{0.000000,0.000000,0.000000}%
\pgfsetstrokecolor{currentstroke}%
\pgfsetdash{}{0pt}%
\pgfpathmoveto{\pgfqpoint{2.143341in}{2.291288in}}%
\pgfpathlineto{\pgfqpoint{2.410872in}{2.508083in}}%
\pgfpathlineto{\pgfqpoint{2.143341in}{2.291288in}}%
\pgfpathclose%
\pgfusepath{fill}%
\end{pgfscope}%
\begin{pgfscope}%
\pgfpathrectangle{\pgfqpoint{0.017500in}{0.017500in}}{\pgfqpoint{6.218282in}{3.580372in}} %
\pgfusepath{clip}%
\pgfsetbuttcap%
\pgfsetroundjoin%
\definecolor{currentfill}{rgb}{0.899534,0.440692,0.344107}%
\pgfsetfillcolor{currentfill}%
\pgfsetlinewidth{0.000000pt}%
\definecolor{currentstroke}{rgb}{0.000000,0.000000,0.000000}%
\pgfsetstrokecolor{currentstroke}%
\pgfsetdash{}{0pt}%
\pgfpathmoveto{\pgfqpoint{2.143341in}{2.291288in}}%
\pgfpathlineto{\pgfqpoint{2.410872in}{2.508083in}}%
\pgfpathlineto{\pgfqpoint{2.143341in}{2.291288in}}%
\pgfpathclose%
\pgfusepath{fill}%
\end{pgfscope}%
\begin{pgfscope}%
\pgfpathrectangle{\pgfqpoint{0.017500in}{0.017500in}}{\pgfqpoint{6.218282in}{3.580372in}} %
\pgfusepath{clip}%
\pgfsetbuttcap%
\pgfsetroundjoin%
\definecolor{currentfill}{rgb}{0.899534,0.440692,0.344107}%
\pgfsetfillcolor{currentfill}%
\pgfsetlinewidth{0.000000pt}%
\definecolor{currentstroke}{rgb}{0.000000,0.000000,0.000000}%
\pgfsetstrokecolor{currentstroke}%
\pgfsetdash{}{0pt}%
\pgfpathmoveto{\pgfqpoint{2.143341in}{2.291288in}}%
\pgfpathlineto{\pgfqpoint{2.410872in}{2.508083in}}%
\pgfpathlineto{\pgfqpoint{2.143341in}{2.291288in}}%
\pgfpathclose%
\pgfusepath{fill}%
\end{pgfscope}%
\begin{pgfscope}%
\pgfpathrectangle{\pgfqpoint{0.017500in}{0.017500in}}{\pgfqpoint{6.218282in}{3.580372in}} %
\pgfusepath{clip}%
\pgfsetbuttcap%
\pgfsetroundjoin%
\definecolor{currentfill}{rgb}{0.899534,0.440692,0.344107}%
\pgfsetfillcolor{currentfill}%
\pgfsetlinewidth{0.000000pt}%
\definecolor{currentstroke}{rgb}{0.000000,0.000000,0.000000}%
\pgfsetstrokecolor{currentstroke}%
\pgfsetdash{}{0pt}%
\pgfpathmoveto{\pgfqpoint{2.143341in}{2.291288in}}%
\pgfpathlineto{\pgfqpoint{2.410872in}{2.508083in}}%
\pgfpathlineto{\pgfqpoint{2.143341in}{2.291288in}}%
\pgfpathclose%
\pgfusepath{fill}%
\end{pgfscope}%
\begin{pgfscope}%
\pgfpathrectangle{\pgfqpoint{0.017500in}{0.017500in}}{\pgfqpoint{6.218282in}{3.580372in}} %
\pgfusepath{clip}%
\pgfsetbuttcap%
\pgfsetroundjoin%
\definecolor{currentfill}{rgb}{0.899534,0.440692,0.344107}%
\pgfsetfillcolor{currentfill}%
\pgfsetlinewidth{0.000000pt}%
\definecolor{currentstroke}{rgb}{0.000000,0.000000,0.000000}%
\pgfsetstrokecolor{currentstroke}%
\pgfsetdash{}{0pt}%
\pgfpathmoveto{\pgfqpoint{2.143341in}{2.291288in}}%
\pgfpathlineto{\pgfqpoint{2.410872in}{2.508083in}}%
\pgfpathlineto{\pgfqpoint{2.143341in}{2.291288in}}%
\pgfpathclose%
\pgfusepath{fill}%
\end{pgfscope}%
\begin{pgfscope}%
\pgfpathrectangle{\pgfqpoint{0.017500in}{0.017500in}}{\pgfqpoint{6.218282in}{3.580372in}} %
\pgfusepath{clip}%
\pgfsetbuttcap%
\pgfsetroundjoin%
\definecolor{currentfill}{rgb}{0.899534,0.440692,0.344107}%
\pgfsetfillcolor{currentfill}%
\pgfsetlinewidth{0.000000pt}%
\definecolor{currentstroke}{rgb}{0.000000,0.000000,0.000000}%
\pgfsetstrokecolor{currentstroke}%
\pgfsetdash{}{0pt}%
\pgfpathmoveto{\pgfqpoint{2.143341in}{2.291288in}}%
\pgfpathlineto{\pgfqpoint{2.410872in}{2.508083in}}%
\pgfpathlineto{\pgfqpoint{2.143341in}{2.291288in}}%
\pgfpathclose%
\pgfusepath{fill}%
\end{pgfscope}%
\begin{pgfscope}%
\pgfpathrectangle{\pgfqpoint{0.017500in}{0.017500in}}{\pgfqpoint{6.218282in}{3.580372in}} %
\pgfusepath{clip}%
\pgfsetbuttcap%
\pgfsetroundjoin%
\definecolor{currentfill}{rgb}{0.899534,0.440692,0.344107}%
\pgfsetfillcolor{currentfill}%
\pgfsetlinewidth{0.000000pt}%
\definecolor{currentstroke}{rgb}{0.000000,0.000000,0.000000}%
\pgfsetstrokecolor{currentstroke}%
\pgfsetdash{}{0pt}%
\pgfpathmoveto{\pgfqpoint{2.143341in}{2.291288in}}%
\pgfpathlineto{\pgfqpoint{2.410872in}{2.508083in}}%
\pgfpathlineto{\pgfqpoint{2.143341in}{2.291288in}}%
\pgfpathclose%
\pgfusepath{fill}%
\end{pgfscope}%
\begin{pgfscope}%
\pgfpathrectangle{\pgfqpoint{0.017500in}{0.017500in}}{\pgfqpoint{6.218282in}{3.580372in}} %
\pgfusepath{clip}%
\pgfsetbuttcap%
\pgfsetroundjoin%
\definecolor{currentfill}{rgb}{0.899534,0.440692,0.344107}%
\pgfsetfillcolor{currentfill}%
\pgfsetlinewidth{0.000000pt}%
\definecolor{currentstroke}{rgb}{0.000000,0.000000,0.000000}%
\pgfsetstrokecolor{currentstroke}%
\pgfsetdash{}{0pt}%
\pgfpathmoveto{\pgfqpoint{2.143341in}{2.291288in}}%
\pgfpathlineto{\pgfqpoint{2.410872in}{2.508083in}}%
\pgfpathlineto{\pgfqpoint{2.143341in}{2.291288in}}%
\pgfpathclose%
\pgfusepath{fill}%
\end{pgfscope}%
\begin{pgfscope}%
\pgfpathrectangle{\pgfqpoint{0.017500in}{0.017500in}}{\pgfqpoint{6.218282in}{3.580372in}} %
\pgfusepath{clip}%
\pgfsetbuttcap%
\pgfsetroundjoin%
\definecolor{currentfill}{rgb}{0.899534,0.440692,0.344107}%
\pgfsetfillcolor{currentfill}%
\pgfsetlinewidth{0.000000pt}%
\definecolor{currentstroke}{rgb}{0.000000,0.000000,0.000000}%
\pgfsetstrokecolor{currentstroke}%
\pgfsetdash{}{0pt}%
\pgfpathmoveto{\pgfqpoint{2.143341in}{2.291288in}}%
\pgfpathlineto{\pgfqpoint{2.410872in}{2.508083in}}%
\pgfpathlineto{\pgfqpoint{2.143341in}{2.291288in}}%
\pgfpathclose%
\pgfusepath{fill}%
\end{pgfscope}%
\begin{pgfscope}%
\pgfpathrectangle{\pgfqpoint{0.017500in}{0.017500in}}{\pgfqpoint{6.218282in}{3.580372in}} %
\pgfusepath{clip}%
\pgfsetbuttcap%
\pgfsetroundjoin%
\definecolor{currentfill}{rgb}{0.899534,0.440692,0.344107}%
\pgfsetfillcolor{currentfill}%
\pgfsetlinewidth{0.000000pt}%
\definecolor{currentstroke}{rgb}{0.000000,0.000000,0.000000}%
\pgfsetstrokecolor{currentstroke}%
\pgfsetdash{}{0pt}%
\pgfpathmoveto{\pgfqpoint{2.143341in}{2.291288in}}%
\pgfpathlineto{\pgfqpoint{2.410872in}{2.508083in}}%
\pgfpathlineto{\pgfqpoint{2.143341in}{2.291288in}}%
\pgfpathclose%
\pgfusepath{fill}%
\end{pgfscope}%
\begin{pgfscope}%
\pgfpathrectangle{\pgfqpoint{0.017500in}{0.017500in}}{\pgfqpoint{6.218282in}{3.580372in}} %
\pgfusepath{clip}%
\pgfsetbuttcap%
\pgfsetroundjoin%
\definecolor{currentfill}{rgb}{0.899534,0.440692,0.344107}%
\pgfsetfillcolor{currentfill}%
\pgfsetlinewidth{0.000000pt}%
\definecolor{currentstroke}{rgb}{0.000000,0.000000,0.000000}%
\pgfsetstrokecolor{currentstroke}%
\pgfsetdash{}{0pt}%
\pgfpathmoveto{\pgfqpoint{2.143341in}{2.291288in}}%
\pgfpathlineto{\pgfqpoint{2.410872in}{2.508083in}}%
\pgfpathlineto{\pgfqpoint{2.143341in}{2.291288in}}%
\pgfpathclose%
\pgfusepath{fill}%
\end{pgfscope}%
\begin{pgfscope}%
\pgfpathrectangle{\pgfqpoint{0.017500in}{0.017500in}}{\pgfqpoint{6.218282in}{3.580372in}} %
\pgfusepath{clip}%
\pgfsetbuttcap%
\pgfsetroundjoin%
\definecolor{currentfill}{rgb}{0.899534,0.440692,0.344107}%
\pgfsetfillcolor{currentfill}%
\pgfsetlinewidth{0.000000pt}%
\definecolor{currentstroke}{rgb}{0.000000,0.000000,0.000000}%
\pgfsetstrokecolor{currentstroke}%
\pgfsetdash{}{0pt}%
\pgfpathmoveto{\pgfqpoint{2.143341in}{2.291288in}}%
\pgfpathlineto{\pgfqpoint{2.410872in}{2.508083in}}%
\pgfpathlineto{\pgfqpoint{2.143341in}{2.291288in}}%
\pgfpathclose%
\pgfusepath{fill}%
\end{pgfscope}%
\begin{pgfscope}%
\pgfpathrectangle{\pgfqpoint{0.017500in}{0.017500in}}{\pgfqpoint{6.218282in}{3.580372in}} %
\pgfusepath{clip}%
\pgfsetbuttcap%
\pgfsetroundjoin%
\definecolor{currentfill}{rgb}{0.899534,0.440692,0.344107}%
\pgfsetfillcolor{currentfill}%
\pgfsetlinewidth{0.000000pt}%
\definecolor{currentstroke}{rgb}{0.000000,0.000000,0.000000}%
\pgfsetstrokecolor{currentstroke}%
\pgfsetdash{}{0pt}%
\pgfpathmoveto{\pgfqpoint{2.143341in}{2.291288in}}%
\pgfpathlineto{\pgfqpoint{2.410872in}{2.508083in}}%
\pgfpathlineto{\pgfqpoint{2.143341in}{2.291288in}}%
\pgfpathclose%
\pgfusepath{fill}%
\end{pgfscope}%
\begin{pgfscope}%
\pgfpathrectangle{\pgfqpoint{0.017500in}{0.017500in}}{\pgfqpoint{6.218282in}{3.580372in}} %
\pgfusepath{clip}%
\pgfsetbuttcap%
\pgfsetroundjoin%
\definecolor{currentfill}{rgb}{0.899534,0.440692,0.344107}%
\pgfsetfillcolor{currentfill}%
\pgfsetlinewidth{0.000000pt}%
\definecolor{currentstroke}{rgb}{0.000000,0.000000,0.000000}%
\pgfsetstrokecolor{currentstroke}%
\pgfsetdash{}{0pt}%
\pgfpathmoveto{\pgfqpoint{2.143341in}{2.291288in}}%
\pgfpathlineto{\pgfqpoint{2.410872in}{2.508083in}}%
\pgfpathlineto{\pgfqpoint{2.143341in}{2.291288in}}%
\pgfpathclose%
\pgfusepath{fill}%
\end{pgfscope}%
\begin{pgfscope}%
\pgfpathrectangle{\pgfqpoint{0.017500in}{0.017500in}}{\pgfqpoint{6.218282in}{3.580372in}} %
\pgfusepath{clip}%
\pgfsetbuttcap%
\pgfsetroundjoin%
\definecolor{currentfill}{rgb}{0.763520,0.178667,0.193396}%
\pgfsetfillcolor{currentfill}%
\pgfsetlinewidth{0.000000pt}%
\definecolor{currentstroke}{rgb}{0.000000,0.000000,0.000000}%
\pgfsetstrokecolor{currentstroke}%
\pgfsetdash{}{0pt}%
\pgfpathmoveto{\pgfqpoint{4.018447in}{2.468167in}}%
\pgfpathlineto{\pgfqpoint{4.267628in}{2.613520in}}%
\pgfpathlineto{\pgfqpoint{4.018447in}{2.468167in}}%
\pgfpathclose%
\pgfusepath{fill}%
\end{pgfscope}%
\begin{pgfscope}%
\pgfpathrectangle{\pgfqpoint{0.017500in}{0.017500in}}{\pgfqpoint{6.218282in}{3.580372in}} %
\pgfusepath{clip}%
\pgfsetbuttcap%
\pgfsetroundjoin%
\definecolor{currentfill}{rgb}{0.763520,0.178667,0.193396}%
\pgfsetfillcolor{currentfill}%
\pgfsetlinewidth{0.000000pt}%
\definecolor{currentstroke}{rgb}{0.000000,0.000000,0.000000}%
\pgfsetstrokecolor{currentstroke}%
\pgfsetdash{}{0pt}%
\pgfpathmoveto{\pgfqpoint{4.018447in}{2.468167in}}%
\pgfpathlineto{\pgfqpoint{4.267628in}{2.613520in}}%
\pgfpathlineto{\pgfqpoint{4.018447in}{2.468167in}}%
\pgfpathclose%
\pgfusepath{fill}%
\end{pgfscope}%
\begin{pgfscope}%
\pgfpathrectangle{\pgfqpoint{0.017500in}{0.017500in}}{\pgfqpoint{6.218282in}{3.580372in}} %
\pgfusepath{clip}%
\pgfsetbuttcap%
\pgfsetroundjoin%
\definecolor{currentfill}{rgb}{0.763520,0.178667,0.193396}%
\pgfsetfillcolor{currentfill}%
\pgfsetlinewidth{0.000000pt}%
\definecolor{currentstroke}{rgb}{0.000000,0.000000,0.000000}%
\pgfsetstrokecolor{currentstroke}%
\pgfsetdash{}{0pt}%
\pgfpathmoveto{\pgfqpoint{4.018447in}{2.468167in}}%
\pgfpathlineto{\pgfqpoint{4.267628in}{2.613520in}}%
\pgfpathlineto{\pgfqpoint{4.018447in}{2.468167in}}%
\pgfpathclose%
\pgfusepath{fill}%
\end{pgfscope}%
\begin{pgfscope}%
\pgfpathrectangle{\pgfqpoint{0.017500in}{0.017500in}}{\pgfqpoint{6.218282in}{3.580372in}} %
\pgfusepath{clip}%
\pgfsetbuttcap%
\pgfsetroundjoin%
\definecolor{currentfill}{rgb}{0.763520,0.178667,0.193396}%
\pgfsetfillcolor{currentfill}%
\pgfsetlinewidth{0.000000pt}%
\definecolor{currentstroke}{rgb}{0.000000,0.000000,0.000000}%
\pgfsetstrokecolor{currentstroke}%
\pgfsetdash{}{0pt}%
\pgfpathmoveto{\pgfqpoint{4.018447in}{2.468167in}}%
\pgfpathlineto{\pgfqpoint{4.267628in}{2.613520in}}%
\pgfpathlineto{\pgfqpoint{4.018447in}{2.468167in}}%
\pgfpathclose%
\pgfusepath{fill}%
\end{pgfscope}%
\begin{pgfscope}%
\pgfpathrectangle{\pgfqpoint{0.017500in}{0.017500in}}{\pgfqpoint{6.218282in}{3.580372in}} %
\pgfusepath{clip}%
\pgfsetbuttcap%
\pgfsetroundjoin%
\definecolor{currentfill}{rgb}{0.763520,0.178667,0.193396}%
\pgfsetfillcolor{currentfill}%
\pgfsetlinewidth{0.000000pt}%
\definecolor{currentstroke}{rgb}{0.000000,0.000000,0.000000}%
\pgfsetstrokecolor{currentstroke}%
\pgfsetdash{}{0pt}%
\pgfpathmoveto{\pgfqpoint{4.018447in}{2.468167in}}%
\pgfpathlineto{\pgfqpoint{4.267628in}{2.613520in}}%
\pgfpathlineto{\pgfqpoint{4.018447in}{2.468167in}}%
\pgfpathclose%
\pgfusepath{fill}%
\end{pgfscope}%
\begin{pgfscope}%
\pgfpathrectangle{\pgfqpoint{0.017500in}{0.017500in}}{\pgfqpoint{6.218282in}{3.580372in}} %
\pgfusepath{clip}%
\pgfsetbuttcap%
\pgfsetroundjoin%
\definecolor{currentfill}{rgb}{0.763520,0.178667,0.193396}%
\pgfsetfillcolor{currentfill}%
\pgfsetlinewidth{0.000000pt}%
\definecolor{currentstroke}{rgb}{0.000000,0.000000,0.000000}%
\pgfsetstrokecolor{currentstroke}%
\pgfsetdash{}{0pt}%
\pgfpathmoveto{\pgfqpoint{4.018447in}{2.468167in}}%
\pgfpathlineto{\pgfqpoint{4.267628in}{2.613520in}}%
\pgfpathlineto{\pgfqpoint{4.018447in}{2.468167in}}%
\pgfpathclose%
\pgfusepath{fill}%
\end{pgfscope}%
\begin{pgfscope}%
\pgfpathrectangle{\pgfqpoint{0.017500in}{0.017500in}}{\pgfqpoint{6.218282in}{3.580372in}} %
\pgfusepath{clip}%
\pgfsetbuttcap%
\pgfsetroundjoin%
\definecolor{currentfill}{rgb}{0.763520,0.178667,0.193396}%
\pgfsetfillcolor{currentfill}%
\pgfsetlinewidth{0.000000pt}%
\definecolor{currentstroke}{rgb}{0.000000,0.000000,0.000000}%
\pgfsetstrokecolor{currentstroke}%
\pgfsetdash{}{0pt}%
\pgfpathmoveto{\pgfqpoint{4.018447in}{2.468167in}}%
\pgfpathlineto{\pgfqpoint{4.267628in}{2.613520in}}%
\pgfpathlineto{\pgfqpoint{4.018447in}{2.468167in}}%
\pgfpathclose%
\pgfusepath{fill}%
\end{pgfscope}%
\begin{pgfscope}%
\pgfpathrectangle{\pgfqpoint{0.017500in}{0.017500in}}{\pgfqpoint{6.218282in}{3.580372in}} %
\pgfusepath{clip}%
\pgfsetbuttcap%
\pgfsetroundjoin%
\definecolor{currentfill}{rgb}{0.763520,0.178667,0.193396}%
\pgfsetfillcolor{currentfill}%
\pgfsetlinewidth{0.000000pt}%
\definecolor{currentstroke}{rgb}{0.000000,0.000000,0.000000}%
\pgfsetstrokecolor{currentstroke}%
\pgfsetdash{}{0pt}%
\pgfpathmoveto{\pgfqpoint{4.018447in}{2.468167in}}%
\pgfpathlineto{\pgfqpoint{4.267628in}{2.613520in}}%
\pgfpathlineto{\pgfqpoint{4.018447in}{2.468167in}}%
\pgfpathclose%
\pgfusepath{fill}%
\end{pgfscope}%
\begin{pgfscope}%
\pgfpathrectangle{\pgfqpoint{0.017500in}{0.017500in}}{\pgfqpoint{6.218282in}{3.580372in}} %
\pgfusepath{clip}%
\pgfsetbuttcap%
\pgfsetroundjoin%
\definecolor{currentfill}{rgb}{0.763520,0.178667,0.193396}%
\pgfsetfillcolor{currentfill}%
\pgfsetlinewidth{0.000000pt}%
\definecolor{currentstroke}{rgb}{0.000000,0.000000,0.000000}%
\pgfsetstrokecolor{currentstroke}%
\pgfsetdash{}{0pt}%
\pgfpathmoveto{\pgfqpoint{4.018447in}{2.468167in}}%
\pgfpathlineto{\pgfqpoint{4.267628in}{2.613520in}}%
\pgfpathlineto{\pgfqpoint{4.018447in}{2.468167in}}%
\pgfpathclose%
\pgfusepath{fill}%
\end{pgfscope}%
\begin{pgfscope}%
\pgfpathrectangle{\pgfqpoint{0.017500in}{0.017500in}}{\pgfqpoint{6.218282in}{3.580372in}} %
\pgfusepath{clip}%
\pgfsetbuttcap%
\pgfsetroundjoin%
\definecolor{currentfill}{rgb}{0.763520,0.178667,0.193396}%
\pgfsetfillcolor{currentfill}%
\pgfsetlinewidth{0.000000pt}%
\definecolor{currentstroke}{rgb}{0.000000,0.000000,0.000000}%
\pgfsetstrokecolor{currentstroke}%
\pgfsetdash{}{0pt}%
\pgfpathmoveto{\pgfqpoint{4.018447in}{2.468167in}}%
\pgfpathlineto{\pgfqpoint{4.267628in}{2.613520in}}%
\pgfpathlineto{\pgfqpoint{4.018447in}{2.468167in}}%
\pgfpathclose%
\pgfusepath{fill}%
\end{pgfscope}%
\begin{pgfscope}%
\pgfpathrectangle{\pgfqpoint{0.017500in}{0.017500in}}{\pgfqpoint{6.218282in}{3.580372in}} %
\pgfusepath{clip}%
\pgfsetbuttcap%
\pgfsetroundjoin%
\definecolor{currentfill}{rgb}{0.763520,0.178667,0.193396}%
\pgfsetfillcolor{currentfill}%
\pgfsetlinewidth{0.000000pt}%
\definecolor{currentstroke}{rgb}{0.000000,0.000000,0.000000}%
\pgfsetstrokecolor{currentstroke}%
\pgfsetdash{}{0pt}%
\pgfpathmoveto{\pgfqpoint{4.018447in}{2.468167in}}%
\pgfpathlineto{\pgfqpoint{4.267628in}{2.613520in}}%
\pgfpathlineto{\pgfqpoint{4.018447in}{2.468167in}}%
\pgfpathclose%
\pgfusepath{fill}%
\end{pgfscope}%
\begin{pgfscope}%
\pgfpathrectangle{\pgfqpoint{0.017500in}{0.017500in}}{\pgfqpoint{6.218282in}{3.580372in}} %
\pgfusepath{clip}%
\pgfsetbuttcap%
\pgfsetroundjoin%
\definecolor{currentfill}{rgb}{0.763520,0.178667,0.193396}%
\pgfsetfillcolor{currentfill}%
\pgfsetlinewidth{0.000000pt}%
\definecolor{currentstroke}{rgb}{0.000000,0.000000,0.000000}%
\pgfsetstrokecolor{currentstroke}%
\pgfsetdash{}{0pt}%
\pgfpathmoveto{\pgfqpoint{4.018447in}{2.468167in}}%
\pgfpathlineto{\pgfqpoint{4.267628in}{2.613520in}}%
\pgfpathlineto{\pgfqpoint{4.018447in}{2.468167in}}%
\pgfpathclose%
\pgfusepath{fill}%
\end{pgfscope}%
\begin{pgfscope}%
\pgfpathrectangle{\pgfqpoint{0.017500in}{0.017500in}}{\pgfqpoint{6.218282in}{3.580372in}} %
\pgfusepath{clip}%
\pgfsetbuttcap%
\pgfsetroundjoin%
\definecolor{currentfill}{rgb}{0.763520,0.178667,0.193396}%
\pgfsetfillcolor{currentfill}%
\pgfsetlinewidth{0.000000pt}%
\definecolor{currentstroke}{rgb}{0.000000,0.000000,0.000000}%
\pgfsetstrokecolor{currentstroke}%
\pgfsetdash{}{0pt}%
\pgfpathmoveto{\pgfqpoint{4.018447in}{2.468167in}}%
\pgfpathlineto{\pgfqpoint{4.267628in}{2.613520in}}%
\pgfpathlineto{\pgfqpoint{4.018447in}{2.468167in}}%
\pgfpathclose%
\pgfusepath{fill}%
\end{pgfscope}%
\begin{pgfscope}%
\pgfpathrectangle{\pgfqpoint{0.017500in}{0.017500in}}{\pgfqpoint{6.218282in}{3.580372in}} %
\pgfusepath{clip}%
\pgfsetbuttcap%
\pgfsetroundjoin%
\definecolor{currentfill}{rgb}{0.763520,0.178667,0.193396}%
\pgfsetfillcolor{currentfill}%
\pgfsetlinewidth{0.000000pt}%
\definecolor{currentstroke}{rgb}{0.000000,0.000000,0.000000}%
\pgfsetstrokecolor{currentstroke}%
\pgfsetdash{}{0pt}%
\pgfpathmoveto{\pgfqpoint{4.018447in}{2.468167in}}%
\pgfpathlineto{\pgfqpoint{4.267628in}{2.613520in}}%
\pgfpathlineto{\pgfqpoint{4.018447in}{2.468167in}}%
\pgfpathclose%
\pgfusepath{fill}%
\end{pgfscope}%
\begin{pgfscope}%
\pgfpathrectangle{\pgfqpoint{0.017500in}{0.017500in}}{\pgfqpoint{6.218282in}{3.580372in}} %
\pgfusepath{clip}%
\pgfsetbuttcap%
\pgfsetroundjoin%
\definecolor{currentfill}{rgb}{0.763520,0.178667,0.193396}%
\pgfsetfillcolor{currentfill}%
\pgfsetlinewidth{0.000000pt}%
\definecolor{currentstroke}{rgb}{0.000000,0.000000,0.000000}%
\pgfsetstrokecolor{currentstroke}%
\pgfsetdash{}{0pt}%
\pgfpathmoveto{\pgfqpoint{4.018447in}{2.468167in}}%
\pgfpathlineto{\pgfqpoint{4.267628in}{2.613520in}}%
\pgfpathlineto{\pgfqpoint{4.018447in}{2.468167in}}%
\pgfpathclose%
\pgfusepath{fill}%
\end{pgfscope}%
\begin{pgfscope}%
\pgfpathrectangle{\pgfqpoint{0.017500in}{0.017500in}}{\pgfqpoint{6.218282in}{3.580372in}} %
\pgfusepath{clip}%
\pgfsetbuttcap%
\pgfsetroundjoin%
\definecolor{currentfill}{rgb}{0.763520,0.178667,0.193396}%
\pgfsetfillcolor{currentfill}%
\pgfsetlinewidth{0.000000pt}%
\definecolor{currentstroke}{rgb}{0.000000,0.000000,0.000000}%
\pgfsetstrokecolor{currentstroke}%
\pgfsetdash{}{0pt}%
\pgfpathmoveto{\pgfqpoint{4.018447in}{2.468167in}}%
\pgfpathlineto{\pgfqpoint{4.267628in}{2.613520in}}%
\pgfpathlineto{\pgfqpoint{4.018447in}{2.468167in}}%
\pgfpathclose%
\pgfusepath{fill}%
\end{pgfscope}%
\begin{pgfscope}%
\pgfpathrectangle{\pgfqpoint{0.017500in}{0.017500in}}{\pgfqpoint{6.218282in}{3.580372in}} %
\pgfusepath{clip}%
\pgfsetbuttcap%
\pgfsetroundjoin%
\definecolor{currentfill}{rgb}{0.763520,0.178667,0.193396}%
\pgfsetfillcolor{currentfill}%
\pgfsetlinewidth{0.000000pt}%
\definecolor{currentstroke}{rgb}{0.000000,0.000000,0.000000}%
\pgfsetstrokecolor{currentstroke}%
\pgfsetdash{}{0pt}%
\pgfpathmoveto{\pgfqpoint{4.018447in}{2.468167in}}%
\pgfpathlineto{\pgfqpoint{4.267628in}{2.613520in}}%
\pgfpathlineto{\pgfqpoint{4.018447in}{2.468167in}}%
\pgfpathclose%
\pgfusepath{fill}%
\end{pgfscope}%
\begin{pgfscope}%
\pgfpathrectangle{\pgfqpoint{0.017500in}{0.017500in}}{\pgfqpoint{6.218282in}{3.580372in}} %
\pgfusepath{clip}%
\pgfsetbuttcap%
\pgfsetroundjoin%
\definecolor{currentfill}{rgb}{0.763520,0.178667,0.193396}%
\pgfsetfillcolor{currentfill}%
\pgfsetlinewidth{0.000000pt}%
\definecolor{currentstroke}{rgb}{0.000000,0.000000,0.000000}%
\pgfsetstrokecolor{currentstroke}%
\pgfsetdash{}{0pt}%
\pgfpathmoveto{\pgfqpoint{4.018447in}{2.468167in}}%
\pgfpathlineto{\pgfqpoint{4.267628in}{2.613520in}}%
\pgfpathlineto{\pgfqpoint{4.018447in}{2.468167in}}%
\pgfpathclose%
\pgfusepath{fill}%
\end{pgfscope}%
\begin{pgfscope}%
\pgfpathrectangle{\pgfqpoint{0.017500in}{0.017500in}}{\pgfqpoint{6.218282in}{3.580372in}} %
\pgfusepath{clip}%
\pgfsetbuttcap%
\pgfsetroundjoin%
\definecolor{currentfill}{rgb}{0.763520,0.178667,0.193396}%
\pgfsetfillcolor{currentfill}%
\pgfsetlinewidth{0.000000pt}%
\definecolor{currentstroke}{rgb}{0.000000,0.000000,0.000000}%
\pgfsetstrokecolor{currentstroke}%
\pgfsetdash{}{0pt}%
\pgfpathmoveto{\pgfqpoint{4.018447in}{2.468167in}}%
\pgfpathlineto{\pgfqpoint{4.267628in}{2.613520in}}%
\pgfpathlineto{\pgfqpoint{4.018447in}{2.468167in}}%
\pgfpathclose%
\pgfusepath{fill}%
\end{pgfscope}%
\begin{pgfscope}%
\pgfpathrectangle{\pgfqpoint{0.017500in}{0.017500in}}{\pgfqpoint{6.218282in}{3.580372in}} %
\pgfusepath{clip}%
\pgfsetbuttcap%
\pgfsetroundjoin%
\definecolor{currentfill}{rgb}{0.763520,0.178667,0.193396}%
\pgfsetfillcolor{currentfill}%
\pgfsetlinewidth{0.000000pt}%
\definecolor{currentstroke}{rgb}{0.000000,0.000000,0.000000}%
\pgfsetstrokecolor{currentstroke}%
\pgfsetdash{}{0pt}%
\pgfpathmoveto{\pgfqpoint{4.018447in}{2.468167in}}%
\pgfpathlineto{\pgfqpoint{4.267628in}{2.613520in}}%
\pgfpathlineto{\pgfqpoint{4.018447in}{2.468167in}}%
\pgfpathclose%
\pgfusepath{fill}%
\end{pgfscope}%
\begin{pgfscope}%
\pgfpathrectangle{\pgfqpoint{0.017500in}{0.017500in}}{\pgfqpoint{6.218282in}{3.580372in}} %
\pgfusepath{clip}%
\pgfsetbuttcap%
\pgfsetroundjoin%
\definecolor{currentfill}{rgb}{0.763520,0.178667,0.193396}%
\pgfsetfillcolor{currentfill}%
\pgfsetlinewidth{0.000000pt}%
\definecolor{currentstroke}{rgb}{0.000000,0.000000,0.000000}%
\pgfsetstrokecolor{currentstroke}%
\pgfsetdash{}{0pt}%
\pgfpathmoveto{\pgfqpoint{4.018447in}{2.468167in}}%
\pgfpathlineto{\pgfqpoint{4.267628in}{2.613520in}}%
\pgfpathlineto{\pgfqpoint{4.018447in}{2.468167in}}%
\pgfpathclose%
\pgfusepath{fill}%
\end{pgfscope}%
\begin{pgfscope}%
\pgfpathrectangle{\pgfqpoint{0.017500in}{0.017500in}}{\pgfqpoint{6.218282in}{3.580372in}} %
\pgfusepath{clip}%
\pgfsetbuttcap%
\pgfsetroundjoin%
\definecolor{currentfill}{rgb}{0.763520,0.178667,0.193396}%
\pgfsetfillcolor{currentfill}%
\pgfsetlinewidth{0.000000pt}%
\definecolor{currentstroke}{rgb}{0.000000,0.000000,0.000000}%
\pgfsetstrokecolor{currentstroke}%
\pgfsetdash{}{0pt}%
\pgfpathmoveto{\pgfqpoint{4.018447in}{2.468167in}}%
\pgfpathlineto{\pgfqpoint{4.267628in}{2.613520in}}%
\pgfpathlineto{\pgfqpoint{4.018447in}{2.468167in}}%
\pgfpathclose%
\pgfusepath{fill}%
\end{pgfscope}%
\begin{pgfscope}%
\pgfpathrectangle{\pgfqpoint{0.017500in}{0.017500in}}{\pgfqpoint{6.218282in}{3.580372in}} %
\pgfusepath{clip}%
\pgfsetbuttcap%
\pgfsetroundjoin%
\definecolor{currentfill}{rgb}{0.763520,0.178667,0.193396}%
\pgfsetfillcolor{currentfill}%
\pgfsetlinewidth{0.000000pt}%
\definecolor{currentstroke}{rgb}{0.000000,0.000000,0.000000}%
\pgfsetstrokecolor{currentstroke}%
\pgfsetdash{}{0pt}%
\pgfpathmoveto{\pgfqpoint{4.018447in}{2.468167in}}%
\pgfpathlineto{\pgfqpoint{4.267628in}{2.613520in}}%
\pgfpathlineto{\pgfqpoint{4.018447in}{2.468167in}}%
\pgfpathclose%
\pgfusepath{fill}%
\end{pgfscope}%
\begin{pgfscope}%
\pgfpathrectangle{\pgfqpoint{0.017500in}{0.017500in}}{\pgfqpoint{6.218282in}{3.580372in}} %
\pgfusepath{clip}%
\pgfsetbuttcap%
\pgfsetroundjoin%
\definecolor{currentfill}{rgb}{0.763520,0.178667,0.193396}%
\pgfsetfillcolor{currentfill}%
\pgfsetlinewidth{0.000000pt}%
\definecolor{currentstroke}{rgb}{0.000000,0.000000,0.000000}%
\pgfsetstrokecolor{currentstroke}%
\pgfsetdash{}{0pt}%
\pgfpathmoveto{\pgfqpoint{4.018447in}{2.468167in}}%
\pgfpathlineto{\pgfqpoint{4.267628in}{2.613520in}}%
\pgfpathlineto{\pgfqpoint{4.018447in}{2.468167in}}%
\pgfpathclose%
\pgfusepath{fill}%
\end{pgfscope}%
\begin{pgfscope}%
\pgfpathrectangle{\pgfqpoint{0.017500in}{0.017500in}}{\pgfqpoint{6.218282in}{3.580372in}} %
\pgfusepath{clip}%
\pgfsetbuttcap%
\pgfsetroundjoin%
\definecolor{currentfill}{rgb}{0.763520,0.178667,0.193396}%
\pgfsetfillcolor{currentfill}%
\pgfsetlinewidth{0.000000pt}%
\definecolor{currentstroke}{rgb}{0.000000,0.000000,0.000000}%
\pgfsetstrokecolor{currentstroke}%
\pgfsetdash{}{0pt}%
\pgfpathmoveto{\pgfqpoint{4.018447in}{2.468167in}}%
\pgfpathlineto{\pgfqpoint{4.267628in}{2.613520in}}%
\pgfpathlineto{\pgfqpoint{4.018447in}{2.468167in}}%
\pgfpathclose%
\pgfusepath{fill}%
\end{pgfscope}%
\begin{pgfscope}%
\pgfpathrectangle{\pgfqpoint{0.017500in}{0.017500in}}{\pgfqpoint{6.218282in}{3.580372in}} %
\pgfusepath{clip}%
\pgfsetbuttcap%
\pgfsetroundjoin%
\definecolor{currentfill}{rgb}{0.763520,0.178667,0.193396}%
\pgfsetfillcolor{currentfill}%
\pgfsetlinewidth{0.000000pt}%
\definecolor{currentstroke}{rgb}{0.000000,0.000000,0.000000}%
\pgfsetstrokecolor{currentstroke}%
\pgfsetdash{}{0pt}%
\pgfpathmoveto{\pgfqpoint{4.018447in}{2.468167in}}%
\pgfpathlineto{\pgfqpoint{4.267628in}{2.613520in}}%
\pgfpathlineto{\pgfqpoint{4.018447in}{2.468167in}}%
\pgfpathclose%
\pgfusepath{fill}%
\end{pgfscope}%
\begin{pgfscope}%
\pgfpathrectangle{\pgfqpoint{0.017500in}{0.017500in}}{\pgfqpoint{6.218282in}{3.580372in}} %
\pgfusepath{clip}%
\pgfsetbuttcap%
\pgfsetroundjoin%
\definecolor{currentfill}{rgb}{0.763520,0.178667,0.193396}%
\pgfsetfillcolor{currentfill}%
\pgfsetlinewidth{0.000000pt}%
\definecolor{currentstroke}{rgb}{0.000000,0.000000,0.000000}%
\pgfsetstrokecolor{currentstroke}%
\pgfsetdash{}{0pt}%
\pgfpathmoveto{\pgfqpoint{4.018447in}{2.468167in}}%
\pgfpathlineto{\pgfqpoint{4.267628in}{2.613520in}}%
\pgfpathlineto{\pgfqpoint{4.018447in}{2.468167in}}%
\pgfpathclose%
\pgfusepath{fill}%
\end{pgfscope}%
\begin{pgfscope}%
\pgfpathrectangle{\pgfqpoint{0.017500in}{0.017500in}}{\pgfqpoint{6.218282in}{3.580372in}} %
\pgfusepath{clip}%
\pgfsetbuttcap%
\pgfsetroundjoin%
\definecolor{currentfill}{rgb}{0.763520,0.178667,0.193396}%
\pgfsetfillcolor{currentfill}%
\pgfsetlinewidth{0.000000pt}%
\definecolor{currentstroke}{rgb}{0.000000,0.000000,0.000000}%
\pgfsetstrokecolor{currentstroke}%
\pgfsetdash{}{0pt}%
\pgfpathmoveto{\pgfqpoint{4.018447in}{2.468167in}}%
\pgfpathlineto{\pgfqpoint{4.267628in}{2.613520in}}%
\pgfpathlineto{\pgfqpoint{4.018447in}{2.468167in}}%
\pgfpathclose%
\pgfusepath{fill}%
\end{pgfscope}%
\begin{pgfscope}%
\pgfpathrectangle{\pgfqpoint{0.017500in}{0.017500in}}{\pgfqpoint{6.218282in}{3.580372in}} %
\pgfusepath{clip}%
\pgfsetbuttcap%
\pgfsetroundjoin%
\definecolor{currentfill}{rgb}{0.763520,0.178667,0.193396}%
\pgfsetfillcolor{currentfill}%
\pgfsetlinewidth{0.000000pt}%
\definecolor{currentstroke}{rgb}{0.000000,0.000000,0.000000}%
\pgfsetstrokecolor{currentstroke}%
\pgfsetdash{}{0pt}%
\pgfpathmoveto{\pgfqpoint{4.018447in}{2.468167in}}%
\pgfpathlineto{\pgfqpoint{4.267628in}{2.613520in}}%
\pgfpathlineto{\pgfqpoint{4.018447in}{2.468167in}}%
\pgfpathclose%
\pgfusepath{fill}%
\end{pgfscope}%
\begin{pgfscope}%
\pgfpathrectangle{\pgfqpoint{0.017500in}{0.017500in}}{\pgfqpoint{6.218282in}{3.580372in}} %
\pgfusepath{clip}%
\pgfsetbuttcap%
\pgfsetroundjoin%
\definecolor{currentfill}{rgb}{0.763520,0.178667,0.193396}%
\pgfsetfillcolor{currentfill}%
\pgfsetlinewidth{0.000000pt}%
\definecolor{currentstroke}{rgb}{0.000000,0.000000,0.000000}%
\pgfsetstrokecolor{currentstroke}%
\pgfsetdash{}{0pt}%
\pgfpathmoveto{\pgfqpoint{4.018447in}{2.468167in}}%
\pgfpathlineto{\pgfqpoint{4.267628in}{2.613520in}}%
\pgfpathlineto{\pgfqpoint{4.018447in}{2.468167in}}%
\pgfpathclose%
\pgfusepath{fill}%
\end{pgfscope}%
\begin{pgfscope}%
\pgfpathrectangle{\pgfqpoint{0.017500in}{0.017500in}}{\pgfqpoint{6.218282in}{3.580372in}} %
\pgfusepath{clip}%
\pgfsetbuttcap%
\pgfsetroundjoin%
\definecolor{currentfill}{rgb}{0.892138,0.425389,0.333289}%
\pgfsetfillcolor{currentfill}%
\pgfsetlinewidth{0.000000pt}%
\definecolor{currentstroke}{rgb}{0.000000,0.000000,0.000000}%
\pgfsetstrokecolor{currentstroke}%
\pgfsetdash{}{0pt}%
\pgfpathmoveto{\pgfqpoint{2.143341in}{2.291288in}}%
\pgfpathlineto{\pgfqpoint{2.590432in}{2.210709in}}%
\pgfpathlineto{\pgfqpoint{2.855513in}{2.451418in}}%
\pgfpathlineto{\pgfqpoint{2.410872in}{2.508083in}}%
\pgfpathlineto{\pgfqpoint{2.143341in}{2.291288in}}%
\pgfpathclose%
\pgfusepath{fill}%
\end{pgfscope}%
\begin{pgfscope}%
\pgfpathrectangle{\pgfqpoint{0.017500in}{0.017500in}}{\pgfqpoint{6.218282in}{3.580372in}} %
\pgfusepath{clip}%
\pgfsetbuttcap%
\pgfsetroundjoin%
\definecolor{currentfill}{rgb}{0.892138,0.425389,0.333289}%
\pgfsetfillcolor{currentfill}%
\pgfsetlinewidth{0.000000pt}%
\definecolor{currentstroke}{rgb}{0.000000,0.000000,0.000000}%
\pgfsetstrokecolor{currentstroke}%
\pgfsetdash{}{0pt}%
\pgfpathmoveto{\pgfqpoint{2.143341in}{2.291288in}}%
\pgfpathlineto{\pgfqpoint{2.590432in}{2.210709in}}%
\pgfpathlineto{\pgfqpoint{2.855513in}{2.451418in}}%
\pgfpathlineto{\pgfqpoint{2.410872in}{2.508083in}}%
\pgfpathlineto{\pgfqpoint{2.143341in}{2.291288in}}%
\pgfpathclose%
\pgfusepath{fill}%
\end{pgfscope}%
\begin{pgfscope}%
\pgfpathrectangle{\pgfqpoint{0.017500in}{0.017500in}}{\pgfqpoint{6.218282in}{3.580372in}} %
\pgfusepath{clip}%
\pgfsetbuttcap%
\pgfsetroundjoin%
\definecolor{currentfill}{rgb}{0.892138,0.425389,0.333289}%
\pgfsetfillcolor{currentfill}%
\pgfsetlinewidth{0.000000pt}%
\definecolor{currentstroke}{rgb}{0.000000,0.000000,0.000000}%
\pgfsetstrokecolor{currentstroke}%
\pgfsetdash{}{0pt}%
\pgfpathmoveto{\pgfqpoint{2.143341in}{2.291288in}}%
\pgfpathlineto{\pgfqpoint{2.590432in}{2.210709in}}%
\pgfpathlineto{\pgfqpoint{2.855513in}{2.451418in}}%
\pgfpathlineto{\pgfqpoint{2.410872in}{2.508083in}}%
\pgfpathlineto{\pgfqpoint{2.143341in}{2.291288in}}%
\pgfpathclose%
\pgfusepath{fill}%
\end{pgfscope}%
\begin{pgfscope}%
\pgfpathrectangle{\pgfqpoint{0.017500in}{0.017500in}}{\pgfqpoint{6.218282in}{3.580372in}} %
\pgfusepath{clip}%
\pgfsetbuttcap%
\pgfsetroundjoin%
\definecolor{currentfill}{rgb}{0.892138,0.425389,0.333289}%
\pgfsetfillcolor{currentfill}%
\pgfsetlinewidth{0.000000pt}%
\definecolor{currentstroke}{rgb}{0.000000,0.000000,0.000000}%
\pgfsetstrokecolor{currentstroke}%
\pgfsetdash{}{0pt}%
\pgfpathmoveto{\pgfqpoint{2.143341in}{2.291288in}}%
\pgfpathlineto{\pgfqpoint{2.590432in}{2.210709in}}%
\pgfpathlineto{\pgfqpoint{2.855513in}{2.451418in}}%
\pgfpathlineto{\pgfqpoint{2.410872in}{2.508083in}}%
\pgfpathlineto{\pgfqpoint{2.143341in}{2.291288in}}%
\pgfpathclose%
\pgfusepath{fill}%
\end{pgfscope}%
\begin{pgfscope}%
\pgfpathrectangle{\pgfqpoint{0.017500in}{0.017500in}}{\pgfqpoint{6.218282in}{3.580372in}} %
\pgfusepath{clip}%
\pgfsetbuttcap%
\pgfsetroundjoin%
\definecolor{currentfill}{rgb}{0.892138,0.425389,0.333289}%
\pgfsetfillcolor{currentfill}%
\pgfsetlinewidth{0.000000pt}%
\definecolor{currentstroke}{rgb}{0.000000,0.000000,0.000000}%
\pgfsetstrokecolor{currentstroke}%
\pgfsetdash{}{0pt}%
\pgfpathmoveto{\pgfqpoint{2.143341in}{2.291288in}}%
\pgfpathlineto{\pgfqpoint{2.590432in}{2.210709in}}%
\pgfpathlineto{\pgfqpoint{2.855513in}{2.451418in}}%
\pgfpathlineto{\pgfqpoint{2.410872in}{2.508083in}}%
\pgfpathlineto{\pgfqpoint{2.143341in}{2.291288in}}%
\pgfpathclose%
\pgfusepath{fill}%
\end{pgfscope}%
\begin{pgfscope}%
\pgfpathrectangle{\pgfqpoint{0.017500in}{0.017500in}}{\pgfqpoint{6.218282in}{3.580372in}} %
\pgfusepath{clip}%
\pgfsetbuttcap%
\pgfsetroundjoin%
\definecolor{currentfill}{rgb}{0.892138,0.425389,0.333289}%
\pgfsetfillcolor{currentfill}%
\pgfsetlinewidth{0.000000pt}%
\definecolor{currentstroke}{rgb}{0.000000,0.000000,0.000000}%
\pgfsetstrokecolor{currentstroke}%
\pgfsetdash{}{0pt}%
\pgfpathmoveto{\pgfqpoint{2.143341in}{2.291288in}}%
\pgfpathlineto{\pgfqpoint{2.590432in}{2.210709in}}%
\pgfpathlineto{\pgfqpoint{2.855513in}{2.451418in}}%
\pgfpathlineto{\pgfqpoint{2.410872in}{2.508083in}}%
\pgfpathlineto{\pgfqpoint{2.143341in}{2.291288in}}%
\pgfpathclose%
\pgfusepath{fill}%
\end{pgfscope}%
\begin{pgfscope}%
\pgfpathrectangle{\pgfqpoint{0.017500in}{0.017500in}}{\pgfqpoint{6.218282in}{3.580372in}} %
\pgfusepath{clip}%
\pgfsetbuttcap%
\pgfsetroundjoin%
\definecolor{currentfill}{rgb}{0.758112,0.168122,0.188827}%
\pgfsetfillcolor{currentfill}%
\pgfsetlinewidth{0.000000pt}%
\definecolor{currentstroke}{rgb}{0.000000,0.000000,0.000000}%
\pgfsetstrokecolor{currentstroke}%
\pgfsetdash{}{0pt}%
\pgfpathmoveto{\pgfqpoint{4.018447in}{2.468167in}}%
\pgfpathlineto{\pgfqpoint{4.478468in}{2.393972in}}%
\pgfpathlineto{\pgfqpoint{4.724651in}{2.549289in}}%
\pgfpathlineto{\pgfqpoint{4.267628in}{2.613520in}}%
\pgfpathlineto{\pgfqpoint{4.018447in}{2.468167in}}%
\pgfpathclose%
\pgfusepath{fill}%
\end{pgfscope}%
\begin{pgfscope}%
\pgfpathrectangle{\pgfqpoint{0.017500in}{0.017500in}}{\pgfqpoint{6.218282in}{3.580372in}} %
\pgfusepath{clip}%
\pgfsetbuttcap%
\pgfsetroundjoin%
\definecolor{currentfill}{rgb}{0.758112,0.168122,0.188827}%
\pgfsetfillcolor{currentfill}%
\pgfsetlinewidth{0.000000pt}%
\definecolor{currentstroke}{rgb}{0.000000,0.000000,0.000000}%
\pgfsetstrokecolor{currentstroke}%
\pgfsetdash{}{0pt}%
\pgfpathmoveto{\pgfqpoint{4.018447in}{2.468167in}}%
\pgfpathlineto{\pgfqpoint{4.478468in}{2.393972in}}%
\pgfpathlineto{\pgfqpoint{4.724651in}{2.549289in}}%
\pgfpathlineto{\pgfqpoint{4.267628in}{2.613520in}}%
\pgfpathlineto{\pgfqpoint{4.018447in}{2.468167in}}%
\pgfpathclose%
\pgfusepath{fill}%
\end{pgfscope}%
\begin{pgfscope}%
\pgfpathrectangle{\pgfqpoint{0.017500in}{0.017500in}}{\pgfqpoint{6.218282in}{3.580372in}} %
\pgfusepath{clip}%
\pgfsetbuttcap%
\pgfsetroundjoin%
\definecolor{currentfill}{rgb}{0.758112,0.168122,0.188827}%
\pgfsetfillcolor{currentfill}%
\pgfsetlinewidth{0.000000pt}%
\definecolor{currentstroke}{rgb}{0.000000,0.000000,0.000000}%
\pgfsetstrokecolor{currentstroke}%
\pgfsetdash{}{0pt}%
\pgfpathmoveto{\pgfqpoint{4.018447in}{2.468167in}}%
\pgfpathlineto{\pgfqpoint{4.478468in}{2.393972in}}%
\pgfpathlineto{\pgfqpoint{4.724651in}{2.549289in}}%
\pgfpathlineto{\pgfqpoint{4.267628in}{2.613520in}}%
\pgfpathlineto{\pgfqpoint{4.018447in}{2.468167in}}%
\pgfpathclose%
\pgfusepath{fill}%
\end{pgfscope}%
\begin{pgfscope}%
\pgfpathrectangle{\pgfqpoint{0.017500in}{0.017500in}}{\pgfqpoint{6.218282in}{3.580372in}} %
\pgfusepath{clip}%
\pgfsetbuttcap%
\pgfsetroundjoin%
\definecolor{currentfill}{rgb}{0.758112,0.168122,0.188827}%
\pgfsetfillcolor{currentfill}%
\pgfsetlinewidth{0.000000pt}%
\definecolor{currentstroke}{rgb}{0.000000,0.000000,0.000000}%
\pgfsetstrokecolor{currentstroke}%
\pgfsetdash{}{0pt}%
\pgfpathmoveto{\pgfqpoint{4.018447in}{2.468167in}}%
\pgfpathlineto{\pgfqpoint{4.478468in}{2.393972in}}%
\pgfpathlineto{\pgfqpoint{4.724651in}{2.549289in}}%
\pgfpathlineto{\pgfqpoint{4.267628in}{2.613520in}}%
\pgfpathlineto{\pgfqpoint{4.018447in}{2.468167in}}%
\pgfpathclose%
\pgfusepath{fill}%
\end{pgfscope}%
\begin{pgfscope}%
\pgfpathrectangle{\pgfqpoint{0.017500in}{0.017500in}}{\pgfqpoint{6.218282in}{3.580372in}} %
\pgfusepath{clip}%
\pgfsetbuttcap%
\pgfsetroundjoin%
\definecolor{currentfill}{rgb}{0.758112,0.168122,0.188827}%
\pgfsetfillcolor{currentfill}%
\pgfsetlinewidth{0.000000pt}%
\definecolor{currentstroke}{rgb}{0.000000,0.000000,0.000000}%
\pgfsetstrokecolor{currentstroke}%
\pgfsetdash{}{0pt}%
\pgfpathmoveto{\pgfqpoint{4.018447in}{2.468167in}}%
\pgfpathlineto{\pgfqpoint{4.478468in}{2.393972in}}%
\pgfpathlineto{\pgfqpoint{4.724651in}{2.549289in}}%
\pgfpathlineto{\pgfqpoint{4.267628in}{2.613520in}}%
\pgfpathlineto{\pgfqpoint{4.018447in}{2.468167in}}%
\pgfpathclose%
\pgfusepath{fill}%
\end{pgfscope}%
\begin{pgfscope}%
\pgfpathrectangle{\pgfqpoint{0.017500in}{0.017500in}}{\pgfqpoint{6.218282in}{3.580372in}} %
\pgfusepath{clip}%
\pgfsetbuttcap%
\pgfsetroundjoin%
\definecolor{currentfill}{rgb}{0.758112,0.168122,0.188827}%
\pgfsetfillcolor{currentfill}%
\pgfsetlinewidth{0.000000pt}%
\definecolor{currentstroke}{rgb}{0.000000,0.000000,0.000000}%
\pgfsetstrokecolor{currentstroke}%
\pgfsetdash{}{0pt}%
\pgfpathmoveto{\pgfqpoint{4.018447in}{2.468167in}}%
\pgfpathlineto{\pgfqpoint{4.478468in}{2.393972in}}%
\pgfpathlineto{\pgfqpoint{4.724651in}{2.549289in}}%
\pgfpathlineto{\pgfqpoint{4.267628in}{2.613520in}}%
\pgfpathlineto{\pgfqpoint{4.018447in}{2.468167in}}%
\pgfpathclose%
\pgfusepath{fill}%
\end{pgfscope}%
\begin{pgfscope}%
\pgfpathrectangle{\pgfqpoint{0.017500in}{0.017500in}}{\pgfqpoint{6.218282in}{3.580372in}} %
\pgfusepath{clip}%
\pgfsetbuttcap%
\pgfsetroundjoin%
\definecolor{currentfill}{rgb}{0.333490,0.446265,0.874452}%
\pgfsetfillcolor{currentfill}%
\pgfsetlinewidth{0.000000pt}%
\definecolor{currentstroke}{rgb}{0.000000,0.000000,0.000000}%
\pgfsetstrokecolor{currentstroke}%
\pgfsetdash{}{0pt}%
\pgfpathmoveto{\pgfqpoint{2.789950in}{0.798346in}}%
\pgfpathlineto{\pgfqpoint{3.236099in}{0.723497in}}%
\pgfpathlineto{\pgfqpoint{3.502561in}{2.073396in}}%
\pgfpathlineto{\pgfqpoint{3.043326in}{2.141782in}}%
\pgfpathlineto{\pgfqpoint{2.789950in}{0.798346in}}%
\pgfpathclose%
\pgfusepath{fill}%
\end{pgfscope}%
\begin{pgfscope}%
\pgfpathrectangle{\pgfqpoint{0.017500in}{0.017500in}}{\pgfqpoint{6.218282in}{3.580372in}} %
\pgfusepath{clip}%
\pgfsetbuttcap%
\pgfsetroundjoin%
\definecolor{currentfill}{rgb}{0.333490,0.446265,0.874452}%
\pgfsetfillcolor{currentfill}%
\pgfsetlinewidth{0.000000pt}%
\definecolor{currentstroke}{rgb}{0.000000,0.000000,0.000000}%
\pgfsetstrokecolor{currentstroke}%
\pgfsetdash{}{0pt}%
\pgfpathmoveto{\pgfqpoint{2.789950in}{0.798346in}}%
\pgfpathlineto{\pgfqpoint{3.236099in}{0.723497in}}%
\pgfpathlineto{\pgfqpoint{3.502561in}{2.073396in}}%
\pgfpathlineto{\pgfqpoint{3.043326in}{2.141782in}}%
\pgfpathlineto{\pgfqpoint{2.789950in}{0.798346in}}%
\pgfpathclose%
\pgfusepath{fill}%
\end{pgfscope}%
\begin{pgfscope}%
\pgfpathrectangle{\pgfqpoint{0.017500in}{0.017500in}}{\pgfqpoint{6.218282in}{3.580372in}} %
\pgfusepath{clip}%
\pgfsetbuttcap%
\pgfsetroundjoin%
\definecolor{currentfill}{rgb}{0.333490,0.446265,0.874452}%
\pgfsetfillcolor{currentfill}%
\pgfsetlinewidth{0.000000pt}%
\definecolor{currentstroke}{rgb}{0.000000,0.000000,0.000000}%
\pgfsetstrokecolor{currentstroke}%
\pgfsetdash{}{0pt}%
\pgfpathmoveto{\pgfqpoint{2.789950in}{0.798346in}}%
\pgfpathlineto{\pgfqpoint{3.236099in}{0.723497in}}%
\pgfpathlineto{\pgfqpoint{3.502561in}{2.073396in}}%
\pgfpathlineto{\pgfqpoint{3.043326in}{2.141782in}}%
\pgfpathlineto{\pgfqpoint{2.789950in}{0.798346in}}%
\pgfpathclose%
\pgfusepath{fill}%
\end{pgfscope}%
\begin{pgfscope}%
\pgfpathrectangle{\pgfqpoint{0.017500in}{0.017500in}}{\pgfqpoint{6.218282in}{3.580372in}} %
\pgfusepath{clip}%
\pgfsetbuttcap%
\pgfsetroundjoin%
\definecolor{currentfill}{rgb}{0.333490,0.446265,0.874452}%
\pgfsetfillcolor{currentfill}%
\pgfsetlinewidth{0.000000pt}%
\definecolor{currentstroke}{rgb}{0.000000,0.000000,0.000000}%
\pgfsetstrokecolor{currentstroke}%
\pgfsetdash{}{0pt}%
\pgfpathmoveto{\pgfqpoint{2.789950in}{0.798346in}}%
\pgfpathlineto{\pgfqpoint{3.236099in}{0.723497in}}%
\pgfpathlineto{\pgfqpoint{3.502561in}{2.073396in}}%
\pgfpathlineto{\pgfqpoint{3.043326in}{2.141782in}}%
\pgfpathlineto{\pgfqpoint{2.789950in}{0.798346in}}%
\pgfpathclose%
\pgfusepath{fill}%
\end{pgfscope}%
\begin{pgfscope}%
\pgfpathrectangle{\pgfqpoint{0.017500in}{0.017500in}}{\pgfqpoint{6.218282in}{3.580372in}} %
\pgfusepath{clip}%
\pgfsetbuttcap%
\pgfsetroundjoin%
\definecolor{currentfill}{rgb}{0.333490,0.446265,0.874452}%
\pgfsetfillcolor{currentfill}%
\pgfsetlinewidth{0.000000pt}%
\definecolor{currentstroke}{rgb}{0.000000,0.000000,0.000000}%
\pgfsetstrokecolor{currentstroke}%
\pgfsetdash{}{0pt}%
\pgfpathmoveto{\pgfqpoint{2.789950in}{0.798346in}}%
\pgfpathlineto{\pgfqpoint{3.236099in}{0.723497in}}%
\pgfpathlineto{\pgfqpoint{3.502561in}{2.073396in}}%
\pgfpathlineto{\pgfqpoint{3.043326in}{2.141782in}}%
\pgfpathlineto{\pgfqpoint{2.789950in}{0.798346in}}%
\pgfpathclose%
\pgfusepath{fill}%
\end{pgfscope}%
\begin{pgfscope}%
\pgfpathrectangle{\pgfqpoint{0.017500in}{0.017500in}}{\pgfqpoint{6.218282in}{3.580372in}} %
\pgfusepath{clip}%
\pgfsetbuttcap%
\pgfsetroundjoin%
\definecolor{currentfill}{rgb}{0.333490,0.446265,0.874452}%
\pgfsetfillcolor{currentfill}%
\pgfsetlinewidth{0.000000pt}%
\definecolor{currentstroke}{rgb}{0.000000,0.000000,0.000000}%
\pgfsetstrokecolor{currentstroke}%
\pgfsetdash{}{0pt}%
\pgfpathmoveto{\pgfqpoint{2.789950in}{0.798346in}}%
\pgfpathlineto{\pgfqpoint{3.236099in}{0.723497in}}%
\pgfpathlineto{\pgfqpoint{3.502561in}{2.073396in}}%
\pgfpathlineto{\pgfqpoint{3.043326in}{2.141782in}}%
\pgfpathlineto{\pgfqpoint{2.789950in}{0.798346in}}%
\pgfpathclose%
\pgfusepath{fill}%
\end{pgfscope}%
\begin{pgfscope}%
\pgfpathrectangle{\pgfqpoint{0.017500in}{0.017500in}}{\pgfqpoint{6.218282in}{3.580372in}} %
\pgfusepath{clip}%
\pgfsetbuttcap%
\pgfsetroundjoin%
\definecolor{currentfill}{rgb}{0.800830,0.250829,0.225696}%
\pgfsetfillcolor{currentfill}%
\pgfsetlinewidth{0.000000pt}%
\definecolor{currentstroke}{rgb}{0.000000,0.000000,0.000000}%
\pgfsetstrokecolor{currentstroke}%
\pgfsetdash{}{0pt}%
\pgfpathmoveto{\pgfqpoint{3.306485in}{2.382499in}}%
\pgfpathlineto{\pgfqpoint{3.564492in}{2.541257in}}%
\pgfpathlineto{\pgfqpoint{3.306485in}{2.382499in}}%
\pgfpathclose%
\pgfusepath{fill}%
\end{pgfscope}%
\begin{pgfscope}%
\pgfpathrectangle{\pgfqpoint{0.017500in}{0.017500in}}{\pgfqpoint{6.218282in}{3.580372in}} %
\pgfusepath{clip}%
\pgfsetbuttcap%
\pgfsetroundjoin%
\definecolor{currentfill}{rgb}{0.800830,0.250829,0.225696}%
\pgfsetfillcolor{currentfill}%
\pgfsetlinewidth{0.000000pt}%
\definecolor{currentstroke}{rgb}{0.000000,0.000000,0.000000}%
\pgfsetstrokecolor{currentstroke}%
\pgfsetdash{}{0pt}%
\pgfpathmoveto{\pgfqpoint{3.306485in}{2.382499in}}%
\pgfpathlineto{\pgfqpoint{3.564492in}{2.541257in}}%
\pgfpathlineto{\pgfqpoint{3.306485in}{2.382499in}}%
\pgfpathclose%
\pgfusepath{fill}%
\end{pgfscope}%
\begin{pgfscope}%
\pgfpathrectangle{\pgfqpoint{0.017500in}{0.017500in}}{\pgfqpoint{6.218282in}{3.580372in}} %
\pgfusepath{clip}%
\pgfsetbuttcap%
\pgfsetroundjoin%
\definecolor{currentfill}{rgb}{0.800830,0.250829,0.225696}%
\pgfsetfillcolor{currentfill}%
\pgfsetlinewidth{0.000000pt}%
\definecolor{currentstroke}{rgb}{0.000000,0.000000,0.000000}%
\pgfsetstrokecolor{currentstroke}%
\pgfsetdash{}{0pt}%
\pgfpathmoveto{\pgfqpoint{3.306485in}{2.382499in}}%
\pgfpathlineto{\pgfqpoint{3.564492in}{2.541257in}}%
\pgfpathlineto{\pgfqpoint{3.306485in}{2.382499in}}%
\pgfpathclose%
\pgfusepath{fill}%
\end{pgfscope}%
\begin{pgfscope}%
\pgfpathrectangle{\pgfqpoint{0.017500in}{0.017500in}}{\pgfqpoint{6.218282in}{3.580372in}} %
\pgfusepath{clip}%
\pgfsetbuttcap%
\pgfsetroundjoin%
\definecolor{currentfill}{rgb}{0.800830,0.250829,0.225696}%
\pgfsetfillcolor{currentfill}%
\pgfsetlinewidth{0.000000pt}%
\definecolor{currentstroke}{rgb}{0.000000,0.000000,0.000000}%
\pgfsetstrokecolor{currentstroke}%
\pgfsetdash{}{0pt}%
\pgfpathmoveto{\pgfqpoint{3.306485in}{2.382499in}}%
\pgfpathlineto{\pgfqpoint{3.564492in}{2.541257in}}%
\pgfpathlineto{\pgfqpoint{3.306485in}{2.382499in}}%
\pgfpathclose%
\pgfusepath{fill}%
\end{pgfscope}%
\begin{pgfscope}%
\pgfpathrectangle{\pgfqpoint{0.017500in}{0.017500in}}{\pgfqpoint{6.218282in}{3.580372in}} %
\pgfusepath{clip}%
\pgfsetbuttcap%
\pgfsetroundjoin%
\definecolor{currentfill}{rgb}{0.800830,0.250829,0.225696}%
\pgfsetfillcolor{currentfill}%
\pgfsetlinewidth{0.000000pt}%
\definecolor{currentstroke}{rgb}{0.000000,0.000000,0.000000}%
\pgfsetstrokecolor{currentstroke}%
\pgfsetdash{}{0pt}%
\pgfpathmoveto{\pgfqpoint{3.306485in}{2.382499in}}%
\pgfpathlineto{\pgfqpoint{3.564492in}{2.541257in}}%
\pgfpathlineto{\pgfqpoint{3.306485in}{2.382499in}}%
\pgfpathclose%
\pgfusepath{fill}%
\end{pgfscope}%
\begin{pgfscope}%
\pgfpathrectangle{\pgfqpoint{0.017500in}{0.017500in}}{\pgfqpoint{6.218282in}{3.580372in}} %
\pgfusepath{clip}%
\pgfsetbuttcap%
\pgfsetroundjoin%
\definecolor{currentfill}{rgb}{0.800830,0.250829,0.225696}%
\pgfsetfillcolor{currentfill}%
\pgfsetlinewidth{0.000000pt}%
\definecolor{currentstroke}{rgb}{0.000000,0.000000,0.000000}%
\pgfsetstrokecolor{currentstroke}%
\pgfsetdash{}{0pt}%
\pgfpathmoveto{\pgfqpoint{3.306485in}{2.382499in}}%
\pgfpathlineto{\pgfqpoint{3.564492in}{2.541257in}}%
\pgfpathlineto{\pgfqpoint{3.306485in}{2.382499in}}%
\pgfpathclose%
\pgfusepath{fill}%
\end{pgfscope}%
\begin{pgfscope}%
\pgfpathrectangle{\pgfqpoint{0.017500in}{0.017500in}}{\pgfqpoint{6.218282in}{3.580372in}} %
\pgfusepath{clip}%
\pgfsetbuttcap%
\pgfsetroundjoin%
\definecolor{currentfill}{rgb}{0.800830,0.250829,0.225696}%
\pgfsetfillcolor{currentfill}%
\pgfsetlinewidth{0.000000pt}%
\definecolor{currentstroke}{rgb}{0.000000,0.000000,0.000000}%
\pgfsetstrokecolor{currentstroke}%
\pgfsetdash{}{0pt}%
\pgfpathmoveto{\pgfqpoint{3.306485in}{2.382499in}}%
\pgfpathlineto{\pgfqpoint{3.564492in}{2.541257in}}%
\pgfpathlineto{\pgfqpoint{3.306485in}{2.382499in}}%
\pgfpathclose%
\pgfusepath{fill}%
\end{pgfscope}%
\begin{pgfscope}%
\pgfpathrectangle{\pgfqpoint{0.017500in}{0.017500in}}{\pgfqpoint{6.218282in}{3.580372in}} %
\pgfusepath{clip}%
\pgfsetbuttcap%
\pgfsetroundjoin%
\definecolor{currentfill}{rgb}{0.800830,0.250829,0.225696}%
\pgfsetfillcolor{currentfill}%
\pgfsetlinewidth{0.000000pt}%
\definecolor{currentstroke}{rgb}{0.000000,0.000000,0.000000}%
\pgfsetstrokecolor{currentstroke}%
\pgfsetdash{}{0pt}%
\pgfpathmoveto{\pgfqpoint{3.306485in}{2.382499in}}%
\pgfpathlineto{\pgfqpoint{3.564492in}{2.541257in}}%
\pgfpathlineto{\pgfqpoint{3.306485in}{2.382499in}}%
\pgfpathclose%
\pgfusepath{fill}%
\end{pgfscope}%
\begin{pgfscope}%
\pgfpathrectangle{\pgfqpoint{0.017500in}{0.017500in}}{\pgfqpoint{6.218282in}{3.580372in}} %
\pgfusepath{clip}%
\pgfsetbuttcap%
\pgfsetroundjoin%
\definecolor{currentfill}{rgb}{0.800830,0.250829,0.225696}%
\pgfsetfillcolor{currentfill}%
\pgfsetlinewidth{0.000000pt}%
\definecolor{currentstroke}{rgb}{0.000000,0.000000,0.000000}%
\pgfsetstrokecolor{currentstroke}%
\pgfsetdash{}{0pt}%
\pgfpathmoveto{\pgfqpoint{3.306485in}{2.382499in}}%
\pgfpathlineto{\pgfqpoint{3.564492in}{2.541257in}}%
\pgfpathlineto{\pgfqpoint{3.306485in}{2.382499in}}%
\pgfpathclose%
\pgfusepath{fill}%
\end{pgfscope}%
\begin{pgfscope}%
\pgfpathrectangle{\pgfqpoint{0.017500in}{0.017500in}}{\pgfqpoint{6.218282in}{3.580372in}} %
\pgfusepath{clip}%
\pgfsetbuttcap%
\pgfsetroundjoin%
\definecolor{currentfill}{rgb}{0.800830,0.250829,0.225696}%
\pgfsetfillcolor{currentfill}%
\pgfsetlinewidth{0.000000pt}%
\definecolor{currentstroke}{rgb}{0.000000,0.000000,0.000000}%
\pgfsetstrokecolor{currentstroke}%
\pgfsetdash{}{0pt}%
\pgfpathmoveto{\pgfqpoint{3.306485in}{2.382499in}}%
\pgfpathlineto{\pgfqpoint{3.564492in}{2.541257in}}%
\pgfpathlineto{\pgfqpoint{3.306485in}{2.382499in}}%
\pgfpathclose%
\pgfusepath{fill}%
\end{pgfscope}%
\begin{pgfscope}%
\pgfpathrectangle{\pgfqpoint{0.017500in}{0.017500in}}{\pgfqpoint{6.218282in}{3.580372in}} %
\pgfusepath{clip}%
\pgfsetbuttcap%
\pgfsetroundjoin%
\definecolor{currentfill}{rgb}{0.800830,0.250829,0.225696}%
\pgfsetfillcolor{currentfill}%
\pgfsetlinewidth{0.000000pt}%
\definecolor{currentstroke}{rgb}{0.000000,0.000000,0.000000}%
\pgfsetstrokecolor{currentstroke}%
\pgfsetdash{}{0pt}%
\pgfpathmoveto{\pgfqpoint{3.306485in}{2.382499in}}%
\pgfpathlineto{\pgfqpoint{3.564492in}{2.541257in}}%
\pgfpathlineto{\pgfqpoint{3.306485in}{2.382499in}}%
\pgfpathclose%
\pgfusepath{fill}%
\end{pgfscope}%
\begin{pgfscope}%
\pgfpathrectangle{\pgfqpoint{0.017500in}{0.017500in}}{\pgfqpoint{6.218282in}{3.580372in}} %
\pgfusepath{clip}%
\pgfsetbuttcap%
\pgfsetroundjoin%
\definecolor{currentfill}{rgb}{0.800830,0.250829,0.225696}%
\pgfsetfillcolor{currentfill}%
\pgfsetlinewidth{0.000000pt}%
\definecolor{currentstroke}{rgb}{0.000000,0.000000,0.000000}%
\pgfsetstrokecolor{currentstroke}%
\pgfsetdash{}{0pt}%
\pgfpathmoveto{\pgfqpoint{3.306485in}{2.382499in}}%
\pgfpathlineto{\pgfqpoint{3.564492in}{2.541257in}}%
\pgfpathlineto{\pgfqpoint{3.306485in}{2.382499in}}%
\pgfpathclose%
\pgfusepath{fill}%
\end{pgfscope}%
\begin{pgfscope}%
\pgfpathrectangle{\pgfqpoint{0.017500in}{0.017500in}}{\pgfqpoint{6.218282in}{3.580372in}} %
\pgfusepath{clip}%
\pgfsetbuttcap%
\pgfsetroundjoin%
\definecolor{currentfill}{rgb}{0.800830,0.250829,0.225696}%
\pgfsetfillcolor{currentfill}%
\pgfsetlinewidth{0.000000pt}%
\definecolor{currentstroke}{rgb}{0.000000,0.000000,0.000000}%
\pgfsetstrokecolor{currentstroke}%
\pgfsetdash{}{0pt}%
\pgfpathmoveto{\pgfqpoint{3.306485in}{2.382499in}}%
\pgfpathlineto{\pgfqpoint{3.564492in}{2.541257in}}%
\pgfpathlineto{\pgfqpoint{3.306485in}{2.382499in}}%
\pgfpathclose%
\pgfusepath{fill}%
\end{pgfscope}%
\begin{pgfscope}%
\pgfpathrectangle{\pgfqpoint{0.017500in}{0.017500in}}{\pgfqpoint{6.218282in}{3.580372in}} %
\pgfusepath{clip}%
\pgfsetbuttcap%
\pgfsetroundjoin%
\definecolor{currentfill}{rgb}{0.800830,0.250829,0.225696}%
\pgfsetfillcolor{currentfill}%
\pgfsetlinewidth{0.000000pt}%
\definecolor{currentstroke}{rgb}{0.000000,0.000000,0.000000}%
\pgfsetstrokecolor{currentstroke}%
\pgfsetdash{}{0pt}%
\pgfpathmoveto{\pgfqpoint{3.306485in}{2.382499in}}%
\pgfpathlineto{\pgfqpoint{3.564492in}{2.541257in}}%
\pgfpathlineto{\pgfqpoint{3.306485in}{2.382499in}}%
\pgfpathclose%
\pgfusepath{fill}%
\end{pgfscope}%
\begin{pgfscope}%
\pgfpathrectangle{\pgfqpoint{0.017500in}{0.017500in}}{\pgfqpoint{6.218282in}{3.580372in}} %
\pgfusepath{clip}%
\pgfsetbuttcap%
\pgfsetroundjoin%
\definecolor{currentfill}{rgb}{0.800830,0.250829,0.225696}%
\pgfsetfillcolor{currentfill}%
\pgfsetlinewidth{0.000000pt}%
\definecolor{currentstroke}{rgb}{0.000000,0.000000,0.000000}%
\pgfsetstrokecolor{currentstroke}%
\pgfsetdash{}{0pt}%
\pgfpathmoveto{\pgfqpoint{3.306485in}{2.382499in}}%
\pgfpathlineto{\pgfqpoint{3.564492in}{2.541257in}}%
\pgfpathlineto{\pgfqpoint{3.306485in}{2.382499in}}%
\pgfpathclose%
\pgfusepath{fill}%
\end{pgfscope}%
\begin{pgfscope}%
\pgfpathrectangle{\pgfqpoint{0.017500in}{0.017500in}}{\pgfqpoint{6.218282in}{3.580372in}} %
\pgfusepath{clip}%
\pgfsetbuttcap%
\pgfsetroundjoin%
\definecolor{currentfill}{rgb}{0.800830,0.250829,0.225696}%
\pgfsetfillcolor{currentfill}%
\pgfsetlinewidth{0.000000pt}%
\definecolor{currentstroke}{rgb}{0.000000,0.000000,0.000000}%
\pgfsetstrokecolor{currentstroke}%
\pgfsetdash{}{0pt}%
\pgfpathmoveto{\pgfqpoint{3.306485in}{2.382499in}}%
\pgfpathlineto{\pgfqpoint{3.564492in}{2.541257in}}%
\pgfpathlineto{\pgfqpoint{3.306485in}{2.382499in}}%
\pgfpathclose%
\pgfusepath{fill}%
\end{pgfscope}%
\begin{pgfscope}%
\pgfpathrectangle{\pgfqpoint{0.017500in}{0.017500in}}{\pgfqpoint{6.218282in}{3.580372in}} %
\pgfusepath{clip}%
\pgfsetbuttcap%
\pgfsetroundjoin%
\definecolor{currentfill}{rgb}{0.800830,0.250829,0.225696}%
\pgfsetfillcolor{currentfill}%
\pgfsetlinewidth{0.000000pt}%
\definecolor{currentstroke}{rgb}{0.000000,0.000000,0.000000}%
\pgfsetstrokecolor{currentstroke}%
\pgfsetdash{}{0pt}%
\pgfpathmoveto{\pgfqpoint{3.306485in}{2.382499in}}%
\pgfpathlineto{\pgfqpoint{3.564492in}{2.541257in}}%
\pgfpathlineto{\pgfqpoint{3.306485in}{2.382499in}}%
\pgfpathclose%
\pgfusepath{fill}%
\end{pgfscope}%
\begin{pgfscope}%
\pgfpathrectangle{\pgfqpoint{0.017500in}{0.017500in}}{\pgfqpoint{6.218282in}{3.580372in}} %
\pgfusepath{clip}%
\pgfsetbuttcap%
\pgfsetroundjoin%
\definecolor{currentfill}{rgb}{0.800830,0.250829,0.225696}%
\pgfsetfillcolor{currentfill}%
\pgfsetlinewidth{0.000000pt}%
\definecolor{currentstroke}{rgb}{0.000000,0.000000,0.000000}%
\pgfsetstrokecolor{currentstroke}%
\pgfsetdash{}{0pt}%
\pgfpathmoveto{\pgfqpoint{3.306485in}{2.382499in}}%
\pgfpathlineto{\pgfqpoint{3.564492in}{2.541257in}}%
\pgfpathlineto{\pgfqpoint{3.306485in}{2.382499in}}%
\pgfpathclose%
\pgfusepath{fill}%
\end{pgfscope}%
\begin{pgfscope}%
\pgfpathrectangle{\pgfqpoint{0.017500in}{0.017500in}}{\pgfqpoint{6.218282in}{3.580372in}} %
\pgfusepath{clip}%
\pgfsetbuttcap%
\pgfsetroundjoin%
\definecolor{currentfill}{rgb}{0.800830,0.250829,0.225696}%
\pgfsetfillcolor{currentfill}%
\pgfsetlinewidth{0.000000pt}%
\definecolor{currentstroke}{rgb}{0.000000,0.000000,0.000000}%
\pgfsetstrokecolor{currentstroke}%
\pgfsetdash{}{0pt}%
\pgfpathmoveto{\pgfqpoint{3.306485in}{2.382499in}}%
\pgfpathlineto{\pgfqpoint{3.564492in}{2.541257in}}%
\pgfpathlineto{\pgfqpoint{3.306485in}{2.382499in}}%
\pgfpathclose%
\pgfusepath{fill}%
\end{pgfscope}%
\begin{pgfscope}%
\pgfpathrectangle{\pgfqpoint{0.017500in}{0.017500in}}{\pgfqpoint{6.218282in}{3.580372in}} %
\pgfusepath{clip}%
\pgfsetbuttcap%
\pgfsetroundjoin%
\definecolor{currentfill}{rgb}{0.800830,0.250829,0.225696}%
\pgfsetfillcolor{currentfill}%
\pgfsetlinewidth{0.000000pt}%
\definecolor{currentstroke}{rgb}{0.000000,0.000000,0.000000}%
\pgfsetstrokecolor{currentstroke}%
\pgfsetdash{}{0pt}%
\pgfpathmoveto{\pgfqpoint{3.306485in}{2.382499in}}%
\pgfpathlineto{\pgfqpoint{3.564492in}{2.541257in}}%
\pgfpathlineto{\pgfqpoint{3.306485in}{2.382499in}}%
\pgfpathclose%
\pgfusepath{fill}%
\end{pgfscope}%
\begin{pgfscope}%
\pgfpathrectangle{\pgfqpoint{0.017500in}{0.017500in}}{\pgfqpoint{6.218282in}{3.580372in}} %
\pgfusepath{clip}%
\pgfsetbuttcap%
\pgfsetroundjoin%
\definecolor{currentfill}{rgb}{0.800830,0.250829,0.225696}%
\pgfsetfillcolor{currentfill}%
\pgfsetlinewidth{0.000000pt}%
\definecolor{currentstroke}{rgb}{0.000000,0.000000,0.000000}%
\pgfsetstrokecolor{currentstroke}%
\pgfsetdash{}{0pt}%
\pgfpathmoveto{\pgfqpoint{3.306485in}{2.382499in}}%
\pgfpathlineto{\pgfqpoint{3.564492in}{2.541257in}}%
\pgfpathlineto{\pgfqpoint{3.306485in}{2.382499in}}%
\pgfpathclose%
\pgfusepath{fill}%
\end{pgfscope}%
\begin{pgfscope}%
\pgfpathrectangle{\pgfqpoint{0.017500in}{0.017500in}}{\pgfqpoint{6.218282in}{3.580372in}} %
\pgfusepath{clip}%
\pgfsetbuttcap%
\pgfsetroundjoin%
\definecolor{currentfill}{rgb}{0.800830,0.250829,0.225696}%
\pgfsetfillcolor{currentfill}%
\pgfsetlinewidth{0.000000pt}%
\definecolor{currentstroke}{rgb}{0.000000,0.000000,0.000000}%
\pgfsetstrokecolor{currentstroke}%
\pgfsetdash{}{0pt}%
\pgfpathmoveto{\pgfqpoint{3.306485in}{2.382499in}}%
\pgfpathlineto{\pgfqpoint{3.564492in}{2.541257in}}%
\pgfpathlineto{\pgfqpoint{3.306485in}{2.382499in}}%
\pgfpathclose%
\pgfusepath{fill}%
\end{pgfscope}%
\begin{pgfscope}%
\pgfpathrectangle{\pgfqpoint{0.017500in}{0.017500in}}{\pgfqpoint{6.218282in}{3.580372in}} %
\pgfusepath{clip}%
\pgfsetbuttcap%
\pgfsetroundjoin%
\definecolor{currentfill}{rgb}{0.800830,0.250829,0.225696}%
\pgfsetfillcolor{currentfill}%
\pgfsetlinewidth{0.000000pt}%
\definecolor{currentstroke}{rgb}{0.000000,0.000000,0.000000}%
\pgfsetstrokecolor{currentstroke}%
\pgfsetdash{}{0pt}%
\pgfpathmoveto{\pgfqpoint{3.306485in}{2.382499in}}%
\pgfpathlineto{\pgfqpoint{3.564492in}{2.541257in}}%
\pgfpathlineto{\pgfqpoint{3.306485in}{2.382499in}}%
\pgfpathclose%
\pgfusepath{fill}%
\end{pgfscope}%
\begin{pgfscope}%
\pgfpathrectangle{\pgfqpoint{0.017500in}{0.017500in}}{\pgfqpoint{6.218282in}{3.580372in}} %
\pgfusepath{clip}%
\pgfsetbuttcap%
\pgfsetroundjoin%
\definecolor{currentfill}{rgb}{0.800830,0.250829,0.225696}%
\pgfsetfillcolor{currentfill}%
\pgfsetlinewidth{0.000000pt}%
\definecolor{currentstroke}{rgb}{0.000000,0.000000,0.000000}%
\pgfsetstrokecolor{currentstroke}%
\pgfsetdash{}{0pt}%
\pgfpathmoveto{\pgfqpoint{3.306485in}{2.382499in}}%
\pgfpathlineto{\pgfqpoint{3.564492in}{2.541257in}}%
\pgfpathlineto{\pgfqpoint{3.306485in}{2.382499in}}%
\pgfpathclose%
\pgfusepath{fill}%
\end{pgfscope}%
\begin{pgfscope}%
\pgfpathrectangle{\pgfqpoint{0.017500in}{0.017500in}}{\pgfqpoint{6.218282in}{3.580372in}} %
\pgfusepath{clip}%
\pgfsetbuttcap%
\pgfsetroundjoin%
\definecolor{currentfill}{rgb}{0.800830,0.250829,0.225696}%
\pgfsetfillcolor{currentfill}%
\pgfsetlinewidth{0.000000pt}%
\definecolor{currentstroke}{rgb}{0.000000,0.000000,0.000000}%
\pgfsetstrokecolor{currentstroke}%
\pgfsetdash{}{0pt}%
\pgfpathmoveto{\pgfqpoint{3.306485in}{2.382499in}}%
\pgfpathlineto{\pgfqpoint{3.564492in}{2.541257in}}%
\pgfpathlineto{\pgfqpoint{3.306485in}{2.382499in}}%
\pgfpathclose%
\pgfusepath{fill}%
\end{pgfscope}%
\begin{pgfscope}%
\pgfpathrectangle{\pgfqpoint{0.017500in}{0.017500in}}{\pgfqpoint{6.218282in}{3.580372in}} %
\pgfusepath{clip}%
\pgfsetbuttcap%
\pgfsetroundjoin%
\definecolor{currentfill}{rgb}{0.800830,0.250829,0.225696}%
\pgfsetfillcolor{currentfill}%
\pgfsetlinewidth{0.000000pt}%
\definecolor{currentstroke}{rgb}{0.000000,0.000000,0.000000}%
\pgfsetstrokecolor{currentstroke}%
\pgfsetdash{}{0pt}%
\pgfpathmoveto{\pgfqpoint{3.306485in}{2.382499in}}%
\pgfpathlineto{\pgfqpoint{3.564492in}{2.541257in}}%
\pgfpathlineto{\pgfqpoint{3.306485in}{2.382499in}}%
\pgfpathclose%
\pgfusepath{fill}%
\end{pgfscope}%
\begin{pgfscope}%
\pgfpathrectangle{\pgfqpoint{0.017500in}{0.017500in}}{\pgfqpoint{6.218282in}{3.580372in}} %
\pgfusepath{clip}%
\pgfsetbuttcap%
\pgfsetroundjoin%
\definecolor{currentfill}{rgb}{0.800830,0.250829,0.225696}%
\pgfsetfillcolor{currentfill}%
\pgfsetlinewidth{0.000000pt}%
\definecolor{currentstroke}{rgb}{0.000000,0.000000,0.000000}%
\pgfsetstrokecolor{currentstroke}%
\pgfsetdash{}{0pt}%
\pgfpathmoveto{\pgfqpoint{3.306485in}{2.382499in}}%
\pgfpathlineto{\pgfqpoint{3.564492in}{2.541257in}}%
\pgfpathlineto{\pgfqpoint{3.306485in}{2.382499in}}%
\pgfpathclose%
\pgfusepath{fill}%
\end{pgfscope}%
\begin{pgfscope}%
\pgfpathrectangle{\pgfqpoint{0.017500in}{0.017500in}}{\pgfqpoint{6.218282in}{3.580372in}} %
\pgfusepath{clip}%
\pgfsetbuttcap%
\pgfsetroundjoin%
\definecolor{currentfill}{rgb}{0.800830,0.250829,0.225696}%
\pgfsetfillcolor{currentfill}%
\pgfsetlinewidth{0.000000pt}%
\definecolor{currentstroke}{rgb}{0.000000,0.000000,0.000000}%
\pgfsetstrokecolor{currentstroke}%
\pgfsetdash{}{0pt}%
\pgfpathmoveto{\pgfqpoint{3.306485in}{2.382499in}}%
\pgfpathlineto{\pgfqpoint{3.564492in}{2.541257in}}%
\pgfpathlineto{\pgfqpoint{3.306485in}{2.382499in}}%
\pgfpathclose%
\pgfusepath{fill}%
\end{pgfscope}%
\begin{pgfscope}%
\pgfpathrectangle{\pgfqpoint{0.017500in}{0.017500in}}{\pgfqpoint{6.218282in}{3.580372in}} %
\pgfusepath{clip}%
\pgfsetbuttcap%
\pgfsetroundjoin%
\definecolor{currentfill}{rgb}{0.800830,0.250829,0.225696}%
\pgfsetfillcolor{currentfill}%
\pgfsetlinewidth{0.000000pt}%
\definecolor{currentstroke}{rgb}{0.000000,0.000000,0.000000}%
\pgfsetstrokecolor{currentstroke}%
\pgfsetdash{}{0pt}%
\pgfpathmoveto{\pgfqpoint{3.306485in}{2.382499in}}%
\pgfpathlineto{\pgfqpoint{3.564492in}{2.541257in}}%
\pgfpathlineto{\pgfqpoint{3.306485in}{2.382499in}}%
\pgfpathclose%
\pgfusepath{fill}%
\end{pgfscope}%
\begin{pgfscope}%
\pgfpathrectangle{\pgfqpoint{0.017500in}{0.017500in}}{\pgfqpoint{6.218282in}{3.580372in}} %
\pgfusepath{clip}%
\pgfsetbuttcap%
\pgfsetroundjoin%
\definecolor{currentfill}{rgb}{0.800830,0.250829,0.225696}%
\pgfsetfillcolor{currentfill}%
\pgfsetlinewidth{0.000000pt}%
\definecolor{currentstroke}{rgb}{0.000000,0.000000,0.000000}%
\pgfsetstrokecolor{currentstroke}%
\pgfsetdash{}{0pt}%
\pgfpathmoveto{\pgfqpoint{3.306485in}{2.382499in}}%
\pgfpathlineto{\pgfqpoint{3.564492in}{2.541257in}}%
\pgfpathlineto{\pgfqpoint{3.306485in}{2.382499in}}%
\pgfpathclose%
\pgfusepath{fill}%
\end{pgfscope}%
\begin{pgfscope}%
\pgfpathrectangle{\pgfqpoint{0.017500in}{0.017500in}}{\pgfqpoint{6.218282in}{3.580372in}} %
\pgfusepath{clip}%
\pgfsetbuttcap%
\pgfsetroundjoin%
\definecolor{currentfill}{rgb}{0.800830,0.250829,0.225696}%
\pgfsetfillcolor{currentfill}%
\pgfsetlinewidth{0.000000pt}%
\definecolor{currentstroke}{rgb}{0.000000,0.000000,0.000000}%
\pgfsetstrokecolor{currentstroke}%
\pgfsetdash{}{0pt}%
\pgfpathmoveto{\pgfqpoint{3.306485in}{2.382499in}}%
\pgfpathlineto{\pgfqpoint{3.763398in}{2.297457in}}%
\pgfpathlineto{\pgfqpoint{4.018447in}{2.468167in}}%
\pgfpathlineto{\pgfqpoint{3.564492in}{2.541257in}}%
\pgfpathlineto{\pgfqpoint{3.306485in}{2.382499in}}%
\pgfpathclose%
\pgfusepath{fill}%
\end{pgfscope}%
\begin{pgfscope}%
\pgfpathrectangle{\pgfqpoint{0.017500in}{0.017500in}}{\pgfqpoint{6.218282in}{3.580372in}} %
\pgfusepath{clip}%
\pgfsetbuttcap%
\pgfsetroundjoin%
\definecolor{currentfill}{rgb}{0.800830,0.250829,0.225696}%
\pgfsetfillcolor{currentfill}%
\pgfsetlinewidth{0.000000pt}%
\definecolor{currentstroke}{rgb}{0.000000,0.000000,0.000000}%
\pgfsetstrokecolor{currentstroke}%
\pgfsetdash{}{0pt}%
\pgfpathmoveto{\pgfqpoint{3.306485in}{2.382499in}}%
\pgfpathlineto{\pgfqpoint{3.763398in}{2.297457in}}%
\pgfpathlineto{\pgfqpoint{4.018447in}{2.468167in}}%
\pgfpathlineto{\pgfqpoint{3.564492in}{2.541257in}}%
\pgfpathlineto{\pgfqpoint{3.306485in}{2.382499in}}%
\pgfpathclose%
\pgfusepath{fill}%
\end{pgfscope}%
\begin{pgfscope}%
\pgfpathrectangle{\pgfqpoint{0.017500in}{0.017500in}}{\pgfqpoint{6.218282in}{3.580372in}} %
\pgfusepath{clip}%
\pgfsetbuttcap%
\pgfsetroundjoin%
\definecolor{currentfill}{rgb}{0.800830,0.250829,0.225696}%
\pgfsetfillcolor{currentfill}%
\pgfsetlinewidth{0.000000pt}%
\definecolor{currentstroke}{rgb}{0.000000,0.000000,0.000000}%
\pgfsetstrokecolor{currentstroke}%
\pgfsetdash{}{0pt}%
\pgfpathmoveto{\pgfqpoint{3.306485in}{2.382499in}}%
\pgfpathlineto{\pgfqpoint{3.763398in}{2.297457in}}%
\pgfpathlineto{\pgfqpoint{4.018447in}{2.468167in}}%
\pgfpathlineto{\pgfqpoint{3.564492in}{2.541257in}}%
\pgfpathlineto{\pgfqpoint{3.306485in}{2.382499in}}%
\pgfpathclose%
\pgfusepath{fill}%
\end{pgfscope}%
\begin{pgfscope}%
\pgfpathrectangle{\pgfqpoint{0.017500in}{0.017500in}}{\pgfqpoint{6.218282in}{3.580372in}} %
\pgfusepath{clip}%
\pgfsetbuttcap%
\pgfsetroundjoin%
\definecolor{currentfill}{rgb}{0.800830,0.250829,0.225696}%
\pgfsetfillcolor{currentfill}%
\pgfsetlinewidth{0.000000pt}%
\definecolor{currentstroke}{rgb}{0.000000,0.000000,0.000000}%
\pgfsetstrokecolor{currentstroke}%
\pgfsetdash{}{0pt}%
\pgfpathmoveto{\pgfqpoint{3.306485in}{2.382499in}}%
\pgfpathlineto{\pgfqpoint{3.763398in}{2.297457in}}%
\pgfpathlineto{\pgfqpoint{4.018447in}{2.468167in}}%
\pgfpathlineto{\pgfqpoint{3.564492in}{2.541257in}}%
\pgfpathlineto{\pgfqpoint{3.306485in}{2.382499in}}%
\pgfpathclose%
\pgfusepath{fill}%
\end{pgfscope}%
\begin{pgfscope}%
\pgfpathrectangle{\pgfqpoint{0.017500in}{0.017500in}}{\pgfqpoint{6.218282in}{3.580372in}} %
\pgfusepath{clip}%
\pgfsetbuttcap%
\pgfsetroundjoin%
\definecolor{currentfill}{rgb}{0.800830,0.250829,0.225696}%
\pgfsetfillcolor{currentfill}%
\pgfsetlinewidth{0.000000pt}%
\definecolor{currentstroke}{rgb}{0.000000,0.000000,0.000000}%
\pgfsetstrokecolor{currentstroke}%
\pgfsetdash{}{0pt}%
\pgfpathmoveto{\pgfqpoint{3.306485in}{2.382499in}}%
\pgfpathlineto{\pgfqpoint{3.763398in}{2.297457in}}%
\pgfpathlineto{\pgfqpoint{4.018447in}{2.468167in}}%
\pgfpathlineto{\pgfqpoint{3.564492in}{2.541257in}}%
\pgfpathlineto{\pgfqpoint{3.306485in}{2.382499in}}%
\pgfpathclose%
\pgfusepath{fill}%
\end{pgfscope}%
\begin{pgfscope}%
\pgfpathrectangle{\pgfqpoint{0.017500in}{0.017500in}}{\pgfqpoint{6.218282in}{3.580372in}} %
\pgfusepath{clip}%
\pgfsetbuttcap%
\pgfsetroundjoin%
\definecolor{currentfill}{rgb}{0.800830,0.250829,0.225696}%
\pgfsetfillcolor{currentfill}%
\pgfsetlinewidth{0.000000pt}%
\definecolor{currentstroke}{rgb}{0.000000,0.000000,0.000000}%
\pgfsetstrokecolor{currentstroke}%
\pgfsetdash{}{0pt}%
\pgfpathmoveto{\pgfqpoint{3.306485in}{2.382499in}}%
\pgfpathlineto{\pgfqpoint{3.763398in}{2.297457in}}%
\pgfpathlineto{\pgfqpoint{4.018447in}{2.468167in}}%
\pgfpathlineto{\pgfqpoint{3.564492in}{2.541257in}}%
\pgfpathlineto{\pgfqpoint{3.306485in}{2.382499in}}%
\pgfpathclose%
\pgfusepath{fill}%
\end{pgfscope}%
\begin{pgfscope}%
\pgfpathrectangle{\pgfqpoint{0.017500in}{0.017500in}}{\pgfqpoint{6.218282in}{3.580372in}} %
\pgfusepath{clip}%
\pgfsetbuttcap%
\pgfsetroundjoin%
\definecolor{currentfill}{rgb}{0.899534,0.440692,0.344107}%
\pgfsetfillcolor{currentfill}%
\pgfsetlinewidth{0.000000pt}%
\definecolor{currentstroke}{rgb}{0.000000,0.000000,0.000000}%
\pgfsetstrokecolor{currentstroke}%
\pgfsetdash{}{0pt}%
\pgfpathmoveto{\pgfqpoint{2.590432in}{2.210709in}}%
\pgfpathlineto{\pgfqpoint{2.855513in}{2.451418in}}%
\pgfpathlineto{\pgfqpoint{2.590432in}{2.210709in}}%
\pgfpathclose%
\pgfusepath{fill}%
\end{pgfscope}%
\begin{pgfscope}%
\pgfpathrectangle{\pgfqpoint{0.017500in}{0.017500in}}{\pgfqpoint{6.218282in}{3.580372in}} %
\pgfusepath{clip}%
\pgfsetbuttcap%
\pgfsetroundjoin%
\definecolor{currentfill}{rgb}{0.899534,0.440692,0.344107}%
\pgfsetfillcolor{currentfill}%
\pgfsetlinewidth{0.000000pt}%
\definecolor{currentstroke}{rgb}{0.000000,0.000000,0.000000}%
\pgfsetstrokecolor{currentstroke}%
\pgfsetdash{}{0pt}%
\pgfpathmoveto{\pgfqpoint{2.590432in}{2.210709in}}%
\pgfpathlineto{\pgfqpoint{2.855513in}{2.451418in}}%
\pgfpathlineto{\pgfqpoint{2.590432in}{2.210709in}}%
\pgfpathclose%
\pgfusepath{fill}%
\end{pgfscope}%
\begin{pgfscope}%
\pgfpathrectangle{\pgfqpoint{0.017500in}{0.017500in}}{\pgfqpoint{6.218282in}{3.580372in}} %
\pgfusepath{clip}%
\pgfsetbuttcap%
\pgfsetroundjoin%
\definecolor{currentfill}{rgb}{0.899534,0.440692,0.344107}%
\pgfsetfillcolor{currentfill}%
\pgfsetlinewidth{0.000000pt}%
\definecolor{currentstroke}{rgb}{0.000000,0.000000,0.000000}%
\pgfsetstrokecolor{currentstroke}%
\pgfsetdash{}{0pt}%
\pgfpathmoveto{\pgfqpoint{2.590432in}{2.210709in}}%
\pgfpathlineto{\pgfqpoint{2.855513in}{2.451418in}}%
\pgfpathlineto{\pgfqpoint{2.590432in}{2.210709in}}%
\pgfpathclose%
\pgfusepath{fill}%
\end{pgfscope}%
\begin{pgfscope}%
\pgfpathrectangle{\pgfqpoint{0.017500in}{0.017500in}}{\pgfqpoint{6.218282in}{3.580372in}} %
\pgfusepath{clip}%
\pgfsetbuttcap%
\pgfsetroundjoin%
\definecolor{currentfill}{rgb}{0.899534,0.440692,0.344107}%
\pgfsetfillcolor{currentfill}%
\pgfsetlinewidth{0.000000pt}%
\definecolor{currentstroke}{rgb}{0.000000,0.000000,0.000000}%
\pgfsetstrokecolor{currentstroke}%
\pgfsetdash{}{0pt}%
\pgfpathmoveto{\pgfqpoint{2.590432in}{2.210709in}}%
\pgfpathlineto{\pgfqpoint{2.855513in}{2.451418in}}%
\pgfpathlineto{\pgfqpoint{2.590432in}{2.210709in}}%
\pgfpathclose%
\pgfusepath{fill}%
\end{pgfscope}%
\begin{pgfscope}%
\pgfpathrectangle{\pgfqpoint{0.017500in}{0.017500in}}{\pgfqpoint{6.218282in}{3.580372in}} %
\pgfusepath{clip}%
\pgfsetbuttcap%
\pgfsetroundjoin%
\definecolor{currentfill}{rgb}{0.899534,0.440692,0.344107}%
\pgfsetfillcolor{currentfill}%
\pgfsetlinewidth{0.000000pt}%
\definecolor{currentstroke}{rgb}{0.000000,0.000000,0.000000}%
\pgfsetstrokecolor{currentstroke}%
\pgfsetdash{}{0pt}%
\pgfpathmoveto{\pgfqpoint{2.590432in}{2.210709in}}%
\pgfpathlineto{\pgfqpoint{2.855513in}{2.451418in}}%
\pgfpathlineto{\pgfqpoint{2.590432in}{2.210709in}}%
\pgfpathclose%
\pgfusepath{fill}%
\end{pgfscope}%
\begin{pgfscope}%
\pgfpathrectangle{\pgfqpoint{0.017500in}{0.017500in}}{\pgfqpoint{6.218282in}{3.580372in}} %
\pgfusepath{clip}%
\pgfsetbuttcap%
\pgfsetroundjoin%
\definecolor{currentfill}{rgb}{0.899534,0.440692,0.344107}%
\pgfsetfillcolor{currentfill}%
\pgfsetlinewidth{0.000000pt}%
\definecolor{currentstroke}{rgb}{0.000000,0.000000,0.000000}%
\pgfsetstrokecolor{currentstroke}%
\pgfsetdash{}{0pt}%
\pgfpathmoveto{\pgfqpoint{2.590432in}{2.210709in}}%
\pgfpathlineto{\pgfqpoint{2.855513in}{2.451418in}}%
\pgfpathlineto{\pgfqpoint{2.590432in}{2.210709in}}%
\pgfpathclose%
\pgfusepath{fill}%
\end{pgfscope}%
\begin{pgfscope}%
\pgfpathrectangle{\pgfqpoint{0.017500in}{0.017500in}}{\pgfqpoint{6.218282in}{3.580372in}} %
\pgfusepath{clip}%
\pgfsetbuttcap%
\pgfsetroundjoin%
\definecolor{currentfill}{rgb}{0.899534,0.440692,0.344107}%
\pgfsetfillcolor{currentfill}%
\pgfsetlinewidth{0.000000pt}%
\definecolor{currentstroke}{rgb}{0.000000,0.000000,0.000000}%
\pgfsetstrokecolor{currentstroke}%
\pgfsetdash{}{0pt}%
\pgfpathmoveto{\pgfqpoint{2.590432in}{2.210709in}}%
\pgfpathlineto{\pgfqpoint{2.855513in}{2.451418in}}%
\pgfpathlineto{\pgfqpoint{2.590432in}{2.210709in}}%
\pgfpathclose%
\pgfusepath{fill}%
\end{pgfscope}%
\begin{pgfscope}%
\pgfpathrectangle{\pgfqpoint{0.017500in}{0.017500in}}{\pgfqpoint{6.218282in}{3.580372in}} %
\pgfusepath{clip}%
\pgfsetbuttcap%
\pgfsetroundjoin%
\definecolor{currentfill}{rgb}{0.899534,0.440692,0.344107}%
\pgfsetfillcolor{currentfill}%
\pgfsetlinewidth{0.000000pt}%
\definecolor{currentstroke}{rgb}{0.000000,0.000000,0.000000}%
\pgfsetstrokecolor{currentstroke}%
\pgfsetdash{}{0pt}%
\pgfpathmoveto{\pgfqpoint{2.590432in}{2.210709in}}%
\pgfpathlineto{\pgfqpoint{2.855513in}{2.451418in}}%
\pgfpathlineto{\pgfqpoint{2.590432in}{2.210709in}}%
\pgfpathclose%
\pgfusepath{fill}%
\end{pgfscope}%
\begin{pgfscope}%
\pgfpathrectangle{\pgfqpoint{0.017500in}{0.017500in}}{\pgfqpoint{6.218282in}{3.580372in}} %
\pgfusepath{clip}%
\pgfsetbuttcap%
\pgfsetroundjoin%
\definecolor{currentfill}{rgb}{0.899534,0.440692,0.344107}%
\pgfsetfillcolor{currentfill}%
\pgfsetlinewidth{0.000000pt}%
\definecolor{currentstroke}{rgb}{0.000000,0.000000,0.000000}%
\pgfsetstrokecolor{currentstroke}%
\pgfsetdash{}{0pt}%
\pgfpathmoveto{\pgfqpoint{2.590432in}{2.210709in}}%
\pgfpathlineto{\pgfqpoint{2.855513in}{2.451418in}}%
\pgfpathlineto{\pgfqpoint{2.590432in}{2.210709in}}%
\pgfpathclose%
\pgfusepath{fill}%
\end{pgfscope}%
\begin{pgfscope}%
\pgfpathrectangle{\pgfqpoint{0.017500in}{0.017500in}}{\pgfqpoint{6.218282in}{3.580372in}} %
\pgfusepath{clip}%
\pgfsetbuttcap%
\pgfsetroundjoin%
\definecolor{currentfill}{rgb}{0.899534,0.440692,0.344107}%
\pgfsetfillcolor{currentfill}%
\pgfsetlinewidth{0.000000pt}%
\definecolor{currentstroke}{rgb}{0.000000,0.000000,0.000000}%
\pgfsetstrokecolor{currentstroke}%
\pgfsetdash{}{0pt}%
\pgfpathmoveto{\pgfqpoint{2.590432in}{2.210709in}}%
\pgfpathlineto{\pgfqpoint{2.855513in}{2.451418in}}%
\pgfpathlineto{\pgfqpoint{2.590432in}{2.210709in}}%
\pgfpathclose%
\pgfusepath{fill}%
\end{pgfscope}%
\begin{pgfscope}%
\pgfpathrectangle{\pgfqpoint{0.017500in}{0.017500in}}{\pgfqpoint{6.218282in}{3.580372in}} %
\pgfusepath{clip}%
\pgfsetbuttcap%
\pgfsetroundjoin%
\definecolor{currentfill}{rgb}{0.899534,0.440692,0.344107}%
\pgfsetfillcolor{currentfill}%
\pgfsetlinewidth{0.000000pt}%
\definecolor{currentstroke}{rgb}{0.000000,0.000000,0.000000}%
\pgfsetstrokecolor{currentstroke}%
\pgfsetdash{}{0pt}%
\pgfpathmoveto{\pgfqpoint{2.590432in}{2.210709in}}%
\pgfpathlineto{\pgfqpoint{2.855513in}{2.451418in}}%
\pgfpathlineto{\pgfqpoint{2.590432in}{2.210709in}}%
\pgfpathclose%
\pgfusepath{fill}%
\end{pgfscope}%
\begin{pgfscope}%
\pgfpathrectangle{\pgfqpoint{0.017500in}{0.017500in}}{\pgfqpoint{6.218282in}{3.580372in}} %
\pgfusepath{clip}%
\pgfsetbuttcap%
\pgfsetroundjoin%
\definecolor{currentfill}{rgb}{0.899534,0.440692,0.344107}%
\pgfsetfillcolor{currentfill}%
\pgfsetlinewidth{0.000000pt}%
\definecolor{currentstroke}{rgb}{0.000000,0.000000,0.000000}%
\pgfsetstrokecolor{currentstroke}%
\pgfsetdash{}{0pt}%
\pgfpathmoveto{\pgfqpoint{2.590432in}{2.210709in}}%
\pgfpathlineto{\pgfqpoint{2.855513in}{2.451418in}}%
\pgfpathlineto{\pgfqpoint{2.590432in}{2.210709in}}%
\pgfpathclose%
\pgfusepath{fill}%
\end{pgfscope}%
\begin{pgfscope}%
\pgfpathrectangle{\pgfqpoint{0.017500in}{0.017500in}}{\pgfqpoint{6.218282in}{3.580372in}} %
\pgfusepath{clip}%
\pgfsetbuttcap%
\pgfsetroundjoin%
\definecolor{currentfill}{rgb}{0.899534,0.440692,0.344107}%
\pgfsetfillcolor{currentfill}%
\pgfsetlinewidth{0.000000pt}%
\definecolor{currentstroke}{rgb}{0.000000,0.000000,0.000000}%
\pgfsetstrokecolor{currentstroke}%
\pgfsetdash{}{0pt}%
\pgfpathmoveto{\pgfqpoint{2.590432in}{2.210709in}}%
\pgfpathlineto{\pgfqpoint{2.855513in}{2.451418in}}%
\pgfpathlineto{\pgfqpoint{2.590432in}{2.210709in}}%
\pgfpathclose%
\pgfusepath{fill}%
\end{pgfscope}%
\begin{pgfscope}%
\pgfpathrectangle{\pgfqpoint{0.017500in}{0.017500in}}{\pgfqpoint{6.218282in}{3.580372in}} %
\pgfusepath{clip}%
\pgfsetbuttcap%
\pgfsetroundjoin%
\definecolor{currentfill}{rgb}{0.899534,0.440692,0.344107}%
\pgfsetfillcolor{currentfill}%
\pgfsetlinewidth{0.000000pt}%
\definecolor{currentstroke}{rgb}{0.000000,0.000000,0.000000}%
\pgfsetstrokecolor{currentstroke}%
\pgfsetdash{}{0pt}%
\pgfpathmoveto{\pgfqpoint{2.590432in}{2.210709in}}%
\pgfpathlineto{\pgfqpoint{2.855513in}{2.451418in}}%
\pgfpathlineto{\pgfqpoint{2.590432in}{2.210709in}}%
\pgfpathclose%
\pgfusepath{fill}%
\end{pgfscope}%
\begin{pgfscope}%
\pgfpathrectangle{\pgfqpoint{0.017500in}{0.017500in}}{\pgfqpoint{6.218282in}{3.580372in}} %
\pgfusepath{clip}%
\pgfsetbuttcap%
\pgfsetroundjoin%
\definecolor{currentfill}{rgb}{0.899534,0.440692,0.344107}%
\pgfsetfillcolor{currentfill}%
\pgfsetlinewidth{0.000000pt}%
\definecolor{currentstroke}{rgb}{0.000000,0.000000,0.000000}%
\pgfsetstrokecolor{currentstroke}%
\pgfsetdash{}{0pt}%
\pgfpathmoveto{\pgfqpoint{2.590432in}{2.210709in}}%
\pgfpathlineto{\pgfqpoint{2.855513in}{2.451418in}}%
\pgfpathlineto{\pgfqpoint{2.590432in}{2.210709in}}%
\pgfpathclose%
\pgfusepath{fill}%
\end{pgfscope}%
\begin{pgfscope}%
\pgfpathrectangle{\pgfqpoint{0.017500in}{0.017500in}}{\pgfqpoint{6.218282in}{3.580372in}} %
\pgfusepath{clip}%
\pgfsetbuttcap%
\pgfsetroundjoin%
\definecolor{currentfill}{rgb}{0.899534,0.440692,0.344107}%
\pgfsetfillcolor{currentfill}%
\pgfsetlinewidth{0.000000pt}%
\definecolor{currentstroke}{rgb}{0.000000,0.000000,0.000000}%
\pgfsetstrokecolor{currentstroke}%
\pgfsetdash{}{0pt}%
\pgfpathmoveto{\pgfqpoint{2.590432in}{2.210709in}}%
\pgfpathlineto{\pgfqpoint{2.855513in}{2.451418in}}%
\pgfpathlineto{\pgfqpoint{2.590432in}{2.210709in}}%
\pgfpathclose%
\pgfusepath{fill}%
\end{pgfscope}%
\begin{pgfscope}%
\pgfpathrectangle{\pgfqpoint{0.017500in}{0.017500in}}{\pgfqpoint{6.218282in}{3.580372in}} %
\pgfusepath{clip}%
\pgfsetbuttcap%
\pgfsetroundjoin%
\definecolor{currentfill}{rgb}{0.899534,0.440692,0.344107}%
\pgfsetfillcolor{currentfill}%
\pgfsetlinewidth{0.000000pt}%
\definecolor{currentstroke}{rgb}{0.000000,0.000000,0.000000}%
\pgfsetstrokecolor{currentstroke}%
\pgfsetdash{}{0pt}%
\pgfpathmoveto{\pgfqpoint{2.590432in}{2.210709in}}%
\pgfpathlineto{\pgfqpoint{2.855513in}{2.451418in}}%
\pgfpathlineto{\pgfqpoint{2.590432in}{2.210709in}}%
\pgfpathclose%
\pgfusepath{fill}%
\end{pgfscope}%
\begin{pgfscope}%
\pgfpathrectangle{\pgfqpoint{0.017500in}{0.017500in}}{\pgfqpoint{6.218282in}{3.580372in}} %
\pgfusepath{clip}%
\pgfsetbuttcap%
\pgfsetroundjoin%
\definecolor{currentfill}{rgb}{0.899534,0.440692,0.344107}%
\pgfsetfillcolor{currentfill}%
\pgfsetlinewidth{0.000000pt}%
\definecolor{currentstroke}{rgb}{0.000000,0.000000,0.000000}%
\pgfsetstrokecolor{currentstroke}%
\pgfsetdash{}{0pt}%
\pgfpathmoveto{\pgfqpoint{2.590432in}{2.210709in}}%
\pgfpathlineto{\pgfqpoint{2.855513in}{2.451418in}}%
\pgfpathlineto{\pgfqpoint{2.590432in}{2.210709in}}%
\pgfpathclose%
\pgfusepath{fill}%
\end{pgfscope}%
\begin{pgfscope}%
\pgfpathrectangle{\pgfqpoint{0.017500in}{0.017500in}}{\pgfqpoint{6.218282in}{3.580372in}} %
\pgfusepath{clip}%
\pgfsetbuttcap%
\pgfsetroundjoin%
\definecolor{currentfill}{rgb}{0.899534,0.440692,0.344107}%
\pgfsetfillcolor{currentfill}%
\pgfsetlinewidth{0.000000pt}%
\definecolor{currentstroke}{rgb}{0.000000,0.000000,0.000000}%
\pgfsetstrokecolor{currentstroke}%
\pgfsetdash{}{0pt}%
\pgfpathmoveto{\pgfqpoint{2.590432in}{2.210709in}}%
\pgfpathlineto{\pgfqpoint{2.855513in}{2.451418in}}%
\pgfpathlineto{\pgfqpoint{2.590432in}{2.210709in}}%
\pgfpathclose%
\pgfusepath{fill}%
\end{pgfscope}%
\begin{pgfscope}%
\pgfpathrectangle{\pgfqpoint{0.017500in}{0.017500in}}{\pgfqpoint{6.218282in}{3.580372in}} %
\pgfusepath{clip}%
\pgfsetbuttcap%
\pgfsetroundjoin%
\definecolor{currentfill}{rgb}{0.899534,0.440692,0.344107}%
\pgfsetfillcolor{currentfill}%
\pgfsetlinewidth{0.000000pt}%
\definecolor{currentstroke}{rgb}{0.000000,0.000000,0.000000}%
\pgfsetstrokecolor{currentstroke}%
\pgfsetdash{}{0pt}%
\pgfpathmoveto{\pgfqpoint{2.590432in}{2.210709in}}%
\pgfpathlineto{\pgfqpoint{2.855513in}{2.451418in}}%
\pgfpathlineto{\pgfqpoint{2.590432in}{2.210709in}}%
\pgfpathclose%
\pgfusepath{fill}%
\end{pgfscope}%
\begin{pgfscope}%
\pgfpathrectangle{\pgfqpoint{0.017500in}{0.017500in}}{\pgfqpoint{6.218282in}{3.580372in}} %
\pgfusepath{clip}%
\pgfsetbuttcap%
\pgfsetroundjoin%
\definecolor{currentfill}{rgb}{0.899534,0.440692,0.344107}%
\pgfsetfillcolor{currentfill}%
\pgfsetlinewidth{0.000000pt}%
\definecolor{currentstroke}{rgb}{0.000000,0.000000,0.000000}%
\pgfsetstrokecolor{currentstroke}%
\pgfsetdash{}{0pt}%
\pgfpathmoveto{\pgfqpoint{2.590432in}{2.210709in}}%
\pgfpathlineto{\pgfqpoint{2.855513in}{2.451418in}}%
\pgfpathlineto{\pgfqpoint{2.590432in}{2.210709in}}%
\pgfpathclose%
\pgfusepath{fill}%
\end{pgfscope}%
\begin{pgfscope}%
\pgfpathrectangle{\pgfqpoint{0.017500in}{0.017500in}}{\pgfqpoint{6.218282in}{3.580372in}} %
\pgfusepath{clip}%
\pgfsetbuttcap%
\pgfsetroundjoin%
\definecolor{currentfill}{rgb}{0.899534,0.440692,0.344107}%
\pgfsetfillcolor{currentfill}%
\pgfsetlinewidth{0.000000pt}%
\definecolor{currentstroke}{rgb}{0.000000,0.000000,0.000000}%
\pgfsetstrokecolor{currentstroke}%
\pgfsetdash{}{0pt}%
\pgfpathmoveto{\pgfqpoint{2.590432in}{2.210709in}}%
\pgfpathlineto{\pgfqpoint{2.855513in}{2.451418in}}%
\pgfpathlineto{\pgfqpoint{2.590432in}{2.210709in}}%
\pgfpathclose%
\pgfusepath{fill}%
\end{pgfscope}%
\begin{pgfscope}%
\pgfpathrectangle{\pgfqpoint{0.017500in}{0.017500in}}{\pgfqpoint{6.218282in}{3.580372in}} %
\pgfusepath{clip}%
\pgfsetbuttcap%
\pgfsetroundjoin%
\definecolor{currentfill}{rgb}{0.899534,0.440692,0.344107}%
\pgfsetfillcolor{currentfill}%
\pgfsetlinewidth{0.000000pt}%
\definecolor{currentstroke}{rgb}{0.000000,0.000000,0.000000}%
\pgfsetstrokecolor{currentstroke}%
\pgfsetdash{}{0pt}%
\pgfpathmoveto{\pgfqpoint{2.590432in}{2.210709in}}%
\pgfpathlineto{\pgfqpoint{2.855513in}{2.451418in}}%
\pgfpathlineto{\pgfqpoint{2.590432in}{2.210709in}}%
\pgfpathclose%
\pgfusepath{fill}%
\end{pgfscope}%
\begin{pgfscope}%
\pgfpathrectangle{\pgfqpoint{0.017500in}{0.017500in}}{\pgfqpoint{6.218282in}{3.580372in}} %
\pgfusepath{clip}%
\pgfsetbuttcap%
\pgfsetroundjoin%
\definecolor{currentfill}{rgb}{0.899534,0.440692,0.344107}%
\pgfsetfillcolor{currentfill}%
\pgfsetlinewidth{0.000000pt}%
\definecolor{currentstroke}{rgb}{0.000000,0.000000,0.000000}%
\pgfsetstrokecolor{currentstroke}%
\pgfsetdash{}{0pt}%
\pgfpathmoveto{\pgfqpoint{2.590432in}{2.210709in}}%
\pgfpathlineto{\pgfqpoint{2.855513in}{2.451418in}}%
\pgfpathlineto{\pgfqpoint{2.590432in}{2.210709in}}%
\pgfpathclose%
\pgfusepath{fill}%
\end{pgfscope}%
\begin{pgfscope}%
\pgfpathrectangle{\pgfqpoint{0.017500in}{0.017500in}}{\pgfqpoint{6.218282in}{3.580372in}} %
\pgfusepath{clip}%
\pgfsetbuttcap%
\pgfsetroundjoin%
\definecolor{currentfill}{rgb}{0.899534,0.440692,0.344107}%
\pgfsetfillcolor{currentfill}%
\pgfsetlinewidth{0.000000pt}%
\definecolor{currentstroke}{rgb}{0.000000,0.000000,0.000000}%
\pgfsetstrokecolor{currentstroke}%
\pgfsetdash{}{0pt}%
\pgfpathmoveto{\pgfqpoint{2.590432in}{2.210709in}}%
\pgfpathlineto{\pgfqpoint{2.855513in}{2.451418in}}%
\pgfpathlineto{\pgfqpoint{2.590432in}{2.210709in}}%
\pgfpathclose%
\pgfusepath{fill}%
\end{pgfscope}%
\begin{pgfscope}%
\pgfpathrectangle{\pgfqpoint{0.017500in}{0.017500in}}{\pgfqpoint{6.218282in}{3.580372in}} %
\pgfusepath{clip}%
\pgfsetbuttcap%
\pgfsetroundjoin%
\definecolor{currentfill}{rgb}{0.899534,0.440692,0.344107}%
\pgfsetfillcolor{currentfill}%
\pgfsetlinewidth{0.000000pt}%
\definecolor{currentstroke}{rgb}{0.000000,0.000000,0.000000}%
\pgfsetstrokecolor{currentstroke}%
\pgfsetdash{}{0pt}%
\pgfpathmoveto{\pgfqpoint{2.590432in}{2.210709in}}%
\pgfpathlineto{\pgfqpoint{2.855513in}{2.451418in}}%
\pgfpathlineto{\pgfqpoint{2.590432in}{2.210709in}}%
\pgfpathclose%
\pgfusepath{fill}%
\end{pgfscope}%
\begin{pgfscope}%
\pgfpathrectangle{\pgfqpoint{0.017500in}{0.017500in}}{\pgfqpoint{6.218282in}{3.580372in}} %
\pgfusepath{clip}%
\pgfsetbuttcap%
\pgfsetroundjoin%
\definecolor{currentfill}{rgb}{0.899534,0.440692,0.344107}%
\pgfsetfillcolor{currentfill}%
\pgfsetlinewidth{0.000000pt}%
\definecolor{currentstroke}{rgb}{0.000000,0.000000,0.000000}%
\pgfsetstrokecolor{currentstroke}%
\pgfsetdash{}{0pt}%
\pgfpathmoveto{\pgfqpoint{2.590432in}{2.210709in}}%
\pgfpathlineto{\pgfqpoint{2.855513in}{2.451418in}}%
\pgfpathlineto{\pgfqpoint{2.590432in}{2.210709in}}%
\pgfpathclose%
\pgfusepath{fill}%
\end{pgfscope}%
\begin{pgfscope}%
\pgfpathrectangle{\pgfqpoint{0.017500in}{0.017500in}}{\pgfqpoint{6.218282in}{3.580372in}} %
\pgfusepath{clip}%
\pgfsetbuttcap%
\pgfsetroundjoin%
\definecolor{currentfill}{rgb}{0.899534,0.440692,0.344107}%
\pgfsetfillcolor{currentfill}%
\pgfsetlinewidth{0.000000pt}%
\definecolor{currentstroke}{rgb}{0.000000,0.000000,0.000000}%
\pgfsetstrokecolor{currentstroke}%
\pgfsetdash{}{0pt}%
\pgfpathmoveto{\pgfqpoint{2.590432in}{2.210709in}}%
\pgfpathlineto{\pgfqpoint{2.855513in}{2.451418in}}%
\pgfpathlineto{\pgfqpoint{2.590432in}{2.210709in}}%
\pgfpathclose%
\pgfusepath{fill}%
\end{pgfscope}%
\begin{pgfscope}%
\pgfpathrectangle{\pgfqpoint{0.017500in}{0.017500in}}{\pgfqpoint{6.218282in}{3.580372in}} %
\pgfusepath{clip}%
\pgfsetbuttcap%
\pgfsetroundjoin%
\definecolor{currentfill}{rgb}{0.899534,0.440692,0.344107}%
\pgfsetfillcolor{currentfill}%
\pgfsetlinewidth{0.000000pt}%
\definecolor{currentstroke}{rgb}{0.000000,0.000000,0.000000}%
\pgfsetstrokecolor{currentstroke}%
\pgfsetdash{}{0pt}%
\pgfpathmoveto{\pgfqpoint{2.590432in}{2.210709in}}%
\pgfpathlineto{\pgfqpoint{2.855513in}{2.451418in}}%
\pgfpathlineto{\pgfqpoint{2.590432in}{2.210709in}}%
\pgfpathclose%
\pgfusepath{fill}%
\end{pgfscope}%
\begin{pgfscope}%
\pgfpathrectangle{\pgfqpoint{0.017500in}{0.017500in}}{\pgfqpoint{6.218282in}{3.580372in}} %
\pgfusepath{clip}%
\pgfsetbuttcap%
\pgfsetroundjoin%
\definecolor{currentfill}{rgb}{0.899534,0.440692,0.344107}%
\pgfsetfillcolor{currentfill}%
\pgfsetlinewidth{0.000000pt}%
\definecolor{currentstroke}{rgb}{0.000000,0.000000,0.000000}%
\pgfsetstrokecolor{currentstroke}%
\pgfsetdash{}{0pt}%
\pgfpathmoveto{\pgfqpoint{2.590432in}{2.210709in}}%
\pgfpathlineto{\pgfqpoint{2.855513in}{2.451418in}}%
\pgfpathlineto{\pgfqpoint{2.590432in}{2.210709in}}%
\pgfpathclose%
\pgfusepath{fill}%
\end{pgfscope}%
\begin{pgfscope}%
\pgfpathrectangle{\pgfqpoint{0.017500in}{0.017500in}}{\pgfqpoint{6.218282in}{3.580372in}} %
\pgfusepath{clip}%
\pgfsetbuttcap%
\pgfsetroundjoin%
\definecolor{currentfill}{rgb}{0.243520,0.319189,0.771672}%
\pgfsetfillcolor{currentfill}%
\pgfsetlinewidth{0.000000pt}%
\definecolor{currentstroke}{rgb}{0.000000,0.000000,0.000000}%
\pgfsetstrokecolor{currentstroke}%
\pgfsetdash{}{0pt}%
\pgfpathmoveto{\pgfqpoint{3.236099in}{0.723497in}}%
\pgfpathlineto{\pgfqpoint{3.502561in}{2.073396in}}%
\pgfpathlineto{\pgfqpoint{3.236099in}{0.723497in}}%
\pgfpathclose%
\pgfusepath{fill}%
\end{pgfscope}%
\begin{pgfscope}%
\pgfpathrectangle{\pgfqpoint{0.017500in}{0.017500in}}{\pgfqpoint{6.218282in}{3.580372in}} %
\pgfusepath{clip}%
\pgfsetbuttcap%
\pgfsetroundjoin%
\definecolor{currentfill}{rgb}{0.243520,0.319189,0.771672}%
\pgfsetfillcolor{currentfill}%
\pgfsetlinewidth{0.000000pt}%
\definecolor{currentstroke}{rgb}{0.000000,0.000000,0.000000}%
\pgfsetstrokecolor{currentstroke}%
\pgfsetdash{}{0pt}%
\pgfpathmoveto{\pgfqpoint{3.236099in}{0.723497in}}%
\pgfpathlineto{\pgfqpoint{3.502561in}{2.073396in}}%
\pgfpathlineto{\pgfqpoint{3.236099in}{0.723497in}}%
\pgfpathclose%
\pgfusepath{fill}%
\end{pgfscope}%
\begin{pgfscope}%
\pgfpathrectangle{\pgfqpoint{0.017500in}{0.017500in}}{\pgfqpoint{6.218282in}{3.580372in}} %
\pgfusepath{clip}%
\pgfsetbuttcap%
\pgfsetroundjoin%
\definecolor{currentfill}{rgb}{0.243520,0.319189,0.771672}%
\pgfsetfillcolor{currentfill}%
\pgfsetlinewidth{0.000000pt}%
\definecolor{currentstroke}{rgb}{0.000000,0.000000,0.000000}%
\pgfsetstrokecolor{currentstroke}%
\pgfsetdash{}{0pt}%
\pgfpathmoveto{\pgfqpoint{3.236099in}{0.723497in}}%
\pgfpathlineto{\pgfqpoint{3.502561in}{2.073396in}}%
\pgfpathlineto{\pgfqpoint{3.236099in}{0.723497in}}%
\pgfpathclose%
\pgfusepath{fill}%
\end{pgfscope}%
\begin{pgfscope}%
\pgfpathrectangle{\pgfqpoint{0.017500in}{0.017500in}}{\pgfqpoint{6.218282in}{3.580372in}} %
\pgfusepath{clip}%
\pgfsetbuttcap%
\pgfsetroundjoin%
\definecolor{currentfill}{rgb}{0.243520,0.319189,0.771672}%
\pgfsetfillcolor{currentfill}%
\pgfsetlinewidth{0.000000pt}%
\definecolor{currentstroke}{rgb}{0.000000,0.000000,0.000000}%
\pgfsetstrokecolor{currentstroke}%
\pgfsetdash{}{0pt}%
\pgfpathmoveto{\pgfqpoint{3.236099in}{0.723497in}}%
\pgfpathlineto{\pgfqpoint{3.502561in}{2.073396in}}%
\pgfpathlineto{\pgfqpoint{3.236099in}{0.723497in}}%
\pgfpathclose%
\pgfusepath{fill}%
\end{pgfscope}%
\begin{pgfscope}%
\pgfpathrectangle{\pgfqpoint{0.017500in}{0.017500in}}{\pgfqpoint{6.218282in}{3.580372in}} %
\pgfusepath{clip}%
\pgfsetbuttcap%
\pgfsetroundjoin%
\definecolor{currentfill}{rgb}{0.243520,0.319189,0.771672}%
\pgfsetfillcolor{currentfill}%
\pgfsetlinewidth{0.000000pt}%
\definecolor{currentstroke}{rgb}{0.000000,0.000000,0.000000}%
\pgfsetstrokecolor{currentstroke}%
\pgfsetdash{}{0pt}%
\pgfpathmoveto{\pgfqpoint{3.236099in}{0.723497in}}%
\pgfpathlineto{\pgfqpoint{3.502561in}{2.073396in}}%
\pgfpathlineto{\pgfqpoint{3.236099in}{0.723497in}}%
\pgfpathclose%
\pgfusepath{fill}%
\end{pgfscope}%
\begin{pgfscope}%
\pgfpathrectangle{\pgfqpoint{0.017500in}{0.017500in}}{\pgfqpoint{6.218282in}{3.580372in}} %
\pgfusepath{clip}%
\pgfsetbuttcap%
\pgfsetroundjoin%
\definecolor{currentfill}{rgb}{0.243520,0.319189,0.771672}%
\pgfsetfillcolor{currentfill}%
\pgfsetlinewidth{0.000000pt}%
\definecolor{currentstroke}{rgb}{0.000000,0.000000,0.000000}%
\pgfsetstrokecolor{currentstroke}%
\pgfsetdash{}{0pt}%
\pgfpathmoveto{\pgfqpoint{3.236099in}{0.723497in}}%
\pgfpathlineto{\pgfqpoint{3.502561in}{2.073396in}}%
\pgfpathlineto{\pgfqpoint{3.236099in}{0.723497in}}%
\pgfpathclose%
\pgfusepath{fill}%
\end{pgfscope}%
\begin{pgfscope}%
\pgfpathrectangle{\pgfqpoint{0.017500in}{0.017500in}}{\pgfqpoint{6.218282in}{3.580372in}} %
\pgfusepath{clip}%
\pgfsetbuttcap%
\pgfsetroundjoin%
\definecolor{currentfill}{rgb}{0.243520,0.319189,0.771672}%
\pgfsetfillcolor{currentfill}%
\pgfsetlinewidth{0.000000pt}%
\definecolor{currentstroke}{rgb}{0.000000,0.000000,0.000000}%
\pgfsetstrokecolor{currentstroke}%
\pgfsetdash{}{0pt}%
\pgfpathmoveto{\pgfqpoint{3.236099in}{0.723497in}}%
\pgfpathlineto{\pgfqpoint{3.502561in}{2.073396in}}%
\pgfpathlineto{\pgfqpoint{3.236099in}{0.723497in}}%
\pgfpathclose%
\pgfusepath{fill}%
\end{pgfscope}%
\begin{pgfscope}%
\pgfpathrectangle{\pgfqpoint{0.017500in}{0.017500in}}{\pgfqpoint{6.218282in}{3.580372in}} %
\pgfusepath{clip}%
\pgfsetbuttcap%
\pgfsetroundjoin%
\definecolor{currentfill}{rgb}{0.243520,0.319189,0.771672}%
\pgfsetfillcolor{currentfill}%
\pgfsetlinewidth{0.000000pt}%
\definecolor{currentstroke}{rgb}{0.000000,0.000000,0.000000}%
\pgfsetstrokecolor{currentstroke}%
\pgfsetdash{}{0pt}%
\pgfpathmoveto{\pgfqpoint{3.236099in}{0.723497in}}%
\pgfpathlineto{\pgfqpoint{3.502561in}{2.073396in}}%
\pgfpathlineto{\pgfqpoint{3.236099in}{0.723497in}}%
\pgfpathclose%
\pgfusepath{fill}%
\end{pgfscope}%
\begin{pgfscope}%
\pgfpathrectangle{\pgfqpoint{0.017500in}{0.017500in}}{\pgfqpoint{6.218282in}{3.580372in}} %
\pgfusepath{clip}%
\pgfsetbuttcap%
\pgfsetroundjoin%
\definecolor{currentfill}{rgb}{0.243520,0.319189,0.771672}%
\pgfsetfillcolor{currentfill}%
\pgfsetlinewidth{0.000000pt}%
\definecolor{currentstroke}{rgb}{0.000000,0.000000,0.000000}%
\pgfsetstrokecolor{currentstroke}%
\pgfsetdash{}{0pt}%
\pgfpathmoveto{\pgfqpoint{3.236099in}{0.723497in}}%
\pgfpathlineto{\pgfqpoint{3.502561in}{2.073396in}}%
\pgfpathlineto{\pgfqpoint{3.236099in}{0.723497in}}%
\pgfpathclose%
\pgfusepath{fill}%
\end{pgfscope}%
\begin{pgfscope}%
\pgfpathrectangle{\pgfqpoint{0.017500in}{0.017500in}}{\pgfqpoint{6.218282in}{3.580372in}} %
\pgfusepath{clip}%
\pgfsetbuttcap%
\pgfsetroundjoin%
\definecolor{currentfill}{rgb}{0.243520,0.319189,0.771672}%
\pgfsetfillcolor{currentfill}%
\pgfsetlinewidth{0.000000pt}%
\definecolor{currentstroke}{rgb}{0.000000,0.000000,0.000000}%
\pgfsetstrokecolor{currentstroke}%
\pgfsetdash{}{0pt}%
\pgfpathmoveto{\pgfqpoint{3.236099in}{0.723497in}}%
\pgfpathlineto{\pgfqpoint{3.502561in}{2.073396in}}%
\pgfpathlineto{\pgfqpoint{3.236099in}{0.723497in}}%
\pgfpathclose%
\pgfusepath{fill}%
\end{pgfscope}%
\begin{pgfscope}%
\pgfpathrectangle{\pgfqpoint{0.017500in}{0.017500in}}{\pgfqpoint{6.218282in}{3.580372in}} %
\pgfusepath{clip}%
\pgfsetbuttcap%
\pgfsetroundjoin%
\definecolor{currentfill}{rgb}{0.243520,0.319189,0.771672}%
\pgfsetfillcolor{currentfill}%
\pgfsetlinewidth{0.000000pt}%
\definecolor{currentstroke}{rgb}{0.000000,0.000000,0.000000}%
\pgfsetstrokecolor{currentstroke}%
\pgfsetdash{}{0pt}%
\pgfpathmoveto{\pgfqpoint{3.236099in}{0.723497in}}%
\pgfpathlineto{\pgfqpoint{3.502561in}{2.073396in}}%
\pgfpathlineto{\pgfqpoint{3.236099in}{0.723497in}}%
\pgfpathclose%
\pgfusepath{fill}%
\end{pgfscope}%
\begin{pgfscope}%
\pgfpathrectangle{\pgfqpoint{0.017500in}{0.017500in}}{\pgfqpoint{6.218282in}{3.580372in}} %
\pgfusepath{clip}%
\pgfsetbuttcap%
\pgfsetroundjoin%
\definecolor{currentfill}{rgb}{0.243520,0.319189,0.771672}%
\pgfsetfillcolor{currentfill}%
\pgfsetlinewidth{0.000000pt}%
\definecolor{currentstroke}{rgb}{0.000000,0.000000,0.000000}%
\pgfsetstrokecolor{currentstroke}%
\pgfsetdash{}{0pt}%
\pgfpathmoveto{\pgfqpoint{3.236099in}{0.723497in}}%
\pgfpathlineto{\pgfqpoint{3.502561in}{2.073396in}}%
\pgfpathlineto{\pgfqpoint{3.236099in}{0.723497in}}%
\pgfpathclose%
\pgfusepath{fill}%
\end{pgfscope}%
\begin{pgfscope}%
\pgfpathrectangle{\pgfqpoint{0.017500in}{0.017500in}}{\pgfqpoint{6.218282in}{3.580372in}} %
\pgfusepath{clip}%
\pgfsetbuttcap%
\pgfsetroundjoin%
\definecolor{currentfill}{rgb}{0.243520,0.319189,0.771672}%
\pgfsetfillcolor{currentfill}%
\pgfsetlinewidth{0.000000pt}%
\definecolor{currentstroke}{rgb}{0.000000,0.000000,0.000000}%
\pgfsetstrokecolor{currentstroke}%
\pgfsetdash{}{0pt}%
\pgfpathmoveto{\pgfqpoint{3.236099in}{0.723497in}}%
\pgfpathlineto{\pgfqpoint{3.502561in}{2.073396in}}%
\pgfpathlineto{\pgfqpoint{3.236099in}{0.723497in}}%
\pgfpathclose%
\pgfusepath{fill}%
\end{pgfscope}%
\begin{pgfscope}%
\pgfpathrectangle{\pgfqpoint{0.017500in}{0.017500in}}{\pgfqpoint{6.218282in}{3.580372in}} %
\pgfusepath{clip}%
\pgfsetbuttcap%
\pgfsetroundjoin%
\definecolor{currentfill}{rgb}{0.243520,0.319189,0.771672}%
\pgfsetfillcolor{currentfill}%
\pgfsetlinewidth{0.000000pt}%
\definecolor{currentstroke}{rgb}{0.000000,0.000000,0.000000}%
\pgfsetstrokecolor{currentstroke}%
\pgfsetdash{}{0pt}%
\pgfpathmoveto{\pgfqpoint{3.236099in}{0.723497in}}%
\pgfpathlineto{\pgfqpoint{3.502561in}{2.073396in}}%
\pgfpathlineto{\pgfqpoint{3.236099in}{0.723497in}}%
\pgfpathclose%
\pgfusepath{fill}%
\end{pgfscope}%
\begin{pgfscope}%
\pgfpathrectangle{\pgfqpoint{0.017500in}{0.017500in}}{\pgfqpoint{6.218282in}{3.580372in}} %
\pgfusepath{clip}%
\pgfsetbuttcap%
\pgfsetroundjoin%
\definecolor{currentfill}{rgb}{0.243520,0.319189,0.771672}%
\pgfsetfillcolor{currentfill}%
\pgfsetlinewidth{0.000000pt}%
\definecolor{currentstroke}{rgb}{0.000000,0.000000,0.000000}%
\pgfsetstrokecolor{currentstroke}%
\pgfsetdash{}{0pt}%
\pgfpathmoveto{\pgfqpoint{3.236099in}{0.723497in}}%
\pgfpathlineto{\pgfqpoint{3.502561in}{2.073396in}}%
\pgfpathlineto{\pgfqpoint{3.236099in}{0.723497in}}%
\pgfpathclose%
\pgfusepath{fill}%
\end{pgfscope}%
\begin{pgfscope}%
\pgfpathrectangle{\pgfqpoint{0.017500in}{0.017500in}}{\pgfqpoint{6.218282in}{3.580372in}} %
\pgfusepath{clip}%
\pgfsetbuttcap%
\pgfsetroundjoin%
\definecolor{currentfill}{rgb}{0.243520,0.319189,0.771672}%
\pgfsetfillcolor{currentfill}%
\pgfsetlinewidth{0.000000pt}%
\definecolor{currentstroke}{rgb}{0.000000,0.000000,0.000000}%
\pgfsetstrokecolor{currentstroke}%
\pgfsetdash{}{0pt}%
\pgfpathmoveto{\pgfqpoint{3.236099in}{0.723497in}}%
\pgfpathlineto{\pgfqpoint{3.502561in}{2.073396in}}%
\pgfpathlineto{\pgfqpoint{3.236099in}{0.723497in}}%
\pgfpathclose%
\pgfusepath{fill}%
\end{pgfscope}%
\begin{pgfscope}%
\pgfpathrectangle{\pgfqpoint{0.017500in}{0.017500in}}{\pgfqpoint{6.218282in}{3.580372in}} %
\pgfusepath{clip}%
\pgfsetbuttcap%
\pgfsetroundjoin%
\definecolor{currentfill}{rgb}{0.243520,0.319189,0.771672}%
\pgfsetfillcolor{currentfill}%
\pgfsetlinewidth{0.000000pt}%
\definecolor{currentstroke}{rgb}{0.000000,0.000000,0.000000}%
\pgfsetstrokecolor{currentstroke}%
\pgfsetdash{}{0pt}%
\pgfpathmoveto{\pgfqpoint{3.236099in}{0.723497in}}%
\pgfpathlineto{\pgfqpoint{3.502561in}{2.073396in}}%
\pgfpathlineto{\pgfqpoint{3.236099in}{0.723497in}}%
\pgfpathclose%
\pgfusepath{fill}%
\end{pgfscope}%
\begin{pgfscope}%
\pgfpathrectangle{\pgfqpoint{0.017500in}{0.017500in}}{\pgfqpoint{6.218282in}{3.580372in}} %
\pgfusepath{clip}%
\pgfsetbuttcap%
\pgfsetroundjoin%
\definecolor{currentfill}{rgb}{0.243520,0.319189,0.771672}%
\pgfsetfillcolor{currentfill}%
\pgfsetlinewidth{0.000000pt}%
\definecolor{currentstroke}{rgb}{0.000000,0.000000,0.000000}%
\pgfsetstrokecolor{currentstroke}%
\pgfsetdash{}{0pt}%
\pgfpathmoveto{\pgfqpoint{3.236099in}{0.723497in}}%
\pgfpathlineto{\pgfqpoint{3.502561in}{2.073396in}}%
\pgfpathlineto{\pgfqpoint{3.236099in}{0.723497in}}%
\pgfpathclose%
\pgfusepath{fill}%
\end{pgfscope}%
\begin{pgfscope}%
\pgfpathrectangle{\pgfqpoint{0.017500in}{0.017500in}}{\pgfqpoint{6.218282in}{3.580372in}} %
\pgfusepath{clip}%
\pgfsetbuttcap%
\pgfsetroundjoin%
\definecolor{currentfill}{rgb}{0.243520,0.319189,0.771672}%
\pgfsetfillcolor{currentfill}%
\pgfsetlinewidth{0.000000pt}%
\definecolor{currentstroke}{rgb}{0.000000,0.000000,0.000000}%
\pgfsetstrokecolor{currentstroke}%
\pgfsetdash{}{0pt}%
\pgfpathmoveto{\pgfqpoint{3.236099in}{0.723497in}}%
\pgfpathlineto{\pgfqpoint{3.502561in}{2.073396in}}%
\pgfpathlineto{\pgfqpoint{3.236099in}{0.723497in}}%
\pgfpathclose%
\pgfusepath{fill}%
\end{pgfscope}%
\begin{pgfscope}%
\pgfpathrectangle{\pgfqpoint{0.017500in}{0.017500in}}{\pgfqpoint{6.218282in}{3.580372in}} %
\pgfusepath{clip}%
\pgfsetbuttcap%
\pgfsetroundjoin%
\definecolor{currentfill}{rgb}{0.243520,0.319189,0.771672}%
\pgfsetfillcolor{currentfill}%
\pgfsetlinewidth{0.000000pt}%
\definecolor{currentstroke}{rgb}{0.000000,0.000000,0.000000}%
\pgfsetstrokecolor{currentstroke}%
\pgfsetdash{}{0pt}%
\pgfpathmoveto{\pgfqpoint{3.236099in}{0.723497in}}%
\pgfpathlineto{\pgfqpoint{3.502561in}{2.073396in}}%
\pgfpathlineto{\pgfqpoint{3.236099in}{0.723497in}}%
\pgfpathclose%
\pgfusepath{fill}%
\end{pgfscope}%
\begin{pgfscope}%
\pgfpathrectangle{\pgfqpoint{0.017500in}{0.017500in}}{\pgfqpoint{6.218282in}{3.580372in}} %
\pgfusepath{clip}%
\pgfsetbuttcap%
\pgfsetroundjoin%
\definecolor{currentfill}{rgb}{0.243520,0.319189,0.771672}%
\pgfsetfillcolor{currentfill}%
\pgfsetlinewidth{0.000000pt}%
\definecolor{currentstroke}{rgb}{0.000000,0.000000,0.000000}%
\pgfsetstrokecolor{currentstroke}%
\pgfsetdash{}{0pt}%
\pgfpathmoveto{\pgfqpoint{3.236099in}{0.723497in}}%
\pgfpathlineto{\pgfqpoint{3.502561in}{2.073396in}}%
\pgfpathlineto{\pgfqpoint{3.236099in}{0.723497in}}%
\pgfpathclose%
\pgfusepath{fill}%
\end{pgfscope}%
\begin{pgfscope}%
\pgfpathrectangle{\pgfqpoint{0.017500in}{0.017500in}}{\pgfqpoint{6.218282in}{3.580372in}} %
\pgfusepath{clip}%
\pgfsetbuttcap%
\pgfsetroundjoin%
\definecolor{currentfill}{rgb}{0.243520,0.319189,0.771672}%
\pgfsetfillcolor{currentfill}%
\pgfsetlinewidth{0.000000pt}%
\definecolor{currentstroke}{rgb}{0.000000,0.000000,0.000000}%
\pgfsetstrokecolor{currentstroke}%
\pgfsetdash{}{0pt}%
\pgfpathmoveto{\pgfqpoint{3.236099in}{0.723497in}}%
\pgfpathlineto{\pgfqpoint{3.502561in}{2.073396in}}%
\pgfpathlineto{\pgfqpoint{3.236099in}{0.723497in}}%
\pgfpathclose%
\pgfusepath{fill}%
\end{pgfscope}%
\begin{pgfscope}%
\pgfpathrectangle{\pgfqpoint{0.017500in}{0.017500in}}{\pgfqpoint{6.218282in}{3.580372in}} %
\pgfusepath{clip}%
\pgfsetbuttcap%
\pgfsetroundjoin%
\definecolor{currentfill}{rgb}{0.243520,0.319189,0.771672}%
\pgfsetfillcolor{currentfill}%
\pgfsetlinewidth{0.000000pt}%
\definecolor{currentstroke}{rgb}{0.000000,0.000000,0.000000}%
\pgfsetstrokecolor{currentstroke}%
\pgfsetdash{}{0pt}%
\pgfpathmoveto{\pgfqpoint{3.236099in}{0.723497in}}%
\pgfpathlineto{\pgfqpoint{3.502561in}{2.073396in}}%
\pgfpathlineto{\pgfqpoint{3.236099in}{0.723497in}}%
\pgfpathclose%
\pgfusepath{fill}%
\end{pgfscope}%
\begin{pgfscope}%
\pgfpathrectangle{\pgfqpoint{0.017500in}{0.017500in}}{\pgfqpoint{6.218282in}{3.580372in}} %
\pgfusepath{clip}%
\pgfsetbuttcap%
\pgfsetroundjoin%
\definecolor{currentfill}{rgb}{0.243520,0.319189,0.771672}%
\pgfsetfillcolor{currentfill}%
\pgfsetlinewidth{0.000000pt}%
\definecolor{currentstroke}{rgb}{0.000000,0.000000,0.000000}%
\pgfsetstrokecolor{currentstroke}%
\pgfsetdash{}{0pt}%
\pgfpathmoveto{\pgfqpoint{3.236099in}{0.723497in}}%
\pgfpathlineto{\pgfqpoint{3.502561in}{2.073396in}}%
\pgfpathlineto{\pgfqpoint{3.236099in}{0.723497in}}%
\pgfpathclose%
\pgfusepath{fill}%
\end{pgfscope}%
\begin{pgfscope}%
\pgfpathrectangle{\pgfqpoint{0.017500in}{0.017500in}}{\pgfqpoint{6.218282in}{3.580372in}} %
\pgfusepath{clip}%
\pgfsetbuttcap%
\pgfsetroundjoin%
\definecolor{currentfill}{rgb}{0.243520,0.319189,0.771672}%
\pgfsetfillcolor{currentfill}%
\pgfsetlinewidth{0.000000pt}%
\definecolor{currentstroke}{rgb}{0.000000,0.000000,0.000000}%
\pgfsetstrokecolor{currentstroke}%
\pgfsetdash{}{0pt}%
\pgfpathmoveto{\pgfqpoint{3.236099in}{0.723497in}}%
\pgfpathlineto{\pgfqpoint{3.502561in}{2.073396in}}%
\pgfpathlineto{\pgfqpoint{3.236099in}{0.723497in}}%
\pgfpathclose%
\pgfusepath{fill}%
\end{pgfscope}%
\begin{pgfscope}%
\pgfpathrectangle{\pgfqpoint{0.017500in}{0.017500in}}{\pgfqpoint{6.218282in}{3.580372in}} %
\pgfusepath{clip}%
\pgfsetbuttcap%
\pgfsetroundjoin%
\definecolor{currentfill}{rgb}{0.243520,0.319189,0.771672}%
\pgfsetfillcolor{currentfill}%
\pgfsetlinewidth{0.000000pt}%
\definecolor{currentstroke}{rgb}{0.000000,0.000000,0.000000}%
\pgfsetstrokecolor{currentstroke}%
\pgfsetdash{}{0pt}%
\pgfpathmoveto{\pgfqpoint{3.236099in}{0.723497in}}%
\pgfpathlineto{\pgfqpoint{3.502561in}{2.073396in}}%
\pgfpathlineto{\pgfqpoint{3.236099in}{0.723497in}}%
\pgfpathclose%
\pgfusepath{fill}%
\end{pgfscope}%
\begin{pgfscope}%
\pgfpathrectangle{\pgfqpoint{0.017500in}{0.017500in}}{\pgfqpoint{6.218282in}{3.580372in}} %
\pgfusepath{clip}%
\pgfsetbuttcap%
\pgfsetroundjoin%
\definecolor{currentfill}{rgb}{0.243520,0.319189,0.771672}%
\pgfsetfillcolor{currentfill}%
\pgfsetlinewidth{0.000000pt}%
\definecolor{currentstroke}{rgb}{0.000000,0.000000,0.000000}%
\pgfsetstrokecolor{currentstroke}%
\pgfsetdash{}{0pt}%
\pgfpathmoveto{\pgfqpoint{3.236099in}{0.723497in}}%
\pgfpathlineto{\pgfqpoint{3.502561in}{2.073396in}}%
\pgfpathlineto{\pgfqpoint{3.236099in}{0.723497in}}%
\pgfpathclose%
\pgfusepath{fill}%
\end{pgfscope}%
\begin{pgfscope}%
\pgfpathrectangle{\pgfqpoint{0.017500in}{0.017500in}}{\pgfqpoint{6.218282in}{3.580372in}} %
\pgfusepath{clip}%
\pgfsetbuttcap%
\pgfsetroundjoin%
\definecolor{currentfill}{rgb}{0.243520,0.319189,0.771672}%
\pgfsetfillcolor{currentfill}%
\pgfsetlinewidth{0.000000pt}%
\definecolor{currentstroke}{rgb}{0.000000,0.000000,0.000000}%
\pgfsetstrokecolor{currentstroke}%
\pgfsetdash{}{0pt}%
\pgfpathmoveto{\pgfqpoint{3.236099in}{0.723497in}}%
\pgfpathlineto{\pgfqpoint{3.502561in}{2.073396in}}%
\pgfpathlineto{\pgfqpoint{3.236099in}{0.723497in}}%
\pgfpathclose%
\pgfusepath{fill}%
\end{pgfscope}%
\begin{pgfscope}%
\pgfpathrectangle{\pgfqpoint{0.017500in}{0.017500in}}{\pgfqpoint{6.218282in}{3.580372in}} %
\pgfusepath{clip}%
\pgfsetbuttcap%
\pgfsetroundjoin%
\definecolor{currentfill}{rgb}{0.243520,0.319189,0.771672}%
\pgfsetfillcolor{currentfill}%
\pgfsetlinewidth{0.000000pt}%
\definecolor{currentstroke}{rgb}{0.000000,0.000000,0.000000}%
\pgfsetstrokecolor{currentstroke}%
\pgfsetdash{}{0pt}%
\pgfpathmoveto{\pgfqpoint{3.236099in}{0.723497in}}%
\pgfpathlineto{\pgfqpoint{3.502561in}{2.073396in}}%
\pgfpathlineto{\pgfqpoint{3.236099in}{0.723497in}}%
\pgfpathclose%
\pgfusepath{fill}%
\end{pgfscope}%
\begin{pgfscope}%
\pgfpathrectangle{\pgfqpoint{0.017500in}{0.017500in}}{\pgfqpoint{6.218282in}{3.580372in}} %
\pgfusepath{clip}%
\pgfsetbuttcap%
\pgfsetroundjoin%
\definecolor{currentfill}{rgb}{0.243520,0.319189,0.771672}%
\pgfsetfillcolor{currentfill}%
\pgfsetlinewidth{0.000000pt}%
\definecolor{currentstroke}{rgb}{0.000000,0.000000,0.000000}%
\pgfsetstrokecolor{currentstroke}%
\pgfsetdash{}{0pt}%
\pgfpathmoveto{\pgfqpoint{3.236099in}{0.723497in}}%
\pgfpathlineto{\pgfqpoint{3.502561in}{2.073396in}}%
\pgfpathlineto{\pgfqpoint{3.236099in}{0.723497in}}%
\pgfpathclose%
\pgfusepath{fill}%
\end{pgfscope}%
\begin{pgfscope}%
\pgfpathrectangle{\pgfqpoint{0.017500in}{0.017500in}}{\pgfqpoint{6.218282in}{3.580372in}} %
\pgfusepath{clip}%
\pgfsetbuttcap%
\pgfsetroundjoin%
\definecolor{currentfill}{rgb}{0.763520,0.178667,0.193396}%
\pgfsetfillcolor{currentfill}%
\pgfsetlinewidth{0.000000pt}%
\definecolor{currentstroke}{rgb}{0.000000,0.000000,0.000000}%
\pgfsetstrokecolor{currentstroke}%
\pgfsetdash{}{0pt}%
\pgfpathmoveto{\pgfqpoint{4.478468in}{2.393972in}}%
\pgfpathlineto{\pgfqpoint{4.724651in}{2.549289in}}%
\pgfpathlineto{\pgfqpoint{4.478468in}{2.393972in}}%
\pgfpathclose%
\pgfusepath{fill}%
\end{pgfscope}%
\begin{pgfscope}%
\pgfpathrectangle{\pgfqpoint{0.017500in}{0.017500in}}{\pgfqpoint{6.218282in}{3.580372in}} %
\pgfusepath{clip}%
\pgfsetbuttcap%
\pgfsetroundjoin%
\definecolor{currentfill}{rgb}{0.763520,0.178667,0.193396}%
\pgfsetfillcolor{currentfill}%
\pgfsetlinewidth{0.000000pt}%
\definecolor{currentstroke}{rgb}{0.000000,0.000000,0.000000}%
\pgfsetstrokecolor{currentstroke}%
\pgfsetdash{}{0pt}%
\pgfpathmoveto{\pgfqpoint{4.478468in}{2.393972in}}%
\pgfpathlineto{\pgfqpoint{4.724651in}{2.549289in}}%
\pgfpathlineto{\pgfqpoint{4.478468in}{2.393972in}}%
\pgfpathclose%
\pgfusepath{fill}%
\end{pgfscope}%
\begin{pgfscope}%
\pgfpathrectangle{\pgfqpoint{0.017500in}{0.017500in}}{\pgfqpoint{6.218282in}{3.580372in}} %
\pgfusepath{clip}%
\pgfsetbuttcap%
\pgfsetroundjoin%
\definecolor{currentfill}{rgb}{0.763520,0.178667,0.193396}%
\pgfsetfillcolor{currentfill}%
\pgfsetlinewidth{0.000000pt}%
\definecolor{currentstroke}{rgb}{0.000000,0.000000,0.000000}%
\pgfsetstrokecolor{currentstroke}%
\pgfsetdash{}{0pt}%
\pgfpathmoveto{\pgfqpoint{4.478468in}{2.393972in}}%
\pgfpathlineto{\pgfqpoint{4.724651in}{2.549289in}}%
\pgfpathlineto{\pgfqpoint{4.478468in}{2.393972in}}%
\pgfpathclose%
\pgfusepath{fill}%
\end{pgfscope}%
\begin{pgfscope}%
\pgfpathrectangle{\pgfqpoint{0.017500in}{0.017500in}}{\pgfqpoint{6.218282in}{3.580372in}} %
\pgfusepath{clip}%
\pgfsetbuttcap%
\pgfsetroundjoin%
\definecolor{currentfill}{rgb}{0.763520,0.178667,0.193396}%
\pgfsetfillcolor{currentfill}%
\pgfsetlinewidth{0.000000pt}%
\definecolor{currentstroke}{rgb}{0.000000,0.000000,0.000000}%
\pgfsetstrokecolor{currentstroke}%
\pgfsetdash{}{0pt}%
\pgfpathmoveto{\pgfqpoint{4.478468in}{2.393972in}}%
\pgfpathlineto{\pgfqpoint{4.724651in}{2.549289in}}%
\pgfpathlineto{\pgfqpoint{4.478468in}{2.393972in}}%
\pgfpathclose%
\pgfusepath{fill}%
\end{pgfscope}%
\begin{pgfscope}%
\pgfpathrectangle{\pgfqpoint{0.017500in}{0.017500in}}{\pgfqpoint{6.218282in}{3.580372in}} %
\pgfusepath{clip}%
\pgfsetbuttcap%
\pgfsetroundjoin%
\definecolor{currentfill}{rgb}{0.763520,0.178667,0.193396}%
\pgfsetfillcolor{currentfill}%
\pgfsetlinewidth{0.000000pt}%
\definecolor{currentstroke}{rgb}{0.000000,0.000000,0.000000}%
\pgfsetstrokecolor{currentstroke}%
\pgfsetdash{}{0pt}%
\pgfpathmoveto{\pgfqpoint{4.478468in}{2.393972in}}%
\pgfpathlineto{\pgfqpoint{4.724651in}{2.549289in}}%
\pgfpathlineto{\pgfqpoint{4.478468in}{2.393972in}}%
\pgfpathclose%
\pgfusepath{fill}%
\end{pgfscope}%
\begin{pgfscope}%
\pgfpathrectangle{\pgfqpoint{0.017500in}{0.017500in}}{\pgfqpoint{6.218282in}{3.580372in}} %
\pgfusepath{clip}%
\pgfsetbuttcap%
\pgfsetroundjoin%
\definecolor{currentfill}{rgb}{0.763520,0.178667,0.193396}%
\pgfsetfillcolor{currentfill}%
\pgfsetlinewidth{0.000000pt}%
\definecolor{currentstroke}{rgb}{0.000000,0.000000,0.000000}%
\pgfsetstrokecolor{currentstroke}%
\pgfsetdash{}{0pt}%
\pgfpathmoveto{\pgfqpoint{4.478468in}{2.393972in}}%
\pgfpathlineto{\pgfqpoint{4.724651in}{2.549289in}}%
\pgfpathlineto{\pgfqpoint{4.478468in}{2.393972in}}%
\pgfpathclose%
\pgfusepath{fill}%
\end{pgfscope}%
\begin{pgfscope}%
\pgfpathrectangle{\pgfqpoint{0.017500in}{0.017500in}}{\pgfqpoint{6.218282in}{3.580372in}} %
\pgfusepath{clip}%
\pgfsetbuttcap%
\pgfsetroundjoin%
\definecolor{currentfill}{rgb}{0.763520,0.178667,0.193396}%
\pgfsetfillcolor{currentfill}%
\pgfsetlinewidth{0.000000pt}%
\definecolor{currentstroke}{rgb}{0.000000,0.000000,0.000000}%
\pgfsetstrokecolor{currentstroke}%
\pgfsetdash{}{0pt}%
\pgfpathmoveto{\pgfqpoint{4.478468in}{2.393972in}}%
\pgfpathlineto{\pgfqpoint{4.724651in}{2.549289in}}%
\pgfpathlineto{\pgfqpoint{4.478468in}{2.393972in}}%
\pgfpathclose%
\pgfusepath{fill}%
\end{pgfscope}%
\begin{pgfscope}%
\pgfpathrectangle{\pgfqpoint{0.017500in}{0.017500in}}{\pgfqpoint{6.218282in}{3.580372in}} %
\pgfusepath{clip}%
\pgfsetbuttcap%
\pgfsetroundjoin%
\definecolor{currentfill}{rgb}{0.763520,0.178667,0.193396}%
\pgfsetfillcolor{currentfill}%
\pgfsetlinewidth{0.000000pt}%
\definecolor{currentstroke}{rgb}{0.000000,0.000000,0.000000}%
\pgfsetstrokecolor{currentstroke}%
\pgfsetdash{}{0pt}%
\pgfpathmoveto{\pgfqpoint{4.478468in}{2.393972in}}%
\pgfpathlineto{\pgfqpoint{4.724651in}{2.549289in}}%
\pgfpathlineto{\pgfqpoint{4.478468in}{2.393972in}}%
\pgfpathclose%
\pgfusepath{fill}%
\end{pgfscope}%
\begin{pgfscope}%
\pgfpathrectangle{\pgfqpoint{0.017500in}{0.017500in}}{\pgfqpoint{6.218282in}{3.580372in}} %
\pgfusepath{clip}%
\pgfsetbuttcap%
\pgfsetroundjoin%
\definecolor{currentfill}{rgb}{0.763520,0.178667,0.193396}%
\pgfsetfillcolor{currentfill}%
\pgfsetlinewidth{0.000000pt}%
\definecolor{currentstroke}{rgb}{0.000000,0.000000,0.000000}%
\pgfsetstrokecolor{currentstroke}%
\pgfsetdash{}{0pt}%
\pgfpathmoveto{\pgfqpoint{4.478468in}{2.393972in}}%
\pgfpathlineto{\pgfqpoint{4.724651in}{2.549289in}}%
\pgfpathlineto{\pgfqpoint{4.478468in}{2.393972in}}%
\pgfpathclose%
\pgfusepath{fill}%
\end{pgfscope}%
\begin{pgfscope}%
\pgfpathrectangle{\pgfqpoint{0.017500in}{0.017500in}}{\pgfqpoint{6.218282in}{3.580372in}} %
\pgfusepath{clip}%
\pgfsetbuttcap%
\pgfsetroundjoin%
\definecolor{currentfill}{rgb}{0.763520,0.178667,0.193396}%
\pgfsetfillcolor{currentfill}%
\pgfsetlinewidth{0.000000pt}%
\definecolor{currentstroke}{rgb}{0.000000,0.000000,0.000000}%
\pgfsetstrokecolor{currentstroke}%
\pgfsetdash{}{0pt}%
\pgfpathmoveto{\pgfqpoint{4.478468in}{2.393972in}}%
\pgfpathlineto{\pgfqpoint{4.724651in}{2.549289in}}%
\pgfpathlineto{\pgfqpoint{4.478468in}{2.393972in}}%
\pgfpathclose%
\pgfusepath{fill}%
\end{pgfscope}%
\begin{pgfscope}%
\pgfpathrectangle{\pgfqpoint{0.017500in}{0.017500in}}{\pgfqpoint{6.218282in}{3.580372in}} %
\pgfusepath{clip}%
\pgfsetbuttcap%
\pgfsetroundjoin%
\definecolor{currentfill}{rgb}{0.763520,0.178667,0.193396}%
\pgfsetfillcolor{currentfill}%
\pgfsetlinewidth{0.000000pt}%
\definecolor{currentstroke}{rgb}{0.000000,0.000000,0.000000}%
\pgfsetstrokecolor{currentstroke}%
\pgfsetdash{}{0pt}%
\pgfpathmoveto{\pgfqpoint{4.478468in}{2.393972in}}%
\pgfpathlineto{\pgfqpoint{4.724651in}{2.549289in}}%
\pgfpathlineto{\pgfqpoint{4.478468in}{2.393972in}}%
\pgfpathclose%
\pgfusepath{fill}%
\end{pgfscope}%
\begin{pgfscope}%
\pgfpathrectangle{\pgfqpoint{0.017500in}{0.017500in}}{\pgfqpoint{6.218282in}{3.580372in}} %
\pgfusepath{clip}%
\pgfsetbuttcap%
\pgfsetroundjoin%
\definecolor{currentfill}{rgb}{0.763520,0.178667,0.193396}%
\pgfsetfillcolor{currentfill}%
\pgfsetlinewidth{0.000000pt}%
\definecolor{currentstroke}{rgb}{0.000000,0.000000,0.000000}%
\pgfsetstrokecolor{currentstroke}%
\pgfsetdash{}{0pt}%
\pgfpathmoveto{\pgfqpoint{4.478468in}{2.393972in}}%
\pgfpathlineto{\pgfqpoint{4.724651in}{2.549289in}}%
\pgfpathlineto{\pgfqpoint{4.478468in}{2.393972in}}%
\pgfpathclose%
\pgfusepath{fill}%
\end{pgfscope}%
\begin{pgfscope}%
\pgfpathrectangle{\pgfqpoint{0.017500in}{0.017500in}}{\pgfqpoint{6.218282in}{3.580372in}} %
\pgfusepath{clip}%
\pgfsetbuttcap%
\pgfsetroundjoin%
\definecolor{currentfill}{rgb}{0.763520,0.178667,0.193396}%
\pgfsetfillcolor{currentfill}%
\pgfsetlinewidth{0.000000pt}%
\definecolor{currentstroke}{rgb}{0.000000,0.000000,0.000000}%
\pgfsetstrokecolor{currentstroke}%
\pgfsetdash{}{0pt}%
\pgfpathmoveto{\pgfqpoint{4.478468in}{2.393972in}}%
\pgfpathlineto{\pgfqpoint{4.724651in}{2.549289in}}%
\pgfpathlineto{\pgfqpoint{4.478468in}{2.393972in}}%
\pgfpathclose%
\pgfusepath{fill}%
\end{pgfscope}%
\begin{pgfscope}%
\pgfpathrectangle{\pgfqpoint{0.017500in}{0.017500in}}{\pgfqpoint{6.218282in}{3.580372in}} %
\pgfusepath{clip}%
\pgfsetbuttcap%
\pgfsetroundjoin%
\definecolor{currentfill}{rgb}{0.763520,0.178667,0.193396}%
\pgfsetfillcolor{currentfill}%
\pgfsetlinewidth{0.000000pt}%
\definecolor{currentstroke}{rgb}{0.000000,0.000000,0.000000}%
\pgfsetstrokecolor{currentstroke}%
\pgfsetdash{}{0pt}%
\pgfpathmoveto{\pgfqpoint{4.478468in}{2.393972in}}%
\pgfpathlineto{\pgfqpoint{4.724651in}{2.549289in}}%
\pgfpathlineto{\pgfqpoint{4.478468in}{2.393972in}}%
\pgfpathclose%
\pgfusepath{fill}%
\end{pgfscope}%
\begin{pgfscope}%
\pgfpathrectangle{\pgfqpoint{0.017500in}{0.017500in}}{\pgfqpoint{6.218282in}{3.580372in}} %
\pgfusepath{clip}%
\pgfsetbuttcap%
\pgfsetroundjoin%
\definecolor{currentfill}{rgb}{0.763520,0.178667,0.193396}%
\pgfsetfillcolor{currentfill}%
\pgfsetlinewidth{0.000000pt}%
\definecolor{currentstroke}{rgb}{0.000000,0.000000,0.000000}%
\pgfsetstrokecolor{currentstroke}%
\pgfsetdash{}{0pt}%
\pgfpathmoveto{\pgfqpoint{4.478468in}{2.393972in}}%
\pgfpathlineto{\pgfqpoint{4.724651in}{2.549289in}}%
\pgfpathlineto{\pgfqpoint{4.478468in}{2.393972in}}%
\pgfpathclose%
\pgfusepath{fill}%
\end{pgfscope}%
\begin{pgfscope}%
\pgfpathrectangle{\pgfqpoint{0.017500in}{0.017500in}}{\pgfqpoint{6.218282in}{3.580372in}} %
\pgfusepath{clip}%
\pgfsetbuttcap%
\pgfsetroundjoin%
\definecolor{currentfill}{rgb}{0.763520,0.178667,0.193396}%
\pgfsetfillcolor{currentfill}%
\pgfsetlinewidth{0.000000pt}%
\definecolor{currentstroke}{rgb}{0.000000,0.000000,0.000000}%
\pgfsetstrokecolor{currentstroke}%
\pgfsetdash{}{0pt}%
\pgfpathmoveto{\pgfqpoint{4.478468in}{2.393972in}}%
\pgfpathlineto{\pgfqpoint{4.724651in}{2.549289in}}%
\pgfpathlineto{\pgfqpoint{4.478468in}{2.393972in}}%
\pgfpathclose%
\pgfusepath{fill}%
\end{pgfscope}%
\begin{pgfscope}%
\pgfpathrectangle{\pgfqpoint{0.017500in}{0.017500in}}{\pgfqpoint{6.218282in}{3.580372in}} %
\pgfusepath{clip}%
\pgfsetbuttcap%
\pgfsetroundjoin%
\definecolor{currentfill}{rgb}{0.763520,0.178667,0.193396}%
\pgfsetfillcolor{currentfill}%
\pgfsetlinewidth{0.000000pt}%
\definecolor{currentstroke}{rgb}{0.000000,0.000000,0.000000}%
\pgfsetstrokecolor{currentstroke}%
\pgfsetdash{}{0pt}%
\pgfpathmoveto{\pgfqpoint{4.478468in}{2.393972in}}%
\pgfpathlineto{\pgfqpoint{4.724651in}{2.549289in}}%
\pgfpathlineto{\pgfqpoint{4.478468in}{2.393972in}}%
\pgfpathclose%
\pgfusepath{fill}%
\end{pgfscope}%
\begin{pgfscope}%
\pgfpathrectangle{\pgfqpoint{0.017500in}{0.017500in}}{\pgfqpoint{6.218282in}{3.580372in}} %
\pgfusepath{clip}%
\pgfsetbuttcap%
\pgfsetroundjoin%
\definecolor{currentfill}{rgb}{0.763520,0.178667,0.193396}%
\pgfsetfillcolor{currentfill}%
\pgfsetlinewidth{0.000000pt}%
\definecolor{currentstroke}{rgb}{0.000000,0.000000,0.000000}%
\pgfsetstrokecolor{currentstroke}%
\pgfsetdash{}{0pt}%
\pgfpathmoveto{\pgfqpoint{4.478468in}{2.393972in}}%
\pgfpathlineto{\pgfqpoint{4.724651in}{2.549289in}}%
\pgfpathlineto{\pgfqpoint{4.478468in}{2.393972in}}%
\pgfpathclose%
\pgfusepath{fill}%
\end{pgfscope}%
\begin{pgfscope}%
\pgfpathrectangle{\pgfqpoint{0.017500in}{0.017500in}}{\pgfqpoint{6.218282in}{3.580372in}} %
\pgfusepath{clip}%
\pgfsetbuttcap%
\pgfsetroundjoin%
\definecolor{currentfill}{rgb}{0.763520,0.178667,0.193396}%
\pgfsetfillcolor{currentfill}%
\pgfsetlinewidth{0.000000pt}%
\definecolor{currentstroke}{rgb}{0.000000,0.000000,0.000000}%
\pgfsetstrokecolor{currentstroke}%
\pgfsetdash{}{0pt}%
\pgfpathmoveto{\pgfqpoint{4.478468in}{2.393972in}}%
\pgfpathlineto{\pgfqpoint{4.724651in}{2.549289in}}%
\pgfpathlineto{\pgfqpoint{4.478468in}{2.393972in}}%
\pgfpathclose%
\pgfusepath{fill}%
\end{pgfscope}%
\begin{pgfscope}%
\pgfpathrectangle{\pgfqpoint{0.017500in}{0.017500in}}{\pgfqpoint{6.218282in}{3.580372in}} %
\pgfusepath{clip}%
\pgfsetbuttcap%
\pgfsetroundjoin%
\definecolor{currentfill}{rgb}{0.763520,0.178667,0.193396}%
\pgfsetfillcolor{currentfill}%
\pgfsetlinewidth{0.000000pt}%
\definecolor{currentstroke}{rgb}{0.000000,0.000000,0.000000}%
\pgfsetstrokecolor{currentstroke}%
\pgfsetdash{}{0pt}%
\pgfpathmoveto{\pgfqpoint{4.478468in}{2.393972in}}%
\pgfpathlineto{\pgfqpoint{4.724651in}{2.549289in}}%
\pgfpathlineto{\pgfqpoint{4.478468in}{2.393972in}}%
\pgfpathclose%
\pgfusepath{fill}%
\end{pgfscope}%
\begin{pgfscope}%
\pgfpathrectangle{\pgfqpoint{0.017500in}{0.017500in}}{\pgfqpoint{6.218282in}{3.580372in}} %
\pgfusepath{clip}%
\pgfsetbuttcap%
\pgfsetroundjoin%
\definecolor{currentfill}{rgb}{0.763520,0.178667,0.193396}%
\pgfsetfillcolor{currentfill}%
\pgfsetlinewidth{0.000000pt}%
\definecolor{currentstroke}{rgb}{0.000000,0.000000,0.000000}%
\pgfsetstrokecolor{currentstroke}%
\pgfsetdash{}{0pt}%
\pgfpathmoveto{\pgfqpoint{4.478468in}{2.393972in}}%
\pgfpathlineto{\pgfqpoint{4.724651in}{2.549289in}}%
\pgfpathlineto{\pgfqpoint{4.478468in}{2.393972in}}%
\pgfpathclose%
\pgfusepath{fill}%
\end{pgfscope}%
\begin{pgfscope}%
\pgfpathrectangle{\pgfqpoint{0.017500in}{0.017500in}}{\pgfqpoint{6.218282in}{3.580372in}} %
\pgfusepath{clip}%
\pgfsetbuttcap%
\pgfsetroundjoin%
\definecolor{currentfill}{rgb}{0.763520,0.178667,0.193396}%
\pgfsetfillcolor{currentfill}%
\pgfsetlinewidth{0.000000pt}%
\definecolor{currentstroke}{rgb}{0.000000,0.000000,0.000000}%
\pgfsetstrokecolor{currentstroke}%
\pgfsetdash{}{0pt}%
\pgfpathmoveto{\pgfqpoint{4.478468in}{2.393972in}}%
\pgfpathlineto{\pgfqpoint{4.724651in}{2.549289in}}%
\pgfpathlineto{\pgfqpoint{4.478468in}{2.393972in}}%
\pgfpathclose%
\pgfusepath{fill}%
\end{pgfscope}%
\begin{pgfscope}%
\pgfpathrectangle{\pgfqpoint{0.017500in}{0.017500in}}{\pgfqpoint{6.218282in}{3.580372in}} %
\pgfusepath{clip}%
\pgfsetbuttcap%
\pgfsetroundjoin%
\definecolor{currentfill}{rgb}{0.763520,0.178667,0.193396}%
\pgfsetfillcolor{currentfill}%
\pgfsetlinewidth{0.000000pt}%
\definecolor{currentstroke}{rgb}{0.000000,0.000000,0.000000}%
\pgfsetstrokecolor{currentstroke}%
\pgfsetdash{}{0pt}%
\pgfpathmoveto{\pgfqpoint{4.478468in}{2.393972in}}%
\pgfpathlineto{\pgfqpoint{4.724651in}{2.549289in}}%
\pgfpathlineto{\pgfqpoint{4.478468in}{2.393972in}}%
\pgfpathclose%
\pgfusepath{fill}%
\end{pgfscope}%
\begin{pgfscope}%
\pgfpathrectangle{\pgfqpoint{0.017500in}{0.017500in}}{\pgfqpoint{6.218282in}{3.580372in}} %
\pgfusepath{clip}%
\pgfsetbuttcap%
\pgfsetroundjoin%
\definecolor{currentfill}{rgb}{0.763520,0.178667,0.193396}%
\pgfsetfillcolor{currentfill}%
\pgfsetlinewidth{0.000000pt}%
\definecolor{currentstroke}{rgb}{0.000000,0.000000,0.000000}%
\pgfsetstrokecolor{currentstroke}%
\pgfsetdash{}{0pt}%
\pgfpathmoveto{\pgfqpoint{4.478468in}{2.393972in}}%
\pgfpathlineto{\pgfqpoint{4.724651in}{2.549289in}}%
\pgfpathlineto{\pgfqpoint{4.478468in}{2.393972in}}%
\pgfpathclose%
\pgfusepath{fill}%
\end{pgfscope}%
\begin{pgfscope}%
\pgfpathrectangle{\pgfqpoint{0.017500in}{0.017500in}}{\pgfqpoint{6.218282in}{3.580372in}} %
\pgfusepath{clip}%
\pgfsetbuttcap%
\pgfsetroundjoin%
\definecolor{currentfill}{rgb}{0.763520,0.178667,0.193396}%
\pgfsetfillcolor{currentfill}%
\pgfsetlinewidth{0.000000pt}%
\definecolor{currentstroke}{rgb}{0.000000,0.000000,0.000000}%
\pgfsetstrokecolor{currentstroke}%
\pgfsetdash{}{0pt}%
\pgfpathmoveto{\pgfqpoint{4.478468in}{2.393972in}}%
\pgfpathlineto{\pgfqpoint{4.724651in}{2.549289in}}%
\pgfpathlineto{\pgfqpoint{4.478468in}{2.393972in}}%
\pgfpathclose%
\pgfusepath{fill}%
\end{pgfscope}%
\begin{pgfscope}%
\pgfpathrectangle{\pgfqpoint{0.017500in}{0.017500in}}{\pgfqpoint{6.218282in}{3.580372in}} %
\pgfusepath{clip}%
\pgfsetbuttcap%
\pgfsetroundjoin%
\definecolor{currentfill}{rgb}{0.763520,0.178667,0.193396}%
\pgfsetfillcolor{currentfill}%
\pgfsetlinewidth{0.000000pt}%
\definecolor{currentstroke}{rgb}{0.000000,0.000000,0.000000}%
\pgfsetstrokecolor{currentstroke}%
\pgfsetdash{}{0pt}%
\pgfpathmoveto{\pgfqpoint{4.478468in}{2.393972in}}%
\pgfpathlineto{\pgfqpoint{4.724651in}{2.549289in}}%
\pgfpathlineto{\pgfqpoint{4.478468in}{2.393972in}}%
\pgfpathclose%
\pgfusepath{fill}%
\end{pgfscope}%
\begin{pgfscope}%
\pgfpathrectangle{\pgfqpoint{0.017500in}{0.017500in}}{\pgfqpoint{6.218282in}{3.580372in}} %
\pgfusepath{clip}%
\pgfsetbuttcap%
\pgfsetroundjoin%
\definecolor{currentfill}{rgb}{0.763520,0.178667,0.193396}%
\pgfsetfillcolor{currentfill}%
\pgfsetlinewidth{0.000000pt}%
\definecolor{currentstroke}{rgb}{0.000000,0.000000,0.000000}%
\pgfsetstrokecolor{currentstroke}%
\pgfsetdash{}{0pt}%
\pgfpathmoveto{\pgfqpoint{4.478468in}{2.393972in}}%
\pgfpathlineto{\pgfqpoint{4.724651in}{2.549289in}}%
\pgfpathlineto{\pgfqpoint{4.478468in}{2.393972in}}%
\pgfpathclose%
\pgfusepath{fill}%
\end{pgfscope}%
\begin{pgfscope}%
\pgfpathrectangle{\pgfqpoint{0.017500in}{0.017500in}}{\pgfqpoint{6.218282in}{3.580372in}} %
\pgfusepath{clip}%
\pgfsetbuttcap%
\pgfsetroundjoin%
\definecolor{currentfill}{rgb}{0.763520,0.178667,0.193396}%
\pgfsetfillcolor{currentfill}%
\pgfsetlinewidth{0.000000pt}%
\definecolor{currentstroke}{rgb}{0.000000,0.000000,0.000000}%
\pgfsetstrokecolor{currentstroke}%
\pgfsetdash{}{0pt}%
\pgfpathmoveto{\pgfqpoint{4.478468in}{2.393972in}}%
\pgfpathlineto{\pgfqpoint{4.724651in}{2.549289in}}%
\pgfpathlineto{\pgfqpoint{4.478468in}{2.393972in}}%
\pgfpathclose%
\pgfusepath{fill}%
\end{pgfscope}%
\begin{pgfscope}%
\pgfpathrectangle{\pgfqpoint{0.017500in}{0.017500in}}{\pgfqpoint{6.218282in}{3.580372in}} %
\pgfusepath{clip}%
\pgfsetbuttcap%
\pgfsetroundjoin%
\definecolor{currentfill}{rgb}{0.763520,0.178667,0.193396}%
\pgfsetfillcolor{currentfill}%
\pgfsetlinewidth{0.000000pt}%
\definecolor{currentstroke}{rgb}{0.000000,0.000000,0.000000}%
\pgfsetstrokecolor{currentstroke}%
\pgfsetdash{}{0pt}%
\pgfpathmoveto{\pgfqpoint{4.478468in}{2.393972in}}%
\pgfpathlineto{\pgfqpoint{4.724651in}{2.549289in}}%
\pgfpathlineto{\pgfqpoint{4.478468in}{2.393972in}}%
\pgfpathclose%
\pgfusepath{fill}%
\end{pgfscope}%
\begin{pgfscope}%
\pgfpathrectangle{\pgfqpoint{0.017500in}{0.017500in}}{\pgfqpoint{6.218282in}{3.580372in}} %
\pgfusepath{clip}%
\pgfsetbuttcap%
\pgfsetroundjoin%
\definecolor{currentfill}{rgb}{0.763520,0.178667,0.193396}%
\pgfsetfillcolor{currentfill}%
\pgfsetlinewidth{0.000000pt}%
\definecolor{currentstroke}{rgb}{0.000000,0.000000,0.000000}%
\pgfsetstrokecolor{currentstroke}%
\pgfsetdash{}{0pt}%
\pgfpathmoveto{\pgfqpoint{4.478468in}{2.393972in}}%
\pgfpathlineto{\pgfqpoint{4.724651in}{2.549289in}}%
\pgfpathlineto{\pgfqpoint{4.478468in}{2.393972in}}%
\pgfpathclose%
\pgfusepath{fill}%
\end{pgfscope}%
\begin{pgfscope}%
\pgfpathrectangle{\pgfqpoint{0.017500in}{0.017500in}}{\pgfqpoint{6.218282in}{3.580372in}} %
\pgfusepath{clip}%
\pgfsetbuttcap%
\pgfsetroundjoin%
\definecolor{currentfill}{rgb}{0.888390,0.417703,0.327898}%
\pgfsetfillcolor{currentfill}%
\pgfsetlinewidth{0.000000pt}%
\definecolor{currentstroke}{rgb}{0.000000,0.000000,0.000000}%
\pgfsetstrokecolor{currentstroke}%
\pgfsetdash{}{0pt}%
\pgfpathmoveto{\pgfqpoint{2.590432in}{2.210709in}}%
\pgfpathlineto{\pgfqpoint{3.043326in}{2.141782in}}%
\pgfpathlineto{\pgfqpoint{3.306485in}{2.382499in}}%
\pgfpathlineto{\pgfqpoint{2.855513in}{2.451418in}}%
\pgfpathlineto{\pgfqpoint{2.590432in}{2.210709in}}%
\pgfpathclose%
\pgfusepath{fill}%
\end{pgfscope}%
\begin{pgfscope}%
\pgfpathrectangle{\pgfqpoint{0.017500in}{0.017500in}}{\pgfqpoint{6.218282in}{3.580372in}} %
\pgfusepath{clip}%
\pgfsetbuttcap%
\pgfsetroundjoin%
\definecolor{currentfill}{rgb}{0.888390,0.417703,0.327898}%
\pgfsetfillcolor{currentfill}%
\pgfsetlinewidth{0.000000pt}%
\definecolor{currentstroke}{rgb}{0.000000,0.000000,0.000000}%
\pgfsetstrokecolor{currentstroke}%
\pgfsetdash{}{0pt}%
\pgfpathmoveto{\pgfqpoint{2.590432in}{2.210709in}}%
\pgfpathlineto{\pgfqpoint{3.043326in}{2.141782in}}%
\pgfpathlineto{\pgfqpoint{3.306485in}{2.382499in}}%
\pgfpathlineto{\pgfqpoint{2.855513in}{2.451418in}}%
\pgfpathlineto{\pgfqpoint{2.590432in}{2.210709in}}%
\pgfpathclose%
\pgfusepath{fill}%
\end{pgfscope}%
\begin{pgfscope}%
\pgfpathrectangle{\pgfqpoint{0.017500in}{0.017500in}}{\pgfqpoint{6.218282in}{3.580372in}} %
\pgfusepath{clip}%
\pgfsetbuttcap%
\pgfsetroundjoin%
\definecolor{currentfill}{rgb}{0.888390,0.417703,0.327898}%
\pgfsetfillcolor{currentfill}%
\pgfsetlinewidth{0.000000pt}%
\definecolor{currentstroke}{rgb}{0.000000,0.000000,0.000000}%
\pgfsetstrokecolor{currentstroke}%
\pgfsetdash{}{0pt}%
\pgfpathmoveto{\pgfqpoint{2.590432in}{2.210709in}}%
\pgfpathlineto{\pgfqpoint{3.043326in}{2.141782in}}%
\pgfpathlineto{\pgfqpoint{3.306485in}{2.382499in}}%
\pgfpathlineto{\pgfqpoint{2.855513in}{2.451418in}}%
\pgfpathlineto{\pgfqpoint{2.590432in}{2.210709in}}%
\pgfpathclose%
\pgfusepath{fill}%
\end{pgfscope}%
\begin{pgfscope}%
\pgfpathrectangle{\pgfqpoint{0.017500in}{0.017500in}}{\pgfqpoint{6.218282in}{3.580372in}} %
\pgfusepath{clip}%
\pgfsetbuttcap%
\pgfsetroundjoin%
\definecolor{currentfill}{rgb}{0.888390,0.417703,0.327898}%
\pgfsetfillcolor{currentfill}%
\pgfsetlinewidth{0.000000pt}%
\definecolor{currentstroke}{rgb}{0.000000,0.000000,0.000000}%
\pgfsetstrokecolor{currentstroke}%
\pgfsetdash{}{0pt}%
\pgfpathmoveto{\pgfqpoint{2.590432in}{2.210709in}}%
\pgfpathlineto{\pgfqpoint{3.043326in}{2.141782in}}%
\pgfpathlineto{\pgfqpoint{3.306485in}{2.382499in}}%
\pgfpathlineto{\pgfqpoint{2.855513in}{2.451418in}}%
\pgfpathlineto{\pgfqpoint{2.590432in}{2.210709in}}%
\pgfpathclose%
\pgfusepath{fill}%
\end{pgfscope}%
\begin{pgfscope}%
\pgfpathrectangle{\pgfqpoint{0.017500in}{0.017500in}}{\pgfqpoint{6.218282in}{3.580372in}} %
\pgfusepath{clip}%
\pgfsetbuttcap%
\pgfsetroundjoin%
\definecolor{currentfill}{rgb}{0.888390,0.417703,0.327898}%
\pgfsetfillcolor{currentfill}%
\pgfsetlinewidth{0.000000pt}%
\definecolor{currentstroke}{rgb}{0.000000,0.000000,0.000000}%
\pgfsetstrokecolor{currentstroke}%
\pgfsetdash{}{0pt}%
\pgfpathmoveto{\pgfqpoint{2.590432in}{2.210709in}}%
\pgfpathlineto{\pgfqpoint{3.043326in}{2.141782in}}%
\pgfpathlineto{\pgfqpoint{3.306485in}{2.382499in}}%
\pgfpathlineto{\pgfqpoint{2.855513in}{2.451418in}}%
\pgfpathlineto{\pgfqpoint{2.590432in}{2.210709in}}%
\pgfpathclose%
\pgfusepath{fill}%
\end{pgfscope}%
\begin{pgfscope}%
\pgfpathrectangle{\pgfqpoint{0.017500in}{0.017500in}}{\pgfqpoint{6.218282in}{3.580372in}} %
\pgfusepath{clip}%
\pgfsetbuttcap%
\pgfsetroundjoin%
\definecolor{currentfill}{rgb}{0.888390,0.417703,0.327898}%
\pgfsetfillcolor{currentfill}%
\pgfsetlinewidth{0.000000pt}%
\definecolor{currentstroke}{rgb}{0.000000,0.000000,0.000000}%
\pgfsetstrokecolor{currentstroke}%
\pgfsetdash{}{0pt}%
\pgfpathmoveto{\pgfqpoint{2.590432in}{2.210709in}}%
\pgfpathlineto{\pgfqpoint{3.043326in}{2.141782in}}%
\pgfpathlineto{\pgfqpoint{3.306485in}{2.382499in}}%
\pgfpathlineto{\pgfqpoint{2.855513in}{2.451418in}}%
\pgfpathlineto{\pgfqpoint{2.590432in}{2.210709in}}%
\pgfpathclose%
\pgfusepath{fill}%
\end{pgfscope}%
\begin{pgfscope}%
\pgfpathrectangle{\pgfqpoint{0.017500in}{0.017500in}}{\pgfqpoint{6.218282in}{3.580372in}} %
\pgfusepath{clip}%
\pgfsetbuttcap%
\pgfsetroundjoin%
\definecolor{currentfill}{rgb}{0.810616,0.268797,0.235428}%
\pgfsetfillcolor{currentfill}%
\pgfsetlinewidth{0.000000pt}%
\definecolor{currentstroke}{rgb}{0.000000,0.000000,0.000000}%
\pgfsetstrokecolor{currentstroke}%
\pgfsetdash{}{0pt}%
\pgfpathmoveto{\pgfqpoint{3.763398in}{2.297457in}}%
\pgfpathlineto{\pgfqpoint{4.018447in}{2.468167in}}%
\pgfpathlineto{\pgfqpoint{3.763398in}{2.297457in}}%
\pgfpathclose%
\pgfusepath{fill}%
\end{pgfscope}%
\begin{pgfscope}%
\pgfpathrectangle{\pgfqpoint{0.017500in}{0.017500in}}{\pgfqpoint{6.218282in}{3.580372in}} %
\pgfusepath{clip}%
\pgfsetbuttcap%
\pgfsetroundjoin%
\definecolor{currentfill}{rgb}{0.810616,0.268797,0.235428}%
\pgfsetfillcolor{currentfill}%
\pgfsetlinewidth{0.000000pt}%
\definecolor{currentstroke}{rgb}{0.000000,0.000000,0.000000}%
\pgfsetstrokecolor{currentstroke}%
\pgfsetdash{}{0pt}%
\pgfpathmoveto{\pgfqpoint{3.763398in}{2.297457in}}%
\pgfpathlineto{\pgfqpoint{4.018447in}{2.468167in}}%
\pgfpathlineto{\pgfqpoint{3.763398in}{2.297457in}}%
\pgfpathclose%
\pgfusepath{fill}%
\end{pgfscope}%
\begin{pgfscope}%
\pgfpathrectangle{\pgfqpoint{0.017500in}{0.017500in}}{\pgfqpoint{6.218282in}{3.580372in}} %
\pgfusepath{clip}%
\pgfsetbuttcap%
\pgfsetroundjoin%
\definecolor{currentfill}{rgb}{0.810616,0.268797,0.235428}%
\pgfsetfillcolor{currentfill}%
\pgfsetlinewidth{0.000000pt}%
\definecolor{currentstroke}{rgb}{0.000000,0.000000,0.000000}%
\pgfsetstrokecolor{currentstroke}%
\pgfsetdash{}{0pt}%
\pgfpathmoveto{\pgfqpoint{3.763398in}{2.297457in}}%
\pgfpathlineto{\pgfqpoint{4.018447in}{2.468167in}}%
\pgfpathlineto{\pgfqpoint{3.763398in}{2.297457in}}%
\pgfpathclose%
\pgfusepath{fill}%
\end{pgfscope}%
\begin{pgfscope}%
\pgfpathrectangle{\pgfqpoint{0.017500in}{0.017500in}}{\pgfqpoint{6.218282in}{3.580372in}} %
\pgfusepath{clip}%
\pgfsetbuttcap%
\pgfsetroundjoin%
\definecolor{currentfill}{rgb}{0.810616,0.268797,0.235428}%
\pgfsetfillcolor{currentfill}%
\pgfsetlinewidth{0.000000pt}%
\definecolor{currentstroke}{rgb}{0.000000,0.000000,0.000000}%
\pgfsetstrokecolor{currentstroke}%
\pgfsetdash{}{0pt}%
\pgfpathmoveto{\pgfqpoint{3.763398in}{2.297457in}}%
\pgfpathlineto{\pgfqpoint{4.018447in}{2.468167in}}%
\pgfpathlineto{\pgfqpoint{3.763398in}{2.297457in}}%
\pgfpathclose%
\pgfusepath{fill}%
\end{pgfscope}%
\begin{pgfscope}%
\pgfpathrectangle{\pgfqpoint{0.017500in}{0.017500in}}{\pgfqpoint{6.218282in}{3.580372in}} %
\pgfusepath{clip}%
\pgfsetbuttcap%
\pgfsetroundjoin%
\definecolor{currentfill}{rgb}{0.810616,0.268797,0.235428}%
\pgfsetfillcolor{currentfill}%
\pgfsetlinewidth{0.000000pt}%
\definecolor{currentstroke}{rgb}{0.000000,0.000000,0.000000}%
\pgfsetstrokecolor{currentstroke}%
\pgfsetdash{}{0pt}%
\pgfpathmoveto{\pgfqpoint{3.763398in}{2.297457in}}%
\pgfpathlineto{\pgfqpoint{4.018447in}{2.468167in}}%
\pgfpathlineto{\pgfqpoint{3.763398in}{2.297457in}}%
\pgfpathclose%
\pgfusepath{fill}%
\end{pgfscope}%
\begin{pgfscope}%
\pgfpathrectangle{\pgfqpoint{0.017500in}{0.017500in}}{\pgfqpoint{6.218282in}{3.580372in}} %
\pgfusepath{clip}%
\pgfsetbuttcap%
\pgfsetroundjoin%
\definecolor{currentfill}{rgb}{0.810616,0.268797,0.235428}%
\pgfsetfillcolor{currentfill}%
\pgfsetlinewidth{0.000000pt}%
\definecolor{currentstroke}{rgb}{0.000000,0.000000,0.000000}%
\pgfsetstrokecolor{currentstroke}%
\pgfsetdash{}{0pt}%
\pgfpathmoveto{\pgfqpoint{3.763398in}{2.297457in}}%
\pgfpathlineto{\pgfqpoint{4.018447in}{2.468167in}}%
\pgfpathlineto{\pgfqpoint{3.763398in}{2.297457in}}%
\pgfpathclose%
\pgfusepath{fill}%
\end{pgfscope}%
\begin{pgfscope}%
\pgfpathrectangle{\pgfqpoint{0.017500in}{0.017500in}}{\pgfqpoint{6.218282in}{3.580372in}} %
\pgfusepath{clip}%
\pgfsetbuttcap%
\pgfsetroundjoin%
\definecolor{currentfill}{rgb}{0.810616,0.268797,0.235428}%
\pgfsetfillcolor{currentfill}%
\pgfsetlinewidth{0.000000pt}%
\definecolor{currentstroke}{rgb}{0.000000,0.000000,0.000000}%
\pgfsetstrokecolor{currentstroke}%
\pgfsetdash{}{0pt}%
\pgfpathmoveto{\pgfqpoint{3.763398in}{2.297457in}}%
\pgfpathlineto{\pgfqpoint{4.018447in}{2.468167in}}%
\pgfpathlineto{\pgfqpoint{3.763398in}{2.297457in}}%
\pgfpathclose%
\pgfusepath{fill}%
\end{pgfscope}%
\begin{pgfscope}%
\pgfpathrectangle{\pgfqpoint{0.017500in}{0.017500in}}{\pgfqpoint{6.218282in}{3.580372in}} %
\pgfusepath{clip}%
\pgfsetbuttcap%
\pgfsetroundjoin%
\definecolor{currentfill}{rgb}{0.810616,0.268797,0.235428}%
\pgfsetfillcolor{currentfill}%
\pgfsetlinewidth{0.000000pt}%
\definecolor{currentstroke}{rgb}{0.000000,0.000000,0.000000}%
\pgfsetstrokecolor{currentstroke}%
\pgfsetdash{}{0pt}%
\pgfpathmoveto{\pgfqpoint{3.763398in}{2.297457in}}%
\pgfpathlineto{\pgfqpoint{4.018447in}{2.468167in}}%
\pgfpathlineto{\pgfqpoint{3.763398in}{2.297457in}}%
\pgfpathclose%
\pgfusepath{fill}%
\end{pgfscope}%
\begin{pgfscope}%
\pgfpathrectangle{\pgfqpoint{0.017500in}{0.017500in}}{\pgfqpoint{6.218282in}{3.580372in}} %
\pgfusepath{clip}%
\pgfsetbuttcap%
\pgfsetroundjoin%
\definecolor{currentfill}{rgb}{0.810616,0.268797,0.235428}%
\pgfsetfillcolor{currentfill}%
\pgfsetlinewidth{0.000000pt}%
\definecolor{currentstroke}{rgb}{0.000000,0.000000,0.000000}%
\pgfsetstrokecolor{currentstroke}%
\pgfsetdash{}{0pt}%
\pgfpathmoveto{\pgfqpoint{3.763398in}{2.297457in}}%
\pgfpathlineto{\pgfqpoint{4.018447in}{2.468167in}}%
\pgfpathlineto{\pgfqpoint{3.763398in}{2.297457in}}%
\pgfpathclose%
\pgfusepath{fill}%
\end{pgfscope}%
\begin{pgfscope}%
\pgfpathrectangle{\pgfqpoint{0.017500in}{0.017500in}}{\pgfqpoint{6.218282in}{3.580372in}} %
\pgfusepath{clip}%
\pgfsetbuttcap%
\pgfsetroundjoin%
\definecolor{currentfill}{rgb}{0.810616,0.268797,0.235428}%
\pgfsetfillcolor{currentfill}%
\pgfsetlinewidth{0.000000pt}%
\definecolor{currentstroke}{rgb}{0.000000,0.000000,0.000000}%
\pgfsetstrokecolor{currentstroke}%
\pgfsetdash{}{0pt}%
\pgfpathmoveto{\pgfqpoint{3.763398in}{2.297457in}}%
\pgfpathlineto{\pgfqpoint{4.018447in}{2.468167in}}%
\pgfpathlineto{\pgfqpoint{3.763398in}{2.297457in}}%
\pgfpathclose%
\pgfusepath{fill}%
\end{pgfscope}%
\begin{pgfscope}%
\pgfpathrectangle{\pgfqpoint{0.017500in}{0.017500in}}{\pgfqpoint{6.218282in}{3.580372in}} %
\pgfusepath{clip}%
\pgfsetbuttcap%
\pgfsetroundjoin%
\definecolor{currentfill}{rgb}{0.810616,0.268797,0.235428}%
\pgfsetfillcolor{currentfill}%
\pgfsetlinewidth{0.000000pt}%
\definecolor{currentstroke}{rgb}{0.000000,0.000000,0.000000}%
\pgfsetstrokecolor{currentstroke}%
\pgfsetdash{}{0pt}%
\pgfpathmoveto{\pgfqpoint{3.763398in}{2.297457in}}%
\pgfpathlineto{\pgfqpoint{4.018447in}{2.468167in}}%
\pgfpathlineto{\pgfqpoint{3.763398in}{2.297457in}}%
\pgfpathclose%
\pgfusepath{fill}%
\end{pgfscope}%
\begin{pgfscope}%
\pgfpathrectangle{\pgfqpoint{0.017500in}{0.017500in}}{\pgfqpoint{6.218282in}{3.580372in}} %
\pgfusepath{clip}%
\pgfsetbuttcap%
\pgfsetroundjoin%
\definecolor{currentfill}{rgb}{0.810616,0.268797,0.235428}%
\pgfsetfillcolor{currentfill}%
\pgfsetlinewidth{0.000000pt}%
\definecolor{currentstroke}{rgb}{0.000000,0.000000,0.000000}%
\pgfsetstrokecolor{currentstroke}%
\pgfsetdash{}{0pt}%
\pgfpathmoveto{\pgfqpoint{3.763398in}{2.297457in}}%
\pgfpathlineto{\pgfqpoint{4.018447in}{2.468167in}}%
\pgfpathlineto{\pgfqpoint{3.763398in}{2.297457in}}%
\pgfpathclose%
\pgfusepath{fill}%
\end{pgfscope}%
\begin{pgfscope}%
\pgfpathrectangle{\pgfqpoint{0.017500in}{0.017500in}}{\pgfqpoint{6.218282in}{3.580372in}} %
\pgfusepath{clip}%
\pgfsetbuttcap%
\pgfsetroundjoin%
\definecolor{currentfill}{rgb}{0.810616,0.268797,0.235428}%
\pgfsetfillcolor{currentfill}%
\pgfsetlinewidth{0.000000pt}%
\definecolor{currentstroke}{rgb}{0.000000,0.000000,0.000000}%
\pgfsetstrokecolor{currentstroke}%
\pgfsetdash{}{0pt}%
\pgfpathmoveto{\pgfqpoint{3.763398in}{2.297457in}}%
\pgfpathlineto{\pgfqpoint{4.018447in}{2.468167in}}%
\pgfpathlineto{\pgfqpoint{3.763398in}{2.297457in}}%
\pgfpathclose%
\pgfusepath{fill}%
\end{pgfscope}%
\begin{pgfscope}%
\pgfpathrectangle{\pgfqpoint{0.017500in}{0.017500in}}{\pgfqpoint{6.218282in}{3.580372in}} %
\pgfusepath{clip}%
\pgfsetbuttcap%
\pgfsetroundjoin%
\definecolor{currentfill}{rgb}{0.810616,0.268797,0.235428}%
\pgfsetfillcolor{currentfill}%
\pgfsetlinewidth{0.000000pt}%
\definecolor{currentstroke}{rgb}{0.000000,0.000000,0.000000}%
\pgfsetstrokecolor{currentstroke}%
\pgfsetdash{}{0pt}%
\pgfpathmoveto{\pgfqpoint{3.763398in}{2.297457in}}%
\pgfpathlineto{\pgfqpoint{4.018447in}{2.468167in}}%
\pgfpathlineto{\pgfqpoint{3.763398in}{2.297457in}}%
\pgfpathclose%
\pgfusepath{fill}%
\end{pgfscope}%
\begin{pgfscope}%
\pgfpathrectangle{\pgfqpoint{0.017500in}{0.017500in}}{\pgfqpoint{6.218282in}{3.580372in}} %
\pgfusepath{clip}%
\pgfsetbuttcap%
\pgfsetroundjoin%
\definecolor{currentfill}{rgb}{0.810616,0.268797,0.235428}%
\pgfsetfillcolor{currentfill}%
\pgfsetlinewidth{0.000000pt}%
\definecolor{currentstroke}{rgb}{0.000000,0.000000,0.000000}%
\pgfsetstrokecolor{currentstroke}%
\pgfsetdash{}{0pt}%
\pgfpathmoveto{\pgfqpoint{3.763398in}{2.297457in}}%
\pgfpathlineto{\pgfqpoint{4.018447in}{2.468167in}}%
\pgfpathlineto{\pgfqpoint{3.763398in}{2.297457in}}%
\pgfpathclose%
\pgfusepath{fill}%
\end{pgfscope}%
\begin{pgfscope}%
\pgfpathrectangle{\pgfqpoint{0.017500in}{0.017500in}}{\pgfqpoint{6.218282in}{3.580372in}} %
\pgfusepath{clip}%
\pgfsetbuttcap%
\pgfsetroundjoin%
\definecolor{currentfill}{rgb}{0.810616,0.268797,0.235428}%
\pgfsetfillcolor{currentfill}%
\pgfsetlinewidth{0.000000pt}%
\definecolor{currentstroke}{rgb}{0.000000,0.000000,0.000000}%
\pgfsetstrokecolor{currentstroke}%
\pgfsetdash{}{0pt}%
\pgfpathmoveto{\pgfqpoint{3.763398in}{2.297457in}}%
\pgfpathlineto{\pgfqpoint{4.018447in}{2.468167in}}%
\pgfpathlineto{\pgfqpoint{3.763398in}{2.297457in}}%
\pgfpathclose%
\pgfusepath{fill}%
\end{pgfscope}%
\begin{pgfscope}%
\pgfpathrectangle{\pgfqpoint{0.017500in}{0.017500in}}{\pgfqpoint{6.218282in}{3.580372in}} %
\pgfusepath{clip}%
\pgfsetbuttcap%
\pgfsetroundjoin%
\definecolor{currentfill}{rgb}{0.810616,0.268797,0.235428}%
\pgfsetfillcolor{currentfill}%
\pgfsetlinewidth{0.000000pt}%
\definecolor{currentstroke}{rgb}{0.000000,0.000000,0.000000}%
\pgfsetstrokecolor{currentstroke}%
\pgfsetdash{}{0pt}%
\pgfpathmoveto{\pgfqpoint{3.763398in}{2.297457in}}%
\pgfpathlineto{\pgfqpoint{4.018447in}{2.468167in}}%
\pgfpathlineto{\pgfqpoint{3.763398in}{2.297457in}}%
\pgfpathclose%
\pgfusepath{fill}%
\end{pgfscope}%
\begin{pgfscope}%
\pgfpathrectangle{\pgfqpoint{0.017500in}{0.017500in}}{\pgfqpoint{6.218282in}{3.580372in}} %
\pgfusepath{clip}%
\pgfsetbuttcap%
\pgfsetroundjoin%
\definecolor{currentfill}{rgb}{0.810616,0.268797,0.235428}%
\pgfsetfillcolor{currentfill}%
\pgfsetlinewidth{0.000000pt}%
\definecolor{currentstroke}{rgb}{0.000000,0.000000,0.000000}%
\pgfsetstrokecolor{currentstroke}%
\pgfsetdash{}{0pt}%
\pgfpathmoveto{\pgfqpoint{3.763398in}{2.297457in}}%
\pgfpathlineto{\pgfqpoint{4.018447in}{2.468167in}}%
\pgfpathlineto{\pgfqpoint{3.763398in}{2.297457in}}%
\pgfpathclose%
\pgfusepath{fill}%
\end{pgfscope}%
\begin{pgfscope}%
\pgfpathrectangle{\pgfqpoint{0.017500in}{0.017500in}}{\pgfqpoint{6.218282in}{3.580372in}} %
\pgfusepath{clip}%
\pgfsetbuttcap%
\pgfsetroundjoin%
\definecolor{currentfill}{rgb}{0.810616,0.268797,0.235428}%
\pgfsetfillcolor{currentfill}%
\pgfsetlinewidth{0.000000pt}%
\definecolor{currentstroke}{rgb}{0.000000,0.000000,0.000000}%
\pgfsetstrokecolor{currentstroke}%
\pgfsetdash{}{0pt}%
\pgfpathmoveto{\pgfqpoint{3.763398in}{2.297457in}}%
\pgfpathlineto{\pgfqpoint{4.018447in}{2.468167in}}%
\pgfpathlineto{\pgfqpoint{3.763398in}{2.297457in}}%
\pgfpathclose%
\pgfusepath{fill}%
\end{pgfscope}%
\begin{pgfscope}%
\pgfpathrectangle{\pgfqpoint{0.017500in}{0.017500in}}{\pgfqpoint{6.218282in}{3.580372in}} %
\pgfusepath{clip}%
\pgfsetbuttcap%
\pgfsetroundjoin%
\definecolor{currentfill}{rgb}{0.810616,0.268797,0.235428}%
\pgfsetfillcolor{currentfill}%
\pgfsetlinewidth{0.000000pt}%
\definecolor{currentstroke}{rgb}{0.000000,0.000000,0.000000}%
\pgfsetstrokecolor{currentstroke}%
\pgfsetdash{}{0pt}%
\pgfpathmoveto{\pgfqpoint{3.763398in}{2.297457in}}%
\pgfpathlineto{\pgfqpoint{4.018447in}{2.468167in}}%
\pgfpathlineto{\pgfqpoint{3.763398in}{2.297457in}}%
\pgfpathclose%
\pgfusepath{fill}%
\end{pgfscope}%
\begin{pgfscope}%
\pgfpathrectangle{\pgfqpoint{0.017500in}{0.017500in}}{\pgfqpoint{6.218282in}{3.580372in}} %
\pgfusepath{clip}%
\pgfsetbuttcap%
\pgfsetroundjoin%
\definecolor{currentfill}{rgb}{0.810616,0.268797,0.235428}%
\pgfsetfillcolor{currentfill}%
\pgfsetlinewidth{0.000000pt}%
\definecolor{currentstroke}{rgb}{0.000000,0.000000,0.000000}%
\pgfsetstrokecolor{currentstroke}%
\pgfsetdash{}{0pt}%
\pgfpathmoveto{\pgfqpoint{3.763398in}{2.297457in}}%
\pgfpathlineto{\pgfqpoint{4.018447in}{2.468167in}}%
\pgfpathlineto{\pgfqpoint{3.763398in}{2.297457in}}%
\pgfpathclose%
\pgfusepath{fill}%
\end{pgfscope}%
\begin{pgfscope}%
\pgfpathrectangle{\pgfqpoint{0.017500in}{0.017500in}}{\pgfqpoint{6.218282in}{3.580372in}} %
\pgfusepath{clip}%
\pgfsetbuttcap%
\pgfsetroundjoin%
\definecolor{currentfill}{rgb}{0.810616,0.268797,0.235428}%
\pgfsetfillcolor{currentfill}%
\pgfsetlinewidth{0.000000pt}%
\definecolor{currentstroke}{rgb}{0.000000,0.000000,0.000000}%
\pgfsetstrokecolor{currentstroke}%
\pgfsetdash{}{0pt}%
\pgfpathmoveto{\pgfqpoint{3.763398in}{2.297457in}}%
\pgfpathlineto{\pgfqpoint{4.018447in}{2.468167in}}%
\pgfpathlineto{\pgfqpoint{3.763398in}{2.297457in}}%
\pgfpathclose%
\pgfusepath{fill}%
\end{pgfscope}%
\begin{pgfscope}%
\pgfpathrectangle{\pgfqpoint{0.017500in}{0.017500in}}{\pgfqpoint{6.218282in}{3.580372in}} %
\pgfusepath{clip}%
\pgfsetbuttcap%
\pgfsetroundjoin%
\definecolor{currentfill}{rgb}{0.810616,0.268797,0.235428}%
\pgfsetfillcolor{currentfill}%
\pgfsetlinewidth{0.000000pt}%
\definecolor{currentstroke}{rgb}{0.000000,0.000000,0.000000}%
\pgfsetstrokecolor{currentstroke}%
\pgfsetdash{}{0pt}%
\pgfpathmoveto{\pgfqpoint{3.763398in}{2.297457in}}%
\pgfpathlineto{\pgfqpoint{4.018447in}{2.468167in}}%
\pgfpathlineto{\pgfqpoint{3.763398in}{2.297457in}}%
\pgfpathclose%
\pgfusepath{fill}%
\end{pgfscope}%
\begin{pgfscope}%
\pgfpathrectangle{\pgfqpoint{0.017500in}{0.017500in}}{\pgfqpoint{6.218282in}{3.580372in}} %
\pgfusepath{clip}%
\pgfsetbuttcap%
\pgfsetroundjoin%
\definecolor{currentfill}{rgb}{0.810616,0.268797,0.235428}%
\pgfsetfillcolor{currentfill}%
\pgfsetlinewidth{0.000000pt}%
\definecolor{currentstroke}{rgb}{0.000000,0.000000,0.000000}%
\pgfsetstrokecolor{currentstroke}%
\pgfsetdash{}{0pt}%
\pgfpathmoveto{\pgfqpoint{3.763398in}{2.297457in}}%
\pgfpathlineto{\pgfqpoint{4.018447in}{2.468167in}}%
\pgfpathlineto{\pgfqpoint{3.763398in}{2.297457in}}%
\pgfpathclose%
\pgfusepath{fill}%
\end{pgfscope}%
\begin{pgfscope}%
\pgfpathrectangle{\pgfqpoint{0.017500in}{0.017500in}}{\pgfqpoint{6.218282in}{3.580372in}} %
\pgfusepath{clip}%
\pgfsetbuttcap%
\pgfsetroundjoin%
\definecolor{currentfill}{rgb}{0.810616,0.268797,0.235428}%
\pgfsetfillcolor{currentfill}%
\pgfsetlinewidth{0.000000pt}%
\definecolor{currentstroke}{rgb}{0.000000,0.000000,0.000000}%
\pgfsetstrokecolor{currentstroke}%
\pgfsetdash{}{0pt}%
\pgfpathmoveto{\pgfqpoint{3.763398in}{2.297457in}}%
\pgfpathlineto{\pgfqpoint{4.018447in}{2.468167in}}%
\pgfpathlineto{\pgfqpoint{3.763398in}{2.297457in}}%
\pgfpathclose%
\pgfusepath{fill}%
\end{pgfscope}%
\begin{pgfscope}%
\pgfpathrectangle{\pgfqpoint{0.017500in}{0.017500in}}{\pgfqpoint{6.218282in}{3.580372in}} %
\pgfusepath{clip}%
\pgfsetbuttcap%
\pgfsetroundjoin%
\definecolor{currentfill}{rgb}{0.810616,0.268797,0.235428}%
\pgfsetfillcolor{currentfill}%
\pgfsetlinewidth{0.000000pt}%
\definecolor{currentstroke}{rgb}{0.000000,0.000000,0.000000}%
\pgfsetstrokecolor{currentstroke}%
\pgfsetdash{}{0pt}%
\pgfpathmoveto{\pgfqpoint{3.763398in}{2.297457in}}%
\pgfpathlineto{\pgfqpoint{4.018447in}{2.468167in}}%
\pgfpathlineto{\pgfqpoint{3.763398in}{2.297457in}}%
\pgfpathclose%
\pgfusepath{fill}%
\end{pgfscope}%
\begin{pgfscope}%
\pgfpathrectangle{\pgfqpoint{0.017500in}{0.017500in}}{\pgfqpoint{6.218282in}{3.580372in}} %
\pgfusepath{clip}%
\pgfsetbuttcap%
\pgfsetroundjoin%
\definecolor{currentfill}{rgb}{0.810616,0.268797,0.235428}%
\pgfsetfillcolor{currentfill}%
\pgfsetlinewidth{0.000000pt}%
\definecolor{currentstroke}{rgb}{0.000000,0.000000,0.000000}%
\pgfsetstrokecolor{currentstroke}%
\pgfsetdash{}{0pt}%
\pgfpathmoveto{\pgfqpoint{3.763398in}{2.297457in}}%
\pgfpathlineto{\pgfqpoint{4.018447in}{2.468167in}}%
\pgfpathlineto{\pgfqpoint{3.763398in}{2.297457in}}%
\pgfpathclose%
\pgfusepath{fill}%
\end{pgfscope}%
\begin{pgfscope}%
\pgfpathrectangle{\pgfqpoint{0.017500in}{0.017500in}}{\pgfqpoint{6.218282in}{3.580372in}} %
\pgfusepath{clip}%
\pgfsetbuttcap%
\pgfsetroundjoin%
\definecolor{currentfill}{rgb}{0.810616,0.268797,0.235428}%
\pgfsetfillcolor{currentfill}%
\pgfsetlinewidth{0.000000pt}%
\definecolor{currentstroke}{rgb}{0.000000,0.000000,0.000000}%
\pgfsetstrokecolor{currentstroke}%
\pgfsetdash{}{0pt}%
\pgfpathmoveto{\pgfqpoint{3.763398in}{2.297457in}}%
\pgfpathlineto{\pgfqpoint{4.018447in}{2.468167in}}%
\pgfpathlineto{\pgfqpoint{3.763398in}{2.297457in}}%
\pgfpathclose%
\pgfusepath{fill}%
\end{pgfscope}%
\begin{pgfscope}%
\pgfpathrectangle{\pgfqpoint{0.017500in}{0.017500in}}{\pgfqpoint{6.218282in}{3.580372in}} %
\pgfusepath{clip}%
\pgfsetbuttcap%
\pgfsetroundjoin%
\definecolor{currentfill}{rgb}{0.810616,0.268797,0.235428}%
\pgfsetfillcolor{currentfill}%
\pgfsetlinewidth{0.000000pt}%
\definecolor{currentstroke}{rgb}{0.000000,0.000000,0.000000}%
\pgfsetstrokecolor{currentstroke}%
\pgfsetdash{}{0pt}%
\pgfpathmoveto{\pgfqpoint{3.763398in}{2.297457in}}%
\pgfpathlineto{\pgfqpoint{4.018447in}{2.468167in}}%
\pgfpathlineto{\pgfqpoint{3.763398in}{2.297457in}}%
\pgfpathclose%
\pgfusepath{fill}%
\end{pgfscope}%
\begin{pgfscope}%
\pgfpathrectangle{\pgfqpoint{0.017500in}{0.017500in}}{\pgfqpoint{6.218282in}{3.580372in}} %
\pgfusepath{clip}%
\pgfsetbuttcap%
\pgfsetroundjoin%
\definecolor{currentfill}{rgb}{0.810616,0.268797,0.235428}%
\pgfsetfillcolor{currentfill}%
\pgfsetlinewidth{0.000000pt}%
\definecolor{currentstroke}{rgb}{0.000000,0.000000,0.000000}%
\pgfsetstrokecolor{currentstroke}%
\pgfsetdash{}{0pt}%
\pgfpathmoveto{\pgfqpoint{3.763398in}{2.297457in}}%
\pgfpathlineto{\pgfqpoint{4.018447in}{2.468167in}}%
\pgfpathlineto{\pgfqpoint{3.763398in}{2.297457in}}%
\pgfpathclose%
\pgfusepath{fill}%
\end{pgfscope}%
\begin{pgfscope}%
\pgfpathrectangle{\pgfqpoint{0.017500in}{0.017500in}}{\pgfqpoint{6.218282in}{3.580372in}} %
\pgfusepath{clip}%
\pgfsetbuttcap%
\pgfsetroundjoin%
\definecolor{currentfill}{rgb}{0.338377,0.452819,0.879317}%
\pgfsetfillcolor{currentfill}%
\pgfsetlinewidth{0.000000pt}%
\definecolor{currentstroke}{rgb}{0.000000,0.000000,0.000000}%
\pgfsetstrokecolor{currentstroke}%
\pgfsetdash{}{0pt}%
\pgfpathmoveto{\pgfqpoint{3.236099in}{0.723497in}}%
\pgfpathlineto{\pgfqpoint{3.688351in}{0.641693in}}%
\pgfpathlineto{\pgfqpoint{3.968076in}{1.994042in}}%
\pgfpathlineto{\pgfqpoint{3.502561in}{2.073396in}}%
\pgfpathlineto{\pgfqpoint{3.236099in}{0.723497in}}%
\pgfpathclose%
\pgfusepath{fill}%
\end{pgfscope}%
\begin{pgfscope}%
\pgfpathrectangle{\pgfqpoint{0.017500in}{0.017500in}}{\pgfqpoint{6.218282in}{3.580372in}} %
\pgfusepath{clip}%
\pgfsetbuttcap%
\pgfsetroundjoin%
\definecolor{currentfill}{rgb}{0.338377,0.452819,0.879317}%
\pgfsetfillcolor{currentfill}%
\pgfsetlinewidth{0.000000pt}%
\definecolor{currentstroke}{rgb}{0.000000,0.000000,0.000000}%
\pgfsetstrokecolor{currentstroke}%
\pgfsetdash{}{0pt}%
\pgfpathmoveto{\pgfqpoint{3.236099in}{0.723497in}}%
\pgfpathlineto{\pgfqpoint{3.688351in}{0.641693in}}%
\pgfpathlineto{\pgfqpoint{3.968076in}{1.994042in}}%
\pgfpathlineto{\pgfqpoint{3.502561in}{2.073396in}}%
\pgfpathlineto{\pgfqpoint{3.236099in}{0.723497in}}%
\pgfpathclose%
\pgfusepath{fill}%
\end{pgfscope}%
\begin{pgfscope}%
\pgfpathrectangle{\pgfqpoint{0.017500in}{0.017500in}}{\pgfqpoint{6.218282in}{3.580372in}} %
\pgfusepath{clip}%
\pgfsetbuttcap%
\pgfsetroundjoin%
\definecolor{currentfill}{rgb}{0.338377,0.452819,0.879317}%
\pgfsetfillcolor{currentfill}%
\pgfsetlinewidth{0.000000pt}%
\definecolor{currentstroke}{rgb}{0.000000,0.000000,0.000000}%
\pgfsetstrokecolor{currentstroke}%
\pgfsetdash{}{0pt}%
\pgfpathmoveto{\pgfqpoint{3.236099in}{0.723497in}}%
\pgfpathlineto{\pgfqpoint{3.688351in}{0.641693in}}%
\pgfpathlineto{\pgfqpoint{3.968076in}{1.994042in}}%
\pgfpathlineto{\pgfqpoint{3.502561in}{2.073396in}}%
\pgfpathlineto{\pgfqpoint{3.236099in}{0.723497in}}%
\pgfpathclose%
\pgfusepath{fill}%
\end{pgfscope}%
\begin{pgfscope}%
\pgfpathrectangle{\pgfqpoint{0.017500in}{0.017500in}}{\pgfqpoint{6.218282in}{3.580372in}} %
\pgfusepath{clip}%
\pgfsetbuttcap%
\pgfsetroundjoin%
\definecolor{currentfill}{rgb}{0.338377,0.452819,0.879317}%
\pgfsetfillcolor{currentfill}%
\pgfsetlinewidth{0.000000pt}%
\definecolor{currentstroke}{rgb}{0.000000,0.000000,0.000000}%
\pgfsetstrokecolor{currentstroke}%
\pgfsetdash{}{0pt}%
\pgfpathmoveto{\pgfqpoint{3.236099in}{0.723497in}}%
\pgfpathlineto{\pgfqpoint{3.688351in}{0.641693in}}%
\pgfpathlineto{\pgfqpoint{3.968076in}{1.994042in}}%
\pgfpathlineto{\pgfqpoint{3.502561in}{2.073396in}}%
\pgfpathlineto{\pgfqpoint{3.236099in}{0.723497in}}%
\pgfpathclose%
\pgfusepath{fill}%
\end{pgfscope}%
\begin{pgfscope}%
\pgfpathrectangle{\pgfqpoint{0.017500in}{0.017500in}}{\pgfqpoint{6.218282in}{3.580372in}} %
\pgfusepath{clip}%
\pgfsetbuttcap%
\pgfsetroundjoin%
\definecolor{currentfill}{rgb}{0.338377,0.452819,0.879317}%
\pgfsetfillcolor{currentfill}%
\pgfsetlinewidth{0.000000pt}%
\definecolor{currentstroke}{rgb}{0.000000,0.000000,0.000000}%
\pgfsetstrokecolor{currentstroke}%
\pgfsetdash{}{0pt}%
\pgfpathmoveto{\pgfqpoint{3.236099in}{0.723497in}}%
\pgfpathlineto{\pgfqpoint{3.688351in}{0.641693in}}%
\pgfpathlineto{\pgfqpoint{3.968076in}{1.994042in}}%
\pgfpathlineto{\pgfqpoint{3.502561in}{2.073396in}}%
\pgfpathlineto{\pgfqpoint{3.236099in}{0.723497in}}%
\pgfpathclose%
\pgfusepath{fill}%
\end{pgfscope}%
\begin{pgfscope}%
\pgfpathrectangle{\pgfqpoint{0.017500in}{0.017500in}}{\pgfqpoint{6.218282in}{3.580372in}} %
\pgfusepath{clip}%
\pgfsetbuttcap%
\pgfsetroundjoin%
\definecolor{currentfill}{rgb}{0.338377,0.452819,0.879317}%
\pgfsetfillcolor{currentfill}%
\pgfsetlinewidth{0.000000pt}%
\definecolor{currentstroke}{rgb}{0.000000,0.000000,0.000000}%
\pgfsetstrokecolor{currentstroke}%
\pgfsetdash{}{0pt}%
\pgfpathmoveto{\pgfqpoint{3.236099in}{0.723497in}}%
\pgfpathlineto{\pgfqpoint{3.688351in}{0.641693in}}%
\pgfpathlineto{\pgfqpoint{3.968076in}{1.994042in}}%
\pgfpathlineto{\pgfqpoint{3.502561in}{2.073396in}}%
\pgfpathlineto{\pgfqpoint{3.236099in}{0.723497in}}%
\pgfpathclose%
\pgfusepath{fill}%
\end{pgfscope}%
\begin{pgfscope}%
\pgfpathrectangle{\pgfqpoint{0.017500in}{0.017500in}}{\pgfqpoint{6.218282in}{3.580372in}} %
\pgfusepath{clip}%
\pgfsetbuttcap%
\pgfsetroundjoin%
\definecolor{currentfill}{rgb}{0.805723,0.259813,0.230562}%
\pgfsetfillcolor{currentfill}%
\pgfsetlinewidth{0.000000pt}%
\definecolor{currentstroke}{rgb}{0.000000,0.000000,0.000000}%
\pgfsetstrokecolor{currentstroke}%
\pgfsetdash{}{0pt}%
\pgfpathmoveto{\pgfqpoint{3.763398in}{2.297457in}}%
\pgfpathlineto{\pgfqpoint{4.226975in}{2.236848in}}%
\pgfpathlineto{\pgfqpoint{4.478468in}{2.393972in}}%
\pgfpathlineto{\pgfqpoint{4.018447in}{2.468167in}}%
\pgfpathlineto{\pgfqpoint{3.763398in}{2.297457in}}%
\pgfpathclose%
\pgfusepath{fill}%
\end{pgfscope}%
\begin{pgfscope}%
\pgfpathrectangle{\pgfqpoint{0.017500in}{0.017500in}}{\pgfqpoint{6.218282in}{3.580372in}} %
\pgfusepath{clip}%
\pgfsetbuttcap%
\pgfsetroundjoin%
\definecolor{currentfill}{rgb}{0.805723,0.259813,0.230562}%
\pgfsetfillcolor{currentfill}%
\pgfsetlinewidth{0.000000pt}%
\definecolor{currentstroke}{rgb}{0.000000,0.000000,0.000000}%
\pgfsetstrokecolor{currentstroke}%
\pgfsetdash{}{0pt}%
\pgfpathmoveto{\pgfqpoint{3.763398in}{2.297457in}}%
\pgfpathlineto{\pgfqpoint{4.226975in}{2.236848in}}%
\pgfpathlineto{\pgfqpoint{4.478468in}{2.393972in}}%
\pgfpathlineto{\pgfqpoint{4.018447in}{2.468167in}}%
\pgfpathlineto{\pgfqpoint{3.763398in}{2.297457in}}%
\pgfpathclose%
\pgfusepath{fill}%
\end{pgfscope}%
\begin{pgfscope}%
\pgfpathrectangle{\pgfqpoint{0.017500in}{0.017500in}}{\pgfqpoint{6.218282in}{3.580372in}} %
\pgfusepath{clip}%
\pgfsetbuttcap%
\pgfsetroundjoin%
\definecolor{currentfill}{rgb}{0.805723,0.259813,0.230562}%
\pgfsetfillcolor{currentfill}%
\pgfsetlinewidth{0.000000pt}%
\definecolor{currentstroke}{rgb}{0.000000,0.000000,0.000000}%
\pgfsetstrokecolor{currentstroke}%
\pgfsetdash{}{0pt}%
\pgfpathmoveto{\pgfqpoint{3.763398in}{2.297457in}}%
\pgfpathlineto{\pgfqpoint{4.226975in}{2.236848in}}%
\pgfpathlineto{\pgfqpoint{4.478468in}{2.393972in}}%
\pgfpathlineto{\pgfqpoint{4.018447in}{2.468167in}}%
\pgfpathlineto{\pgfqpoint{3.763398in}{2.297457in}}%
\pgfpathclose%
\pgfusepath{fill}%
\end{pgfscope}%
\begin{pgfscope}%
\pgfpathrectangle{\pgfqpoint{0.017500in}{0.017500in}}{\pgfqpoint{6.218282in}{3.580372in}} %
\pgfusepath{clip}%
\pgfsetbuttcap%
\pgfsetroundjoin%
\definecolor{currentfill}{rgb}{0.805723,0.259813,0.230562}%
\pgfsetfillcolor{currentfill}%
\pgfsetlinewidth{0.000000pt}%
\definecolor{currentstroke}{rgb}{0.000000,0.000000,0.000000}%
\pgfsetstrokecolor{currentstroke}%
\pgfsetdash{}{0pt}%
\pgfpathmoveto{\pgfqpoint{3.763398in}{2.297457in}}%
\pgfpathlineto{\pgfqpoint{4.226975in}{2.236848in}}%
\pgfpathlineto{\pgfqpoint{4.478468in}{2.393972in}}%
\pgfpathlineto{\pgfqpoint{4.018447in}{2.468167in}}%
\pgfpathlineto{\pgfqpoint{3.763398in}{2.297457in}}%
\pgfpathclose%
\pgfusepath{fill}%
\end{pgfscope}%
\begin{pgfscope}%
\pgfpathrectangle{\pgfqpoint{0.017500in}{0.017500in}}{\pgfqpoint{6.218282in}{3.580372in}} %
\pgfusepath{clip}%
\pgfsetbuttcap%
\pgfsetroundjoin%
\definecolor{currentfill}{rgb}{0.805723,0.259813,0.230562}%
\pgfsetfillcolor{currentfill}%
\pgfsetlinewidth{0.000000pt}%
\definecolor{currentstroke}{rgb}{0.000000,0.000000,0.000000}%
\pgfsetstrokecolor{currentstroke}%
\pgfsetdash{}{0pt}%
\pgfpathmoveto{\pgfqpoint{3.763398in}{2.297457in}}%
\pgfpathlineto{\pgfqpoint{4.226975in}{2.236848in}}%
\pgfpathlineto{\pgfqpoint{4.478468in}{2.393972in}}%
\pgfpathlineto{\pgfqpoint{4.018447in}{2.468167in}}%
\pgfpathlineto{\pgfqpoint{3.763398in}{2.297457in}}%
\pgfpathclose%
\pgfusepath{fill}%
\end{pgfscope}%
\begin{pgfscope}%
\pgfpathrectangle{\pgfqpoint{0.017500in}{0.017500in}}{\pgfqpoint{6.218282in}{3.580372in}} %
\pgfusepath{clip}%
\pgfsetbuttcap%
\pgfsetroundjoin%
\definecolor{currentfill}{rgb}{0.805723,0.259813,0.230562}%
\pgfsetfillcolor{currentfill}%
\pgfsetlinewidth{0.000000pt}%
\definecolor{currentstroke}{rgb}{0.000000,0.000000,0.000000}%
\pgfsetstrokecolor{currentstroke}%
\pgfsetdash{}{0pt}%
\pgfpathmoveto{\pgfqpoint{3.763398in}{2.297457in}}%
\pgfpathlineto{\pgfqpoint{4.226975in}{2.236848in}}%
\pgfpathlineto{\pgfqpoint{4.478468in}{2.393972in}}%
\pgfpathlineto{\pgfqpoint{4.018447in}{2.468167in}}%
\pgfpathlineto{\pgfqpoint{3.763398in}{2.297457in}}%
\pgfpathclose%
\pgfusepath{fill}%
\end{pgfscope}%
\begin{pgfscope}%
\pgfpathrectangle{\pgfqpoint{0.017500in}{0.017500in}}{\pgfqpoint{6.218282in}{3.580372in}} %
\pgfusepath{clip}%
\pgfsetbuttcap%
\pgfsetroundjoin%
\definecolor{currentfill}{rgb}{0.243520,0.319189,0.771672}%
\pgfsetfillcolor{currentfill}%
\pgfsetlinewidth{0.000000pt}%
\definecolor{currentstroke}{rgb}{0.000000,0.000000,0.000000}%
\pgfsetstrokecolor{currentstroke}%
\pgfsetdash{}{0pt}%
\pgfpathmoveto{\pgfqpoint{3.688351in}{0.641693in}}%
\pgfpathlineto{\pgfqpoint{3.968076in}{1.994042in}}%
\pgfpathlineto{\pgfqpoint{3.688351in}{0.641693in}}%
\pgfpathclose%
\pgfusepath{fill}%
\end{pgfscope}%
\begin{pgfscope}%
\pgfpathrectangle{\pgfqpoint{0.017500in}{0.017500in}}{\pgfqpoint{6.218282in}{3.580372in}} %
\pgfusepath{clip}%
\pgfsetbuttcap%
\pgfsetroundjoin%
\definecolor{currentfill}{rgb}{0.243520,0.319189,0.771672}%
\pgfsetfillcolor{currentfill}%
\pgfsetlinewidth{0.000000pt}%
\definecolor{currentstroke}{rgb}{0.000000,0.000000,0.000000}%
\pgfsetstrokecolor{currentstroke}%
\pgfsetdash{}{0pt}%
\pgfpathmoveto{\pgfqpoint{3.688351in}{0.641693in}}%
\pgfpathlineto{\pgfqpoint{3.968076in}{1.994042in}}%
\pgfpathlineto{\pgfqpoint{3.688351in}{0.641693in}}%
\pgfpathclose%
\pgfusepath{fill}%
\end{pgfscope}%
\begin{pgfscope}%
\pgfpathrectangle{\pgfqpoint{0.017500in}{0.017500in}}{\pgfqpoint{6.218282in}{3.580372in}} %
\pgfusepath{clip}%
\pgfsetbuttcap%
\pgfsetroundjoin%
\definecolor{currentfill}{rgb}{0.243520,0.319189,0.771672}%
\pgfsetfillcolor{currentfill}%
\pgfsetlinewidth{0.000000pt}%
\definecolor{currentstroke}{rgb}{0.000000,0.000000,0.000000}%
\pgfsetstrokecolor{currentstroke}%
\pgfsetdash{}{0pt}%
\pgfpathmoveto{\pgfqpoint{3.688351in}{0.641693in}}%
\pgfpathlineto{\pgfqpoint{3.968076in}{1.994042in}}%
\pgfpathlineto{\pgfqpoint{3.688351in}{0.641693in}}%
\pgfpathclose%
\pgfusepath{fill}%
\end{pgfscope}%
\begin{pgfscope}%
\pgfpathrectangle{\pgfqpoint{0.017500in}{0.017500in}}{\pgfqpoint{6.218282in}{3.580372in}} %
\pgfusepath{clip}%
\pgfsetbuttcap%
\pgfsetroundjoin%
\definecolor{currentfill}{rgb}{0.243520,0.319189,0.771672}%
\pgfsetfillcolor{currentfill}%
\pgfsetlinewidth{0.000000pt}%
\definecolor{currentstroke}{rgb}{0.000000,0.000000,0.000000}%
\pgfsetstrokecolor{currentstroke}%
\pgfsetdash{}{0pt}%
\pgfpathmoveto{\pgfqpoint{3.688351in}{0.641693in}}%
\pgfpathlineto{\pgfqpoint{3.968076in}{1.994042in}}%
\pgfpathlineto{\pgfqpoint{3.688351in}{0.641693in}}%
\pgfpathclose%
\pgfusepath{fill}%
\end{pgfscope}%
\begin{pgfscope}%
\pgfpathrectangle{\pgfqpoint{0.017500in}{0.017500in}}{\pgfqpoint{6.218282in}{3.580372in}} %
\pgfusepath{clip}%
\pgfsetbuttcap%
\pgfsetroundjoin%
\definecolor{currentfill}{rgb}{0.243520,0.319189,0.771672}%
\pgfsetfillcolor{currentfill}%
\pgfsetlinewidth{0.000000pt}%
\definecolor{currentstroke}{rgb}{0.000000,0.000000,0.000000}%
\pgfsetstrokecolor{currentstroke}%
\pgfsetdash{}{0pt}%
\pgfpathmoveto{\pgfqpoint{3.688351in}{0.641693in}}%
\pgfpathlineto{\pgfqpoint{3.968076in}{1.994042in}}%
\pgfpathlineto{\pgfqpoint{3.688351in}{0.641693in}}%
\pgfpathclose%
\pgfusepath{fill}%
\end{pgfscope}%
\begin{pgfscope}%
\pgfpathrectangle{\pgfqpoint{0.017500in}{0.017500in}}{\pgfqpoint{6.218282in}{3.580372in}} %
\pgfusepath{clip}%
\pgfsetbuttcap%
\pgfsetroundjoin%
\definecolor{currentfill}{rgb}{0.243520,0.319189,0.771672}%
\pgfsetfillcolor{currentfill}%
\pgfsetlinewidth{0.000000pt}%
\definecolor{currentstroke}{rgb}{0.000000,0.000000,0.000000}%
\pgfsetstrokecolor{currentstroke}%
\pgfsetdash{}{0pt}%
\pgfpathmoveto{\pgfqpoint{3.688351in}{0.641693in}}%
\pgfpathlineto{\pgfqpoint{3.968076in}{1.994042in}}%
\pgfpathlineto{\pgfqpoint{3.688351in}{0.641693in}}%
\pgfpathclose%
\pgfusepath{fill}%
\end{pgfscope}%
\begin{pgfscope}%
\pgfpathrectangle{\pgfqpoint{0.017500in}{0.017500in}}{\pgfqpoint{6.218282in}{3.580372in}} %
\pgfusepath{clip}%
\pgfsetbuttcap%
\pgfsetroundjoin%
\definecolor{currentfill}{rgb}{0.243520,0.319189,0.771672}%
\pgfsetfillcolor{currentfill}%
\pgfsetlinewidth{0.000000pt}%
\definecolor{currentstroke}{rgb}{0.000000,0.000000,0.000000}%
\pgfsetstrokecolor{currentstroke}%
\pgfsetdash{}{0pt}%
\pgfpathmoveto{\pgfqpoint{3.688351in}{0.641693in}}%
\pgfpathlineto{\pgfqpoint{3.968076in}{1.994042in}}%
\pgfpathlineto{\pgfqpoint{3.688351in}{0.641693in}}%
\pgfpathclose%
\pgfusepath{fill}%
\end{pgfscope}%
\begin{pgfscope}%
\pgfpathrectangle{\pgfqpoint{0.017500in}{0.017500in}}{\pgfqpoint{6.218282in}{3.580372in}} %
\pgfusepath{clip}%
\pgfsetbuttcap%
\pgfsetroundjoin%
\definecolor{currentfill}{rgb}{0.243520,0.319189,0.771672}%
\pgfsetfillcolor{currentfill}%
\pgfsetlinewidth{0.000000pt}%
\definecolor{currentstroke}{rgb}{0.000000,0.000000,0.000000}%
\pgfsetstrokecolor{currentstroke}%
\pgfsetdash{}{0pt}%
\pgfpathmoveto{\pgfqpoint{3.688351in}{0.641693in}}%
\pgfpathlineto{\pgfqpoint{3.968076in}{1.994042in}}%
\pgfpathlineto{\pgfqpoint{3.688351in}{0.641693in}}%
\pgfpathclose%
\pgfusepath{fill}%
\end{pgfscope}%
\begin{pgfscope}%
\pgfpathrectangle{\pgfqpoint{0.017500in}{0.017500in}}{\pgfqpoint{6.218282in}{3.580372in}} %
\pgfusepath{clip}%
\pgfsetbuttcap%
\pgfsetroundjoin%
\definecolor{currentfill}{rgb}{0.243520,0.319189,0.771672}%
\pgfsetfillcolor{currentfill}%
\pgfsetlinewidth{0.000000pt}%
\definecolor{currentstroke}{rgb}{0.000000,0.000000,0.000000}%
\pgfsetstrokecolor{currentstroke}%
\pgfsetdash{}{0pt}%
\pgfpathmoveto{\pgfqpoint{3.688351in}{0.641693in}}%
\pgfpathlineto{\pgfqpoint{3.968076in}{1.994042in}}%
\pgfpathlineto{\pgfqpoint{3.688351in}{0.641693in}}%
\pgfpathclose%
\pgfusepath{fill}%
\end{pgfscope}%
\begin{pgfscope}%
\pgfpathrectangle{\pgfqpoint{0.017500in}{0.017500in}}{\pgfqpoint{6.218282in}{3.580372in}} %
\pgfusepath{clip}%
\pgfsetbuttcap%
\pgfsetroundjoin%
\definecolor{currentfill}{rgb}{0.243520,0.319189,0.771672}%
\pgfsetfillcolor{currentfill}%
\pgfsetlinewidth{0.000000pt}%
\definecolor{currentstroke}{rgb}{0.000000,0.000000,0.000000}%
\pgfsetstrokecolor{currentstroke}%
\pgfsetdash{}{0pt}%
\pgfpathmoveto{\pgfqpoint{3.688351in}{0.641693in}}%
\pgfpathlineto{\pgfqpoint{3.968076in}{1.994042in}}%
\pgfpathlineto{\pgfqpoint{3.688351in}{0.641693in}}%
\pgfpathclose%
\pgfusepath{fill}%
\end{pgfscope}%
\begin{pgfscope}%
\pgfpathrectangle{\pgfqpoint{0.017500in}{0.017500in}}{\pgfqpoint{6.218282in}{3.580372in}} %
\pgfusepath{clip}%
\pgfsetbuttcap%
\pgfsetroundjoin%
\definecolor{currentfill}{rgb}{0.243520,0.319189,0.771672}%
\pgfsetfillcolor{currentfill}%
\pgfsetlinewidth{0.000000pt}%
\definecolor{currentstroke}{rgb}{0.000000,0.000000,0.000000}%
\pgfsetstrokecolor{currentstroke}%
\pgfsetdash{}{0pt}%
\pgfpathmoveto{\pgfqpoint{3.688351in}{0.641693in}}%
\pgfpathlineto{\pgfqpoint{3.968076in}{1.994042in}}%
\pgfpathlineto{\pgfqpoint{3.688351in}{0.641693in}}%
\pgfpathclose%
\pgfusepath{fill}%
\end{pgfscope}%
\begin{pgfscope}%
\pgfpathrectangle{\pgfqpoint{0.017500in}{0.017500in}}{\pgfqpoint{6.218282in}{3.580372in}} %
\pgfusepath{clip}%
\pgfsetbuttcap%
\pgfsetroundjoin%
\definecolor{currentfill}{rgb}{0.243520,0.319189,0.771672}%
\pgfsetfillcolor{currentfill}%
\pgfsetlinewidth{0.000000pt}%
\definecolor{currentstroke}{rgb}{0.000000,0.000000,0.000000}%
\pgfsetstrokecolor{currentstroke}%
\pgfsetdash{}{0pt}%
\pgfpathmoveto{\pgfqpoint{3.688351in}{0.641693in}}%
\pgfpathlineto{\pgfqpoint{3.968076in}{1.994042in}}%
\pgfpathlineto{\pgfqpoint{3.688351in}{0.641693in}}%
\pgfpathclose%
\pgfusepath{fill}%
\end{pgfscope}%
\begin{pgfscope}%
\pgfpathrectangle{\pgfqpoint{0.017500in}{0.017500in}}{\pgfqpoint{6.218282in}{3.580372in}} %
\pgfusepath{clip}%
\pgfsetbuttcap%
\pgfsetroundjoin%
\definecolor{currentfill}{rgb}{0.243520,0.319189,0.771672}%
\pgfsetfillcolor{currentfill}%
\pgfsetlinewidth{0.000000pt}%
\definecolor{currentstroke}{rgb}{0.000000,0.000000,0.000000}%
\pgfsetstrokecolor{currentstroke}%
\pgfsetdash{}{0pt}%
\pgfpathmoveto{\pgfqpoint{3.688351in}{0.641693in}}%
\pgfpathlineto{\pgfqpoint{3.968076in}{1.994042in}}%
\pgfpathlineto{\pgfqpoint{3.688351in}{0.641693in}}%
\pgfpathclose%
\pgfusepath{fill}%
\end{pgfscope}%
\begin{pgfscope}%
\pgfpathrectangle{\pgfqpoint{0.017500in}{0.017500in}}{\pgfqpoint{6.218282in}{3.580372in}} %
\pgfusepath{clip}%
\pgfsetbuttcap%
\pgfsetroundjoin%
\definecolor{currentfill}{rgb}{0.243520,0.319189,0.771672}%
\pgfsetfillcolor{currentfill}%
\pgfsetlinewidth{0.000000pt}%
\definecolor{currentstroke}{rgb}{0.000000,0.000000,0.000000}%
\pgfsetstrokecolor{currentstroke}%
\pgfsetdash{}{0pt}%
\pgfpathmoveto{\pgfqpoint{3.688351in}{0.641693in}}%
\pgfpathlineto{\pgfqpoint{3.968076in}{1.994042in}}%
\pgfpathlineto{\pgfqpoint{3.688351in}{0.641693in}}%
\pgfpathclose%
\pgfusepath{fill}%
\end{pgfscope}%
\begin{pgfscope}%
\pgfpathrectangle{\pgfqpoint{0.017500in}{0.017500in}}{\pgfqpoint{6.218282in}{3.580372in}} %
\pgfusepath{clip}%
\pgfsetbuttcap%
\pgfsetroundjoin%
\definecolor{currentfill}{rgb}{0.243520,0.319189,0.771672}%
\pgfsetfillcolor{currentfill}%
\pgfsetlinewidth{0.000000pt}%
\definecolor{currentstroke}{rgb}{0.000000,0.000000,0.000000}%
\pgfsetstrokecolor{currentstroke}%
\pgfsetdash{}{0pt}%
\pgfpathmoveto{\pgfqpoint{3.688351in}{0.641693in}}%
\pgfpathlineto{\pgfqpoint{3.968076in}{1.994042in}}%
\pgfpathlineto{\pgfqpoint{3.688351in}{0.641693in}}%
\pgfpathclose%
\pgfusepath{fill}%
\end{pgfscope}%
\begin{pgfscope}%
\pgfpathrectangle{\pgfqpoint{0.017500in}{0.017500in}}{\pgfqpoint{6.218282in}{3.580372in}} %
\pgfusepath{clip}%
\pgfsetbuttcap%
\pgfsetroundjoin%
\definecolor{currentfill}{rgb}{0.243520,0.319189,0.771672}%
\pgfsetfillcolor{currentfill}%
\pgfsetlinewidth{0.000000pt}%
\definecolor{currentstroke}{rgb}{0.000000,0.000000,0.000000}%
\pgfsetstrokecolor{currentstroke}%
\pgfsetdash{}{0pt}%
\pgfpathmoveto{\pgfqpoint{3.688351in}{0.641693in}}%
\pgfpathlineto{\pgfqpoint{3.968076in}{1.994042in}}%
\pgfpathlineto{\pgfqpoint{3.688351in}{0.641693in}}%
\pgfpathclose%
\pgfusepath{fill}%
\end{pgfscope}%
\begin{pgfscope}%
\pgfpathrectangle{\pgfqpoint{0.017500in}{0.017500in}}{\pgfqpoint{6.218282in}{3.580372in}} %
\pgfusepath{clip}%
\pgfsetbuttcap%
\pgfsetroundjoin%
\definecolor{currentfill}{rgb}{0.243520,0.319189,0.771672}%
\pgfsetfillcolor{currentfill}%
\pgfsetlinewidth{0.000000pt}%
\definecolor{currentstroke}{rgb}{0.000000,0.000000,0.000000}%
\pgfsetstrokecolor{currentstroke}%
\pgfsetdash{}{0pt}%
\pgfpathmoveto{\pgfqpoint{3.688351in}{0.641693in}}%
\pgfpathlineto{\pgfqpoint{3.968076in}{1.994042in}}%
\pgfpathlineto{\pgfqpoint{3.688351in}{0.641693in}}%
\pgfpathclose%
\pgfusepath{fill}%
\end{pgfscope}%
\begin{pgfscope}%
\pgfpathrectangle{\pgfqpoint{0.017500in}{0.017500in}}{\pgfqpoint{6.218282in}{3.580372in}} %
\pgfusepath{clip}%
\pgfsetbuttcap%
\pgfsetroundjoin%
\definecolor{currentfill}{rgb}{0.243520,0.319189,0.771672}%
\pgfsetfillcolor{currentfill}%
\pgfsetlinewidth{0.000000pt}%
\definecolor{currentstroke}{rgb}{0.000000,0.000000,0.000000}%
\pgfsetstrokecolor{currentstroke}%
\pgfsetdash{}{0pt}%
\pgfpathmoveto{\pgfqpoint{3.688351in}{0.641693in}}%
\pgfpathlineto{\pgfqpoint{3.968076in}{1.994042in}}%
\pgfpathlineto{\pgfqpoint{3.688351in}{0.641693in}}%
\pgfpathclose%
\pgfusepath{fill}%
\end{pgfscope}%
\begin{pgfscope}%
\pgfpathrectangle{\pgfqpoint{0.017500in}{0.017500in}}{\pgfqpoint{6.218282in}{3.580372in}} %
\pgfusepath{clip}%
\pgfsetbuttcap%
\pgfsetroundjoin%
\definecolor{currentfill}{rgb}{0.243520,0.319189,0.771672}%
\pgfsetfillcolor{currentfill}%
\pgfsetlinewidth{0.000000pt}%
\definecolor{currentstroke}{rgb}{0.000000,0.000000,0.000000}%
\pgfsetstrokecolor{currentstroke}%
\pgfsetdash{}{0pt}%
\pgfpathmoveto{\pgfqpoint{3.688351in}{0.641693in}}%
\pgfpathlineto{\pgfqpoint{3.968076in}{1.994042in}}%
\pgfpathlineto{\pgfqpoint{3.688351in}{0.641693in}}%
\pgfpathclose%
\pgfusepath{fill}%
\end{pgfscope}%
\begin{pgfscope}%
\pgfpathrectangle{\pgfqpoint{0.017500in}{0.017500in}}{\pgfqpoint{6.218282in}{3.580372in}} %
\pgfusepath{clip}%
\pgfsetbuttcap%
\pgfsetroundjoin%
\definecolor{currentfill}{rgb}{0.243520,0.319189,0.771672}%
\pgfsetfillcolor{currentfill}%
\pgfsetlinewidth{0.000000pt}%
\definecolor{currentstroke}{rgb}{0.000000,0.000000,0.000000}%
\pgfsetstrokecolor{currentstroke}%
\pgfsetdash{}{0pt}%
\pgfpathmoveto{\pgfqpoint{3.688351in}{0.641693in}}%
\pgfpathlineto{\pgfqpoint{3.968076in}{1.994042in}}%
\pgfpathlineto{\pgfqpoint{3.688351in}{0.641693in}}%
\pgfpathclose%
\pgfusepath{fill}%
\end{pgfscope}%
\begin{pgfscope}%
\pgfpathrectangle{\pgfqpoint{0.017500in}{0.017500in}}{\pgfqpoint{6.218282in}{3.580372in}} %
\pgfusepath{clip}%
\pgfsetbuttcap%
\pgfsetroundjoin%
\definecolor{currentfill}{rgb}{0.243520,0.319189,0.771672}%
\pgfsetfillcolor{currentfill}%
\pgfsetlinewidth{0.000000pt}%
\definecolor{currentstroke}{rgb}{0.000000,0.000000,0.000000}%
\pgfsetstrokecolor{currentstroke}%
\pgfsetdash{}{0pt}%
\pgfpathmoveto{\pgfqpoint{3.688351in}{0.641693in}}%
\pgfpathlineto{\pgfqpoint{3.968076in}{1.994042in}}%
\pgfpathlineto{\pgfqpoint{3.688351in}{0.641693in}}%
\pgfpathclose%
\pgfusepath{fill}%
\end{pgfscope}%
\begin{pgfscope}%
\pgfpathrectangle{\pgfqpoint{0.017500in}{0.017500in}}{\pgfqpoint{6.218282in}{3.580372in}} %
\pgfusepath{clip}%
\pgfsetbuttcap%
\pgfsetroundjoin%
\definecolor{currentfill}{rgb}{0.243520,0.319189,0.771672}%
\pgfsetfillcolor{currentfill}%
\pgfsetlinewidth{0.000000pt}%
\definecolor{currentstroke}{rgb}{0.000000,0.000000,0.000000}%
\pgfsetstrokecolor{currentstroke}%
\pgfsetdash{}{0pt}%
\pgfpathmoveto{\pgfqpoint{3.688351in}{0.641693in}}%
\pgfpathlineto{\pgfqpoint{3.968076in}{1.994042in}}%
\pgfpathlineto{\pgfqpoint{3.688351in}{0.641693in}}%
\pgfpathclose%
\pgfusepath{fill}%
\end{pgfscope}%
\begin{pgfscope}%
\pgfpathrectangle{\pgfqpoint{0.017500in}{0.017500in}}{\pgfqpoint{6.218282in}{3.580372in}} %
\pgfusepath{clip}%
\pgfsetbuttcap%
\pgfsetroundjoin%
\definecolor{currentfill}{rgb}{0.243520,0.319189,0.771672}%
\pgfsetfillcolor{currentfill}%
\pgfsetlinewidth{0.000000pt}%
\definecolor{currentstroke}{rgb}{0.000000,0.000000,0.000000}%
\pgfsetstrokecolor{currentstroke}%
\pgfsetdash{}{0pt}%
\pgfpathmoveto{\pgfqpoint{3.688351in}{0.641693in}}%
\pgfpathlineto{\pgfqpoint{3.968076in}{1.994042in}}%
\pgfpathlineto{\pgfqpoint{3.688351in}{0.641693in}}%
\pgfpathclose%
\pgfusepath{fill}%
\end{pgfscope}%
\begin{pgfscope}%
\pgfpathrectangle{\pgfqpoint{0.017500in}{0.017500in}}{\pgfqpoint{6.218282in}{3.580372in}} %
\pgfusepath{clip}%
\pgfsetbuttcap%
\pgfsetroundjoin%
\definecolor{currentfill}{rgb}{0.243520,0.319189,0.771672}%
\pgfsetfillcolor{currentfill}%
\pgfsetlinewidth{0.000000pt}%
\definecolor{currentstroke}{rgb}{0.000000,0.000000,0.000000}%
\pgfsetstrokecolor{currentstroke}%
\pgfsetdash{}{0pt}%
\pgfpathmoveto{\pgfqpoint{3.688351in}{0.641693in}}%
\pgfpathlineto{\pgfqpoint{3.968076in}{1.994042in}}%
\pgfpathlineto{\pgfqpoint{3.688351in}{0.641693in}}%
\pgfpathclose%
\pgfusepath{fill}%
\end{pgfscope}%
\begin{pgfscope}%
\pgfpathrectangle{\pgfqpoint{0.017500in}{0.017500in}}{\pgfqpoint{6.218282in}{3.580372in}} %
\pgfusepath{clip}%
\pgfsetbuttcap%
\pgfsetroundjoin%
\definecolor{currentfill}{rgb}{0.243520,0.319189,0.771672}%
\pgfsetfillcolor{currentfill}%
\pgfsetlinewidth{0.000000pt}%
\definecolor{currentstroke}{rgb}{0.000000,0.000000,0.000000}%
\pgfsetstrokecolor{currentstroke}%
\pgfsetdash{}{0pt}%
\pgfpathmoveto{\pgfqpoint{3.688351in}{0.641693in}}%
\pgfpathlineto{\pgfqpoint{3.968076in}{1.994042in}}%
\pgfpathlineto{\pgfqpoint{3.688351in}{0.641693in}}%
\pgfpathclose%
\pgfusepath{fill}%
\end{pgfscope}%
\begin{pgfscope}%
\pgfpathrectangle{\pgfqpoint{0.017500in}{0.017500in}}{\pgfqpoint{6.218282in}{3.580372in}} %
\pgfusepath{clip}%
\pgfsetbuttcap%
\pgfsetroundjoin%
\definecolor{currentfill}{rgb}{0.243520,0.319189,0.771672}%
\pgfsetfillcolor{currentfill}%
\pgfsetlinewidth{0.000000pt}%
\definecolor{currentstroke}{rgb}{0.000000,0.000000,0.000000}%
\pgfsetstrokecolor{currentstroke}%
\pgfsetdash{}{0pt}%
\pgfpathmoveto{\pgfqpoint{3.688351in}{0.641693in}}%
\pgfpathlineto{\pgfqpoint{3.968076in}{1.994042in}}%
\pgfpathlineto{\pgfqpoint{3.688351in}{0.641693in}}%
\pgfpathclose%
\pgfusepath{fill}%
\end{pgfscope}%
\begin{pgfscope}%
\pgfpathrectangle{\pgfqpoint{0.017500in}{0.017500in}}{\pgfqpoint{6.218282in}{3.580372in}} %
\pgfusepath{clip}%
\pgfsetbuttcap%
\pgfsetroundjoin%
\definecolor{currentfill}{rgb}{0.243520,0.319189,0.771672}%
\pgfsetfillcolor{currentfill}%
\pgfsetlinewidth{0.000000pt}%
\definecolor{currentstroke}{rgb}{0.000000,0.000000,0.000000}%
\pgfsetstrokecolor{currentstroke}%
\pgfsetdash{}{0pt}%
\pgfpathmoveto{\pgfqpoint{3.688351in}{0.641693in}}%
\pgfpathlineto{\pgfqpoint{3.968076in}{1.994042in}}%
\pgfpathlineto{\pgfqpoint{3.688351in}{0.641693in}}%
\pgfpathclose%
\pgfusepath{fill}%
\end{pgfscope}%
\begin{pgfscope}%
\pgfpathrectangle{\pgfqpoint{0.017500in}{0.017500in}}{\pgfqpoint{6.218282in}{3.580372in}} %
\pgfusepath{clip}%
\pgfsetbuttcap%
\pgfsetroundjoin%
\definecolor{currentfill}{rgb}{0.243520,0.319189,0.771672}%
\pgfsetfillcolor{currentfill}%
\pgfsetlinewidth{0.000000pt}%
\definecolor{currentstroke}{rgb}{0.000000,0.000000,0.000000}%
\pgfsetstrokecolor{currentstroke}%
\pgfsetdash{}{0pt}%
\pgfpathmoveto{\pgfqpoint{3.688351in}{0.641693in}}%
\pgfpathlineto{\pgfqpoint{3.968076in}{1.994042in}}%
\pgfpathlineto{\pgfqpoint{3.688351in}{0.641693in}}%
\pgfpathclose%
\pgfusepath{fill}%
\end{pgfscope}%
\begin{pgfscope}%
\pgfpathrectangle{\pgfqpoint{0.017500in}{0.017500in}}{\pgfqpoint{6.218282in}{3.580372in}} %
\pgfusepath{clip}%
\pgfsetbuttcap%
\pgfsetroundjoin%
\definecolor{currentfill}{rgb}{0.243520,0.319189,0.771672}%
\pgfsetfillcolor{currentfill}%
\pgfsetlinewidth{0.000000pt}%
\definecolor{currentstroke}{rgb}{0.000000,0.000000,0.000000}%
\pgfsetstrokecolor{currentstroke}%
\pgfsetdash{}{0pt}%
\pgfpathmoveto{\pgfqpoint{3.688351in}{0.641693in}}%
\pgfpathlineto{\pgfqpoint{3.968076in}{1.994042in}}%
\pgfpathlineto{\pgfqpoint{3.688351in}{0.641693in}}%
\pgfpathclose%
\pgfusepath{fill}%
\end{pgfscope}%
\begin{pgfscope}%
\pgfpathrectangle{\pgfqpoint{0.017500in}{0.017500in}}{\pgfqpoint{6.218282in}{3.580372in}} %
\pgfusepath{clip}%
\pgfsetbuttcap%
\pgfsetroundjoin%
\definecolor{currentfill}{rgb}{0.243520,0.319189,0.771672}%
\pgfsetfillcolor{currentfill}%
\pgfsetlinewidth{0.000000pt}%
\definecolor{currentstroke}{rgb}{0.000000,0.000000,0.000000}%
\pgfsetstrokecolor{currentstroke}%
\pgfsetdash{}{0pt}%
\pgfpathmoveto{\pgfqpoint{3.688351in}{0.641693in}}%
\pgfpathlineto{\pgfqpoint{3.968076in}{1.994042in}}%
\pgfpathlineto{\pgfqpoint{3.688351in}{0.641693in}}%
\pgfpathclose%
\pgfusepath{fill}%
\end{pgfscope}%
\begin{pgfscope}%
\pgfpathrectangle{\pgfqpoint{0.017500in}{0.017500in}}{\pgfqpoint{6.218282in}{3.580372in}} %
\pgfusepath{clip}%
\pgfsetbuttcap%
\pgfsetroundjoin%
\definecolor{currentfill}{rgb}{0.895885,0.433075,0.338681}%
\pgfsetfillcolor{currentfill}%
\pgfsetlinewidth{0.000000pt}%
\definecolor{currentstroke}{rgb}{0.000000,0.000000,0.000000}%
\pgfsetstrokecolor{currentstroke}%
\pgfsetdash{}{0pt}%
\pgfpathmoveto{\pgfqpoint{3.043326in}{2.141782in}}%
\pgfpathlineto{\pgfqpoint{3.306485in}{2.382499in}}%
\pgfpathlineto{\pgfqpoint{3.043326in}{2.141782in}}%
\pgfpathclose%
\pgfusepath{fill}%
\end{pgfscope}%
\begin{pgfscope}%
\pgfpathrectangle{\pgfqpoint{0.017500in}{0.017500in}}{\pgfqpoint{6.218282in}{3.580372in}} %
\pgfusepath{clip}%
\pgfsetbuttcap%
\pgfsetroundjoin%
\definecolor{currentfill}{rgb}{0.895885,0.433075,0.338681}%
\pgfsetfillcolor{currentfill}%
\pgfsetlinewidth{0.000000pt}%
\definecolor{currentstroke}{rgb}{0.000000,0.000000,0.000000}%
\pgfsetstrokecolor{currentstroke}%
\pgfsetdash{}{0pt}%
\pgfpathmoveto{\pgfqpoint{3.043326in}{2.141782in}}%
\pgfpathlineto{\pgfqpoint{3.306485in}{2.382499in}}%
\pgfpathlineto{\pgfqpoint{3.043326in}{2.141782in}}%
\pgfpathclose%
\pgfusepath{fill}%
\end{pgfscope}%
\begin{pgfscope}%
\pgfpathrectangle{\pgfqpoint{0.017500in}{0.017500in}}{\pgfqpoint{6.218282in}{3.580372in}} %
\pgfusepath{clip}%
\pgfsetbuttcap%
\pgfsetroundjoin%
\definecolor{currentfill}{rgb}{0.895885,0.433075,0.338681}%
\pgfsetfillcolor{currentfill}%
\pgfsetlinewidth{0.000000pt}%
\definecolor{currentstroke}{rgb}{0.000000,0.000000,0.000000}%
\pgfsetstrokecolor{currentstroke}%
\pgfsetdash{}{0pt}%
\pgfpathmoveto{\pgfqpoint{3.043326in}{2.141782in}}%
\pgfpathlineto{\pgfqpoint{3.306485in}{2.382499in}}%
\pgfpathlineto{\pgfqpoint{3.043326in}{2.141782in}}%
\pgfpathclose%
\pgfusepath{fill}%
\end{pgfscope}%
\begin{pgfscope}%
\pgfpathrectangle{\pgfqpoint{0.017500in}{0.017500in}}{\pgfqpoint{6.218282in}{3.580372in}} %
\pgfusepath{clip}%
\pgfsetbuttcap%
\pgfsetroundjoin%
\definecolor{currentfill}{rgb}{0.895885,0.433075,0.338681}%
\pgfsetfillcolor{currentfill}%
\pgfsetlinewidth{0.000000pt}%
\definecolor{currentstroke}{rgb}{0.000000,0.000000,0.000000}%
\pgfsetstrokecolor{currentstroke}%
\pgfsetdash{}{0pt}%
\pgfpathmoveto{\pgfqpoint{3.043326in}{2.141782in}}%
\pgfpathlineto{\pgfqpoint{3.306485in}{2.382499in}}%
\pgfpathlineto{\pgfqpoint{3.043326in}{2.141782in}}%
\pgfpathclose%
\pgfusepath{fill}%
\end{pgfscope}%
\begin{pgfscope}%
\pgfpathrectangle{\pgfqpoint{0.017500in}{0.017500in}}{\pgfqpoint{6.218282in}{3.580372in}} %
\pgfusepath{clip}%
\pgfsetbuttcap%
\pgfsetroundjoin%
\definecolor{currentfill}{rgb}{0.895885,0.433075,0.338681}%
\pgfsetfillcolor{currentfill}%
\pgfsetlinewidth{0.000000pt}%
\definecolor{currentstroke}{rgb}{0.000000,0.000000,0.000000}%
\pgfsetstrokecolor{currentstroke}%
\pgfsetdash{}{0pt}%
\pgfpathmoveto{\pgfqpoint{3.043326in}{2.141782in}}%
\pgfpathlineto{\pgfqpoint{3.306485in}{2.382499in}}%
\pgfpathlineto{\pgfqpoint{3.043326in}{2.141782in}}%
\pgfpathclose%
\pgfusepath{fill}%
\end{pgfscope}%
\begin{pgfscope}%
\pgfpathrectangle{\pgfqpoint{0.017500in}{0.017500in}}{\pgfqpoint{6.218282in}{3.580372in}} %
\pgfusepath{clip}%
\pgfsetbuttcap%
\pgfsetroundjoin%
\definecolor{currentfill}{rgb}{0.895885,0.433075,0.338681}%
\pgfsetfillcolor{currentfill}%
\pgfsetlinewidth{0.000000pt}%
\definecolor{currentstroke}{rgb}{0.000000,0.000000,0.000000}%
\pgfsetstrokecolor{currentstroke}%
\pgfsetdash{}{0pt}%
\pgfpathmoveto{\pgfqpoint{3.043326in}{2.141782in}}%
\pgfpathlineto{\pgfqpoint{3.306485in}{2.382499in}}%
\pgfpathlineto{\pgfqpoint{3.043326in}{2.141782in}}%
\pgfpathclose%
\pgfusepath{fill}%
\end{pgfscope}%
\begin{pgfscope}%
\pgfpathrectangle{\pgfqpoint{0.017500in}{0.017500in}}{\pgfqpoint{6.218282in}{3.580372in}} %
\pgfusepath{clip}%
\pgfsetbuttcap%
\pgfsetroundjoin%
\definecolor{currentfill}{rgb}{0.895885,0.433075,0.338681}%
\pgfsetfillcolor{currentfill}%
\pgfsetlinewidth{0.000000pt}%
\definecolor{currentstroke}{rgb}{0.000000,0.000000,0.000000}%
\pgfsetstrokecolor{currentstroke}%
\pgfsetdash{}{0pt}%
\pgfpathmoveto{\pgfqpoint{3.043326in}{2.141782in}}%
\pgfpathlineto{\pgfqpoint{3.306485in}{2.382499in}}%
\pgfpathlineto{\pgfqpoint{3.043326in}{2.141782in}}%
\pgfpathclose%
\pgfusepath{fill}%
\end{pgfscope}%
\begin{pgfscope}%
\pgfpathrectangle{\pgfqpoint{0.017500in}{0.017500in}}{\pgfqpoint{6.218282in}{3.580372in}} %
\pgfusepath{clip}%
\pgfsetbuttcap%
\pgfsetroundjoin%
\definecolor{currentfill}{rgb}{0.895885,0.433075,0.338681}%
\pgfsetfillcolor{currentfill}%
\pgfsetlinewidth{0.000000pt}%
\definecolor{currentstroke}{rgb}{0.000000,0.000000,0.000000}%
\pgfsetstrokecolor{currentstroke}%
\pgfsetdash{}{0pt}%
\pgfpathmoveto{\pgfqpoint{3.043326in}{2.141782in}}%
\pgfpathlineto{\pgfqpoint{3.306485in}{2.382499in}}%
\pgfpathlineto{\pgfqpoint{3.043326in}{2.141782in}}%
\pgfpathclose%
\pgfusepath{fill}%
\end{pgfscope}%
\begin{pgfscope}%
\pgfpathrectangle{\pgfqpoint{0.017500in}{0.017500in}}{\pgfqpoint{6.218282in}{3.580372in}} %
\pgfusepath{clip}%
\pgfsetbuttcap%
\pgfsetroundjoin%
\definecolor{currentfill}{rgb}{0.895885,0.433075,0.338681}%
\pgfsetfillcolor{currentfill}%
\pgfsetlinewidth{0.000000pt}%
\definecolor{currentstroke}{rgb}{0.000000,0.000000,0.000000}%
\pgfsetstrokecolor{currentstroke}%
\pgfsetdash{}{0pt}%
\pgfpathmoveto{\pgfqpoint{3.043326in}{2.141782in}}%
\pgfpathlineto{\pgfqpoint{3.306485in}{2.382499in}}%
\pgfpathlineto{\pgfqpoint{3.043326in}{2.141782in}}%
\pgfpathclose%
\pgfusepath{fill}%
\end{pgfscope}%
\begin{pgfscope}%
\pgfpathrectangle{\pgfqpoint{0.017500in}{0.017500in}}{\pgfqpoint{6.218282in}{3.580372in}} %
\pgfusepath{clip}%
\pgfsetbuttcap%
\pgfsetroundjoin%
\definecolor{currentfill}{rgb}{0.895885,0.433075,0.338681}%
\pgfsetfillcolor{currentfill}%
\pgfsetlinewidth{0.000000pt}%
\definecolor{currentstroke}{rgb}{0.000000,0.000000,0.000000}%
\pgfsetstrokecolor{currentstroke}%
\pgfsetdash{}{0pt}%
\pgfpathmoveto{\pgfqpoint{3.043326in}{2.141782in}}%
\pgfpathlineto{\pgfqpoint{3.306485in}{2.382499in}}%
\pgfpathlineto{\pgfqpoint{3.043326in}{2.141782in}}%
\pgfpathclose%
\pgfusepath{fill}%
\end{pgfscope}%
\begin{pgfscope}%
\pgfpathrectangle{\pgfqpoint{0.017500in}{0.017500in}}{\pgfqpoint{6.218282in}{3.580372in}} %
\pgfusepath{clip}%
\pgfsetbuttcap%
\pgfsetroundjoin%
\definecolor{currentfill}{rgb}{0.895885,0.433075,0.338681}%
\pgfsetfillcolor{currentfill}%
\pgfsetlinewidth{0.000000pt}%
\definecolor{currentstroke}{rgb}{0.000000,0.000000,0.000000}%
\pgfsetstrokecolor{currentstroke}%
\pgfsetdash{}{0pt}%
\pgfpathmoveto{\pgfqpoint{3.043326in}{2.141782in}}%
\pgfpathlineto{\pgfqpoint{3.306485in}{2.382499in}}%
\pgfpathlineto{\pgfqpoint{3.043326in}{2.141782in}}%
\pgfpathclose%
\pgfusepath{fill}%
\end{pgfscope}%
\begin{pgfscope}%
\pgfpathrectangle{\pgfqpoint{0.017500in}{0.017500in}}{\pgfqpoint{6.218282in}{3.580372in}} %
\pgfusepath{clip}%
\pgfsetbuttcap%
\pgfsetroundjoin%
\definecolor{currentfill}{rgb}{0.895885,0.433075,0.338681}%
\pgfsetfillcolor{currentfill}%
\pgfsetlinewidth{0.000000pt}%
\definecolor{currentstroke}{rgb}{0.000000,0.000000,0.000000}%
\pgfsetstrokecolor{currentstroke}%
\pgfsetdash{}{0pt}%
\pgfpathmoveto{\pgfqpoint{3.043326in}{2.141782in}}%
\pgfpathlineto{\pgfqpoint{3.306485in}{2.382499in}}%
\pgfpathlineto{\pgfqpoint{3.043326in}{2.141782in}}%
\pgfpathclose%
\pgfusepath{fill}%
\end{pgfscope}%
\begin{pgfscope}%
\pgfpathrectangle{\pgfqpoint{0.017500in}{0.017500in}}{\pgfqpoint{6.218282in}{3.580372in}} %
\pgfusepath{clip}%
\pgfsetbuttcap%
\pgfsetroundjoin%
\definecolor{currentfill}{rgb}{0.895885,0.433075,0.338681}%
\pgfsetfillcolor{currentfill}%
\pgfsetlinewidth{0.000000pt}%
\definecolor{currentstroke}{rgb}{0.000000,0.000000,0.000000}%
\pgfsetstrokecolor{currentstroke}%
\pgfsetdash{}{0pt}%
\pgfpathmoveto{\pgfqpoint{3.043326in}{2.141782in}}%
\pgfpathlineto{\pgfqpoint{3.306485in}{2.382499in}}%
\pgfpathlineto{\pgfqpoint{3.043326in}{2.141782in}}%
\pgfpathclose%
\pgfusepath{fill}%
\end{pgfscope}%
\begin{pgfscope}%
\pgfpathrectangle{\pgfqpoint{0.017500in}{0.017500in}}{\pgfqpoint{6.218282in}{3.580372in}} %
\pgfusepath{clip}%
\pgfsetbuttcap%
\pgfsetroundjoin%
\definecolor{currentfill}{rgb}{0.895885,0.433075,0.338681}%
\pgfsetfillcolor{currentfill}%
\pgfsetlinewidth{0.000000pt}%
\definecolor{currentstroke}{rgb}{0.000000,0.000000,0.000000}%
\pgfsetstrokecolor{currentstroke}%
\pgfsetdash{}{0pt}%
\pgfpathmoveto{\pgfqpoint{3.043326in}{2.141782in}}%
\pgfpathlineto{\pgfqpoint{3.306485in}{2.382499in}}%
\pgfpathlineto{\pgfqpoint{3.043326in}{2.141782in}}%
\pgfpathclose%
\pgfusepath{fill}%
\end{pgfscope}%
\begin{pgfscope}%
\pgfpathrectangle{\pgfqpoint{0.017500in}{0.017500in}}{\pgfqpoint{6.218282in}{3.580372in}} %
\pgfusepath{clip}%
\pgfsetbuttcap%
\pgfsetroundjoin%
\definecolor{currentfill}{rgb}{0.895885,0.433075,0.338681}%
\pgfsetfillcolor{currentfill}%
\pgfsetlinewidth{0.000000pt}%
\definecolor{currentstroke}{rgb}{0.000000,0.000000,0.000000}%
\pgfsetstrokecolor{currentstroke}%
\pgfsetdash{}{0pt}%
\pgfpathmoveto{\pgfqpoint{3.043326in}{2.141782in}}%
\pgfpathlineto{\pgfqpoint{3.306485in}{2.382499in}}%
\pgfpathlineto{\pgfqpoint{3.043326in}{2.141782in}}%
\pgfpathclose%
\pgfusepath{fill}%
\end{pgfscope}%
\begin{pgfscope}%
\pgfpathrectangle{\pgfqpoint{0.017500in}{0.017500in}}{\pgfqpoint{6.218282in}{3.580372in}} %
\pgfusepath{clip}%
\pgfsetbuttcap%
\pgfsetroundjoin%
\definecolor{currentfill}{rgb}{0.895885,0.433075,0.338681}%
\pgfsetfillcolor{currentfill}%
\pgfsetlinewidth{0.000000pt}%
\definecolor{currentstroke}{rgb}{0.000000,0.000000,0.000000}%
\pgfsetstrokecolor{currentstroke}%
\pgfsetdash{}{0pt}%
\pgfpathmoveto{\pgfqpoint{3.043326in}{2.141782in}}%
\pgfpathlineto{\pgfqpoint{3.306485in}{2.382499in}}%
\pgfpathlineto{\pgfqpoint{3.043326in}{2.141782in}}%
\pgfpathclose%
\pgfusepath{fill}%
\end{pgfscope}%
\begin{pgfscope}%
\pgfpathrectangle{\pgfqpoint{0.017500in}{0.017500in}}{\pgfqpoint{6.218282in}{3.580372in}} %
\pgfusepath{clip}%
\pgfsetbuttcap%
\pgfsetroundjoin%
\definecolor{currentfill}{rgb}{0.895885,0.433075,0.338681}%
\pgfsetfillcolor{currentfill}%
\pgfsetlinewidth{0.000000pt}%
\definecolor{currentstroke}{rgb}{0.000000,0.000000,0.000000}%
\pgfsetstrokecolor{currentstroke}%
\pgfsetdash{}{0pt}%
\pgfpathmoveto{\pgfqpoint{3.043326in}{2.141782in}}%
\pgfpathlineto{\pgfqpoint{3.306485in}{2.382499in}}%
\pgfpathlineto{\pgfqpoint{3.043326in}{2.141782in}}%
\pgfpathclose%
\pgfusepath{fill}%
\end{pgfscope}%
\begin{pgfscope}%
\pgfpathrectangle{\pgfqpoint{0.017500in}{0.017500in}}{\pgfqpoint{6.218282in}{3.580372in}} %
\pgfusepath{clip}%
\pgfsetbuttcap%
\pgfsetroundjoin%
\definecolor{currentfill}{rgb}{0.895885,0.433075,0.338681}%
\pgfsetfillcolor{currentfill}%
\pgfsetlinewidth{0.000000pt}%
\definecolor{currentstroke}{rgb}{0.000000,0.000000,0.000000}%
\pgfsetstrokecolor{currentstroke}%
\pgfsetdash{}{0pt}%
\pgfpathmoveto{\pgfqpoint{3.043326in}{2.141782in}}%
\pgfpathlineto{\pgfqpoint{3.306485in}{2.382499in}}%
\pgfpathlineto{\pgfqpoint{3.043326in}{2.141782in}}%
\pgfpathclose%
\pgfusepath{fill}%
\end{pgfscope}%
\begin{pgfscope}%
\pgfpathrectangle{\pgfqpoint{0.017500in}{0.017500in}}{\pgfqpoint{6.218282in}{3.580372in}} %
\pgfusepath{clip}%
\pgfsetbuttcap%
\pgfsetroundjoin%
\definecolor{currentfill}{rgb}{0.895885,0.433075,0.338681}%
\pgfsetfillcolor{currentfill}%
\pgfsetlinewidth{0.000000pt}%
\definecolor{currentstroke}{rgb}{0.000000,0.000000,0.000000}%
\pgfsetstrokecolor{currentstroke}%
\pgfsetdash{}{0pt}%
\pgfpathmoveto{\pgfqpoint{3.043326in}{2.141782in}}%
\pgfpathlineto{\pgfqpoint{3.306485in}{2.382499in}}%
\pgfpathlineto{\pgfqpoint{3.043326in}{2.141782in}}%
\pgfpathclose%
\pgfusepath{fill}%
\end{pgfscope}%
\begin{pgfscope}%
\pgfpathrectangle{\pgfqpoint{0.017500in}{0.017500in}}{\pgfqpoint{6.218282in}{3.580372in}} %
\pgfusepath{clip}%
\pgfsetbuttcap%
\pgfsetroundjoin%
\definecolor{currentfill}{rgb}{0.895885,0.433075,0.338681}%
\pgfsetfillcolor{currentfill}%
\pgfsetlinewidth{0.000000pt}%
\definecolor{currentstroke}{rgb}{0.000000,0.000000,0.000000}%
\pgfsetstrokecolor{currentstroke}%
\pgfsetdash{}{0pt}%
\pgfpathmoveto{\pgfqpoint{3.043326in}{2.141782in}}%
\pgfpathlineto{\pgfqpoint{3.306485in}{2.382499in}}%
\pgfpathlineto{\pgfqpoint{3.043326in}{2.141782in}}%
\pgfpathclose%
\pgfusepath{fill}%
\end{pgfscope}%
\begin{pgfscope}%
\pgfpathrectangle{\pgfqpoint{0.017500in}{0.017500in}}{\pgfqpoint{6.218282in}{3.580372in}} %
\pgfusepath{clip}%
\pgfsetbuttcap%
\pgfsetroundjoin%
\definecolor{currentfill}{rgb}{0.895885,0.433075,0.338681}%
\pgfsetfillcolor{currentfill}%
\pgfsetlinewidth{0.000000pt}%
\definecolor{currentstroke}{rgb}{0.000000,0.000000,0.000000}%
\pgfsetstrokecolor{currentstroke}%
\pgfsetdash{}{0pt}%
\pgfpathmoveto{\pgfqpoint{3.043326in}{2.141782in}}%
\pgfpathlineto{\pgfqpoint{3.306485in}{2.382499in}}%
\pgfpathlineto{\pgfqpoint{3.043326in}{2.141782in}}%
\pgfpathclose%
\pgfusepath{fill}%
\end{pgfscope}%
\begin{pgfscope}%
\pgfpathrectangle{\pgfqpoint{0.017500in}{0.017500in}}{\pgfqpoint{6.218282in}{3.580372in}} %
\pgfusepath{clip}%
\pgfsetbuttcap%
\pgfsetroundjoin%
\definecolor{currentfill}{rgb}{0.895885,0.433075,0.338681}%
\pgfsetfillcolor{currentfill}%
\pgfsetlinewidth{0.000000pt}%
\definecolor{currentstroke}{rgb}{0.000000,0.000000,0.000000}%
\pgfsetstrokecolor{currentstroke}%
\pgfsetdash{}{0pt}%
\pgfpathmoveto{\pgfqpoint{3.043326in}{2.141782in}}%
\pgfpathlineto{\pgfqpoint{3.306485in}{2.382499in}}%
\pgfpathlineto{\pgfqpoint{3.043326in}{2.141782in}}%
\pgfpathclose%
\pgfusepath{fill}%
\end{pgfscope}%
\begin{pgfscope}%
\pgfpathrectangle{\pgfqpoint{0.017500in}{0.017500in}}{\pgfqpoint{6.218282in}{3.580372in}} %
\pgfusepath{clip}%
\pgfsetbuttcap%
\pgfsetroundjoin%
\definecolor{currentfill}{rgb}{0.895885,0.433075,0.338681}%
\pgfsetfillcolor{currentfill}%
\pgfsetlinewidth{0.000000pt}%
\definecolor{currentstroke}{rgb}{0.000000,0.000000,0.000000}%
\pgfsetstrokecolor{currentstroke}%
\pgfsetdash{}{0pt}%
\pgfpathmoveto{\pgfqpoint{3.043326in}{2.141782in}}%
\pgfpathlineto{\pgfqpoint{3.306485in}{2.382499in}}%
\pgfpathlineto{\pgfqpoint{3.043326in}{2.141782in}}%
\pgfpathclose%
\pgfusepath{fill}%
\end{pgfscope}%
\begin{pgfscope}%
\pgfpathrectangle{\pgfqpoint{0.017500in}{0.017500in}}{\pgfqpoint{6.218282in}{3.580372in}} %
\pgfusepath{clip}%
\pgfsetbuttcap%
\pgfsetroundjoin%
\definecolor{currentfill}{rgb}{0.895885,0.433075,0.338681}%
\pgfsetfillcolor{currentfill}%
\pgfsetlinewidth{0.000000pt}%
\definecolor{currentstroke}{rgb}{0.000000,0.000000,0.000000}%
\pgfsetstrokecolor{currentstroke}%
\pgfsetdash{}{0pt}%
\pgfpathmoveto{\pgfqpoint{3.043326in}{2.141782in}}%
\pgfpathlineto{\pgfqpoint{3.306485in}{2.382499in}}%
\pgfpathlineto{\pgfqpoint{3.043326in}{2.141782in}}%
\pgfpathclose%
\pgfusepath{fill}%
\end{pgfscope}%
\begin{pgfscope}%
\pgfpathrectangle{\pgfqpoint{0.017500in}{0.017500in}}{\pgfqpoint{6.218282in}{3.580372in}} %
\pgfusepath{clip}%
\pgfsetbuttcap%
\pgfsetroundjoin%
\definecolor{currentfill}{rgb}{0.895885,0.433075,0.338681}%
\pgfsetfillcolor{currentfill}%
\pgfsetlinewidth{0.000000pt}%
\definecolor{currentstroke}{rgb}{0.000000,0.000000,0.000000}%
\pgfsetstrokecolor{currentstroke}%
\pgfsetdash{}{0pt}%
\pgfpathmoveto{\pgfqpoint{3.043326in}{2.141782in}}%
\pgfpathlineto{\pgfqpoint{3.306485in}{2.382499in}}%
\pgfpathlineto{\pgfqpoint{3.043326in}{2.141782in}}%
\pgfpathclose%
\pgfusepath{fill}%
\end{pgfscope}%
\begin{pgfscope}%
\pgfpathrectangle{\pgfqpoint{0.017500in}{0.017500in}}{\pgfqpoint{6.218282in}{3.580372in}} %
\pgfusepath{clip}%
\pgfsetbuttcap%
\pgfsetroundjoin%
\definecolor{currentfill}{rgb}{0.895885,0.433075,0.338681}%
\pgfsetfillcolor{currentfill}%
\pgfsetlinewidth{0.000000pt}%
\definecolor{currentstroke}{rgb}{0.000000,0.000000,0.000000}%
\pgfsetstrokecolor{currentstroke}%
\pgfsetdash{}{0pt}%
\pgfpathmoveto{\pgfqpoint{3.043326in}{2.141782in}}%
\pgfpathlineto{\pgfqpoint{3.306485in}{2.382499in}}%
\pgfpathlineto{\pgfqpoint{3.043326in}{2.141782in}}%
\pgfpathclose%
\pgfusepath{fill}%
\end{pgfscope}%
\begin{pgfscope}%
\pgfpathrectangle{\pgfqpoint{0.017500in}{0.017500in}}{\pgfqpoint{6.218282in}{3.580372in}} %
\pgfusepath{clip}%
\pgfsetbuttcap%
\pgfsetroundjoin%
\definecolor{currentfill}{rgb}{0.895885,0.433075,0.338681}%
\pgfsetfillcolor{currentfill}%
\pgfsetlinewidth{0.000000pt}%
\definecolor{currentstroke}{rgb}{0.000000,0.000000,0.000000}%
\pgfsetstrokecolor{currentstroke}%
\pgfsetdash{}{0pt}%
\pgfpathmoveto{\pgfqpoint{3.043326in}{2.141782in}}%
\pgfpathlineto{\pgfqpoint{3.306485in}{2.382499in}}%
\pgfpathlineto{\pgfqpoint{3.043326in}{2.141782in}}%
\pgfpathclose%
\pgfusepath{fill}%
\end{pgfscope}%
\begin{pgfscope}%
\pgfpathrectangle{\pgfqpoint{0.017500in}{0.017500in}}{\pgfqpoint{6.218282in}{3.580372in}} %
\pgfusepath{clip}%
\pgfsetbuttcap%
\pgfsetroundjoin%
\definecolor{currentfill}{rgb}{0.895885,0.433075,0.338681}%
\pgfsetfillcolor{currentfill}%
\pgfsetlinewidth{0.000000pt}%
\definecolor{currentstroke}{rgb}{0.000000,0.000000,0.000000}%
\pgfsetstrokecolor{currentstroke}%
\pgfsetdash{}{0pt}%
\pgfpathmoveto{\pgfqpoint{3.043326in}{2.141782in}}%
\pgfpathlineto{\pgfqpoint{3.306485in}{2.382499in}}%
\pgfpathlineto{\pgfqpoint{3.043326in}{2.141782in}}%
\pgfpathclose%
\pgfusepath{fill}%
\end{pgfscope}%
\begin{pgfscope}%
\pgfpathrectangle{\pgfqpoint{0.017500in}{0.017500in}}{\pgfqpoint{6.218282in}{3.580372in}} %
\pgfusepath{clip}%
\pgfsetbuttcap%
\pgfsetroundjoin%
\definecolor{currentfill}{rgb}{0.895885,0.433075,0.338681}%
\pgfsetfillcolor{currentfill}%
\pgfsetlinewidth{0.000000pt}%
\definecolor{currentstroke}{rgb}{0.000000,0.000000,0.000000}%
\pgfsetstrokecolor{currentstroke}%
\pgfsetdash{}{0pt}%
\pgfpathmoveto{\pgfqpoint{3.043326in}{2.141782in}}%
\pgfpathlineto{\pgfqpoint{3.306485in}{2.382499in}}%
\pgfpathlineto{\pgfqpoint{3.043326in}{2.141782in}}%
\pgfpathclose%
\pgfusepath{fill}%
\end{pgfscope}%
\begin{pgfscope}%
\pgfpathrectangle{\pgfqpoint{0.017500in}{0.017500in}}{\pgfqpoint{6.218282in}{3.580372in}} %
\pgfusepath{clip}%
\pgfsetbuttcap%
\pgfsetroundjoin%
\definecolor{currentfill}{rgb}{0.895885,0.433075,0.338681}%
\pgfsetfillcolor{currentfill}%
\pgfsetlinewidth{0.000000pt}%
\definecolor{currentstroke}{rgb}{0.000000,0.000000,0.000000}%
\pgfsetstrokecolor{currentstroke}%
\pgfsetdash{}{0pt}%
\pgfpathmoveto{\pgfqpoint{3.043326in}{2.141782in}}%
\pgfpathlineto{\pgfqpoint{3.306485in}{2.382499in}}%
\pgfpathlineto{\pgfqpoint{3.043326in}{2.141782in}}%
\pgfpathclose%
\pgfusepath{fill}%
\end{pgfscope}%
\begin{pgfscope}%
\pgfpathrectangle{\pgfqpoint{0.017500in}{0.017500in}}{\pgfqpoint{6.218282in}{3.580372in}} %
\pgfusepath{clip}%
\pgfsetbuttcap%
\pgfsetroundjoin%
\definecolor{currentfill}{rgb}{0.888390,0.417703,0.327898}%
\pgfsetfillcolor{currentfill}%
\pgfsetlinewidth{0.000000pt}%
\definecolor{currentstroke}{rgb}{0.000000,0.000000,0.000000}%
\pgfsetstrokecolor{currentstroke}%
\pgfsetdash{}{0pt}%
\pgfpathmoveto{\pgfqpoint{3.043326in}{2.141782in}}%
\pgfpathlineto{\pgfqpoint{3.502561in}{2.073396in}}%
\pgfpathlineto{\pgfqpoint{3.763398in}{2.297457in}}%
\pgfpathlineto{\pgfqpoint{3.306485in}{2.382499in}}%
\pgfpathlineto{\pgfqpoint{3.043326in}{2.141782in}}%
\pgfpathclose%
\pgfusepath{fill}%
\end{pgfscope}%
\begin{pgfscope}%
\pgfpathrectangle{\pgfqpoint{0.017500in}{0.017500in}}{\pgfqpoint{6.218282in}{3.580372in}} %
\pgfusepath{clip}%
\pgfsetbuttcap%
\pgfsetroundjoin%
\definecolor{currentfill}{rgb}{0.888390,0.417703,0.327898}%
\pgfsetfillcolor{currentfill}%
\pgfsetlinewidth{0.000000pt}%
\definecolor{currentstroke}{rgb}{0.000000,0.000000,0.000000}%
\pgfsetstrokecolor{currentstroke}%
\pgfsetdash{}{0pt}%
\pgfpathmoveto{\pgfqpoint{3.043326in}{2.141782in}}%
\pgfpathlineto{\pgfqpoint{3.502561in}{2.073396in}}%
\pgfpathlineto{\pgfqpoint{3.763398in}{2.297457in}}%
\pgfpathlineto{\pgfqpoint{3.306485in}{2.382499in}}%
\pgfpathlineto{\pgfqpoint{3.043326in}{2.141782in}}%
\pgfpathclose%
\pgfusepath{fill}%
\end{pgfscope}%
\begin{pgfscope}%
\pgfpathrectangle{\pgfqpoint{0.017500in}{0.017500in}}{\pgfqpoint{6.218282in}{3.580372in}} %
\pgfusepath{clip}%
\pgfsetbuttcap%
\pgfsetroundjoin%
\definecolor{currentfill}{rgb}{0.888390,0.417703,0.327898}%
\pgfsetfillcolor{currentfill}%
\pgfsetlinewidth{0.000000pt}%
\definecolor{currentstroke}{rgb}{0.000000,0.000000,0.000000}%
\pgfsetstrokecolor{currentstroke}%
\pgfsetdash{}{0pt}%
\pgfpathmoveto{\pgfqpoint{3.043326in}{2.141782in}}%
\pgfpathlineto{\pgfqpoint{3.502561in}{2.073396in}}%
\pgfpathlineto{\pgfqpoint{3.763398in}{2.297457in}}%
\pgfpathlineto{\pgfqpoint{3.306485in}{2.382499in}}%
\pgfpathlineto{\pgfqpoint{3.043326in}{2.141782in}}%
\pgfpathclose%
\pgfusepath{fill}%
\end{pgfscope}%
\begin{pgfscope}%
\pgfpathrectangle{\pgfqpoint{0.017500in}{0.017500in}}{\pgfqpoint{6.218282in}{3.580372in}} %
\pgfusepath{clip}%
\pgfsetbuttcap%
\pgfsetroundjoin%
\definecolor{currentfill}{rgb}{0.888390,0.417703,0.327898}%
\pgfsetfillcolor{currentfill}%
\pgfsetlinewidth{0.000000pt}%
\definecolor{currentstroke}{rgb}{0.000000,0.000000,0.000000}%
\pgfsetstrokecolor{currentstroke}%
\pgfsetdash{}{0pt}%
\pgfpathmoveto{\pgfqpoint{3.043326in}{2.141782in}}%
\pgfpathlineto{\pgfqpoint{3.502561in}{2.073396in}}%
\pgfpathlineto{\pgfqpoint{3.763398in}{2.297457in}}%
\pgfpathlineto{\pgfqpoint{3.306485in}{2.382499in}}%
\pgfpathlineto{\pgfqpoint{3.043326in}{2.141782in}}%
\pgfpathclose%
\pgfusepath{fill}%
\end{pgfscope}%
\begin{pgfscope}%
\pgfpathrectangle{\pgfqpoint{0.017500in}{0.017500in}}{\pgfqpoint{6.218282in}{3.580372in}} %
\pgfusepath{clip}%
\pgfsetbuttcap%
\pgfsetroundjoin%
\definecolor{currentfill}{rgb}{0.888390,0.417703,0.327898}%
\pgfsetfillcolor{currentfill}%
\pgfsetlinewidth{0.000000pt}%
\definecolor{currentstroke}{rgb}{0.000000,0.000000,0.000000}%
\pgfsetstrokecolor{currentstroke}%
\pgfsetdash{}{0pt}%
\pgfpathmoveto{\pgfqpoint{3.043326in}{2.141782in}}%
\pgfpathlineto{\pgfqpoint{3.502561in}{2.073396in}}%
\pgfpathlineto{\pgfqpoint{3.763398in}{2.297457in}}%
\pgfpathlineto{\pgfqpoint{3.306485in}{2.382499in}}%
\pgfpathlineto{\pgfqpoint{3.043326in}{2.141782in}}%
\pgfpathclose%
\pgfusepath{fill}%
\end{pgfscope}%
\begin{pgfscope}%
\pgfpathrectangle{\pgfqpoint{0.017500in}{0.017500in}}{\pgfqpoint{6.218282in}{3.580372in}} %
\pgfusepath{clip}%
\pgfsetbuttcap%
\pgfsetroundjoin%
\definecolor{currentfill}{rgb}{0.888390,0.417703,0.327898}%
\pgfsetfillcolor{currentfill}%
\pgfsetlinewidth{0.000000pt}%
\definecolor{currentstroke}{rgb}{0.000000,0.000000,0.000000}%
\pgfsetstrokecolor{currentstroke}%
\pgfsetdash{}{0pt}%
\pgfpathmoveto{\pgfqpoint{3.043326in}{2.141782in}}%
\pgfpathlineto{\pgfqpoint{3.502561in}{2.073396in}}%
\pgfpathlineto{\pgfqpoint{3.763398in}{2.297457in}}%
\pgfpathlineto{\pgfqpoint{3.306485in}{2.382499in}}%
\pgfpathlineto{\pgfqpoint{3.043326in}{2.141782in}}%
\pgfpathclose%
\pgfusepath{fill}%
\end{pgfscope}%
\begin{pgfscope}%
\pgfpathrectangle{\pgfqpoint{0.017500in}{0.017500in}}{\pgfqpoint{6.218282in}{3.580372in}} %
\pgfusepath{clip}%
\pgfsetbuttcap%
\pgfsetroundjoin%
\definecolor{currentfill}{rgb}{0.800830,0.250829,0.225696}%
\pgfsetfillcolor{currentfill}%
\pgfsetlinewidth{0.000000pt}%
\definecolor{currentstroke}{rgb}{0.000000,0.000000,0.000000}%
\pgfsetstrokecolor{currentstroke}%
\pgfsetdash{}{0pt}%
\pgfpathmoveto{\pgfqpoint{4.226975in}{2.236848in}}%
\pgfpathlineto{\pgfqpoint{4.478468in}{2.393972in}}%
\pgfpathlineto{\pgfqpoint{4.226975in}{2.236848in}}%
\pgfpathclose%
\pgfusepath{fill}%
\end{pgfscope}%
\begin{pgfscope}%
\pgfpathrectangle{\pgfqpoint{0.017500in}{0.017500in}}{\pgfqpoint{6.218282in}{3.580372in}} %
\pgfusepath{clip}%
\pgfsetbuttcap%
\pgfsetroundjoin%
\definecolor{currentfill}{rgb}{0.800830,0.250829,0.225696}%
\pgfsetfillcolor{currentfill}%
\pgfsetlinewidth{0.000000pt}%
\definecolor{currentstroke}{rgb}{0.000000,0.000000,0.000000}%
\pgfsetstrokecolor{currentstroke}%
\pgfsetdash{}{0pt}%
\pgfpathmoveto{\pgfqpoint{4.226975in}{2.236848in}}%
\pgfpathlineto{\pgfqpoint{4.478468in}{2.393972in}}%
\pgfpathlineto{\pgfqpoint{4.226975in}{2.236848in}}%
\pgfpathclose%
\pgfusepath{fill}%
\end{pgfscope}%
\begin{pgfscope}%
\pgfpathrectangle{\pgfqpoint{0.017500in}{0.017500in}}{\pgfqpoint{6.218282in}{3.580372in}} %
\pgfusepath{clip}%
\pgfsetbuttcap%
\pgfsetroundjoin%
\definecolor{currentfill}{rgb}{0.800830,0.250829,0.225696}%
\pgfsetfillcolor{currentfill}%
\pgfsetlinewidth{0.000000pt}%
\definecolor{currentstroke}{rgb}{0.000000,0.000000,0.000000}%
\pgfsetstrokecolor{currentstroke}%
\pgfsetdash{}{0pt}%
\pgfpathmoveto{\pgfqpoint{4.226975in}{2.236848in}}%
\pgfpathlineto{\pgfqpoint{4.478468in}{2.393972in}}%
\pgfpathlineto{\pgfqpoint{4.226975in}{2.236848in}}%
\pgfpathclose%
\pgfusepath{fill}%
\end{pgfscope}%
\begin{pgfscope}%
\pgfpathrectangle{\pgfqpoint{0.017500in}{0.017500in}}{\pgfqpoint{6.218282in}{3.580372in}} %
\pgfusepath{clip}%
\pgfsetbuttcap%
\pgfsetroundjoin%
\definecolor{currentfill}{rgb}{0.800830,0.250829,0.225696}%
\pgfsetfillcolor{currentfill}%
\pgfsetlinewidth{0.000000pt}%
\definecolor{currentstroke}{rgb}{0.000000,0.000000,0.000000}%
\pgfsetstrokecolor{currentstroke}%
\pgfsetdash{}{0pt}%
\pgfpathmoveto{\pgfqpoint{4.226975in}{2.236848in}}%
\pgfpathlineto{\pgfqpoint{4.478468in}{2.393972in}}%
\pgfpathlineto{\pgfqpoint{4.226975in}{2.236848in}}%
\pgfpathclose%
\pgfusepath{fill}%
\end{pgfscope}%
\begin{pgfscope}%
\pgfpathrectangle{\pgfqpoint{0.017500in}{0.017500in}}{\pgfqpoint{6.218282in}{3.580372in}} %
\pgfusepath{clip}%
\pgfsetbuttcap%
\pgfsetroundjoin%
\definecolor{currentfill}{rgb}{0.800830,0.250829,0.225696}%
\pgfsetfillcolor{currentfill}%
\pgfsetlinewidth{0.000000pt}%
\definecolor{currentstroke}{rgb}{0.000000,0.000000,0.000000}%
\pgfsetstrokecolor{currentstroke}%
\pgfsetdash{}{0pt}%
\pgfpathmoveto{\pgfqpoint{4.226975in}{2.236848in}}%
\pgfpathlineto{\pgfqpoint{4.478468in}{2.393972in}}%
\pgfpathlineto{\pgfqpoint{4.226975in}{2.236848in}}%
\pgfpathclose%
\pgfusepath{fill}%
\end{pgfscope}%
\begin{pgfscope}%
\pgfpathrectangle{\pgfqpoint{0.017500in}{0.017500in}}{\pgfqpoint{6.218282in}{3.580372in}} %
\pgfusepath{clip}%
\pgfsetbuttcap%
\pgfsetroundjoin%
\definecolor{currentfill}{rgb}{0.800830,0.250829,0.225696}%
\pgfsetfillcolor{currentfill}%
\pgfsetlinewidth{0.000000pt}%
\definecolor{currentstroke}{rgb}{0.000000,0.000000,0.000000}%
\pgfsetstrokecolor{currentstroke}%
\pgfsetdash{}{0pt}%
\pgfpathmoveto{\pgfqpoint{4.226975in}{2.236848in}}%
\pgfpathlineto{\pgfqpoint{4.478468in}{2.393972in}}%
\pgfpathlineto{\pgfqpoint{4.226975in}{2.236848in}}%
\pgfpathclose%
\pgfusepath{fill}%
\end{pgfscope}%
\begin{pgfscope}%
\pgfpathrectangle{\pgfqpoint{0.017500in}{0.017500in}}{\pgfqpoint{6.218282in}{3.580372in}} %
\pgfusepath{clip}%
\pgfsetbuttcap%
\pgfsetroundjoin%
\definecolor{currentfill}{rgb}{0.800830,0.250829,0.225696}%
\pgfsetfillcolor{currentfill}%
\pgfsetlinewidth{0.000000pt}%
\definecolor{currentstroke}{rgb}{0.000000,0.000000,0.000000}%
\pgfsetstrokecolor{currentstroke}%
\pgfsetdash{}{0pt}%
\pgfpathmoveto{\pgfqpoint{4.226975in}{2.236848in}}%
\pgfpathlineto{\pgfqpoint{4.478468in}{2.393972in}}%
\pgfpathlineto{\pgfqpoint{4.226975in}{2.236848in}}%
\pgfpathclose%
\pgfusepath{fill}%
\end{pgfscope}%
\begin{pgfscope}%
\pgfpathrectangle{\pgfqpoint{0.017500in}{0.017500in}}{\pgfqpoint{6.218282in}{3.580372in}} %
\pgfusepath{clip}%
\pgfsetbuttcap%
\pgfsetroundjoin%
\definecolor{currentfill}{rgb}{0.800830,0.250829,0.225696}%
\pgfsetfillcolor{currentfill}%
\pgfsetlinewidth{0.000000pt}%
\definecolor{currentstroke}{rgb}{0.000000,0.000000,0.000000}%
\pgfsetstrokecolor{currentstroke}%
\pgfsetdash{}{0pt}%
\pgfpathmoveto{\pgfqpoint{4.226975in}{2.236848in}}%
\pgfpathlineto{\pgfqpoint{4.478468in}{2.393972in}}%
\pgfpathlineto{\pgfqpoint{4.226975in}{2.236848in}}%
\pgfpathclose%
\pgfusepath{fill}%
\end{pgfscope}%
\begin{pgfscope}%
\pgfpathrectangle{\pgfqpoint{0.017500in}{0.017500in}}{\pgfqpoint{6.218282in}{3.580372in}} %
\pgfusepath{clip}%
\pgfsetbuttcap%
\pgfsetroundjoin%
\definecolor{currentfill}{rgb}{0.800830,0.250829,0.225696}%
\pgfsetfillcolor{currentfill}%
\pgfsetlinewidth{0.000000pt}%
\definecolor{currentstroke}{rgb}{0.000000,0.000000,0.000000}%
\pgfsetstrokecolor{currentstroke}%
\pgfsetdash{}{0pt}%
\pgfpathmoveto{\pgfqpoint{4.226975in}{2.236848in}}%
\pgfpathlineto{\pgfqpoint{4.478468in}{2.393972in}}%
\pgfpathlineto{\pgfqpoint{4.226975in}{2.236848in}}%
\pgfpathclose%
\pgfusepath{fill}%
\end{pgfscope}%
\begin{pgfscope}%
\pgfpathrectangle{\pgfqpoint{0.017500in}{0.017500in}}{\pgfqpoint{6.218282in}{3.580372in}} %
\pgfusepath{clip}%
\pgfsetbuttcap%
\pgfsetroundjoin%
\definecolor{currentfill}{rgb}{0.800830,0.250829,0.225696}%
\pgfsetfillcolor{currentfill}%
\pgfsetlinewidth{0.000000pt}%
\definecolor{currentstroke}{rgb}{0.000000,0.000000,0.000000}%
\pgfsetstrokecolor{currentstroke}%
\pgfsetdash{}{0pt}%
\pgfpathmoveto{\pgfqpoint{4.226975in}{2.236848in}}%
\pgfpathlineto{\pgfqpoint{4.478468in}{2.393972in}}%
\pgfpathlineto{\pgfqpoint{4.226975in}{2.236848in}}%
\pgfpathclose%
\pgfusepath{fill}%
\end{pgfscope}%
\begin{pgfscope}%
\pgfpathrectangle{\pgfqpoint{0.017500in}{0.017500in}}{\pgfqpoint{6.218282in}{3.580372in}} %
\pgfusepath{clip}%
\pgfsetbuttcap%
\pgfsetroundjoin%
\definecolor{currentfill}{rgb}{0.800830,0.250829,0.225696}%
\pgfsetfillcolor{currentfill}%
\pgfsetlinewidth{0.000000pt}%
\definecolor{currentstroke}{rgb}{0.000000,0.000000,0.000000}%
\pgfsetstrokecolor{currentstroke}%
\pgfsetdash{}{0pt}%
\pgfpathmoveto{\pgfqpoint{4.226975in}{2.236848in}}%
\pgfpathlineto{\pgfqpoint{4.478468in}{2.393972in}}%
\pgfpathlineto{\pgfqpoint{4.226975in}{2.236848in}}%
\pgfpathclose%
\pgfusepath{fill}%
\end{pgfscope}%
\begin{pgfscope}%
\pgfpathrectangle{\pgfqpoint{0.017500in}{0.017500in}}{\pgfqpoint{6.218282in}{3.580372in}} %
\pgfusepath{clip}%
\pgfsetbuttcap%
\pgfsetroundjoin%
\definecolor{currentfill}{rgb}{0.800830,0.250829,0.225696}%
\pgfsetfillcolor{currentfill}%
\pgfsetlinewidth{0.000000pt}%
\definecolor{currentstroke}{rgb}{0.000000,0.000000,0.000000}%
\pgfsetstrokecolor{currentstroke}%
\pgfsetdash{}{0pt}%
\pgfpathmoveto{\pgfqpoint{4.226975in}{2.236848in}}%
\pgfpathlineto{\pgfqpoint{4.478468in}{2.393972in}}%
\pgfpathlineto{\pgfqpoint{4.226975in}{2.236848in}}%
\pgfpathclose%
\pgfusepath{fill}%
\end{pgfscope}%
\begin{pgfscope}%
\pgfpathrectangle{\pgfqpoint{0.017500in}{0.017500in}}{\pgfqpoint{6.218282in}{3.580372in}} %
\pgfusepath{clip}%
\pgfsetbuttcap%
\pgfsetroundjoin%
\definecolor{currentfill}{rgb}{0.800830,0.250829,0.225696}%
\pgfsetfillcolor{currentfill}%
\pgfsetlinewidth{0.000000pt}%
\definecolor{currentstroke}{rgb}{0.000000,0.000000,0.000000}%
\pgfsetstrokecolor{currentstroke}%
\pgfsetdash{}{0pt}%
\pgfpathmoveto{\pgfqpoint{4.226975in}{2.236848in}}%
\pgfpathlineto{\pgfqpoint{4.478468in}{2.393972in}}%
\pgfpathlineto{\pgfqpoint{4.226975in}{2.236848in}}%
\pgfpathclose%
\pgfusepath{fill}%
\end{pgfscope}%
\begin{pgfscope}%
\pgfpathrectangle{\pgfqpoint{0.017500in}{0.017500in}}{\pgfqpoint{6.218282in}{3.580372in}} %
\pgfusepath{clip}%
\pgfsetbuttcap%
\pgfsetroundjoin%
\definecolor{currentfill}{rgb}{0.800830,0.250829,0.225696}%
\pgfsetfillcolor{currentfill}%
\pgfsetlinewidth{0.000000pt}%
\definecolor{currentstroke}{rgb}{0.000000,0.000000,0.000000}%
\pgfsetstrokecolor{currentstroke}%
\pgfsetdash{}{0pt}%
\pgfpathmoveto{\pgfqpoint{4.226975in}{2.236848in}}%
\pgfpathlineto{\pgfqpoint{4.478468in}{2.393972in}}%
\pgfpathlineto{\pgfqpoint{4.226975in}{2.236848in}}%
\pgfpathclose%
\pgfusepath{fill}%
\end{pgfscope}%
\begin{pgfscope}%
\pgfpathrectangle{\pgfqpoint{0.017500in}{0.017500in}}{\pgfqpoint{6.218282in}{3.580372in}} %
\pgfusepath{clip}%
\pgfsetbuttcap%
\pgfsetroundjoin%
\definecolor{currentfill}{rgb}{0.800830,0.250829,0.225696}%
\pgfsetfillcolor{currentfill}%
\pgfsetlinewidth{0.000000pt}%
\definecolor{currentstroke}{rgb}{0.000000,0.000000,0.000000}%
\pgfsetstrokecolor{currentstroke}%
\pgfsetdash{}{0pt}%
\pgfpathmoveto{\pgfqpoint{4.226975in}{2.236848in}}%
\pgfpathlineto{\pgfqpoint{4.478468in}{2.393972in}}%
\pgfpathlineto{\pgfqpoint{4.226975in}{2.236848in}}%
\pgfpathclose%
\pgfusepath{fill}%
\end{pgfscope}%
\begin{pgfscope}%
\pgfpathrectangle{\pgfqpoint{0.017500in}{0.017500in}}{\pgfqpoint{6.218282in}{3.580372in}} %
\pgfusepath{clip}%
\pgfsetbuttcap%
\pgfsetroundjoin%
\definecolor{currentfill}{rgb}{0.800830,0.250829,0.225696}%
\pgfsetfillcolor{currentfill}%
\pgfsetlinewidth{0.000000pt}%
\definecolor{currentstroke}{rgb}{0.000000,0.000000,0.000000}%
\pgfsetstrokecolor{currentstroke}%
\pgfsetdash{}{0pt}%
\pgfpathmoveto{\pgfqpoint{4.226975in}{2.236848in}}%
\pgfpathlineto{\pgfqpoint{4.478468in}{2.393972in}}%
\pgfpathlineto{\pgfqpoint{4.226975in}{2.236848in}}%
\pgfpathclose%
\pgfusepath{fill}%
\end{pgfscope}%
\begin{pgfscope}%
\pgfpathrectangle{\pgfqpoint{0.017500in}{0.017500in}}{\pgfqpoint{6.218282in}{3.580372in}} %
\pgfusepath{clip}%
\pgfsetbuttcap%
\pgfsetroundjoin%
\definecolor{currentfill}{rgb}{0.800830,0.250829,0.225696}%
\pgfsetfillcolor{currentfill}%
\pgfsetlinewidth{0.000000pt}%
\definecolor{currentstroke}{rgb}{0.000000,0.000000,0.000000}%
\pgfsetstrokecolor{currentstroke}%
\pgfsetdash{}{0pt}%
\pgfpathmoveto{\pgfqpoint{4.226975in}{2.236848in}}%
\pgfpathlineto{\pgfqpoint{4.478468in}{2.393972in}}%
\pgfpathlineto{\pgfqpoint{4.226975in}{2.236848in}}%
\pgfpathclose%
\pgfusepath{fill}%
\end{pgfscope}%
\begin{pgfscope}%
\pgfpathrectangle{\pgfqpoint{0.017500in}{0.017500in}}{\pgfqpoint{6.218282in}{3.580372in}} %
\pgfusepath{clip}%
\pgfsetbuttcap%
\pgfsetroundjoin%
\definecolor{currentfill}{rgb}{0.800830,0.250829,0.225696}%
\pgfsetfillcolor{currentfill}%
\pgfsetlinewidth{0.000000pt}%
\definecolor{currentstroke}{rgb}{0.000000,0.000000,0.000000}%
\pgfsetstrokecolor{currentstroke}%
\pgfsetdash{}{0pt}%
\pgfpathmoveto{\pgfqpoint{4.226975in}{2.236848in}}%
\pgfpathlineto{\pgfqpoint{4.478468in}{2.393972in}}%
\pgfpathlineto{\pgfqpoint{4.226975in}{2.236848in}}%
\pgfpathclose%
\pgfusepath{fill}%
\end{pgfscope}%
\begin{pgfscope}%
\pgfpathrectangle{\pgfqpoint{0.017500in}{0.017500in}}{\pgfqpoint{6.218282in}{3.580372in}} %
\pgfusepath{clip}%
\pgfsetbuttcap%
\pgfsetroundjoin%
\definecolor{currentfill}{rgb}{0.800830,0.250829,0.225696}%
\pgfsetfillcolor{currentfill}%
\pgfsetlinewidth{0.000000pt}%
\definecolor{currentstroke}{rgb}{0.000000,0.000000,0.000000}%
\pgfsetstrokecolor{currentstroke}%
\pgfsetdash{}{0pt}%
\pgfpathmoveto{\pgfqpoint{4.226975in}{2.236848in}}%
\pgfpathlineto{\pgfqpoint{4.478468in}{2.393972in}}%
\pgfpathlineto{\pgfqpoint{4.226975in}{2.236848in}}%
\pgfpathclose%
\pgfusepath{fill}%
\end{pgfscope}%
\begin{pgfscope}%
\pgfpathrectangle{\pgfqpoint{0.017500in}{0.017500in}}{\pgfqpoint{6.218282in}{3.580372in}} %
\pgfusepath{clip}%
\pgfsetbuttcap%
\pgfsetroundjoin%
\definecolor{currentfill}{rgb}{0.800830,0.250829,0.225696}%
\pgfsetfillcolor{currentfill}%
\pgfsetlinewidth{0.000000pt}%
\definecolor{currentstroke}{rgb}{0.000000,0.000000,0.000000}%
\pgfsetstrokecolor{currentstroke}%
\pgfsetdash{}{0pt}%
\pgfpathmoveto{\pgfqpoint{4.226975in}{2.236848in}}%
\pgfpathlineto{\pgfqpoint{4.478468in}{2.393972in}}%
\pgfpathlineto{\pgfqpoint{4.226975in}{2.236848in}}%
\pgfpathclose%
\pgfusepath{fill}%
\end{pgfscope}%
\begin{pgfscope}%
\pgfpathrectangle{\pgfqpoint{0.017500in}{0.017500in}}{\pgfqpoint{6.218282in}{3.580372in}} %
\pgfusepath{clip}%
\pgfsetbuttcap%
\pgfsetroundjoin%
\definecolor{currentfill}{rgb}{0.800830,0.250829,0.225696}%
\pgfsetfillcolor{currentfill}%
\pgfsetlinewidth{0.000000pt}%
\definecolor{currentstroke}{rgb}{0.000000,0.000000,0.000000}%
\pgfsetstrokecolor{currentstroke}%
\pgfsetdash{}{0pt}%
\pgfpathmoveto{\pgfqpoint{4.226975in}{2.236848in}}%
\pgfpathlineto{\pgfqpoint{4.478468in}{2.393972in}}%
\pgfpathlineto{\pgfqpoint{4.226975in}{2.236848in}}%
\pgfpathclose%
\pgfusepath{fill}%
\end{pgfscope}%
\begin{pgfscope}%
\pgfpathrectangle{\pgfqpoint{0.017500in}{0.017500in}}{\pgfqpoint{6.218282in}{3.580372in}} %
\pgfusepath{clip}%
\pgfsetbuttcap%
\pgfsetroundjoin%
\definecolor{currentfill}{rgb}{0.800830,0.250829,0.225696}%
\pgfsetfillcolor{currentfill}%
\pgfsetlinewidth{0.000000pt}%
\definecolor{currentstroke}{rgb}{0.000000,0.000000,0.000000}%
\pgfsetstrokecolor{currentstroke}%
\pgfsetdash{}{0pt}%
\pgfpathmoveto{\pgfqpoint{4.226975in}{2.236848in}}%
\pgfpathlineto{\pgfqpoint{4.478468in}{2.393972in}}%
\pgfpathlineto{\pgfqpoint{4.226975in}{2.236848in}}%
\pgfpathclose%
\pgfusepath{fill}%
\end{pgfscope}%
\begin{pgfscope}%
\pgfpathrectangle{\pgfqpoint{0.017500in}{0.017500in}}{\pgfqpoint{6.218282in}{3.580372in}} %
\pgfusepath{clip}%
\pgfsetbuttcap%
\pgfsetroundjoin%
\definecolor{currentfill}{rgb}{0.800830,0.250829,0.225696}%
\pgfsetfillcolor{currentfill}%
\pgfsetlinewidth{0.000000pt}%
\definecolor{currentstroke}{rgb}{0.000000,0.000000,0.000000}%
\pgfsetstrokecolor{currentstroke}%
\pgfsetdash{}{0pt}%
\pgfpathmoveto{\pgfqpoint{4.226975in}{2.236848in}}%
\pgfpathlineto{\pgfqpoint{4.478468in}{2.393972in}}%
\pgfpathlineto{\pgfqpoint{4.226975in}{2.236848in}}%
\pgfpathclose%
\pgfusepath{fill}%
\end{pgfscope}%
\begin{pgfscope}%
\pgfpathrectangle{\pgfqpoint{0.017500in}{0.017500in}}{\pgfqpoint{6.218282in}{3.580372in}} %
\pgfusepath{clip}%
\pgfsetbuttcap%
\pgfsetroundjoin%
\definecolor{currentfill}{rgb}{0.800830,0.250829,0.225696}%
\pgfsetfillcolor{currentfill}%
\pgfsetlinewidth{0.000000pt}%
\definecolor{currentstroke}{rgb}{0.000000,0.000000,0.000000}%
\pgfsetstrokecolor{currentstroke}%
\pgfsetdash{}{0pt}%
\pgfpathmoveto{\pgfqpoint{4.226975in}{2.236848in}}%
\pgfpathlineto{\pgfqpoint{4.478468in}{2.393972in}}%
\pgfpathlineto{\pgfqpoint{4.226975in}{2.236848in}}%
\pgfpathclose%
\pgfusepath{fill}%
\end{pgfscope}%
\begin{pgfscope}%
\pgfpathrectangle{\pgfqpoint{0.017500in}{0.017500in}}{\pgfqpoint{6.218282in}{3.580372in}} %
\pgfusepath{clip}%
\pgfsetbuttcap%
\pgfsetroundjoin%
\definecolor{currentfill}{rgb}{0.800830,0.250829,0.225696}%
\pgfsetfillcolor{currentfill}%
\pgfsetlinewidth{0.000000pt}%
\definecolor{currentstroke}{rgb}{0.000000,0.000000,0.000000}%
\pgfsetstrokecolor{currentstroke}%
\pgfsetdash{}{0pt}%
\pgfpathmoveto{\pgfqpoint{4.226975in}{2.236848in}}%
\pgfpathlineto{\pgfqpoint{4.478468in}{2.393972in}}%
\pgfpathlineto{\pgfqpoint{4.226975in}{2.236848in}}%
\pgfpathclose%
\pgfusepath{fill}%
\end{pgfscope}%
\begin{pgfscope}%
\pgfpathrectangle{\pgfqpoint{0.017500in}{0.017500in}}{\pgfqpoint{6.218282in}{3.580372in}} %
\pgfusepath{clip}%
\pgfsetbuttcap%
\pgfsetroundjoin%
\definecolor{currentfill}{rgb}{0.800830,0.250829,0.225696}%
\pgfsetfillcolor{currentfill}%
\pgfsetlinewidth{0.000000pt}%
\definecolor{currentstroke}{rgb}{0.000000,0.000000,0.000000}%
\pgfsetstrokecolor{currentstroke}%
\pgfsetdash{}{0pt}%
\pgfpathmoveto{\pgfqpoint{4.226975in}{2.236848in}}%
\pgfpathlineto{\pgfqpoint{4.478468in}{2.393972in}}%
\pgfpathlineto{\pgfqpoint{4.226975in}{2.236848in}}%
\pgfpathclose%
\pgfusepath{fill}%
\end{pgfscope}%
\begin{pgfscope}%
\pgfpathrectangle{\pgfqpoint{0.017500in}{0.017500in}}{\pgfqpoint{6.218282in}{3.580372in}} %
\pgfusepath{clip}%
\pgfsetbuttcap%
\pgfsetroundjoin%
\definecolor{currentfill}{rgb}{0.800830,0.250829,0.225696}%
\pgfsetfillcolor{currentfill}%
\pgfsetlinewidth{0.000000pt}%
\definecolor{currentstroke}{rgb}{0.000000,0.000000,0.000000}%
\pgfsetstrokecolor{currentstroke}%
\pgfsetdash{}{0pt}%
\pgfpathmoveto{\pgfqpoint{4.226975in}{2.236848in}}%
\pgfpathlineto{\pgfqpoint{4.478468in}{2.393972in}}%
\pgfpathlineto{\pgfqpoint{4.226975in}{2.236848in}}%
\pgfpathclose%
\pgfusepath{fill}%
\end{pgfscope}%
\begin{pgfscope}%
\pgfpathrectangle{\pgfqpoint{0.017500in}{0.017500in}}{\pgfqpoint{6.218282in}{3.580372in}} %
\pgfusepath{clip}%
\pgfsetbuttcap%
\pgfsetroundjoin%
\definecolor{currentfill}{rgb}{0.800830,0.250829,0.225696}%
\pgfsetfillcolor{currentfill}%
\pgfsetlinewidth{0.000000pt}%
\definecolor{currentstroke}{rgb}{0.000000,0.000000,0.000000}%
\pgfsetstrokecolor{currentstroke}%
\pgfsetdash{}{0pt}%
\pgfpathmoveto{\pgfqpoint{4.226975in}{2.236848in}}%
\pgfpathlineto{\pgfqpoint{4.478468in}{2.393972in}}%
\pgfpathlineto{\pgfqpoint{4.226975in}{2.236848in}}%
\pgfpathclose%
\pgfusepath{fill}%
\end{pgfscope}%
\begin{pgfscope}%
\pgfpathrectangle{\pgfqpoint{0.017500in}{0.017500in}}{\pgfqpoint{6.218282in}{3.580372in}} %
\pgfusepath{clip}%
\pgfsetbuttcap%
\pgfsetroundjoin%
\definecolor{currentfill}{rgb}{0.800830,0.250829,0.225696}%
\pgfsetfillcolor{currentfill}%
\pgfsetlinewidth{0.000000pt}%
\definecolor{currentstroke}{rgb}{0.000000,0.000000,0.000000}%
\pgfsetstrokecolor{currentstroke}%
\pgfsetdash{}{0pt}%
\pgfpathmoveto{\pgfqpoint{4.226975in}{2.236848in}}%
\pgfpathlineto{\pgfqpoint{4.478468in}{2.393972in}}%
\pgfpathlineto{\pgfqpoint{4.226975in}{2.236848in}}%
\pgfpathclose%
\pgfusepath{fill}%
\end{pgfscope}%
\begin{pgfscope}%
\pgfpathrectangle{\pgfqpoint{0.017500in}{0.017500in}}{\pgfqpoint{6.218282in}{3.580372in}} %
\pgfusepath{clip}%
\pgfsetbuttcap%
\pgfsetroundjoin%
\definecolor{currentfill}{rgb}{0.800830,0.250829,0.225696}%
\pgfsetfillcolor{currentfill}%
\pgfsetlinewidth{0.000000pt}%
\definecolor{currentstroke}{rgb}{0.000000,0.000000,0.000000}%
\pgfsetstrokecolor{currentstroke}%
\pgfsetdash{}{0pt}%
\pgfpathmoveto{\pgfqpoint{4.226975in}{2.236848in}}%
\pgfpathlineto{\pgfqpoint{4.478468in}{2.393972in}}%
\pgfpathlineto{\pgfqpoint{4.226975in}{2.236848in}}%
\pgfpathclose%
\pgfusepath{fill}%
\end{pgfscope}%
\begin{pgfscope}%
\pgfpathrectangle{\pgfqpoint{0.017500in}{0.017500in}}{\pgfqpoint{6.218282in}{3.580372in}} %
\pgfusepath{clip}%
\pgfsetbuttcap%
\pgfsetroundjoin%
\definecolor{currentfill}{rgb}{0.895885,0.433075,0.338681}%
\pgfsetfillcolor{currentfill}%
\pgfsetlinewidth{0.000000pt}%
\definecolor{currentstroke}{rgb}{0.000000,0.000000,0.000000}%
\pgfsetstrokecolor{currentstroke}%
\pgfsetdash{}{0pt}%
\pgfpathmoveto{\pgfqpoint{3.502561in}{2.073396in}}%
\pgfpathlineto{\pgfqpoint{3.763398in}{2.297457in}}%
\pgfpathlineto{\pgfqpoint{3.502561in}{2.073396in}}%
\pgfpathclose%
\pgfusepath{fill}%
\end{pgfscope}%
\begin{pgfscope}%
\pgfpathrectangle{\pgfqpoint{0.017500in}{0.017500in}}{\pgfqpoint{6.218282in}{3.580372in}} %
\pgfusepath{clip}%
\pgfsetbuttcap%
\pgfsetroundjoin%
\definecolor{currentfill}{rgb}{0.895885,0.433075,0.338681}%
\pgfsetfillcolor{currentfill}%
\pgfsetlinewidth{0.000000pt}%
\definecolor{currentstroke}{rgb}{0.000000,0.000000,0.000000}%
\pgfsetstrokecolor{currentstroke}%
\pgfsetdash{}{0pt}%
\pgfpathmoveto{\pgfqpoint{3.502561in}{2.073396in}}%
\pgfpathlineto{\pgfqpoint{3.763398in}{2.297457in}}%
\pgfpathlineto{\pgfqpoint{3.502561in}{2.073396in}}%
\pgfpathclose%
\pgfusepath{fill}%
\end{pgfscope}%
\begin{pgfscope}%
\pgfpathrectangle{\pgfqpoint{0.017500in}{0.017500in}}{\pgfqpoint{6.218282in}{3.580372in}} %
\pgfusepath{clip}%
\pgfsetbuttcap%
\pgfsetroundjoin%
\definecolor{currentfill}{rgb}{0.895885,0.433075,0.338681}%
\pgfsetfillcolor{currentfill}%
\pgfsetlinewidth{0.000000pt}%
\definecolor{currentstroke}{rgb}{0.000000,0.000000,0.000000}%
\pgfsetstrokecolor{currentstroke}%
\pgfsetdash{}{0pt}%
\pgfpathmoveto{\pgfqpoint{3.502561in}{2.073396in}}%
\pgfpathlineto{\pgfqpoint{3.763398in}{2.297457in}}%
\pgfpathlineto{\pgfqpoint{3.502561in}{2.073396in}}%
\pgfpathclose%
\pgfusepath{fill}%
\end{pgfscope}%
\begin{pgfscope}%
\pgfpathrectangle{\pgfqpoint{0.017500in}{0.017500in}}{\pgfqpoint{6.218282in}{3.580372in}} %
\pgfusepath{clip}%
\pgfsetbuttcap%
\pgfsetroundjoin%
\definecolor{currentfill}{rgb}{0.895885,0.433075,0.338681}%
\pgfsetfillcolor{currentfill}%
\pgfsetlinewidth{0.000000pt}%
\definecolor{currentstroke}{rgb}{0.000000,0.000000,0.000000}%
\pgfsetstrokecolor{currentstroke}%
\pgfsetdash{}{0pt}%
\pgfpathmoveto{\pgfqpoint{3.502561in}{2.073396in}}%
\pgfpathlineto{\pgfqpoint{3.763398in}{2.297457in}}%
\pgfpathlineto{\pgfqpoint{3.502561in}{2.073396in}}%
\pgfpathclose%
\pgfusepath{fill}%
\end{pgfscope}%
\begin{pgfscope}%
\pgfpathrectangle{\pgfqpoint{0.017500in}{0.017500in}}{\pgfqpoint{6.218282in}{3.580372in}} %
\pgfusepath{clip}%
\pgfsetbuttcap%
\pgfsetroundjoin%
\definecolor{currentfill}{rgb}{0.895885,0.433075,0.338681}%
\pgfsetfillcolor{currentfill}%
\pgfsetlinewidth{0.000000pt}%
\definecolor{currentstroke}{rgb}{0.000000,0.000000,0.000000}%
\pgfsetstrokecolor{currentstroke}%
\pgfsetdash{}{0pt}%
\pgfpathmoveto{\pgfqpoint{3.502561in}{2.073396in}}%
\pgfpathlineto{\pgfqpoint{3.763398in}{2.297457in}}%
\pgfpathlineto{\pgfqpoint{3.502561in}{2.073396in}}%
\pgfpathclose%
\pgfusepath{fill}%
\end{pgfscope}%
\begin{pgfscope}%
\pgfpathrectangle{\pgfqpoint{0.017500in}{0.017500in}}{\pgfqpoint{6.218282in}{3.580372in}} %
\pgfusepath{clip}%
\pgfsetbuttcap%
\pgfsetroundjoin%
\definecolor{currentfill}{rgb}{0.895885,0.433075,0.338681}%
\pgfsetfillcolor{currentfill}%
\pgfsetlinewidth{0.000000pt}%
\definecolor{currentstroke}{rgb}{0.000000,0.000000,0.000000}%
\pgfsetstrokecolor{currentstroke}%
\pgfsetdash{}{0pt}%
\pgfpathmoveto{\pgfqpoint{3.502561in}{2.073396in}}%
\pgfpathlineto{\pgfqpoint{3.763398in}{2.297457in}}%
\pgfpathlineto{\pgfqpoint{3.502561in}{2.073396in}}%
\pgfpathclose%
\pgfusepath{fill}%
\end{pgfscope}%
\begin{pgfscope}%
\pgfpathrectangle{\pgfqpoint{0.017500in}{0.017500in}}{\pgfqpoint{6.218282in}{3.580372in}} %
\pgfusepath{clip}%
\pgfsetbuttcap%
\pgfsetroundjoin%
\definecolor{currentfill}{rgb}{0.895885,0.433075,0.338681}%
\pgfsetfillcolor{currentfill}%
\pgfsetlinewidth{0.000000pt}%
\definecolor{currentstroke}{rgb}{0.000000,0.000000,0.000000}%
\pgfsetstrokecolor{currentstroke}%
\pgfsetdash{}{0pt}%
\pgfpathmoveto{\pgfqpoint{3.502561in}{2.073396in}}%
\pgfpathlineto{\pgfqpoint{3.763398in}{2.297457in}}%
\pgfpathlineto{\pgfqpoint{3.502561in}{2.073396in}}%
\pgfpathclose%
\pgfusepath{fill}%
\end{pgfscope}%
\begin{pgfscope}%
\pgfpathrectangle{\pgfqpoint{0.017500in}{0.017500in}}{\pgfqpoint{6.218282in}{3.580372in}} %
\pgfusepath{clip}%
\pgfsetbuttcap%
\pgfsetroundjoin%
\definecolor{currentfill}{rgb}{0.895885,0.433075,0.338681}%
\pgfsetfillcolor{currentfill}%
\pgfsetlinewidth{0.000000pt}%
\definecolor{currentstroke}{rgb}{0.000000,0.000000,0.000000}%
\pgfsetstrokecolor{currentstroke}%
\pgfsetdash{}{0pt}%
\pgfpathmoveto{\pgfqpoint{3.502561in}{2.073396in}}%
\pgfpathlineto{\pgfqpoint{3.763398in}{2.297457in}}%
\pgfpathlineto{\pgfqpoint{3.502561in}{2.073396in}}%
\pgfpathclose%
\pgfusepath{fill}%
\end{pgfscope}%
\begin{pgfscope}%
\pgfpathrectangle{\pgfqpoint{0.017500in}{0.017500in}}{\pgfqpoint{6.218282in}{3.580372in}} %
\pgfusepath{clip}%
\pgfsetbuttcap%
\pgfsetroundjoin%
\definecolor{currentfill}{rgb}{0.895885,0.433075,0.338681}%
\pgfsetfillcolor{currentfill}%
\pgfsetlinewidth{0.000000pt}%
\definecolor{currentstroke}{rgb}{0.000000,0.000000,0.000000}%
\pgfsetstrokecolor{currentstroke}%
\pgfsetdash{}{0pt}%
\pgfpathmoveto{\pgfqpoint{3.502561in}{2.073396in}}%
\pgfpathlineto{\pgfqpoint{3.763398in}{2.297457in}}%
\pgfpathlineto{\pgfqpoint{3.502561in}{2.073396in}}%
\pgfpathclose%
\pgfusepath{fill}%
\end{pgfscope}%
\begin{pgfscope}%
\pgfpathrectangle{\pgfqpoint{0.017500in}{0.017500in}}{\pgfqpoint{6.218282in}{3.580372in}} %
\pgfusepath{clip}%
\pgfsetbuttcap%
\pgfsetroundjoin%
\definecolor{currentfill}{rgb}{0.895885,0.433075,0.338681}%
\pgfsetfillcolor{currentfill}%
\pgfsetlinewidth{0.000000pt}%
\definecolor{currentstroke}{rgb}{0.000000,0.000000,0.000000}%
\pgfsetstrokecolor{currentstroke}%
\pgfsetdash{}{0pt}%
\pgfpathmoveto{\pgfqpoint{3.502561in}{2.073396in}}%
\pgfpathlineto{\pgfqpoint{3.763398in}{2.297457in}}%
\pgfpathlineto{\pgfqpoint{3.502561in}{2.073396in}}%
\pgfpathclose%
\pgfusepath{fill}%
\end{pgfscope}%
\begin{pgfscope}%
\pgfpathrectangle{\pgfqpoint{0.017500in}{0.017500in}}{\pgfqpoint{6.218282in}{3.580372in}} %
\pgfusepath{clip}%
\pgfsetbuttcap%
\pgfsetroundjoin%
\definecolor{currentfill}{rgb}{0.895885,0.433075,0.338681}%
\pgfsetfillcolor{currentfill}%
\pgfsetlinewidth{0.000000pt}%
\definecolor{currentstroke}{rgb}{0.000000,0.000000,0.000000}%
\pgfsetstrokecolor{currentstroke}%
\pgfsetdash{}{0pt}%
\pgfpathmoveto{\pgfqpoint{3.502561in}{2.073396in}}%
\pgfpathlineto{\pgfqpoint{3.763398in}{2.297457in}}%
\pgfpathlineto{\pgfqpoint{3.502561in}{2.073396in}}%
\pgfpathclose%
\pgfusepath{fill}%
\end{pgfscope}%
\begin{pgfscope}%
\pgfpathrectangle{\pgfqpoint{0.017500in}{0.017500in}}{\pgfqpoint{6.218282in}{3.580372in}} %
\pgfusepath{clip}%
\pgfsetbuttcap%
\pgfsetroundjoin%
\definecolor{currentfill}{rgb}{0.895885,0.433075,0.338681}%
\pgfsetfillcolor{currentfill}%
\pgfsetlinewidth{0.000000pt}%
\definecolor{currentstroke}{rgb}{0.000000,0.000000,0.000000}%
\pgfsetstrokecolor{currentstroke}%
\pgfsetdash{}{0pt}%
\pgfpathmoveto{\pgfqpoint{3.502561in}{2.073396in}}%
\pgfpathlineto{\pgfqpoint{3.763398in}{2.297457in}}%
\pgfpathlineto{\pgfqpoint{3.502561in}{2.073396in}}%
\pgfpathclose%
\pgfusepath{fill}%
\end{pgfscope}%
\begin{pgfscope}%
\pgfpathrectangle{\pgfqpoint{0.017500in}{0.017500in}}{\pgfqpoint{6.218282in}{3.580372in}} %
\pgfusepath{clip}%
\pgfsetbuttcap%
\pgfsetroundjoin%
\definecolor{currentfill}{rgb}{0.895885,0.433075,0.338681}%
\pgfsetfillcolor{currentfill}%
\pgfsetlinewidth{0.000000pt}%
\definecolor{currentstroke}{rgb}{0.000000,0.000000,0.000000}%
\pgfsetstrokecolor{currentstroke}%
\pgfsetdash{}{0pt}%
\pgfpathmoveto{\pgfqpoint{3.502561in}{2.073396in}}%
\pgfpathlineto{\pgfqpoint{3.763398in}{2.297457in}}%
\pgfpathlineto{\pgfqpoint{3.502561in}{2.073396in}}%
\pgfpathclose%
\pgfusepath{fill}%
\end{pgfscope}%
\begin{pgfscope}%
\pgfpathrectangle{\pgfqpoint{0.017500in}{0.017500in}}{\pgfqpoint{6.218282in}{3.580372in}} %
\pgfusepath{clip}%
\pgfsetbuttcap%
\pgfsetroundjoin%
\definecolor{currentfill}{rgb}{0.895885,0.433075,0.338681}%
\pgfsetfillcolor{currentfill}%
\pgfsetlinewidth{0.000000pt}%
\definecolor{currentstroke}{rgb}{0.000000,0.000000,0.000000}%
\pgfsetstrokecolor{currentstroke}%
\pgfsetdash{}{0pt}%
\pgfpathmoveto{\pgfqpoint{3.502561in}{2.073396in}}%
\pgfpathlineto{\pgfqpoint{3.763398in}{2.297457in}}%
\pgfpathlineto{\pgfqpoint{3.502561in}{2.073396in}}%
\pgfpathclose%
\pgfusepath{fill}%
\end{pgfscope}%
\begin{pgfscope}%
\pgfpathrectangle{\pgfqpoint{0.017500in}{0.017500in}}{\pgfqpoint{6.218282in}{3.580372in}} %
\pgfusepath{clip}%
\pgfsetbuttcap%
\pgfsetroundjoin%
\definecolor{currentfill}{rgb}{0.895885,0.433075,0.338681}%
\pgfsetfillcolor{currentfill}%
\pgfsetlinewidth{0.000000pt}%
\definecolor{currentstroke}{rgb}{0.000000,0.000000,0.000000}%
\pgfsetstrokecolor{currentstroke}%
\pgfsetdash{}{0pt}%
\pgfpathmoveto{\pgfqpoint{3.502561in}{2.073396in}}%
\pgfpathlineto{\pgfqpoint{3.763398in}{2.297457in}}%
\pgfpathlineto{\pgfqpoint{3.502561in}{2.073396in}}%
\pgfpathclose%
\pgfusepath{fill}%
\end{pgfscope}%
\begin{pgfscope}%
\pgfpathrectangle{\pgfqpoint{0.017500in}{0.017500in}}{\pgfqpoint{6.218282in}{3.580372in}} %
\pgfusepath{clip}%
\pgfsetbuttcap%
\pgfsetroundjoin%
\definecolor{currentfill}{rgb}{0.895885,0.433075,0.338681}%
\pgfsetfillcolor{currentfill}%
\pgfsetlinewidth{0.000000pt}%
\definecolor{currentstroke}{rgb}{0.000000,0.000000,0.000000}%
\pgfsetstrokecolor{currentstroke}%
\pgfsetdash{}{0pt}%
\pgfpathmoveto{\pgfqpoint{3.502561in}{2.073396in}}%
\pgfpathlineto{\pgfqpoint{3.763398in}{2.297457in}}%
\pgfpathlineto{\pgfqpoint{3.502561in}{2.073396in}}%
\pgfpathclose%
\pgfusepath{fill}%
\end{pgfscope}%
\begin{pgfscope}%
\pgfpathrectangle{\pgfqpoint{0.017500in}{0.017500in}}{\pgfqpoint{6.218282in}{3.580372in}} %
\pgfusepath{clip}%
\pgfsetbuttcap%
\pgfsetroundjoin%
\definecolor{currentfill}{rgb}{0.895885,0.433075,0.338681}%
\pgfsetfillcolor{currentfill}%
\pgfsetlinewidth{0.000000pt}%
\definecolor{currentstroke}{rgb}{0.000000,0.000000,0.000000}%
\pgfsetstrokecolor{currentstroke}%
\pgfsetdash{}{0pt}%
\pgfpathmoveto{\pgfqpoint{3.502561in}{2.073396in}}%
\pgfpathlineto{\pgfqpoint{3.763398in}{2.297457in}}%
\pgfpathlineto{\pgfqpoint{3.502561in}{2.073396in}}%
\pgfpathclose%
\pgfusepath{fill}%
\end{pgfscope}%
\begin{pgfscope}%
\pgfpathrectangle{\pgfqpoint{0.017500in}{0.017500in}}{\pgfqpoint{6.218282in}{3.580372in}} %
\pgfusepath{clip}%
\pgfsetbuttcap%
\pgfsetroundjoin%
\definecolor{currentfill}{rgb}{0.895885,0.433075,0.338681}%
\pgfsetfillcolor{currentfill}%
\pgfsetlinewidth{0.000000pt}%
\definecolor{currentstroke}{rgb}{0.000000,0.000000,0.000000}%
\pgfsetstrokecolor{currentstroke}%
\pgfsetdash{}{0pt}%
\pgfpathmoveto{\pgfqpoint{3.502561in}{2.073396in}}%
\pgfpathlineto{\pgfqpoint{3.763398in}{2.297457in}}%
\pgfpathlineto{\pgfqpoint{3.502561in}{2.073396in}}%
\pgfpathclose%
\pgfusepath{fill}%
\end{pgfscope}%
\begin{pgfscope}%
\pgfpathrectangle{\pgfqpoint{0.017500in}{0.017500in}}{\pgfqpoint{6.218282in}{3.580372in}} %
\pgfusepath{clip}%
\pgfsetbuttcap%
\pgfsetroundjoin%
\definecolor{currentfill}{rgb}{0.895885,0.433075,0.338681}%
\pgfsetfillcolor{currentfill}%
\pgfsetlinewidth{0.000000pt}%
\definecolor{currentstroke}{rgb}{0.000000,0.000000,0.000000}%
\pgfsetstrokecolor{currentstroke}%
\pgfsetdash{}{0pt}%
\pgfpathmoveto{\pgfqpoint{3.502561in}{2.073396in}}%
\pgfpathlineto{\pgfqpoint{3.763398in}{2.297457in}}%
\pgfpathlineto{\pgfqpoint{3.502561in}{2.073396in}}%
\pgfpathclose%
\pgfusepath{fill}%
\end{pgfscope}%
\begin{pgfscope}%
\pgfpathrectangle{\pgfqpoint{0.017500in}{0.017500in}}{\pgfqpoint{6.218282in}{3.580372in}} %
\pgfusepath{clip}%
\pgfsetbuttcap%
\pgfsetroundjoin%
\definecolor{currentfill}{rgb}{0.895885,0.433075,0.338681}%
\pgfsetfillcolor{currentfill}%
\pgfsetlinewidth{0.000000pt}%
\definecolor{currentstroke}{rgb}{0.000000,0.000000,0.000000}%
\pgfsetstrokecolor{currentstroke}%
\pgfsetdash{}{0pt}%
\pgfpathmoveto{\pgfqpoint{3.502561in}{2.073396in}}%
\pgfpathlineto{\pgfqpoint{3.763398in}{2.297457in}}%
\pgfpathlineto{\pgfqpoint{3.502561in}{2.073396in}}%
\pgfpathclose%
\pgfusepath{fill}%
\end{pgfscope}%
\begin{pgfscope}%
\pgfpathrectangle{\pgfqpoint{0.017500in}{0.017500in}}{\pgfqpoint{6.218282in}{3.580372in}} %
\pgfusepath{clip}%
\pgfsetbuttcap%
\pgfsetroundjoin%
\definecolor{currentfill}{rgb}{0.895885,0.433075,0.338681}%
\pgfsetfillcolor{currentfill}%
\pgfsetlinewidth{0.000000pt}%
\definecolor{currentstroke}{rgb}{0.000000,0.000000,0.000000}%
\pgfsetstrokecolor{currentstroke}%
\pgfsetdash{}{0pt}%
\pgfpathmoveto{\pgfqpoint{3.502561in}{2.073396in}}%
\pgfpathlineto{\pgfqpoint{3.763398in}{2.297457in}}%
\pgfpathlineto{\pgfqpoint{3.502561in}{2.073396in}}%
\pgfpathclose%
\pgfusepath{fill}%
\end{pgfscope}%
\begin{pgfscope}%
\pgfpathrectangle{\pgfqpoint{0.017500in}{0.017500in}}{\pgfqpoint{6.218282in}{3.580372in}} %
\pgfusepath{clip}%
\pgfsetbuttcap%
\pgfsetroundjoin%
\definecolor{currentfill}{rgb}{0.895885,0.433075,0.338681}%
\pgfsetfillcolor{currentfill}%
\pgfsetlinewidth{0.000000pt}%
\definecolor{currentstroke}{rgb}{0.000000,0.000000,0.000000}%
\pgfsetstrokecolor{currentstroke}%
\pgfsetdash{}{0pt}%
\pgfpathmoveto{\pgfqpoint{3.502561in}{2.073396in}}%
\pgfpathlineto{\pgfqpoint{3.763398in}{2.297457in}}%
\pgfpathlineto{\pgfqpoint{3.502561in}{2.073396in}}%
\pgfpathclose%
\pgfusepath{fill}%
\end{pgfscope}%
\begin{pgfscope}%
\pgfpathrectangle{\pgfqpoint{0.017500in}{0.017500in}}{\pgfqpoint{6.218282in}{3.580372in}} %
\pgfusepath{clip}%
\pgfsetbuttcap%
\pgfsetroundjoin%
\definecolor{currentfill}{rgb}{0.895885,0.433075,0.338681}%
\pgfsetfillcolor{currentfill}%
\pgfsetlinewidth{0.000000pt}%
\definecolor{currentstroke}{rgb}{0.000000,0.000000,0.000000}%
\pgfsetstrokecolor{currentstroke}%
\pgfsetdash{}{0pt}%
\pgfpathmoveto{\pgfqpoint{3.502561in}{2.073396in}}%
\pgfpathlineto{\pgfqpoint{3.763398in}{2.297457in}}%
\pgfpathlineto{\pgfqpoint{3.502561in}{2.073396in}}%
\pgfpathclose%
\pgfusepath{fill}%
\end{pgfscope}%
\begin{pgfscope}%
\pgfpathrectangle{\pgfqpoint{0.017500in}{0.017500in}}{\pgfqpoint{6.218282in}{3.580372in}} %
\pgfusepath{clip}%
\pgfsetbuttcap%
\pgfsetroundjoin%
\definecolor{currentfill}{rgb}{0.895885,0.433075,0.338681}%
\pgfsetfillcolor{currentfill}%
\pgfsetlinewidth{0.000000pt}%
\definecolor{currentstroke}{rgb}{0.000000,0.000000,0.000000}%
\pgfsetstrokecolor{currentstroke}%
\pgfsetdash{}{0pt}%
\pgfpathmoveto{\pgfqpoint{3.502561in}{2.073396in}}%
\pgfpathlineto{\pgfqpoint{3.763398in}{2.297457in}}%
\pgfpathlineto{\pgfqpoint{3.502561in}{2.073396in}}%
\pgfpathclose%
\pgfusepath{fill}%
\end{pgfscope}%
\begin{pgfscope}%
\pgfpathrectangle{\pgfqpoint{0.017500in}{0.017500in}}{\pgfqpoint{6.218282in}{3.580372in}} %
\pgfusepath{clip}%
\pgfsetbuttcap%
\pgfsetroundjoin%
\definecolor{currentfill}{rgb}{0.895885,0.433075,0.338681}%
\pgfsetfillcolor{currentfill}%
\pgfsetlinewidth{0.000000pt}%
\definecolor{currentstroke}{rgb}{0.000000,0.000000,0.000000}%
\pgfsetstrokecolor{currentstroke}%
\pgfsetdash{}{0pt}%
\pgfpathmoveto{\pgfqpoint{3.502561in}{2.073396in}}%
\pgfpathlineto{\pgfqpoint{3.763398in}{2.297457in}}%
\pgfpathlineto{\pgfqpoint{3.502561in}{2.073396in}}%
\pgfpathclose%
\pgfusepath{fill}%
\end{pgfscope}%
\begin{pgfscope}%
\pgfpathrectangle{\pgfqpoint{0.017500in}{0.017500in}}{\pgfqpoint{6.218282in}{3.580372in}} %
\pgfusepath{clip}%
\pgfsetbuttcap%
\pgfsetroundjoin%
\definecolor{currentfill}{rgb}{0.895885,0.433075,0.338681}%
\pgfsetfillcolor{currentfill}%
\pgfsetlinewidth{0.000000pt}%
\definecolor{currentstroke}{rgb}{0.000000,0.000000,0.000000}%
\pgfsetstrokecolor{currentstroke}%
\pgfsetdash{}{0pt}%
\pgfpathmoveto{\pgfqpoint{3.502561in}{2.073396in}}%
\pgfpathlineto{\pgfqpoint{3.763398in}{2.297457in}}%
\pgfpathlineto{\pgfqpoint{3.502561in}{2.073396in}}%
\pgfpathclose%
\pgfusepath{fill}%
\end{pgfscope}%
\begin{pgfscope}%
\pgfpathrectangle{\pgfqpoint{0.017500in}{0.017500in}}{\pgfqpoint{6.218282in}{3.580372in}} %
\pgfusepath{clip}%
\pgfsetbuttcap%
\pgfsetroundjoin%
\definecolor{currentfill}{rgb}{0.895885,0.433075,0.338681}%
\pgfsetfillcolor{currentfill}%
\pgfsetlinewidth{0.000000pt}%
\definecolor{currentstroke}{rgb}{0.000000,0.000000,0.000000}%
\pgfsetstrokecolor{currentstroke}%
\pgfsetdash{}{0pt}%
\pgfpathmoveto{\pgfqpoint{3.502561in}{2.073396in}}%
\pgfpathlineto{\pgfqpoint{3.763398in}{2.297457in}}%
\pgfpathlineto{\pgfqpoint{3.502561in}{2.073396in}}%
\pgfpathclose%
\pgfusepath{fill}%
\end{pgfscope}%
\begin{pgfscope}%
\pgfpathrectangle{\pgfqpoint{0.017500in}{0.017500in}}{\pgfqpoint{6.218282in}{3.580372in}} %
\pgfusepath{clip}%
\pgfsetbuttcap%
\pgfsetroundjoin%
\definecolor{currentfill}{rgb}{0.895885,0.433075,0.338681}%
\pgfsetfillcolor{currentfill}%
\pgfsetlinewidth{0.000000pt}%
\definecolor{currentstroke}{rgb}{0.000000,0.000000,0.000000}%
\pgfsetstrokecolor{currentstroke}%
\pgfsetdash{}{0pt}%
\pgfpathmoveto{\pgfqpoint{3.502561in}{2.073396in}}%
\pgfpathlineto{\pgfqpoint{3.763398in}{2.297457in}}%
\pgfpathlineto{\pgfqpoint{3.502561in}{2.073396in}}%
\pgfpathclose%
\pgfusepath{fill}%
\end{pgfscope}%
\begin{pgfscope}%
\pgfpathrectangle{\pgfqpoint{0.017500in}{0.017500in}}{\pgfqpoint{6.218282in}{3.580372in}} %
\pgfusepath{clip}%
\pgfsetbuttcap%
\pgfsetroundjoin%
\definecolor{currentfill}{rgb}{0.895885,0.433075,0.338681}%
\pgfsetfillcolor{currentfill}%
\pgfsetlinewidth{0.000000pt}%
\definecolor{currentstroke}{rgb}{0.000000,0.000000,0.000000}%
\pgfsetstrokecolor{currentstroke}%
\pgfsetdash{}{0pt}%
\pgfpathmoveto{\pgfqpoint{3.502561in}{2.073396in}}%
\pgfpathlineto{\pgfqpoint{3.763398in}{2.297457in}}%
\pgfpathlineto{\pgfqpoint{3.502561in}{2.073396in}}%
\pgfpathclose%
\pgfusepath{fill}%
\end{pgfscope}%
\begin{pgfscope}%
\pgfpathrectangle{\pgfqpoint{0.017500in}{0.017500in}}{\pgfqpoint{6.218282in}{3.580372in}} %
\pgfusepath{clip}%
\pgfsetbuttcap%
\pgfsetroundjoin%
\definecolor{currentfill}{rgb}{0.895885,0.433075,0.338681}%
\pgfsetfillcolor{currentfill}%
\pgfsetlinewidth{0.000000pt}%
\definecolor{currentstroke}{rgb}{0.000000,0.000000,0.000000}%
\pgfsetstrokecolor{currentstroke}%
\pgfsetdash{}{0pt}%
\pgfpathmoveto{\pgfqpoint{3.502561in}{2.073396in}}%
\pgfpathlineto{\pgfqpoint{3.763398in}{2.297457in}}%
\pgfpathlineto{\pgfqpoint{3.502561in}{2.073396in}}%
\pgfpathclose%
\pgfusepath{fill}%
\end{pgfscope}%
\begin{pgfscope}%
\pgfpathrectangle{\pgfqpoint{0.017500in}{0.017500in}}{\pgfqpoint{6.218282in}{3.580372in}} %
\pgfusepath{clip}%
\pgfsetbuttcap%
\pgfsetroundjoin%
\definecolor{currentfill}{rgb}{0.888390,0.417703,0.327898}%
\pgfsetfillcolor{currentfill}%
\pgfsetlinewidth{0.000000pt}%
\definecolor{currentstroke}{rgb}{0.000000,0.000000,0.000000}%
\pgfsetstrokecolor{currentstroke}%
\pgfsetdash{}{0pt}%
\pgfpathmoveto{\pgfqpoint{3.502561in}{2.073396in}}%
\pgfpathlineto{\pgfqpoint{3.968076in}{1.994042in}}%
\pgfpathlineto{\pgfqpoint{4.226975in}{2.236848in}}%
\pgfpathlineto{\pgfqpoint{3.763398in}{2.297457in}}%
\pgfpathlineto{\pgfqpoint{3.502561in}{2.073396in}}%
\pgfpathclose%
\pgfusepath{fill}%
\end{pgfscope}%
\begin{pgfscope}%
\pgfpathrectangle{\pgfqpoint{0.017500in}{0.017500in}}{\pgfqpoint{6.218282in}{3.580372in}} %
\pgfusepath{clip}%
\pgfsetbuttcap%
\pgfsetroundjoin%
\definecolor{currentfill}{rgb}{0.888390,0.417703,0.327898}%
\pgfsetfillcolor{currentfill}%
\pgfsetlinewidth{0.000000pt}%
\definecolor{currentstroke}{rgb}{0.000000,0.000000,0.000000}%
\pgfsetstrokecolor{currentstroke}%
\pgfsetdash{}{0pt}%
\pgfpathmoveto{\pgfqpoint{3.502561in}{2.073396in}}%
\pgfpathlineto{\pgfqpoint{3.968076in}{1.994042in}}%
\pgfpathlineto{\pgfqpoint{4.226975in}{2.236848in}}%
\pgfpathlineto{\pgfqpoint{3.763398in}{2.297457in}}%
\pgfpathlineto{\pgfqpoint{3.502561in}{2.073396in}}%
\pgfpathclose%
\pgfusepath{fill}%
\end{pgfscope}%
\begin{pgfscope}%
\pgfpathrectangle{\pgfqpoint{0.017500in}{0.017500in}}{\pgfqpoint{6.218282in}{3.580372in}} %
\pgfusepath{clip}%
\pgfsetbuttcap%
\pgfsetroundjoin%
\definecolor{currentfill}{rgb}{0.888390,0.417703,0.327898}%
\pgfsetfillcolor{currentfill}%
\pgfsetlinewidth{0.000000pt}%
\definecolor{currentstroke}{rgb}{0.000000,0.000000,0.000000}%
\pgfsetstrokecolor{currentstroke}%
\pgfsetdash{}{0pt}%
\pgfpathmoveto{\pgfqpoint{3.502561in}{2.073396in}}%
\pgfpathlineto{\pgfqpoint{3.968076in}{1.994042in}}%
\pgfpathlineto{\pgfqpoint{4.226975in}{2.236848in}}%
\pgfpathlineto{\pgfqpoint{3.763398in}{2.297457in}}%
\pgfpathlineto{\pgfqpoint{3.502561in}{2.073396in}}%
\pgfpathclose%
\pgfusepath{fill}%
\end{pgfscope}%
\begin{pgfscope}%
\pgfpathrectangle{\pgfqpoint{0.017500in}{0.017500in}}{\pgfqpoint{6.218282in}{3.580372in}} %
\pgfusepath{clip}%
\pgfsetbuttcap%
\pgfsetroundjoin%
\definecolor{currentfill}{rgb}{0.888390,0.417703,0.327898}%
\pgfsetfillcolor{currentfill}%
\pgfsetlinewidth{0.000000pt}%
\definecolor{currentstroke}{rgb}{0.000000,0.000000,0.000000}%
\pgfsetstrokecolor{currentstroke}%
\pgfsetdash{}{0pt}%
\pgfpathmoveto{\pgfqpoint{3.502561in}{2.073396in}}%
\pgfpathlineto{\pgfqpoint{3.968076in}{1.994042in}}%
\pgfpathlineto{\pgfqpoint{4.226975in}{2.236848in}}%
\pgfpathlineto{\pgfqpoint{3.763398in}{2.297457in}}%
\pgfpathlineto{\pgfqpoint{3.502561in}{2.073396in}}%
\pgfpathclose%
\pgfusepath{fill}%
\end{pgfscope}%
\begin{pgfscope}%
\pgfpathrectangle{\pgfqpoint{0.017500in}{0.017500in}}{\pgfqpoint{6.218282in}{3.580372in}} %
\pgfusepath{clip}%
\pgfsetbuttcap%
\pgfsetroundjoin%
\definecolor{currentfill}{rgb}{0.888390,0.417703,0.327898}%
\pgfsetfillcolor{currentfill}%
\pgfsetlinewidth{0.000000pt}%
\definecolor{currentstroke}{rgb}{0.000000,0.000000,0.000000}%
\pgfsetstrokecolor{currentstroke}%
\pgfsetdash{}{0pt}%
\pgfpathmoveto{\pgfqpoint{3.502561in}{2.073396in}}%
\pgfpathlineto{\pgfqpoint{3.968076in}{1.994042in}}%
\pgfpathlineto{\pgfqpoint{4.226975in}{2.236848in}}%
\pgfpathlineto{\pgfqpoint{3.763398in}{2.297457in}}%
\pgfpathlineto{\pgfqpoint{3.502561in}{2.073396in}}%
\pgfpathclose%
\pgfusepath{fill}%
\end{pgfscope}%
\begin{pgfscope}%
\pgfpathrectangle{\pgfqpoint{0.017500in}{0.017500in}}{\pgfqpoint{6.218282in}{3.580372in}} %
\pgfusepath{clip}%
\pgfsetbuttcap%
\pgfsetroundjoin%
\definecolor{currentfill}{rgb}{0.888390,0.417703,0.327898}%
\pgfsetfillcolor{currentfill}%
\pgfsetlinewidth{0.000000pt}%
\definecolor{currentstroke}{rgb}{0.000000,0.000000,0.000000}%
\pgfsetstrokecolor{currentstroke}%
\pgfsetdash{}{0pt}%
\pgfpathmoveto{\pgfqpoint{3.502561in}{2.073396in}}%
\pgfpathlineto{\pgfqpoint{3.968076in}{1.994042in}}%
\pgfpathlineto{\pgfqpoint{4.226975in}{2.236848in}}%
\pgfpathlineto{\pgfqpoint{3.763398in}{2.297457in}}%
\pgfpathlineto{\pgfqpoint{3.502561in}{2.073396in}}%
\pgfpathclose%
\pgfusepath{fill}%
\end{pgfscope}%
\begin{pgfscope}%
\pgfpathrectangle{\pgfqpoint{0.017500in}{0.017500in}}{\pgfqpoint{6.218282in}{3.580372in}} %
\pgfusepath{clip}%
\pgfsetbuttcap%
\pgfsetroundjoin%
\definecolor{currentfill}{rgb}{0.892138,0.425389,0.333289}%
\pgfsetfillcolor{currentfill}%
\pgfsetlinewidth{0.000000pt}%
\definecolor{currentstroke}{rgb}{0.000000,0.000000,0.000000}%
\pgfsetstrokecolor{currentstroke}%
\pgfsetdash{}{0pt}%
\pgfpathmoveto{\pgfqpoint{3.968076in}{1.994042in}}%
\pgfpathlineto{\pgfqpoint{4.226975in}{2.236848in}}%
\pgfpathlineto{\pgfqpoint{3.968076in}{1.994042in}}%
\pgfpathclose%
\pgfusepath{fill}%
\end{pgfscope}%
\begin{pgfscope}%
\pgfpathrectangle{\pgfqpoint{0.017500in}{0.017500in}}{\pgfqpoint{6.218282in}{3.580372in}} %
\pgfusepath{clip}%
\pgfsetbuttcap%
\pgfsetroundjoin%
\definecolor{currentfill}{rgb}{0.892138,0.425389,0.333289}%
\pgfsetfillcolor{currentfill}%
\pgfsetlinewidth{0.000000pt}%
\definecolor{currentstroke}{rgb}{0.000000,0.000000,0.000000}%
\pgfsetstrokecolor{currentstroke}%
\pgfsetdash{}{0pt}%
\pgfpathmoveto{\pgfqpoint{3.968076in}{1.994042in}}%
\pgfpathlineto{\pgfqpoint{4.226975in}{2.236848in}}%
\pgfpathlineto{\pgfqpoint{3.968076in}{1.994042in}}%
\pgfpathclose%
\pgfusepath{fill}%
\end{pgfscope}%
\begin{pgfscope}%
\pgfpathrectangle{\pgfqpoint{0.017500in}{0.017500in}}{\pgfqpoint{6.218282in}{3.580372in}} %
\pgfusepath{clip}%
\pgfsetbuttcap%
\pgfsetroundjoin%
\definecolor{currentfill}{rgb}{0.892138,0.425389,0.333289}%
\pgfsetfillcolor{currentfill}%
\pgfsetlinewidth{0.000000pt}%
\definecolor{currentstroke}{rgb}{0.000000,0.000000,0.000000}%
\pgfsetstrokecolor{currentstroke}%
\pgfsetdash{}{0pt}%
\pgfpathmoveto{\pgfqpoint{3.968076in}{1.994042in}}%
\pgfpathlineto{\pgfqpoint{4.226975in}{2.236848in}}%
\pgfpathlineto{\pgfqpoint{3.968076in}{1.994042in}}%
\pgfpathclose%
\pgfusepath{fill}%
\end{pgfscope}%
\begin{pgfscope}%
\pgfpathrectangle{\pgfqpoint{0.017500in}{0.017500in}}{\pgfqpoint{6.218282in}{3.580372in}} %
\pgfusepath{clip}%
\pgfsetbuttcap%
\pgfsetroundjoin%
\definecolor{currentfill}{rgb}{0.892138,0.425389,0.333289}%
\pgfsetfillcolor{currentfill}%
\pgfsetlinewidth{0.000000pt}%
\definecolor{currentstroke}{rgb}{0.000000,0.000000,0.000000}%
\pgfsetstrokecolor{currentstroke}%
\pgfsetdash{}{0pt}%
\pgfpathmoveto{\pgfqpoint{3.968076in}{1.994042in}}%
\pgfpathlineto{\pgfqpoint{4.226975in}{2.236848in}}%
\pgfpathlineto{\pgfqpoint{3.968076in}{1.994042in}}%
\pgfpathclose%
\pgfusepath{fill}%
\end{pgfscope}%
\begin{pgfscope}%
\pgfpathrectangle{\pgfqpoint{0.017500in}{0.017500in}}{\pgfqpoint{6.218282in}{3.580372in}} %
\pgfusepath{clip}%
\pgfsetbuttcap%
\pgfsetroundjoin%
\definecolor{currentfill}{rgb}{0.892138,0.425389,0.333289}%
\pgfsetfillcolor{currentfill}%
\pgfsetlinewidth{0.000000pt}%
\definecolor{currentstroke}{rgb}{0.000000,0.000000,0.000000}%
\pgfsetstrokecolor{currentstroke}%
\pgfsetdash{}{0pt}%
\pgfpathmoveto{\pgfqpoint{3.968076in}{1.994042in}}%
\pgfpathlineto{\pgfqpoint{4.226975in}{2.236848in}}%
\pgfpathlineto{\pgfqpoint{3.968076in}{1.994042in}}%
\pgfpathclose%
\pgfusepath{fill}%
\end{pgfscope}%
\begin{pgfscope}%
\pgfpathrectangle{\pgfqpoint{0.017500in}{0.017500in}}{\pgfqpoint{6.218282in}{3.580372in}} %
\pgfusepath{clip}%
\pgfsetbuttcap%
\pgfsetroundjoin%
\definecolor{currentfill}{rgb}{0.892138,0.425389,0.333289}%
\pgfsetfillcolor{currentfill}%
\pgfsetlinewidth{0.000000pt}%
\definecolor{currentstroke}{rgb}{0.000000,0.000000,0.000000}%
\pgfsetstrokecolor{currentstroke}%
\pgfsetdash{}{0pt}%
\pgfpathmoveto{\pgfqpoint{3.968076in}{1.994042in}}%
\pgfpathlineto{\pgfqpoint{4.226975in}{2.236848in}}%
\pgfpathlineto{\pgfqpoint{3.968076in}{1.994042in}}%
\pgfpathclose%
\pgfusepath{fill}%
\end{pgfscope}%
\begin{pgfscope}%
\pgfpathrectangle{\pgfqpoint{0.017500in}{0.017500in}}{\pgfqpoint{6.218282in}{3.580372in}} %
\pgfusepath{clip}%
\pgfsetbuttcap%
\pgfsetroundjoin%
\definecolor{currentfill}{rgb}{0.892138,0.425389,0.333289}%
\pgfsetfillcolor{currentfill}%
\pgfsetlinewidth{0.000000pt}%
\definecolor{currentstroke}{rgb}{0.000000,0.000000,0.000000}%
\pgfsetstrokecolor{currentstroke}%
\pgfsetdash{}{0pt}%
\pgfpathmoveto{\pgfqpoint{3.968076in}{1.994042in}}%
\pgfpathlineto{\pgfqpoint{4.226975in}{2.236848in}}%
\pgfpathlineto{\pgfqpoint{3.968076in}{1.994042in}}%
\pgfpathclose%
\pgfusepath{fill}%
\end{pgfscope}%
\begin{pgfscope}%
\pgfpathrectangle{\pgfqpoint{0.017500in}{0.017500in}}{\pgfqpoint{6.218282in}{3.580372in}} %
\pgfusepath{clip}%
\pgfsetbuttcap%
\pgfsetroundjoin%
\definecolor{currentfill}{rgb}{0.892138,0.425389,0.333289}%
\pgfsetfillcolor{currentfill}%
\pgfsetlinewidth{0.000000pt}%
\definecolor{currentstroke}{rgb}{0.000000,0.000000,0.000000}%
\pgfsetstrokecolor{currentstroke}%
\pgfsetdash{}{0pt}%
\pgfpathmoveto{\pgfqpoint{3.968076in}{1.994042in}}%
\pgfpathlineto{\pgfqpoint{4.226975in}{2.236848in}}%
\pgfpathlineto{\pgfqpoint{3.968076in}{1.994042in}}%
\pgfpathclose%
\pgfusepath{fill}%
\end{pgfscope}%
\begin{pgfscope}%
\pgfpathrectangle{\pgfqpoint{0.017500in}{0.017500in}}{\pgfqpoint{6.218282in}{3.580372in}} %
\pgfusepath{clip}%
\pgfsetbuttcap%
\pgfsetroundjoin%
\definecolor{currentfill}{rgb}{0.892138,0.425389,0.333289}%
\pgfsetfillcolor{currentfill}%
\pgfsetlinewidth{0.000000pt}%
\definecolor{currentstroke}{rgb}{0.000000,0.000000,0.000000}%
\pgfsetstrokecolor{currentstroke}%
\pgfsetdash{}{0pt}%
\pgfpathmoveto{\pgfqpoint{3.968076in}{1.994042in}}%
\pgfpathlineto{\pgfqpoint{4.226975in}{2.236848in}}%
\pgfpathlineto{\pgfqpoint{3.968076in}{1.994042in}}%
\pgfpathclose%
\pgfusepath{fill}%
\end{pgfscope}%
\begin{pgfscope}%
\pgfpathrectangle{\pgfqpoint{0.017500in}{0.017500in}}{\pgfqpoint{6.218282in}{3.580372in}} %
\pgfusepath{clip}%
\pgfsetbuttcap%
\pgfsetroundjoin%
\definecolor{currentfill}{rgb}{0.892138,0.425389,0.333289}%
\pgfsetfillcolor{currentfill}%
\pgfsetlinewidth{0.000000pt}%
\definecolor{currentstroke}{rgb}{0.000000,0.000000,0.000000}%
\pgfsetstrokecolor{currentstroke}%
\pgfsetdash{}{0pt}%
\pgfpathmoveto{\pgfqpoint{3.968076in}{1.994042in}}%
\pgfpathlineto{\pgfqpoint{4.226975in}{2.236848in}}%
\pgfpathlineto{\pgfqpoint{3.968076in}{1.994042in}}%
\pgfpathclose%
\pgfusepath{fill}%
\end{pgfscope}%
\begin{pgfscope}%
\pgfpathrectangle{\pgfqpoint{0.017500in}{0.017500in}}{\pgfqpoint{6.218282in}{3.580372in}} %
\pgfusepath{clip}%
\pgfsetbuttcap%
\pgfsetroundjoin%
\definecolor{currentfill}{rgb}{0.892138,0.425389,0.333289}%
\pgfsetfillcolor{currentfill}%
\pgfsetlinewidth{0.000000pt}%
\definecolor{currentstroke}{rgb}{0.000000,0.000000,0.000000}%
\pgfsetstrokecolor{currentstroke}%
\pgfsetdash{}{0pt}%
\pgfpathmoveto{\pgfqpoint{3.968076in}{1.994042in}}%
\pgfpathlineto{\pgfqpoint{4.226975in}{2.236848in}}%
\pgfpathlineto{\pgfqpoint{3.968076in}{1.994042in}}%
\pgfpathclose%
\pgfusepath{fill}%
\end{pgfscope}%
\begin{pgfscope}%
\pgfpathrectangle{\pgfqpoint{0.017500in}{0.017500in}}{\pgfqpoint{6.218282in}{3.580372in}} %
\pgfusepath{clip}%
\pgfsetbuttcap%
\pgfsetroundjoin%
\definecolor{currentfill}{rgb}{0.892138,0.425389,0.333289}%
\pgfsetfillcolor{currentfill}%
\pgfsetlinewidth{0.000000pt}%
\definecolor{currentstroke}{rgb}{0.000000,0.000000,0.000000}%
\pgfsetstrokecolor{currentstroke}%
\pgfsetdash{}{0pt}%
\pgfpathmoveto{\pgfqpoint{3.968076in}{1.994042in}}%
\pgfpathlineto{\pgfqpoint{4.226975in}{2.236848in}}%
\pgfpathlineto{\pgfqpoint{3.968076in}{1.994042in}}%
\pgfpathclose%
\pgfusepath{fill}%
\end{pgfscope}%
\begin{pgfscope}%
\pgfpathrectangle{\pgfqpoint{0.017500in}{0.017500in}}{\pgfqpoint{6.218282in}{3.580372in}} %
\pgfusepath{clip}%
\pgfsetbuttcap%
\pgfsetroundjoin%
\definecolor{currentfill}{rgb}{0.892138,0.425389,0.333289}%
\pgfsetfillcolor{currentfill}%
\pgfsetlinewidth{0.000000pt}%
\definecolor{currentstroke}{rgb}{0.000000,0.000000,0.000000}%
\pgfsetstrokecolor{currentstroke}%
\pgfsetdash{}{0pt}%
\pgfpathmoveto{\pgfqpoint{3.968076in}{1.994042in}}%
\pgfpathlineto{\pgfqpoint{4.226975in}{2.236848in}}%
\pgfpathlineto{\pgfqpoint{3.968076in}{1.994042in}}%
\pgfpathclose%
\pgfusepath{fill}%
\end{pgfscope}%
\begin{pgfscope}%
\pgfpathrectangle{\pgfqpoint{0.017500in}{0.017500in}}{\pgfqpoint{6.218282in}{3.580372in}} %
\pgfusepath{clip}%
\pgfsetbuttcap%
\pgfsetroundjoin%
\definecolor{currentfill}{rgb}{0.892138,0.425389,0.333289}%
\pgfsetfillcolor{currentfill}%
\pgfsetlinewidth{0.000000pt}%
\definecolor{currentstroke}{rgb}{0.000000,0.000000,0.000000}%
\pgfsetstrokecolor{currentstroke}%
\pgfsetdash{}{0pt}%
\pgfpathmoveto{\pgfqpoint{3.968076in}{1.994042in}}%
\pgfpathlineto{\pgfqpoint{4.226975in}{2.236848in}}%
\pgfpathlineto{\pgfqpoint{3.968076in}{1.994042in}}%
\pgfpathclose%
\pgfusepath{fill}%
\end{pgfscope}%
\begin{pgfscope}%
\pgfpathrectangle{\pgfqpoint{0.017500in}{0.017500in}}{\pgfqpoint{6.218282in}{3.580372in}} %
\pgfusepath{clip}%
\pgfsetbuttcap%
\pgfsetroundjoin%
\definecolor{currentfill}{rgb}{0.892138,0.425389,0.333289}%
\pgfsetfillcolor{currentfill}%
\pgfsetlinewidth{0.000000pt}%
\definecolor{currentstroke}{rgb}{0.000000,0.000000,0.000000}%
\pgfsetstrokecolor{currentstroke}%
\pgfsetdash{}{0pt}%
\pgfpathmoveto{\pgfqpoint{3.968076in}{1.994042in}}%
\pgfpathlineto{\pgfqpoint{4.226975in}{2.236848in}}%
\pgfpathlineto{\pgfqpoint{3.968076in}{1.994042in}}%
\pgfpathclose%
\pgfusepath{fill}%
\end{pgfscope}%
\begin{pgfscope}%
\pgfpathrectangle{\pgfqpoint{0.017500in}{0.017500in}}{\pgfqpoint{6.218282in}{3.580372in}} %
\pgfusepath{clip}%
\pgfsetbuttcap%
\pgfsetroundjoin%
\definecolor{currentfill}{rgb}{0.892138,0.425389,0.333289}%
\pgfsetfillcolor{currentfill}%
\pgfsetlinewidth{0.000000pt}%
\definecolor{currentstroke}{rgb}{0.000000,0.000000,0.000000}%
\pgfsetstrokecolor{currentstroke}%
\pgfsetdash{}{0pt}%
\pgfpathmoveto{\pgfqpoint{3.968076in}{1.994042in}}%
\pgfpathlineto{\pgfqpoint{4.226975in}{2.236848in}}%
\pgfpathlineto{\pgfqpoint{3.968076in}{1.994042in}}%
\pgfpathclose%
\pgfusepath{fill}%
\end{pgfscope}%
\begin{pgfscope}%
\pgfpathrectangle{\pgfqpoint{0.017500in}{0.017500in}}{\pgfqpoint{6.218282in}{3.580372in}} %
\pgfusepath{clip}%
\pgfsetbuttcap%
\pgfsetroundjoin%
\definecolor{currentfill}{rgb}{0.892138,0.425389,0.333289}%
\pgfsetfillcolor{currentfill}%
\pgfsetlinewidth{0.000000pt}%
\definecolor{currentstroke}{rgb}{0.000000,0.000000,0.000000}%
\pgfsetstrokecolor{currentstroke}%
\pgfsetdash{}{0pt}%
\pgfpathmoveto{\pgfqpoint{3.968076in}{1.994042in}}%
\pgfpathlineto{\pgfqpoint{4.226975in}{2.236848in}}%
\pgfpathlineto{\pgfqpoint{3.968076in}{1.994042in}}%
\pgfpathclose%
\pgfusepath{fill}%
\end{pgfscope}%
\begin{pgfscope}%
\pgfpathrectangle{\pgfqpoint{0.017500in}{0.017500in}}{\pgfqpoint{6.218282in}{3.580372in}} %
\pgfusepath{clip}%
\pgfsetbuttcap%
\pgfsetroundjoin%
\definecolor{currentfill}{rgb}{0.892138,0.425389,0.333289}%
\pgfsetfillcolor{currentfill}%
\pgfsetlinewidth{0.000000pt}%
\definecolor{currentstroke}{rgb}{0.000000,0.000000,0.000000}%
\pgfsetstrokecolor{currentstroke}%
\pgfsetdash{}{0pt}%
\pgfpathmoveto{\pgfqpoint{3.968076in}{1.994042in}}%
\pgfpathlineto{\pgfqpoint{4.226975in}{2.236848in}}%
\pgfpathlineto{\pgfqpoint{3.968076in}{1.994042in}}%
\pgfpathclose%
\pgfusepath{fill}%
\end{pgfscope}%
\begin{pgfscope}%
\pgfpathrectangle{\pgfqpoint{0.017500in}{0.017500in}}{\pgfqpoint{6.218282in}{3.580372in}} %
\pgfusepath{clip}%
\pgfsetbuttcap%
\pgfsetroundjoin%
\definecolor{currentfill}{rgb}{0.892138,0.425389,0.333289}%
\pgfsetfillcolor{currentfill}%
\pgfsetlinewidth{0.000000pt}%
\definecolor{currentstroke}{rgb}{0.000000,0.000000,0.000000}%
\pgfsetstrokecolor{currentstroke}%
\pgfsetdash{}{0pt}%
\pgfpathmoveto{\pgfqpoint{3.968076in}{1.994042in}}%
\pgfpathlineto{\pgfqpoint{4.226975in}{2.236848in}}%
\pgfpathlineto{\pgfqpoint{3.968076in}{1.994042in}}%
\pgfpathclose%
\pgfusepath{fill}%
\end{pgfscope}%
\begin{pgfscope}%
\pgfpathrectangle{\pgfqpoint{0.017500in}{0.017500in}}{\pgfqpoint{6.218282in}{3.580372in}} %
\pgfusepath{clip}%
\pgfsetbuttcap%
\pgfsetroundjoin%
\definecolor{currentfill}{rgb}{0.892138,0.425389,0.333289}%
\pgfsetfillcolor{currentfill}%
\pgfsetlinewidth{0.000000pt}%
\definecolor{currentstroke}{rgb}{0.000000,0.000000,0.000000}%
\pgfsetstrokecolor{currentstroke}%
\pgfsetdash{}{0pt}%
\pgfpathmoveto{\pgfqpoint{3.968076in}{1.994042in}}%
\pgfpathlineto{\pgfqpoint{4.226975in}{2.236848in}}%
\pgfpathlineto{\pgfqpoint{3.968076in}{1.994042in}}%
\pgfpathclose%
\pgfusepath{fill}%
\end{pgfscope}%
\begin{pgfscope}%
\pgfpathrectangle{\pgfqpoint{0.017500in}{0.017500in}}{\pgfqpoint{6.218282in}{3.580372in}} %
\pgfusepath{clip}%
\pgfsetbuttcap%
\pgfsetroundjoin%
\definecolor{currentfill}{rgb}{0.892138,0.425389,0.333289}%
\pgfsetfillcolor{currentfill}%
\pgfsetlinewidth{0.000000pt}%
\definecolor{currentstroke}{rgb}{0.000000,0.000000,0.000000}%
\pgfsetstrokecolor{currentstroke}%
\pgfsetdash{}{0pt}%
\pgfpathmoveto{\pgfqpoint{3.968076in}{1.994042in}}%
\pgfpathlineto{\pgfqpoint{4.226975in}{2.236848in}}%
\pgfpathlineto{\pgfqpoint{3.968076in}{1.994042in}}%
\pgfpathclose%
\pgfusepath{fill}%
\end{pgfscope}%
\begin{pgfscope}%
\pgfpathrectangle{\pgfqpoint{0.017500in}{0.017500in}}{\pgfqpoint{6.218282in}{3.580372in}} %
\pgfusepath{clip}%
\pgfsetbuttcap%
\pgfsetroundjoin%
\definecolor{currentfill}{rgb}{0.892138,0.425389,0.333289}%
\pgfsetfillcolor{currentfill}%
\pgfsetlinewidth{0.000000pt}%
\definecolor{currentstroke}{rgb}{0.000000,0.000000,0.000000}%
\pgfsetstrokecolor{currentstroke}%
\pgfsetdash{}{0pt}%
\pgfpathmoveto{\pgfqpoint{3.968076in}{1.994042in}}%
\pgfpathlineto{\pgfqpoint{4.226975in}{2.236848in}}%
\pgfpathlineto{\pgfqpoint{3.968076in}{1.994042in}}%
\pgfpathclose%
\pgfusepath{fill}%
\end{pgfscope}%
\begin{pgfscope}%
\pgfpathrectangle{\pgfqpoint{0.017500in}{0.017500in}}{\pgfqpoint{6.218282in}{3.580372in}} %
\pgfusepath{clip}%
\pgfsetbuttcap%
\pgfsetroundjoin%
\definecolor{currentfill}{rgb}{0.892138,0.425389,0.333289}%
\pgfsetfillcolor{currentfill}%
\pgfsetlinewidth{0.000000pt}%
\definecolor{currentstroke}{rgb}{0.000000,0.000000,0.000000}%
\pgfsetstrokecolor{currentstroke}%
\pgfsetdash{}{0pt}%
\pgfpathmoveto{\pgfqpoint{3.968076in}{1.994042in}}%
\pgfpathlineto{\pgfqpoint{4.226975in}{2.236848in}}%
\pgfpathlineto{\pgfqpoint{3.968076in}{1.994042in}}%
\pgfpathclose%
\pgfusepath{fill}%
\end{pgfscope}%
\begin{pgfscope}%
\pgfpathrectangle{\pgfqpoint{0.017500in}{0.017500in}}{\pgfqpoint{6.218282in}{3.580372in}} %
\pgfusepath{clip}%
\pgfsetbuttcap%
\pgfsetroundjoin%
\definecolor{currentfill}{rgb}{0.892138,0.425389,0.333289}%
\pgfsetfillcolor{currentfill}%
\pgfsetlinewidth{0.000000pt}%
\definecolor{currentstroke}{rgb}{0.000000,0.000000,0.000000}%
\pgfsetstrokecolor{currentstroke}%
\pgfsetdash{}{0pt}%
\pgfpathmoveto{\pgfqpoint{3.968076in}{1.994042in}}%
\pgfpathlineto{\pgfqpoint{4.226975in}{2.236848in}}%
\pgfpathlineto{\pgfqpoint{3.968076in}{1.994042in}}%
\pgfpathclose%
\pgfusepath{fill}%
\end{pgfscope}%
\begin{pgfscope}%
\pgfpathrectangle{\pgfqpoint{0.017500in}{0.017500in}}{\pgfqpoint{6.218282in}{3.580372in}} %
\pgfusepath{clip}%
\pgfsetbuttcap%
\pgfsetroundjoin%
\definecolor{currentfill}{rgb}{0.892138,0.425389,0.333289}%
\pgfsetfillcolor{currentfill}%
\pgfsetlinewidth{0.000000pt}%
\definecolor{currentstroke}{rgb}{0.000000,0.000000,0.000000}%
\pgfsetstrokecolor{currentstroke}%
\pgfsetdash{}{0pt}%
\pgfpathmoveto{\pgfqpoint{3.968076in}{1.994042in}}%
\pgfpathlineto{\pgfqpoint{4.226975in}{2.236848in}}%
\pgfpathlineto{\pgfqpoint{3.968076in}{1.994042in}}%
\pgfpathclose%
\pgfusepath{fill}%
\end{pgfscope}%
\begin{pgfscope}%
\pgfpathrectangle{\pgfqpoint{0.017500in}{0.017500in}}{\pgfqpoint{6.218282in}{3.580372in}} %
\pgfusepath{clip}%
\pgfsetbuttcap%
\pgfsetroundjoin%
\definecolor{currentfill}{rgb}{0.892138,0.425389,0.333289}%
\pgfsetfillcolor{currentfill}%
\pgfsetlinewidth{0.000000pt}%
\definecolor{currentstroke}{rgb}{0.000000,0.000000,0.000000}%
\pgfsetstrokecolor{currentstroke}%
\pgfsetdash{}{0pt}%
\pgfpathmoveto{\pgfqpoint{3.968076in}{1.994042in}}%
\pgfpathlineto{\pgfqpoint{4.226975in}{2.236848in}}%
\pgfpathlineto{\pgfqpoint{3.968076in}{1.994042in}}%
\pgfpathclose%
\pgfusepath{fill}%
\end{pgfscope}%
\begin{pgfscope}%
\pgfpathrectangle{\pgfqpoint{0.017500in}{0.017500in}}{\pgfqpoint{6.218282in}{3.580372in}} %
\pgfusepath{clip}%
\pgfsetbuttcap%
\pgfsetroundjoin%
\definecolor{currentfill}{rgb}{0.892138,0.425389,0.333289}%
\pgfsetfillcolor{currentfill}%
\pgfsetlinewidth{0.000000pt}%
\definecolor{currentstroke}{rgb}{0.000000,0.000000,0.000000}%
\pgfsetstrokecolor{currentstroke}%
\pgfsetdash{}{0pt}%
\pgfpathmoveto{\pgfqpoint{3.968076in}{1.994042in}}%
\pgfpathlineto{\pgfqpoint{4.226975in}{2.236848in}}%
\pgfpathlineto{\pgfqpoint{3.968076in}{1.994042in}}%
\pgfpathclose%
\pgfusepath{fill}%
\end{pgfscope}%
\begin{pgfscope}%
\pgfpathrectangle{\pgfqpoint{0.017500in}{0.017500in}}{\pgfqpoint{6.218282in}{3.580372in}} %
\pgfusepath{clip}%
\pgfsetbuttcap%
\pgfsetroundjoin%
\definecolor{currentfill}{rgb}{0.892138,0.425389,0.333289}%
\pgfsetfillcolor{currentfill}%
\pgfsetlinewidth{0.000000pt}%
\definecolor{currentstroke}{rgb}{0.000000,0.000000,0.000000}%
\pgfsetstrokecolor{currentstroke}%
\pgfsetdash{}{0pt}%
\pgfpathmoveto{\pgfqpoint{3.968076in}{1.994042in}}%
\pgfpathlineto{\pgfqpoint{4.226975in}{2.236848in}}%
\pgfpathlineto{\pgfqpoint{3.968076in}{1.994042in}}%
\pgfpathclose%
\pgfusepath{fill}%
\end{pgfscope}%
\begin{pgfscope}%
\pgfpathrectangle{\pgfqpoint{0.017500in}{0.017500in}}{\pgfqpoint{6.218282in}{3.580372in}} %
\pgfusepath{clip}%
\pgfsetbuttcap%
\pgfsetroundjoin%
\definecolor{currentfill}{rgb}{0.892138,0.425389,0.333289}%
\pgfsetfillcolor{currentfill}%
\pgfsetlinewidth{0.000000pt}%
\definecolor{currentstroke}{rgb}{0.000000,0.000000,0.000000}%
\pgfsetstrokecolor{currentstroke}%
\pgfsetdash{}{0pt}%
\pgfpathmoveto{\pgfqpoint{3.968076in}{1.994042in}}%
\pgfpathlineto{\pgfqpoint{4.226975in}{2.236848in}}%
\pgfpathlineto{\pgfqpoint{3.968076in}{1.994042in}}%
\pgfpathclose%
\pgfusepath{fill}%
\end{pgfscope}%
\begin{pgfscope}%
\pgfpathrectangle{\pgfqpoint{0.017500in}{0.017500in}}{\pgfqpoint{6.218282in}{3.580372in}} %
\pgfusepath{clip}%
\pgfsetbuttcap%
\pgfsetroundjoin%
\definecolor{currentfill}{rgb}{0.892138,0.425389,0.333289}%
\pgfsetfillcolor{currentfill}%
\pgfsetlinewidth{0.000000pt}%
\definecolor{currentstroke}{rgb}{0.000000,0.000000,0.000000}%
\pgfsetstrokecolor{currentstroke}%
\pgfsetdash{}{0pt}%
\pgfpathmoveto{\pgfqpoint{3.968076in}{1.994042in}}%
\pgfpathlineto{\pgfqpoint{4.226975in}{2.236848in}}%
\pgfpathlineto{\pgfqpoint{3.968076in}{1.994042in}}%
\pgfpathclose%
\pgfusepath{fill}%
\end{pgfscope}%
\end{pgfpicture}%
\makeatother%
\endgroup%

	\caption{Learning surface for SemEval and CONLL2000 Chunk.}
\end{figure}
\newpage
\vspace*{2cm}
\begin{figure}[h!]
	\hspace*{-2cm}%% Creator: Matplotlib, PGF backend
%%
%% To include the figure in your LaTeX document, write
%%   \input{<filename>.pgf}
%%
%% Make sure the required packages are loaded in your preamble
%%   \usepackage{pgf}
%%
%% Figures using additional raster images can only be included by \input if
%% they are in the same directory as the main LaTeX file. For loading figures
%% from other directories you can use the `import` package
%%   \usepackage{import}
%% and then include the figures with
%%   \import{<path to file>}{<filename>.pgf}
%%
%% Matplotlib used the following preamble
%%   \usepackage{fontspec}
%%   \setmainfont{Palatino}
%%   \setsansfont{Lucida Grande}
%%   \setmonofont{Andale Mono}
%%
\begingroup%
\makeatletter%
\begin{pgfpicture}%
\pgfpathrectangle{\pgfpointorigin}{\pgfqpoint{6.253282in}{3.615372in}}%
\pgfusepath{use as bounding box, clip}%
\begin{pgfscope}%
\pgfsetbuttcap%
\pgfsetmiterjoin%
\definecolor{currentfill}{rgb}{1.000000,1.000000,1.000000}%
\pgfsetfillcolor{currentfill}%
\pgfsetlinewidth{0.000000pt}%
\definecolor{currentstroke}{rgb}{1.000000,1.000000,1.000000}%
\pgfsetstrokecolor{currentstroke}%
\pgfsetdash{}{0pt}%
\pgfpathmoveto{\pgfqpoint{0.000000in}{0.000000in}}%
\pgfpathlineto{\pgfqpoint{6.253282in}{0.000000in}}%
\pgfpathlineto{\pgfqpoint{6.253282in}{3.615372in}}%
\pgfpathlineto{\pgfqpoint{0.000000in}{3.615372in}}%
\pgfpathclose%
\pgfusepath{fill}%
\end{pgfscope}%
\begin{pgfscope}%
\pgfsetbuttcap%
\pgfsetmiterjoin%
\definecolor{currentfill}{rgb}{1.000000,1.000000,1.000000}%
\pgfsetfillcolor{currentfill}%
\pgfsetlinewidth{0.000000pt}%
\definecolor{currentstroke}{rgb}{0.000000,0.000000,0.000000}%
\pgfsetstrokecolor{currentstroke}%
\pgfsetstrokeopacity{0.000000}%
\pgfsetdash{}{0pt}%
\pgfpathmoveto{\pgfqpoint{0.017500in}{0.017500in}}%
\pgfpathlineto{\pgfqpoint{6.235782in}{0.017500in}}%
\pgfpathlineto{\pgfqpoint{6.235782in}{3.597872in}}%
\pgfpathlineto{\pgfqpoint{0.017500in}{3.597872in}}%
\pgfpathclose%
\pgfusepath{fill}%
\end{pgfscope}%
\begin{pgfscope}%
\pgfsetbuttcap%
\pgfsetmiterjoin%
\definecolor{currentfill}{rgb}{0.950000,0.950000,0.950000}%
\pgfsetfillcolor{currentfill}%
\pgfsetfillopacity{0.500000}%
\pgfsetlinewidth{0.200750pt}%
\definecolor{currentstroke}{rgb}{0.950000,0.950000,0.950000}%
\pgfsetstrokecolor{currentstroke}%
\pgfsetstrokeopacity{0.500000}%
\pgfsetdash{}{0pt}%
\pgfpathmoveto{\pgfqpoint{1.233525in}{0.981833in}}%
\pgfpathlineto{\pgfqpoint{2.713039in}{1.712806in}}%
\pgfpathlineto{\pgfqpoint{2.692063in}{3.119783in}}%
\pgfpathlineto{\pgfqpoint{1.144557in}{2.450885in}}%
\pgfusepath{stroke,fill}%
\end{pgfscope}%
\begin{pgfscope}%
\pgfsetbuttcap%
\pgfsetmiterjoin%
\definecolor{currentfill}{rgb}{0.900000,0.900000,0.900000}%
\pgfsetfillcolor{currentfill}%
\pgfsetfillopacity{0.500000}%
\pgfsetlinewidth{0.200750pt}%
\definecolor{currentstroke}{rgb}{0.900000,0.900000,0.900000}%
\pgfsetstrokecolor{currentstroke}%
\pgfsetstrokeopacity{0.500000}%
\pgfsetdash{}{0pt}%
\pgfpathmoveto{\pgfqpoint{2.713039in}{1.712806in}}%
\pgfpathlineto{\pgfqpoint{5.135278in}{1.301993in}}%
\pgfpathlineto{\pgfqpoint{5.219480in}{2.744306in}}%
\pgfpathlineto{\pgfqpoint{2.692063in}{3.119783in}}%
\pgfusepath{stroke,fill}%
\end{pgfscope}%
\begin{pgfscope}%
\pgfsetbuttcap%
\pgfsetmiterjoin%
\definecolor{currentfill}{rgb}{0.925000,0.925000,0.925000}%
\pgfsetfillcolor{currentfill}%
\pgfsetfillopacity{0.500000}%
\pgfsetlinewidth{0.200750pt}%
\definecolor{currentstroke}{rgb}{0.925000,0.925000,0.925000}%
\pgfsetstrokecolor{currentstroke}%
\pgfsetstrokeopacity{0.500000}%
\pgfsetdash{}{0pt}%
\pgfpathmoveto{\pgfqpoint{1.233525in}{0.981833in}}%
\pgfpathlineto{\pgfqpoint{3.760968in}{0.515142in}}%
\pgfpathlineto{\pgfqpoint{5.135278in}{1.301993in}}%
\pgfpathlineto{\pgfqpoint{2.713039in}{1.712806in}}%
\pgfusepath{stroke,fill}%
\end{pgfscope}%
\begin{pgfscope}%
\pgfsetrectcap%
\pgfsetroundjoin%
\pgfsetlinewidth{0.501875pt}%
\definecolor{currentstroke}{rgb}{0.000000,0.000000,0.000000}%
\pgfsetstrokecolor{currentstroke}%
\pgfsetdash{}{0pt}%
\pgfpathmoveto{\pgfqpoint{1.233525in}{0.981833in}}%
\pgfpathlineto{\pgfqpoint{3.760968in}{0.515142in}}%
\pgfusepath{stroke}%
\end{pgfscope}%
\begin{pgfscope}%
\pgftext[x=2.292998in,y=0.574477in,left,base,rotate=349.538214]{\rmfamily\fontsize{10.000000}{12.000000}\selectfont }%
\end{pgfscope}%
\begin{pgfscope}%
\pgftext[x=1.953595in,y=0.474946in,left,base,rotate=349.538214]{\rmfamily\fontsize{10.000000}{12.000000}\selectfont fraction of}%
\end{pgfscope}%
\begin{pgfscope}%
\pgftext[x=1.482715in,y=0.399693in,left,base,rotate=349.538214]{\rmfamily\fontsize{10.000000}{12.000000}\selectfont CONLL2000 Chunk data}%
\end{pgfscope}%
\begin{pgfscope}%
\pgfsetbuttcap%
\pgfsetroundjoin%
\pgfsetlinewidth{0.803000pt}%
\definecolor{currentstroke}{rgb}{0.690196,0.690196,0.690196}%
\pgfsetstrokecolor{currentstroke}%
\pgfsetdash{}{0pt}%
\pgfpathmoveto{\pgfqpoint{1.388733in}{0.953174in}}%
\pgfpathlineto{\pgfqpoint{2.862115in}{1.687522in}}%
\pgfpathlineto{\pgfqpoint{2.847388in}{3.096708in}}%
\pgfusepath{stroke}%
\end{pgfscope}%
\begin{pgfscope}%
\pgfsetbuttcap%
\pgfsetroundjoin%
\pgfsetlinewidth{0.803000pt}%
\definecolor{currentstroke}{rgb}{0.690196,0.690196,0.690196}%
\pgfsetstrokecolor{currentstroke}%
\pgfsetdash{}{0pt}%
\pgfpathmoveto{\pgfqpoint{1.818260in}{0.873862in}}%
\pgfpathlineto{\pgfqpoint{3.274447in}{1.617591in}}%
\pgfpathlineto{\pgfqpoint{3.277159in}{3.032860in}}%
\pgfusepath{stroke}%
\end{pgfscope}%
\begin{pgfscope}%
\pgfsetbuttcap%
\pgfsetroundjoin%
\pgfsetlinewidth{0.803000pt}%
\definecolor{currentstroke}{rgb}{0.690196,0.690196,0.690196}%
\pgfsetstrokecolor{currentstroke}%
\pgfsetdash{}{0pt}%
\pgfpathmoveto{\pgfqpoint{2.253467in}{0.793501in}}%
\pgfpathlineto{\pgfqpoint{3.691895in}{1.546791in}}%
\pgfpathlineto{\pgfqpoint{3.712491in}{2.968187in}}%
\pgfusepath{stroke}%
\end{pgfscope}%
\begin{pgfscope}%
\pgfsetbuttcap%
\pgfsetroundjoin%
\pgfsetlinewidth{0.803000pt}%
\definecolor{currentstroke}{rgb}{0.690196,0.690196,0.690196}%
\pgfsetstrokecolor{currentstroke}%
\pgfsetdash{}{0pt}%
\pgfpathmoveto{\pgfqpoint{2.694467in}{0.712071in}}%
\pgfpathlineto{\pgfqpoint{4.114556in}{1.475108in}}%
\pgfpathlineto{\pgfqpoint{4.153492in}{2.902671in}}%
\pgfusepath{stroke}%
\end{pgfscope}%
\begin{pgfscope}%
\pgfsetbuttcap%
\pgfsetroundjoin%
\pgfsetlinewidth{0.803000pt}%
\definecolor{currentstroke}{rgb}{0.690196,0.690196,0.690196}%
\pgfsetstrokecolor{currentstroke}%
\pgfsetdash{}{0pt}%
\pgfpathmoveto{\pgfqpoint{3.141375in}{0.629550in}}%
\pgfpathlineto{\pgfqpoint{4.542526in}{1.402524in}}%
\pgfpathlineto{\pgfqpoint{4.600274in}{2.836296in}}%
\pgfusepath{stroke}%
\end{pgfscope}%
\begin{pgfscope}%
\pgfsetbuttcap%
\pgfsetroundjoin%
\pgfsetlinewidth{0.803000pt}%
\definecolor{currentstroke}{rgb}{0.690196,0.690196,0.690196}%
\pgfsetstrokecolor{currentstroke}%
\pgfsetdash{}{0pt}%
\pgfpathmoveto{\pgfqpoint{3.594311in}{0.545915in}}%
\pgfpathlineto{\pgfqpoint{4.975908in}{1.329022in}}%
\pgfpathlineto{\pgfqpoint{5.052952in}{2.769046in}}%
\pgfusepath{stroke}%
\end{pgfscope}%
\begin{pgfscope}%
\pgfsetrectcap%
\pgfsetroundjoin%
\pgfsetlinewidth{0.803000pt}%
\definecolor{currentstroke}{rgb}{0.000000,0.000000,0.000000}%
\pgfsetstrokecolor{currentstroke}%
\pgfsetdash{}{0pt}%
\pgfpathmoveto{\pgfqpoint{1.401276in}{0.959425in}}%
\pgfpathlineto{\pgfqpoint{1.363606in}{0.940650in}}%
\pgfusepath{stroke}%
\end{pgfscope}%
\begin{pgfscope}%
\pgftext[x=1.295334in,y=0.815921in,,top]{\rmfamily\fontsize{8.000000}{9.600000}\selectfont 0.0}%
\end{pgfscope}%
\begin{pgfscope}%
\pgfsetrectcap%
\pgfsetroundjoin%
\pgfsetlinewidth{0.803000pt}%
\definecolor{currentstroke}{rgb}{0.000000,0.000000,0.000000}%
\pgfsetstrokecolor{currentstroke}%
\pgfsetdash{}{0pt}%
\pgfpathmoveto{\pgfqpoint{1.830663in}{0.880196in}}%
\pgfpathlineto{\pgfqpoint{1.793417in}{0.861173in}}%
\pgfusepath{stroke}%
\end{pgfscope}%
\begin{pgfscope}%
\pgftext[x=1.724832in,y=0.735598in,,top]{\rmfamily\fontsize{8.000000}{9.600000}\selectfont 0.2}%
\end{pgfscope}%
\begin{pgfscope}%
\pgfsetrectcap%
\pgfsetroundjoin%
\pgfsetlinewidth{0.803000pt}%
\definecolor{currentstroke}{rgb}{0.000000,0.000000,0.000000}%
\pgfsetstrokecolor{currentstroke}%
\pgfsetdash{}{0pt}%
\pgfpathmoveto{\pgfqpoint{2.265723in}{0.799919in}}%
\pgfpathlineto{\pgfqpoint{2.228917in}{0.780644in}}%
\pgfusepath{stroke}%
\end{pgfscope}%
\begin{pgfscope}%
\pgftext[x=2.160017in,y=0.654212in,,top]{\rmfamily\fontsize{8.000000}{9.600000}\selectfont 0.4}%
\end{pgfscope}%
\begin{pgfscope}%
\pgfsetrectcap%
\pgfsetroundjoin%
\pgfsetlinewidth{0.803000pt}%
\definecolor{currentstroke}{rgb}{0.000000,0.000000,0.000000}%
\pgfsetstrokecolor{currentstroke}%
\pgfsetdash{}{0pt}%
\pgfpathmoveto{\pgfqpoint{2.706571in}{0.718575in}}%
\pgfpathlineto{\pgfqpoint{2.670219in}{0.699042in}}%
\pgfusepath{stroke}%
\end{pgfscope}%
\begin{pgfscope}%
\pgftext[x=2.601002in,y=0.571741in,,top]{\rmfamily\fontsize{8.000000}{9.600000}\selectfont 0.6}%
\end{pgfscope}%
\begin{pgfscope}%
\pgfsetrectcap%
\pgfsetroundjoin%
\pgfsetlinewidth{0.803000pt}%
\definecolor{currentstroke}{rgb}{0.000000,0.000000,0.000000}%
\pgfsetstrokecolor{currentstroke}%
\pgfsetdash{}{0pt}%
\pgfpathmoveto{\pgfqpoint{3.153323in}{0.636141in}}%
\pgfpathlineto{\pgfqpoint{3.117440in}{0.616346in}}%
\pgfusepath{stroke}%
\end{pgfscope}%
\begin{pgfscope}%
\pgftext[x=3.047904in,y=0.488164in,,top]{\rmfamily\fontsize{8.000000}{9.600000}\selectfont 0.8}%
\end{pgfscope}%
\begin{pgfscope}%
\pgfsetrectcap%
\pgfsetroundjoin%
\pgfsetlinewidth{0.803000pt}%
\definecolor{currentstroke}{rgb}{0.000000,0.000000,0.000000}%
\pgfsetstrokecolor{currentstroke}%
\pgfsetdash{}{0pt}%
\pgfpathmoveto{\pgfqpoint{3.606097in}{0.552596in}}%
\pgfpathlineto{\pgfqpoint{3.570701in}{0.532532in}}%
\pgfusepath{stroke}%
\end{pgfscope}%
\begin{pgfscope}%
\pgftext[x=3.500842in,y=0.403458in,,top]{\rmfamily\fontsize{8.000000}{9.600000}\selectfont 1.0}%
\end{pgfscope}%
\begin{pgfscope}%
\pgfsetrectcap%
\pgfsetroundjoin%
\pgfsetlinewidth{0.501875pt}%
\definecolor{currentstroke}{rgb}{0.000000,0.000000,0.000000}%
\pgfsetstrokecolor{currentstroke}%
\pgfsetdash{}{0pt}%
\pgfpathmoveto{\pgfqpoint{5.135278in}{1.301993in}}%
\pgfpathlineto{\pgfqpoint{3.760968in}{0.515142in}}%
\pgfusepath{stroke}%
\end{pgfscope}%
\begin{pgfscope}%
\pgftext[x=4.748349in,y=0.779064in,left,base,rotate=29.792981]{\rmfamily\fontsize{10.000000}{12.000000}\selectfont }%
\end{pgfscope}%
\begin{pgfscope}%
\pgftext[x=4.553640in,y=0.483787in,left,base,rotate=29.792981]{\rmfamily\fontsize{10.000000}{12.000000}\selectfont fraction of}%
\end{pgfscope}%
\begin{pgfscope}%
\pgftext[x=4.542818in,y=0.293793in,left,base,rotate=29.792981]{\rmfamily\fontsize{10.000000}{12.000000}\selectfont SemEval data}%
\end{pgfscope}%
\begin{pgfscope}%
\pgfsetbuttcap%
\pgfsetroundjoin%
\pgfsetlinewidth{0.803000pt}%
\definecolor{currentstroke}{rgb}{0.690196,0.690196,0.690196}%
\pgfsetstrokecolor{currentstroke}%
\pgfsetdash{}{0pt}%
\pgfpathmoveto{\pgfqpoint{1.249232in}{2.496130in}}%
\pgfpathlineto{\pgfqpoint{1.333347in}{1.031151in}}%
\pgfpathlineto{\pgfqpoint{3.853895in}{0.568347in}}%
\pgfusepath{stroke}%
\end{pgfscope}%
\begin{pgfscope}%
\pgfsetbuttcap%
\pgfsetroundjoin%
\pgfsetlinewidth{0.803000pt}%
\definecolor{currentstroke}{rgb}{0.690196,0.690196,0.690196}%
\pgfsetstrokecolor{currentstroke}%
\pgfsetdash{}{0pt}%
\pgfpathmoveto{\pgfqpoint{1.531798in}{2.618267in}}%
\pgfpathlineto{\pgfqpoint{1.602996in}{1.164374in}}%
\pgfpathlineto{\pgfqpoint{4.104771in}{0.711984in}}%
\pgfusepath{stroke}%
\end{pgfscope}%
\begin{pgfscope}%
\pgfsetbuttcap%
\pgfsetroundjoin%
\pgfsetlinewidth{0.803000pt}%
\definecolor{currentstroke}{rgb}{0.690196,0.690196,0.690196}%
\pgfsetstrokecolor{currentstroke}%
\pgfsetdash{}{0pt}%
\pgfpathmoveto{\pgfqpoint{1.807886in}{2.737604in}}%
\pgfpathlineto{\pgfqpoint{1.866722in}{1.294672in}}%
\pgfpathlineto{\pgfqpoint{4.349930in}{0.852348in}}%
\pgfusepath{stroke}%
\end{pgfscope}%
\begin{pgfscope}%
\pgfsetbuttcap%
\pgfsetroundjoin%
\pgfsetlinewidth{0.803000pt}%
\definecolor{currentstroke}{rgb}{0.690196,0.690196,0.690196}%
\pgfsetstrokecolor{currentstroke}%
\pgfsetdash{}{0pt}%
\pgfpathmoveto{\pgfqpoint{2.077717in}{2.854237in}}%
\pgfpathlineto{\pgfqpoint{2.124720in}{1.422139in}}%
\pgfpathlineto{\pgfqpoint{4.589565in}{0.989549in}}%
\pgfusepath{stroke}%
\end{pgfscope}%
\begin{pgfscope}%
\pgfsetbuttcap%
\pgfsetroundjoin%
\pgfsetlinewidth{0.803000pt}%
\definecolor{currentstroke}{rgb}{0.690196,0.690196,0.690196}%
\pgfsetstrokecolor{currentstroke}%
\pgfsetdash{}{0pt}%
\pgfpathmoveto{\pgfqpoint{2.341501in}{2.968255in}}%
\pgfpathlineto{\pgfqpoint{2.377173in}{1.546866in}}%
\pgfpathlineto{\pgfqpoint{4.823859in}{1.123693in}}%
\pgfusepath{stroke}%
\end{pgfscope}%
\begin{pgfscope}%
\pgfsetbuttcap%
\pgfsetroundjoin%
\pgfsetlinewidth{0.803000pt}%
\definecolor{currentstroke}{rgb}{0.690196,0.690196,0.690196}%
\pgfsetstrokecolor{currentstroke}%
\pgfsetdash{}{0pt}%
\pgfpathmoveto{\pgfqpoint{2.599439in}{3.079747in}}%
\pgfpathlineto{\pgfqpoint{2.624258in}{1.668942in}}%
\pgfpathlineto{\pgfqpoint{5.052991in}{1.254881in}}%
\pgfusepath{stroke}%
\end{pgfscope}%
\begin{pgfscope}%
\pgfsetrectcap%
\pgfsetroundjoin%
\pgfsetlinewidth{0.803000pt}%
\definecolor{currentstroke}{rgb}{0.000000,0.000000,0.000000}%
\pgfsetstrokecolor{currentstroke}%
\pgfsetdash{}{0pt}%
\pgfpathmoveto{\pgfqpoint{3.832959in}{0.572191in}}%
\pgfpathlineto{\pgfqpoint{3.895805in}{0.560652in}}%
\pgfusepath{stroke}%
\end{pgfscope}%
\begin{pgfscope}%
\pgftext[x=4.015897in,y=0.448445in,,top]{\rmfamily\fontsize{8.000000}{9.600000}\selectfont 0.0}%
\end{pgfscope}%
\begin{pgfscope}%
\pgfsetrectcap%
\pgfsetroundjoin%
\pgfsetlinewidth{0.803000pt}%
\definecolor{currentstroke}{rgb}{0.000000,0.000000,0.000000}%
\pgfsetstrokecolor{currentstroke}%
\pgfsetdash{}{0pt}%
\pgfpathmoveto{\pgfqpoint{4.084000in}{0.715740in}}%
\pgfpathlineto{\pgfqpoint{4.146351in}{0.704465in}}%
\pgfusepath{stroke}%
\end{pgfscope}%
\begin{pgfscope}%
\pgftext[x=4.264814in,y=0.593411in,,top]{\rmfamily\fontsize{8.000000}{9.600000}\selectfont 0.2}%
\end{pgfscope}%
\begin{pgfscope}%
\pgfsetrectcap%
\pgfsetroundjoin%
\pgfsetlinewidth{0.803000pt}%
\definecolor{currentstroke}{rgb}{0.000000,0.000000,0.000000}%
\pgfsetstrokecolor{currentstroke}%
\pgfsetdash{}{0pt}%
\pgfpathmoveto{\pgfqpoint{4.329322in}{0.856019in}}%
\pgfpathlineto{\pgfqpoint{4.391184in}{0.845000in}}%
\pgfusepath{stroke}%
\end{pgfscope}%
\begin{pgfscope}%
\pgftext[x=4.508060in,y=0.735074in,,top]{\rmfamily\fontsize{8.000000}{9.600000}\selectfont 0.4}%
\end{pgfscope}%
\begin{pgfscope}%
\pgfsetrectcap%
\pgfsetroundjoin%
\pgfsetlinewidth{0.803000pt}%
\definecolor{currentstroke}{rgb}{0.000000,0.000000,0.000000}%
\pgfsetstrokecolor{currentstroke}%
\pgfsetdash{}{0pt}%
\pgfpathmoveto{\pgfqpoint{4.569117in}{0.993138in}}%
\pgfpathlineto{\pgfqpoint{4.630496in}{0.982365in}}%
\pgfusepath{stroke}%
\end{pgfscope}%
\begin{pgfscope}%
\pgftext[x=4.745827in,y=0.873546in,,top]{\rmfamily\fontsize{8.000000}{9.600000}\selectfont 0.6}%
\end{pgfscope}%
\begin{pgfscope}%
\pgfsetrectcap%
\pgfsetroundjoin%
\pgfsetlinewidth{0.803000pt}%
\definecolor{currentstroke}{rgb}{0.000000,0.000000,0.000000}%
\pgfsetstrokecolor{currentstroke}%
\pgfsetdash{}{0pt}%
\pgfpathmoveto{\pgfqpoint{4.803571in}{1.127202in}}%
\pgfpathlineto{\pgfqpoint{4.864473in}{1.116668in}}%
\pgfusepath{stroke}%
\end{pgfscope}%
\begin{pgfscope}%
\pgftext[x=4.978298in,y=1.008934in,,top]{\rmfamily\fontsize{8.000000}{9.600000}\selectfont 0.8}%
\end{pgfscope}%
\begin{pgfscope}%
\pgfsetrectcap%
\pgfsetroundjoin%
\pgfsetlinewidth{0.803000pt}%
\definecolor{currentstroke}{rgb}{0.000000,0.000000,0.000000}%
\pgfsetstrokecolor{currentstroke}%
\pgfsetdash{}{0pt}%
\pgfpathmoveto{\pgfqpoint{5.032859in}{1.258313in}}%
\pgfpathlineto{\pgfqpoint{5.093291in}{1.248010in}}%
\pgfusepath{stroke}%
\end{pgfscope}%
\begin{pgfscope}%
\pgftext[x=5.205648in,y=1.141340in,,top]{\rmfamily\fontsize{8.000000}{9.600000}\selectfont 1.0}%
\end{pgfscope}%
\begin{pgfscope}%
\pgfsetrectcap%
\pgfsetroundjoin%
\pgfsetlinewidth{0.501875pt}%
\definecolor{currentstroke}{rgb}{0.000000,0.000000,0.000000}%
\pgfsetstrokecolor{currentstroke}%
\pgfsetdash{}{0pt}%
\pgfpathmoveto{\pgfqpoint{5.135278in}{1.301993in}}%
\pgfpathlineto{\pgfqpoint{5.219480in}{2.744306in}}%
\pgfusepath{stroke}%
\end{pgfscope}%
\begin{pgfscope}%
\pgftext[x=5.653991in,y=2.051089in,left,base,rotate=86.658871]{\rmfamily\fontsize{10.000000}{12.000000}\selectfont }%
\end{pgfscope}%
\begin{pgfscope}%
\pgftext[x=5.798047in,y=1.781819in,left,base,rotate=86.658871]{\rmfamily\fontsize{10.000000}{12.000000}\selectfont mean F1}%
\end{pgfscope}%
\begin{pgfscope}%
\pgfsetbuttcap%
\pgfsetroundjoin%
\pgfsetlinewidth{0.803000pt}%
\definecolor{currentstroke}{rgb}{0.690196,0.690196,0.690196}%
\pgfsetstrokecolor{currentstroke}%
\pgfsetdash{}{0pt}%
\pgfpathmoveto{\pgfqpoint{5.143397in}{1.441079in}}%
\pgfpathlineto{\pgfqpoint{2.711014in}{1.848672in}}%
\pgfpathlineto{\pgfqpoint{1.224955in}{1.123344in}}%
\pgfusepath{stroke}%
\end{pgfscope}%
\begin{pgfscope}%
\pgfsetbuttcap%
\pgfsetroundjoin%
\pgfsetlinewidth{0.803000pt}%
\definecolor{currentstroke}{rgb}{0.690196,0.690196,0.690196}%
\pgfsetstrokecolor{currentstroke}%
\pgfsetdash{}{0pt}%
\pgfpathmoveto{\pgfqpoint{5.155442in}{1.647401in}}%
\pgfpathlineto{\pgfqpoint{2.708010in}{2.050144in}}%
\pgfpathlineto{\pgfqpoint{1.212238in}{1.333324in}}%
\pgfusepath{stroke}%
\end{pgfscope}%
\begin{pgfscope}%
\pgfsetbuttcap%
\pgfsetroundjoin%
\pgfsetlinewidth{0.803000pt}%
\definecolor{currentstroke}{rgb}{0.690196,0.690196,0.690196}%
\pgfsetstrokecolor{currentstroke}%
\pgfsetdash{}{0pt}%
\pgfpathmoveto{\pgfqpoint{5.167639in}{1.856311in}}%
\pgfpathlineto{\pgfqpoint{2.704970in}{2.254053in}}%
\pgfpathlineto{\pgfqpoint{1.199357in}{1.546011in}}%
\pgfusepath{stroke}%
\end{pgfscope}%
\begin{pgfscope}%
\pgfsetbuttcap%
\pgfsetroundjoin%
\pgfsetlinewidth{0.803000pt}%
\definecolor{currentstroke}{rgb}{0.690196,0.690196,0.690196}%
\pgfsetstrokecolor{currentstroke}%
\pgfsetdash{}{0pt}%
\pgfpathmoveto{\pgfqpoint{5.179989in}{2.067857in}}%
\pgfpathlineto{\pgfqpoint{2.701893in}{2.460443in}}%
\pgfpathlineto{\pgfqpoint{1.186310in}{1.761458in}}%
\pgfusepath{stroke}%
\end{pgfscope}%
\begin{pgfscope}%
\pgfsetbuttcap%
\pgfsetroundjoin%
\pgfsetlinewidth{0.803000pt}%
\definecolor{currentstroke}{rgb}{0.690196,0.690196,0.690196}%
\pgfsetstrokecolor{currentstroke}%
\pgfsetdash{}{0pt}%
\pgfpathmoveto{\pgfqpoint{5.192496in}{2.282090in}}%
\pgfpathlineto{\pgfqpoint{2.698778in}{2.669360in}}%
\pgfpathlineto{\pgfqpoint{1.173091in}{1.979718in}}%
\pgfusepath{stroke}%
\end{pgfscope}%
\begin{pgfscope}%
\pgfsetbuttcap%
\pgfsetroundjoin%
\pgfsetlinewidth{0.803000pt}%
\definecolor{currentstroke}{rgb}{0.690196,0.690196,0.690196}%
\pgfsetstrokecolor{currentstroke}%
\pgfsetdash{}{0pt}%
\pgfpathmoveto{\pgfqpoint{5.205162in}{2.499062in}}%
\pgfpathlineto{\pgfqpoint{2.695625in}{2.880851in}}%
\pgfpathlineto{\pgfqpoint{1.159699in}{2.200846in}}%
\pgfusepath{stroke}%
\end{pgfscope}%
\begin{pgfscope}%
\pgfsetrectcap%
\pgfsetroundjoin%
\pgfsetlinewidth{0.803000pt}%
\definecolor{currentstroke}{rgb}{0.000000,0.000000,0.000000}%
\pgfsetstrokecolor{currentstroke}%
\pgfsetdash{}{0pt}%
\pgfpathmoveto{\pgfqpoint{5.123235in}{1.444458in}}%
\pgfpathlineto{\pgfqpoint{5.183758in}{1.434316in}}%
\pgfusepath{stroke}%
\end{pgfscope}%
\begin{pgfscope}%
\pgftext[x=5.381724in,y=1.460404in,,top]{\rmfamily\fontsize{8.000000}{9.600000}\selectfont 0.1}%
\end{pgfscope}%
\begin{pgfscope}%
\pgfsetrectcap%
\pgfsetroundjoin%
\pgfsetlinewidth{0.803000pt}%
\definecolor{currentstroke}{rgb}{0.000000,0.000000,0.000000}%
\pgfsetstrokecolor{currentstroke}%
\pgfsetdash{}{0pt}%
\pgfpathmoveto{\pgfqpoint{5.135151in}{1.650740in}}%
\pgfpathlineto{\pgfqpoint{5.196062in}{1.640717in}}%
\pgfusepath{stroke}%
\end{pgfscope}%
\begin{pgfscope}%
\pgftext[x=5.395231in,y=1.666500in,,top]{\rmfamily\fontsize{8.000000}{9.600000}\selectfont 0.2}%
\end{pgfscope}%
\begin{pgfscope}%
\pgfsetrectcap%
\pgfsetroundjoin%
\pgfsetlinewidth{0.803000pt}%
\definecolor{currentstroke}{rgb}{0.000000,0.000000,0.000000}%
\pgfsetstrokecolor{currentstroke}%
\pgfsetdash{}{0pt}%
\pgfpathmoveto{\pgfqpoint{5.147216in}{1.859609in}}%
\pgfpathlineto{\pgfqpoint{5.208520in}{1.849708in}}%
\pgfusepath{stroke}%
\end{pgfscope}%
\begin{pgfscope}%
\pgftext[x=5.408908in,y=1.875177in,,top]{\rmfamily\fontsize{8.000000}{9.600000}\selectfont 0.3}%
\end{pgfscope}%
\begin{pgfscope}%
\pgfsetrectcap%
\pgfsetroundjoin%
\pgfsetlinewidth{0.803000pt}%
\definecolor{currentstroke}{rgb}{0.000000,0.000000,0.000000}%
\pgfsetstrokecolor{currentstroke}%
\pgfsetdash{}{0pt}%
\pgfpathmoveto{\pgfqpoint{5.159434in}{2.071113in}}%
\pgfpathlineto{\pgfqpoint{5.221135in}{2.061338in}}%
\pgfusepath{stroke}%
\end{pgfscope}%
\begin{pgfscope}%
\pgftext[x=5.422757in,y=2.086482in,,top]{\rmfamily\fontsize{8.000000}{9.600000}\selectfont 0.4}%
\end{pgfscope}%
\begin{pgfscope}%
\pgfsetrectcap%
\pgfsetroundjoin%
\pgfsetlinewidth{0.803000pt}%
\definecolor{currentstroke}{rgb}{0.000000,0.000000,0.000000}%
\pgfsetstrokecolor{currentstroke}%
\pgfsetdash{}{0pt}%
\pgfpathmoveto{\pgfqpoint{5.171806in}{2.285303in}}%
\pgfpathlineto{\pgfqpoint{5.233911in}{2.275658in}}%
\pgfusepath{stroke}%
\end{pgfscope}%
\begin{pgfscope}%
\pgftext[x=5.436782in,y=2.300467in,,top]{\rmfamily\fontsize{8.000000}{9.600000}\selectfont 0.5}%
\end{pgfscope}%
\begin{pgfscope}%
\pgfsetrectcap%
\pgfsetroundjoin%
\pgfsetlinewidth{0.803000pt}%
\definecolor{currentstroke}{rgb}{0.000000,0.000000,0.000000}%
\pgfsetstrokecolor{currentstroke}%
\pgfsetdash{}{0pt}%
\pgfpathmoveto{\pgfqpoint{5.184337in}{2.502230in}}%
\pgfpathlineto{\pgfqpoint{5.246851in}{2.492720in}}%
\pgfusepath{stroke}%
\end{pgfscope}%
\begin{pgfscope}%
\pgftext[x=5.450985in,y=2.517183in,,top]{\rmfamily\fontsize{8.000000}{9.600000}\selectfont 0.6}%
\end{pgfscope}%
\begin{pgfscope}%
\pgfpathrectangle{\pgfqpoint{0.017500in}{0.017500in}}{\pgfqpoint{6.218282in}{3.580372in}} %
\pgfusepath{clip}%
\pgfsetbuttcap%
\pgfsetroundjoin%
\definecolor{currentfill}{rgb}{0.906154,0.842091,0.806151}%
\pgfsetfillcolor{currentfill}%
\pgfsetlinewidth{0.000000pt}%
\definecolor{currentstroke}{rgb}{0.000000,0.000000,0.000000}%
\pgfsetstrokecolor{currentstroke}%
\pgfsetdash{}{0pt}%
\pgfpathmoveto{\pgfqpoint{2.755563in}{3.027050in}}%
\pgfpathlineto{\pgfqpoint{1.486050in}{1.042638in}}%
\pgfpathlineto{\pgfqpoint{2.755563in}{3.027050in}}%
\pgfpathclose%
\pgfusepath{fill}%
\end{pgfscope}%
\begin{pgfscope}%
\pgfpathrectangle{\pgfqpoint{0.017500in}{0.017500in}}{\pgfqpoint{6.218282in}{3.580372in}} %
\pgfusepath{clip}%
\pgfsetbuttcap%
\pgfsetroundjoin%
\definecolor{currentfill}{rgb}{0.906154,0.842091,0.806151}%
\pgfsetfillcolor{currentfill}%
\pgfsetlinewidth{0.000000pt}%
\definecolor{currentstroke}{rgb}{0.000000,0.000000,0.000000}%
\pgfsetstrokecolor{currentstroke}%
\pgfsetdash{}{0pt}%
\pgfpathmoveto{\pgfqpoint{2.755563in}{3.027050in}}%
\pgfpathlineto{\pgfqpoint{1.486050in}{1.042638in}}%
\pgfpathlineto{\pgfqpoint{2.755563in}{3.027050in}}%
\pgfpathclose%
\pgfusepath{fill}%
\end{pgfscope}%
\begin{pgfscope}%
\pgfpathrectangle{\pgfqpoint{0.017500in}{0.017500in}}{\pgfqpoint{6.218282in}{3.580372in}} %
\pgfusepath{clip}%
\pgfsetbuttcap%
\pgfsetroundjoin%
\definecolor{currentfill}{rgb}{0.906154,0.842091,0.806151}%
\pgfsetfillcolor{currentfill}%
\pgfsetlinewidth{0.000000pt}%
\definecolor{currentstroke}{rgb}{0.000000,0.000000,0.000000}%
\pgfsetstrokecolor{currentstroke}%
\pgfsetdash{}{0pt}%
\pgfpathmoveto{\pgfqpoint{2.755563in}{3.027050in}}%
\pgfpathlineto{\pgfqpoint{1.486050in}{1.042638in}}%
\pgfpathlineto{\pgfqpoint{2.755563in}{3.027050in}}%
\pgfpathclose%
\pgfusepath{fill}%
\end{pgfscope}%
\begin{pgfscope}%
\pgfpathrectangle{\pgfqpoint{0.017500in}{0.017500in}}{\pgfqpoint{6.218282in}{3.580372in}} %
\pgfusepath{clip}%
\pgfsetbuttcap%
\pgfsetroundjoin%
\definecolor{currentfill}{rgb}{0.906154,0.842091,0.806151}%
\pgfsetfillcolor{currentfill}%
\pgfsetlinewidth{0.000000pt}%
\definecolor{currentstroke}{rgb}{0.000000,0.000000,0.000000}%
\pgfsetstrokecolor{currentstroke}%
\pgfsetdash{}{0pt}%
\pgfpathmoveto{\pgfqpoint{2.755563in}{3.027050in}}%
\pgfpathlineto{\pgfqpoint{1.486050in}{1.042638in}}%
\pgfpathlineto{\pgfqpoint{2.755563in}{3.027050in}}%
\pgfpathclose%
\pgfusepath{fill}%
\end{pgfscope}%
\begin{pgfscope}%
\pgfpathrectangle{\pgfqpoint{0.017500in}{0.017500in}}{\pgfqpoint{6.218282in}{3.580372in}} %
\pgfusepath{clip}%
\pgfsetbuttcap%
\pgfsetroundjoin%
\definecolor{currentfill}{rgb}{0.906154,0.842091,0.806151}%
\pgfsetfillcolor{currentfill}%
\pgfsetlinewidth{0.000000pt}%
\definecolor{currentstroke}{rgb}{0.000000,0.000000,0.000000}%
\pgfsetstrokecolor{currentstroke}%
\pgfsetdash{}{0pt}%
\pgfpathmoveto{\pgfqpoint{2.755563in}{3.027050in}}%
\pgfpathlineto{\pgfqpoint{1.486050in}{1.042638in}}%
\pgfpathlineto{\pgfqpoint{2.755563in}{3.027050in}}%
\pgfpathclose%
\pgfusepath{fill}%
\end{pgfscope}%
\begin{pgfscope}%
\pgfpathrectangle{\pgfqpoint{0.017500in}{0.017500in}}{\pgfqpoint{6.218282in}{3.580372in}} %
\pgfusepath{clip}%
\pgfsetbuttcap%
\pgfsetroundjoin%
\definecolor{currentfill}{rgb}{0.906154,0.842091,0.806151}%
\pgfsetfillcolor{currentfill}%
\pgfsetlinewidth{0.000000pt}%
\definecolor{currentstroke}{rgb}{0.000000,0.000000,0.000000}%
\pgfsetstrokecolor{currentstroke}%
\pgfsetdash{}{0pt}%
\pgfpathmoveto{\pgfqpoint{2.755563in}{3.027050in}}%
\pgfpathlineto{\pgfqpoint{1.486050in}{1.042638in}}%
\pgfpathlineto{\pgfqpoint{2.755563in}{3.027050in}}%
\pgfpathclose%
\pgfusepath{fill}%
\end{pgfscope}%
\begin{pgfscope}%
\pgfpathrectangle{\pgfqpoint{0.017500in}{0.017500in}}{\pgfqpoint{6.218282in}{3.580372in}} %
\pgfusepath{clip}%
\pgfsetbuttcap%
\pgfsetroundjoin%
\definecolor{currentfill}{rgb}{0.906154,0.842091,0.806151}%
\pgfsetfillcolor{currentfill}%
\pgfsetlinewidth{0.000000pt}%
\definecolor{currentstroke}{rgb}{0.000000,0.000000,0.000000}%
\pgfsetstrokecolor{currentstroke}%
\pgfsetdash{}{0pt}%
\pgfpathmoveto{\pgfqpoint{2.755563in}{3.027050in}}%
\pgfpathlineto{\pgfqpoint{1.486050in}{1.042638in}}%
\pgfpathlineto{\pgfqpoint{2.755563in}{3.027050in}}%
\pgfpathclose%
\pgfusepath{fill}%
\end{pgfscope}%
\begin{pgfscope}%
\pgfpathrectangle{\pgfqpoint{0.017500in}{0.017500in}}{\pgfqpoint{6.218282in}{3.580372in}} %
\pgfusepath{clip}%
\pgfsetbuttcap%
\pgfsetroundjoin%
\definecolor{currentfill}{rgb}{0.906154,0.842091,0.806151}%
\pgfsetfillcolor{currentfill}%
\pgfsetlinewidth{0.000000pt}%
\definecolor{currentstroke}{rgb}{0.000000,0.000000,0.000000}%
\pgfsetstrokecolor{currentstroke}%
\pgfsetdash{}{0pt}%
\pgfpathmoveto{\pgfqpoint{2.755563in}{3.027050in}}%
\pgfpathlineto{\pgfqpoint{1.486050in}{1.042638in}}%
\pgfpathlineto{\pgfqpoint{2.755563in}{3.027050in}}%
\pgfpathclose%
\pgfusepath{fill}%
\end{pgfscope}%
\begin{pgfscope}%
\pgfpathrectangle{\pgfqpoint{0.017500in}{0.017500in}}{\pgfqpoint{6.218282in}{3.580372in}} %
\pgfusepath{clip}%
\pgfsetbuttcap%
\pgfsetroundjoin%
\definecolor{currentfill}{rgb}{0.906154,0.842091,0.806151}%
\pgfsetfillcolor{currentfill}%
\pgfsetlinewidth{0.000000pt}%
\definecolor{currentstroke}{rgb}{0.000000,0.000000,0.000000}%
\pgfsetstrokecolor{currentstroke}%
\pgfsetdash{}{0pt}%
\pgfpathmoveto{\pgfqpoint{2.755563in}{3.027050in}}%
\pgfpathlineto{\pgfqpoint{1.486050in}{1.042638in}}%
\pgfpathlineto{\pgfqpoint{2.755563in}{3.027050in}}%
\pgfpathclose%
\pgfusepath{fill}%
\end{pgfscope}%
\begin{pgfscope}%
\pgfpathrectangle{\pgfqpoint{0.017500in}{0.017500in}}{\pgfqpoint{6.218282in}{3.580372in}} %
\pgfusepath{clip}%
\pgfsetbuttcap%
\pgfsetroundjoin%
\definecolor{currentfill}{rgb}{0.906154,0.842091,0.806151}%
\pgfsetfillcolor{currentfill}%
\pgfsetlinewidth{0.000000pt}%
\definecolor{currentstroke}{rgb}{0.000000,0.000000,0.000000}%
\pgfsetstrokecolor{currentstroke}%
\pgfsetdash{}{0pt}%
\pgfpathmoveto{\pgfqpoint{2.755563in}{3.027050in}}%
\pgfpathlineto{\pgfqpoint{1.486050in}{1.042638in}}%
\pgfpathlineto{\pgfqpoint{2.755563in}{3.027050in}}%
\pgfpathclose%
\pgfusepath{fill}%
\end{pgfscope}%
\begin{pgfscope}%
\pgfpathrectangle{\pgfqpoint{0.017500in}{0.017500in}}{\pgfqpoint{6.218282in}{3.580372in}} %
\pgfusepath{clip}%
\pgfsetbuttcap%
\pgfsetroundjoin%
\definecolor{currentfill}{rgb}{0.906154,0.842091,0.806151}%
\pgfsetfillcolor{currentfill}%
\pgfsetlinewidth{0.000000pt}%
\definecolor{currentstroke}{rgb}{0.000000,0.000000,0.000000}%
\pgfsetstrokecolor{currentstroke}%
\pgfsetdash{}{0pt}%
\pgfpathmoveto{\pgfqpoint{2.755563in}{3.027050in}}%
\pgfpathlineto{\pgfqpoint{1.486050in}{1.042638in}}%
\pgfpathlineto{\pgfqpoint{2.755563in}{3.027050in}}%
\pgfpathclose%
\pgfusepath{fill}%
\end{pgfscope}%
\begin{pgfscope}%
\pgfpathrectangle{\pgfqpoint{0.017500in}{0.017500in}}{\pgfqpoint{6.218282in}{3.580372in}} %
\pgfusepath{clip}%
\pgfsetbuttcap%
\pgfsetroundjoin%
\definecolor{currentfill}{rgb}{0.906154,0.842091,0.806151}%
\pgfsetfillcolor{currentfill}%
\pgfsetlinewidth{0.000000pt}%
\definecolor{currentstroke}{rgb}{0.000000,0.000000,0.000000}%
\pgfsetstrokecolor{currentstroke}%
\pgfsetdash{}{0pt}%
\pgfpathmoveto{\pgfqpoint{2.755563in}{3.027050in}}%
\pgfpathlineto{\pgfqpoint{1.486050in}{1.042638in}}%
\pgfpathlineto{\pgfqpoint{2.755563in}{3.027050in}}%
\pgfpathclose%
\pgfusepath{fill}%
\end{pgfscope}%
\begin{pgfscope}%
\pgfpathrectangle{\pgfqpoint{0.017500in}{0.017500in}}{\pgfqpoint{6.218282in}{3.580372in}} %
\pgfusepath{clip}%
\pgfsetbuttcap%
\pgfsetroundjoin%
\definecolor{currentfill}{rgb}{0.906154,0.842091,0.806151}%
\pgfsetfillcolor{currentfill}%
\pgfsetlinewidth{0.000000pt}%
\definecolor{currentstroke}{rgb}{0.000000,0.000000,0.000000}%
\pgfsetstrokecolor{currentstroke}%
\pgfsetdash{}{0pt}%
\pgfpathmoveto{\pgfqpoint{2.755563in}{3.027050in}}%
\pgfpathlineto{\pgfqpoint{1.486050in}{1.042638in}}%
\pgfpathlineto{\pgfqpoint{2.755563in}{3.027050in}}%
\pgfpathclose%
\pgfusepath{fill}%
\end{pgfscope}%
\begin{pgfscope}%
\pgfpathrectangle{\pgfqpoint{0.017500in}{0.017500in}}{\pgfqpoint{6.218282in}{3.580372in}} %
\pgfusepath{clip}%
\pgfsetbuttcap%
\pgfsetroundjoin%
\definecolor{currentfill}{rgb}{0.906154,0.842091,0.806151}%
\pgfsetfillcolor{currentfill}%
\pgfsetlinewidth{0.000000pt}%
\definecolor{currentstroke}{rgb}{0.000000,0.000000,0.000000}%
\pgfsetstrokecolor{currentstroke}%
\pgfsetdash{}{0pt}%
\pgfpathmoveto{\pgfqpoint{2.755563in}{3.027050in}}%
\pgfpathlineto{\pgfqpoint{1.486050in}{1.042638in}}%
\pgfpathlineto{\pgfqpoint{2.755563in}{3.027050in}}%
\pgfpathclose%
\pgfusepath{fill}%
\end{pgfscope}%
\begin{pgfscope}%
\pgfpathrectangle{\pgfqpoint{0.017500in}{0.017500in}}{\pgfqpoint{6.218282in}{3.580372in}} %
\pgfusepath{clip}%
\pgfsetbuttcap%
\pgfsetroundjoin%
\definecolor{currentfill}{rgb}{0.906154,0.842091,0.806151}%
\pgfsetfillcolor{currentfill}%
\pgfsetlinewidth{0.000000pt}%
\definecolor{currentstroke}{rgb}{0.000000,0.000000,0.000000}%
\pgfsetstrokecolor{currentstroke}%
\pgfsetdash{}{0pt}%
\pgfpathmoveto{\pgfqpoint{2.755563in}{3.027050in}}%
\pgfpathlineto{\pgfqpoint{1.486050in}{1.042638in}}%
\pgfpathlineto{\pgfqpoint{2.755563in}{3.027050in}}%
\pgfpathclose%
\pgfusepath{fill}%
\end{pgfscope}%
\begin{pgfscope}%
\pgfpathrectangle{\pgfqpoint{0.017500in}{0.017500in}}{\pgfqpoint{6.218282in}{3.580372in}} %
\pgfusepath{clip}%
\pgfsetbuttcap%
\pgfsetroundjoin%
\definecolor{currentfill}{rgb}{0.906154,0.842091,0.806151}%
\pgfsetfillcolor{currentfill}%
\pgfsetlinewidth{0.000000pt}%
\definecolor{currentstroke}{rgb}{0.000000,0.000000,0.000000}%
\pgfsetstrokecolor{currentstroke}%
\pgfsetdash{}{0pt}%
\pgfpathmoveto{\pgfqpoint{2.755563in}{3.027050in}}%
\pgfpathlineto{\pgfqpoint{1.486050in}{1.042638in}}%
\pgfpathlineto{\pgfqpoint{2.755563in}{3.027050in}}%
\pgfpathclose%
\pgfusepath{fill}%
\end{pgfscope}%
\begin{pgfscope}%
\pgfpathrectangle{\pgfqpoint{0.017500in}{0.017500in}}{\pgfqpoint{6.218282in}{3.580372in}} %
\pgfusepath{clip}%
\pgfsetbuttcap%
\pgfsetroundjoin%
\definecolor{currentfill}{rgb}{0.906154,0.842091,0.806151}%
\pgfsetfillcolor{currentfill}%
\pgfsetlinewidth{0.000000pt}%
\definecolor{currentstroke}{rgb}{0.000000,0.000000,0.000000}%
\pgfsetstrokecolor{currentstroke}%
\pgfsetdash{}{0pt}%
\pgfpathmoveto{\pgfqpoint{2.755563in}{3.027050in}}%
\pgfpathlineto{\pgfqpoint{1.486050in}{1.042638in}}%
\pgfpathlineto{\pgfqpoint{2.755563in}{3.027050in}}%
\pgfpathclose%
\pgfusepath{fill}%
\end{pgfscope}%
\begin{pgfscope}%
\pgfpathrectangle{\pgfqpoint{0.017500in}{0.017500in}}{\pgfqpoint{6.218282in}{3.580372in}} %
\pgfusepath{clip}%
\pgfsetbuttcap%
\pgfsetroundjoin%
\definecolor{currentfill}{rgb}{0.906154,0.842091,0.806151}%
\pgfsetfillcolor{currentfill}%
\pgfsetlinewidth{0.000000pt}%
\definecolor{currentstroke}{rgb}{0.000000,0.000000,0.000000}%
\pgfsetstrokecolor{currentstroke}%
\pgfsetdash{}{0pt}%
\pgfpathmoveto{\pgfqpoint{2.755563in}{3.027050in}}%
\pgfpathlineto{\pgfqpoint{1.486050in}{1.042638in}}%
\pgfpathlineto{\pgfqpoint{2.755563in}{3.027050in}}%
\pgfpathclose%
\pgfusepath{fill}%
\end{pgfscope}%
\begin{pgfscope}%
\pgfpathrectangle{\pgfqpoint{0.017500in}{0.017500in}}{\pgfqpoint{6.218282in}{3.580372in}} %
\pgfusepath{clip}%
\pgfsetbuttcap%
\pgfsetroundjoin%
\definecolor{currentfill}{rgb}{0.906154,0.842091,0.806151}%
\pgfsetfillcolor{currentfill}%
\pgfsetlinewidth{0.000000pt}%
\definecolor{currentstroke}{rgb}{0.000000,0.000000,0.000000}%
\pgfsetstrokecolor{currentstroke}%
\pgfsetdash{}{0pt}%
\pgfpathmoveto{\pgfqpoint{2.755563in}{3.027050in}}%
\pgfpathlineto{\pgfqpoint{1.486050in}{1.042638in}}%
\pgfpathlineto{\pgfqpoint{2.755563in}{3.027050in}}%
\pgfpathclose%
\pgfusepath{fill}%
\end{pgfscope}%
\begin{pgfscope}%
\pgfpathrectangle{\pgfqpoint{0.017500in}{0.017500in}}{\pgfqpoint{6.218282in}{3.580372in}} %
\pgfusepath{clip}%
\pgfsetbuttcap%
\pgfsetroundjoin%
\definecolor{currentfill}{rgb}{0.906154,0.842091,0.806151}%
\pgfsetfillcolor{currentfill}%
\pgfsetlinewidth{0.000000pt}%
\definecolor{currentstroke}{rgb}{0.000000,0.000000,0.000000}%
\pgfsetstrokecolor{currentstroke}%
\pgfsetdash{}{0pt}%
\pgfpathmoveto{\pgfqpoint{2.755563in}{3.027050in}}%
\pgfpathlineto{\pgfqpoint{1.486050in}{1.042638in}}%
\pgfpathlineto{\pgfqpoint{2.755563in}{3.027050in}}%
\pgfpathclose%
\pgfusepath{fill}%
\end{pgfscope}%
\begin{pgfscope}%
\pgfpathrectangle{\pgfqpoint{0.017500in}{0.017500in}}{\pgfqpoint{6.218282in}{3.580372in}} %
\pgfusepath{clip}%
\pgfsetbuttcap%
\pgfsetroundjoin%
\definecolor{currentfill}{rgb}{0.906154,0.842091,0.806151}%
\pgfsetfillcolor{currentfill}%
\pgfsetlinewidth{0.000000pt}%
\definecolor{currentstroke}{rgb}{0.000000,0.000000,0.000000}%
\pgfsetstrokecolor{currentstroke}%
\pgfsetdash{}{0pt}%
\pgfpathmoveto{\pgfqpoint{2.755563in}{3.027050in}}%
\pgfpathlineto{\pgfqpoint{1.486050in}{1.042638in}}%
\pgfpathlineto{\pgfqpoint{2.755563in}{3.027050in}}%
\pgfpathclose%
\pgfusepath{fill}%
\end{pgfscope}%
\begin{pgfscope}%
\pgfpathrectangle{\pgfqpoint{0.017500in}{0.017500in}}{\pgfqpoint{6.218282in}{3.580372in}} %
\pgfusepath{clip}%
\pgfsetbuttcap%
\pgfsetroundjoin%
\definecolor{currentfill}{rgb}{0.906154,0.842091,0.806151}%
\pgfsetfillcolor{currentfill}%
\pgfsetlinewidth{0.000000pt}%
\definecolor{currentstroke}{rgb}{0.000000,0.000000,0.000000}%
\pgfsetstrokecolor{currentstroke}%
\pgfsetdash{}{0pt}%
\pgfpathmoveto{\pgfqpoint{2.755563in}{3.027050in}}%
\pgfpathlineto{\pgfqpoint{1.486050in}{1.042638in}}%
\pgfpathlineto{\pgfqpoint{2.755563in}{3.027050in}}%
\pgfpathclose%
\pgfusepath{fill}%
\end{pgfscope}%
\begin{pgfscope}%
\pgfpathrectangle{\pgfqpoint{0.017500in}{0.017500in}}{\pgfqpoint{6.218282in}{3.580372in}} %
\pgfusepath{clip}%
\pgfsetbuttcap%
\pgfsetroundjoin%
\definecolor{currentfill}{rgb}{0.906154,0.842091,0.806151}%
\pgfsetfillcolor{currentfill}%
\pgfsetlinewidth{0.000000pt}%
\definecolor{currentstroke}{rgb}{0.000000,0.000000,0.000000}%
\pgfsetstrokecolor{currentstroke}%
\pgfsetdash{}{0pt}%
\pgfpathmoveto{\pgfqpoint{2.755563in}{3.027050in}}%
\pgfpathlineto{\pgfqpoint{1.486050in}{1.042638in}}%
\pgfpathlineto{\pgfqpoint{2.755563in}{3.027050in}}%
\pgfpathclose%
\pgfusepath{fill}%
\end{pgfscope}%
\begin{pgfscope}%
\pgfpathrectangle{\pgfqpoint{0.017500in}{0.017500in}}{\pgfqpoint{6.218282in}{3.580372in}} %
\pgfusepath{clip}%
\pgfsetbuttcap%
\pgfsetroundjoin%
\definecolor{currentfill}{rgb}{0.906154,0.842091,0.806151}%
\pgfsetfillcolor{currentfill}%
\pgfsetlinewidth{0.000000pt}%
\definecolor{currentstroke}{rgb}{0.000000,0.000000,0.000000}%
\pgfsetstrokecolor{currentstroke}%
\pgfsetdash{}{0pt}%
\pgfpathmoveto{\pgfqpoint{2.755563in}{3.027050in}}%
\pgfpathlineto{\pgfqpoint{1.486050in}{1.042638in}}%
\pgfpathlineto{\pgfqpoint{2.755563in}{3.027050in}}%
\pgfpathclose%
\pgfusepath{fill}%
\end{pgfscope}%
\begin{pgfscope}%
\pgfpathrectangle{\pgfqpoint{0.017500in}{0.017500in}}{\pgfqpoint{6.218282in}{3.580372in}} %
\pgfusepath{clip}%
\pgfsetbuttcap%
\pgfsetroundjoin%
\definecolor{currentfill}{rgb}{0.906154,0.842091,0.806151}%
\pgfsetfillcolor{currentfill}%
\pgfsetlinewidth{0.000000pt}%
\definecolor{currentstroke}{rgb}{0.000000,0.000000,0.000000}%
\pgfsetstrokecolor{currentstroke}%
\pgfsetdash{}{0pt}%
\pgfpathmoveto{\pgfqpoint{2.755563in}{3.027050in}}%
\pgfpathlineto{\pgfqpoint{1.486050in}{1.042638in}}%
\pgfpathlineto{\pgfqpoint{2.755563in}{3.027050in}}%
\pgfpathclose%
\pgfusepath{fill}%
\end{pgfscope}%
\begin{pgfscope}%
\pgfpathrectangle{\pgfqpoint{0.017500in}{0.017500in}}{\pgfqpoint{6.218282in}{3.580372in}} %
\pgfusepath{clip}%
\pgfsetbuttcap%
\pgfsetroundjoin%
\definecolor{currentfill}{rgb}{0.705673,0.015556,0.150233}%
\pgfsetfillcolor{currentfill}%
\pgfsetlinewidth{0.000000pt}%
\definecolor{currentstroke}{rgb}{0.000000,0.000000,0.000000}%
\pgfsetstrokecolor{currentstroke}%
\pgfsetdash{}{0pt}%
\pgfpathmoveto{\pgfqpoint{2.499217in}{2.903684in}}%
\pgfpathlineto{\pgfqpoint{2.755563in}{3.027050in}}%
\pgfpathlineto{\pgfqpoint{2.499217in}{2.903684in}}%
\pgfpathclose%
\pgfusepath{fill}%
\end{pgfscope}%
\begin{pgfscope}%
\pgfpathrectangle{\pgfqpoint{0.017500in}{0.017500in}}{\pgfqpoint{6.218282in}{3.580372in}} %
\pgfusepath{clip}%
\pgfsetbuttcap%
\pgfsetroundjoin%
\definecolor{currentfill}{rgb}{0.705673,0.015556,0.150233}%
\pgfsetfillcolor{currentfill}%
\pgfsetlinewidth{0.000000pt}%
\definecolor{currentstroke}{rgb}{0.000000,0.000000,0.000000}%
\pgfsetstrokecolor{currentstroke}%
\pgfsetdash{}{0pt}%
\pgfpathmoveto{\pgfqpoint{2.499217in}{2.903684in}}%
\pgfpathlineto{\pgfqpoint{2.755563in}{3.027050in}}%
\pgfpathlineto{\pgfqpoint{2.499217in}{2.903684in}}%
\pgfpathclose%
\pgfusepath{fill}%
\end{pgfscope}%
\begin{pgfscope}%
\pgfpathrectangle{\pgfqpoint{0.017500in}{0.017500in}}{\pgfqpoint{6.218282in}{3.580372in}} %
\pgfusepath{clip}%
\pgfsetbuttcap%
\pgfsetroundjoin%
\definecolor{currentfill}{rgb}{0.705673,0.015556,0.150233}%
\pgfsetfillcolor{currentfill}%
\pgfsetlinewidth{0.000000pt}%
\definecolor{currentstroke}{rgb}{0.000000,0.000000,0.000000}%
\pgfsetstrokecolor{currentstroke}%
\pgfsetdash{}{0pt}%
\pgfpathmoveto{\pgfqpoint{2.499217in}{2.903684in}}%
\pgfpathlineto{\pgfqpoint{2.755563in}{3.027050in}}%
\pgfpathlineto{\pgfqpoint{2.499217in}{2.903684in}}%
\pgfpathclose%
\pgfusepath{fill}%
\end{pgfscope}%
\begin{pgfscope}%
\pgfpathrectangle{\pgfqpoint{0.017500in}{0.017500in}}{\pgfqpoint{6.218282in}{3.580372in}} %
\pgfusepath{clip}%
\pgfsetbuttcap%
\pgfsetroundjoin%
\definecolor{currentfill}{rgb}{0.705673,0.015556,0.150233}%
\pgfsetfillcolor{currentfill}%
\pgfsetlinewidth{0.000000pt}%
\definecolor{currentstroke}{rgb}{0.000000,0.000000,0.000000}%
\pgfsetstrokecolor{currentstroke}%
\pgfsetdash{}{0pt}%
\pgfpathmoveto{\pgfqpoint{2.499217in}{2.903684in}}%
\pgfpathlineto{\pgfqpoint{2.755563in}{3.027050in}}%
\pgfpathlineto{\pgfqpoint{2.499217in}{2.903684in}}%
\pgfpathclose%
\pgfusepath{fill}%
\end{pgfscope}%
\begin{pgfscope}%
\pgfpathrectangle{\pgfqpoint{0.017500in}{0.017500in}}{\pgfqpoint{6.218282in}{3.580372in}} %
\pgfusepath{clip}%
\pgfsetbuttcap%
\pgfsetroundjoin%
\definecolor{currentfill}{rgb}{0.705673,0.015556,0.150233}%
\pgfsetfillcolor{currentfill}%
\pgfsetlinewidth{0.000000pt}%
\definecolor{currentstroke}{rgb}{0.000000,0.000000,0.000000}%
\pgfsetstrokecolor{currentstroke}%
\pgfsetdash{}{0pt}%
\pgfpathmoveto{\pgfqpoint{2.499217in}{2.903684in}}%
\pgfpathlineto{\pgfqpoint{2.755563in}{3.027050in}}%
\pgfpathlineto{\pgfqpoint{2.499217in}{2.903684in}}%
\pgfpathclose%
\pgfusepath{fill}%
\end{pgfscope}%
\begin{pgfscope}%
\pgfpathrectangle{\pgfqpoint{0.017500in}{0.017500in}}{\pgfqpoint{6.218282in}{3.580372in}} %
\pgfusepath{clip}%
\pgfsetbuttcap%
\pgfsetroundjoin%
\definecolor{currentfill}{rgb}{0.705673,0.015556,0.150233}%
\pgfsetfillcolor{currentfill}%
\pgfsetlinewidth{0.000000pt}%
\definecolor{currentstroke}{rgb}{0.000000,0.000000,0.000000}%
\pgfsetstrokecolor{currentstroke}%
\pgfsetdash{}{0pt}%
\pgfpathmoveto{\pgfqpoint{2.499217in}{2.903684in}}%
\pgfpathlineto{\pgfqpoint{2.755563in}{3.027050in}}%
\pgfpathlineto{\pgfqpoint{2.499217in}{2.903684in}}%
\pgfpathclose%
\pgfusepath{fill}%
\end{pgfscope}%
\begin{pgfscope}%
\pgfpathrectangle{\pgfqpoint{0.017500in}{0.017500in}}{\pgfqpoint{6.218282in}{3.580372in}} %
\pgfusepath{clip}%
\pgfsetbuttcap%
\pgfsetroundjoin%
\definecolor{currentfill}{rgb}{0.705673,0.015556,0.150233}%
\pgfsetfillcolor{currentfill}%
\pgfsetlinewidth{0.000000pt}%
\definecolor{currentstroke}{rgb}{0.000000,0.000000,0.000000}%
\pgfsetstrokecolor{currentstroke}%
\pgfsetdash{}{0pt}%
\pgfpathmoveto{\pgfqpoint{2.499217in}{2.903684in}}%
\pgfpathlineto{\pgfqpoint{2.755563in}{3.027050in}}%
\pgfpathlineto{\pgfqpoint{2.499217in}{2.903684in}}%
\pgfpathclose%
\pgfusepath{fill}%
\end{pgfscope}%
\begin{pgfscope}%
\pgfpathrectangle{\pgfqpoint{0.017500in}{0.017500in}}{\pgfqpoint{6.218282in}{3.580372in}} %
\pgfusepath{clip}%
\pgfsetbuttcap%
\pgfsetroundjoin%
\definecolor{currentfill}{rgb}{0.705673,0.015556,0.150233}%
\pgfsetfillcolor{currentfill}%
\pgfsetlinewidth{0.000000pt}%
\definecolor{currentstroke}{rgb}{0.000000,0.000000,0.000000}%
\pgfsetstrokecolor{currentstroke}%
\pgfsetdash{}{0pt}%
\pgfpathmoveto{\pgfqpoint{2.499217in}{2.903684in}}%
\pgfpathlineto{\pgfqpoint{2.755563in}{3.027050in}}%
\pgfpathlineto{\pgfqpoint{2.499217in}{2.903684in}}%
\pgfpathclose%
\pgfusepath{fill}%
\end{pgfscope}%
\begin{pgfscope}%
\pgfpathrectangle{\pgfqpoint{0.017500in}{0.017500in}}{\pgfqpoint{6.218282in}{3.580372in}} %
\pgfusepath{clip}%
\pgfsetbuttcap%
\pgfsetroundjoin%
\definecolor{currentfill}{rgb}{0.705673,0.015556,0.150233}%
\pgfsetfillcolor{currentfill}%
\pgfsetlinewidth{0.000000pt}%
\definecolor{currentstroke}{rgb}{0.000000,0.000000,0.000000}%
\pgfsetstrokecolor{currentstroke}%
\pgfsetdash{}{0pt}%
\pgfpathmoveto{\pgfqpoint{2.499217in}{2.903684in}}%
\pgfpathlineto{\pgfqpoint{2.755563in}{3.027050in}}%
\pgfpathlineto{\pgfqpoint{2.499217in}{2.903684in}}%
\pgfpathclose%
\pgfusepath{fill}%
\end{pgfscope}%
\begin{pgfscope}%
\pgfpathrectangle{\pgfqpoint{0.017500in}{0.017500in}}{\pgfqpoint{6.218282in}{3.580372in}} %
\pgfusepath{clip}%
\pgfsetbuttcap%
\pgfsetroundjoin%
\definecolor{currentfill}{rgb}{0.705673,0.015556,0.150233}%
\pgfsetfillcolor{currentfill}%
\pgfsetlinewidth{0.000000pt}%
\definecolor{currentstroke}{rgb}{0.000000,0.000000,0.000000}%
\pgfsetstrokecolor{currentstroke}%
\pgfsetdash{}{0pt}%
\pgfpathmoveto{\pgfqpoint{2.499217in}{2.903684in}}%
\pgfpathlineto{\pgfqpoint{2.755563in}{3.027050in}}%
\pgfpathlineto{\pgfqpoint{2.499217in}{2.903684in}}%
\pgfpathclose%
\pgfusepath{fill}%
\end{pgfscope}%
\begin{pgfscope}%
\pgfpathrectangle{\pgfqpoint{0.017500in}{0.017500in}}{\pgfqpoint{6.218282in}{3.580372in}} %
\pgfusepath{clip}%
\pgfsetbuttcap%
\pgfsetroundjoin%
\definecolor{currentfill}{rgb}{0.705673,0.015556,0.150233}%
\pgfsetfillcolor{currentfill}%
\pgfsetlinewidth{0.000000pt}%
\definecolor{currentstroke}{rgb}{0.000000,0.000000,0.000000}%
\pgfsetstrokecolor{currentstroke}%
\pgfsetdash{}{0pt}%
\pgfpathmoveto{\pgfqpoint{2.499217in}{2.903684in}}%
\pgfpathlineto{\pgfqpoint{2.755563in}{3.027050in}}%
\pgfpathlineto{\pgfqpoint{2.499217in}{2.903684in}}%
\pgfpathclose%
\pgfusepath{fill}%
\end{pgfscope}%
\begin{pgfscope}%
\pgfpathrectangle{\pgfqpoint{0.017500in}{0.017500in}}{\pgfqpoint{6.218282in}{3.580372in}} %
\pgfusepath{clip}%
\pgfsetbuttcap%
\pgfsetroundjoin%
\definecolor{currentfill}{rgb}{0.705673,0.015556,0.150233}%
\pgfsetfillcolor{currentfill}%
\pgfsetlinewidth{0.000000pt}%
\definecolor{currentstroke}{rgb}{0.000000,0.000000,0.000000}%
\pgfsetstrokecolor{currentstroke}%
\pgfsetdash{}{0pt}%
\pgfpathmoveto{\pgfqpoint{2.499217in}{2.903684in}}%
\pgfpathlineto{\pgfqpoint{2.755563in}{3.027050in}}%
\pgfpathlineto{\pgfqpoint{2.499217in}{2.903684in}}%
\pgfpathclose%
\pgfusepath{fill}%
\end{pgfscope}%
\begin{pgfscope}%
\pgfpathrectangle{\pgfqpoint{0.017500in}{0.017500in}}{\pgfqpoint{6.218282in}{3.580372in}} %
\pgfusepath{clip}%
\pgfsetbuttcap%
\pgfsetroundjoin%
\definecolor{currentfill}{rgb}{0.705673,0.015556,0.150233}%
\pgfsetfillcolor{currentfill}%
\pgfsetlinewidth{0.000000pt}%
\definecolor{currentstroke}{rgb}{0.000000,0.000000,0.000000}%
\pgfsetstrokecolor{currentstroke}%
\pgfsetdash{}{0pt}%
\pgfpathmoveto{\pgfqpoint{2.499217in}{2.903684in}}%
\pgfpathlineto{\pgfqpoint{2.755563in}{3.027050in}}%
\pgfpathlineto{\pgfqpoint{2.499217in}{2.903684in}}%
\pgfpathclose%
\pgfusepath{fill}%
\end{pgfscope}%
\begin{pgfscope}%
\pgfpathrectangle{\pgfqpoint{0.017500in}{0.017500in}}{\pgfqpoint{6.218282in}{3.580372in}} %
\pgfusepath{clip}%
\pgfsetbuttcap%
\pgfsetroundjoin%
\definecolor{currentfill}{rgb}{0.705673,0.015556,0.150233}%
\pgfsetfillcolor{currentfill}%
\pgfsetlinewidth{0.000000pt}%
\definecolor{currentstroke}{rgb}{0.000000,0.000000,0.000000}%
\pgfsetstrokecolor{currentstroke}%
\pgfsetdash{}{0pt}%
\pgfpathmoveto{\pgfqpoint{2.499217in}{2.903684in}}%
\pgfpathlineto{\pgfqpoint{2.755563in}{3.027050in}}%
\pgfpathlineto{\pgfqpoint{2.499217in}{2.903684in}}%
\pgfpathclose%
\pgfusepath{fill}%
\end{pgfscope}%
\begin{pgfscope}%
\pgfpathrectangle{\pgfqpoint{0.017500in}{0.017500in}}{\pgfqpoint{6.218282in}{3.580372in}} %
\pgfusepath{clip}%
\pgfsetbuttcap%
\pgfsetroundjoin%
\definecolor{currentfill}{rgb}{0.705673,0.015556,0.150233}%
\pgfsetfillcolor{currentfill}%
\pgfsetlinewidth{0.000000pt}%
\definecolor{currentstroke}{rgb}{0.000000,0.000000,0.000000}%
\pgfsetstrokecolor{currentstroke}%
\pgfsetdash{}{0pt}%
\pgfpathmoveto{\pgfqpoint{2.499217in}{2.903684in}}%
\pgfpathlineto{\pgfqpoint{2.755563in}{3.027050in}}%
\pgfpathlineto{\pgfqpoint{2.499217in}{2.903684in}}%
\pgfpathclose%
\pgfusepath{fill}%
\end{pgfscope}%
\begin{pgfscope}%
\pgfpathrectangle{\pgfqpoint{0.017500in}{0.017500in}}{\pgfqpoint{6.218282in}{3.580372in}} %
\pgfusepath{clip}%
\pgfsetbuttcap%
\pgfsetroundjoin%
\definecolor{currentfill}{rgb}{0.705673,0.015556,0.150233}%
\pgfsetfillcolor{currentfill}%
\pgfsetlinewidth{0.000000pt}%
\definecolor{currentstroke}{rgb}{0.000000,0.000000,0.000000}%
\pgfsetstrokecolor{currentstroke}%
\pgfsetdash{}{0pt}%
\pgfpathmoveto{\pgfqpoint{2.499217in}{2.903684in}}%
\pgfpathlineto{\pgfqpoint{2.755563in}{3.027050in}}%
\pgfpathlineto{\pgfqpoint{2.499217in}{2.903684in}}%
\pgfpathclose%
\pgfusepath{fill}%
\end{pgfscope}%
\begin{pgfscope}%
\pgfpathrectangle{\pgfqpoint{0.017500in}{0.017500in}}{\pgfqpoint{6.218282in}{3.580372in}} %
\pgfusepath{clip}%
\pgfsetbuttcap%
\pgfsetroundjoin%
\definecolor{currentfill}{rgb}{0.705673,0.015556,0.150233}%
\pgfsetfillcolor{currentfill}%
\pgfsetlinewidth{0.000000pt}%
\definecolor{currentstroke}{rgb}{0.000000,0.000000,0.000000}%
\pgfsetstrokecolor{currentstroke}%
\pgfsetdash{}{0pt}%
\pgfpathmoveto{\pgfqpoint{2.499217in}{2.903684in}}%
\pgfpathlineto{\pgfqpoint{2.755563in}{3.027050in}}%
\pgfpathlineto{\pgfqpoint{2.499217in}{2.903684in}}%
\pgfpathclose%
\pgfusepath{fill}%
\end{pgfscope}%
\begin{pgfscope}%
\pgfpathrectangle{\pgfqpoint{0.017500in}{0.017500in}}{\pgfqpoint{6.218282in}{3.580372in}} %
\pgfusepath{clip}%
\pgfsetbuttcap%
\pgfsetroundjoin%
\definecolor{currentfill}{rgb}{0.705673,0.015556,0.150233}%
\pgfsetfillcolor{currentfill}%
\pgfsetlinewidth{0.000000pt}%
\definecolor{currentstroke}{rgb}{0.000000,0.000000,0.000000}%
\pgfsetstrokecolor{currentstroke}%
\pgfsetdash{}{0pt}%
\pgfpathmoveto{\pgfqpoint{2.499217in}{2.903684in}}%
\pgfpathlineto{\pgfqpoint{2.755563in}{3.027050in}}%
\pgfpathlineto{\pgfqpoint{2.499217in}{2.903684in}}%
\pgfpathclose%
\pgfusepath{fill}%
\end{pgfscope}%
\begin{pgfscope}%
\pgfpathrectangle{\pgfqpoint{0.017500in}{0.017500in}}{\pgfqpoint{6.218282in}{3.580372in}} %
\pgfusepath{clip}%
\pgfsetbuttcap%
\pgfsetroundjoin%
\definecolor{currentfill}{rgb}{0.705673,0.015556,0.150233}%
\pgfsetfillcolor{currentfill}%
\pgfsetlinewidth{0.000000pt}%
\definecolor{currentstroke}{rgb}{0.000000,0.000000,0.000000}%
\pgfsetstrokecolor{currentstroke}%
\pgfsetdash{}{0pt}%
\pgfpathmoveto{\pgfqpoint{2.499217in}{2.903684in}}%
\pgfpathlineto{\pgfqpoint{2.755563in}{3.027050in}}%
\pgfpathlineto{\pgfqpoint{2.499217in}{2.903684in}}%
\pgfpathclose%
\pgfusepath{fill}%
\end{pgfscope}%
\begin{pgfscope}%
\pgfpathrectangle{\pgfqpoint{0.017500in}{0.017500in}}{\pgfqpoint{6.218282in}{3.580372in}} %
\pgfusepath{clip}%
\pgfsetbuttcap%
\pgfsetroundjoin%
\definecolor{currentfill}{rgb}{0.705673,0.015556,0.150233}%
\pgfsetfillcolor{currentfill}%
\pgfsetlinewidth{0.000000pt}%
\definecolor{currentstroke}{rgb}{0.000000,0.000000,0.000000}%
\pgfsetstrokecolor{currentstroke}%
\pgfsetdash{}{0pt}%
\pgfpathmoveto{\pgfqpoint{2.499217in}{2.903684in}}%
\pgfpathlineto{\pgfqpoint{2.755563in}{3.027050in}}%
\pgfpathlineto{\pgfqpoint{2.499217in}{2.903684in}}%
\pgfpathclose%
\pgfusepath{fill}%
\end{pgfscope}%
\begin{pgfscope}%
\pgfpathrectangle{\pgfqpoint{0.017500in}{0.017500in}}{\pgfqpoint{6.218282in}{3.580372in}} %
\pgfusepath{clip}%
\pgfsetbuttcap%
\pgfsetroundjoin%
\definecolor{currentfill}{rgb}{0.705673,0.015556,0.150233}%
\pgfsetfillcolor{currentfill}%
\pgfsetlinewidth{0.000000pt}%
\definecolor{currentstroke}{rgb}{0.000000,0.000000,0.000000}%
\pgfsetstrokecolor{currentstroke}%
\pgfsetdash{}{0pt}%
\pgfpathmoveto{\pgfqpoint{2.499217in}{2.903684in}}%
\pgfpathlineto{\pgfqpoint{2.755563in}{3.027050in}}%
\pgfpathlineto{\pgfqpoint{2.499217in}{2.903684in}}%
\pgfpathclose%
\pgfusepath{fill}%
\end{pgfscope}%
\begin{pgfscope}%
\pgfpathrectangle{\pgfqpoint{0.017500in}{0.017500in}}{\pgfqpoint{6.218282in}{3.580372in}} %
\pgfusepath{clip}%
\pgfsetbuttcap%
\pgfsetroundjoin%
\definecolor{currentfill}{rgb}{0.705673,0.015556,0.150233}%
\pgfsetfillcolor{currentfill}%
\pgfsetlinewidth{0.000000pt}%
\definecolor{currentstroke}{rgb}{0.000000,0.000000,0.000000}%
\pgfsetstrokecolor{currentstroke}%
\pgfsetdash{}{0pt}%
\pgfpathmoveto{\pgfqpoint{2.499217in}{2.903684in}}%
\pgfpathlineto{\pgfqpoint{2.755563in}{3.027050in}}%
\pgfpathlineto{\pgfqpoint{2.499217in}{2.903684in}}%
\pgfpathclose%
\pgfusepath{fill}%
\end{pgfscope}%
\begin{pgfscope}%
\pgfpathrectangle{\pgfqpoint{0.017500in}{0.017500in}}{\pgfqpoint{6.218282in}{3.580372in}} %
\pgfusepath{clip}%
\pgfsetbuttcap%
\pgfsetroundjoin%
\definecolor{currentfill}{rgb}{0.705673,0.015556,0.150233}%
\pgfsetfillcolor{currentfill}%
\pgfsetlinewidth{0.000000pt}%
\definecolor{currentstroke}{rgb}{0.000000,0.000000,0.000000}%
\pgfsetstrokecolor{currentstroke}%
\pgfsetdash{}{0pt}%
\pgfpathmoveto{\pgfqpoint{2.499217in}{2.903684in}}%
\pgfpathlineto{\pgfqpoint{2.755563in}{3.027050in}}%
\pgfpathlineto{\pgfqpoint{2.499217in}{2.903684in}}%
\pgfpathclose%
\pgfusepath{fill}%
\end{pgfscope}%
\begin{pgfscope}%
\pgfpathrectangle{\pgfqpoint{0.017500in}{0.017500in}}{\pgfqpoint{6.218282in}{3.580372in}} %
\pgfusepath{clip}%
\pgfsetbuttcap%
\pgfsetroundjoin%
\definecolor{currentfill}{rgb}{0.705673,0.015556,0.150233}%
\pgfsetfillcolor{currentfill}%
\pgfsetlinewidth{0.000000pt}%
\definecolor{currentstroke}{rgb}{0.000000,0.000000,0.000000}%
\pgfsetstrokecolor{currentstroke}%
\pgfsetdash{}{0pt}%
\pgfpathmoveto{\pgfqpoint{2.499217in}{2.903684in}}%
\pgfpathlineto{\pgfqpoint{2.755563in}{3.027050in}}%
\pgfpathlineto{\pgfqpoint{2.499217in}{2.903684in}}%
\pgfpathclose%
\pgfusepath{fill}%
\end{pgfscope}%
\begin{pgfscope}%
\pgfpathrectangle{\pgfqpoint{0.017500in}{0.017500in}}{\pgfqpoint{6.218282in}{3.580372in}} %
\pgfusepath{clip}%
\pgfsetbuttcap%
\pgfsetroundjoin%
\definecolor{currentfill}{rgb}{0.705673,0.015556,0.150233}%
\pgfsetfillcolor{currentfill}%
\pgfsetlinewidth{0.000000pt}%
\definecolor{currentstroke}{rgb}{0.000000,0.000000,0.000000}%
\pgfsetstrokecolor{currentstroke}%
\pgfsetdash{}{0pt}%
\pgfpathmoveto{\pgfqpoint{2.499217in}{2.903684in}}%
\pgfpathlineto{\pgfqpoint{2.755563in}{3.027050in}}%
\pgfpathlineto{\pgfqpoint{2.499217in}{2.903684in}}%
\pgfpathclose%
\pgfusepath{fill}%
\end{pgfscope}%
\begin{pgfscope}%
\pgfpathrectangle{\pgfqpoint{0.017500in}{0.017500in}}{\pgfqpoint{6.218282in}{3.580372in}} %
\pgfusepath{clip}%
\pgfsetbuttcap%
\pgfsetroundjoin%
\definecolor{currentfill}{rgb}{0.705673,0.015556,0.150233}%
\pgfsetfillcolor{currentfill}%
\pgfsetlinewidth{0.000000pt}%
\definecolor{currentstroke}{rgb}{0.000000,0.000000,0.000000}%
\pgfsetstrokecolor{currentstroke}%
\pgfsetdash{}{0pt}%
\pgfpathmoveto{\pgfqpoint{2.499217in}{2.903684in}}%
\pgfpathlineto{\pgfqpoint{2.755563in}{3.027050in}}%
\pgfpathlineto{\pgfqpoint{2.499217in}{2.903684in}}%
\pgfpathclose%
\pgfusepath{fill}%
\end{pgfscope}%
\begin{pgfscope}%
\pgfpathrectangle{\pgfqpoint{0.017500in}{0.017500in}}{\pgfqpoint{6.218282in}{3.580372in}} %
\pgfusepath{clip}%
\pgfsetbuttcap%
\pgfsetroundjoin%
\definecolor{currentfill}{rgb}{0.705673,0.015556,0.150233}%
\pgfsetfillcolor{currentfill}%
\pgfsetlinewidth{0.000000pt}%
\definecolor{currentstroke}{rgb}{0.000000,0.000000,0.000000}%
\pgfsetstrokecolor{currentstroke}%
\pgfsetdash{}{0pt}%
\pgfpathmoveto{\pgfqpoint{2.499217in}{2.903684in}}%
\pgfpathlineto{\pgfqpoint{2.755563in}{3.027050in}}%
\pgfpathlineto{\pgfqpoint{2.499217in}{2.903684in}}%
\pgfpathclose%
\pgfusepath{fill}%
\end{pgfscope}%
\begin{pgfscope}%
\pgfpathrectangle{\pgfqpoint{0.017500in}{0.017500in}}{\pgfqpoint{6.218282in}{3.580372in}} %
\pgfusepath{clip}%
\pgfsetbuttcap%
\pgfsetroundjoin%
\definecolor{currentfill}{rgb}{0.705673,0.015556,0.150233}%
\pgfsetfillcolor{currentfill}%
\pgfsetlinewidth{0.000000pt}%
\definecolor{currentstroke}{rgb}{0.000000,0.000000,0.000000}%
\pgfsetstrokecolor{currentstroke}%
\pgfsetdash{}{0pt}%
\pgfpathmoveto{\pgfqpoint{2.499217in}{2.903684in}}%
\pgfpathlineto{\pgfqpoint{2.755563in}{3.027050in}}%
\pgfpathlineto{\pgfqpoint{2.499217in}{2.903684in}}%
\pgfpathclose%
\pgfusepath{fill}%
\end{pgfscope}%
\begin{pgfscope}%
\pgfpathrectangle{\pgfqpoint{0.017500in}{0.017500in}}{\pgfqpoint{6.218282in}{3.580372in}} %
\pgfusepath{clip}%
\pgfsetbuttcap%
\pgfsetroundjoin%
\definecolor{currentfill}{rgb}{0.705673,0.015556,0.150233}%
\pgfsetfillcolor{currentfill}%
\pgfsetlinewidth{0.000000pt}%
\definecolor{currentstroke}{rgb}{0.000000,0.000000,0.000000}%
\pgfsetstrokecolor{currentstroke}%
\pgfsetdash{}{0pt}%
\pgfpathmoveto{\pgfqpoint{2.499217in}{2.903684in}}%
\pgfpathlineto{\pgfqpoint{2.755563in}{3.027050in}}%
\pgfpathlineto{\pgfqpoint{2.499217in}{2.903684in}}%
\pgfpathclose%
\pgfusepath{fill}%
\end{pgfscope}%
\begin{pgfscope}%
\pgfpathrectangle{\pgfqpoint{0.017500in}{0.017500in}}{\pgfqpoint{6.218282in}{3.580372in}} %
\pgfusepath{clip}%
\pgfsetbuttcap%
\pgfsetroundjoin%
\definecolor{currentfill}{rgb}{0.705673,0.015556,0.150233}%
\pgfsetfillcolor{currentfill}%
\pgfsetlinewidth{0.000000pt}%
\definecolor{currentstroke}{rgb}{0.000000,0.000000,0.000000}%
\pgfsetstrokecolor{currentstroke}%
\pgfsetdash{}{0pt}%
\pgfpathmoveto{\pgfqpoint{2.499217in}{2.903684in}}%
\pgfpathlineto{\pgfqpoint{2.755563in}{3.027050in}}%
\pgfpathlineto{\pgfqpoint{2.499217in}{2.903684in}}%
\pgfpathclose%
\pgfusepath{fill}%
\end{pgfscope}%
\begin{pgfscope}%
\pgfpathrectangle{\pgfqpoint{0.017500in}{0.017500in}}{\pgfqpoint{6.218282in}{3.580372in}} %
\pgfusepath{clip}%
\pgfsetbuttcap%
\pgfsetroundjoin%
\definecolor{currentfill}{rgb}{0.809329,0.852974,0.922323}%
\pgfsetfillcolor{currentfill}%
\pgfsetlinewidth{0.000000pt}%
\definecolor{currentstroke}{rgb}{0.000000,0.000000,0.000000}%
\pgfsetstrokecolor{currentstroke}%
\pgfsetdash{}{0pt}%
\pgfpathmoveto{\pgfqpoint{2.755563in}{3.027050in}}%
\pgfpathlineto{\pgfqpoint{3.186131in}{2.954270in}}%
\pgfpathlineto{\pgfqpoint{1.914830in}{0.967594in}}%
\pgfpathlineto{\pgfqpoint{1.486050in}{1.042638in}}%
\pgfpathlineto{\pgfqpoint{2.755563in}{3.027050in}}%
\pgfpathclose%
\pgfusepath{fill}%
\end{pgfscope}%
\begin{pgfscope}%
\pgfpathrectangle{\pgfqpoint{0.017500in}{0.017500in}}{\pgfqpoint{6.218282in}{3.580372in}} %
\pgfusepath{clip}%
\pgfsetbuttcap%
\pgfsetroundjoin%
\definecolor{currentfill}{rgb}{0.809329,0.852974,0.922323}%
\pgfsetfillcolor{currentfill}%
\pgfsetlinewidth{0.000000pt}%
\definecolor{currentstroke}{rgb}{0.000000,0.000000,0.000000}%
\pgfsetstrokecolor{currentstroke}%
\pgfsetdash{}{0pt}%
\pgfpathmoveto{\pgfqpoint{2.755563in}{3.027050in}}%
\pgfpathlineto{\pgfqpoint{3.186131in}{2.954270in}}%
\pgfpathlineto{\pgfqpoint{1.914830in}{0.967594in}}%
\pgfpathlineto{\pgfqpoint{1.486050in}{1.042638in}}%
\pgfpathlineto{\pgfqpoint{2.755563in}{3.027050in}}%
\pgfpathclose%
\pgfusepath{fill}%
\end{pgfscope}%
\begin{pgfscope}%
\pgfpathrectangle{\pgfqpoint{0.017500in}{0.017500in}}{\pgfqpoint{6.218282in}{3.580372in}} %
\pgfusepath{clip}%
\pgfsetbuttcap%
\pgfsetroundjoin%
\definecolor{currentfill}{rgb}{0.809329,0.852974,0.922323}%
\pgfsetfillcolor{currentfill}%
\pgfsetlinewidth{0.000000pt}%
\definecolor{currentstroke}{rgb}{0.000000,0.000000,0.000000}%
\pgfsetstrokecolor{currentstroke}%
\pgfsetdash{}{0pt}%
\pgfpathmoveto{\pgfqpoint{2.755563in}{3.027050in}}%
\pgfpathlineto{\pgfqpoint{3.186131in}{2.954270in}}%
\pgfpathlineto{\pgfqpoint{1.914830in}{0.967594in}}%
\pgfpathlineto{\pgfqpoint{1.486050in}{1.042638in}}%
\pgfpathlineto{\pgfqpoint{2.755563in}{3.027050in}}%
\pgfpathclose%
\pgfusepath{fill}%
\end{pgfscope}%
\begin{pgfscope}%
\pgfpathrectangle{\pgfqpoint{0.017500in}{0.017500in}}{\pgfqpoint{6.218282in}{3.580372in}} %
\pgfusepath{clip}%
\pgfsetbuttcap%
\pgfsetroundjoin%
\definecolor{currentfill}{rgb}{0.809329,0.852974,0.922323}%
\pgfsetfillcolor{currentfill}%
\pgfsetlinewidth{0.000000pt}%
\definecolor{currentstroke}{rgb}{0.000000,0.000000,0.000000}%
\pgfsetstrokecolor{currentstroke}%
\pgfsetdash{}{0pt}%
\pgfpathmoveto{\pgfqpoint{2.755563in}{3.027050in}}%
\pgfpathlineto{\pgfqpoint{3.186131in}{2.954270in}}%
\pgfpathlineto{\pgfqpoint{1.914830in}{0.967594in}}%
\pgfpathlineto{\pgfqpoint{1.486050in}{1.042638in}}%
\pgfpathlineto{\pgfqpoint{2.755563in}{3.027050in}}%
\pgfpathclose%
\pgfusepath{fill}%
\end{pgfscope}%
\begin{pgfscope}%
\pgfpathrectangle{\pgfqpoint{0.017500in}{0.017500in}}{\pgfqpoint{6.218282in}{3.580372in}} %
\pgfusepath{clip}%
\pgfsetbuttcap%
\pgfsetroundjoin%
\definecolor{currentfill}{rgb}{0.809329,0.852974,0.922323}%
\pgfsetfillcolor{currentfill}%
\pgfsetlinewidth{0.000000pt}%
\definecolor{currentstroke}{rgb}{0.000000,0.000000,0.000000}%
\pgfsetstrokecolor{currentstroke}%
\pgfsetdash{}{0pt}%
\pgfpathmoveto{\pgfqpoint{2.755563in}{3.027050in}}%
\pgfpathlineto{\pgfqpoint{3.186131in}{2.954270in}}%
\pgfpathlineto{\pgfqpoint{1.914830in}{0.967594in}}%
\pgfpathlineto{\pgfqpoint{1.486050in}{1.042638in}}%
\pgfpathlineto{\pgfqpoint{2.755563in}{3.027050in}}%
\pgfpathclose%
\pgfusepath{fill}%
\end{pgfscope}%
\begin{pgfscope}%
\pgfpathrectangle{\pgfqpoint{0.017500in}{0.017500in}}{\pgfqpoint{6.218282in}{3.580372in}} %
\pgfusepath{clip}%
\pgfsetbuttcap%
\pgfsetroundjoin%
\definecolor{currentfill}{rgb}{0.711554,0.033337,0.154485}%
\pgfsetfillcolor{currentfill}%
\pgfsetlinewidth{0.000000pt}%
\definecolor{currentstroke}{rgb}{0.000000,0.000000,0.000000}%
\pgfsetstrokecolor{currentstroke}%
\pgfsetdash{}{0pt}%
\pgfpathmoveto{\pgfqpoint{2.499217in}{2.903684in}}%
\pgfpathlineto{\pgfqpoint{2.932968in}{2.820668in}}%
\pgfpathlineto{\pgfqpoint{3.186131in}{2.954270in}}%
\pgfpathlineto{\pgfqpoint{2.755563in}{3.027050in}}%
\pgfpathlineto{\pgfqpoint{2.499217in}{2.903684in}}%
\pgfpathclose%
\pgfusepath{fill}%
\end{pgfscope}%
\begin{pgfscope}%
\pgfpathrectangle{\pgfqpoint{0.017500in}{0.017500in}}{\pgfqpoint{6.218282in}{3.580372in}} %
\pgfusepath{clip}%
\pgfsetbuttcap%
\pgfsetroundjoin%
\definecolor{currentfill}{rgb}{0.711554,0.033337,0.154485}%
\pgfsetfillcolor{currentfill}%
\pgfsetlinewidth{0.000000pt}%
\definecolor{currentstroke}{rgb}{0.000000,0.000000,0.000000}%
\pgfsetstrokecolor{currentstroke}%
\pgfsetdash{}{0pt}%
\pgfpathmoveto{\pgfqpoint{2.499217in}{2.903684in}}%
\pgfpathlineto{\pgfqpoint{2.932968in}{2.820668in}}%
\pgfpathlineto{\pgfqpoint{3.186131in}{2.954270in}}%
\pgfpathlineto{\pgfqpoint{2.755563in}{3.027050in}}%
\pgfpathlineto{\pgfqpoint{2.499217in}{2.903684in}}%
\pgfpathclose%
\pgfusepath{fill}%
\end{pgfscope}%
\begin{pgfscope}%
\pgfpathrectangle{\pgfqpoint{0.017500in}{0.017500in}}{\pgfqpoint{6.218282in}{3.580372in}} %
\pgfusepath{clip}%
\pgfsetbuttcap%
\pgfsetroundjoin%
\definecolor{currentfill}{rgb}{0.711554,0.033337,0.154485}%
\pgfsetfillcolor{currentfill}%
\pgfsetlinewidth{0.000000pt}%
\definecolor{currentstroke}{rgb}{0.000000,0.000000,0.000000}%
\pgfsetstrokecolor{currentstroke}%
\pgfsetdash{}{0pt}%
\pgfpathmoveto{\pgfqpoint{2.499217in}{2.903684in}}%
\pgfpathlineto{\pgfqpoint{2.932968in}{2.820668in}}%
\pgfpathlineto{\pgfqpoint{3.186131in}{2.954270in}}%
\pgfpathlineto{\pgfqpoint{2.755563in}{3.027050in}}%
\pgfpathlineto{\pgfqpoint{2.499217in}{2.903684in}}%
\pgfpathclose%
\pgfusepath{fill}%
\end{pgfscope}%
\begin{pgfscope}%
\pgfpathrectangle{\pgfqpoint{0.017500in}{0.017500in}}{\pgfqpoint{6.218282in}{3.580372in}} %
\pgfusepath{clip}%
\pgfsetbuttcap%
\pgfsetroundjoin%
\definecolor{currentfill}{rgb}{0.711554,0.033337,0.154485}%
\pgfsetfillcolor{currentfill}%
\pgfsetlinewidth{0.000000pt}%
\definecolor{currentstroke}{rgb}{0.000000,0.000000,0.000000}%
\pgfsetstrokecolor{currentstroke}%
\pgfsetdash{}{0pt}%
\pgfpathmoveto{\pgfqpoint{2.499217in}{2.903684in}}%
\pgfpathlineto{\pgfqpoint{2.932968in}{2.820668in}}%
\pgfpathlineto{\pgfqpoint{3.186131in}{2.954270in}}%
\pgfpathlineto{\pgfqpoint{2.755563in}{3.027050in}}%
\pgfpathlineto{\pgfqpoint{2.499217in}{2.903684in}}%
\pgfpathclose%
\pgfusepath{fill}%
\end{pgfscope}%
\begin{pgfscope}%
\pgfpathrectangle{\pgfqpoint{0.017500in}{0.017500in}}{\pgfqpoint{6.218282in}{3.580372in}} %
\pgfusepath{clip}%
\pgfsetbuttcap%
\pgfsetroundjoin%
\definecolor{currentfill}{rgb}{0.711554,0.033337,0.154485}%
\pgfsetfillcolor{currentfill}%
\pgfsetlinewidth{0.000000pt}%
\definecolor{currentstroke}{rgb}{0.000000,0.000000,0.000000}%
\pgfsetstrokecolor{currentstroke}%
\pgfsetdash{}{0pt}%
\pgfpathmoveto{\pgfqpoint{2.499217in}{2.903684in}}%
\pgfpathlineto{\pgfqpoint{2.932968in}{2.820668in}}%
\pgfpathlineto{\pgfqpoint{3.186131in}{2.954270in}}%
\pgfpathlineto{\pgfqpoint{2.755563in}{3.027050in}}%
\pgfpathlineto{\pgfqpoint{2.499217in}{2.903684in}}%
\pgfpathclose%
\pgfusepath{fill}%
\end{pgfscope}%
\begin{pgfscope}%
\pgfpathrectangle{\pgfqpoint{0.017500in}{0.017500in}}{\pgfqpoint{6.218282in}{3.580372in}} %
\pgfusepath{clip}%
\pgfsetbuttcap%
\pgfsetroundjoin%
\definecolor{currentfill}{rgb}{0.711554,0.033337,0.154485}%
\pgfsetfillcolor{currentfill}%
\pgfsetlinewidth{0.000000pt}%
\definecolor{currentstroke}{rgb}{0.000000,0.000000,0.000000}%
\pgfsetstrokecolor{currentstroke}%
\pgfsetdash{}{0pt}%
\pgfpathmoveto{\pgfqpoint{2.499217in}{2.903684in}}%
\pgfpathlineto{\pgfqpoint{2.932968in}{2.820668in}}%
\pgfpathlineto{\pgfqpoint{3.186131in}{2.954270in}}%
\pgfpathlineto{\pgfqpoint{2.755563in}{3.027050in}}%
\pgfpathlineto{\pgfqpoint{2.499217in}{2.903684in}}%
\pgfpathclose%
\pgfusepath{fill}%
\end{pgfscope}%
\begin{pgfscope}%
\pgfpathrectangle{\pgfqpoint{0.017500in}{0.017500in}}{\pgfqpoint{6.218282in}{3.580372in}} %
\pgfusepath{clip}%
\pgfsetbuttcap%
\pgfsetroundjoin%
\definecolor{currentfill}{rgb}{0.902849,0.844796,0.811970}%
\pgfsetfillcolor{currentfill}%
\pgfsetlinewidth{0.000000pt}%
\definecolor{currentstroke}{rgb}{0.000000,0.000000,0.000000}%
\pgfsetstrokecolor{currentstroke}%
\pgfsetdash{}{0pt}%
\pgfpathmoveto{\pgfqpoint{3.186131in}{2.954270in}}%
\pgfpathlineto{\pgfqpoint{1.914830in}{0.967594in}}%
\pgfpathlineto{\pgfqpoint{3.186131in}{2.954270in}}%
\pgfpathclose%
\pgfusepath{fill}%
\end{pgfscope}%
\begin{pgfscope}%
\pgfpathrectangle{\pgfqpoint{0.017500in}{0.017500in}}{\pgfqpoint{6.218282in}{3.580372in}} %
\pgfusepath{clip}%
\pgfsetbuttcap%
\pgfsetroundjoin%
\definecolor{currentfill}{rgb}{0.902849,0.844796,0.811970}%
\pgfsetfillcolor{currentfill}%
\pgfsetlinewidth{0.000000pt}%
\definecolor{currentstroke}{rgb}{0.000000,0.000000,0.000000}%
\pgfsetstrokecolor{currentstroke}%
\pgfsetdash{}{0pt}%
\pgfpathmoveto{\pgfqpoint{3.186131in}{2.954270in}}%
\pgfpathlineto{\pgfqpoint{1.914830in}{0.967594in}}%
\pgfpathlineto{\pgfqpoint{3.186131in}{2.954270in}}%
\pgfpathclose%
\pgfusepath{fill}%
\end{pgfscope}%
\begin{pgfscope}%
\pgfpathrectangle{\pgfqpoint{0.017500in}{0.017500in}}{\pgfqpoint{6.218282in}{3.580372in}} %
\pgfusepath{clip}%
\pgfsetbuttcap%
\pgfsetroundjoin%
\definecolor{currentfill}{rgb}{0.902849,0.844796,0.811970}%
\pgfsetfillcolor{currentfill}%
\pgfsetlinewidth{0.000000pt}%
\definecolor{currentstroke}{rgb}{0.000000,0.000000,0.000000}%
\pgfsetstrokecolor{currentstroke}%
\pgfsetdash{}{0pt}%
\pgfpathmoveto{\pgfqpoint{3.186131in}{2.954270in}}%
\pgfpathlineto{\pgfqpoint{1.914830in}{0.967594in}}%
\pgfpathlineto{\pgfqpoint{3.186131in}{2.954270in}}%
\pgfpathclose%
\pgfusepath{fill}%
\end{pgfscope}%
\begin{pgfscope}%
\pgfpathrectangle{\pgfqpoint{0.017500in}{0.017500in}}{\pgfqpoint{6.218282in}{3.580372in}} %
\pgfusepath{clip}%
\pgfsetbuttcap%
\pgfsetroundjoin%
\definecolor{currentfill}{rgb}{0.902849,0.844796,0.811970}%
\pgfsetfillcolor{currentfill}%
\pgfsetlinewidth{0.000000pt}%
\definecolor{currentstroke}{rgb}{0.000000,0.000000,0.000000}%
\pgfsetstrokecolor{currentstroke}%
\pgfsetdash{}{0pt}%
\pgfpathmoveto{\pgfqpoint{3.186131in}{2.954270in}}%
\pgfpathlineto{\pgfqpoint{1.914830in}{0.967594in}}%
\pgfpathlineto{\pgfqpoint{3.186131in}{2.954270in}}%
\pgfpathclose%
\pgfusepath{fill}%
\end{pgfscope}%
\begin{pgfscope}%
\pgfpathrectangle{\pgfqpoint{0.017500in}{0.017500in}}{\pgfqpoint{6.218282in}{3.580372in}} %
\pgfusepath{clip}%
\pgfsetbuttcap%
\pgfsetroundjoin%
\definecolor{currentfill}{rgb}{0.902849,0.844796,0.811970}%
\pgfsetfillcolor{currentfill}%
\pgfsetlinewidth{0.000000pt}%
\definecolor{currentstroke}{rgb}{0.000000,0.000000,0.000000}%
\pgfsetstrokecolor{currentstroke}%
\pgfsetdash{}{0pt}%
\pgfpathmoveto{\pgfqpoint{3.186131in}{2.954270in}}%
\pgfpathlineto{\pgfqpoint{1.914830in}{0.967594in}}%
\pgfpathlineto{\pgfqpoint{3.186131in}{2.954270in}}%
\pgfpathclose%
\pgfusepath{fill}%
\end{pgfscope}%
\begin{pgfscope}%
\pgfpathrectangle{\pgfqpoint{0.017500in}{0.017500in}}{\pgfqpoint{6.218282in}{3.580372in}} %
\pgfusepath{clip}%
\pgfsetbuttcap%
\pgfsetroundjoin%
\definecolor{currentfill}{rgb}{0.902849,0.844796,0.811970}%
\pgfsetfillcolor{currentfill}%
\pgfsetlinewidth{0.000000pt}%
\definecolor{currentstroke}{rgb}{0.000000,0.000000,0.000000}%
\pgfsetstrokecolor{currentstroke}%
\pgfsetdash{}{0pt}%
\pgfpathmoveto{\pgfqpoint{3.186131in}{2.954270in}}%
\pgfpathlineto{\pgfqpoint{1.914830in}{0.967594in}}%
\pgfpathlineto{\pgfqpoint{3.186131in}{2.954270in}}%
\pgfpathclose%
\pgfusepath{fill}%
\end{pgfscope}%
\begin{pgfscope}%
\pgfpathrectangle{\pgfqpoint{0.017500in}{0.017500in}}{\pgfqpoint{6.218282in}{3.580372in}} %
\pgfusepath{clip}%
\pgfsetbuttcap%
\pgfsetroundjoin%
\definecolor{currentfill}{rgb}{0.902849,0.844796,0.811970}%
\pgfsetfillcolor{currentfill}%
\pgfsetlinewidth{0.000000pt}%
\definecolor{currentstroke}{rgb}{0.000000,0.000000,0.000000}%
\pgfsetstrokecolor{currentstroke}%
\pgfsetdash{}{0pt}%
\pgfpathmoveto{\pgfqpoint{3.186131in}{2.954270in}}%
\pgfpathlineto{\pgfqpoint{1.914830in}{0.967594in}}%
\pgfpathlineto{\pgfqpoint{3.186131in}{2.954270in}}%
\pgfpathclose%
\pgfusepath{fill}%
\end{pgfscope}%
\begin{pgfscope}%
\pgfpathrectangle{\pgfqpoint{0.017500in}{0.017500in}}{\pgfqpoint{6.218282in}{3.580372in}} %
\pgfusepath{clip}%
\pgfsetbuttcap%
\pgfsetroundjoin%
\definecolor{currentfill}{rgb}{0.902849,0.844796,0.811970}%
\pgfsetfillcolor{currentfill}%
\pgfsetlinewidth{0.000000pt}%
\definecolor{currentstroke}{rgb}{0.000000,0.000000,0.000000}%
\pgfsetstrokecolor{currentstroke}%
\pgfsetdash{}{0pt}%
\pgfpathmoveto{\pgfqpoint{3.186131in}{2.954270in}}%
\pgfpathlineto{\pgfqpoint{1.914830in}{0.967594in}}%
\pgfpathlineto{\pgfqpoint{3.186131in}{2.954270in}}%
\pgfpathclose%
\pgfusepath{fill}%
\end{pgfscope}%
\begin{pgfscope}%
\pgfpathrectangle{\pgfqpoint{0.017500in}{0.017500in}}{\pgfqpoint{6.218282in}{3.580372in}} %
\pgfusepath{clip}%
\pgfsetbuttcap%
\pgfsetroundjoin%
\definecolor{currentfill}{rgb}{0.902849,0.844796,0.811970}%
\pgfsetfillcolor{currentfill}%
\pgfsetlinewidth{0.000000pt}%
\definecolor{currentstroke}{rgb}{0.000000,0.000000,0.000000}%
\pgfsetstrokecolor{currentstroke}%
\pgfsetdash{}{0pt}%
\pgfpathmoveto{\pgfqpoint{3.186131in}{2.954270in}}%
\pgfpathlineto{\pgfqpoint{1.914830in}{0.967594in}}%
\pgfpathlineto{\pgfqpoint{3.186131in}{2.954270in}}%
\pgfpathclose%
\pgfusepath{fill}%
\end{pgfscope}%
\begin{pgfscope}%
\pgfpathrectangle{\pgfqpoint{0.017500in}{0.017500in}}{\pgfqpoint{6.218282in}{3.580372in}} %
\pgfusepath{clip}%
\pgfsetbuttcap%
\pgfsetroundjoin%
\definecolor{currentfill}{rgb}{0.902849,0.844796,0.811970}%
\pgfsetfillcolor{currentfill}%
\pgfsetlinewidth{0.000000pt}%
\definecolor{currentstroke}{rgb}{0.000000,0.000000,0.000000}%
\pgfsetstrokecolor{currentstroke}%
\pgfsetdash{}{0pt}%
\pgfpathmoveto{\pgfqpoint{3.186131in}{2.954270in}}%
\pgfpathlineto{\pgfqpoint{1.914830in}{0.967594in}}%
\pgfpathlineto{\pgfqpoint{3.186131in}{2.954270in}}%
\pgfpathclose%
\pgfusepath{fill}%
\end{pgfscope}%
\begin{pgfscope}%
\pgfpathrectangle{\pgfqpoint{0.017500in}{0.017500in}}{\pgfqpoint{6.218282in}{3.580372in}} %
\pgfusepath{clip}%
\pgfsetbuttcap%
\pgfsetroundjoin%
\definecolor{currentfill}{rgb}{0.902849,0.844796,0.811970}%
\pgfsetfillcolor{currentfill}%
\pgfsetlinewidth{0.000000pt}%
\definecolor{currentstroke}{rgb}{0.000000,0.000000,0.000000}%
\pgfsetstrokecolor{currentstroke}%
\pgfsetdash{}{0pt}%
\pgfpathmoveto{\pgfqpoint{3.186131in}{2.954270in}}%
\pgfpathlineto{\pgfqpoint{1.914830in}{0.967594in}}%
\pgfpathlineto{\pgfqpoint{3.186131in}{2.954270in}}%
\pgfpathclose%
\pgfusepath{fill}%
\end{pgfscope}%
\begin{pgfscope}%
\pgfpathrectangle{\pgfqpoint{0.017500in}{0.017500in}}{\pgfqpoint{6.218282in}{3.580372in}} %
\pgfusepath{clip}%
\pgfsetbuttcap%
\pgfsetroundjoin%
\definecolor{currentfill}{rgb}{0.902849,0.844796,0.811970}%
\pgfsetfillcolor{currentfill}%
\pgfsetlinewidth{0.000000pt}%
\definecolor{currentstroke}{rgb}{0.000000,0.000000,0.000000}%
\pgfsetstrokecolor{currentstroke}%
\pgfsetdash{}{0pt}%
\pgfpathmoveto{\pgfqpoint{3.186131in}{2.954270in}}%
\pgfpathlineto{\pgfqpoint{1.914830in}{0.967594in}}%
\pgfpathlineto{\pgfqpoint{3.186131in}{2.954270in}}%
\pgfpathclose%
\pgfusepath{fill}%
\end{pgfscope}%
\begin{pgfscope}%
\pgfpathrectangle{\pgfqpoint{0.017500in}{0.017500in}}{\pgfqpoint{6.218282in}{3.580372in}} %
\pgfusepath{clip}%
\pgfsetbuttcap%
\pgfsetroundjoin%
\definecolor{currentfill}{rgb}{0.902849,0.844796,0.811970}%
\pgfsetfillcolor{currentfill}%
\pgfsetlinewidth{0.000000pt}%
\definecolor{currentstroke}{rgb}{0.000000,0.000000,0.000000}%
\pgfsetstrokecolor{currentstroke}%
\pgfsetdash{}{0pt}%
\pgfpathmoveto{\pgfqpoint{3.186131in}{2.954270in}}%
\pgfpathlineto{\pgfqpoint{1.914830in}{0.967594in}}%
\pgfpathlineto{\pgfqpoint{3.186131in}{2.954270in}}%
\pgfpathclose%
\pgfusepath{fill}%
\end{pgfscope}%
\begin{pgfscope}%
\pgfpathrectangle{\pgfqpoint{0.017500in}{0.017500in}}{\pgfqpoint{6.218282in}{3.580372in}} %
\pgfusepath{clip}%
\pgfsetbuttcap%
\pgfsetroundjoin%
\definecolor{currentfill}{rgb}{0.902849,0.844796,0.811970}%
\pgfsetfillcolor{currentfill}%
\pgfsetlinewidth{0.000000pt}%
\definecolor{currentstroke}{rgb}{0.000000,0.000000,0.000000}%
\pgfsetstrokecolor{currentstroke}%
\pgfsetdash{}{0pt}%
\pgfpathmoveto{\pgfqpoint{3.186131in}{2.954270in}}%
\pgfpathlineto{\pgfqpoint{1.914830in}{0.967594in}}%
\pgfpathlineto{\pgfqpoint{3.186131in}{2.954270in}}%
\pgfpathclose%
\pgfusepath{fill}%
\end{pgfscope}%
\begin{pgfscope}%
\pgfpathrectangle{\pgfqpoint{0.017500in}{0.017500in}}{\pgfqpoint{6.218282in}{3.580372in}} %
\pgfusepath{clip}%
\pgfsetbuttcap%
\pgfsetroundjoin%
\definecolor{currentfill}{rgb}{0.902849,0.844796,0.811970}%
\pgfsetfillcolor{currentfill}%
\pgfsetlinewidth{0.000000pt}%
\definecolor{currentstroke}{rgb}{0.000000,0.000000,0.000000}%
\pgfsetstrokecolor{currentstroke}%
\pgfsetdash{}{0pt}%
\pgfpathmoveto{\pgfqpoint{3.186131in}{2.954270in}}%
\pgfpathlineto{\pgfqpoint{1.914830in}{0.967594in}}%
\pgfpathlineto{\pgfqpoint{3.186131in}{2.954270in}}%
\pgfpathclose%
\pgfusepath{fill}%
\end{pgfscope}%
\begin{pgfscope}%
\pgfpathrectangle{\pgfqpoint{0.017500in}{0.017500in}}{\pgfqpoint{6.218282in}{3.580372in}} %
\pgfusepath{clip}%
\pgfsetbuttcap%
\pgfsetroundjoin%
\definecolor{currentfill}{rgb}{0.902849,0.844796,0.811970}%
\pgfsetfillcolor{currentfill}%
\pgfsetlinewidth{0.000000pt}%
\definecolor{currentstroke}{rgb}{0.000000,0.000000,0.000000}%
\pgfsetstrokecolor{currentstroke}%
\pgfsetdash{}{0pt}%
\pgfpathmoveto{\pgfqpoint{3.186131in}{2.954270in}}%
\pgfpathlineto{\pgfqpoint{1.914830in}{0.967594in}}%
\pgfpathlineto{\pgfqpoint{3.186131in}{2.954270in}}%
\pgfpathclose%
\pgfusepath{fill}%
\end{pgfscope}%
\begin{pgfscope}%
\pgfpathrectangle{\pgfqpoint{0.017500in}{0.017500in}}{\pgfqpoint{6.218282in}{3.580372in}} %
\pgfusepath{clip}%
\pgfsetbuttcap%
\pgfsetroundjoin%
\definecolor{currentfill}{rgb}{0.902849,0.844796,0.811970}%
\pgfsetfillcolor{currentfill}%
\pgfsetlinewidth{0.000000pt}%
\definecolor{currentstroke}{rgb}{0.000000,0.000000,0.000000}%
\pgfsetstrokecolor{currentstroke}%
\pgfsetdash{}{0pt}%
\pgfpathmoveto{\pgfqpoint{3.186131in}{2.954270in}}%
\pgfpathlineto{\pgfqpoint{1.914830in}{0.967594in}}%
\pgfpathlineto{\pgfqpoint{3.186131in}{2.954270in}}%
\pgfpathclose%
\pgfusepath{fill}%
\end{pgfscope}%
\begin{pgfscope}%
\pgfpathrectangle{\pgfqpoint{0.017500in}{0.017500in}}{\pgfqpoint{6.218282in}{3.580372in}} %
\pgfusepath{clip}%
\pgfsetbuttcap%
\pgfsetroundjoin%
\definecolor{currentfill}{rgb}{0.902849,0.844796,0.811970}%
\pgfsetfillcolor{currentfill}%
\pgfsetlinewidth{0.000000pt}%
\definecolor{currentstroke}{rgb}{0.000000,0.000000,0.000000}%
\pgfsetstrokecolor{currentstroke}%
\pgfsetdash{}{0pt}%
\pgfpathmoveto{\pgfqpoint{3.186131in}{2.954270in}}%
\pgfpathlineto{\pgfqpoint{1.914830in}{0.967594in}}%
\pgfpathlineto{\pgfqpoint{3.186131in}{2.954270in}}%
\pgfpathclose%
\pgfusepath{fill}%
\end{pgfscope}%
\begin{pgfscope}%
\pgfpathrectangle{\pgfqpoint{0.017500in}{0.017500in}}{\pgfqpoint{6.218282in}{3.580372in}} %
\pgfusepath{clip}%
\pgfsetbuttcap%
\pgfsetroundjoin%
\definecolor{currentfill}{rgb}{0.902849,0.844796,0.811970}%
\pgfsetfillcolor{currentfill}%
\pgfsetlinewidth{0.000000pt}%
\definecolor{currentstroke}{rgb}{0.000000,0.000000,0.000000}%
\pgfsetstrokecolor{currentstroke}%
\pgfsetdash{}{0pt}%
\pgfpathmoveto{\pgfqpoint{3.186131in}{2.954270in}}%
\pgfpathlineto{\pgfqpoint{1.914830in}{0.967594in}}%
\pgfpathlineto{\pgfqpoint{3.186131in}{2.954270in}}%
\pgfpathclose%
\pgfusepath{fill}%
\end{pgfscope}%
\begin{pgfscope}%
\pgfpathrectangle{\pgfqpoint{0.017500in}{0.017500in}}{\pgfqpoint{6.218282in}{3.580372in}} %
\pgfusepath{clip}%
\pgfsetbuttcap%
\pgfsetroundjoin%
\definecolor{currentfill}{rgb}{0.902849,0.844796,0.811970}%
\pgfsetfillcolor{currentfill}%
\pgfsetlinewidth{0.000000pt}%
\definecolor{currentstroke}{rgb}{0.000000,0.000000,0.000000}%
\pgfsetstrokecolor{currentstroke}%
\pgfsetdash{}{0pt}%
\pgfpathmoveto{\pgfqpoint{3.186131in}{2.954270in}}%
\pgfpathlineto{\pgfqpoint{1.914830in}{0.967594in}}%
\pgfpathlineto{\pgfqpoint{3.186131in}{2.954270in}}%
\pgfpathclose%
\pgfusepath{fill}%
\end{pgfscope}%
\begin{pgfscope}%
\pgfpathrectangle{\pgfqpoint{0.017500in}{0.017500in}}{\pgfqpoint{6.218282in}{3.580372in}} %
\pgfusepath{clip}%
\pgfsetbuttcap%
\pgfsetroundjoin%
\definecolor{currentfill}{rgb}{0.902849,0.844796,0.811970}%
\pgfsetfillcolor{currentfill}%
\pgfsetlinewidth{0.000000pt}%
\definecolor{currentstroke}{rgb}{0.000000,0.000000,0.000000}%
\pgfsetstrokecolor{currentstroke}%
\pgfsetdash{}{0pt}%
\pgfpathmoveto{\pgfqpoint{3.186131in}{2.954270in}}%
\pgfpathlineto{\pgfqpoint{1.914830in}{0.967594in}}%
\pgfpathlineto{\pgfqpoint{3.186131in}{2.954270in}}%
\pgfpathclose%
\pgfusepath{fill}%
\end{pgfscope}%
\begin{pgfscope}%
\pgfpathrectangle{\pgfqpoint{0.017500in}{0.017500in}}{\pgfqpoint{6.218282in}{3.580372in}} %
\pgfusepath{clip}%
\pgfsetbuttcap%
\pgfsetroundjoin%
\definecolor{currentfill}{rgb}{0.902849,0.844796,0.811970}%
\pgfsetfillcolor{currentfill}%
\pgfsetlinewidth{0.000000pt}%
\definecolor{currentstroke}{rgb}{0.000000,0.000000,0.000000}%
\pgfsetstrokecolor{currentstroke}%
\pgfsetdash{}{0pt}%
\pgfpathmoveto{\pgfqpoint{3.186131in}{2.954270in}}%
\pgfpathlineto{\pgfqpoint{1.914830in}{0.967594in}}%
\pgfpathlineto{\pgfqpoint{3.186131in}{2.954270in}}%
\pgfpathclose%
\pgfusepath{fill}%
\end{pgfscope}%
\begin{pgfscope}%
\pgfpathrectangle{\pgfqpoint{0.017500in}{0.017500in}}{\pgfqpoint{6.218282in}{3.580372in}} %
\pgfusepath{clip}%
\pgfsetbuttcap%
\pgfsetroundjoin%
\definecolor{currentfill}{rgb}{0.902849,0.844796,0.811970}%
\pgfsetfillcolor{currentfill}%
\pgfsetlinewidth{0.000000pt}%
\definecolor{currentstroke}{rgb}{0.000000,0.000000,0.000000}%
\pgfsetstrokecolor{currentstroke}%
\pgfsetdash{}{0pt}%
\pgfpathmoveto{\pgfqpoint{3.186131in}{2.954270in}}%
\pgfpathlineto{\pgfqpoint{1.914830in}{0.967594in}}%
\pgfpathlineto{\pgfqpoint{3.186131in}{2.954270in}}%
\pgfpathclose%
\pgfusepath{fill}%
\end{pgfscope}%
\begin{pgfscope}%
\pgfpathrectangle{\pgfqpoint{0.017500in}{0.017500in}}{\pgfqpoint{6.218282in}{3.580372in}} %
\pgfusepath{clip}%
\pgfsetbuttcap%
\pgfsetroundjoin%
\definecolor{currentfill}{rgb}{0.902849,0.844796,0.811970}%
\pgfsetfillcolor{currentfill}%
\pgfsetlinewidth{0.000000pt}%
\definecolor{currentstroke}{rgb}{0.000000,0.000000,0.000000}%
\pgfsetstrokecolor{currentstroke}%
\pgfsetdash{}{0pt}%
\pgfpathmoveto{\pgfqpoint{3.186131in}{2.954270in}}%
\pgfpathlineto{\pgfqpoint{1.914830in}{0.967594in}}%
\pgfpathlineto{\pgfqpoint{3.186131in}{2.954270in}}%
\pgfpathclose%
\pgfusepath{fill}%
\end{pgfscope}%
\begin{pgfscope}%
\pgfpathrectangle{\pgfqpoint{0.017500in}{0.017500in}}{\pgfqpoint{6.218282in}{3.580372in}} %
\pgfusepath{clip}%
\pgfsetbuttcap%
\pgfsetroundjoin%
\definecolor{currentfill}{rgb}{0.902849,0.844796,0.811970}%
\pgfsetfillcolor{currentfill}%
\pgfsetlinewidth{0.000000pt}%
\definecolor{currentstroke}{rgb}{0.000000,0.000000,0.000000}%
\pgfsetstrokecolor{currentstroke}%
\pgfsetdash{}{0pt}%
\pgfpathmoveto{\pgfqpoint{3.186131in}{2.954270in}}%
\pgfpathlineto{\pgfqpoint{1.914830in}{0.967594in}}%
\pgfpathlineto{\pgfqpoint{3.186131in}{2.954270in}}%
\pgfpathclose%
\pgfusepath{fill}%
\end{pgfscope}%
\begin{pgfscope}%
\pgfpathrectangle{\pgfqpoint{0.017500in}{0.017500in}}{\pgfqpoint{6.218282in}{3.580372in}} %
\pgfusepath{clip}%
\pgfsetbuttcap%
\pgfsetroundjoin%
\definecolor{currentfill}{rgb}{0.723315,0.068898,0.162989}%
\pgfsetfillcolor{currentfill}%
\pgfsetlinewidth{0.000000pt}%
\definecolor{currentstroke}{rgb}{0.000000,0.000000,0.000000}%
\pgfsetstrokecolor{currentstroke}%
\pgfsetdash{}{0pt}%
\pgfpathmoveto{\pgfqpoint{2.932968in}{2.820668in}}%
\pgfpathlineto{\pgfqpoint{3.186131in}{2.954270in}}%
\pgfpathlineto{\pgfqpoint{2.932968in}{2.820668in}}%
\pgfpathclose%
\pgfusepath{fill}%
\end{pgfscope}%
\begin{pgfscope}%
\pgfpathrectangle{\pgfqpoint{0.017500in}{0.017500in}}{\pgfqpoint{6.218282in}{3.580372in}} %
\pgfusepath{clip}%
\pgfsetbuttcap%
\pgfsetroundjoin%
\definecolor{currentfill}{rgb}{0.723315,0.068898,0.162989}%
\pgfsetfillcolor{currentfill}%
\pgfsetlinewidth{0.000000pt}%
\definecolor{currentstroke}{rgb}{0.000000,0.000000,0.000000}%
\pgfsetstrokecolor{currentstroke}%
\pgfsetdash{}{0pt}%
\pgfpathmoveto{\pgfqpoint{2.932968in}{2.820668in}}%
\pgfpathlineto{\pgfqpoint{3.186131in}{2.954270in}}%
\pgfpathlineto{\pgfqpoint{2.932968in}{2.820668in}}%
\pgfpathclose%
\pgfusepath{fill}%
\end{pgfscope}%
\begin{pgfscope}%
\pgfpathrectangle{\pgfqpoint{0.017500in}{0.017500in}}{\pgfqpoint{6.218282in}{3.580372in}} %
\pgfusepath{clip}%
\pgfsetbuttcap%
\pgfsetroundjoin%
\definecolor{currentfill}{rgb}{0.723315,0.068898,0.162989}%
\pgfsetfillcolor{currentfill}%
\pgfsetlinewidth{0.000000pt}%
\definecolor{currentstroke}{rgb}{0.000000,0.000000,0.000000}%
\pgfsetstrokecolor{currentstroke}%
\pgfsetdash{}{0pt}%
\pgfpathmoveto{\pgfqpoint{2.932968in}{2.820668in}}%
\pgfpathlineto{\pgfqpoint{3.186131in}{2.954270in}}%
\pgfpathlineto{\pgfqpoint{2.932968in}{2.820668in}}%
\pgfpathclose%
\pgfusepath{fill}%
\end{pgfscope}%
\begin{pgfscope}%
\pgfpathrectangle{\pgfqpoint{0.017500in}{0.017500in}}{\pgfqpoint{6.218282in}{3.580372in}} %
\pgfusepath{clip}%
\pgfsetbuttcap%
\pgfsetroundjoin%
\definecolor{currentfill}{rgb}{0.723315,0.068898,0.162989}%
\pgfsetfillcolor{currentfill}%
\pgfsetlinewidth{0.000000pt}%
\definecolor{currentstroke}{rgb}{0.000000,0.000000,0.000000}%
\pgfsetstrokecolor{currentstroke}%
\pgfsetdash{}{0pt}%
\pgfpathmoveto{\pgfqpoint{2.932968in}{2.820668in}}%
\pgfpathlineto{\pgfqpoint{3.186131in}{2.954270in}}%
\pgfpathlineto{\pgfqpoint{2.932968in}{2.820668in}}%
\pgfpathclose%
\pgfusepath{fill}%
\end{pgfscope}%
\begin{pgfscope}%
\pgfpathrectangle{\pgfqpoint{0.017500in}{0.017500in}}{\pgfqpoint{6.218282in}{3.580372in}} %
\pgfusepath{clip}%
\pgfsetbuttcap%
\pgfsetroundjoin%
\definecolor{currentfill}{rgb}{0.723315,0.068898,0.162989}%
\pgfsetfillcolor{currentfill}%
\pgfsetlinewidth{0.000000pt}%
\definecolor{currentstroke}{rgb}{0.000000,0.000000,0.000000}%
\pgfsetstrokecolor{currentstroke}%
\pgfsetdash{}{0pt}%
\pgfpathmoveto{\pgfqpoint{2.932968in}{2.820668in}}%
\pgfpathlineto{\pgfqpoint{3.186131in}{2.954270in}}%
\pgfpathlineto{\pgfqpoint{2.932968in}{2.820668in}}%
\pgfpathclose%
\pgfusepath{fill}%
\end{pgfscope}%
\begin{pgfscope}%
\pgfpathrectangle{\pgfqpoint{0.017500in}{0.017500in}}{\pgfqpoint{6.218282in}{3.580372in}} %
\pgfusepath{clip}%
\pgfsetbuttcap%
\pgfsetroundjoin%
\definecolor{currentfill}{rgb}{0.723315,0.068898,0.162989}%
\pgfsetfillcolor{currentfill}%
\pgfsetlinewidth{0.000000pt}%
\definecolor{currentstroke}{rgb}{0.000000,0.000000,0.000000}%
\pgfsetstrokecolor{currentstroke}%
\pgfsetdash{}{0pt}%
\pgfpathmoveto{\pgfqpoint{2.932968in}{2.820668in}}%
\pgfpathlineto{\pgfqpoint{3.186131in}{2.954270in}}%
\pgfpathlineto{\pgfqpoint{2.932968in}{2.820668in}}%
\pgfpathclose%
\pgfusepath{fill}%
\end{pgfscope}%
\begin{pgfscope}%
\pgfpathrectangle{\pgfqpoint{0.017500in}{0.017500in}}{\pgfqpoint{6.218282in}{3.580372in}} %
\pgfusepath{clip}%
\pgfsetbuttcap%
\pgfsetroundjoin%
\definecolor{currentfill}{rgb}{0.723315,0.068898,0.162989}%
\pgfsetfillcolor{currentfill}%
\pgfsetlinewidth{0.000000pt}%
\definecolor{currentstroke}{rgb}{0.000000,0.000000,0.000000}%
\pgfsetstrokecolor{currentstroke}%
\pgfsetdash{}{0pt}%
\pgfpathmoveto{\pgfqpoint{2.932968in}{2.820668in}}%
\pgfpathlineto{\pgfqpoint{3.186131in}{2.954270in}}%
\pgfpathlineto{\pgfqpoint{2.932968in}{2.820668in}}%
\pgfpathclose%
\pgfusepath{fill}%
\end{pgfscope}%
\begin{pgfscope}%
\pgfpathrectangle{\pgfqpoint{0.017500in}{0.017500in}}{\pgfqpoint{6.218282in}{3.580372in}} %
\pgfusepath{clip}%
\pgfsetbuttcap%
\pgfsetroundjoin%
\definecolor{currentfill}{rgb}{0.723315,0.068898,0.162989}%
\pgfsetfillcolor{currentfill}%
\pgfsetlinewidth{0.000000pt}%
\definecolor{currentstroke}{rgb}{0.000000,0.000000,0.000000}%
\pgfsetstrokecolor{currentstroke}%
\pgfsetdash{}{0pt}%
\pgfpathmoveto{\pgfqpoint{2.932968in}{2.820668in}}%
\pgfpathlineto{\pgfqpoint{3.186131in}{2.954270in}}%
\pgfpathlineto{\pgfqpoint{2.932968in}{2.820668in}}%
\pgfpathclose%
\pgfusepath{fill}%
\end{pgfscope}%
\begin{pgfscope}%
\pgfpathrectangle{\pgfqpoint{0.017500in}{0.017500in}}{\pgfqpoint{6.218282in}{3.580372in}} %
\pgfusepath{clip}%
\pgfsetbuttcap%
\pgfsetroundjoin%
\definecolor{currentfill}{rgb}{0.723315,0.068898,0.162989}%
\pgfsetfillcolor{currentfill}%
\pgfsetlinewidth{0.000000pt}%
\definecolor{currentstroke}{rgb}{0.000000,0.000000,0.000000}%
\pgfsetstrokecolor{currentstroke}%
\pgfsetdash{}{0pt}%
\pgfpathmoveto{\pgfqpoint{2.932968in}{2.820668in}}%
\pgfpathlineto{\pgfqpoint{3.186131in}{2.954270in}}%
\pgfpathlineto{\pgfqpoint{2.932968in}{2.820668in}}%
\pgfpathclose%
\pgfusepath{fill}%
\end{pgfscope}%
\begin{pgfscope}%
\pgfpathrectangle{\pgfqpoint{0.017500in}{0.017500in}}{\pgfqpoint{6.218282in}{3.580372in}} %
\pgfusepath{clip}%
\pgfsetbuttcap%
\pgfsetroundjoin%
\definecolor{currentfill}{rgb}{0.723315,0.068898,0.162989}%
\pgfsetfillcolor{currentfill}%
\pgfsetlinewidth{0.000000pt}%
\definecolor{currentstroke}{rgb}{0.000000,0.000000,0.000000}%
\pgfsetstrokecolor{currentstroke}%
\pgfsetdash{}{0pt}%
\pgfpathmoveto{\pgfqpoint{2.932968in}{2.820668in}}%
\pgfpathlineto{\pgfqpoint{3.186131in}{2.954270in}}%
\pgfpathlineto{\pgfqpoint{2.932968in}{2.820668in}}%
\pgfpathclose%
\pgfusepath{fill}%
\end{pgfscope}%
\begin{pgfscope}%
\pgfpathrectangle{\pgfqpoint{0.017500in}{0.017500in}}{\pgfqpoint{6.218282in}{3.580372in}} %
\pgfusepath{clip}%
\pgfsetbuttcap%
\pgfsetroundjoin%
\definecolor{currentfill}{rgb}{0.723315,0.068898,0.162989}%
\pgfsetfillcolor{currentfill}%
\pgfsetlinewidth{0.000000pt}%
\definecolor{currentstroke}{rgb}{0.000000,0.000000,0.000000}%
\pgfsetstrokecolor{currentstroke}%
\pgfsetdash{}{0pt}%
\pgfpathmoveto{\pgfqpoint{2.932968in}{2.820668in}}%
\pgfpathlineto{\pgfqpoint{3.186131in}{2.954270in}}%
\pgfpathlineto{\pgfqpoint{2.932968in}{2.820668in}}%
\pgfpathclose%
\pgfusepath{fill}%
\end{pgfscope}%
\begin{pgfscope}%
\pgfpathrectangle{\pgfqpoint{0.017500in}{0.017500in}}{\pgfqpoint{6.218282in}{3.580372in}} %
\pgfusepath{clip}%
\pgfsetbuttcap%
\pgfsetroundjoin%
\definecolor{currentfill}{rgb}{0.723315,0.068898,0.162989}%
\pgfsetfillcolor{currentfill}%
\pgfsetlinewidth{0.000000pt}%
\definecolor{currentstroke}{rgb}{0.000000,0.000000,0.000000}%
\pgfsetstrokecolor{currentstroke}%
\pgfsetdash{}{0pt}%
\pgfpathmoveto{\pgfqpoint{2.932968in}{2.820668in}}%
\pgfpathlineto{\pgfqpoint{3.186131in}{2.954270in}}%
\pgfpathlineto{\pgfqpoint{2.932968in}{2.820668in}}%
\pgfpathclose%
\pgfusepath{fill}%
\end{pgfscope}%
\begin{pgfscope}%
\pgfpathrectangle{\pgfqpoint{0.017500in}{0.017500in}}{\pgfqpoint{6.218282in}{3.580372in}} %
\pgfusepath{clip}%
\pgfsetbuttcap%
\pgfsetroundjoin%
\definecolor{currentfill}{rgb}{0.723315,0.068898,0.162989}%
\pgfsetfillcolor{currentfill}%
\pgfsetlinewidth{0.000000pt}%
\definecolor{currentstroke}{rgb}{0.000000,0.000000,0.000000}%
\pgfsetstrokecolor{currentstroke}%
\pgfsetdash{}{0pt}%
\pgfpathmoveto{\pgfqpoint{2.932968in}{2.820668in}}%
\pgfpathlineto{\pgfqpoint{3.186131in}{2.954270in}}%
\pgfpathlineto{\pgfqpoint{2.932968in}{2.820668in}}%
\pgfpathclose%
\pgfusepath{fill}%
\end{pgfscope}%
\begin{pgfscope}%
\pgfpathrectangle{\pgfqpoint{0.017500in}{0.017500in}}{\pgfqpoint{6.218282in}{3.580372in}} %
\pgfusepath{clip}%
\pgfsetbuttcap%
\pgfsetroundjoin%
\definecolor{currentfill}{rgb}{0.723315,0.068898,0.162989}%
\pgfsetfillcolor{currentfill}%
\pgfsetlinewidth{0.000000pt}%
\definecolor{currentstroke}{rgb}{0.000000,0.000000,0.000000}%
\pgfsetstrokecolor{currentstroke}%
\pgfsetdash{}{0pt}%
\pgfpathmoveto{\pgfqpoint{2.932968in}{2.820668in}}%
\pgfpathlineto{\pgfqpoint{3.186131in}{2.954270in}}%
\pgfpathlineto{\pgfqpoint{2.932968in}{2.820668in}}%
\pgfpathclose%
\pgfusepath{fill}%
\end{pgfscope}%
\begin{pgfscope}%
\pgfpathrectangle{\pgfqpoint{0.017500in}{0.017500in}}{\pgfqpoint{6.218282in}{3.580372in}} %
\pgfusepath{clip}%
\pgfsetbuttcap%
\pgfsetroundjoin%
\definecolor{currentfill}{rgb}{0.723315,0.068898,0.162989}%
\pgfsetfillcolor{currentfill}%
\pgfsetlinewidth{0.000000pt}%
\definecolor{currentstroke}{rgb}{0.000000,0.000000,0.000000}%
\pgfsetstrokecolor{currentstroke}%
\pgfsetdash{}{0pt}%
\pgfpathmoveto{\pgfqpoint{2.932968in}{2.820668in}}%
\pgfpathlineto{\pgfqpoint{3.186131in}{2.954270in}}%
\pgfpathlineto{\pgfqpoint{2.932968in}{2.820668in}}%
\pgfpathclose%
\pgfusepath{fill}%
\end{pgfscope}%
\begin{pgfscope}%
\pgfpathrectangle{\pgfqpoint{0.017500in}{0.017500in}}{\pgfqpoint{6.218282in}{3.580372in}} %
\pgfusepath{clip}%
\pgfsetbuttcap%
\pgfsetroundjoin%
\definecolor{currentfill}{rgb}{0.723315,0.068898,0.162989}%
\pgfsetfillcolor{currentfill}%
\pgfsetlinewidth{0.000000pt}%
\definecolor{currentstroke}{rgb}{0.000000,0.000000,0.000000}%
\pgfsetstrokecolor{currentstroke}%
\pgfsetdash{}{0pt}%
\pgfpathmoveto{\pgfqpoint{2.932968in}{2.820668in}}%
\pgfpathlineto{\pgfqpoint{3.186131in}{2.954270in}}%
\pgfpathlineto{\pgfqpoint{2.932968in}{2.820668in}}%
\pgfpathclose%
\pgfusepath{fill}%
\end{pgfscope}%
\begin{pgfscope}%
\pgfpathrectangle{\pgfqpoint{0.017500in}{0.017500in}}{\pgfqpoint{6.218282in}{3.580372in}} %
\pgfusepath{clip}%
\pgfsetbuttcap%
\pgfsetroundjoin%
\definecolor{currentfill}{rgb}{0.723315,0.068898,0.162989}%
\pgfsetfillcolor{currentfill}%
\pgfsetlinewidth{0.000000pt}%
\definecolor{currentstroke}{rgb}{0.000000,0.000000,0.000000}%
\pgfsetstrokecolor{currentstroke}%
\pgfsetdash{}{0pt}%
\pgfpathmoveto{\pgfqpoint{2.932968in}{2.820668in}}%
\pgfpathlineto{\pgfqpoint{3.186131in}{2.954270in}}%
\pgfpathlineto{\pgfqpoint{2.932968in}{2.820668in}}%
\pgfpathclose%
\pgfusepath{fill}%
\end{pgfscope}%
\begin{pgfscope}%
\pgfpathrectangle{\pgfqpoint{0.017500in}{0.017500in}}{\pgfqpoint{6.218282in}{3.580372in}} %
\pgfusepath{clip}%
\pgfsetbuttcap%
\pgfsetroundjoin%
\definecolor{currentfill}{rgb}{0.723315,0.068898,0.162989}%
\pgfsetfillcolor{currentfill}%
\pgfsetlinewidth{0.000000pt}%
\definecolor{currentstroke}{rgb}{0.000000,0.000000,0.000000}%
\pgfsetstrokecolor{currentstroke}%
\pgfsetdash{}{0pt}%
\pgfpathmoveto{\pgfqpoint{2.932968in}{2.820668in}}%
\pgfpathlineto{\pgfqpoint{3.186131in}{2.954270in}}%
\pgfpathlineto{\pgfqpoint{2.932968in}{2.820668in}}%
\pgfpathclose%
\pgfusepath{fill}%
\end{pgfscope}%
\begin{pgfscope}%
\pgfpathrectangle{\pgfqpoint{0.017500in}{0.017500in}}{\pgfqpoint{6.218282in}{3.580372in}} %
\pgfusepath{clip}%
\pgfsetbuttcap%
\pgfsetroundjoin%
\definecolor{currentfill}{rgb}{0.723315,0.068898,0.162989}%
\pgfsetfillcolor{currentfill}%
\pgfsetlinewidth{0.000000pt}%
\definecolor{currentstroke}{rgb}{0.000000,0.000000,0.000000}%
\pgfsetstrokecolor{currentstroke}%
\pgfsetdash{}{0pt}%
\pgfpathmoveto{\pgfqpoint{2.932968in}{2.820668in}}%
\pgfpathlineto{\pgfqpoint{3.186131in}{2.954270in}}%
\pgfpathlineto{\pgfqpoint{2.932968in}{2.820668in}}%
\pgfpathclose%
\pgfusepath{fill}%
\end{pgfscope}%
\begin{pgfscope}%
\pgfpathrectangle{\pgfqpoint{0.017500in}{0.017500in}}{\pgfqpoint{6.218282in}{3.580372in}} %
\pgfusepath{clip}%
\pgfsetbuttcap%
\pgfsetroundjoin%
\definecolor{currentfill}{rgb}{0.723315,0.068898,0.162989}%
\pgfsetfillcolor{currentfill}%
\pgfsetlinewidth{0.000000pt}%
\definecolor{currentstroke}{rgb}{0.000000,0.000000,0.000000}%
\pgfsetstrokecolor{currentstroke}%
\pgfsetdash{}{0pt}%
\pgfpathmoveto{\pgfqpoint{2.932968in}{2.820668in}}%
\pgfpathlineto{\pgfqpoint{3.186131in}{2.954270in}}%
\pgfpathlineto{\pgfqpoint{2.932968in}{2.820668in}}%
\pgfpathclose%
\pgfusepath{fill}%
\end{pgfscope}%
\begin{pgfscope}%
\pgfpathrectangle{\pgfqpoint{0.017500in}{0.017500in}}{\pgfqpoint{6.218282in}{3.580372in}} %
\pgfusepath{clip}%
\pgfsetbuttcap%
\pgfsetroundjoin%
\definecolor{currentfill}{rgb}{0.723315,0.068898,0.162989}%
\pgfsetfillcolor{currentfill}%
\pgfsetlinewidth{0.000000pt}%
\definecolor{currentstroke}{rgb}{0.000000,0.000000,0.000000}%
\pgfsetstrokecolor{currentstroke}%
\pgfsetdash{}{0pt}%
\pgfpathmoveto{\pgfqpoint{2.932968in}{2.820668in}}%
\pgfpathlineto{\pgfqpoint{3.186131in}{2.954270in}}%
\pgfpathlineto{\pgfqpoint{2.932968in}{2.820668in}}%
\pgfpathclose%
\pgfusepath{fill}%
\end{pgfscope}%
\begin{pgfscope}%
\pgfpathrectangle{\pgfqpoint{0.017500in}{0.017500in}}{\pgfqpoint{6.218282in}{3.580372in}} %
\pgfusepath{clip}%
\pgfsetbuttcap%
\pgfsetroundjoin%
\definecolor{currentfill}{rgb}{0.723315,0.068898,0.162989}%
\pgfsetfillcolor{currentfill}%
\pgfsetlinewidth{0.000000pt}%
\definecolor{currentstroke}{rgb}{0.000000,0.000000,0.000000}%
\pgfsetstrokecolor{currentstroke}%
\pgfsetdash{}{0pt}%
\pgfpathmoveto{\pgfqpoint{2.932968in}{2.820668in}}%
\pgfpathlineto{\pgfqpoint{3.186131in}{2.954270in}}%
\pgfpathlineto{\pgfqpoint{2.932968in}{2.820668in}}%
\pgfpathclose%
\pgfusepath{fill}%
\end{pgfscope}%
\begin{pgfscope}%
\pgfpathrectangle{\pgfqpoint{0.017500in}{0.017500in}}{\pgfqpoint{6.218282in}{3.580372in}} %
\pgfusepath{clip}%
\pgfsetbuttcap%
\pgfsetroundjoin%
\definecolor{currentfill}{rgb}{0.723315,0.068898,0.162989}%
\pgfsetfillcolor{currentfill}%
\pgfsetlinewidth{0.000000pt}%
\definecolor{currentstroke}{rgb}{0.000000,0.000000,0.000000}%
\pgfsetstrokecolor{currentstroke}%
\pgfsetdash{}{0pt}%
\pgfpathmoveto{\pgfqpoint{2.932968in}{2.820668in}}%
\pgfpathlineto{\pgfqpoint{3.186131in}{2.954270in}}%
\pgfpathlineto{\pgfqpoint{2.932968in}{2.820668in}}%
\pgfpathclose%
\pgfusepath{fill}%
\end{pgfscope}%
\begin{pgfscope}%
\pgfpathrectangle{\pgfqpoint{0.017500in}{0.017500in}}{\pgfqpoint{6.218282in}{3.580372in}} %
\pgfusepath{clip}%
\pgfsetbuttcap%
\pgfsetroundjoin%
\definecolor{currentfill}{rgb}{0.723315,0.068898,0.162989}%
\pgfsetfillcolor{currentfill}%
\pgfsetlinewidth{0.000000pt}%
\definecolor{currentstroke}{rgb}{0.000000,0.000000,0.000000}%
\pgfsetstrokecolor{currentstroke}%
\pgfsetdash{}{0pt}%
\pgfpathmoveto{\pgfqpoint{2.932968in}{2.820668in}}%
\pgfpathlineto{\pgfqpoint{3.186131in}{2.954270in}}%
\pgfpathlineto{\pgfqpoint{2.932968in}{2.820668in}}%
\pgfpathclose%
\pgfusepath{fill}%
\end{pgfscope}%
\begin{pgfscope}%
\pgfpathrectangle{\pgfqpoint{0.017500in}{0.017500in}}{\pgfqpoint{6.218282in}{3.580372in}} %
\pgfusepath{clip}%
\pgfsetbuttcap%
\pgfsetroundjoin%
\definecolor{currentfill}{rgb}{0.723315,0.068898,0.162989}%
\pgfsetfillcolor{currentfill}%
\pgfsetlinewidth{0.000000pt}%
\definecolor{currentstroke}{rgb}{0.000000,0.000000,0.000000}%
\pgfsetstrokecolor{currentstroke}%
\pgfsetdash{}{0pt}%
\pgfpathmoveto{\pgfqpoint{2.932968in}{2.820668in}}%
\pgfpathlineto{\pgfqpoint{3.186131in}{2.954270in}}%
\pgfpathlineto{\pgfqpoint{2.932968in}{2.820668in}}%
\pgfpathclose%
\pgfusepath{fill}%
\end{pgfscope}%
\begin{pgfscope}%
\pgfpathrectangle{\pgfqpoint{0.017500in}{0.017500in}}{\pgfqpoint{6.218282in}{3.580372in}} %
\pgfusepath{clip}%
\pgfsetbuttcap%
\pgfsetroundjoin%
\definecolor{currentfill}{rgb}{0.723315,0.068898,0.162989}%
\pgfsetfillcolor{currentfill}%
\pgfsetlinewidth{0.000000pt}%
\definecolor{currentstroke}{rgb}{0.000000,0.000000,0.000000}%
\pgfsetstrokecolor{currentstroke}%
\pgfsetdash{}{0pt}%
\pgfpathmoveto{\pgfqpoint{2.932968in}{2.820668in}}%
\pgfpathlineto{\pgfqpoint{3.186131in}{2.954270in}}%
\pgfpathlineto{\pgfqpoint{2.932968in}{2.820668in}}%
\pgfpathclose%
\pgfusepath{fill}%
\end{pgfscope}%
\begin{pgfscope}%
\pgfpathrectangle{\pgfqpoint{0.017500in}{0.017500in}}{\pgfqpoint{6.218282in}{3.580372in}} %
\pgfusepath{clip}%
\pgfsetbuttcap%
\pgfsetroundjoin%
\definecolor{currentfill}{rgb}{0.723315,0.068898,0.162989}%
\pgfsetfillcolor{currentfill}%
\pgfsetlinewidth{0.000000pt}%
\definecolor{currentstroke}{rgb}{0.000000,0.000000,0.000000}%
\pgfsetstrokecolor{currentstroke}%
\pgfsetdash{}{0pt}%
\pgfpathmoveto{\pgfqpoint{2.932968in}{2.820668in}}%
\pgfpathlineto{\pgfqpoint{3.186131in}{2.954270in}}%
\pgfpathlineto{\pgfqpoint{2.932968in}{2.820668in}}%
\pgfpathclose%
\pgfusepath{fill}%
\end{pgfscope}%
\begin{pgfscope}%
\pgfpathrectangle{\pgfqpoint{0.017500in}{0.017500in}}{\pgfqpoint{6.218282in}{3.580372in}} %
\pgfusepath{clip}%
\pgfsetbuttcap%
\pgfsetroundjoin%
\definecolor{currentfill}{rgb}{0.723315,0.068898,0.162989}%
\pgfsetfillcolor{currentfill}%
\pgfsetlinewidth{0.000000pt}%
\definecolor{currentstroke}{rgb}{0.000000,0.000000,0.000000}%
\pgfsetstrokecolor{currentstroke}%
\pgfsetdash{}{0pt}%
\pgfpathmoveto{\pgfqpoint{2.932968in}{2.820668in}}%
\pgfpathlineto{\pgfqpoint{3.186131in}{2.954270in}}%
\pgfpathlineto{\pgfqpoint{2.932968in}{2.820668in}}%
\pgfpathclose%
\pgfusepath{fill}%
\end{pgfscope}%
\begin{pgfscope}%
\pgfpathrectangle{\pgfqpoint{0.017500in}{0.017500in}}{\pgfqpoint{6.218282in}{3.580372in}} %
\pgfusepath{clip}%
\pgfsetbuttcap%
\pgfsetroundjoin%
\definecolor{currentfill}{rgb}{0.723315,0.068898,0.162989}%
\pgfsetfillcolor{currentfill}%
\pgfsetlinewidth{0.000000pt}%
\definecolor{currentstroke}{rgb}{0.000000,0.000000,0.000000}%
\pgfsetstrokecolor{currentstroke}%
\pgfsetdash{}{0pt}%
\pgfpathmoveto{\pgfqpoint{2.932968in}{2.820668in}}%
\pgfpathlineto{\pgfqpoint{3.186131in}{2.954270in}}%
\pgfpathlineto{\pgfqpoint{2.932968in}{2.820668in}}%
\pgfpathclose%
\pgfusepath{fill}%
\end{pgfscope}%
\begin{pgfscope}%
\pgfpathrectangle{\pgfqpoint{0.017500in}{0.017500in}}{\pgfqpoint{6.218282in}{3.580372in}} %
\pgfusepath{clip}%
\pgfsetbuttcap%
\pgfsetroundjoin%
\definecolor{currentfill}{rgb}{0.723315,0.068898,0.162989}%
\pgfsetfillcolor{currentfill}%
\pgfsetlinewidth{0.000000pt}%
\definecolor{currentstroke}{rgb}{0.000000,0.000000,0.000000}%
\pgfsetstrokecolor{currentstroke}%
\pgfsetdash{}{0pt}%
\pgfpathmoveto{\pgfqpoint{2.932968in}{2.820668in}}%
\pgfpathlineto{\pgfqpoint{3.186131in}{2.954270in}}%
\pgfpathlineto{\pgfqpoint{2.932968in}{2.820668in}}%
\pgfpathclose%
\pgfusepath{fill}%
\end{pgfscope}%
\begin{pgfscope}%
\pgfpathrectangle{\pgfqpoint{0.017500in}{0.017500in}}{\pgfqpoint{6.218282in}{3.580372in}} %
\pgfusepath{clip}%
\pgfsetbuttcap%
\pgfsetroundjoin%
\definecolor{currentfill}{rgb}{0.804965,0.851666,0.926165}%
\pgfsetfillcolor{currentfill}%
\pgfsetlinewidth{0.000000pt}%
\definecolor{currentstroke}{rgb}{0.000000,0.000000,0.000000}%
\pgfsetstrokecolor{currentstroke}%
\pgfsetdash{}{0pt}%
\pgfpathmoveto{\pgfqpoint{3.186131in}{2.954270in}}%
\pgfpathlineto{\pgfqpoint{3.622205in}{2.891176in}}%
\pgfpathlineto{\pgfqpoint{2.349853in}{0.872712in}}%
\pgfpathlineto{\pgfqpoint{1.914830in}{0.967594in}}%
\pgfpathlineto{\pgfqpoint{3.186131in}{2.954270in}}%
\pgfpathclose%
\pgfusepath{fill}%
\end{pgfscope}%
\begin{pgfscope}%
\pgfpathrectangle{\pgfqpoint{0.017500in}{0.017500in}}{\pgfqpoint{6.218282in}{3.580372in}} %
\pgfusepath{clip}%
\pgfsetbuttcap%
\pgfsetroundjoin%
\definecolor{currentfill}{rgb}{0.804965,0.851666,0.926165}%
\pgfsetfillcolor{currentfill}%
\pgfsetlinewidth{0.000000pt}%
\definecolor{currentstroke}{rgb}{0.000000,0.000000,0.000000}%
\pgfsetstrokecolor{currentstroke}%
\pgfsetdash{}{0pt}%
\pgfpathmoveto{\pgfqpoint{3.186131in}{2.954270in}}%
\pgfpathlineto{\pgfqpoint{3.622205in}{2.891176in}}%
\pgfpathlineto{\pgfqpoint{2.349853in}{0.872712in}}%
\pgfpathlineto{\pgfqpoint{1.914830in}{0.967594in}}%
\pgfpathlineto{\pgfqpoint{3.186131in}{2.954270in}}%
\pgfpathclose%
\pgfusepath{fill}%
\end{pgfscope}%
\begin{pgfscope}%
\pgfpathrectangle{\pgfqpoint{0.017500in}{0.017500in}}{\pgfqpoint{6.218282in}{3.580372in}} %
\pgfusepath{clip}%
\pgfsetbuttcap%
\pgfsetroundjoin%
\definecolor{currentfill}{rgb}{0.804965,0.851666,0.926165}%
\pgfsetfillcolor{currentfill}%
\pgfsetlinewidth{0.000000pt}%
\definecolor{currentstroke}{rgb}{0.000000,0.000000,0.000000}%
\pgfsetstrokecolor{currentstroke}%
\pgfsetdash{}{0pt}%
\pgfpathmoveto{\pgfqpoint{3.186131in}{2.954270in}}%
\pgfpathlineto{\pgfqpoint{3.622205in}{2.891176in}}%
\pgfpathlineto{\pgfqpoint{2.349853in}{0.872712in}}%
\pgfpathlineto{\pgfqpoint{1.914830in}{0.967594in}}%
\pgfpathlineto{\pgfqpoint{3.186131in}{2.954270in}}%
\pgfpathclose%
\pgfusepath{fill}%
\end{pgfscope}%
\begin{pgfscope}%
\pgfpathrectangle{\pgfqpoint{0.017500in}{0.017500in}}{\pgfqpoint{6.218282in}{3.580372in}} %
\pgfusepath{clip}%
\pgfsetbuttcap%
\pgfsetroundjoin%
\definecolor{currentfill}{rgb}{0.804965,0.851666,0.926165}%
\pgfsetfillcolor{currentfill}%
\pgfsetlinewidth{0.000000pt}%
\definecolor{currentstroke}{rgb}{0.000000,0.000000,0.000000}%
\pgfsetstrokecolor{currentstroke}%
\pgfsetdash{}{0pt}%
\pgfpathmoveto{\pgfqpoint{3.186131in}{2.954270in}}%
\pgfpathlineto{\pgfqpoint{3.622205in}{2.891176in}}%
\pgfpathlineto{\pgfqpoint{2.349853in}{0.872712in}}%
\pgfpathlineto{\pgfqpoint{1.914830in}{0.967594in}}%
\pgfpathlineto{\pgfqpoint{3.186131in}{2.954270in}}%
\pgfpathclose%
\pgfusepath{fill}%
\end{pgfscope}%
\begin{pgfscope}%
\pgfpathrectangle{\pgfqpoint{0.017500in}{0.017500in}}{\pgfqpoint{6.218282in}{3.580372in}} %
\pgfusepath{clip}%
\pgfsetbuttcap%
\pgfsetroundjoin%
\definecolor{currentfill}{rgb}{0.804965,0.851666,0.926165}%
\pgfsetfillcolor{currentfill}%
\pgfsetlinewidth{0.000000pt}%
\definecolor{currentstroke}{rgb}{0.000000,0.000000,0.000000}%
\pgfsetstrokecolor{currentstroke}%
\pgfsetdash{}{0pt}%
\pgfpathmoveto{\pgfqpoint{3.186131in}{2.954270in}}%
\pgfpathlineto{\pgfqpoint{3.622205in}{2.891176in}}%
\pgfpathlineto{\pgfqpoint{2.349853in}{0.872712in}}%
\pgfpathlineto{\pgfqpoint{1.914830in}{0.967594in}}%
\pgfpathlineto{\pgfqpoint{3.186131in}{2.954270in}}%
\pgfpathclose%
\pgfusepath{fill}%
\end{pgfscope}%
\begin{pgfscope}%
\pgfpathrectangle{\pgfqpoint{0.017500in}{0.017500in}}{\pgfqpoint{6.218282in}{3.580372in}} %
\pgfusepath{clip}%
\pgfsetbuttcap%
\pgfsetroundjoin%
\definecolor{currentfill}{rgb}{0.717435,0.051118,0.158737}%
\pgfsetfillcolor{currentfill}%
\pgfsetlinewidth{0.000000pt}%
\definecolor{currentstroke}{rgb}{0.000000,0.000000,0.000000}%
\pgfsetstrokecolor{currentstroke}%
\pgfsetdash{}{0pt}%
\pgfpathmoveto{\pgfqpoint{2.932968in}{2.820668in}}%
\pgfpathlineto{\pgfqpoint{3.372093in}{2.761754in}}%
\pgfpathlineto{\pgfqpoint{3.622205in}{2.891176in}}%
\pgfpathlineto{\pgfqpoint{3.186131in}{2.954270in}}%
\pgfpathlineto{\pgfqpoint{2.932968in}{2.820668in}}%
\pgfpathclose%
\pgfusepath{fill}%
\end{pgfscope}%
\begin{pgfscope}%
\pgfpathrectangle{\pgfqpoint{0.017500in}{0.017500in}}{\pgfqpoint{6.218282in}{3.580372in}} %
\pgfusepath{clip}%
\pgfsetbuttcap%
\pgfsetroundjoin%
\definecolor{currentfill}{rgb}{0.717435,0.051118,0.158737}%
\pgfsetfillcolor{currentfill}%
\pgfsetlinewidth{0.000000pt}%
\definecolor{currentstroke}{rgb}{0.000000,0.000000,0.000000}%
\pgfsetstrokecolor{currentstroke}%
\pgfsetdash{}{0pt}%
\pgfpathmoveto{\pgfqpoint{2.932968in}{2.820668in}}%
\pgfpathlineto{\pgfqpoint{3.372093in}{2.761754in}}%
\pgfpathlineto{\pgfqpoint{3.622205in}{2.891176in}}%
\pgfpathlineto{\pgfqpoint{3.186131in}{2.954270in}}%
\pgfpathlineto{\pgfqpoint{2.932968in}{2.820668in}}%
\pgfpathclose%
\pgfusepath{fill}%
\end{pgfscope}%
\begin{pgfscope}%
\pgfpathrectangle{\pgfqpoint{0.017500in}{0.017500in}}{\pgfqpoint{6.218282in}{3.580372in}} %
\pgfusepath{clip}%
\pgfsetbuttcap%
\pgfsetroundjoin%
\definecolor{currentfill}{rgb}{0.717435,0.051118,0.158737}%
\pgfsetfillcolor{currentfill}%
\pgfsetlinewidth{0.000000pt}%
\definecolor{currentstroke}{rgb}{0.000000,0.000000,0.000000}%
\pgfsetstrokecolor{currentstroke}%
\pgfsetdash{}{0pt}%
\pgfpathmoveto{\pgfqpoint{2.932968in}{2.820668in}}%
\pgfpathlineto{\pgfqpoint{3.372093in}{2.761754in}}%
\pgfpathlineto{\pgfqpoint{3.622205in}{2.891176in}}%
\pgfpathlineto{\pgfqpoint{3.186131in}{2.954270in}}%
\pgfpathlineto{\pgfqpoint{2.932968in}{2.820668in}}%
\pgfpathclose%
\pgfusepath{fill}%
\end{pgfscope}%
\begin{pgfscope}%
\pgfpathrectangle{\pgfqpoint{0.017500in}{0.017500in}}{\pgfqpoint{6.218282in}{3.580372in}} %
\pgfusepath{clip}%
\pgfsetbuttcap%
\pgfsetroundjoin%
\definecolor{currentfill}{rgb}{0.717435,0.051118,0.158737}%
\pgfsetfillcolor{currentfill}%
\pgfsetlinewidth{0.000000pt}%
\definecolor{currentstroke}{rgb}{0.000000,0.000000,0.000000}%
\pgfsetstrokecolor{currentstroke}%
\pgfsetdash{}{0pt}%
\pgfpathmoveto{\pgfqpoint{2.932968in}{2.820668in}}%
\pgfpathlineto{\pgfqpoint{3.372093in}{2.761754in}}%
\pgfpathlineto{\pgfqpoint{3.622205in}{2.891176in}}%
\pgfpathlineto{\pgfqpoint{3.186131in}{2.954270in}}%
\pgfpathlineto{\pgfqpoint{2.932968in}{2.820668in}}%
\pgfpathclose%
\pgfusepath{fill}%
\end{pgfscope}%
\begin{pgfscope}%
\pgfpathrectangle{\pgfqpoint{0.017500in}{0.017500in}}{\pgfqpoint{6.218282in}{3.580372in}} %
\pgfusepath{clip}%
\pgfsetbuttcap%
\pgfsetroundjoin%
\definecolor{currentfill}{rgb}{0.717435,0.051118,0.158737}%
\pgfsetfillcolor{currentfill}%
\pgfsetlinewidth{0.000000pt}%
\definecolor{currentstroke}{rgb}{0.000000,0.000000,0.000000}%
\pgfsetstrokecolor{currentstroke}%
\pgfsetdash{}{0pt}%
\pgfpathmoveto{\pgfqpoint{2.932968in}{2.820668in}}%
\pgfpathlineto{\pgfqpoint{3.372093in}{2.761754in}}%
\pgfpathlineto{\pgfqpoint{3.622205in}{2.891176in}}%
\pgfpathlineto{\pgfqpoint{3.186131in}{2.954270in}}%
\pgfpathlineto{\pgfqpoint{2.932968in}{2.820668in}}%
\pgfpathclose%
\pgfusepath{fill}%
\end{pgfscope}%
\begin{pgfscope}%
\pgfpathrectangle{\pgfqpoint{0.017500in}{0.017500in}}{\pgfqpoint{6.218282in}{3.580372in}} %
\pgfusepath{clip}%
\pgfsetbuttcap%
\pgfsetroundjoin%
\definecolor{currentfill}{rgb}{0.717435,0.051118,0.158737}%
\pgfsetfillcolor{currentfill}%
\pgfsetlinewidth{0.000000pt}%
\definecolor{currentstroke}{rgb}{0.000000,0.000000,0.000000}%
\pgfsetstrokecolor{currentstroke}%
\pgfsetdash{}{0pt}%
\pgfpathmoveto{\pgfqpoint{2.932968in}{2.820668in}}%
\pgfpathlineto{\pgfqpoint{3.372093in}{2.761754in}}%
\pgfpathlineto{\pgfqpoint{3.622205in}{2.891176in}}%
\pgfpathlineto{\pgfqpoint{3.186131in}{2.954270in}}%
\pgfpathlineto{\pgfqpoint{2.932968in}{2.820668in}}%
\pgfpathclose%
\pgfusepath{fill}%
\end{pgfscope}%
\begin{pgfscope}%
\pgfpathrectangle{\pgfqpoint{0.017500in}{0.017500in}}{\pgfqpoint{6.218282in}{3.580372in}} %
\pgfusepath{clip}%
\pgfsetbuttcap%
\pgfsetroundjoin%
\definecolor{currentfill}{rgb}{0.740957,0.122240,0.175744}%
\pgfsetfillcolor{currentfill}%
\pgfsetlinewidth{0.000000pt}%
\definecolor{currentstroke}{rgb}{0.000000,0.000000,0.000000}%
\pgfsetstrokecolor{currentstroke}%
\pgfsetdash{}{0pt}%
\pgfpathmoveto{\pgfqpoint{2.237924in}{2.752812in}}%
\pgfpathlineto{\pgfqpoint{2.499217in}{2.903684in}}%
\pgfpathlineto{\pgfqpoint{2.237924in}{2.752812in}}%
\pgfpathclose%
\pgfusepath{fill}%
\end{pgfscope}%
\begin{pgfscope}%
\pgfpathrectangle{\pgfqpoint{0.017500in}{0.017500in}}{\pgfqpoint{6.218282in}{3.580372in}} %
\pgfusepath{clip}%
\pgfsetbuttcap%
\pgfsetroundjoin%
\definecolor{currentfill}{rgb}{0.740957,0.122240,0.175744}%
\pgfsetfillcolor{currentfill}%
\pgfsetlinewidth{0.000000pt}%
\definecolor{currentstroke}{rgb}{0.000000,0.000000,0.000000}%
\pgfsetstrokecolor{currentstroke}%
\pgfsetdash{}{0pt}%
\pgfpathmoveto{\pgfqpoint{2.237924in}{2.752812in}}%
\pgfpathlineto{\pgfqpoint{2.499217in}{2.903684in}}%
\pgfpathlineto{\pgfqpoint{2.237924in}{2.752812in}}%
\pgfpathclose%
\pgfusepath{fill}%
\end{pgfscope}%
\begin{pgfscope}%
\pgfpathrectangle{\pgfqpoint{0.017500in}{0.017500in}}{\pgfqpoint{6.218282in}{3.580372in}} %
\pgfusepath{clip}%
\pgfsetbuttcap%
\pgfsetroundjoin%
\definecolor{currentfill}{rgb}{0.740957,0.122240,0.175744}%
\pgfsetfillcolor{currentfill}%
\pgfsetlinewidth{0.000000pt}%
\definecolor{currentstroke}{rgb}{0.000000,0.000000,0.000000}%
\pgfsetstrokecolor{currentstroke}%
\pgfsetdash{}{0pt}%
\pgfpathmoveto{\pgfqpoint{2.237924in}{2.752812in}}%
\pgfpathlineto{\pgfqpoint{2.499217in}{2.903684in}}%
\pgfpathlineto{\pgfqpoint{2.237924in}{2.752812in}}%
\pgfpathclose%
\pgfusepath{fill}%
\end{pgfscope}%
\begin{pgfscope}%
\pgfpathrectangle{\pgfqpoint{0.017500in}{0.017500in}}{\pgfqpoint{6.218282in}{3.580372in}} %
\pgfusepath{clip}%
\pgfsetbuttcap%
\pgfsetroundjoin%
\definecolor{currentfill}{rgb}{0.740957,0.122240,0.175744}%
\pgfsetfillcolor{currentfill}%
\pgfsetlinewidth{0.000000pt}%
\definecolor{currentstroke}{rgb}{0.000000,0.000000,0.000000}%
\pgfsetstrokecolor{currentstroke}%
\pgfsetdash{}{0pt}%
\pgfpathmoveto{\pgfqpoint{2.237924in}{2.752812in}}%
\pgfpathlineto{\pgfqpoint{2.499217in}{2.903684in}}%
\pgfpathlineto{\pgfqpoint{2.237924in}{2.752812in}}%
\pgfpathclose%
\pgfusepath{fill}%
\end{pgfscope}%
\begin{pgfscope}%
\pgfpathrectangle{\pgfqpoint{0.017500in}{0.017500in}}{\pgfqpoint{6.218282in}{3.580372in}} %
\pgfusepath{clip}%
\pgfsetbuttcap%
\pgfsetroundjoin%
\definecolor{currentfill}{rgb}{0.740957,0.122240,0.175744}%
\pgfsetfillcolor{currentfill}%
\pgfsetlinewidth{0.000000pt}%
\definecolor{currentstroke}{rgb}{0.000000,0.000000,0.000000}%
\pgfsetstrokecolor{currentstroke}%
\pgfsetdash{}{0pt}%
\pgfpathmoveto{\pgfqpoint{2.237924in}{2.752812in}}%
\pgfpathlineto{\pgfqpoint{2.499217in}{2.903684in}}%
\pgfpathlineto{\pgfqpoint{2.237924in}{2.752812in}}%
\pgfpathclose%
\pgfusepath{fill}%
\end{pgfscope}%
\begin{pgfscope}%
\pgfpathrectangle{\pgfqpoint{0.017500in}{0.017500in}}{\pgfqpoint{6.218282in}{3.580372in}} %
\pgfusepath{clip}%
\pgfsetbuttcap%
\pgfsetroundjoin%
\definecolor{currentfill}{rgb}{0.740957,0.122240,0.175744}%
\pgfsetfillcolor{currentfill}%
\pgfsetlinewidth{0.000000pt}%
\definecolor{currentstroke}{rgb}{0.000000,0.000000,0.000000}%
\pgfsetstrokecolor{currentstroke}%
\pgfsetdash{}{0pt}%
\pgfpathmoveto{\pgfqpoint{2.237924in}{2.752812in}}%
\pgfpathlineto{\pgfqpoint{2.499217in}{2.903684in}}%
\pgfpathlineto{\pgfqpoint{2.237924in}{2.752812in}}%
\pgfpathclose%
\pgfusepath{fill}%
\end{pgfscope}%
\begin{pgfscope}%
\pgfpathrectangle{\pgfqpoint{0.017500in}{0.017500in}}{\pgfqpoint{6.218282in}{3.580372in}} %
\pgfusepath{clip}%
\pgfsetbuttcap%
\pgfsetroundjoin%
\definecolor{currentfill}{rgb}{0.740957,0.122240,0.175744}%
\pgfsetfillcolor{currentfill}%
\pgfsetlinewidth{0.000000pt}%
\definecolor{currentstroke}{rgb}{0.000000,0.000000,0.000000}%
\pgfsetstrokecolor{currentstroke}%
\pgfsetdash{}{0pt}%
\pgfpathmoveto{\pgfqpoint{2.237924in}{2.752812in}}%
\pgfpathlineto{\pgfqpoint{2.499217in}{2.903684in}}%
\pgfpathlineto{\pgfqpoint{2.237924in}{2.752812in}}%
\pgfpathclose%
\pgfusepath{fill}%
\end{pgfscope}%
\begin{pgfscope}%
\pgfpathrectangle{\pgfqpoint{0.017500in}{0.017500in}}{\pgfqpoint{6.218282in}{3.580372in}} %
\pgfusepath{clip}%
\pgfsetbuttcap%
\pgfsetroundjoin%
\definecolor{currentfill}{rgb}{0.740957,0.122240,0.175744}%
\pgfsetfillcolor{currentfill}%
\pgfsetlinewidth{0.000000pt}%
\definecolor{currentstroke}{rgb}{0.000000,0.000000,0.000000}%
\pgfsetstrokecolor{currentstroke}%
\pgfsetdash{}{0pt}%
\pgfpathmoveto{\pgfqpoint{2.237924in}{2.752812in}}%
\pgfpathlineto{\pgfqpoint{2.499217in}{2.903684in}}%
\pgfpathlineto{\pgfqpoint{2.237924in}{2.752812in}}%
\pgfpathclose%
\pgfusepath{fill}%
\end{pgfscope}%
\begin{pgfscope}%
\pgfpathrectangle{\pgfqpoint{0.017500in}{0.017500in}}{\pgfqpoint{6.218282in}{3.580372in}} %
\pgfusepath{clip}%
\pgfsetbuttcap%
\pgfsetroundjoin%
\definecolor{currentfill}{rgb}{0.740957,0.122240,0.175744}%
\pgfsetfillcolor{currentfill}%
\pgfsetlinewidth{0.000000pt}%
\definecolor{currentstroke}{rgb}{0.000000,0.000000,0.000000}%
\pgfsetstrokecolor{currentstroke}%
\pgfsetdash{}{0pt}%
\pgfpathmoveto{\pgfqpoint{2.237924in}{2.752812in}}%
\pgfpathlineto{\pgfqpoint{2.499217in}{2.903684in}}%
\pgfpathlineto{\pgfqpoint{2.237924in}{2.752812in}}%
\pgfpathclose%
\pgfusepath{fill}%
\end{pgfscope}%
\begin{pgfscope}%
\pgfpathrectangle{\pgfqpoint{0.017500in}{0.017500in}}{\pgfqpoint{6.218282in}{3.580372in}} %
\pgfusepath{clip}%
\pgfsetbuttcap%
\pgfsetroundjoin%
\definecolor{currentfill}{rgb}{0.740957,0.122240,0.175744}%
\pgfsetfillcolor{currentfill}%
\pgfsetlinewidth{0.000000pt}%
\definecolor{currentstroke}{rgb}{0.000000,0.000000,0.000000}%
\pgfsetstrokecolor{currentstroke}%
\pgfsetdash{}{0pt}%
\pgfpathmoveto{\pgfqpoint{2.237924in}{2.752812in}}%
\pgfpathlineto{\pgfqpoint{2.499217in}{2.903684in}}%
\pgfpathlineto{\pgfqpoint{2.237924in}{2.752812in}}%
\pgfpathclose%
\pgfusepath{fill}%
\end{pgfscope}%
\begin{pgfscope}%
\pgfpathrectangle{\pgfqpoint{0.017500in}{0.017500in}}{\pgfqpoint{6.218282in}{3.580372in}} %
\pgfusepath{clip}%
\pgfsetbuttcap%
\pgfsetroundjoin%
\definecolor{currentfill}{rgb}{0.740957,0.122240,0.175744}%
\pgfsetfillcolor{currentfill}%
\pgfsetlinewidth{0.000000pt}%
\definecolor{currentstroke}{rgb}{0.000000,0.000000,0.000000}%
\pgfsetstrokecolor{currentstroke}%
\pgfsetdash{}{0pt}%
\pgfpathmoveto{\pgfqpoint{2.237924in}{2.752812in}}%
\pgfpathlineto{\pgfqpoint{2.499217in}{2.903684in}}%
\pgfpathlineto{\pgfqpoint{2.237924in}{2.752812in}}%
\pgfpathclose%
\pgfusepath{fill}%
\end{pgfscope}%
\begin{pgfscope}%
\pgfpathrectangle{\pgfqpoint{0.017500in}{0.017500in}}{\pgfqpoint{6.218282in}{3.580372in}} %
\pgfusepath{clip}%
\pgfsetbuttcap%
\pgfsetroundjoin%
\definecolor{currentfill}{rgb}{0.740957,0.122240,0.175744}%
\pgfsetfillcolor{currentfill}%
\pgfsetlinewidth{0.000000pt}%
\definecolor{currentstroke}{rgb}{0.000000,0.000000,0.000000}%
\pgfsetstrokecolor{currentstroke}%
\pgfsetdash{}{0pt}%
\pgfpathmoveto{\pgfqpoint{2.237924in}{2.752812in}}%
\pgfpathlineto{\pgfqpoint{2.499217in}{2.903684in}}%
\pgfpathlineto{\pgfqpoint{2.237924in}{2.752812in}}%
\pgfpathclose%
\pgfusepath{fill}%
\end{pgfscope}%
\begin{pgfscope}%
\pgfpathrectangle{\pgfqpoint{0.017500in}{0.017500in}}{\pgfqpoint{6.218282in}{3.580372in}} %
\pgfusepath{clip}%
\pgfsetbuttcap%
\pgfsetroundjoin%
\definecolor{currentfill}{rgb}{0.740957,0.122240,0.175744}%
\pgfsetfillcolor{currentfill}%
\pgfsetlinewidth{0.000000pt}%
\definecolor{currentstroke}{rgb}{0.000000,0.000000,0.000000}%
\pgfsetstrokecolor{currentstroke}%
\pgfsetdash{}{0pt}%
\pgfpathmoveto{\pgfqpoint{2.237924in}{2.752812in}}%
\pgfpathlineto{\pgfqpoint{2.499217in}{2.903684in}}%
\pgfpathlineto{\pgfqpoint{2.237924in}{2.752812in}}%
\pgfpathclose%
\pgfusepath{fill}%
\end{pgfscope}%
\begin{pgfscope}%
\pgfpathrectangle{\pgfqpoint{0.017500in}{0.017500in}}{\pgfqpoint{6.218282in}{3.580372in}} %
\pgfusepath{clip}%
\pgfsetbuttcap%
\pgfsetroundjoin%
\definecolor{currentfill}{rgb}{0.740957,0.122240,0.175744}%
\pgfsetfillcolor{currentfill}%
\pgfsetlinewidth{0.000000pt}%
\definecolor{currentstroke}{rgb}{0.000000,0.000000,0.000000}%
\pgfsetstrokecolor{currentstroke}%
\pgfsetdash{}{0pt}%
\pgfpathmoveto{\pgfqpoint{2.237924in}{2.752812in}}%
\pgfpathlineto{\pgfqpoint{2.499217in}{2.903684in}}%
\pgfpathlineto{\pgfqpoint{2.237924in}{2.752812in}}%
\pgfpathclose%
\pgfusepath{fill}%
\end{pgfscope}%
\begin{pgfscope}%
\pgfpathrectangle{\pgfqpoint{0.017500in}{0.017500in}}{\pgfqpoint{6.218282in}{3.580372in}} %
\pgfusepath{clip}%
\pgfsetbuttcap%
\pgfsetroundjoin%
\definecolor{currentfill}{rgb}{0.740957,0.122240,0.175744}%
\pgfsetfillcolor{currentfill}%
\pgfsetlinewidth{0.000000pt}%
\definecolor{currentstroke}{rgb}{0.000000,0.000000,0.000000}%
\pgfsetstrokecolor{currentstroke}%
\pgfsetdash{}{0pt}%
\pgfpathmoveto{\pgfqpoint{2.237924in}{2.752812in}}%
\pgfpathlineto{\pgfqpoint{2.499217in}{2.903684in}}%
\pgfpathlineto{\pgfqpoint{2.237924in}{2.752812in}}%
\pgfpathclose%
\pgfusepath{fill}%
\end{pgfscope}%
\begin{pgfscope}%
\pgfpathrectangle{\pgfqpoint{0.017500in}{0.017500in}}{\pgfqpoint{6.218282in}{3.580372in}} %
\pgfusepath{clip}%
\pgfsetbuttcap%
\pgfsetroundjoin%
\definecolor{currentfill}{rgb}{0.740957,0.122240,0.175744}%
\pgfsetfillcolor{currentfill}%
\pgfsetlinewidth{0.000000pt}%
\definecolor{currentstroke}{rgb}{0.000000,0.000000,0.000000}%
\pgfsetstrokecolor{currentstroke}%
\pgfsetdash{}{0pt}%
\pgfpathmoveto{\pgfqpoint{2.237924in}{2.752812in}}%
\pgfpathlineto{\pgfqpoint{2.499217in}{2.903684in}}%
\pgfpathlineto{\pgfqpoint{2.237924in}{2.752812in}}%
\pgfpathclose%
\pgfusepath{fill}%
\end{pgfscope}%
\begin{pgfscope}%
\pgfpathrectangle{\pgfqpoint{0.017500in}{0.017500in}}{\pgfqpoint{6.218282in}{3.580372in}} %
\pgfusepath{clip}%
\pgfsetbuttcap%
\pgfsetroundjoin%
\definecolor{currentfill}{rgb}{0.740957,0.122240,0.175744}%
\pgfsetfillcolor{currentfill}%
\pgfsetlinewidth{0.000000pt}%
\definecolor{currentstroke}{rgb}{0.000000,0.000000,0.000000}%
\pgfsetstrokecolor{currentstroke}%
\pgfsetdash{}{0pt}%
\pgfpathmoveto{\pgfqpoint{2.237924in}{2.752812in}}%
\pgfpathlineto{\pgfqpoint{2.499217in}{2.903684in}}%
\pgfpathlineto{\pgfqpoint{2.237924in}{2.752812in}}%
\pgfpathclose%
\pgfusepath{fill}%
\end{pgfscope}%
\begin{pgfscope}%
\pgfpathrectangle{\pgfqpoint{0.017500in}{0.017500in}}{\pgfqpoint{6.218282in}{3.580372in}} %
\pgfusepath{clip}%
\pgfsetbuttcap%
\pgfsetroundjoin%
\definecolor{currentfill}{rgb}{0.740957,0.122240,0.175744}%
\pgfsetfillcolor{currentfill}%
\pgfsetlinewidth{0.000000pt}%
\definecolor{currentstroke}{rgb}{0.000000,0.000000,0.000000}%
\pgfsetstrokecolor{currentstroke}%
\pgfsetdash{}{0pt}%
\pgfpathmoveto{\pgfqpoint{2.237924in}{2.752812in}}%
\pgfpathlineto{\pgfqpoint{2.499217in}{2.903684in}}%
\pgfpathlineto{\pgfqpoint{2.237924in}{2.752812in}}%
\pgfpathclose%
\pgfusepath{fill}%
\end{pgfscope}%
\begin{pgfscope}%
\pgfpathrectangle{\pgfqpoint{0.017500in}{0.017500in}}{\pgfqpoint{6.218282in}{3.580372in}} %
\pgfusepath{clip}%
\pgfsetbuttcap%
\pgfsetroundjoin%
\definecolor{currentfill}{rgb}{0.740957,0.122240,0.175744}%
\pgfsetfillcolor{currentfill}%
\pgfsetlinewidth{0.000000pt}%
\definecolor{currentstroke}{rgb}{0.000000,0.000000,0.000000}%
\pgfsetstrokecolor{currentstroke}%
\pgfsetdash{}{0pt}%
\pgfpathmoveto{\pgfqpoint{2.237924in}{2.752812in}}%
\pgfpathlineto{\pgfqpoint{2.499217in}{2.903684in}}%
\pgfpathlineto{\pgfqpoint{2.237924in}{2.752812in}}%
\pgfpathclose%
\pgfusepath{fill}%
\end{pgfscope}%
\begin{pgfscope}%
\pgfpathrectangle{\pgfqpoint{0.017500in}{0.017500in}}{\pgfqpoint{6.218282in}{3.580372in}} %
\pgfusepath{clip}%
\pgfsetbuttcap%
\pgfsetroundjoin%
\definecolor{currentfill}{rgb}{0.740957,0.122240,0.175744}%
\pgfsetfillcolor{currentfill}%
\pgfsetlinewidth{0.000000pt}%
\definecolor{currentstroke}{rgb}{0.000000,0.000000,0.000000}%
\pgfsetstrokecolor{currentstroke}%
\pgfsetdash{}{0pt}%
\pgfpathmoveto{\pgfqpoint{2.237924in}{2.752812in}}%
\pgfpathlineto{\pgfqpoint{2.499217in}{2.903684in}}%
\pgfpathlineto{\pgfqpoint{2.237924in}{2.752812in}}%
\pgfpathclose%
\pgfusepath{fill}%
\end{pgfscope}%
\begin{pgfscope}%
\pgfpathrectangle{\pgfqpoint{0.017500in}{0.017500in}}{\pgfqpoint{6.218282in}{3.580372in}} %
\pgfusepath{clip}%
\pgfsetbuttcap%
\pgfsetroundjoin%
\definecolor{currentfill}{rgb}{0.740957,0.122240,0.175744}%
\pgfsetfillcolor{currentfill}%
\pgfsetlinewidth{0.000000pt}%
\definecolor{currentstroke}{rgb}{0.000000,0.000000,0.000000}%
\pgfsetstrokecolor{currentstroke}%
\pgfsetdash{}{0pt}%
\pgfpathmoveto{\pgfqpoint{2.237924in}{2.752812in}}%
\pgfpathlineto{\pgfqpoint{2.499217in}{2.903684in}}%
\pgfpathlineto{\pgfqpoint{2.237924in}{2.752812in}}%
\pgfpathclose%
\pgfusepath{fill}%
\end{pgfscope}%
\begin{pgfscope}%
\pgfpathrectangle{\pgfqpoint{0.017500in}{0.017500in}}{\pgfqpoint{6.218282in}{3.580372in}} %
\pgfusepath{clip}%
\pgfsetbuttcap%
\pgfsetroundjoin%
\definecolor{currentfill}{rgb}{0.740957,0.122240,0.175744}%
\pgfsetfillcolor{currentfill}%
\pgfsetlinewidth{0.000000pt}%
\definecolor{currentstroke}{rgb}{0.000000,0.000000,0.000000}%
\pgfsetstrokecolor{currentstroke}%
\pgfsetdash{}{0pt}%
\pgfpathmoveto{\pgfqpoint{2.237924in}{2.752812in}}%
\pgfpathlineto{\pgfqpoint{2.499217in}{2.903684in}}%
\pgfpathlineto{\pgfqpoint{2.237924in}{2.752812in}}%
\pgfpathclose%
\pgfusepath{fill}%
\end{pgfscope}%
\begin{pgfscope}%
\pgfpathrectangle{\pgfqpoint{0.017500in}{0.017500in}}{\pgfqpoint{6.218282in}{3.580372in}} %
\pgfusepath{clip}%
\pgfsetbuttcap%
\pgfsetroundjoin%
\definecolor{currentfill}{rgb}{0.740957,0.122240,0.175744}%
\pgfsetfillcolor{currentfill}%
\pgfsetlinewidth{0.000000pt}%
\definecolor{currentstroke}{rgb}{0.000000,0.000000,0.000000}%
\pgfsetstrokecolor{currentstroke}%
\pgfsetdash{}{0pt}%
\pgfpathmoveto{\pgfqpoint{2.237924in}{2.752812in}}%
\pgfpathlineto{\pgfqpoint{2.499217in}{2.903684in}}%
\pgfpathlineto{\pgfqpoint{2.237924in}{2.752812in}}%
\pgfpathclose%
\pgfusepath{fill}%
\end{pgfscope}%
\begin{pgfscope}%
\pgfpathrectangle{\pgfqpoint{0.017500in}{0.017500in}}{\pgfqpoint{6.218282in}{3.580372in}} %
\pgfusepath{clip}%
\pgfsetbuttcap%
\pgfsetroundjoin%
\definecolor{currentfill}{rgb}{0.740957,0.122240,0.175744}%
\pgfsetfillcolor{currentfill}%
\pgfsetlinewidth{0.000000pt}%
\definecolor{currentstroke}{rgb}{0.000000,0.000000,0.000000}%
\pgfsetstrokecolor{currentstroke}%
\pgfsetdash{}{0pt}%
\pgfpathmoveto{\pgfqpoint{2.237924in}{2.752812in}}%
\pgfpathlineto{\pgfqpoint{2.499217in}{2.903684in}}%
\pgfpathlineto{\pgfqpoint{2.237924in}{2.752812in}}%
\pgfpathclose%
\pgfusepath{fill}%
\end{pgfscope}%
\begin{pgfscope}%
\pgfpathrectangle{\pgfqpoint{0.017500in}{0.017500in}}{\pgfqpoint{6.218282in}{3.580372in}} %
\pgfusepath{clip}%
\pgfsetbuttcap%
\pgfsetroundjoin%
\definecolor{currentfill}{rgb}{0.740957,0.122240,0.175744}%
\pgfsetfillcolor{currentfill}%
\pgfsetlinewidth{0.000000pt}%
\definecolor{currentstroke}{rgb}{0.000000,0.000000,0.000000}%
\pgfsetstrokecolor{currentstroke}%
\pgfsetdash{}{0pt}%
\pgfpathmoveto{\pgfqpoint{2.237924in}{2.752812in}}%
\pgfpathlineto{\pgfqpoint{2.499217in}{2.903684in}}%
\pgfpathlineto{\pgfqpoint{2.237924in}{2.752812in}}%
\pgfpathclose%
\pgfusepath{fill}%
\end{pgfscope}%
\begin{pgfscope}%
\pgfpathrectangle{\pgfqpoint{0.017500in}{0.017500in}}{\pgfqpoint{6.218282in}{3.580372in}} %
\pgfusepath{clip}%
\pgfsetbuttcap%
\pgfsetroundjoin%
\definecolor{currentfill}{rgb}{0.740957,0.122240,0.175744}%
\pgfsetfillcolor{currentfill}%
\pgfsetlinewidth{0.000000pt}%
\definecolor{currentstroke}{rgb}{0.000000,0.000000,0.000000}%
\pgfsetstrokecolor{currentstroke}%
\pgfsetdash{}{0pt}%
\pgfpathmoveto{\pgfqpoint{2.237924in}{2.752812in}}%
\pgfpathlineto{\pgfqpoint{2.499217in}{2.903684in}}%
\pgfpathlineto{\pgfqpoint{2.237924in}{2.752812in}}%
\pgfpathclose%
\pgfusepath{fill}%
\end{pgfscope}%
\begin{pgfscope}%
\pgfpathrectangle{\pgfqpoint{0.017500in}{0.017500in}}{\pgfqpoint{6.218282in}{3.580372in}} %
\pgfusepath{clip}%
\pgfsetbuttcap%
\pgfsetroundjoin%
\definecolor{currentfill}{rgb}{0.740957,0.122240,0.175744}%
\pgfsetfillcolor{currentfill}%
\pgfsetlinewidth{0.000000pt}%
\definecolor{currentstroke}{rgb}{0.000000,0.000000,0.000000}%
\pgfsetstrokecolor{currentstroke}%
\pgfsetdash{}{0pt}%
\pgfpathmoveto{\pgfqpoint{2.237924in}{2.752812in}}%
\pgfpathlineto{\pgfqpoint{2.499217in}{2.903684in}}%
\pgfpathlineto{\pgfqpoint{2.237924in}{2.752812in}}%
\pgfpathclose%
\pgfusepath{fill}%
\end{pgfscope}%
\begin{pgfscope}%
\pgfpathrectangle{\pgfqpoint{0.017500in}{0.017500in}}{\pgfqpoint{6.218282in}{3.580372in}} %
\pgfusepath{clip}%
\pgfsetbuttcap%
\pgfsetroundjoin%
\definecolor{currentfill}{rgb}{0.740957,0.122240,0.175744}%
\pgfsetfillcolor{currentfill}%
\pgfsetlinewidth{0.000000pt}%
\definecolor{currentstroke}{rgb}{0.000000,0.000000,0.000000}%
\pgfsetstrokecolor{currentstroke}%
\pgfsetdash{}{0pt}%
\pgfpathmoveto{\pgfqpoint{2.237924in}{2.752812in}}%
\pgfpathlineto{\pgfqpoint{2.499217in}{2.903684in}}%
\pgfpathlineto{\pgfqpoint{2.237924in}{2.752812in}}%
\pgfpathclose%
\pgfusepath{fill}%
\end{pgfscope}%
\begin{pgfscope}%
\pgfpathrectangle{\pgfqpoint{0.017500in}{0.017500in}}{\pgfqpoint{6.218282in}{3.580372in}} %
\pgfusepath{clip}%
\pgfsetbuttcap%
\pgfsetroundjoin%
\definecolor{currentfill}{rgb}{0.740957,0.122240,0.175744}%
\pgfsetfillcolor{currentfill}%
\pgfsetlinewidth{0.000000pt}%
\definecolor{currentstroke}{rgb}{0.000000,0.000000,0.000000}%
\pgfsetstrokecolor{currentstroke}%
\pgfsetdash{}{0pt}%
\pgfpathmoveto{\pgfqpoint{2.237924in}{2.752812in}}%
\pgfpathlineto{\pgfqpoint{2.499217in}{2.903684in}}%
\pgfpathlineto{\pgfqpoint{2.237924in}{2.752812in}}%
\pgfpathclose%
\pgfusepath{fill}%
\end{pgfscope}%
\begin{pgfscope}%
\pgfpathrectangle{\pgfqpoint{0.017500in}{0.017500in}}{\pgfqpoint{6.218282in}{3.580372in}} %
\pgfusepath{clip}%
\pgfsetbuttcap%
\pgfsetroundjoin%
\definecolor{currentfill}{rgb}{0.740957,0.122240,0.175744}%
\pgfsetfillcolor{currentfill}%
\pgfsetlinewidth{0.000000pt}%
\definecolor{currentstroke}{rgb}{0.000000,0.000000,0.000000}%
\pgfsetstrokecolor{currentstroke}%
\pgfsetdash{}{0pt}%
\pgfpathmoveto{\pgfqpoint{2.237924in}{2.752812in}}%
\pgfpathlineto{\pgfqpoint{2.499217in}{2.903684in}}%
\pgfpathlineto{\pgfqpoint{2.237924in}{2.752812in}}%
\pgfpathclose%
\pgfusepath{fill}%
\end{pgfscope}%
\begin{pgfscope}%
\pgfpathrectangle{\pgfqpoint{0.017500in}{0.017500in}}{\pgfqpoint{6.218282in}{3.580372in}} %
\pgfusepath{clip}%
\pgfsetbuttcap%
\pgfsetroundjoin%
\definecolor{currentfill}{rgb}{0.899543,0.847500,0.817789}%
\pgfsetfillcolor{currentfill}%
\pgfsetlinewidth{0.000000pt}%
\definecolor{currentstroke}{rgb}{0.000000,0.000000,0.000000}%
\pgfsetstrokecolor{currentstroke}%
\pgfsetdash{}{0pt}%
\pgfpathmoveto{\pgfqpoint{3.622205in}{2.891176in}}%
\pgfpathlineto{\pgfqpoint{2.349853in}{0.872712in}}%
\pgfpathlineto{\pgfqpoint{3.622205in}{2.891176in}}%
\pgfpathclose%
\pgfusepath{fill}%
\end{pgfscope}%
\begin{pgfscope}%
\pgfpathrectangle{\pgfqpoint{0.017500in}{0.017500in}}{\pgfqpoint{6.218282in}{3.580372in}} %
\pgfusepath{clip}%
\pgfsetbuttcap%
\pgfsetroundjoin%
\definecolor{currentfill}{rgb}{0.899543,0.847500,0.817789}%
\pgfsetfillcolor{currentfill}%
\pgfsetlinewidth{0.000000pt}%
\definecolor{currentstroke}{rgb}{0.000000,0.000000,0.000000}%
\pgfsetstrokecolor{currentstroke}%
\pgfsetdash{}{0pt}%
\pgfpathmoveto{\pgfqpoint{3.622205in}{2.891176in}}%
\pgfpathlineto{\pgfqpoint{2.349853in}{0.872712in}}%
\pgfpathlineto{\pgfqpoint{3.622205in}{2.891176in}}%
\pgfpathclose%
\pgfusepath{fill}%
\end{pgfscope}%
\begin{pgfscope}%
\pgfpathrectangle{\pgfqpoint{0.017500in}{0.017500in}}{\pgfqpoint{6.218282in}{3.580372in}} %
\pgfusepath{clip}%
\pgfsetbuttcap%
\pgfsetroundjoin%
\definecolor{currentfill}{rgb}{0.899543,0.847500,0.817789}%
\pgfsetfillcolor{currentfill}%
\pgfsetlinewidth{0.000000pt}%
\definecolor{currentstroke}{rgb}{0.000000,0.000000,0.000000}%
\pgfsetstrokecolor{currentstroke}%
\pgfsetdash{}{0pt}%
\pgfpathmoveto{\pgfqpoint{3.622205in}{2.891176in}}%
\pgfpathlineto{\pgfqpoint{2.349853in}{0.872712in}}%
\pgfpathlineto{\pgfqpoint{3.622205in}{2.891176in}}%
\pgfpathclose%
\pgfusepath{fill}%
\end{pgfscope}%
\begin{pgfscope}%
\pgfpathrectangle{\pgfqpoint{0.017500in}{0.017500in}}{\pgfqpoint{6.218282in}{3.580372in}} %
\pgfusepath{clip}%
\pgfsetbuttcap%
\pgfsetroundjoin%
\definecolor{currentfill}{rgb}{0.899543,0.847500,0.817789}%
\pgfsetfillcolor{currentfill}%
\pgfsetlinewidth{0.000000pt}%
\definecolor{currentstroke}{rgb}{0.000000,0.000000,0.000000}%
\pgfsetstrokecolor{currentstroke}%
\pgfsetdash{}{0pt}%
\pgfpathmoveto{\pgfqpoint{3.622205in}{2.891176in}}%
\pgfpathlineto{\pgfqpoint{2.349853in}{0.872712in}}%
\pgfpathlineto{\pgfqpoint{3.622205in}{2.891176in}}%
\pgfpathclose%
\pgfusepath{fill}%
\end{pgfscope}%
\begin{pgfscope}%
\pgfpathrectangle{\pgfqpoint{0.017500in}{0.017500in}}{\pgfqpoint{6.218282in}{3.580372in}} %
\pgfusepath{clip}%
\pgfsetbuttcap%
\pgfsetroundjoin%
\definecolor{currentfill}{rgb}{0.899543,0.847500,0.817789}%
\pgfsetfillcolor{currentfill}%
\pgfsetlinewidth{0.000000pt}%
\definecolor{currentstroke}{rgb}{0.000000,0.000000,0.000000}%
\pgfsetstrokecolor{currentstroke}%
\pgfsetdash{}{0pt}%
\pgfpathmoveto{\pgfqpoint{3.622205in}{2.891176in}}%
\pgfpathlineto{\pgfqpoint{2.349853in}{0.872712in}}%
\pgfpathlineto{\pgfqpoint{3.622205in}{2.891176in}}%
\pgfpathclose%
\pgfusepath{fill}%
\end{pgfscope}%
\begin{pgfscope}%
\pgfpathrectangle{\pgfqpoint{0.017500in}{0.017500in}}{\pgfqpoint{6.218282in}{3.580372in}} %
\pgfusepath{clip}%
\pgfsetbuttcap%
\pgfsetroundjoin%
\definecolor{currentfill}{rgb}{0.899543,0.847500,0.817789}%
\pgfsetfillcolor{currentfill}%
\pgfsetlinewidth{0.000000pt}%
\definecolor{currentstroke}{rgb}{0.000000,0.000000,0.000000}%
\pgfsetstrokecolor{currentstroke}%
\pgfsetdash{}{0pt}%
\pgfpathmoveto{\pgfqpoint{3.622205in}{2.891176in}}%
\pgfpathlineto{\pgfqpoint{2.349853in}{0.872712in}}%
\pgfpathlineto{\pgfqpoint{3.622205in}{2.891176in}}%
\pgfpathclose%
\pgfusepath{fill}%
\end{pgfscope}%
\begin{pgfscope}%
\pgfpathrectangle{\pgfqpoint{0.017500in}{0.017500in}}{\pgfqpoint{6.218282in}{3.580372in}} %
\pgfusepath{clip}%
\pgfsetbuttcap%
\pgfsetroundjoin%
\definecolor{currentfill}{rgb}{0.899543,0.847500,0.817789}%
\pgfsetfillcolor{currentfill}%
\pgfsetlinewidth{0.000000pt}%
\definecolor{currentstroke}{rgb}{0.000000,0.000000,0.000000}%
\pgfsetstrokecolor{currentstroke}%
\pgfsetdash{}{0pt}%
\pgfpathmoveto{\pgfqpoint{3.622205in}{2.891176in}}%
\pgfpathlineto{\pgfqpoint{2.349853in}{0.872712in}}%
\pgfpathlineto{\pgfqpoint{3.622205in}{2.891176in}}%
\pgfpathclose%
\pgfusepath{fill}%
\end{pgfscope}%
\begin{pgfscope}%
\pgfpathrectangle{\pgfqpoint{0.017500in}{0.017500in}}{\pgfqpoint{6.218282in}{3.580372in}} %
\pgfusepath{clip}%
\pgfsetbuttcap%
\pgfsetroundjoin%
\definecolor{currentfill}{rgb}{0.899543,0.847500,0.817789}%
\pgfsetfillcolor{currentfill}%
\pgfsetlinewidth{0.000000pt}%
\definecolor{currentstroke}{rgb}{0.000000,0.000000,0.000000}%
\pgfsetstrokecolor{currentstroke}%
\pgfsetdash{}{0pt}%
\pgfpathmoveto{\pgfqpoint{3.622205in}{2.891176in}}%
\pgfpathlineto{\pgfqpoint{2.349853in}{0.872712in}}%
\pgfpathlineto{\pgfqpoint{3.622205in}{2.891176in}}%
\pgfpathclose%
\pgfusepath{fill}%
\end{pgfscope}%
\begin{pgfscope}%
\pgfpathrectangle{\pgfqpoint{0.017500in}{0.017500in}}{\pgfqpoint{6.218282in}{3.580372in}} %
\pgfusepath{clip}%
\pgfsetbuttcap%
\pgfsetroundjoin%
\definecolor{currentfill}{rgb}{0.899543,0.847500,0.817789}%
\pgfsetfillcolor{currentfill}%
\pgfsetlinewidth{0.000000pt}%
\definecolor{currentstroke}{rgb}{0.000000,0.000000,0.000000}%
\pgfsetstrokecolor{currentstroke}%
\pgfsetdash{}{0pt}%
\pgfpathmoveto{\pgfqpoint{3.622205in}{2.891176in}}%
\pgfpathlineto{\pgfqpoint{2.349853in}{0.872712in}}%
\pgfpathlineto{\pgfqpoint{3.622205in}{2.891176in}}%
\pgfpathclose%
\pgfusepath{fill}%
\end{pgfscope}%
\begin{pgfscope}%
\pgfpathrectangle{\pgfqpoint{0.017500in}{0.017500in}}{\pgfqpoint{6.218282in}{3.580372in}} %
\pgfusepath{clip}%
\pgfsetbuttcap%
\pgfsetroundjoin%
\definecolor{currentfill}{rgb}{0.899543,0.847500,0.817789}%
\pgfsetfillcolor{currentfill}%
\pgfsetlinewidth{0.000000pt}%
\definecolor{currentstroke}{rgb}{0.000000,0.000000,0.000000}%
\pgfsetstrokecolor{currentstroke}%
\pgfsetdash{}{0pt}%
\pgfpathmoveto{\pgfqpoint{3.622205in}{2.891176in}}%
\pgfpathlineto{\pgfqpoint{2.349853in}{0.872712in}}%
\pgfpathlineto{\pgfqpoint{3.622205in}{2.891176in}}%
\pgfpathclose%
\pgfusepath{fill}%
\end{pgfscope}%
\begin{pgfscope}%
\pgfpathrectangle{\pgfqpoint{0.017500in}{0.017500in}}{\pgfqpoint{6.218282in}{3.580372in}} %
\pgfusepath{clip}%
\pgfsetbuttcap%
\pgfsetroundjoin%
\definecolor{currentfill}{rgb}{0.899543,0.847500,0.817789}%
\pgfsetfillcolor{currentfill}%
\pgfsetlinewidth{0.000000pt}%
\definecolor{currentstroke}{rgb}{0.000000,0.000000,0.000000}%
\pgfsetstrokecolor{currentstroke}%
\pgfsetdash{}{0pt}%
\pgfpathmoveto{\pgfqpoint{3.622205in}{2.891176in}}%
\pgfpathlineto{\pgfqpoint{2.349853in}{0.872712in}}%
\pgfpathlineto{\pgfqpoint{3.622205in}{2.891176in}}%
\pgfpathclose%
\pgfusepath{fill}%
\end{pgfscope}%
\begin{pgfscope}%
\pgfpathrectangle{\pgfqpoint{0.017500in}{0.017500in}}{\pgfqpoint{6.218282in}{3.580372in}} %
\pgfusepath{clip}%
\pgfsetbuttcap%
\pgfsetroundjoin%
\definecolor{currentfill}{rgb}{0.899543,0.847500,0.817789}%
\pgfsetfillcolor{currentfill}%
\pgfsetlinewidth{0.000000pt}%
\definecolor{currentstroke}{rgb}{0.000000,0.000000,0.000000}%
\pgfsetstrokecolor{currentstroke}%
\pgfsetdash{}{0pt}%
\pgfpathmoveto{\pgfqpoint{3.622205in}{2.891176in}}%
\pgfpathlineto{\pgfqpoint{2.349853in}{0.872712in}}%
\pgfpathlineto{\pgfqpoint{3.622205in}{2.891176in}}%
\pgfpathclose%
\pgfusepath{fill}%
\end{pgfscope}%
\begin{pgfscope}%
\pgfpathrectangle{\pgfqpoint{0.017500in}{0.017500in}}{\pgfqpoint{6.218282in}{3.580372in}} %
\pgfusepath{clip}%
\pgfsetbuttcap%
\pgfsetroundjoin%
\definecolor{currentfill}{rgb}{0.899543,0.847500,0.817789}%
\pgfsetfillcolor{currentfill}%
\pgfsetlinewidth{0.000000pt}%
\definecolor{currentstroke}{rgb}{0.000000,0.000000,0.000000}%
\pgfsetstrokecolor{currentstroke}%
\pgfsetdash{}{0pt}%
\pgfpathmoveto{\pgfqpoint{3.622205in}{2.891176in}}%
\pgfpathlineto{\pgfqpoint{2.349853in}{0.872712in}}%
\pgfpathlineto{\pgfqpoint{3.622205in}{2.891176in}}%
\pgfpathclose%
\pgfusepath{fill}%
\end{pgfscope}%
\begin{pgfscope}%
\pgfpathrectangle{\pgfqpoint{0.017500in}{0.017500in}}{\pgfqpoint{6.218282in}{3.580372in}} %
\pgfusepath{clip}%
\pgfsetbuttcap%
\pgfsetroundjoin%
\definecolor{currentfill}{rgb}{0.899543,0.847500,0.817789}%
\pgfsetfillcolor{currentfill}%
\pgfsetlinewidth{0.000000pt}%
\definecolor{currentstroke}{rgb}{0.000000,0.000000,0.000000}%
\pgfsetstrokecolor{currentstroke}%
\pgfsetdash{}{0pt}%
\pgfpathmoveto{\pgfqpoint{3.622205in}{2.891176in}}%
\pgfpathlineto{\pgfqpoint{2.349853in}{0.872712in}}%
\pgfpathlineto{\pgfqpoint{3.622205in}{2.891176in}}%
\pgfpathclose%
\pgfusepath{fill}%
\end{pgfscope}%
\begin{pgfscope}%
\pgfpathrectangle{\pgfqpoint{0.017500in}{0.017500in}}{\pgfqpoint{6.218282in}{3.580372in}} %
\pgfusepath{clip}%
\pgfsetbuttcap%
\pgfsetroundjoin%
\definecolor{currentfill}{rgb}{0.899543,0.847500,0.817789}%
\pgfsetfillcolor{currentfill}%
\pgfsetlinewidth{0.000000pt}%
\definecolor{currentstroke}{rgb}{0.000000,0.000000,0.000000}%
\pgfsetstrokecolor{currentstroke}%
\pgfsetdash{}{0pt}%
\pgfpathmoveto{\pgfqpoint{3.622205in}{2.891176in}}%
\pgfpathlineto{\pgfqpoint{2.349853in}{0.872712in}}%
\pgfpathlineto{\pgfqpoint{3.622205in}{2.891176in}}%
\pgfpathclose%
\pgfusepath{fill}%
\end{pgfscope}%
\begin{pgfscope}%
\pgfpathrectangle{\pgfqpoint{0.017500in}{0.017500in}}{\pgfqpoint{6.218282in}{3.580372in}} %
\pgfusepath{clip}%
\pgfsetbuttcap%
\pgfsetroundjoin%
\definecolor{currentfill}{rgb}{0.899543,0.847500,0.817789}%
\pgfsetfillcolor{currentfill}%
\pgfsetlinewidth{0.000000pt}%
\definecolor{currentstroke}{rgb}{0.000000,0.000000,0.000000}%
\pgfsetstrokecolor{currentstroke}%
\pgfsetdash{}{0pt}%
\pgfpathmoveto{\pgfqpoint{3.622205in}{2.891176in}}%
\pgfpathlineto{\pgfqpoint{2.349853in}{0.872712in}}%
\pgfpathlineto{\pgfqpoint{3.622205in}{2.891176in}}%
\pgfpathclose%
\pgfusepath{fill}%
\end{pgfscope}%
\begin{pgfscope}%
\pgfpathrectangle{\pgfqpoint{0.017500in}{0.017500in}}{\pgfqpoint{6.218282in}{3.580372in}} %
\pgfusepath{clip}%
\pgfsetbuttcap%
\pgfsetroundjoin%
\definecolor{currentfill}{rgb}{0.899543,0.847500,0.817789}%
\pgfsetfillcolor{currentfill}%
\pgfsetlinewidth{0.000000pt}%
\definecolor{currentstroke}{rgb}{0.000000,0.000000,0.000000}%
\pgfsetstrokecolor{currentstroke}%
\pgfsetdash{}{0pt}%
\pgfpathmoveto{\pgfqpoint{3.622205in}{2.891176in}}%
\pgfpathlineto{\pgfqpoint{2.349853in}{0.872712in}}%
\pgfpathlineto{\pgfqpoint{3.622205in}{2.891176in}}%
\pgfpathclose%
\pgfusepath{fill}%
\end{pgfscope}%
\begin{pgfscope}%
\pgfpathrectangle{\pgfqpoint{0.017500in}{0.017500in}}{\pgfqpoint{6.218282in}{3.580372in}} %
\pgfusepath{clip}%
\pgfsetbuttcap%
\pgfsetroundjoin%
\definecolor{currentfill}{rgb}{0.899543,0.847500,0.817789}%
\pgfsetfillcolor{currentfill}%
\pgfsetlinewidth{0.000000pt}%
\definecolor{currentstroke}{rgb}{0.000000,0.000000,0.000000}%
\pgfsetstrokecolor{currentstroke}%
\pgfsetdash{}{0pt}%
\pgfpathmoveto{\pgfqpoint{3.622205in}{2.891176in}}%
\pgfpathlineto{\pgfqpoint{2.349853in}{0.872712in}}%
\pgfpathlineto{\pgfqpoint{3.622205in}{2.891176in}}%
\pgfpathclose%
\pgfusepath{fill}%
\end{pgfscope}%
\begin{pgfscope}%
\pgfpathrectangle{\pgfqpoint{0.017500in}{0.017500in}}{\pgfqpoint{6.218282in}{3.580372in}} %
\pgfusepath{clip}%
\pgfsetbuttcap%
\pgfsetroundjoin%
\definecolor{currentfill}{rgb}{0.899543,0.847500,0.817789}%
\pgfsetfillcolor{currentfill}%
\pgfsetlinewidth{0.000000pt}%
\definecolor{currentstroke}{rgb}{0.000000,0.000000,0.000000}%
\pgfsetstrokecolor{currentstroke}%
\pgfsetdash{}{0pt}%
\pgfpathmoveto{\pgfqpoint{3.622205in}{2.891176in}}%
\pgfpathlineto{\pgfqpoint{2.349853in}{0.872712in}}%
\pgfpathlineto{\pgfqpoint{3.622205in}{2.891176in}}%
\pgfpathclose%
\pgfusepath{fill}%
\end{pgfscope}%
\begin{pgfscope}%
\pgfpathrectangle{\pgfqpoint{0.017500in}{0.017500in}}{\pgfqpoint{6.218282in}{3.580372in}} %
\pgfusepath{clip}%
\pgfsetbuttcap%
\pgfsetroundjoin%
\definecolor{currentfill}{rgb}{0.899543,0.847500,0.817789}%
\pgfsetfillcolor{currentfill}%
\pgfsetlinewidth{0.000000pt}%
\definecolor{currentstroke}{rgb}{0.000000,0.000000,0.000000}%
\pgfsetstrokecolor{currentstroke}%
\pgfsetdash{}{0pt}%
\pgfpathmoveto{\pgfqpoint{3.622205in}{2.891176in}}%
\pgfpathlineto{\pgfqpoint{2.349853in}{0.872712in}}%
\pgfpathlineto{\pgfqpoint{3.622205in}{2.891176in}}%
\pgfpathclose%
\pgfusepath{fill}%
\end{pgfscope}%
\begin{pgfscope}%
\pgfpathrectangle{\pgfqpoint{0.017500in}{0.017500in}}{\pgfqpoint{6.218282in}{3.580372in}} %
\pgfusepath{clip}%
\pgfsetbuttcap%
\pgfsetroundjoin%
\definecolor{currentfill}{rgb}{0.899543,0.847500,0.817789}%
\pgfsetfillcolor{currentfill}%
\pgfsetlinewidth{0.000000pt}%
\definecolor{currentstroke}{rgb}{0.000000,0.000000,0.000000}%
\pgfsetstrokecolor{currentstroke}%
\pgfsetdash{}{0pt}%
\pgfpathmoveto{\pgfqpoint{3.622205in}{2.891176in}}%
\pgfpathlineto{\pgfqpoint{2.349853in}{0.872712in}}%
\pgfpathlineto{\pgfqpoint{3.622205in}{2.891176in}}%
\pgfpathclose%
\pgfusepath{fill}%
\end{pgfscope}%
\begin{pgfscope}%
\pgfpathrectangle{\pgfqpoint{0.017500in}{0.017500in}}{\pgfqpoint{6.218282in}{3.580372in}} %
\pgfusepath{clip}%
\pgfsetbuttcap%
\pgfsetroundjoin%
\definecolor{currentfill}{rgb}{0.899543,0.847500,0.817789}%
\pgfsetfillcolor{currentfill}%
\pgfsetlinewidth{0.000000pt}%
\definecolor{currentstroke}{rgb}{0.000000,0.000000,0.000000}%
\pgfsetstrokecolor{currentstroke}%
\pgfsetdash{}{0pt}%
\pgfpathmoveto{\pgfqpoint{3.622205in}{2.891176in}}%
\pgfpathlineto{\pgfqpoint{2.349853in}{0.872712in}}%
\pgfpathlineto{\pgfqpoint{3.622205in}{2.891176in}}%
\pgfpathclose%
\pgfusepath{fill}%
\end{pgfscope}%
\begin{pgfscope}%
\pgfpathrectangle{\pgfqpoint{0.017500in}{0.017500in}}{\pgfqpoint{6.218282in}{3.580372in}} %
\pgfusepath{clip}%
\pgfsetbuttcap%
\pgfsetroundjoin%
\definecolor{currentfill}{rgb}{0.899543,0.847500,0.817789}%
\pgfsetfillcolor{currentfill}%
\pgfsetlinewidth{0.000000pt}%
\definecolor{currentstroke}{rgb}{0.000000,0.000000,0.000000}%
\pgfsetstrokecolor{currentstroke}%
\pgfsetdash{}{0pt}%
\pgfpathmoveto{\pgfqpoint{3.622205in}{2.891176in}}%
\pgfpathlineto{\pgfqpoint{2.349853in}{0.872712in}}%
\pgfpathlineto{\pgfqpoint{3.622205in}{2.891176in}}%
\pgfpathclose%
\pgfusepath{fill}%
\end{pgfscope}%
\begin{pgfscope}%
\pgfpathrectangle{\pgfqpoint{0.017500in}{0.017500in}}{\pgfqpoint{6.218282in}{3.580372in}} %
\pgfusepath{clip}%
\pgfsetbuttcap%
\pgfsetroundjoin%
\definecolor{currentfill}{rgb}{0.899543,0.847500,0.817789}%
\pgfsetfillcolor{currentfill}%
\pgfsetlinewidth{0.000000pt}%
\definecolor{currentstroke}{rgb}{0.000000,0.000000,0.000000}%
\pgfsetstrokecolor{currentstroke}%
\pgfsetdash{}{0pt}%
\pgfpathmoveto{\pgfqpoint{3.622205in}{2.891176in}}%
\pgfpathlineto{\pgfqpoint{2.349853in}{0.872712in}}%
\pgfpathlineto{\pgfqpoint{3.622205in}{2.891176in}}%
\pgfpathclose%
\pgfusepath{fill}%
\end{pgfscope}%
\begin{pgfscope}%
\pgfpathrectangle{\pgfqpoint{0.017500in}{0.017500in}}{\pgfqpoint{6.218282in}{3.580372in}} %
\pgfusepath{clip}%
\pgfsetbuttcap%
\pgfsetroundjoin%
\definecolor{currentfill}{rgb}{0.899543,0.847500,0.817789}%
\pgfsetfillcolor{currentfill}%
\pgfsetlinewidth{0.000000pt}%
\definecolor{currentstroke}{rgb}{0.000000,0.000000,0.000000}%
\pgfsetstrokecolor{currentstroke}%
\pgfsetdash{}{0pt}%
\pgfpathmoveto{\pgfqpoint{3.622205in}{2.891176in}}%
\pgfpathlineto{\pgfqpoint{2.349853in}{0.872712in}}%
\pgfpathlineto{\pgfqpoint{3.622205in}{2.891176in}}%
\pgfpathclose%
\pgfusepath{fill}%
\end{pgfscope}%
\begin{pgfscope}%
\pgfpathrectangle{\pgfqpoint{0.017500in}{0.017500in}}{\pgfqpoint{6.218282in}{3.580372in}} %
\pgfusepath{clip}%
\pgfsetbuttcap%
\pgfsetroundjoin%
\definecolor{currentfill}{rgb}{0.746838,0.140021,0.179996}%
\pgfsetfillcolor{currentfill}%
\pgfsetlinewidth{0.000000pt}%
\definecolor{currentstroke}{rgb}{0.000000,0.000000,0.000000}%
\pgfsetstrokecolor{currentstroke}%
\pgfsetdash{}{0pt}%
\pgfpathmoveto{\pgfqpoint{2.237924in}{2.752812in}}%
\pgfpathlineto{\pgfqpoint{2.674254in}{2.688576in}}%
\pgfpathlineto{\pgfqpoint{2.932968in}{2.820668in}}%
\pgfpathlineto{\pgfqpoint{2.499217in}{2.903684in}}%
\pgfpathlineto{\pgfqpoint{2.237924in}{2.752812in}}%
\pgfpathclose%
\pgfusepath{fill}%
\end{pgfscope}%
\begin{pgfscope}%
\pgfpathrectangle{\pgfqpoint{0.017500in}{0.017500in}}{\pgfqpoint{6.218282in}{3.580372in}} %
\pgfusepath{clip}%
\pgfsetbuttcap%
\pgfsetroundjoin%
\definecolor{currentfill}{rgb}{0.746838,0.140021,0.179996}%
\pgfsetfillcolor{currentfill}%
\pgfsetlinewidth{0.000000pt}%
\definecolor{currentstroke}{rgb}{0.000000,0.000000,0.000000}%
\pgfsetstrokecolor{currentstroke}%
\pgfsetdash{}{0pt}%
\pgfpathmoveto{\pgfqpoint{2.237924in}{2.752812in}}%
\pgfpathlineto{\pgfqpoint{2.674254in}{2.688576in}}%
\pgfpathlineto{\pgfqpoint{2.932968in}{2.820668in}}%
\pgfpathlineto{\pgfqpoint{2.499217in}{2.903684in}}%
\pgfpathlineto{\pgfqpoint{2.237924in}{2.752812in}}%
\pgfpathclose%
\pgfusepath{fill}%
\end{pgfscope}%
\begin{pgfscope}%
\pgfpathrectangle{\pgfqpoint{0.017500in}{0.017500in}}{\pgfqpoint{6.218282in}{3.580372in}} %
\pgfusepath{clip}%
\pgfsetbuttcap%
\pgfsetroundjoin%
\definecolor{currentfill}{rgb}{0.746838,0.140021,0.179996}%
\pgfsetfillcolor{currentfill}%
\pgfsetlinewidth{0.000000pt}%
\definecolor{currentstroke}{rgb}{0.000000,0.000000,0.000000}%
\pgfsetstrokecolor{currentstroke}%
\pgfsetdash{}{0pt}%
\pgfpathmoveto{\pgfqpoint{2.237924in}{2.752812in}}%
\pgfpathlineto{\pgfqpoint{2.674254in}{2.688576in}}%
\pgfpathlineto{\pgfqpoint{2.932968in}{2.820668in}}%
\pgfpathlineto{\pgfqpoint{2.499217in}{2.903684in}}%
\pgfpathlineto{\pgfqpoint{2.237924in}{2.752812in}}%
\pgfpathclose%
\pgfusepath{fill}%
\end{pgfscope}%
\begin{pgfscope}%
\pgfpathrectangle{\pgfqpoint{0.017500in}{0.017500in}}{\pgfqpoint{6.218282in}{3.580372in}} %
\pgfusepath{clip}%
\pgfsetbuttcap%
\pgfsetroundjoin%
\definecolor{currentfill}{rgb}{0.746838,0.140021,0.179996}%
\pgfsetfillcolor{currentfill}%
\pgfsetlinewidth{0.000000pt}%
\definecolor{currentstroke}{rgb}{0.000000,0.000000,0.000000}%
\pgfsetstrokecolor{currentstroke}%
\pgfsetdash{}{0pt}%
\pgfpathmoveto{\pgfqpoint{2.237924in}{2.752812in}}%
\pgfpathlineto{\pgfqpoint{2.674254in}{2.688576in}}%
\pgfpathlineto{\pgfqpoint{2.932968in}{2.820668in}}%
\pgfpathlineto{\pgfqpoint{2.499217in}{2.903684in}}%
\pgfpathlineto{\pgfqpoint{2.237924in}{2.752812in}}%
\pgfpathclose%
\pgfusepath{fill}%
\end{pgfscope}%
\begin{pgfscope}%
\pgfpathrectangle{\pgfqpoint{0.017500in}{0.017500in}}{\pgfqpoint{6.218282in}{3.580372in}} %
\pgfusepath{clip}%
\pgfsetbuttcap%
\pgfsetroundjoin%
\definecolor{currentfill}{rgb}{0.746838,0.140021,0.179996}%
\pgfsetfillcolor{currentfill}%
\pgfsetlinewidth{0.000000pt}%
\definecolor{currentstroke}{rgb}{0.000000,0.000000,0.000000}%
\pgfsetstrokecolor{currentstroke}%
\pgfsetdash{}{0pt}%
\pgfpathmoveto{\pgfqpoint{2.237924in}{2.752812in}}%
\pgfpathlineto{\pgfqpoint{2.674254in}{2.688576in}}%
\pgfpathlineto{\pgfqpoint{2.932968in}{2.820668in}}%
\pgfpathlineto{\pgfqpoint{2.499217in}{2.903684in}}%
\pgfpathlineto{\pgfqpoint{2.237924in}{2.752812in}}%
\pgfpathclose%
\pgfusepath{fill}%
\end{pgfscope}%
\begin{pgfscope}%
\pgfpathrectangle{\pgfqpoint{0.017500in}{0.017500in}}{\pgfqpoint{6.218282in}{3.580372in}} %
\pgfusepath{clip}%
\pgfsetbuttcap%
\pgfsetroundjoin%
\definecolor{currentfill}{rgb}{0.746838,0.140021,0.179996}%
\pgfsetfillcolor{currentfill}%
\pgfsetlinewidth{0.000000pt}%
\definecolor{currentstroke}{rgb}{0.000000,0.000000,0.000000}%
\pgfsetstrokecolor{currentstroke}%
\pgfsetdash{}{0pt}%
\pgfpathmoveto{\pgfqpoint{2.237924in}{2.752812in}}%
\pgfpathlineto{\pgfqpoint{2.674254in}{2.688576in}}%
\pgfpathlineto{\pgfqpoint{2.932968in}{2.820668in}}%
\pgfpathlineto{\pgfqpoint{2.499217in}{2.903684in}}%
\pgfpathlineto{\pgfqpoint{2.237924in}{2.752812in}}%
\pgfpathclose%
\pgfusepath{fill}%
\end{pgfscope}%
\begin{pgfscope}%
\pgfpathrectangle{\pgfqpoint{0.017500in}{0.017500in}}{\pgfqpoint{6.218282in}{3.580372in}} %
\pgfusepath{clip}%
\pgfsetbuttcap%
\pgfsetroundjoin%
\definecolor{currentfill}{rgb}{0.717435,0.051118,0.158737}%
\pgfsetfillcolor{currentfill}%
\pgfsetlinewidth{0.000000pt}%
\definecolor{currentstroke}{rgb}{0.000000,0.000000,0.000000}%
\pgfsetstrokecolor{currentstroke}%
\pgfsetdash{}{0pt}%
\pgfpathmoveto{\pgfqpoint{3.372093in}{2.761754in}}%
\pgfpathlineto{\pgfqpoint{3.622205in}{2.891176in}}%
\pgfpathlineto{\pgfqpoint{3.372093in}{2.761754in}}%
\pgfpathclose%
\pgfusepath{fill}%
\end{pgfscope}%
\begin{pgfscope}%
\pgfpathrectangle{\pgfqpoint{0.017500in}{0.017500in}}{\pgfqpoint{6.218282in}{3.580372in}} %
\pgfusepath{clip}%
\pgfsetbuttcap%
\pgfsetroundjoin%
\definecolor{currentfill}{rgb}{0.717435,0.051118,0.158737}%
\pgfsetfillcolor{currentfill}%
\pgfsetlinewidth{0.000000pt}%
\definecolor{currentstroke}{rgb}{0.000000,0.000000,0.000000}%
\pgfsetstrokecolor{currentstroke}%
\pgfsetdash{}{0pt}%
\pgfpathmoveto{\pgfqpoint{3.372093in}{2.761754in}}%
\pgfpathlineto{\pgfqpoint{3.622205in}{2.891176in}}%
\pgfpathlineto{\pgfqpoint{3.372093in}{2.761754in}}%
\pgfpathclose%
\pgfusepath{fill}%
\end{pgfscope}%
\begin{pgfscope}%
\pgfpathrectangle{\pgfqpoint{0.017500in}{0.017500in}}{\pgfqpoint{6.218282in}{3.580372in}} %
\pgfusepath{clip}%
\pgfsetbuttcap%
\pgfsetroundjoin%
\definecolor{currentfill}{rgb}{0.717435,0.051118,0.158737}%
\pgfsetfillcolor{currentfill}%
\pgfsetlinewidth{0.000000pt}%
\definecolor{currentstroke}{rgb}{0.000000,0.000000,0.000000}%
\pgfsetstrokecolor{currentstroke}%
\pgfsetdash{}{0pt}%
\pgfpathmoveto{\pgfqpoint{3.372093in}{2.761754in}}%
\pgfpathlineto{\pgfqpoint{3.622205in}{2.891176in}}%
\pgfpathlineto{\pgfqpoint{3.372093in}{2.761754in}}%
\pgfpathclose%
\pgfusepath{fill}%
\end{pgfscope}%
\begin{pgfscope}%
\pgfpathrectangle{\pgfqpoint{0.017500in}{0.017500in}}{\pgfqpoint{6.218282in}{3.580372in}} %
\pgfusepath{clip}%
\pgfsetbuttcap%
\pgfsetroundjoin%
\definecolor{currentfill}{rgb}{0.717435,0.051118,0.158737}%
\pgfsetfillcolor{currentfill}%
\pgfsetlinewidth{0.000000pt}%
\definecolor{currentstroke}{rgb}{0.000000,0.000000,0.000000}%
\pgfsetstrokecolor{currentstroke}%
\pgfsetdash{}{0pt}%
\pgfpathmoveto{\pgfqpoint{3.372093in}{2.761754in}}%
\pgfpathlineto{\pgfqpoint{3.622205in}{2.891176in}}%
\pgfpathlineto{\pgfqpoint{3.372093in}{2.761754in}}%
\pgfpathclose%
\pgfusepath{fill}%
\end{pgfscope}%
\begin{pgfscope}%
\pgfpathrectangle{\pgfqpoint{0.017500in}{0.017500in}}{\pgfqpoint{6.218282in}{3.580372in}} %
\pgfusepath{clip}%
\pgfsetbuttcap%
\pgfsetroundjoin%
\definecolor{currentfill}{rgb}{0.717435,0.051118,0.158737}%
\pgfsetfillcolor{currentfill}%
\pgfsetlinewidth{0.000000pt}%
\definecolor{currentstroke}{rgb}{0.000000,0.000000,0.000000}%
\pgfsetstrokecolor{currentstroke}%
\pgfsetdash{}{0pt}%
\pgfpathmoveto{\pgfqpoint{3.372093in}{2.761754in}}%
\pgfpathlineto{\pgfqpoint{3.622205in}{2.891176in}}%
\pgfpathlineto{\pgfqpoint{3.372093in}{2.761754in}}%
\pgfpathclose%
\pgfusepath{fill}%
\end{pgfscope}%
\begin{pgfscope}%
\pgfpathrectangle{\pgfqpoint{0.017500in}{0.017500in}}{\pgfqpoint{6.218282in}{3.580372in}} %
\pgfusepath{clip}%
\pgfsetbuttcap%
\pgfsetroundjoin%
\definecolor{currentfill}{rgb}{0.717435,0.051118,0.158737}%
\pgfsetfillcolor{currentfill}%
\pgfsetlinewidth{0.000000pt}%
\definecolor{currentstroke}{rgb}{0.000000,0.000000,0.000000}%
\pgfsetstrokecolor{currentstroke}%
\pgfsetdash{}{0pt}%
\pgfpathmoveto{\pgfqpoint{3.372093in}{2.761754in}}%
\pgfpathlineto{\pgfqpoint{3.622205in}{2.891176in}}%
\pgfpathlineto{\pgfqpoint{3.372093in}{2.761754in}}%
\pgfpathclose%
\pgfusepath{fill}%
\end{pgfscope}%
\begin{pgfscope}%
\pgfpathrectangle{\pgfqpoint{0.017500in}{0.017500in}}{\pgfqpoint{6.218282in}{3.580372in}} %
\pgfusepath{clip}%
\pgfsetbuttcap%
\pgfsetroundjoin%
\definecolor{currentfill}{rgb}{0.717435,0.051118,0.158737}%
\pgfsetfillcolor{currentfill}%
\pgfsetlinewidth{0.000000pt}%
\definecolor{currentstroke}{rgb}{0.000000,0.000000,0.000000}%
\pgfsetstrokecolor{currentstroke}%
\pgfsetdash{}{0pt}%
\pgfpathmoveto{\pgfqpoint{3.372093in}{2.761754in}}%
\pgfpathlineto{\pgfqpoint{3.622205in}{2.891176in}}%
\pgfpathlineto{\pgfqpoint{3.372093in}{2.761754in}}%
\pgfpathclose%
\pgfusepath{fill}%
\end{pgfscope}%
\begin{pgfscope}%
\pgfpathrectangle{\pgfqpoint{0.017500in}{0.017500in}}{\pgfqpoint{6.218282in}{3.580372in}} %
\pgfusepath{clip}%
\pgfsetbuttcap%
\pgfsetroundjoin%
\definecolor{currentfill}{rgb}{0.717435,0.051118,0.158737}%
\pgfsetfillcolor{currentfill}%
\pgfsetlinewidth{0.000000pt}%
\definecolor{currentstroke}{rgb}{0.000000,0.000000,0.000000}%
\pgfsetstrokecolor{currentstroke}%
\pgfsetdash{}{0pt}%
\pgfpathmoveto{\pgfqpoint{3.372093in}{2.761754in}}%
\pgfpathlineto{\pgfqpoint{3.622205in}{2.891176in}}%
\pgfpathlineto{\pgfqpoint{3.372093in}{2.761754in}}%
\pgfpathclose%
\pgfusepath{fill}%
\end{pgfscope}%
\begin{pgfscope}%
\pgfpathrectangle{\pgfqpoint{0.017500in}{0.017500in}}{\pgfqpoint{6.218282in}{3.580372in}} %
\pgfusepath{clip}%
\pgfsetbuttcap%
\pgfsetroundjoin%
\definecolor{currentfill}{rgb}{0.717435,0.051118,0.158737}%
\pgfsetfillcolor{currentfill}%
\pgfsetlinewidth{0.000000pt}%
\definecolor{currentstroke}{rgb}{0.000000,0.000000,0.000000}%
\pgfsetstrokecolor{currentstroke}%
\pgfsetdash{}{0pt}%
\pgfpathmoveto{\pgfqpoint{3.372093in}{2.761754in}}%
\pgfpathlineto{\pgfqpoint{3.622205in}{2.891176in}}%
\pgfpathlineto{\pgfqpoint{3.372093in}{2.761754in}}%
\pgfpathclose%
\pgfusepath{fill}%
\end{pgfscope}%
\begin{pgfscope}%
\pgfpathrectangle{\pgfqpoint{0.017500in}{0.017500in}}{\pgfqpoint{6.218282in}{3.580372in}} %
\pgfusepath{clip}%
\pgfsetbuttcap%
\pgfsetroundjoin%
\definecolor{currentfill}{rgb}{0.717435,0.051118,0.158737}%
\pgfsetfillcolor{currentfill}%
\pgfsetlinewidth{0.000000pt}%
\definecolor{currentstroke}{rgb}{0.000000,0.000000,0.000000}%
\pgfsetstrokecolor{currentstroke}%
\pgfsetdash{}{0pt}%
\pgfpathmoveto{\pgfqpoint{3.372093in}{2.761754in}}%
\pgfpathlineto{\pgfqpoint{3.622205in}{2.891176in}}%
\pgfpathlineto{\pgfqpoint{3.372093in}{2.761754in}}%
\pgfpathclose%
\pgfusepath{fill}%
\end{pgfscope}%
\begin{pgfscope}%
\pgfpathrectangle{\pgfqpoint{0.017500in}{0.017500in}}{\pgfqpoint{6.218282in}{3.580372in}} %
\pgfusepath{clip}%
\pgfsetbuttcap%
\pgfsetroundjoin%
\definecolor{currentfill}{rgb}{0.717435,0.051118,0.158737}%
\pgfsetfillcolor{currentfill}%
\pgfsetlinewidth{0.000000pt}%
\definecolor{currentstroke}{rgb}{0.000000,0.000000,0.000000}%
\pgfsetstrokecolor{currentstroke}%
\pgfsetdash{}{0pt}%
\pgfpathmoveto{\pgfqpoint{3.372093in}{2.761754in}}%
\pgfpathlineto{\pgfqpoint{3.622205in}{2.891176in}}%
\pgfpathlineto{\pgfqpoint{3.372093in}{2.761754in}}%
\pgfpathclose%
\pgfusepath{fill}%
\end{pgfscope}%
\begin{pgfscope}%
\pgfpathrectangle{\pgfqpoint{0.017500in}{0.017500in}}{\pgfqpoint{6.218282in}{3.580372in}} %
\pgfusepath{clip}%
\pgfsetbuttcap%
\pgfsetroundjoin%
\definecolor{currentfill}{rgb}{0.717435,0.051118,0.158737}%
\pgfsetfillcolor{currentfill}%
\pgfsetlinewidth{0.000000pt}%
\definecolor{currentstroke}{rgb}{0.000000,0.000000,0.000000}%
\pgfsetstrokecolor{currentstroke}%
\pgfsetdash{}{0pt}%
\pgfpathmoveto{\pgfqpoint{3.372093in}{2.761754in}}%
\pgfpathlineto{\pgfqpoint{3.622205in}{2.891176in}}%
\pgfpathlineto{\pgfqpoint{3.372093in}{2.761754in}}%
\pgfpathclose%
\pgfusepath{fill}%
\end{pgfscope}%
\begin{pgfscope}%
\pgfpathrectangle{\pgfqpoint{0.017500in}{0.017500in}}{\pgfqpoint{6.218282in}{3.580372in}} %
\pgfusepath{clip}%
\pgfsetbuttcap%
\pgfsetroundjoin%
\definecolor{currentfill}{rgb}{0.717435,0.051118,0.158737}%
\pgfsetfillcolor{currentfill}%
\pgfsetlinewidth{0.000000pt}%
\definecolor{currentstroke}{rgb}{0.000000,0.000000,0.000000}%
\pgfsetstrokecolor{currentstroke}%
\pgfsetdash{}{0pt}%
\pgfpathmoveto{\pgfqpoint{3.372093in}{2.761754in}}%
\pgfpathlineto{\pgfqpoint{3.622205in}{2.891176in}}%
\pgfpathlineto{\pgfqpoint{3.372093in}{2.761754in}}%
\pgfpathclose%
\pgfusepath{fill}%
\end{pgfscope}%
\begin{pgfscope}%
\pgfpathrectangle{\pgfqpoint{0.017500in}{0.017500in}}{\pgfqpoint{6.218282in}{3.580372in}} %
\pgfusepath{clip}%
\pgfsetbuttcap%
\pgfsetroundjoin%
\definecolor{currentfill}{rgb}{0.717435,0.051118,0.158737}%
\pgfsetfillcolor{currentfill}%
\pgfsetlinewidth{0.000000pt}%
\definecolor{currentstroke}{rgb}{0.000000,0.000000,0.000000}%
\pgfsetstrokecolor{currentstroke}%
\pgfsetdash{}{0pt}%
\pgfpathmoveto{\pgfqpoint{3.372093in}{2.761754in}}%
\pgfpathlineto{\pgfqpoint{3.622205in}{2.891176in}}%
\pgfpathlineto{\pgfqpoint{3.372093in}{2.761754in}}%
\pgfpathclose%
\pgfusepath{fill}%
\end{pgfscope}%
\begin{pgfscope}%
\pgfpathrectangle{\pgfqpoint{0.017500in}{0.017500in}}{\pgfqpoint{6.218282in}{3.580372in}} %
\pgfusepath{clip}%
\pgfsetbuttcap%
\pgfsetroundjoin%
\definecolor{currentfill}{rgb}{0.717435,0.051118,0.158737}%
\pgfsetfillcolor{currentfill}%
\pgfsetlinewidth{0.000000pt}%
\definecolor{currentstroke}{rgb}{0.000000,0.000000,0.000000}%
\pgfsetstrokecolor{currentstroke}%
\pgfsetdash{}{0pt}%
\pgfpathmoveto{\pgfqpoint{3.372093in}{2.761754in}}%
\pgfpathlineto{\pgfqpoint{3.622205in}{2.891176in}}%
\pgfpathlineto{\pgfqpoint{3.372093in}{2.761754in}}%
\pgfpathclose%
\pgfusepath{fill}%
\end{pgfscope}%
\begin{pgfscope}%
\pgfpathrectangle{\pgfqpoint{0.017500in}{0.017500in}}{\pgfqpoint{6.218282in}{3.580372in}} %
\pgfusepath{clip}%
\pgfsetbuttcap%
\pgfsetroundjoin%
\definecolor{currentfill}{rgb}{0.717435,0.051118,0.158737}%
\pgfsetfillcolor{currentfill}%
\pgfsetlinewidth{0.000000pt}%
\definecolor{currentstroke}{rgb}{0.000000,0.000000,0.000000}%
\pgfsetstrokecolor{currentstroke}%
\pgfsetdash{}{0pt}%
\pgfpathmoveto{\pgfqpoint{3.372093in}{2.761754in}}%
\pgfpathlineto{\pgfqpoint{3.622205in}{2.891176in}}%
\pgfpathlineto{\pgfqpoint{3.372093in}{2.761754in}}%
\pgfpathclose%
\pgfusepath{fill}%
\end{pgfscope}%
\begin{pgfscope}%
\pgfpathrectangle{\pgfqpoint{0.017500in}{0.017500in}}{\pgfqpoint{6.218282in}{3.580372in}} %
\pgfusepath{clip}%
\pgfsetbuttcap%
\pgfsetroundjoin%
\definecolor{currentfill}{rgb}{0.717435,0.051118,0.158737}%
\pgfsetfillcolor{currentfill}%
\pgfsetlinewidth{0.000000pt}%
\definecolor{currentstroke}{rgb}{0.000000,0.000000,0.000000}%
\pgfsetstrokecolor{currentstroke}%
\pgfsetdash{}{0pt}%
\pgfpathmoveto{\pgfqpoint{3.372093in}{2.761754in}}%
\pgfpathlineto{\pgfqpoint{3.622205in}{2.891176in}}%
\pgfpathlineto{\pgfqpoint{3.372093in}{2.761754in}}%
\pgfpathclose%
\pgfusepath{fill}%
\end{pgfscope}%
\begin{pgfscope}%
\pgfpathrectangle{\pgfqpoint{0.017500in}{0.017500in}}{\pgfqpoint{6.218282in}{3.580372in}} %
\pgfusepath{clip}%
\pgfsetbuttcap%
\pgfsetroundjoin%
\definecolor{currentfill}{rgb}{0.717435,0.051118,0.158737}%
\pgfsetfillcolor{currentfill}%
\pgfsetlinewidth{0.000000pt}%
\definecolor{currentstroke}{rgb}{0.000000,0.000000,0.000000}%
\pgfsetstrokecolor{currentstroke}%
\pgfsetdash{}{0pt}%
\pgfpathmoveto{\pgfqpoint{3.372093in}{2.761754in}}%
\pgfpathlineto{\pgfqpoint{3.622205in}{2.891176in}}%
\pgfpathlineto{\pgfqpoint{3.372093in}{2.761754in}}%
\pgfpathclose%
\pgfusepath{fill}%
\end{pgfscope}%
\begin{pgfscope}%
\pgfpathrectangle{\pgfqpoint{0.017500in}{0.017500in}}{\pgfqpoint{6.218282in}{3.580372in}} %
\pgfusepath{clip}%
\pgfsetbuttcap%
\pgfsetroundjoin%
\definecolor{currentfill}{rgb}{0.717435,0.051118,0.158737}%
\pgfsetfillcolor{currentfill}%
\pgfsetlinewidth{0.000000pt}%
\definecolor{currentstroke}{rgb}{0.000000,0.000000,0.000000}%
\pgfsetstrokecolor{currentstroke}%
\pgfsetdash{}{0pt}%
\pgfpathmoveto{\pgfqpoint{3.372093in}{2.761754in}}%
\pgfpathlineto{\pgfqpoint{3.622205in}{2.891176in}}%
\pgfpathlineto{\pgfqpoint{3.372093in}{2.761754in}}%
\pgfpathclose%
\pgfusepath{fill}%
\end{pgfscope}%
\begin{pgfscope}%
\pgfpathrectangle{\pgfqpoint{0.017500in}{0.017500in}}{\pgfqpoint{6.218282in}{3.580372in}} %
\pgfusepath{clip}%
\pgfsetbuttcap%
\pgfsetroundjoin%
\definecolor{currentfill}{rgb}{0.717435,0.051118,0.158737}%
\pgfsetfillcolor{currentfill}%
\pgfsetlinewidth{0.000000pt}%
\definecolor{currentstroke}{rgb}{0.000000,0.000000,0.000000}%
\pgfsetstrokecolor{currentstroke}%
\pgfsetdash{}{0pt}%
\pgfpathmoveto{\pgfqpoint{3.372093in}{2.761754in}}%
\pgfpathlineto{\pgfqpoint{3.622205in}{2.891176in}}%
\pgfpathlineto{\pgfqpoint{3.372093in}{2.761754in}}%
\pgfpathclose%
\pgfusepath{fill}%
\end{pgfscope}%
\begin{pgfscope}%
\pgfpathrectangle{\pgfqpoint{0.017500in}{0.017500in}}{\pgfqpoint{6.218282in}{3.580372in}} %
\pgfusepath{clip}%
\pgfsetbuttcap%
\pgfsetroundjoin%
\definecolor{currentfill}{rgb}{0.717435,0.051118,0.158737}%
\pgfsetfillcolor{currentfill}%
\pgfsetlinewidth{0.000000pt}%
\definecolor{currentstroke}{rgb}{0.000000,0.000000,0.000000}%
\pgfsetstrokecolor{currentstroke}%
\pgfsetdash{}{0pt}%
\pgfpathmoveto{\pgfqpoint{3.372093in}{2.761754in}}%
\pgfpathlineto{\pgfqpoint{3.622205in}{2.891176in}}%
\pgfpathlineto{\pgfqpoint{3.372093in}{2.761754in}}%
\pgfpathclose%
\pgfusepath{fill}%
\end{pgfscope}%
\begin{pgfscope}%
\pgfpathrectangle{\pgfqpoint{0.017500in}{0.017500in}}{\pgfqpoint{6.218282in}{3.580372in}} %
\pgfusepath{clip}%
\pgfsetbuttcap%
\pgfsetroundjoin%
\definecolor{currentfill}{rgb}{0.717435,0.051118,0.158737}%
\pgfsetfillcolor{currentfill}%
\pgfsetlinewidth{0.000000pt}%
\definecolor{currentstroke}{rgb}{0.000000,0.000000,0.000000}%
\pgfsetstrokecolor{currentstroke}%
\pgfsetdash{}{0pt}%
\pgfpathmoveto{\pgfqpoint{3.372093in}{2.761754in}}%
\pgfpathlineto{\pgfqpoint{3.622205in}{2.891176in}}%
\pgfpathlineto{\pgfqpoint{3.372093in}{2.761754in}}%
\pgfpathclose%
\pgfusepath{fill}%
\end{pgfscope}%
\begin{pgfscope}%
\pgfpathrectangle{\pgfqpoint{0.017500in}{0.017500in}}{\pgfqpoint{6.218282in}{3.580372in}} %
\pgfusepath{clip}%
\pgfsetbuttcap%
\pgfsetroundjoin%
\definecolor{currentfill}{rgb}{0.717435,0.051118,0.158737}%
\pgfsetfillcolor{currentfill}%
\pgfsetlinewidth{0.000000pt}%
\definecolor{currentstroke}{rgb}{0.000000,0.000000,0.000000}%
\pgfsetstrokecolor{currentstroke}%
\pgfsetdash{}{0pt}%
\pgfpathmoveto{\pgfqpoint{3.372093in}{2.761754in}}%
\pgfpathlineto{\pgfqpoint{3.622205in}{2.891176in}}%
\pgfpathlineto{\pgfqpoint{3.372093in}{2.761754in}}%
\pgfpathclose%
\pgfusepath{fill}%
\end{pgfscope}%
\begin{pgfscope}%
\pgfpathrectangle{\pgfqpoint{0.017500in}{0.017500in}}{\pgfqpoint{6.218282in}{3.580372in}} %
\pgfusepath{clip}%
\pgfsetbuttcap%
\pgfsetroundjoin%
\definecolor{currentfill}{rgb}{0.717435,0.051118,0.158737}%
\pgfsetfillcolor{currentfill}%
\pgfsetlinewidth{0.000000pt}%
\definecolor{currentstroke}{rgb}{0.000000,0.000000,0.000000}%
\pgfsetstrokecolor{currentstroke}%
\pgfsetdash{}{0pt}%
\pgfpathmoveto{\pgfqpoint{3.372093in}{2.761754in}}%
\pgfpathlineto{\pgfqpoint{3.622205in}{2.891176in}}%
\pgfpathlineto{\pgfqpoint{3.372093in}{2.761754in}}%
\pgfpathclose%
\pgfusepath{fill}%
\end{pgfscope}%
\begin{pgfscope}%
\pgfpathrectangle{\pgfqpoint{0.017500in}{0.017500in}}{\pgfqpoint{6.218282in}{3.580372in}} %
\pgfusepath{clip}%
\pgfsetbuttcap%
\pgfsetroundjoin%
\definecolor{currentfill}{rgb}{0.717435,0.051118,0.158737}%
\pgfsetfillcolor{currentfill}%
\pgfsetlinewidth{0.000000pt}%
\definecolor{currentstroke}{rgb}{0.000000,0.000000,0.000000}%
\pgfsetstrokecolor{currentstroke}%
\pgfsetdash{}{0pt}%
\pgfpathmoveto{\pgfqpoint{3.372093in}{2.761754in}}%
\pgfpathlineto{\pgfqpoint{3.622205in}{2.891176in}}%
\pgfpathlineto{\pgfqpoint{3.372093in}{2.761754in}}%
\pgfpathclose%
\pgfusepath{fill}%
\end{pgfscope}%
\begin{pgfscope}%
\pgfpathrectangle{\pgfqpoint{0.017500in}{0.017500in}}{\pgfqpoint{6.218282in}{3.580372in}} %
\pgfusepath{clip}%
\pgfsetbuttcap%
\pgfsetroundjoin%
\definecolor{currentfill}{rgb}{0.717435,0.051118,0.158737}%
\pgfsetfillcolor{currentfill}%
\pgfsetlinewidth{0.000000pt}%
\definecolor{currentstroke}{rgb}{0.000000,0.000000,0.000000}%
\pgfsetstrokecolor{currentstroke}%
\pgfsetdash{}{0pt}%
\pgfpathmoveto{\pgfqpoint{3.372093in}{2.761754in}}%
\pgfpathlineto{\pgfqpoint{3.622205in}{2.891176in}}%
\pgfpathlineto{\pgfqpoint{3.372093in}{2.761754in}}%
\pgfpathclose%
\pgfusepath{fill}%
\end{pgfscope}%
\begin{pgfscope}%
\pgfpathrectangle{\pgfqpoint{0.017500in}{0.017500in}}{\pgfqpoint{6.218282in}{3.580372in}} %
\pgfusepath{clip}%
\pgfsetbuttcap%
\pgfsetroundjoin%
\definecolor{currentfill}{rgb}{0.717435,0.051118,0.158737}%
\pgfsetfillcolor{currentfill}%
\pgfsetlinewidth{0.000000pt}%
\definecolor{currentstroke}{rgb}{0.000000,0.000000,0.000000}%
\pgfsetstrokecolor{currentstroke}%
\pgfsetdash{}{0pt}%
\pgfpathmoveto{\pgfqpoint{3.372093in}{2.761754in}}%
\pgfpathlineto{\pgfqpoint{3.622205in}{2.891176in}}%
\pgfpathlineto{\pgfqpoint{3.372093in}{2.761754in}}%
\pgfpathclose%
\pgfusepath{fill}%
\end{pgfscope}%
\begin{pgfscope}%
\pgfpathrectangle{\pgfqpoint{0.017500in}{0.017500in}}{\pgfqpoint{6.218282in}{3.580372in}} %
\pgfusepath{clip}%
\pgfsetbuttcap%
\pgfsetroundjoin%
\definecolor{currentfill}{rgb}{0.717435,0.051118,0.158737}%
\pgfsetfillcolor{currentfill}%
\pgfsetlinewidth{0.000000pt}%
\definecolor{currentstroke}{rgb}{0.000000,0.000000,0.000000}%
\pgfsetstrokecolor{currentstroke}%
\pgfsetdash{}{0pt}%
\pgfpathmoveto{\pgfqpoint{3.372093in}{2.761754in}}%
\pgfpathlineto{\pgfqpoint{3.622205in}{2.891176in}}%
\pgfpathlineto{\pgfqpoint{3.372093in}{2.761754in}}%
\pgfpathclose%
\pgfusepath{fill}%
\end{pgfscope}%
\begin{pgfscope}%
\pgfpathrectangle{\pgfqpoint{0.017500in}{0.017500in}}{\pgfqpoint{6.218282in}{3.580372in}} %
\pgfusepath{clip}%
\pgfsetbuttcap%
\pgfsetroundjoin%
\definecolor{currentfill}{rgb}{0.717435,0.051118,0.158737}%
\pgfsetfillcolor{currentfill}%
\pgfsetlinewidth{0.000000pt}%
\definecolor{currentstroke}{rgb}{0.000000,0.000000,0.000000}%
\pgfsetstrokecolor{currentstroke}%
\pgfsetdash{}{0pt}%
\pgfpathmoveto{\pgfqpoint{3.372093in}{2.761754in}}%
\pgfpathlineto{\pgfqpoint{3.622205in}{2.891176in}}%
\pgfpathlineto{\pgfqpoint{3.372093in}{2.761754in}}%
\pgfpathclose%
\pgfusepath{fill}%
\end{pgfscope}%
\begin{pgfscope}%
\pgfpathrectangle{\pgfqpoint{0.017500in}{0.017500in}}{\pgfqpoint{6.218282in}{3.580372in}} %
\pgfusepath{clip}%
\pgfsetbuttcap%
\pgfsetroundjoin%
\definecolor{currentfill}{rgb}{0.717435,0.051118,0.158737}%
\pgfsetfillcolor{currentfill}%
\pgfsetlinewidth{0.000000pt}%
\definecolor{currentstroke}{rgb}{0.000000,0.000000,0.000000}%
\pgfsetstrokecolor{currentstroke}%
\pgfsetdash{}{0pt}%
\pgfpathmoveto{\pgfqpoint{3.372093in}{2.761754in}}%
\pgfpathlineto{\pgfqpoint{3.622205in}{2.891176in}}%
\pgfpathlineto{\pgfqpoint{3.372093in}{2.761754in}}%
\pgfpathclose%
\pgfusepath{fill}%
\end{pgfscope}%
\begin{pgfscope}%
\pgfpathrectangle{\pgfqpoint{0.017500in}{0.017500in}}{\pgfqpoint{6.218282in}{3.580372in}} %
\pgfusepath{clip}%
\pgfsetbuttcap%
\pgfsetroundjoin%
\definecolor{currentfill}{rgb}{0.800601,0.850358,0.930008}%
\pgfsetfillcolor{currentfill}%
\pgfsetlinewidth{0.000000pt}%
\definecolor{currentstroke}{rgb}{0.000000,0.000000,0.000000}%
\pgfsetstrokecolor{currentstroke}%
\pgfsetdash{}{0pt}%
\pgfpathmoveto{\pgfqpoint{3.622205in}{2.891176in}}%
\pgfpathlineto{\pgfqpoint{4.063701in}{2.813820in}}%
\pgfpathlineto{\pgfqpoint{2.789950in}{0.798346in}}%
\pgfpathlineto{\pgfqpoint{2.349853in}{0.872712in}}%
\pgfpathlineto{\pgfqpoint{3.622205in}{2.891176in}}%
\pgfpathclose%
\pgfusepath{fill}%
\end{pgfscope}%
\begin{pgfscope}%
\pgfpathrectangle{\pgfqpoint{0.017500in}{0.017500in}}{\pgfqpoint{6.218282in}{3.580372in}} %
\pgfusepath{clip}%
\pgfsetbuttcap%
\pgfsetroundjoin%
\definecolor{currentfill}{rgb}{0.800601,0.850358,0.930008}%
\pgfsetfillcolor{currentfill}%
\pgfsetlinewidth{0.000000pt}%
\definecolor{currentstroke}{rgb}{0.000000,0.000000,0.000000}%
\pgfsetstrokecolor{currentstroke}%
\pgfsetdash{}{0pt}%
\pgfpathmoveto{\pgfqpoint{3.622205in}{2.891176in}}%
\pgfpathlineto{\pgfqpoint{4.063701in}{2.813820in}}%
\pgfpathlineto{\pgfqpoint{2.789950in}{0.798346in}}%
\pgfpathlineto{\pgfqpoint{2.349853in}{0.872712in}}%
\pgfpathlineto{\pgfqpoint{3.622205in}{2.891176in}}%
\pgfpathclose%
\pgfusepath{fill}%
\end{pgfscope}%
\begin{pgfscope}%
\pgfpathrectangle{\pgfqpoint{0.017500in}{0.017500in}}{\pgfqpoint{6.218282in}{3.580372in}} %
\pgfusepath{clip}%
\pgfsetbuttcap%
\pgfsetroundjoin%
\definecolor{currentfill}{rgb}{0.800601,0.850358,0.930008}%
\pgfsetfillcolor{currentfill}%
\pgfsetlinewidth{0.000000pt}%
\definecolor{currentstroke}{rgb}{0.000000,0.000000,0.000000}%
\pgfsetstrokecolor{currentstroke}%
\pgfsetdash{}{0pt}%
\pgfpathmoveto{\pgfqpoint{3.622205in}{2.891176in}}%
\pgfpathlineto{\pgfqpoint{4.063701in}{2.813820in}}%
\pgfpathlineto{\pgfqpoint{2.789950in}{0.798346in}}%
\pgfpathlineto{\pgfqpoint{2.349853in}{0.872712in}}%
\pgfpathlineto{\pgfqpoint{3.622205in}{2.891176in}}%
\pgfpathclose%
\pgfusepath{fill}%
\end{pgfscope}%
\begin{pgfscope}%
\pgfpathrectangle{\pgfqpoint{0.017500in}{0.017500in}}{\pgfqpoint{6.218282in}{3.580372in}} %
\pgfusepath{clip}%
\pgfsetbuttcap%
\pgfsetroundjoin%
\definecolor{currentfill}{rgb}{0.800601,0.850358,0.930008}%
\pgfsetfillcolor{currentfill}%
\pgfsetlinewidth{0.000000pt}%
\definecolor{currentstroke}{rgb}{0.000000,0.000000,0.000000}%
\pgfsetstrokecolor{currentstroke}%
\pgfsetdash{}{0pt}%
\pgfpathmoveto{\pgfqpoint{3.622205in}{2.891176in}}%
\pgfpathlineto{\pgfqpoint{4.063701in}{2.813820in}}%
\pgfpathlineto{\pgfqpoint{2.789950in}{0.798346in}}%
\pgfpathlineto{\pgfqpoint{2.349853in}{0.872712in}}%
\pgfpathlineto{\pgfqpoint{3.622205in}{2.891176in}}%
\pgfpathclose%
\pgfusepath{fill}%
\end{pgfscope}%
\begin{pgfscope}%
\pgfpathrectangle{\pgfqpoint{0.017500in}{0.017500in}}{\pgfqpoint{6.218282in}{3.580372in}} %
\pgfusepath{clip}%
\pgfsetbuttcap%
\pgfsetroundjoin%
\definecolor{currentfill}{rgb}{0.800601,0.850358,0.930008}%
\pgfsetfillcolor{currentfill}%
\pgfsetlinewidth{0.000000pt}%
\definecolor{currentstroke}{rgb}{0.000000,0.000000,0.000000}%
\pgfsetstrokecolor{currentstroke}%
\pgfsetdash{}{0pt}%
\pgfpathmoveto{\pgfqpoint{3.622205in}{2.891176in}}%
\pgfpathlineto{\pgfqpoint{4.063701in}{2.813820in}}%
\pgfpathlineto{\pgfqpoint{2.789950in}{0.798346in}}%
\pgfpathlineto{\pgfqpoint{2.349853in}{0.872712in}}%
\pgfpathlineto{\pgfqpoint{3.622205in}{2.891176in}}%
\pgfpathclose%
\pgfusepath{fill}%
\end{pgfscope}%
\begin{pgfscope}%
\pgfpathrectangle{\pgfqpoint{0.017500in}{0.017500in}}{\pgfqpoint{6.218282in}{3.580372in}} %
\pgfusepath{clip}%
\pgfsetbuttcap%
\pgfsetroundjoin%
\definecolor{currentfill}{rgb}{0.723315,0.068898,0.162989}%
\pgfsetfillcolor{currentfill}%
\pgfsetlinewidth{0.000000pt}%
\definecolor{currentstroke}{rgb}{0.000000,0.000000,0.000000}%
\pgfsetstrokecolor{currentstroke}%
\pgfsetdash{}{0pt}%
\pgfpathmoveto{\pgfqpoint{3.372093in}{2.761754in}}%
\pgfpathlineto{\pgfqpoint{3.816789in}{2.677291in}}%
\pgfpathlineto{\pgfqpoint{4.063701in}{2.813820in}}%
\pgfpathlineto{\pgfqpoint{3.622205in}{2.891176in}}%
\pgfpathlineto{\pgfqpoint{3.372093in}{2.761754in}}%
\pgfpathclose%
\pgfusepath{fill}%
\end{pgfscope}%
\begin{pgfscope}%
\pgfpathrectangle{\pgfqpoint{0.017500in}{0.017500in}}{\pgfqpoint{6.218282in}{3.580372in}} %
\pgfusepath{clip}%
\pgfsetbuttcap%
\pgfsetroundjoin%
\definecolor{currentfill}{rgb}{0.723315,0.068898,0.162989}%
\pgfsetfillcolor{currentfill}%
\pgfsetlinewidth{0.000000pt}%
\definecolor{currentstroke}{rgb}{0.000000,0.000000,0.000000}%
\pgfsetstrokecolor{currentstroke}%
\pgfsetdash{}{0pt}%
\pgfpathmoveto{\pgfqpoint{3.372093in}{2.761754in}}%
\pgfpathlineto{\pgfqpoint{3.816789in}{2.677291in}}%
\pgfpathlineto{\pgfqpoint{4.063701in}{2.813820in}}%
\pgfpathlineto{\pgfqpoint{3.622205in}{2.891176in}}%
\pgfpathlineto{\pgfqpoint{3.372093in}{2.761754in}}%
\pgfpathclose%
\pgfusepath{fill}%
\end{pgfscope}%
\begin{pgfscope}%
\pgfpathrectangle{\pgfqpoint{0.017500in}{0.017500in}}{\pgfqpoint{6.218282in}{3.580372in}} %
\pgfusepath{clip}%
\pgfsetbuttcap%
\pgfsetroundjoin%
\definecolor{currentfill}{rgb}{0.723315,0.068898,0.162989}%
\pgfsetfillcolor{currentfill}%
\pgfsetlinewidth{0.000000pt}%
\definecolor{currentstroke}{rgb}{0.000000,0.000000,0.000000}%
\pgfsetstrokecolor{currentstroke}%
\pgfsetdash{}{0pt}%
\pgfpathmoveto{\pgfqpoint{3.372093in}{2.761754in}}%
\pgfpathlineto{\pgfqpoint{3.816789in}{2.677291in}}%
\pgfpathlineto{\pgfqpoint{4.063701in}{2.813820in}}%
\pgfpathlineto{\pgfqpoint{3.622205in}{2.891176in}}%
\pgfpathlineto{\pgfqpoint{3.372093in}{2.761754in}}%
\pgfpathclose%
\pgfusepath{fill}%
\end{pgfscope}%
\begin{pgfscope}%
\pgfpathrectangle{\pgfqpoint{0.017500in}{0.017500in}}{\pgfqpoint{6.218282in}{3.580372in}} %
\pgfusepath{clip}%
\pgfsetbuttcap%
\pgfsetroundjoin%
\definecolor{currentfill}{rgb}{0.723315,0.068898,0.162989}%
\pgfsetfillcolor{currentfill}%
\pgfsetlinewidth{0.000000pt}%
\definecolor{currentstroke}{rgb}{0.000000,0.000000,0.000000}%
\pgfsetstrokecolor{currentstroke}%
\pgfsetdash{}{0pt}%
\pgfpathmoveto{\pgfqpoint{3.372093in}{2.761754in}}%
\pgfpathlineto{\pgfqpoint{3.816789in}{2.677291in}}%
\pgfpathlineto{\pgfqpoint{4.063701in}{2.813820in}}%
\pgfpathlineto{\pgfqpoint{3.622205in}{2.891176in}}%
\pgfpathlineto{\pgfqpoint{3.372093in}{2.761754in}}%
\pgfpathclose%
\pgfusepath{fill}%
\end{pgfscope}%
\begin{pgfscope}%
\pgfpathrectangle{\pgfqpoint{0.017500in}{0.017500in}}{\pgfqpoint{6.218282in}{3.580372in}} %
\pgfusepath{clip}%
\pgfsetbuttcap%
\pgfsetroundjoin%
\definecolor{currentfill}{rgb}{0.723315,0.068898,0.162989}%
\pgfsetfillcolor{currentfill}%
\pgfsetlinewidth{0.000000pt}%
\definecolor{currentstroke}{rgb}{0.000000,0.000000,0.000000}%
\pgfsetstrokecolor{currentstroke}%
\pgfsetdash{}{0pt}%
\pgfpathmoveto{\pgfqpoint{3.372093in}{2.761754in}}%
\pgfpathlineto{\pgfqpoint{3.816789in}{2.677291in}}%
\pgfpathlineto{\pgfqpoint{4.063701in}{2.813820in}}%
\pgfpathlineto{\pgfqpoint{3.622205in}{2.891176in}}%
\pgfpathlineto{\pgfqpoint{3.372093in}{2.761754in}}%
\pgfpathclose%
\pgfusepath{fill}%
\end{pgfscope}%
\begin{pgfscope}%
\pgfpathrectangle{\pgfqpoint{0.017500in}{0.017500in}}{\pgfqpoint{6.218282in}{3.580372in}} %
\pgfusepath{clip}%
\pgfsetbuttcap%
\pgfsetroundjoin%
\definecolor{currentfill}{rgb}{0.723315,0.068898,0.162989}%
\pgfsetfillcolor{currentfill}%
\pgfsetlinewidth{0.000000pt}%
\definecolor{currentstroke}{rgb}{0.000000,0.000000,0.000000}%
\pgfsetstrokecolor{currentstroke}%
\pgfsetdash{}{0pt}%
\pgfpathmoveto{\pgfqpoint{3.372093in}{2.761754in}}%
\pgfpathlineto{\pgfqpoint{3.816789in}{2.677291in}}%
\pgfpathlineto{\pgfqpoint{4.063701in}{2.813820in}}%
\pgfpathlineto{\pgfqpoint{3.622205in}{2.891176in}}%
\pgfpathlineto{\pgfqpoint{3.372093in}{2.761754in}}%
\pgfpathclose%
\pgfusepath{fill}%
\end{pgfscope}%
\begin{pgfscope}%
\pgfpathrectangle{\pgfqpoint{0.017500in}{0.017500in}}{\pgfqpoint{6.218282in}{3.580372in}} %
\pgfusepath{clip}%
\pgfsetbuttcap%
\pgfsetroundjoin%
\definecolor{currentfill}{rgb}{0.243520,0.319189,0.771672}%
\pgfsetfillcolor{currentfill}%
\pgfsetlinewidth{0.000000pt}%
\definecolor{currentstroke}{rgb}{0.000000,0.000000,0.000000}%
\pgfsetstrokecolor{currentstroke}%
\pgfsetdash{}{0pt}%
\pgfpathmoveto{\pgfqpoint{1.486050in}{1.042638in}}%
\pgfpathlineto{\pgfqpoint{1.701778in}{2.375902in}}%
\pgfpathlineto{\pgfqpoint{1.486050in}{1.042638in}}%
\pgfpathclose%
\pgfusepath{fill}%
\end{pgfscope}%
\begin{pgfscope}%
\pgfpathrectangle{\pgfqpoint{0.017500in}{0.017500in}}{\pgfqpoint{6.218282in}{3.580372in}} %
\pgfusepath{clip}%
\pgfsetbuttcap%
\pgfsetroundjoin%
\definecolor{currentfill}{rgb}{0.243520,0.319189,0.771672}%
\pgfsetfillcolor{currentfill}%
\pgfsetlinewidth{0.000000pt}%
\definecolor{currentstroke}{rgb}{0.000000,0.000000,0.000000}%
\pgfsetstrokecolor{currentstroke}%
\pgfsetdash{}{0pt}%
\pgfpathmoveto{\pgfqpoint{1.486050in}{1.042638in}}%
\pgfpathlineto{\pgfqpoint{1.701778in}{2.375902in}}%
\pgfpathlineto{\pgfqpoint{1.486050in}{1.042638in}}%
\pgfpathclose%
\pgfusepath{fill}%
\end{pgfscope}%
\begin{pgfscope}%
\pgfpathrectangle{\pgfqpoint{0.017500in}{0.017500in}}{\pgfqpoint{6.218282in}{3.580372in}} %
\pgfusepath{clip}%
\pgfsetbuttcap%
\pgfsetroundjoin%
\definecolor{currentfill}{rgb}{0.243520,0.319189,0.771672}%
\pgfsetfillcolor{currentfill}%
\pgfsetlinewidth{0.000000pt}%
\definecolor{currentstroke}{rgb}{0.000000,0.000000,0.000000}%
\pgfsetstrokecolor{currentstroke}%
\pgfsetdash{}{0pt}%
\pgfpathmoveto{\pgfqpoint{1.486050in}{1.042638in}}%
\pgfpathlineto{\pgfqpoint{1.701778in}{2.375902in}}%
\pgfpathlineto{\pgfqpoint{1.486050in}{1.042638in}}%
\pgfpathclose%
\pgfusepath{fill}%
\end{pgfscope}%
\begin{pgfscope}%
\pgfpathrectangle{\pgfqpoint{0.017500in}{0.017500in}}{\pgfqpoint{6.218282in}{3.580372in}} %
\pgfusepath{clip}%
\pgfsetbuttcap%
\pgfsetroundjoin%
\definecolor{currentfill}{rgb}{0.243520,0.319189,0.771672}%
\pgfsetfillcolor{currentfill}%
\pgfsetlinewidth{0.000000pt}%
\definecolor{currentstroke}{rgb}{0.000000,0.000000,0.000000}%
\pgfsetstrokecolor{currentstroke}%
\pgfsetdash{}{0pt}%
\pgfpathmoveto{\pgfqpoint{1.486050in}{1.042638in}}%
\pgfpathlineto{\pgfqpoint{1.701778in}{2.375902in}}%
\pgfpathlineto{\pgfqpoint{1.486050in}{1.042638in}}%
\pgfpathclose%
\pgfusepath{fill}%
\end{pgfscope}%
\begin{pgfscope}%
\pgfpathrectangle{\pgfqpoint{0.017500in}{0.017500in}}{\pgfqpoint{6.218282in}{3.580372in}} %
\pgfusepath{clip}%
\pgfsetbuttcap%
\pgfsetroundjoin%
\definecolor{currentfill}{rgb}{0.243520,0.319189,0.771672}%
\pgfsetfillcolor{currentfill}%
\pgfsetlinewidth{0.000000pt}%
\definecolor{currentstroke}{rgb}{0.000000,0.000000,0.000000}%
\pgfsetstrokecolor{currentstroke}%
\pgfsetdash{}{0pt}%
\pgfpathmoveto{\pgfqpoint{1.486050in}{1.042638in}}%
\pgfpathlineto{\pgfqpoint{1.701778in}{2.375902in}}%
\pgfpathlineto{\pgfqpoint{1.486050in}{1.042638in}}%
\pgfpathclose%
\pgfusepath{fill}%
\end{pgfscope}%
\begin{pgfscope}%
\pgfpathrectangle{\pgfqpoint{0.017500in}{0.017500in}}{\pgfqpoint{6.218282in}{3.580372in}} %
\pgfusepath{clip}%
\pgfsetbuttcap%
\pgfsetroundjoin%
\definecolor{currentfill}{rgb}{0.243520,0.319189,0.771672}%
\pgfsetfillcolor{currentfill}%
\pgfsetlinewidth{0.000000pt}%
\definecolor{currentstroke}{rgb}{0.000000,0.000000,0.000000}%
\pgfsetstrokecolor{currentstroke}%
\pgfsetdash{}{0pt}%
\pgfpathmoveto{\pgfqpoint{1.486050in}{1.042638in}}%
\pgfpathlineto{\pgfqpoint{1.701778in}{2.375902in}}%
\pgfpathlineto{\pgfqpoint{1.486050in}{1.042638in}}%
\pgfpathclose%
\pgfusepath{fill}%
\end{pgfscope}%
\begin{pgfscope}%
\pgfpathrectangle{\pgfqpoint{0.017500in}{0.017500in}}{\pgfqpoint{6.218282in}{3.580372in}} %
\pgfusepath{clip}%
\pgfsetbuttcap%
\pgfsetroundjoin%
\definecolor{currentfill}{rgb}{0.243520,0.319189,0.771672}%
\pgfsetfillcolor{currentfill}%
\pgfsetlinewidth{0.000000pt}%
\definecolor{currentstroke}{rgb}{0.000000,0.000000,0.000000}%
\pgfsetstrokecolor{currentstroke}%
\pgfsetdash{}{0pt}%
\pgfpathmoveto{\pgfqpoint{1.486050in}{1.042638in}}%
\pgfpathlineto{\pgfqpoint{1.701778in}{2.375902in}}%
\pgfpathlineto{\pgfqpoint{1.486050in}{1.042638in}}%
\pgfpathclose%
\pgfusepath{fill}%
\end{pgfscope}%
\begin{pgfscope}%
\pgfpathrectangle{\pgfqpoint{0.017500in}{0.017500in}}{\pgfqpoint{6.218282in}{3.580372in}} %
\pgfusepath{clip}%
\pgfsetbuttcap%
\pgfsetroundjoin%
\definecolor{currentfill}{rgb}{0.243520,0.319189,0.771672}%
\pgfsetfillcolor{currentfill}%
\pgfsetlinewidth{0.000000pt}%
\definecolor{currentstroke}{rgb}{0.000000,0.000000,0.000000}%
\pgfsetstrokecolor{currentstroke}%
\pgfsetdash{}{0pt}%
\pgfpathmoveto{\pgfqpoint{1.486050in}{1.042638in}}%
\pgfpathlineto{\pgfqpoint{1.701778in}{2.375902in}}%
\pgfpathlineto{\pgfqpoint{1.486050in}{1.042638in}}%
\pgfpathclose%
\pgfusepath{fill}%
\end{pgfscope}%
\begin{pgfscope}%
\pgfpathrectangle{\pgfqpoint{0.017500in}{0.017500in}}{\pgfqpoint{6.218282in}{3.580372in}} %
\pgfusepath{clip}%
\pgfsetbuttcap%
\pgfsetroundjoin%
\definecolor{currentfill}{rgb}{0.243520,0.319189,0.771672}%
\pgfsetfillcolor{currentfill}%
\pgfsetlinewidth{0.000000pt}%
\definecolor{currentstroke}{rgb}{0.000000,0.000000,0.000000}%
\pgfsetstrokecolor{currentstroke}%
\pgfsetdash{}{0pt}%
\pgfpathmoveto{\pgfqpoint{1.486050in}{1.042638in}}%
\pgfpathlineto{\pgfqpoint{1.701778in}{2.375902in}}%
\pgfpathlineto{\pgfqpoint{1.486050in}{1.042638in}}%
\pgfpathclose%
\pgfusepath{fill}%
\end{pgfscope}%
\begin{pgfscope}%
\pgfpathrectangle{\pgfqpoint{0.017500in}{0.017500in}}{\pgfqpoint{6.218282in}{3.580372in}} %
\pgfusepath{clip}%
\pgfsetbuttcap%
\pgfsetroundjoin%
\definecolor{currentfill}{rgb}{0.243520,0.319189,0.771672}%
\pgfsetfillcolor{currentfill}%
\pgfsetlinewidth{0.000000pt}%
\definecolor{currentstroke}{rgb}{0.000000,0.000000,0.000000}%
\pgfsetstrokecolor{currentstroke}%
\pgfsetdash{}{0pt}%
\pgfpathmoveto{\pgfqpoint{1.486050in}{1.042638in}}%
\pgfpathlineto{\pgfqpoint{1.701778in}{2.375902in}}%
\pgfpathlineto{\pgfqpoint{1.486050in}{1.042638in}}%
\pgfpathclose%
\pgfusepath{fill}%
\end{pgfscope}%
\begin{pgfscope}%
\pgfpathrectangle{\pgfqpoint{0.017500in}{0.017500in}}{\pgfqpoint{6.218282in}{3.580372in}} %
\pgfusepath{clip}%
\pgfsetbuttcap%
\pgfsetroundjoin%
\definecolor{currentfill}{rgb}{0.243520,0.319189,0.771672}%
\pgfsetfillcolor{currentfill}%
\pgfsetlinewidth{0.000000pt}%
\definecolor{currentstroke}{rgb}{0.000000,0.000000,0.000000}%
\pgfsetstrokecolor{currentstroke}%
\pgfsetdash{}{0pt}%
\pgfpathmoveto{\pgfqpoint{1.486050in}{1.042638in}}%
\pgfpathlineto{\pgfqpoint{1.701778in}{2.375902in}}%
\pgfpathlineto{\pgfqpoint{1.486050in}{1.042638in}}%
\pgfpathclose%
\pgfusepath{fill}%
\end{pgfscope}%
\begin{pgfscope}%
\pgfpathrectangle{\pgfqpoint{0.017500in}{0.017500in}}{\pgfqpoint{6.218282in}{3.580372in}} %
\pgfusepath{clip}%
\pgfsetbuttcap%
\pgfsetroundjoin%
\definecolor{currentfill}{rgb}{0.243520,0.319189,0.771672}%
\pgfsetfillcolor{currentfill}%
\pgfsetlinewidth{0.000000pt}%
\definecolor{currentstroke}{rgb}{0.000000,0.000000,0.000000}%
\pgfsetstrokecolor{currentstroke}%
\pgfsetdash{}{0pt}%
\pgfpathmoveto{\pgfqpoint{1.486050in}{1.042638in}}%
\pgfpathlineto{\pgfqpoint{1.701778in}{2.375902in}}%
\pgfpathlineto{\pgfqpoint{1.486050in}{1.042638in}}%
\pgfpathclose%
\pgfusepath{fill}%
\end{pgfscope}%
\begin{pgfscope}%
\pgfpathrectangle{\pgfqpoint{0.017500in}{0.017500in}}{\pgfqpoint{6.218282in}{3.580372in}} %
\pgfusepath{clip}%
\pgfsetbuttcap%
\pgfsetroundjoin%
\definecolor{currentfill}{rgb}{0.243520,0.319189,0.771672}%
\pgfsetfillcolor{currentfill}%
\pgfsetlinewidth{0.000000pt}%
\definecolor{currentstroke}{rgb}{0.000000,0.000000,0.000000}%
\pgfsetstrokecolor{currentstroke}%
\pgfsetdash{}{0pt}%
\pgfpathmoveto{\pgfqpoint{1.486050in}{1.042638in}}%
\pgfpathlineto{\pgfqpoint{1.701778in}{2.375902in}}%
\pgfpathlineto{\pgfqpoint{1.486050in}{1.042638in}}%
\pgfpathclose%
\pgfusepath{fill}%
\end{pgfscope}%
\begin{pgfscope}%
\pgfpathrectangle{\pgfqpoint{0.017500in}{0.017500in}}{\pgfqpoint{6.218282in}{3.580372in}} %
\pgfusepath{clip}%
\pgfsetbuttcap%
\pgfsetroundjoin%
\definecolor{currentfill}{rgb}{0.243520,0.319189,0.771672}%
\pgfsetfillcolor{currentfill}%
\pgfsetlinewidth{0.000000pt}%
\definecolor{currentstroke}{rgb}{0.000000,0.000000,0.000000}%
\pgfsetstrokecolor{currentstroke}%
\pgfsetdash{}{0pt}%
\pgfpathmoveto{\pgfqpoint{1.486050in}{1.042638in}}%
\pgfpathlineto{\pgfqpoint{1.701778in}{2.375902in}}%
\pgfpathlineto{\pgfqpoint{1.486050in}{1.042638in}}%
\pgfpathclose%
\pgfusepath{fill}%
\end{pgfscope}%
\begin{pgfscope}%
\pgfpathrectangle{\pgfqpoint{0.017500in}{0.017500in}}{\pgfqpoint{6.218282in}{3.580372in}} %
\pgfusepath{clip}%
\pgfsetbuttcap%
\pgfsetroundjoin%
\definecolor{currentfill}{rgb}{0.243520,0.319189,0.771672}%
\pgfsetfillcolor{currentfill}%
\pgfsetlinewidth{0.000000pt}%
\definecolor{currentstroke}{rgb}{0.000000,0.000000,0.000000}%
\pgfsetstrokecolor{currentstroke}%
\pgfsetdash{}{0pt}%
\pgfpathmoveto{\pgfqpoint{1.486050in}{1.042638in}}%
\pgfpathlineto{\pgfqpoint{1.701778in}{2.375902in}}%
\pgfpathlineto{\pgfqpoint{1.486050in}{1.042638in}}%
\pgfpathclose%
\pgfusepath{fill}%
\end{pgfscope}%
\begin{pgfscope}%
\pgfpathrectangle{\pgfqpoint{0.017500in}{0.017500in}}{\pgfqpoint{6.218282in}{3.580372in}} %
\pgfusepath{clip}%
\pgfsetbuttcap%
\pgfsetroundjoin%
\definecolor{currentfill}{rgb}{0.243520,0.319189,0.771672}%
\pgfsetfillcolor{currentfill}%
\pgfsetlinewidth{0.000000pt}%
\definecolor{currentstroke}{rgb}{0.000000,0.000000,0.000000}%
\pgfsetstrokecolor{currentstroke}%
\pgfsetdash{}{0pt}%
\pgfpathmoveto{\pgfqpoint{1.486050in}{1.042638in}}%
\pgfpathlineto{\pgfqpoint{1.701778in}{2.375902in}}%
\pgfpathlineto{\pgfqpoint{1.486050in}{1.042638in}}%
\pgfpathclose%
\pgfusepath{fill}%
\end{pgfscope}%
\begin{pgfscope}%
\pgfpathrectangle{\pgfqpoint{0.017500in}{0.017500in}}{\pgfqpoint{6.218282in}{3.580372in}} %
\pgfusepath{clip}%
\pgfsetbuttcap%
\pgfsetroundjoin%
\definecolor{currentfill}{rgb}{0.243520,0.319189,0.771672}%
\pgfsetfillcolor{currentfill}%
\pgfsetlinewidth{0.000000pt}%
\definecolor{currentstroke}{rgb}{0.000000,0.000000,0.000000}%
\pgfsetstrokecolor{currentstroke}%
\pgfsetdash{}{0pt}%
\pgfpathmoveto{\pgfqpoint{1.486050in}{1.042638in}}%
\pgfpathlineto{\pgfqpoint{1.701778in}{2.375902in}}%
\pgfpathlineto{\pgfqpoint{1.486050in}{1.042638in}}%
\pgfpathclose%
\pgfusepath{fill}%
\end{pgfscope}%
\begin{pgfscope}%
\pgfpathrectangle{\pgfqpoint{0.017500in}{0.017500in}}{\pgfqpoint{6.218282in}{3.580372in}} %
\pgfusepath{clip}%
\pgfsetbuttcap%
\pgfsetroundjoin%
\definecolor{currentfill}{rgb}{0.243520,0.319189,0.771672}%
\pgfsetfillcolor{currentfill}%
\pgfsetlinewidth{0.000000pt}%
\definecolor{currentstroke}{rgb}{0.000000,0.000000,0.000000}%
\pgfsetstrokecolor{currentstroke}%
\pgfsetdash{}{0pt}%
\pgfpathmoveto{\pgfqpoint{1.486050in}{1.042638in}}%
\pgfpathlineto{\pgfqpoint{1.701778in}{2.375902in}}%
\pgfpathlineto{\pgfqpoint{1.486050in}{1.042638in}}%
\pgfpathclose%
\pgfusepath{fill}%
\end{pgfscope}%
\begin{pgfscope}%
\pgfpathrectangle{\pgfqpoint{0.017500in}{0.017500in}}{\pgfqpoint{6.218282in}{3.580372in}} %
\pgfusepath{clip}%
\pgfsetbuttcap%
\pgfsetroundjoin%
\definecolor{currentfill}{rgb}{0.243520,0.319189,0.771672}%
\pgfsetfillcolor{currentfill}%
\pgfsetlinewidth{0.000000pt}%
\definecolor{currentstroke}{rgb}{0.000000,0.000000,0.000000}%
\pgfsetstrokecolor{currentstroke}%
\pgfsetdash{}{0pt}%
\pgfpathmoveto{\pgfqpoint{1.486050in}{1.042638in}}%
\pgfpathlineto{\pgfqpoint{1.701778in}{2.375902in}}%
\pgfpathlineto{\pgfqpoint{1.486050in}{1.042638in}}%
\pgfpathclose%
\pgfusepath{fill}%
\end{pgfscope}%
\begin{pgfscope}%
\pgfpathrectangle{\pgfqpoint{0.017500in}{0.017500in}}{\pgfqpoint{6.218282in}{3.580372in}} %
\pgfusepath{clip}%
\pgfsetbuttcap%
\pgfsetroundjoin%
\definecolor{currentfill}{rgb}{0.243520,0.319189,0.771672}%
\pgfsetfillcolor{currentfill}%
\pgfsetlinewidth{0.000000pt}%
\definecolor{currentstroke}{rgb}{0.000000,0.000000,0.000000}%
\pgfsetstrokecolor{currentstroke}%
\pgfsetdash{}{0pt}%
\pgfpathmoveto{\pgfqpoint{1.486050in}{1.042638in}}%
\pgfpathlineto{\pgfqpoint{1.701778in}{2.375902in}}%
\pgfpathlineto{\pgfqpoint{1.486050in}{1.042638in}}%
\pgfpathclose%
\pgfusepath{fill}%
\end{pgfscope}%
\begin{pgfscope}%
\pgfpathrectangle{\pgfqpoint{0.017500in}{0.017500in}}{\pgfqpoint{6.218282in}{3.580372in}} %
\pgfusepath{clip}%
\pgfsetbuttcap%
\pgfsetroundjoin%
\definecolor{currentfill}{rgb}{0.243520,0.319189,0.771672}%
\pgfsetfillcolor{currentfill}%
\pgfsetlinewidth{0.000000pt}%
\definecolor{currentstroke}{rgb}{0.000000,0.000000,0.000000}%
\pgfsetstrokecolor{currentstroke}%
\pgfsetdash{}{0pt}%
\pgfpathmoveto{\pgfqpoint{1.486050in}{1.042638in}}%
\pgfpathlineto{\pgfqpoint{1.701778in}{2.375902in}}%
\pgfpathlineto{\pgfqpoint{1.486050in}{1.042638in}}%
\pgfpathclose%
\pgfusepath{fill}%
\end{pgfscope}%
\begin{pgfscope}%
\pgfpathrectangle{\pgfqpoint{0.017500in}{0.017500in}}{\pgfqpoint{6.218282in}{3.580372in}} %
\pgfusepath{clip}%
\pgfsetbuttcap%
\pgfsetroundjoin%
\definecolor{currentfill}{rgb}{0.243520,0.319189,0.771672}%
\pgfsetfillcolor{currentfill}%
\pgfsetlinewidth{0.000000pt}%
\definecolor{currentstroke}{rgb}{0.000000,0.000000,0.000000}%
\pgfsetstrokecolor{currentstroke}%
\pgfsetdash{}{0pt}%
\pgfpathmoveto{\pgfqpoint{1.486050in}{1.042638in}}%
\pgfpathlineto{\pgfqpoint{1.701778in}{2.375902in}}%
\pgfpathlineto{\pgfqpoint{1.486050in}{1.042638in}}%
\pgfpathclose%
\pgfusepath{fill}%
\end{pgfscope}%
\begin{pgfscope}%
\pgfpathrectangle{\pgfqpoint{0.017500in}{0.017500in}}{\pgfqpoint{6.218282in}{3.580372in}} %
\pgfusepath{clip}%
\pgfsetbuttcap%
\pgfsetroundjoin%
\definecolor{currentfill}{rgb}{0.243520,0.319189,0.771672}%
\pgfsetfillcolor{currentfill}%
\pgfsetlinewidth{0.000000pt}%
\definecolor{currentstroke}{rgb}{0.000000,0.000000,0.000000}%
\pgfsetstrokecolor{currentstroke}%
\pgfsetdash{}{0pt}%
\pgfpathmoveto{\pgfqpoint{1.486050in}{1.042638in}}%
\pgfpathlineto{\pgfqpoint{1.701778in}{2.375902in}}%
\pgfpathlineto{\pgfqpoint{1.486050in}{1.042638in}}%
\pgfpathclose%
\pgfusepath{fill}%
\end{pgfscope}%
\begin{pgfscope}%
\pgfpathrectangle{\pgfqpoint{0.017500in}{0.017500in}}{\pgfqpoint{6.218282in}{3.580372in}} %
\pgfusepath{clip}%
\pgfsetbuttcap%
\pgfsetroundjoin%
\definecolor{currentfill}{rgb}{0.243520,0.319189,0.771672}%
\pgfsetfillcolor{currentfill}%
\pgfsetlinewidth{0.000000pt}%
\definecolor{currentstroke}{rgb}{0.000000,0.000000,0.000000}%
\pgfsetstrokecolor{currentstroke}%
\pgfsetdash{}{0pt}%
\pgfpathmoveto{\pgfqpoint{1.486050in}{1.042638in}}%
\pgfpathlineto{\pgfqpoint{1.701778in}{2.375902in}}%
\pgfpathlineto{\pgfqpoint{1.486050in}{1.042638in}}%
\pgfpathclose%
\pgfusepath{fill}%
\end{pgfscope}%
\begin{pgfscope}%
\pgfpathrectangle{\pgfqpoint{0.017500in}{0.017500in}}{\pgfqpoint{6.218282in}{3.580372in}} %
\pgfusepath{clip}%
\pgfsetbuttcap%
\pgfsetroundjoin%
\definecolor{currentfill}{rgb}{0.243520,0.319189,0.771672}%
\pgfsetfillcolor{currentfill}%
\pgfsetlinewidth{0.000000pt}%
\definecolor{currentstroke}{rgb}{0.000000,0.000000,0.000000}%
\pgfsetstrokecolor{currentstroke}%
\pgfsetdash{}{0pt}%
\pgfpathmoveto{\pgfqpoint{1.486050in}{1.042638in}}%
\pgfpathlineto{\pgfqpoint{1.701778in}{2.375902in}}%
\pgfpathlineto{\pgfqpoint{1.486050in}{1.042638in}}%
\pgfpathclose%
\pgfusepath{fill}%
\end{pgfscope}%
\begin{pgfscope}%
\pgfpathrectangle{\pgfqpoint{0.017500in}{0.017500in}}{\pgfqpoint{6.218282in}{3.580372in}} %
\pgfusepath{clip}%
\pgfsetbuttcap%
\pgfsetroundjoin%
\definecolor{currentfill}{rgb}{0.243520,0.319189,0.771672}%
\pgfsetfillcolor{currentfill}%
\pgfsetlinewidth{0.000000pt}%
\definecolor{currentstroke}{rgb}{0.000000,0.000000,0.000000}%
\pgfsetstrokecolor{currentstroke}%
\pgfsetdash{}{0pt}%
\pgfpathmoveto{\pgfqpoint{1.486050in}{1.042638in}}%
\pgfpathlineto{\pgfqpoint{1.701778in}{2.375902in}}%
\pgfpathlineto{\pgfqpoint{1.486050in}{1.042638in}}%
\pgfpathclose%
\pgfusepath{fill}%
\end{pgfscope}%
\begin{pgfscope}%
\pgfpathrectangle{\pgfqpoint{0.017500in}{0.017500in}}{\pgfqpoint{6.218282in}{3.580372in}} %
\pgfusepath{clip}%
\pgfsetbuttcap%
\pgfsetroundjoin%
\definecolor{currentfill}{rgb}{0.243520,0.319189,0.771672}%
\pgfsetfillcolor{currentfill}%
\pgfsetlinewidth{0.000000pt}%
\definecolor{currentstroke}{rgb}{0.000000,0.000000,0.000000}%
\pgfsetstrokecolor{currentstroke}%
\pgfsetdash{}{0pt}%
\pgfpathmoveto{\pgfqpoint{1.486050in}{1.042638in}}%
\pgfpathlineto{\pgfqpoint{1.701778in}{2.375902in}}%
\pgfpathlineto{\pgfqpoint{1.486050in}{1.042638in}}%
\pgfpathclose%
\pgfusepath{fill}%
\end{pgfscope}%
\begin{pgfscope}%
\pgfpathrectangle{\pgfqpoint{0.017500in}{0.017500in}}{\pgfqpoint{6.218282in}{3.580372in}} %
\pgfusepath{clip}%
\pgfsetbuttcap%
\pgfsetroundjoin%
\definecolor{currentfill}{rgb}{0.243520,0.319189,0.771672}%
\pgfsetfillcolor{currentfill}%
\pgfsetlinewidth{0.000000pt}%
\definecolor{currentstroke}{rgb}{0.000000,0.000000,0.000000}%
\pgfsetstrokecolor{currentstroke}%
\pgfsetdash{}{0pt}%
\pgfpathmoveto{\pgfqpoint{1.486050in}{1.042638in}}%
\pgfpathlineto{\pgfqpoint{1.701778in}{2.375902in}}%
\pgfpathlineto{\pgfqpoint{1.486050in}{1.042638in}}%
\pgfpathclose%
\pgfusepath{fill}%
\end{pgfscope}%
\begin{pgfscope}%
\pgfpathrectangle{\pgfqpoint{0.017500in}{0.017500in}}{\pgfqpoint{6.218282in}{3.580372in}} %
\pgfusepath{clip}%
\pgfsetbuttcap%
\pgfsetroundjoin%
\definecolor{currentfill}{rgb}{0.243520,0.319189,0.771672}%
\pgfsetfillcolor{currentfill}%
\pgfsetlinewidth{0.000000pt}%
\definecolor{currentstroke}{rgb}{0.000000,0.000000,0.000000}%
\pgfsetstrokecolor{currentstroke}%
\pgfsetdash{}{0pt}%
\pgfpathmoveto{\pgfqpoint{1.486050in}{1.042638in}}%
\pgfpathlineto{\pgfqpoint{1.701778in}{2.375902in}}%
\pgfpathlineto{\pgfqpoint{1.486050in}{1.042638in}}%
\pgfpathclose%
\pgfusepath{fill}%
\end{pgfscope}%
\begin{pgfscope}%
\pgfpathrectangle{\pgfqpoint{0.017500in}{0.017500in}}{\pgfqpoint{6.218282in}{3.580372in}} %
\pgfusepath{clip}%
\pgfsetbuttcap%
\pgfsetroundjoin%
\definecolor{currentfill}{rgb}{0.243520,0.319189,0.771672}%
\pgfsetfillcolor{currentfill}%
\pgfsetlinewidth{0.000000pt}%
\definecolor{currentstroke}{rgb}{0.000000,0.000000,0.000000}%
\pgfsetstrokecolor{currentstroke}%
\pgfsetdash{}{0pt}%
\pgfpathmoveto{\pgfqpoint{1.486050in}{1.042638in}}%
\pgfpathlineto{\pgfqpoint{1.701778in}{2.375902in}}%
\pgfpathlineto{\pgfqpoint{1.486050in}{1.042638in}}%
\pgfpathclose%
\pgfusepath{fill}%
\end{pgfscope}%
\begin{pgfscope}%
\pgfpathrectangle{\pgfqpoint{0.017500in}{0.017500in}}{\pgfqpoint{6.218282in}{3.580372in}} %
\pgfusepath{clip}%
\pgfsetbuttcap%
\pgfsetroundjoin%
\definecolor{currentfill}{rgb}{0.746838,0.140021,0.179996}%
\pgfsetfillcolor{currentfill}%
\pgfsetlinewidth{0.000000pt}%
\definecolor{currentstroke}{rgb}{0.000000,0.000000,0.000000}%
\pgfsetstrokecolor{currentstroke}%
\pgfsetdash{}{0pt}%
\pgfpathmoveto{\pgfqpoint{2.674254in}{2.688576in}}%
\pgfpathlineto{\pgfqpoint{2.932968in}{2.820668in}}%
\pgfpathlineto{\pgfqpoint{2.674254in}{2.688576in}}%
\pgfpathclose%
\pgfusepath{fill}%
\end{pgfscope}%
\begin{pgfscope}%
\pgfpathrectangle{\pgfqpoint{0.017500in}{0.017500in}}{\pgfqpoint{6.218282in}{3.580372in}} %
\pgfusepath{clip}%
\pgfsetbuttcap%
\pgfsetroundjoin%
\definecolor{currentfill}{rgb}{0.746838,0.140021,0.179996}%
\pgfsetfillcolor{currentfill}%
\pgfsetlinewidth{0.000000pt}%
\definecolor{currentstroke}{rgb}{0.000000,0.000000,0.000000}%
\pgfsetstrokecolor{currentstroke}%
\pgfsetdash{}{0pt}%
\pgfpathmoveto{\pgfqpoint{2.674254in}{2.688576in}}%
\pgfpathlineto{\pgfqpoint{2.932968in}{2.820668in}}%
\pgfpathlineto{\pgfqpoint{2.674254in}{2.688576in}}%
\pgfpathclose%
\pgfusepath{fill}%
\end{pgfscope}%
\begin{pgfscope}%
\pgfpathrectangle{\pgfqpoint{0.017500in}{0.017500in}}{\pgfqpoint{6.218282in}{3.580372in}} %
\pgfusepath{clip}%
\pgfsetbuttcap%
\pgfsetroundjoin%
\definecolor{currentfill}{rgb}{0.746838,0.140021,0.179996}%
\pgfsetfillcolor{currentfill}%
\pgfsetlinewidth{0.000000pt}%
\definecolor{currentstroke}{rgb}{0.000000,0.000000,0.000000}%
\pgfsetstrokecolor{currentstroke}%
\pgfsetdash{}{0pt}%
\pgfpathmoveto{\pgfqpoint{2.674254in}{2.688576in}}%
\pgfpathlineto{\pgfqpoint{2.932968in}{2.820668in}}%
\pgfpathlineto{\pgfqpoint{2.674254in}{2.688576in}}%
\pgfpathclose%
\pgfusepath{fill}%
\end{pgfscope}%
\begin{pgfscope}%
\pgfpathrectangle{\pgfqpoint{0.017500in}{0.017500in}}{\pgfqpoint{6.218282in}{3.580372in}} %
\pgfusepath{clip}%
\pgfsetbuttcap%
\pgfsetroundjoin%
\definecolor{currentfill}{rgb}{0.746838,0.140021,0.179996}%
\pgfsetfillcolor{currentfill}%
\pgfsetlinewidth{0.000000pt}%
\definecolor{currentstroke}{rgb}{0.000000,0.000000,0.000000}%
\pgfsetstrokecolor{currentstroke}%
\pgfsetdash{}{0pt}%
\pgfpathmoveto{\pgfqpoint{2.674254in}{2.688576in}}%
\pgfpathlineto{\pgfqpoint{2.932968in}{2.820668in}}%
\pgfpathlineto{\pgfqpoint{2.674254in}{2.688576in}}%
\pgfpathclose%
\pgfusepath{fill}%
\end{pgfscope}%
\begin{pgfscope}%
\pgfpathrectangle{\pgfqpoint{0.017500in}{0.017500in}}{\pgfqpoint{6.218282in}{3.580372in}} %
\pgfusepath{clip}%
\pgfsetbuttcap%
\pgfsetroundjoin%
\definecolor{currentfill}{rgb}{0.746838,0.140021,0.179996}%
\pgfsetfillcolor{currentfill}%
\pgfsetlinewidth{0.000000pt}%
\definecolor{currentstroke}{rgb}{0.000000,0.000000,0.000000}%
\pgfsetstrokecolor{currentstroke}%
\pgfsetdash{}{0pt}%
\pgfpathmoveto{\pgfqpoint{2.674254in}{2.688576in}}%
\pgfpathlineto{\pgfqpoint{2.932968in}{2.820668in}}%
\pgfpathlineto{\pgfqpoint{2.674254in}{2.688576in}}%
\pgfpathclose%
\pgfusepath{fill}%
\end{pgfscope}%
\begin{pgfscope}%
\pgfpathrectangle{\pgfqpoint{0.017500in}{0.017500in}}{\pgfqpoint{6.218282in}{3.580372in}} %
\pgfusepath{clip}%
\pgfsetbuttcap%
\pgfsetroundjoin%
\definecolor{currentfill}{rgb}{0.746838,0.140021,0.179996}%
\pgfsetfillcolor{currentfill}%
\pgfsetlinewidth{0.000000pt}%
\definecolor{currentstroke}{rgb}{0.000000,0.000000,0.000000}%
\pgfsetstrokecolor{currentstroke}%
\pgfsetdash{}{0pt}%
\pgfpathmoveto{\pgfqpoint{2.674254in}{2.688576in}}%
\pgfpathlineto{\pgfqpoint{2.932968in}{2.820668in}}%
\pgfpathlineto{\pgfqpoint{2.674254in}{2.688576in}}%
\pgfpathclose%
\pgfusepath{fill}%
\end{pgfscope}%
\begin{pgfscope}%
\pgfpathrectangle{\pgfqpoint{0.017500in}{0.017500in}}{\pgfqpoint{6.218282in}{3.580372in}} %
\pgfusepath{clip}%
\pgfsetbuttcap%
\pgfsetroundjoin%
\definecolor{currentfill}{rgb}{0.746838,0.140021,0.179996}%
\pgfsetfillcolor{currentfill}%
\pgfsetlinewidth{0.000000pt}%
\definecolor{currentstroke}{rgb}{0.000000,0.000000,0.000000}%
\pgfsetstrokecolor{currentstroke}%
\pgfsetdash{}{0pt}%
\pgfpathmoveto{\pgfqpoint{2.674254in}{2.688576in}}%
\pgfpathlineto{\pgfqpoint{2.932968in}{2.820668in}}%
\pgfpathlineto{\pgfqpoint{2.674254in}{2.688576in}}%
\pgfpathclose%
\pgfusepath{fill}%
\end{pgfscope}%
\begin{pgfscope}%
\pgfpathrectangle{\pgfqpoint{0.017500in}{0.017500in}}{\pgfqpoint{6.218282in}{3.580372in}} %
\pgfusepath{clip}%
\pgfsetbuttcap%
\pgfsetroundjoin%
\definecolor{currentfill}{rgb}{0.746838,0.140021,0.179996}%
\pgfsetfillcolor{currentfill}%
\pgfsetlinewidth{0.000000pt}%
\definecolor{currentstroke}{rgb}{0.000000,0.000000,0.000000}%
\pgfsetstrokecolor{currentstroke}%
\pgfsetdash{}{0pt}%
\pgfpathmoveto{\pgfqpoint{2.674254in}{2.688576in}}%
\pgfpathlineto{\pgfqpoint{2.932968in}{2.820668in}}%
\pgfpathlineto{\pgfqpoint{2.674254in}{2.688576in}}%
\pgfpathclose%
\pgfusepath{fill}%
\end{pgfscope}%
\begin{pgfscope}%
\pgfpathrectangle{\pgfqpoint{0.017500in}{0.017500in}}{\pgfqpoint{6.218282in}{3.580372in}} %
\pgfusepath{clip}%
\pgfsetbuttcap%
\pgfsetroundjoin%
\definecolor{currentfill}{rgb}{0.746838,0.140021,0.179996}%
\pgfsetfillcolor{currentfill}%
\pgfsetlinewidth{0.000000pt}%
\definecolor{currentstroke}{rgb}{0.000000,0.000000,0.000000}%
\pgfsetstrokecolor{currentstroke}%
\pgfsetdash{}{0pt}%
\pgfpathmoveto{\pgfqpoint{2.674254in}{2.688576in}}%
\pgfpathlineto{\pgfqpoint{2.932968in}{2.820668in}}%
\pgfpathlineto{\pgfqpoint{2.674254in}{2.688576in}}%
\pgfpathclose%
\pgfusepath{fill}%
\end{pgfscope}%
\begin{pgfscope}%
\pgfpathrectangle{\pgfqpoint{0.017500in}{0.017500in}}{\pgfqpoint{6.218282in}{3.580372in}} %
\pgfusepath{clip}%
\pgfsetbuttcap%
\pgfsetroundjoin%
\definecolor{currentfill}{rgb}{0.746838,0.140021,0.179996}%
\pgfsetfillcolor{currentfill}%
\pgfsetlinewidth{0.000000pt}%
\definecolor{currentstroke}{rgb}{0.000000,0.000000,0.000000}%
\pgfsetstrokecolor{currentstroke}%
\pgfsetdash{}{0pt}%
\pgfpathmoveto{\pgfqpoint{2.674254in}{2.688576in}}%
\pgfpathlineto{\pgfqpoint{2.932968in}{2.820668in}}%
\pgfpathlineto{\pgfqpoint{2.674254in}{2.688576in}}%
\pgfpathclose%
\pgfusepath{fill}%
\end{pgfscope}%
\begin{pgfscope}%
\pgfpathrectangle{\pgfqpoint{0.017500in}{0.017500in}}{\pgfqpoint{6.218282in}{3.580372in}} %
\pgfusepath{clip}%
\pgfsetbuttcap%
\pgfsetroundjoin%
\definecolor{currentfill}{rgb}{0.746838,0.140021,0.179996}%
\pgfsetfillcolor{currentfill}%
\pgfsetlinewidth{0.000000pt}%
\definecolor{currentstroke}{rgb}{0.000000,0.000000,0.000000}%
\pgfsetstrokecolor{currentstroke}%
\pgfsetdash{}{0pt}%
\pgfpathmoveto{\pgfqpoint{2.674254in}{2.688576in}}%
\pgfpathlineto{\pgfqpoint{2.932968in}{2.820668in}}%
\pgfpathlineto{\pgfqpoint{2.674254in}{2.688576in}}%
\pgfpathclose%
\pgfusepath{fill}%
\end{pgfscope}%
\begin{pgfscope}%
\pgfpathrectangle{\pgfqpoint{0.017500in}{0.017500in}}{\pgfqpoint{6.218282in}{3.580372in}} %
\pgfusepath{clip}%
\pgfsetbuttcap%
\pgfsetroundjoin%
\definecolor{currentfill}{rgb}{0.746838,0.140021,0.179996}%
\pgfsetfillcolor{currentfill}%
\pgfsetlinewidth{0.000000pt}%
\definecolor{currentstroke}{rgb}{0.000000,0.000000,0.000000}%
\pgfsetstrokecolor{currentstroke}%
\pgfsetdash{}{0pt}%
\pgfpathmoveto{\pgfqpoint{2.674254in}{2.688576in}}%
\pgfpathlineto{\pgfqpoint{2.932968in}{2.820668in}}%
\pgfpathlineto{\pgfqpoint{2.674254in}{2.688576in}}%
\pgfpathclose%
\pgfusepath{fill}%
\end{pgfscope}%
\begin{pgfscope}%
\pgfpathrectangle{\pgfqpoint{0.017500in}{0.017500in}}{\pgfqpoint{6.218282in}{3.580372in}} %
\pgfusepath{clip}%
\pgfsetbuttcap%
\pgfsetroundjoin%
\definecolor{currentfill}{rgb}{0.746838,0.140021,0.179996}%
\pgfsetfillcolor{currentfill}%
\pgfsetlinewidth{0.000000pt}%
\definecolor{currentstroke}{rgb}{0.000000,0.000000,0.000000}%
\pgfsetstrokecolor{currentstroke}%
\pgfsetdash{}{0pt}%
\pgfpathmoveto{\pgfqpoint{2.674254in}{2.688576in}}%
\pgfpathlineto{\pgfqpoint{2.932968in}{2.820668in}}%
\pgfpathlineto{\pgfqpoint{2.674254in}{2.688576in}}%
\pgfpathclose%
\pgfusepath{fill}%
\end{pgfscope}%
\begin{pgfscope}%
\pgfpathrectangle{\pgfqpoint{0.017500in}{0.017500in}}{\pgfqpoint{6.218282in}{3.580372in}} %
\pgfusepath{clip}%
\pgfsetbuttcap%
\pgfsetroundjoin%
\definecolor{currentfill}{rgb}{0.746838,0.140021,0.179996}%
\pgfsetfillcolor{currentfill}%
\pgfsetlinewidth{0.000000pt}%
\definecolor{currentstroke}{rgb}{0.000000,0.000000,0.000000}%
\pgfsetstrokecolor{currentstroke}%
\pgfsetdash{}{0pt}%
\pgfpathmoveto{\pgfqpoint{2.674254in}{2.688576in}}%
\pgfpathlineto{\pgfqpoint{2.932968in}{2.820668in}}%
\pgfpathlineto{\pgfqpoint{2.674254in}{2.688576in}}%
\pgfpathclose%
\pgfusepath{fill}%
\end{pgfscope}%
\begin{pgfscope}%
\pgfpathrectangle{\pgfqpoint{0.017500in}{0.017500in}}{\pgfqpoint{6.218282in}{3.580372in}} %
\pgfusepath{clip}%
\pgfsetbuttcap%
\pgfsetroundjoin%
\definecolor{currentfill}{rgb}{0.746838,0.140021,0.179996}%
\pgfsetfillcolor{currentfill}%
\pgfsetlinewidth{0.000000pt}%
\definecolor{currentstroke}{rgb}{0.000000,0.000000,0.000000}%
\pgfsetstrokecolor{currentstroke}%
\pgfsetdash{}{0pt}%
\pgfpathmoveto{\pgfqpoint{2.674254in}{2.688576in}}%
\pgfpathlineto{\pgfqpoint{2.932968in}{2.820668in}}%
\pgfpathlineto{\pgfqpoint{2.674254in}{2.688576in}}%
\pgfpathclose%
\pgfusepath{fill}%
\end{pgfscope}%
\begin{pgfscope}%
\pgfpathrectangle{\pgfqpoint{0.017500in}{0.017500in}}{\pgfqpoint{6.218282in}{3.580372in}} %
\pgfusepath{clip}%
\pgfsetbuttcap%
\pgfsetroundjoin%
\definecolor{currentfill}{rgb}{0.746838,0.140021,0.179996}%
\pgfsetfillcolor{currentfill}%
\pgfsetlinewidth{0.000000pt}%
\definecolor{currentstroke}{rgb}{0.000000,0.000000,0.000000}%
\pgfsetstrokecolor{currentstroke}%
\pgfsetdash{}{0pt}%
\pgfpathmoveto{\pgfqpoint{2.674254in}{2.688576in}}%
\pgfpathlineto{\pgfqpoint{2.932968in}{2.820668in}}%
\pgfpathlineto{\pgfqpoint{2.674254in}{2.688576in}}%
\pgfpathclose%
\pgfusepath{fill}%
\end{pgfscope}%
\begin{pgfscope}%
\pgfpathrectangle{\pgfqpoint{0.017500in}{0.017500in}}{\pgfqpoint{6.218282in}{3.580372in}} %
\pgfusepath{clip}%
\pgfsetbuttcap%
\pgfsetroundjoin%
\definecolor{currentfill}{rgb}{0.746838,0.140021,0.179996}%
\pgfsetfillcolor{currentfill}%
\pgfsetlinewidth{0.000000pt}%
\definecolor{currentstroke}{rgb}{0.000000,0.000000,0.000000}%
\pgfsetstrokecolor{currentstroke}%
\pgfsetdash{}{0pt}%
\pgfpathmoveto{\pgfqpoint{2.674254in}{2.688576in}}%
\pgfpathlineto{\pgfqpoint{2.932968in}{2.820668in}}%
\pgfpathlineto{\pgfqpoint{2.674254in}{2.688576in}}%
\pgfpathclose%
\pgfusepath{fill}%
\end{pgfscope}%
\begin{pgfscope}%
\pgfpathrectangle{\pgfqpoint{0.017500in}{0.017500in}}{\pgfqpoint{6.218282in}{3.580372in}} %
\pgfusepath{clip}%
\pgfsetbuttcap%
\pgfsetroundjoin%
\definecolor{currentfill}{rgb}{0.746838,0.140021,0.179996}%
\pgfsetfillcolor{currentfill}%
\pgfsetlinewidth{0.000000pt}%
\definecolor{currentstroke}{rgb}{0.000000,0.000000,0.000000}%
\pgfsetstrokecolor{currentstroke}%
\pgfsetdash{}{0pt}%
\pgfpathmoveto{\pgfqpoint{2.674254in}{2.688576in}}%
\pgfpathlineto{\pgfqpoint{2.932968in}{2.820668in}}%
\pgfpathlineto{\pgfqpoint{2.674254in}{2.688576in}}%
\pgfpathclose%
\pgfusepath{fill}%
\end{pgfscope}%
\begin{pgfscope}%
\pgfpathrectangle{\pgfqpoint{0.017500in}{0.017500in}}{\pgfqpoint{6.218282in}{3.580372in}} %
\pgfusepath{clip}%
\pgfsetbuttcap%
\pgfsetroundjoin%
\definecolor{currentfill}{rgb}{0.746838,0.140021,0.179996}%
\pgfsetfillcolor{currentfill}%
\pgfsetlinewidth{0.000000pt}%
\definecolor{currentstroke}{rgb}{0.000000,0.000000,0.000000}%
\pgfsetstrokecolor{currentstroke}%
\pgfsetdash{}{0pt}%
\pgfpathmoveto{\pgfqpoint{2.674254in}{2.688576in}}%
\pgfpathlineto{\pgfqpoint{2.932968in}{2.820668in}}%
\pgfpathlineto{\pgfqpoint{2.674254in}{2.688576in}}%
\pgfpathclose%
\pgfusepath{fill}%
\end{pgfscope}%
\begin{pgfscope}%
\pgfpathrectangle{\pgfqpoint{0.017500in}{0.017500in}}{\pgfqpoint{6.218282in}{3.580372in}} %
\pgfusepath{clip}%
\pgfsetbuttcap%
\pgfsetroundjoin%
\definecolor{currentfill}{rgb}{0.746838,0.140021,0.179996}%
\pgfsetfillcolor{currentfill}%
\pgfsetlinewidth{0.000000pt}%
\definecolor{currentstroke}{rgb}{0.000000,0.000000,0.000000}%
\pgfsetstrokecolor{currentstroke}%
\pgfsetdash{}{0pt}%
\pgfpathmoveto{\pgfqpoint{2.674254in}{2.688576in}}%
\pgfpathlineto{\pgfqpoint{2.932968in}{2.820668in}}%
\pgfpathlineto{\pgfqpoint{2.674254in}{2.688576in}}%
\pgfpathclose%
\pgfusepath{fill}%
\end{pgfscope}%
\begin{pgfscope}%
\pgfpathrectangle{\pgfqpoint{0.017500in}{0.017500in}}{\pgfqpoint{6.218282in}{3.580372in}} %
\pgfusepath{clip}%
\pgfsetbuttcap%
\pgfsetroundjoin%
\definecolor{currentfill}{rgb}{0.746838,0.140021,0.179996}%
\pgfsetfillcolor{currentfill}%
\pgfsetlinewidth{0.000000pt}%
\definecolor{currentstroke}{rgb}{0.000000,0.000000,0.000000}%
\pgfsetstrokecolor{currentstroke}%
\pgfsetdash{}{0pt}%
\pgfpathmoveto{\pgfqpoint{2.674254in}{2.688576in}}%
\pgfpathlineto{\pgfqpoint{2.932968in}{2.820668in}}%
\pgfpathlineto{\pgfqpoint{2.674254in}{2.688576in}}%
\pgfpathclose%
\pgfusepath{fill}%
\end{pgfscope}%
\begin{pgfscope}%
\pgfpathrectangle{\pgfqpoint{0.017500in}{0.017500in}}{\pgfqpoint{6.218282in}{3.580372in}} %
\pgfusepath{clip}%
\pgfsetbuttcap%
\pgfsetroundjoin%
\definecolor{currentfill}{rgb}{0.746838,0.140021,0.179996}%
\pgfsetfillcolor{currentfill}%
\pgfsetlinewidth{0.000000pt}%
\definecolor{currentstroke}{rgb}{0.000000,0.000000,0.000000}%
\pgfsetstrokecolor{currentstroke}%
\pgfsetdash{}{0pt}%
\pgfpathmoveto{\pgfqpoint{2.674254in}{2.688576in}}%
\pgfpathlineto{\pgfqpoint{2.932968in}{2.820668in}}%
\pgfpathlineto{\pgfqpoint{2.674254in}{2.688576in}}%
\pgfpathclose%
\pgfusepath{fill}%
\end{pgfscope}%
\begin{pgfscope}%
\pgfpathrectangle{\pgfqpoint{0.017500in}{0.017500in}}{\pgfqpoint{6.218282in}{3.580372in}} %
\pgfusepath{clip}%
\pgfsetbuttcap%
\pgfsetroundjoin%
\definecolor{currentfill}{rgb}{0.746838,0.140021,0.179996}%
\pgfsetfillcolor{currentfill}%
\pgfsetlinewidth{0.000000pt}%
\definecolor{currentstroke}{rgb}{0.000000,0.000000,0.000000}%
\pgfsetstrokecolor{currentstroke}%
\pgfsetdash{}{0pt}%
\pgfpathmoveto{\pgfqpoint{2.674254in}{2.688576in}}%
\pgfpathlineto{\pgfqpoint{2.932968in}{2.820668in}}%
\pgfpathlineto{\pgfqpoint{2.674254in}{2.688576in}}%
\pgfpathclose%
\pgfusepath{fill}%
\end{pgfscope}%
\begin{pgfscope}%
\pgfpathrectangle{\pgfqpoint{0.017500in}{0.017500in}}{\pgfqpoint{6.218282in}{3.580372in}} %
\pgfusepath{clip}%
\pgfsetbuttcap%
\pgfsetroundjoin%
\definecolor{currentfill}{rgb}{0.746838,0.140021,0.179996}%
\pgfsetfillcolor{currentfill}%
\pgfsetlinewidth{0.000000pt}%
\definecolor{currentstroke}{rgb}{0.000000,0.000000,0.000000}%
\pgfsetstrokecolor{currentstroke}%
\pgfsetdash{}{0pt}%
\pgfpathmoveto{\pgfqpoint{2.674254in}{2.688576in}}%
\pgfpathlineto{\pgfqpoint{2.932968in}{2.820668in}}%
\pgfpathlineto{\pgfqpoint{2.674254in}{2.688576in}}%
\pgfpathclose%
\pgfusepath{fill}%
\end{pgfscope}%
\begin{pgfscope}%
\pgfpathrectangle{\pgfqpoint{0.017500in}{0.017500in}}{\pgfqpoint{6.218282in}{3.580372in}} %
\pgfusepath{clip}%
\pgfsetbuttcap%
\pgfsetroundjoin%
\definecolor{currentfill}{rgb}{0.746838,0.140021,0.179996}%
\pgfsetfillcolor{currentfill}%
\pgfsetlinewidth{0.000000pt}%
\definecolor{currentstroke}{rgb}{0.000000,0.000000,0.000000}%
\pgfsetstrokecolor{currentstroke}%
\pgfsetdash{}{0pt}%
\pgfpathmoveto{\pgfqpoint{2.674254in}{2.688576in}}%
\pgfpathlineto{\pgfqpoint{2.932968in}{2.820668in}}%
\pgfpathlineto{\pgfqpoint{2.674254in}{2.688576in}}%
\pgfpathclose%
\pgfusepath{fill}%
\end{pgfscope}%
\begin{pgfscope}%
\pgfpathrectangle{\pgfqpoint{0.017500in}{0.017500in}}{\pgfqpoint{6.218282in}{3.580372in}} %
\pgfusepath{clip}%
\pgfsetbuttcap%
\pgfsetroundjoin%
\definecolor{currentfill}{rgb}{0.746838,0.140021,0.179996}%
\pgfsetfillcolor{currentfill}%
\pgfsetlinewidth{0.000000pt}%
\definecolor{currentstroke}{rgb}{0.000000,0.000000,0.000000}%
\pgfsetstrokecolor{currentstroke}%
\pgfsetdash{}{0pt}%
\pgfpathmoveto{\pgfqpoint{2.674254in}{2.688576in}}%
\pgfpathlineto{\pgfqpoint{2.932968in}{2.820668in}}%
\pgfpathlineto{\pgfqpoint{2.674254in}{2.688576in}}%
\pgfpathclose%
\pgfusepath{fill}%
\end{pgfscope}%
\begin{pgfscope}%
\pgfpathrectangle{\pgfqpoint{0.017500in}{0.017500in}}{\pgfqpoint{6.218282in}{3.580372in}} %
\pgfusepath{clip}%
\pgfsetbuttcap%
\pgfsetroundjoin%
\definecolor{currentfill}{rgb}{0.746838,0.140021,0.179996}%
\pgfsetfillcolor{currentfill}%
\pgfsetlinewidth{0.000000pt}%
\definecolor{currentstroke}{rgb}{0.000000,0.000000,0.000000}%
\pgfsetstrokecolor{currentstroke}%
\pgfsetdash{}{0pt}%
\pgfpathmoveto{\pgfqpoint{2.674254in}{2.688576in}}%
\pgfpathlineto{\pgfqpoint{2.932968in}{2.820668in}}%
\pgfpathlineto{\pgfqpoint{2.674254in}{2.688576in}}%
\pgfpathclose%
\pgfusepath{fill}%
\end{pgfscope}%
\begin{pgfscope}%
\pgfpathrectangle{\pgfqpoint{0.017500in}{0.017500in}}{\pgfqpoint{6.218282in}{3.580372in}} %
\pgfusepath{clip}%
\pgfsetbuttcap%
\pgfsetroundjoin%
\definecolor{currentfill}{rgb}{0.746838,0.140021,0.179996}%
\pgfsetfillcolor{currentfill}%
\pgfsetlinewidth{0.000000pt}%
\definecolor{currentstroke}{rgb}{0.000000,0.000000,0.000000}%
\pgfsetstrokecolor{currentstroke}%
\pgfsetdash{}{0pt}%
\pgfpathmoveto{\pgfqpoint{2.674254in}{2.688576in}}%
\pgfpathlineto{\pgfqpoint{2.932968in}{2.820668in}}%
\pgfpathlineto{\pgfqpoint{2.674254in}{2.688576in}}%
\pgfpathclose%
\pgfusepath{fill}%
\end{pgfscope}%
\begin{pgfscope}%
\pgfpathrectangle{\pgfqpoint{0.017500in}{0.017500in}}{\pgfqpoint{6.218282in}{3.580372in}} %
\pgfusepath{clip}%
\pgfsetbuttcap%
\pgfsetroundjoin%
\definecolor{currentfill}{rgb}{0.746838,0.140021,0.179996}%
\pgfsetfillcolor{currentfill}%
\pgfsetlinewidth{0.000000pt}%
\definecolor{currentstroke}{rgb}{0.000000,0.000000,0.000000}%
\pgfsetstrokecolor{currentstroke}%
\pgfsetdash{}{0pt}%
\pgfpathmoveto{\pgfqpoint{2.674254in}{2.688576in}}%
\pgfpathlineto{\pgfqpoint{2.932968in}{2.820668in}}%
\pgfpathlineto{\pgfqpoint{2.674254in}{2.688576in}}%
\pgfpathclose%
\pgfusepath{fill}%
\end{pgfscope}%
\begin{pgfscope}%
\pgfpathrectangle{\pgfqpoint{0.017500in}{0.017500in}}{\pgfqpoint{6.218282in}{3.580372in}} %
\pgfusepath{clip}%
\pgfsetbuttcap%
\pgfsetroundjoin%
\definecolor{currentfill}{rgb}{0.746838,0.140021,0.179996}%
\pgfsetfillcolor{currentfill}%
\pgfsetlinewidth{0.000000pt}%
\definecolor{currentstroke}{rgb}{0.000000,0.000000,0.000000}%
\pgfsetstrokecolor{currentstroke}%
\pgfsetdash{}{0pt}%
\pgfpathmoveto{\pgfqpoint{2.674254in}{2.688576in}}%
\pgfpathlineto{\pgfqpoint{2.932968in}{2.820668in}}%
\pgfpathlineto{\pgfqpoint{2.674254in}{2.688576in}}%
\pgfpathclose%
\pgfusepath{fill}%
\end{pgfscope}%
\begin{pgfscope}%
\pgfpathrectangle{\pgfqpoint{0.017500in}{0.017500in}}{\pgfqpoint{6.218282in}{3.580372in}} %
\pgfusepath{clip}%
\pgfsetbuttcap%
\pgfsetroundjoin%
\definecolor{currentfill}{rgb}{0.895882,0.849906,0.823499}%
\pgfsetfillcolor{currentfill}%
\pgfsetlinewidth{0.000000pt}%
\definecolor{currentstroke}{rgb}{0.000000,0.000000,0.000000}%
\pgfsetstrokecolor{currentstroke}%
\pgfsetdash{}{0pt}%
\pgfpathmoveto{\pgfqpoint{4.063701in}{2.813820in}}%
\pgfpathlineto{\pgfqpoint{2.789950in}{0.798346in}}%
\pgfpathlineto{\pgfqpoint{4.063701in}{2.813820in}}%
\pgfpathclose%
\pgfusepath{fill}%
\end{pgfscope}%
\begin{pgfscope}%
\pgfpathrectangle{\pgfqpoint{0.017500in}{0.017500in}}{\pgfqpoint{6.218282in}{3.580372in}} %
\pgfusepath{clip}%
\pgfsetbuttcap%
\pgfsetroundjoin%
\definecolor{currentfill}{rgb}{0.895882,0.849906,0.823499}%
\pgfsetfillcolor{currentfill}%
\pgfsetlinewidth{0.000000pt}%
\definecolor{currentstroke}{rgb}{0.000000,0.000000,0.000000}%
\pgfsetstrokecolor{currentstroke}%
\pgfsetdash{}{0pt}%
\pgfpathmoveto{\pgfqpoint{4.063701in}{2.813820in}}%
\pgfpathlineto{\pgfqpoint{2.789950in}{0.798346in}}%
\pgfpathlineto{\pgfqpoint{4.063701in}{2.813820in}}%
\pgfpathclose%
\pgfusepath{fill}%
\end{pgfscope}%
\begin{pgfscope}%
\pgfpathrectangle{\pgfqpoint{0.017500in}{0.017500in}}{\pgfqpoint{6.218282in}{3.580372in}} %
\pgfusepath{clip}%
\pgfsetbuttcap%
\pgfsetroundjoin%
\definecolor{currentfill}{rgb}{0.895882,0.849906,0.823499}%
\pgfsetfillcolor{currentfill}%
\pgfsetlinewidth{0.000000pt}%
\definecolor{currentstroke}{rgb}{0.000000,0.000000,0.000000}%
\pgfsetstrokecolor{currentstroke}%
\pgfsetdash{}{0pt}%
\pgfpathmoveto{\pgfqpoint{4.063701in}{2.813820in}}%
\pgfpathlineto{\pgfqpoint{2.789950in}{0.798346in}}%
\pgfpathlineto{\pgfqpoint{4.063701in}{2.813820in}}%
\pgfpathclose%
\pgfusepath{fill}%
\end{pgfscope}%
\begin{pgfscope}%
\pgfpathrectangle{\pgfqpoint{0.017500in}{0.017500in}}{\pgfqpoint{6.218282in}{3.580372in}} %
\pgfusepath{clip}%
\pgfsetbuttcap%
\pgfsetroundjoin%
\definecolor{currentfill}{rgb}{0.895882,0.849906,0.823499}%
\pgfsetfillcolor{currentfill}%
\pgfsetlinewidth{0.000000pt}%
\definecolor{currentstroke}{rgb}{0.000000,0.000000,0.000000}%
\pgfsetstrokecolor{currentstroke}%
\pgfsetdash{}{0pt}%
\pgfpathmoveto{\pgfqpoint{4.063701in}{2.813820in}}%
\pgfpathlineto{\pgfqpoint{2.789950in}{0.798346in}}%
\pgfpathlineto{\pgfqpoint{4.063701in}{2.813820in}}%
\pgfpathclose%
\pgfusepath{fill}%
\end{pgfscope}%
\begin{pgfscope}%
\pgfpathrectangle{\pgfqpoint{0.017500in}{0.017500in}}{\pgfqpoint{6.218282in}{3.580372in}} %
\pgfusepath{clip}%
\pgfsetbuttcap%
\pgfsetroundjoin%
\definecolor{currentfill}{rgb}{0.895882,0.849906,0.823499}%
\pgfsetfillcolor{currentfill}%
\pgfsetlinewidth{0.000000pt}%
\definecolor{currentstroke}{rgb}{0.000000,0.000000,0.000000}%
\pgfsetstrokecolor{currentstroke}%
\pgfsetdash{}{0pt}%
\pgfpathmoveto{\pgfqpoint{4.063701in}{2.813820in}}%
\pgfpathlineto{\pgfqpoint{2.789950in}{0.798346in}}%
\pgfpathlineto{\pgfqpoint{4.063701in}{2.813820in}}%
\pgfpathclose%
\pgfusepath{fill}%
\end{pgfscope}%
\begin{pgfscope}%
\pgfpathrectangle{\pgfqpoint{0.017500in}{0.017500in}}{\pgfqpoint{6.218282in}{3.580372in}} %
\pgfusepath{clip}%
\pgfsetbuttcap%
\pgfsetroundjoin%
\definecolor{currentfill}{rgb}{0.895882,0.849906,0.823499}%
\pgfsetfillcolor{currentfill}%
\pgfsetlinewidth{0.000000pt}%
\definecolor{currentstroke}{rgb}{0.000000,0.000000,0.000000}%
\pgfsetstrokecolor{currentstroke}%
\pgfsetdash{}{0pt}%
\pgfpathmoveto{\pgfqpoint{4.063701in}{2.813820in}}%
\pgfpathlineto{\pgfqpoint{2.789950in}{0.798346in}}%
\pgfpathlineto{\pgfqpoint{4.063701in}{2.813820in}}%
\pgfpathclose%
\pgfusepath{fill}%
\end{pgfscope}%
\begin{pgfscope}%
\pgfpathrectangle{\pgfqpoint{0.017500in}{0.017500in}}{\pgfqpoint{6.218282in}{3.580372in}} %
\pgfusepath{clip}%
\pgfsetbuttcap%
\pgfsetroundjoin%
\definecolor{currentfill}{rgb}{0.895882,0.849906,0.823499}%
\pgfsetfillcolor{currentfill}%
\pgfsetlinewidth{0.000000pt}%
\definecolor{currentstroke}{rgb}{0.000000,0.000000,0.000000}%
\pgfsetstrokecolor{currentstroke}%
\pgfsetdash{}{0pt}%
\pgfpathmoveto{\pgfqpoint{4.063701in}{2.813820in}}%
\pgfpathlineto{\pgfqpoint{2.789950in}{0.798346in}}%
\pgfpathlineto{\pgfqpoint{4.063701in}{2.813820in}}%
\pgfpathclose%
\pgfusepath{fill}%
\end{pgfscope}%
\begin{pgfscope}%
\pgfpathrectangle{\pgfqpoint{0.017500in}{0.017500in}}{\pgfqpoint{6.218282in}{3.580372in}} %
\pgfusepath{clip}%
\pgfsetbuttcap%
\pgfsetroundjoin%
\definecolor{currentfill}{rgb}{0.895882,0.849906,0.823499}%
\pgfsetfillcolor{currentfill}%
\pgfsetlinewidth{0.000000pt}%
\definecolor{currentstroke}{rgb}{0.000000,0.000000,0.000000}%
\pgfsetstrokecolor{currentstroke}%
\pgfsetdash{}{0pt}%
\pgfpathmoveto{\pgfqpoint{4.063701in}{2.813820in}}%
\pgfpathlineto{\pgfqpoint{2.789950in}{0.798346in}}%
\pgfpathlineto{\pgfqpoint{4.063701in}{2.813820in}}%
\pgfpathclose%
\pgfusepath{fill}%
\end{pgfscope}%
\begin{pgfscope}%
\pgfpathrectangle{\pgfqpoint{0.017500in}{0.017500in}}{\pgfqpoint{6.218282in}{3.580372in}} %
\pgfusepath{clip}%
\pgfsetbuttcap%
\pgfsetroundjoin%
\definecolor{currentfill}{rgb}{0.895882,0.849906,0.823499}%
\pgfsetfillcolor{currentfill}%
\pgfsetlinewidth{0.000000pt}%
\definecolor{currentstroke}{rgb}{0.000000,0.000000,0.000000}%
\pgfsetstrokecolor{currentstroke}%
\pgfsetdash{}{0pt}%
\pgfpathmoveto{\pgfqpoint{4.063701in}{2.813820in}}%
\pgfpathlineto{\pgfqpoint{2.789950in}{0.798346in}}%
\pgfpathlineto{\pgfqpoint{4.063701in}{2.813820in}}%
\pgfpathclose%
\pgfusepath{fill}%
\end{pgfscope}%
\begin{pgfscope}%
\pgfpathrectangle{\pgfqpoint{0.017500in}{0.017500in}}{\pgfqpoint{6.218282in}{3.580372in}} %
\pgfusepath{clip}%
\pgfsetbuttcap%
\pgfsetroundjoin%
\definecolor{currentfill}{rgb}{0.895882,0.849906,0.823499}%
\pgfsetfillcolor{currentfill}%
\pgfsetlinewidth{0.000000pt}%
\definecolor{currentstroke}{rgb}{0.000000,0.000000,0.000000}%
\pgfsetstrokecolor{currentstroke}%
\pgfsetdash{}{0pt}%
\pgfpathmoveto{\pgfqpoint{4.063701in}{2.813820in}}%
\pgfpathlineto{\pgfqpoint{2.789950in}{0.798346in}}%
\pgfpathlineto{\pgfqpoint{4.063701in}{2.813820in}}%
\pgfpathclose%
\pgfusepath{fill}%
\end{pgfscope}%
\begin{pgfscope}%
\pgfpathrectangle{\pgfqpoint{0.017500in}{0.017500in}}{\pgfqpoint{6.218282in}{3.580372in}} %
\pgfusepath{clip}%
\pgfsetbuttcap%
\pgfsetroundjoin%
\definecolor{currentfill}{rgb}{0.895882,0.849906,0.823499}%
\pgfsetfillcolor{currentfill}%
\pgfsetlinewidth{0.000000pt}%
\definecolor{currentstroke}{rgb}{0.000000,0.000000,0.000000}%
\pgfsetstrokecolor{currentstroke}%
\pgfsetdash{}{0pt}%
\pgfpathmoveto{\pgfqpoint{4.063701in}{2.813820in}}%
\pgfpathlineto{\pgfqpoint{2.789950in}{0.798346in}}%
\pgfpathlineto{\pgfqpoint{4.063701in}{2.813820in}}%
\pgfpathclose%
\pgfusepath{fill}%
\end{pgfscope}%
\begin{pgfscope}%
\pgfpathrectangle{\pgfqpoint{0.017500in}{0.017500in}}{\pgfqpoint{6.218282in}{3.580372in}} %
\pgfusepath{clip}%
\pgfsetbuttcap%
\pgfsetroundjoin%
\definecolor{currentfill}{rgb}{0.895882,0.849906,0.823499}%
\pgfsetfillcolor{currentfill}%
\pgfsetlinewidth{0.000000pt}%
\definecolor{currentstroke}{rgb}{0.000000,0.000000,0.000000}%
\pgfsetstrokecolor{currentstroke}%
\pgfsetdash{}{0pt}%
\pgfpathmoveto{\pgfqpoint{4.063701in}{2.813820in}}%
\pgfpathlineto{\pgfqpoint{2.789950in}{0.798346in}}%
\pgfpathlineto{\pgfqpoint{4.063701in}{2.813820in}}%
\pgfpathclose%
\pgfusepath{fill}%
\end{pgfscope}%
\begin{pgfscope}%
\pgfpathrectangle{\pgfqpoint{0.017500in}{0.017500in}}{\pgfqpoint{6.218282in}{3.580372in}} %
\pgfusepath{clip}%
\pgfsetbuttcap%
\pgfsetroundjoin%
\definecolor{currentfill}{rgb}{0.895882,0.849906,0.823499}%
\pgfsetfillcolor{currentfill}%
\pgfsetlinewidth{0.000000pt}%
\definecolor{currentstroke}{rgb}{0.000000,0.000000,0.000000}%
\pgfsetstrokecolor{currentstroke}%
\pgfsetdash{}{0pt}%
\pgfpathmoveto{\pgfqpoint{4.063701in}{2.813820in}}%
\pgfpathlineto{\pgfqpoint{2.789950in}{0.798346in}}%
\pgfpathlineto{\pgfqpoint{4.063701in}{2.813820in}}%
\pgfpathclose%
\pgfusepath{fill}%
\end{pgfscope}%
\begin{pgfscope}%
\pgfpathrectangle{\pgfqpoint{0.017500in}{0.017500in}}{\pgfqpoint{6.218282in}{3.580372in}} %
\pgfusepath{clip}%
\pgfsetbuttcap%
\pgfsetroundjoin%
\definecolor{currentfill}{rgb}{0.895882,0.849906,0.823499}%
\pgfsetfillcolor{currentfill}%
\pgfsetlinewidth{0.000000pt}%
\definecolor{currentstroke}{rgb}{0.000000,0.000000,0.000000}%
\pgfsetstrokecolor{currentstroke}%
\pgfsetdash{}{0pt}%
\pgfpathmoveto{\pgfqpoint{4.063701in}{2.813820in}}%
\pgfpathlineto{\pgfqpoint{2.789950in}{0.798346in}}%
\pgfpathlineto{\pgfqpoint{4.063701in}{2.813820in}}%
\pgfpathclose%
\pgfusepath{fill}%
\end{pgfscope}%
\begin{pgfscope}%
\pgfpathrectangle{\pgfqpoint{0.017500in}{0.017500in}}{\pgfqpoint{6.218282in}{3.580372in}} %
\pgfusepath{clip}%
\pgfsetbuttcap%
\pgfsetroundjoin%
\definecolor{currentfill}{rgb}{0.895882,0.849906,0.823499}%
\pgfsetfillcolor{currentfill}%
\pgfsetlinewidth{0.000000pt}%
\definecolor{currentstroke}{rgb}{0.000000,0.000000,0.000000}%
\pgfsetstrokecolor{currentstroke}%
\pgfsetdash{}{0pt}%
\pgfpathmoveto{\pgfqpoint{4.063701in}{2.813820in}}%
\pgfpathlineto{\pgfqpoint{2.789950in}{0.798346in}}%
\pgfpathlineto{\pgfqpoint{4.063701in}{2.813820in}}%
\pgfpathclose%
\pgfusepath{fill}%
\end{pgfscope}%
\begin{pgfscope}%
\pgfpathrectangle{\pgfqpoint{0.017500in}{0.017500in}}{\pgfqpoint{6.218282in}{3.580372in}} %
\pgfusepath{clip}%
\pgfsetbuttcap%
\pgfsetroundjoin%
\definecolor{currentfill}{rgb}{0.895882,0.849906,0.823499}%
\pgfsetfillcolor{currentfill}%
\pgfsetlinewidth{0.000000pt}%
\definecolor{currentstroke}{rgb}{0.000000,0.000000,0.000000}%
\pgfsetstrokecolor{currentstroke}%
\pgfsetdash{}{0pt}%
\pgfpathmoveto{\pgfqpoint{4.063701in}{2.813820in}}%
\pgfpathlineto{\pgfqpoint{2.789950in}{0.798346in}}%
\pgfpathlineto{\pgfqpoint{4.063701in}{2.813820in}}%
\pgfpathclose%
\pgfusepath{fill}%
\end{pgfscope}%
\begin{pgfscope}%
\pgfpathrectangle{\pgfqpoint{0.017500in}{0.017500in}}{\pgfqpoint{6.218282in}{3.580372in}} %
\pgfusepath{clip}%
\pgfsetbuttcap%
\pgfsetroundjoin%
\definecolor{currentfill}{rgb}{0.895882,0.849906,0.823499}%
\pgfsetfillcolor{currentfill}%
\pgfsetlinewidth{0.000000pt}%
\definecolor{currentstroke}{rgb}{0.000000,0.000000,0.000000}%
\pgfsetstrokecolor{currentstroke}%
\pgfsetdash{}{0pt}%
\pgfpathmoveto{\pgfqpoint{4.063701in}{2.813820in}}%
\pgfpathlineto{\pgfqpoint{2.789950in}{0.798346in}}%
\pgfpathlineto{\pgfqpoint{4.063701in}{2.813820in}}%
\pgfpathclose%
\pgfusepath{fill}%
\end{pgfscope}%
\begin{pgfscope}%
\pgfpathrectangle{\pgfqpoint{0.017500in}{0.017500in}}{\pgfqpoint{6.218282in}{3.580372in}} %
\pgfusepath{clip}%
\pgfsetbuttcap%
\pgfsetroundjoin%
\definecolor{currentfill}{rgb}{0.895882,0.849906,0.823499}%
\pgfsetfillcolor{currentfill}%
\pgfsetlinewidth{0.000000pt}%
\definecolor{currentstroke}{rgb}{0.000000,0.000000,0.000000}%
\pgfsetstrokecolor{currentstroke}%
\pgfsetdash{}{0pt}%
\pgfpathmoveto{\pgfqpoint{4.063701in}{2.813820in}}%
\pgfpathlineto{\pgfqpoint{2.789950in}{0.798346in}}%
\pgfpathlineto{\pgfqpoint{4.063701in}{2.813820in}}%
\pgfpathclose%
\pgfusepath{fill}%
\end{pgfscope}%
\begin{pgfscope}%
\pgfpathrectangle{\pgfqpoint{0.017500in}{0.017500in}}{\pgfqpoint{6.218282in}{3.580372in}} %
\pgfusepath{clip}%
\pgfsetbuttcap%
\pgfsetroundjoin%
\definecolor{currentfill}{rgb}{0.895882,0.849906,0.823499}%
\pgfsetfillcolor{currentfill}%
\pgfsetlinewidth{0.000000pt}%
\definecolor{currentstroke}{rgb}{0.000000,0.000000,0.000000}%
\pgfsetstrokecolor{currentstroke}%
\pgfsetdash{}{0pt}%
\pgfpathmoveto{\pgfqpoint{4.063701in}{2.813820in}}%
\pgfpathlineto{\pgfqpoint{2.789950in}{0.798346in}}%
\pgfpathlineto{\pgfqpoint{4.063701in}{2.813820in}}%
\pgfpathclose%
\pgfusepath{fill}%
\end{pgfscope}%
\begin{pgfscope}%
\pgfpathrectangle{\pgfqpoint{0.017500in}{0.017500in}}{\pgfqpoint{6.218282in}{3.580372in}} %
\pgfusepath{clip}%
\pgfsetbuttcap%
\pgfsetroundjoin%
\definecolor{currentfill}{rgb}{0.895882,0.849906,0.823499}%
\pgfsetfillcolor{currentfill}%
\pgfsetlinewidth{0.000000pt}%
\definecolor{currentstroke}{rgb}{0.000000,0.000000,0.000000}%
\pgfsetstrokecolor{currentstroke}%
\pgfsetdash{}{0pt}%
\pgfpathmoveto{\pgfqpoint{4.063701in}{2.813820in}}%
\pgfpathlineto{\pgfqpoint{2.789950in}{0.798346in}}%
\pgfpathlineto{\pgfqpoint{4.063701in}{2.813820in}}%
\pgfpathclose%
\pgfusepath{fill}%
\end{pgfscope}%
\begin{pgfscope}%
\pgfpathrectangle{\pgfqpoint{0.017500in}{0.017500in}}{\pgfqpoint{6.218282in}{3.580372in}} %
\pgfusepath{clip}%
\pgfsetbuttcap%
\pgfsetroundjoin%
\definecolor{currentfill}{rgb}{0.895882,0.849906,0.823499}%
\pgfsetfillcolor{currentfill}%
\pgfsetlinewidth{0.000000pt}%
\definecolor{currentstroke}{rgb}{0.000000,0.000000,0.000000}%
\pgfsetstrokecolor{currentstroke}%
\pgfsetdash{}{0pt}%
\pgfpathmoveto{\pgfqpoint{4.063701in}{2.813820in}}%
\pgfpathlineto{\pgfqpoint{2.789950in}{0.798346in}}%
\pgfpathlineto{\pgfqpoint{4.063701in}{2.813820in}}%
\pgfpathclose%
\pgfusepath{fill}%
\end{pgfscope}%
\begin{pgfscope}%
\pgfpathrectangle{\pgfqpoint{0.017500in}{0.017500in}}{\pgfqpoint{6.218282in}{3.580372in}} %
\pgfusepath{clip}%
\pgfsetbuttcap%
\pgfsetroundjoin%
\definecolor{currentfill}{rgb}{0.895882,0.849906,0.823499}%
\pgfsetfillcolor{currentfill}%
\pgfsetlinewidth{0.000000pt}%
\definecolor{currentstroke}{rgb}{0.000000,0.000000,0.000000}%
\pgfsetstrokecolor{currentstroke}%
\pgfsetdash{}{0pt}%
\pgfpathmoveto{\pgfqpoint{4.063701in}{2.813820in}}%
\pgfpathlineto{\pgfqpoint{2.789950in}{0.798346in}}%
\pgfpathlineto{\pgfqpoint{4.063701in}{2.813820in}}%
\pgfpathclose%
\pgfusepath{fill}%
\end{pgfscope}%
\begin{pgfscope}%
\pgfpathrectangle{\pgfqpoint{0.017500in}{0.017500in}}{\pgfqpoint{6.218282in}{3.580372in}} %
\pgfusepath{clip}%
\pgfsetbuttcap%
\pgfsetroundjoin%
\definecolor{currentfill}{rgb}{0.895882,0.849906,0.823499}%
\pgfsetfillcolor{currentfill}%
\pgfsetlinewidth{0.000000pt}%
\definecolor{currentstroke}{rgb}{0.000000,0.000000,0.000000}%
\pgfsetstrokecolor{currentstroke}%
\pgfsetdash{}{0pt}%
\pgfpathmoveto{\pgfqpoint{4.063701in}{2.813820in}}%
\pgfpathlineto{\pgfqpoint{2.789950in}{0.798346in}}%
\pgfpathlineto{\pgfqpoint{4.063701in}{2.813820in}}%
\pgfpathclose%
\pgfusepath{fill}%
\end{pgfscope}%
\begin{pgfscope}%
\pgfpathrectangle{\pgfqpoint{0.017500in}{0.017500in}}{\pgfqpoint{6.218282in}{3.580372in}} %
\pgfusepath{clip}%
\pgfsetbuttcap%
\pgfsetroundjoin%
\definecolor{currentfill}{rgb}{0.895882,0.849906,0.823499}%
\pgfsetfillcolor{currentfill}%
\pgfsetlinewidth{0.000000pt}%
\definecolor{currentstroke}{rgb}{0.000000,0.000000,0.000000}%
\pgfsetstrokecolor{currentstroke}%
\pgfsetdash{}{0pt}%
\pgfpathmoveto{\pgfqpoint{4.063701in}{2.813820in}}%
\pgfpathlineto{\pgfqpoint{2.789950in}{0.798346in}}%
\pgfpathlineto{\pgfqpoint{4.063701in}{2.813820in}}%
\pgfpathclose%
\pgfusepath{fill}%
\end{pgfscope}%
\begin{pgfscope}%
\pgfpathrectangle{\pgfqpoint{0.017500in}{0.017500in}}{\pgfqpoint{6.218282in}{3.580372in}} %
\pgfusepath{clip}%
\pgfsetbuttcap%
\pgfsetroundjoin%
\definecolor{currentfill}{rgb}{0.895882,0.849906,0.823499}%
\pgfsetfillcolor{currentfill}%
\pgfsetlinewidth{0.000000pt}%
\definecolor{currentstroke}{rgb}{0.000000,0.000000,0.000000}%
\pgfsetstrokecolor{currentstroke}%
\pgfsetdash{}{0pt}%
\pgfpathmoveto{\pgfqpoint{4.063701in}{2.813820in}}%
\pgfpathlineto{\pgfqpoint{2.789950in}{0.798346in}}%
\pgfpathlineto{\pgfqpoint{4.063701in}{2.813820in}}%
\pgfpathclose%
\pgfusepath{fill}%
\end{pgfscope}%
\begin{pgfscope}%
\pgfpathrectangle{\pgfqpoint{0.017500in}{0.017500in}}{\pgfqpoint{6.218282in}{3.580372in}} %
\pgfusepath{clip}%
\pgfsetbuttcap%
\pgfsetroundjoin%
\definecolor{currentfill}{rgb}{0.746838,0.140021,0.179996}%
\pgfsetfillcolor{currentfill}%
\pgfsetlinewidth{0.000000pt}%
\definecolor{currentstroke}{rgb}{0.000000,0.000000,0.000000}%
\pgfsetstrokecolor{currentstroke}%
\pgfsetdash{}{0pt}%
\pgfpathmoveto{\pgfqpoint{2.674254in}{2.688576in}}%
\pgfpathlineto{\pgfqpoint{3.116513in}{2.603812in}}%
\pgfpathlineto{\pgfqpoint{3.372093in}{2.761754in}}%
\pgfpathlineto{\pgfqpoint{2.932968in}{2.820668in}}%
\pgfpathlineto{\pgfqpoint{2.674254in}{2.688576in}}%
\pgfpathclose%
\pgfusepath{fill}%
\end{pgfscope}%
\begin{pgfscope}%
\pgfpathrectangle{\pgfqpoint{0.017500in}{0.017500in}}{\pgfqpoint{6.218282in}{3.580372in}} %
\pgfusepath{clip}%
\pgfsetbuttcap%
\pgfsetroundjoin%
\definecolor{currentfill}{rgb}{0.746838,0.140021,0.179996}%
\pgfsetfillcolor{currentfill}%
\pgfsetlinewidth{0.000000pt}%
\definecolor{currentstroke}{rgb}{0.000000,0.000000,0.000000}%
\pgfsetstrokecolor{currentstroke}%
\pgfsetdash{}{0pt}%
\pgfpathmoveto{\pgfqpoint{2.674254in}{2.688576in}}%
\pgfpathlineto{\pgfqpoint{3.116513in}{2.603812in}}%
\pgfpathlineto{\pgfqpoint{3.372093in}{2.761754in}}%
\pgfpathlineto{\pgfqpoint{2.932968in}{2.820668in}}%
\pgfpathlineto{\pgfqpoint{2.674254in}{2.688576in}}%
\pgfpathclose%
\pgfusepath{fill}%
\end{pgfscope}%
\begin{pgfscope}%
\pgfpathrectangle{\pgfqpoint{0.017500in}{0.017500in}}{\pgfqpoint{6.218282in}{3.580372in}} %
\pgfusepath{clip}%
\pgfsetbuttcap%
\pgfsetroundjoin%
\definecolor{currentfill}{rgb}{0.746838,0.140021,0.179996}%
\pgfsetfillcolor{currentfill}%
\pgfsetlinewidth{0.000000pt}%
\definecolor{currentstroke}{rgb}{0.000000,0.000000,0.000000}%
\pgfsetstrokecolor{currentstroke}%
\pgfsetdash{}{0pt}%
\pgfpathmoveto{\pgfqpoint{2.674254in}{2.688576in}}%
\pgfpathlineto{\pgfqpoint{3.116513in}{2.603812in}}%
\pgfpathlineto{\pgfqpoint{3.372093in}{2.761754in}}%
\pgfpathlineto{\pgfqpoint{2.932968in}{2.820668in}}%
\pgfpathlineto{\pgfqpoint{2.674254in}{2.688576in}}%
\pgfpathclose%
\pgfusepath{fill}%
\end{pgfscope}%
\begin{pgfscope}%
\pgfpathrectangle{\pgfqpoint{0.017500in}{0.017500in}}{\pgfqpoint{6.218282in}{3.580372in}} %
\pgfusepath{clip}%
\pgfsetbuttcap%
\pgfsetroundjoin%
\definecolor{currentfill}{rgb}{0.746838,0.140021,0.179996}%
\pgfsetfillcolor{currentfill}%
\pgfsetlinewidth{0.000000pt}%
\definecolor{currentstroke}{rgb}{0.000000,0.000000,0.000000}%
\pgfsetstrokecolor{currentstroke}%
\pgfsetdash{}{0pt}%
\pgfpathmoveto{\pgfqpoint{2.674254in}{2.688576in}}%
\pgfpathlineto{\pgfqpoint{3.116513in}{2.603812in}}%
\pgfpathlineto{\pgfqpoint{3.372093in}{2.761754in}}%
\pgfpathlineto{\pgfqpoint{2.932968in}{2.820668in}}%
\pgfpathlineto{\pgfqpoint{2.674254in}{2.688576in}}%
\pgfpathclose%
\pgfusepath{fill}%
\end{pgfscope}%
\begin{pgfscope}%
\pgfpathrectangle{\pgfqpoint{0.017500in}{0.017500in}}{\pgfqpoint{6.218282in}{3.580372in}} %
\pgfusepath{clip}%
\pgfsetbuttcap%
\pgfsetroundjoin%
\definecolor{currentfill}{rgb}{0.746838,0.140021,0.179996}%
\pgfsetfillcolor{currentfill}%
\pgfsetlinewidth{0.000000pt}%
\definecolor{currentstroke}{rgb}{0.000000,0.000000,0.000000}%
\pgfsetstrokecolor{currentstroke}%
\pgfsetdash{}{0pt}%
\pgfpathmoveto{\pgfqpoint{2.674254in}{2.688576in}}%
\pgfpathlineto{\pgfqpoint{3.116513in}{2.603812in}}%
\pgfpathlineto{\pgfqpoint{3.372093in}{2.761754in}}%
\pgfpathlineto{\pgfqpoint{2.932968in}{2.820668in}}%
\pgfpathlineto{\pgfqpoint{2.674254in}{2.688576in}}%
\pgfpathclose%
\pgfusepath{fill}%
\end{pgfscope}%
\begin{pgfscope}%
\pgfpathrectangle{\pgfqpoint{0.017500in}{0.017500in}}{\pgfqpoint{6.218282in}{3.580372in}} %
\pgfusepath{clip}%
\pgfsetbuttcap%
\pgfsetroundjoin%
\definecolor{currentfill}{rgb}{0.746838,0.140021,0.179996}%
\pgfsetfillcolor{currentfill}%
\pgfsetlinewidth{0.000000pt}%
\definecolor{currentstroke}{rgb}{0.000000,0.000000,0.000000}%
\pgfsetstrokecolor{currentstroke}%
\pgfsetdash{}{0pt}%
\pgfpathmoveto{\pgfqpoint{2.674254in}{2.688576in}}%
\pgfpathlineto{\pgfqpoint{3.116513in}{2.603812in}}%
\pgfpathlineto{\pgfqpoint{3.372093in}{2.761754in}}%
\pgfpathlineto{\pgfqpoint{2.932968in}{2.820668in}}%
\pgfpathlineto{\pgfqpoint{2.674254in}{2.688576in}}%
\pgfpathclose%
\pgfusepath{fill}%
\end{pgfscope}%
\begin{pgfscope}%
\pgfpathrectangle{\pgfqpoint{0.017500in}{0.017500in}}{\pgfqpoint{6.218282in}{3.580372in}} %
\pgfusepath{clip}%
\pgfsetbuttcap%
\pgfsetroundjoin%
\definecolor{currentfill}{rgb}{0.735077,0.104460,0.171492}%
\pgfsetfillcolor{currentfill}%
\pgfsetlinewidth{0.000000pt}%
\definecolor{currentstroke}{rgb}{0.000000,0.000000,0.000000}%
\pgfsetstrokecolor{currentstroke}%
\pgfsetdash{}{0pt}%
\pgfpathmoveto{\pgfqpoint{3.816789in}{2.677291in}}%
\pgfpathlineto{\pgfqpoint{4.063701in}{2.813820in}}%
\pgfpathlineto{\pgfqpoint{3.816789in}{2.677291in}}%
\pgfpathclose%
\pgfusepath{fill}%
\end{pgfscope}%
\begin{pgfscope}%
\pgfpathrectangle{\pgfqpoint{0.017500in}{0.017500in}}{\pgfqpoint{6.218282in}{3.580372in}} %
\pgfusepath{clip}%
\pgfsetbuttcap%
\pgfsetroundjoin%
\definecolor{currentfill}{rgb}{0.735077,0.104460,0.171492}%
\pgfsetfillcolor{currentfill}%
\pgfsetlinewidth{0.000000pt}%
\definecolor{currentstroke}{rgb}{0.000000,0.000000,0.000000}%
\pgfsetstrokecolor{currentstroke}%
\pgfsetdash{}{0pt}%
\pgfpathmoveto{\pgfqpoint{3.816789in}{2.677291in}}%
\pgfpathlineto{\pgfqpoint{4.063701in}{2.813820in}}%
\pgfpathlineto{\pgfqpoint{3.816789in}{2.677291in}}%
\pgfpathclose%
\pgfusepath{fill}%
\end{pgfscope}%
\begin{pgfscope}%
\pgfpathrectangle{\pgfqpoint{0.017500in}{0.017500in}}{\pgfqpoint{6.218282in}{3.580372in}} %
\pgfusepath{clip}%
\pgfsetbuttcap%
\pgfsetroundjoin%
\definecolor{currentfill}{rgb}{0.735077,0.104460,0.171492}%
\pgfsetfillcolor{currentfill}%
\pgfsetlinewidth{0.000000pt}%
\definecolor{currentstroke}{rgb}{0.000000,0.000000,0.000000}%
\pgfsetstrokecolor{currentstroke}%
\pgfsetdash{}{0pt}%
\pgfpathmoveto{\pgfqpoint{3.816789in}{2.677291in}}%
\pgfpathlineto{\pgfqpoint{4.063701in}{2.813820in}}%
\pgfpathlineto{\pgfqpoint{3.816789in}{2.677291in}}%
\pgfpathclose%
\pgfusepath{fill}%
\end{pgfscope}%
\begin{pgfscope}%
\pgfpathrectangle{\pgfqpoint{0.017500in}{0.017500in}}{\pgfqpoint{6.218282in}{3.580372in}} %
\pgfusepath{clip}%
\pgfsetbuttcap%
\pgfsetroundjoin%
\definecolor{currentfill}{rgb}{0.735077,0.104460,0.171492}%
\pgfsetfillcolor{currentfill}%
\pgfsetlinewidth{0.000000pt}%
\definecolor{currentstroke}{rgb}{0.000000,0.000000,0.000000}%
\pgfsetstrokecolor{currentstroke}%
\pgfsetdash{}{0pt}%
\pgfpathmoveto{\pgfqpoint{3.816789in}{2.677291in}}%
\pgfpathlineto{\pgfqpoint{4.063701in}{2.813820in}}%
\pgfpathlineto{\pgfqpoint{3.816789in}{2.677291in}}%
\pgfpathclose%
\pgfusepath{fill}%
\end{pgfscope}%
\begin{pgfscope}%
\pgfpathrectangle{\pgfqpoint{0.017500in}{0.017500in}}{\pgfqpoint{6.218282in}{3.580372in}} %
\pgfusepath{clip}%
\pgfsetbuttcap%
\pgfsetroundjoin%
\definecolor{currentfill}{rgb}{0.735077,0.104460,0.171492}%
\pgfsetfillcolor{currentfill}%
\pgfsetlinewidth{0.000000pt}%
\definecolor{currentstroke}{rgb}{0.000000,0.000000,0.000000}%
\pgfsetstrokecolor{currentstroke}%
\pgfsetdash{}{0pt}%
\pgfpathmoveto{\pgfqpoint{3.816789in}{2.677291in}}%
\pgfpathlineto{\pgfqpoint{4.063701in}{2.813820in}}%
\pgfpathlineto{\pgfqpoint{3.816789in}{2.677291in}}%
\pgfpathclose%
\pgfusepath{fill}%
\end{pgfscope}%
\begin{pgfscope}%
\pgfpathrectangle{\pgfqpoint{0.017500in}{0.017500in}}{\pgfqpoint{6.218282in}{3.580372in}} %
\pgfusepath{clip}%
\pgfsetbuttcap%
\pgfsetroundjoin%
\definecolor{currentfill}{rgb}{0.735077,0.104460,0.171492}%
\pgfsetfillcolor{currentfill}%
\pgfsetlinewidth{0.000000pt}%
\definecolor{currentstroke}{rgb}{0.000000,0.000000,0.000000}%
\pgfsetstrokecolor{currentstroke}%
\pgfsetdash{}{0pt}%
\pgfpathmoveto{\pgfqpoint{3.816789in}{2.677291in}}%
\pgfpathlineto{\pgfqpoint{4.063701in}{2.813820in}}%
\pgfpathlineto{\pgfqpoint{3.816789in}{2.677291in}}%
\pgfpathclose%
\pgfusepath{fill}%
\end{pgfscope}%
\begin{pgfscope}%
\pgfpathrectangle{\pgfqpoint{0.017500in}{0.017500in}}{\pgfqpoint{6.218282in}{3.580372in}} %
\pgfusepath{clip}%
\pgfsetbuttcap%
\pgfsetroundjoin%
\definecolor{currentfill}{rgb}{0.735077,0.104460,0.171492}%
\pgfsetfillcolor{currentfill}%
\pgfsetlinewidth{0.000000pt}%
\definecolor{currentstroke}{rgb}{0.000000,0.000000,0.000000}%
\pgfsetstrokecolor{currentstroke}%
\pgfsetdash{}{0pt}%
\pgfpathmoveto{\pgfqpoint{3.816789in}{2.677291in}}%
\pgfpathlineto{\pgfqpoint{4.063701in}{2.813820in}}%
\pgfpathlineto{\pgfqpoint{3.816789in}{2.677291in}}%
\pgfpathclose%
\pgfusepath{fill}%
\end{pgfscope}%
\begin{pgfscope}%
\pgfpathrectangle{\pgfqpoint{0.017500in}{0.017500in}}{\pgfqpoint{6.218282in}{3.580372in}} %
\pgfusepath{clip}%
\pgfsetbuttcap%
\pgfsetroundjoin%
\definecolor{currentfill}{rgb}{0.735077,0.104460,0.171492}%
\pgfsetfillcolor{currentfill}%
\pgfsetlinewidth{0.000000pt}%
\definecolor{currentstroke}{rgb}{0.000000,0.000000,0.000000}%
\pgfsetstrokecolor{currentstroke}%
\pgfsetdash{}{0pt}%
\pgfpathmoveto{\pgfqpoint{3.816789in}{2.677291in}}%
\pgfpathlineto{\pgfqpoint{4.063701in}{2.813820in}}%
\pgfpathlineto{\pgfqpoint{3.816789in}{2.677291in}}%
\pgfpathclose%
\pgfusepath{fill}%
\end{pgfscope}%
\begin{pgfscope}%
\pgfpathrectangle{\pgfqpoint{0.017500in}{0.017500in}}{\pgfqpoint{6.218282in}{3.580372in}} %
\pgfusepath{clip}%
\pgfsetbuttcap%
\pgfsetroundjoin%
\definecolor{currentfill}{rgb}{0.735077,0.104460,0.171492}%
\pgfsetfillcolor{currentfill}%
\pgfsetlinewidth{0.000000pt}%
\definecolor{currentstroke}{rgb}{0.000000,0.000000,0.000000}%
\pgfsetstrokecolor{currentstroke}%
\pgfsetdash{}{0pt}%
\pgfpathmoveto{\pgfqpoint{3.816789in}{2.677291in}}%
\pgfpathlineto{\pgfqpoint{4.063701in}{2.813820in}}%
\pgfpathlineto{\pgfqpoint{3.816789in}{2.677291in}}%
\pgfpathclose%
\pgfusepath{fill}%
\end{pgfscope}%
\begin{pgfscope}%
\pgfpathrectangle{\pgfqpoint{0.017500in}{0.017500in}}{\pgfqpoint{6.218282in}{3.580372in}} %
\pgfusepath{clip}%
\pgfsetbuttcap%
\pgfsetroundjoin%
\definecolor{currentfill}{rgb}{0.735077,0.104460,0.171492}%
\pgfsetfillcolor{currentfill}%
\pgfsetlinewidth{0.000000pt}%
\definecolor{currentstroke}{rgb}{0.000000,0.000000,0.000000}%
\pgfsetstrokecolor{currentstroke}%
\pgfsetdash{}{0pt}%
\pgfpathmoveto{\pgfqpoint{3.816789in}{2.677291in}}%
\pgfpathlineto{\pgfqpoint{4.063701in}{2.813820in}}%
\pgfpathlineto{\pgfqpoint{3.816789in}{2.677291in}}%
\pgfpathclose%
\pgfusepath{fill}%
\end{pgfscope}%
\begin{pgfscope}%
\pgfpathrectangle{\pgfqpoint{0.017500in}{0.017500in}}{\pgfqpoint{6.218282in}{3.580372in}} %
\pgfusepath{clip}%
\pgfsetbuttcap%
\pgfsetroundjoin%
\definecolor{currentfill}{rgb}{0.735077,0.104460,0.171492}%
\pgfsetfillcolor{currentfill}%
\pgfsetlinewidth{0.000000pt}%
\definecolor{currentstroke}{rgb}{0.000000,0.000000,0.000000}%
\pgfsetstrokecolor{currentstroke}%
\pgfsetdash{}{0pt}%
\pgfpathmoveto{\pgfqpoint{3.816789in}{2.677291in}}%
\pgfpathlineto{\pgfqpoint{4.063701in}{2.813820in}}%
\pgfpathlineto{\pgfqpoint{3.816789in}{2.677291in}}%
\pgfpathclose%
\pgfusepath{fill}%
\end{pgfscope}%
\begin{pgfscope}%
\pgfpathrectangle{\pgfqpoint{0.017500in}{0.017500in}}{\pgfqpoint{6.218282in}{3.580372in}} %
\pgfusepath{clip}%
\pgfsetbuttcap%
\pgfsetroundjoin%
\definecolor{currentfill}{rgb}{0.735077,0.104460,0.171492}%
\pgfsetfillcolor{currentfill}%
\pgfsetlinewidth{0.000000pt}%
\definecolor{currentstroke}{rgb}{0.000000,0.000000,0.000000}%
\pgfsetstrokecolor{currentstroke}%
\pgfsetdash{}{0pt}%
\pgfpathmoveto{\pgfqpoint{3.816789in}{2.677291in}}%
\pgfpathlineto{\pgfqpoint{4.063701in}{2.813820in}}%
\pgfpathlineto{\pgfqpoint{3.816789in}{2.677291in}}%
\pgfpathclose%
\pgfusepath{fill}%
\end{pgfscope}%
\begin{pgfscope}%
\pgfpathrectangle{\pgfqpoint{0.017500in}{0.017500in}}{\pgfqpoint{6.218282in}{3.580372in}} %
\pgfusepath{clip}%
\pgfsetbuttcap%
\pgfsetroundjoin%
\definecolor{currentfill}{rgb}{0.735077,0.104460,0.171492}%
\pgfsetfillcolor{currentfill}%
\pgfsetlinewidth{0.000000pt}%
\definecolor{currentstroke}{rgb}{0.000000,0.000000,0.000000}%
\pgfsetstrokecolor{currentstroke}%
\pgfsetdash{}{0pt}%
\pgfpathmoveto{\pgfqpoint{3.816789in}{2.677291in}}%
\pgfpathlineto{\pgfqpoint{4.063701in}{2.813820in}}%
\pgfpathlineto{\pgfqpoint{3.816789in}{2.677291in}}%
\pgfpathclose%
\pgfusepath{fill}%
\end{pgfscope}%
\begin{pgfscope}%
\pgfpathrectangle{\pgfqpoint{0.017500in}{0.017500in}}{\pgfqpoint{6.218282in}{3.580372in}} %
\pgfusepath{clip}%
\pgfsetbuttcap%
\pgfsetroundjoin%
\definecolor{currentfill}{rgb}{0.735077,0.104460,0.171492}%
\pgfsetfillcolor{currentfill}%
\pgfsetlinewidth{0.000000pt}%
\definecolor{currentstroke}{rgb}{0.000000,0.000000,0.000000}%
\pgfsetstrokecolor{currentstroke}%
\pgfsetdash{}{0pt}%
\pgfpathmoveto{\pgfqpoint{3.816789in}{2.677291in}}%
\pgfpathlineto{\pgfqpoint{4.063701in}{2.813820in}}%
\pgfpathlineto{\pgfqpoint{3.816789in}{2.677291in}}%
\pgfpathclose%
\pgfusepath{fill}%
\end{pgfscope}%
\begin{pgfscope}%
\pgfpathrectangle{\pgfqpoint{0.017500in}{0.017500in}}{\pgfqpoint{6.218282in}{3.580372in}} %
\pgfusepath{clip}%
\pgfsetbuttcap%
\pgfsetroundjoin%
\definecolor{currentfill}{rgb}{0.735077,0.104460,0.171492}%
\pgfsetfillcolor{currentfill}%
\pgfsetlinewidth{0.000000pt}%
\definecolor{currentstroke}{rgb}{0.000000,0.000000,0.000000}%
\pgfsetstrokecolor{currentstroke}%
\pgfsetdash{}{0pt}%
\pgfpathmoveto{\pgfqpoint{3.816789in}{2.677291in}}%
\pgfpathlineto{\pgfqpoint{4.063701in}{2.813820in}}%
\pgfpathlineto{\pgfqpoint{3.816789in}{2.677291in}}%
\pgfpathclose%
\pgfusepath{fill}%
\end{pgfscope}%
\begin{pgfscope}%
\pgfpathrectangle{\pgfqpoint{0.017500in}{0.017500in}}{\pgfqpoint{6.218282in}{3.580372in}} %
\pgfusepath{clip}%
\pgfsetbuttcap%
\pgfsetroundjoin%
\definecolor{currentfill}{rgb}{0.735077,0.104460,0.171492}%
\pgfsetfillcolor{currentfill}%
\pgfsetlinewidth{0.000000pt}%
\definecolor{currentstroke}{rgb}{0.000000,0.000000,0.000000}%
\pgfsetstrokecolor{currentstroke}%
\pgfsetdash{}{0pt}%
\pgfpathmoveto{\pgfqpoint{3.816789in}{2.677291in}}%
\pgfpathlineto{\pgfqpoint{4.063701in}{2.813820in}}%
\pgfpathlineto{\pgfqpoint{3.816789in}{2.677291in}}%
\pgfpathclose%
\pgfusepath{fill}%
\end{pgfscope}%
\begin{pgfscope}%
\pgfpathrectangle{\pgfqpoint{0.017500in}{0.017500in}}{\pgfqpoint{6.218282in}{3.580372in}} %
\pgfusepath{clip}%
\pgfsetbuttcap%
\pgfsetroundjoin%
\definecolor{currentfill}{rgb}{0.735077,0.104460,0.171492}%
\pgfsetfillcolor{currentfill}%
\pgfsetlinewidth{0.000000pt}%
\definecolor{currentstroke}{rgb}{0.000000,0.000000,0.000000}%
\pgfsetstrokecolor{currentstroke}%
\pgfsetdash{}{0pt}%
\pgfpathmoveto{\pgfqpoint{3.816789in}{2.677291in}}%
\pgfpathlineto{\pgfqpoint{4.063701in}{2.813820in}}%
\pgfpathlineto{\pgfqpoint{3.816789in}{2.677291in}}%
\pgfpathclose%
\pgfusepath{fill}%
\end{pgfscope}%
\begin{pgfscope}%
\pgfpathrectangle{\pgfqpoint{0.017500in}{0.017500in}}{\pgfqpoint{6.218282in}{3.580372in}} %
\pgfusepath{clip}%
\pgfsetbuttcap%
\pgfsetroundjoin%
\definecolor{currentfill}{rgb}{0.735077,0.104460,0.171492}%
\pgfsetfillcolor{currentfill}%
\pgfsetlinewidth{0.000000pt}%
\definecolor{currentstroke}{rgb}{0.000000,0.000000,0.000000}%
\pgfsetstrokecolor{currentstroke}%
\pgfsetdash{}{0pt}%
\pgfpathmoveto{\pgfqpoint{3.816789in}{2.677291in}}%
\pgfpathlineto{\pgfqpoint{4.063701in}{2.813820in}}%
\pgfpathlineto{\pgfqpoint{3.816789in}{2.677291in}}%
\pgfpathclose%
\pgfusepath{fill}%
\end{pgfscope}%
\begin{pgfscope}%
\pgfpathrectangle{\pgfqpoint{0.017500in}{0.017500in}}{\pgfqpoint{6.218282in}{3.580372in}} %
\pgfusepath{clip}%
\pgfsetbuttcap%
\pgfsetroundjoin%
\definecolor{currentfill}{rgb}{0.735077,0.104460,0.171492}%
\pgfsetfillcolor{currentfill}%
\pgfsetlinewidth{0.000000pt}%
\definecolor{currentstroke}{rgb}{0.000000,0.000000,0.000000}%
\pgfsetstrokecolor{currentstroke}%
\pgfsetdash{}{0pt}%
\pgfpathmoveto{\pgfqpoint{3.816789in}{2.677291in}}%
\pgfpathlineto{\pgfqpoint{4.063701in}{2.813820in}}%
\pgfpathlineto{\pgfqpoint{3.816789in}{2.677291in}}%
\pgfpathclose%
\pgfusepath{fill}%
\end{pgfscope}%
\begin{pgfscope}%
\pgfpathrectangle{\pgfqpoint{0.017500in}{0.017500in}}{\pgfqpoint{6.218282in}{3.580372in}} %
\pgfusepath{clip}%
\pgfsetbuttcap%
\pgfsetroundjoin%
\definecolor{currentfill}{rgb}{0.735077,0.104460,0.171492}%
\pgfsetfillcolor{currentfill}%
\pgfsetlinewidth{0.000000pt}%
\definecolor{currentstroke}{rgb}{0.000000,0.000000,0.000000}%
\pgfsetstrokecolor{currentstroke}%
\pgfsetdash{}{0pt}%
\pgfpathmoveto{\pgfqpoint{3.816789in}{2.677291in}}%
\pgfpathlineto{\pgfqpoint{4.063701in}{2.813820in}}%
\pgfpathlineto{\pgfqpoint{3.816789in}{2.677291in}}%
\pgfpathclose%
\pgfusepath{fill}%
\end{pgfscope}%
\begin{pgfscope}%
\pgfpathrectangle{\pgfqpoint{0.017500in}{0.017500in}}{\pgfqpoint{6.218282in}{3.580372in}} %
\pgfusepath{clip}%
\pgfsetbuttcap%
\pgfsetroundjoin%
\definecolor{currentfill}{rgb}{0.735077,0.104460,0.171492}%
\pgfsetfillcolor{currentfill}%
\pgfsetlinewidth{0.000000pt}%
\definecolor{currentstroke}{rgb}{0.000000,0.000000,0.000000}%
\pgfsetstrokecolor{currentstroke}%
\pgfsetdash{}{0pt}%
\pgfpathmoveto{\pgfqpoint{3.816789in}{2.677291in}}%
\pgfpathlineto{\pgfqpoint{4.063701in}{2.813820in}}%
\pgfpathlineto{\pgfqpoint{3.816789in}{2.677291in}}%
\pgfpathclose%
\pgfusepath{fill}%
\end{pgfscope}%
\begin{pgfscope}%
\pgfpathrectangle{\pgfqpoint{0.017500in}{0.017500in}}{\pgfqpoint{6.218282in}{3.580372in}} %
\pgfusepath{clip}%
\pgfsetbuttcap%
\pgfsetroundjoin%
\definecolor{currentfill}{rgb}{0.735077,0.104460,0.171492}%
\pgfsetfillcolor{currentfill}%
\pgfsetlinewidth{0.000000pt}%
\definecolor{currentstroke}{rgb}{0.000000,0.000000,0.000000}%
\pgfsetstrokecolor{currentstroke}%
\pgfsetdash{}{0pt}%
\pgfpathmoveto{\pgfqpoint{3.816789in}{2.677291in}}%
\pgfpathlineto{\pgfqpoint{4.063701in}{2.813820in}}%
\pgfpathlineto{\pgfqpoint{3.816789in}{2.677291in}}%
\pgfpathclose%
\pgfusepath{fill}%
\end{pgfscope}%
\begin{pgfscope}%
\pgfpathrectangle{\pgfqpoint{0.017500in}{0.017500in}}{\pgfqpoint{6.218282in}{3.580372in}} %
\pgfusepath{clip}%
\pgfsetbuttcap%
\pgfsetroundjoin%
\definecolor{currentfill}{rgb}{0.735077,0.104460,0.171492}%
\pgfsetfillcolor{currentfill}%
\pgfsetlinewidth{0.000000pt}%
\definecolor{currentstroke}{rgb}{0.000000,0.000000,0.000000}%
\pgfsetstrokecolor{currentstroke}%
\pgfsetdash{}{0pt}%
\pgfpathmoveto{\pgfqpoint{3.816789in}{2.677291in}}%
\pgfpathlineto{\pgfqpoint{4.063701in}{2.813820in}}%
\pgfpathlineto{\pgfqpoint{3.816789in}{2.677291in}}%
\pgfpathclose%
\pgfusepath{fill}%
\end{pgfscope}%
\begin{pgfscope}%
\pgfpathrectangle{\pgfqpoint{0.017500in}{0.017500in}}{\pgfqpoint{6.218282in}{3.580372in}} %
\pgfusepath{clip}%
\pgfsetbuttcap%
\pgfsetroundjoin%
\definecolor{currentfill}{rgb}{0.735077,0.104460,0.171492}%
\pgfsetfillcolor{currentfill}%
\pgfsetlinewidth{0.000000pt}%
\definecolor{currentstroke}{rgb}{0.000000,0.000000,0.000000}%
\pgfsetstrokecolor{currentstroke}%
\pgfsetdash{}{0pt}%
\pgfpathmoveto{\pgfqpoint{3.816789in}{2.677291in}}%
\pgfpathlineto{\pgfqpoint{4.063701in}{2.813820in}}%
\pgfpathlineto{\pgfqpoint{3.816789in}{2.677291in}}%
\pgfpathclose%
\pgfusepath{fill}%
\end{pgfscope}%
\begin{pgfscope}%
\pgfpathrectangle{\pgfqpoint{0.017500in}{0.017500in}}{\pgfqpoint{6.218282in}{3.580372in}} %
\pgfusepath{clip}%
\pgfsetbuttcap%
\pgfsetroundjoin%
\definecolor{currentfill}{rgb}{0.735077,0.104460,0.171492}%
\pgfsetfillcolor{currentfill}%
\pgfsetlinewidth{0.000000pt}%
\definecolor{currentstroke}{rgb}{0.000000,0.000000,0.000000}%
\pgfsetstrokecolor{currentstroke}%
\pgfsetdash{}{0pt}%
\pgfpathmoveto{\pgfqpoint{3.816789in}{2.677291in}}%
\pgfpathlineto{\pgfqpoint{4.063701in}{2.813820in}}%
\pgfpathlineto{\pgfqpoint{3.816789in}{2.677291in}}%
\pgfpathclose%
\pgfusepath{fill}%
\end{pgfscope}%
\begin{pgfscope}%
\pgfpathrectangle{\pgfqpoint{0.017500in}{0.017500in}}{\pgfqpoint{6.218282in}{3.580372in}} %
\pgfusepath{clip}%
\pgfsetbuttcap%
\pgfsetroundjoin%
\definecolor{currentfill}{rgb}{0.735077,0.104460,0.171492}%
\pgfsetfillcolor{currentfill}%
\pgfsetlinewidth{0.000000pt}%
\definecolor{currentstroke}{rgb}{0.000000,0.000000,0.000000}%
\pgfsetstrokecolor{currentstroke}%
\pgfsetdash{}{0pt}%
\pgfpathmoveto{\pgfqpoint{3.816789in}{2.677291in}}%
\pgfpathlineto{\pgfqpoint{4.063701in}{2.813820in}}%
\pgfpathlineto{\pgfqpoint{3.816789in}{2.677291in}}%
\pgfpathclose%
\pgfusepath{fill}%
\end{pgfscope}%
\begin{pgfscope}%
\pgfpathrectangle{\pgfqpoint{0.017500in}{0.017500in}}{\pgfqpoint{6.218282in}{3.580372in}} %
\pgfusepath{clip}%
\pgfsetbuttcap%
\pgfsetroundjoin%
\definecolor{currentfill}{rgb}{0.735077,0.104460,0.171492}%
\pgfsetfillcolor{currentfill}%
\pgfsetlinewidth{0.000000pt}%
\definecolor{currentstroke}{rgb}{0.000000,0.000000,0.000000}%
\pgfsetstrokecolor{currentstroke}%
\pgfsetdash{}{0pt}%
\pgfpathmoveto{\pgfqpoint{3.816789in}{2.677291in}}%
\pgfpathlineto{\pgfqpoint{4.063701in}{2.813820in}}%
\pgfpathlineto{\pgfqpoint{3.816789in}{2.677291in}}%
\pgfpathclose%
\pgfusepath{fill}%
\end{pgfscope}%
\begin{pgfscope}%
\pgfpathrectangle{\pgfqpoint{0.017500in}{0.017500in}}{\pgfqpoint{6.218282in}{3.580372in}} %
\pgfusepath{clip}%
\pgfsetbuttcap%
\pgfsetroundjoin%
\definecolor{currentfill}{rgb}{0.735077,0.104460,0.171492}%
\pgfsetfillcolor{currentfill}%
\pgfsetlinewidth{0.000000pt}%
\definecolor{currentstroke}{rgb}{0.000000,0.000000,0.000000}%
\pgfsetstrokecolor{currentstroke}%
\pgfsetdash{}{0pt}%
\pgfpathmoveto{\pgfqpoint{3.816789in}{2.677291in}}%
\pgfpathlineto{\pgfqpoint{4.063701in}{2.813820in}}%
\pgfpathlineto{\pgfqpoint{3.816789in}{2.677291in}}%
\pgfpathclose%
\pgfusepath{fill}%
\end{pgfscope}%
\begin{pgfscope}%
\pgfpathrectangle{\pgfqpoint{0.017500in}{0.017500in}}{\pgfqpoint{6.218282in}{3.580372in}} %
\pgfusepath{clip}%
\pgfsetbuttcap%
\pgfsetroundjoin%
\definecolor{currentfill}{rgb}{0.735077,0.104460,0.171492}%
\pgfsetfillcolor{currentfill}%
\pgfsetlinewidth{0.000000pt}%
\definecolor{currentstroke}{rgb}{0.000000,0.000000,0.000000}%
\pgfsetstrokecolor{currentstroke}%
\pgfsetdash{}{0pt}%
\pgfpathmoveto{\pgfqpoint{3.816789in}{2.677291in}}%
\pgfpathlineto{\pgfqpoint{4.063701in}{2.813820in}}%
\pgfpathlineto{\pgfqpoint{3.816789in}{2.677291in}}%
\pgfpathclose%
\pgfusepath{fill}%
\end{pgfscope}%
\begin{pgfscope}%
\pgfpathrectangle{\pgfqpoint{0.017500in}{0.017500in}}{\pgfqpoint{6.218282in}{3.580372in}} %
\pgfusepath{clip}%
\pgfsetbuttcap%
\pgfsetroundjoin%
\definecolor{currentfill}{rgb}{0.735077,0.104460,0.171492}%
\pgfsetfillcolor{currentfill}%
\pgfsetlinewidth{0.000000pt}%
\definecolor{currentstroke}{rgb}{0.000000,0.000000,0.000000}%
\pgfsetstrokecolor{currentstroke}%
\pgfsetdash{}{0pt}%
\pgfpathmoveto{\pgfqpoint{3.816789in}{2.677291in}}%
\pgfpathlineto{\pgfqpoint{4.063701in}{2.813820in}}%
\pgfpathlineto{\pgfqpoint{3.816789in}{2.677291in}}%
\pgfpathclose%
\pgfusepath{fill}%
\end{pgfscope}%
\begin{pgfscope}%
\pgfpathrectangle{\pgfqpoint{0.017500in}{0.017500in}}{\pgfqpoint{6.218282in}{3.580372in}} %
\pgfusepath{clip}%
\pgfsetbuttcap%
\pgfsetroundjoin%
\definecolor{currentfill}{rgb}{0.805723,0.259813,0.230562}%
\pgfsetfillcolor{currentfill}%
\pgfsetlinewidth{0.000000pt}%
\definecolor{currentstroke}{rgb}{0.000000,0.000000,0.000000}%
\pgfsetstrokecolor{currentstroke}%
\pgfsetdash{}{0pt}%
\pgfpathmoveto{\pgfqpoint{1.971633in}{2.586781in}}%
\pgfpathlineto{\pgfqpoint{2.237924in}{2.752812in}}%
\pgfpathlineto{\pgfqpoint{1.971633in}{2.586781in}}%
\pgfpathclose%
\pgfusepath{fill}%
\end{pgfscope}%
\begin{pgfscope}%
\pgfpathrectangle{\pgfqpoint{0.017500in}{0.017500in}}{\pgfqpoint{6.218282in}{3.580372in}} %
\pgfusepath{clip}%
\pgfsetbuttcap%
\pgfsetroundjoin%
\definecolor{currentfill}{rgb}{0.805723,0.259813,0.230562}%
\pgfsetfillcolor{currentfill}%
\pgfsetlinewidth{0.000000pt}%
\definecolor{currentstroke}{rgb}{0.000000,0.000000,0.000000}%
\pgfsetstrokecolor{currentstroke}%
\pgfsetdash{}{0pt}%
\pgfpathmoveto{\pgfqpoint{1.971633in}{2.586781in}}%
\pgfpathlineto{\pgfqpoint{2.237924in}{2.752812in}}%
\pgfpathlineto{\pgfqpoint{1.971633in}{2.586781in}}%
\pgfpathclose%
\pgfusepath{fill}%
\end{pgfscope}%
\begin{pgfscope}%
\pgfpathrectangle{\pgfqpoint{0.017500in}{0.017500in}}{\pgfqpoint{6.218282in}{3.580372in}} %
\pgfusepath{clip}%
\pgfsetbuttcap%
\pgfsetroundjoin%
\definecolor{currentfill}{rgb}{0.805723,0.259813,0.230562}%
\pgfsetfillcolor{currentfill}%
\pgfsetlinewidth{0.000000pt}%
\definecolor{currentstroke}{rgb}{0.000000,0.000000,0.000000}%
\pgfsetstrokecolor{currentstroke}%
\pgfsetdash{}{0pt}%
\pgfpathmoveto{\pgfqpoint{1.971633in}{2.586781in}}%
\pgfpathlineto{\pgfqpoint{2.237924in}{2.752812in}}%
\pgfpathlineto{\pgfqpoint{1.971633in}{2.586781in}}%
\pgfpathclose%
\pgfusepath{fill}%
\end{pgfscope}%
\begin{pgfscope}%
\pgfpathrectangle{\pgfqpoint{0.017500in}{0.017500in}}{\pgfqpoint{6.218282in}{3.580372in}} %
\pgfusepath{clip}%
\pgfsetbuttcap%
\pgfsetroundjoin%
\definecolor{currentfill}{rgb}{0.805723,0.259813,0.230562}%
\pgfsetfillcolor{currentfill}%
\pgfsetlinewidth{0.000000pt}%
\definecolor{currentstroke}{rgb}{0.000000,0.000000,0.000000}%
\pgfsetstrokecolor{currentstroke}%
\pgfsetdash{}{0pt}%
\pgfpathmoveto{\pgfqpoint{1.971633in}{2.586781in}}%
\pgfpathlineto{\pgfqpoint{2.237924in}{2.752812in}}%
\pgfpathlineto{\pgfqpoint{1.971633in}{2.586781in}}%
\pgfpathclose%
\pgfusepath{fill}%
\end{pgfscope}%
\begin{pgfscope}%
\pgfpathrectangle{\pgfqpoint{0.017500in}{0.017500in}}{\pgfqpoint{6.218282in}{3.580372in}} %
\pgfusepath{clip}%
\pgfsetbuttcap%
\pgfsetroundjoin%
\definecolor{currentfill}{rgb}{0.805723,0.259813,0.230562}%
\pgfsetfillcolor{currentfill}%
\pgfsetlinewidth{0.000000pt}%
\definecolor{currentstroke}{rgb}{0.000000,0.000000,0.000000}%
\pgfsetstrokecolor{currentstroke}%
\pgfsetdash{}{0pt}%
\pgfpathmoveto{\pgfqpoint{1.971633in}{2.586781in}}%
\pgfpathlineto{\pgfqpoint{2.237924in}{2.752812in}}%
\pgfpathlineto{\pgfqpoint{1.971633in}{2.586781in}}%
\pgfpathclose%
\pgfusepath{fill}%
\end{pgfscope}%
\begin{pgfscope}%
\pgfpathrectangle{\pgfqpoint{0.017500in}{0.017500in}}{\pgfqpoint{6.218282in}{3.580372in}} %
\pgfusepath{clip}%
\pgfsetbuttcap%
\pgfsetroundjoin%
\definecolor{currentfill}{rgb}{0.805723,0.259813,0.230562}%
\pgfsetfillcolor{currentfill}%
\pgfsetlinewidth{0.000000pt}%
\definecolor{currentstroke}{rgb}{0.000000,0.000000,0.000000}%
\pgfsetstrokecolor{currentstroke}%
\pgfsetdash{}{0pt}%
\pgfpathmoveto{\pgfqpoint{1.971633in}{2.586781in}}%
\pgfpathlineto{\pgfqpoint{2.237924in}{2.752812in}}%
\pgfpathlineto{\pgfqpoint{1.971633in}{2.586781in}}%
\pgfpathclose%
\pgfusepath{fill}%
\end{pgfscope}%
\begin{pgfscope}%
\pgfpathrectangle{\pgfqpoint{0.017500in}{0.017500in}}{\pgfqpoint{6.218282in}{3.580372in}} %
\pgfusepath{clip}%
\pgfsetbuttcap%
\pgfsetroundjoin%
\definecolor{currentfill}{rgb}{0.805723,0.259813,0.230562}%
\pgfsetfillcolor{currentfill}%
\pgfsetlinewidth{0.000000pt}%
\definecolor{currentstroke}{rgb}{0.000000,0.000000,0.000000}%
\pgfsetstrokecolor{currentstroke}%
\pgfsetdash{}{0pt}%
\pgfpathmoveto{\pgfqpoint{1.971633in}{2.586781in}}%
\pgfpathlineto{\pgfqpoint{2.237924in}{2.752812in}}%
\pgfpathlineto{\pgfqpoint{1.971633in}{2.586781in}}%
\pgfpathclose%
\pgfusepath{fill}%
\end{pgfscope}%
\begin{pgfscope}%
\pgfpathrectangle{\pgfqpoint{0.017500in}{0.017500in}}{\pgfqpoint{6.218282in}{3.580372in}} %
\pgfusepath{clip}%
\pgfsetbuttcap%
\pgfsetroundjoin%
\definecolor{currentfill}{rgb}{0.805723,0.259813,0.230562}%
\pgfsetfillcolor{currentfill}%
\pgfsetlinewidth{0.000000pt}%
\definecolor{currentstroke}{rgb}{0.000000,0.000000,0.000000}%
\pgfsetstrokecolor{currentstroke}%
\pgfsetdash{}{0pt}%
\pgfpathmoveto{\pgfqpoint{1.971633in}{2.586781in}}%
\pgfpathlineto{\pgfqpoint{2.237924in}{2.752812in}}%
\pgfpathlineto{\pgfqpoint{1.971633in}{2.586781in}}%
\pgfpathclose%
\pgfusepath{fill}%
\end{pgfscope}%
\begin{pgfscope}%
\pgfpathrectangle{\pgfqpoint{0.017500in}{0.017500in}}{\pgfqpoint{6.218282in}{3.580372in}} %
\pgfusepath{clip}%
\pgfsetbuttcap%
\pgfsetroundjoin%
\definecolor{currentfill}{rgb}{0.805723,0.259813,0.230562}%
\pgfsetfillcolor{currentfill}%
\pgfsetlinewidth{0.000000pt}%
\definecolor{currentstroke}{rgb}{0.000000,0.000000,0.000000}%
\pgfsetstrokecolor{currentstroke}%
\pgfsetdash{}{0pt}%
\pgfpathmoveto{\pgfqpoint{1.971633in}{2.586781in}}%
\pgfpathlineto{\pgfqpoint{2.237924in}{2.752812in}}%
\pgfpathlineto{\pgfqpoint{1.971633in}{2.586781in}}%
\pgfpathclose%
\pgfusepath{fill}%
\end{pgfscope}%
\begin{pgfscope}%
\pgfpathrectangle{\pgfqpoint{0.017500in}{0.017500in}}{\pgfqpoint{6.218282in}{3.580372in}} %
\pgfusepath{clip}%
\pgfsetbuttcap%
\pgfsetroundjoin%
\definecolor{currentfill}{rgb}{0.805723,0.259813,0.230562}%
\pgfsetfillcolor{currentfill}%
\pgfsetlinewidth{0.000000pt}%
\definecolor{currentstroke}{rgb}{0.000000,0.000000,0.000000}%
\pgfsetstrokecolor{currentstroke}%
\pgfsetdash{}{0pt}%
\pgfpathmoveto{\pgfqpoint{1.971633in}{2.586781in}}%
\pgfpathlineto{\pgfqpoint{2.237924in}{2.752812in}}%
\pgfpathlineto{\pgfqpoint{1.971633in}{2.586781in}}%
\pgfpathclose%
\pgfusepath{fill}%
\end{pgfscope}%
\begin{pgfscope}%
\pgfpathrectangle{\pgfqpoint{0.017500in}{0.017500in}}{\pgfqpoint{6.218282in}{3.580372in}} %
\pgfusepath{clip}%
\pgfsetbuttcap%
\pgfsetroundjoin%
\definecolor{currentfill}{rgb}{0.805723,0.259813,0.230562}%
\pgfsetfillcolor{currentfill}%
\pgfsetlinewidth{0.000000pt}%
\definecolor{currentstroke}{rgb}{0.000000,0.000000,0.000000}%
\pgfsetstrokecolor{currentstroke}%
\pgfsetdash{}{0pt}%
\pgfpathmoveto{\pgfqpoint{1.971633in}{2.586781in}}%
\pgfpathlineto{\pgfqpoint{2.237924in}{2.752812in}}%
\pgfpathlineto{\pgfqpoint{1.971633in}{2.586781in}}%
\pgfpathclose%
\pgfusepath{fill}%
\end{pgfscope}%
\begin{pgfscope}%
\pgfpathrectangle{\pgfqpoint{0.017500in}{0.017500in}}{\pgfqpoint{6.218282in}{3.580372in}} %
\pgfusepath{clip}%
\pgfsetbuttcap%
\pgfsetroundjoin%
\definecolor{currentfill}{rgb}{0.805723,0.259813,0.230562}%
\pgfsetfillcolor{currentfill}%
\pgfsetlinewidth{0.000000pt}%
\definecolor{currentstroke}{rgb}{0.000000,0.000000,0.000000}%
\pgfsetstrokecolor{currentstroke}%
\pgfsetdash{}{0pt}%
\pgfpathmoveto{\pgfqpoint{1.971633in}{2.586781in}}%
\pgfpathlineto{\pgfqpoint{2.237924in}{2.752812in}}%
\pgfpathlineto{\pgfqpoint{1.971633in}{2.586781in}}%
\pgfpathclose%
\pgfusepath{fill}%
\end{pgfscope}%
\begin{pgfscope}%
\pgfpathrectangle{\pgfqpoint{0.017500in}{0.017500in}}{\pgfqpoint{6.218282in}{3.580372in}} %
\pgfusepath{clip}%
\pgfsetbuttcap%
\pgfsetroundjoin%
\definecolor{currentfill}{rgb}{0.805723,0.259813,0.230562}%
\pgfsetfillcolor{currentfill}%
\pgfsetlinewidth{0.000000pt}%
\definecolor{currentstroke}{rgb}{0.000000,0.000000,0.000000}%
\pgfsetstrokecolor{currentstroke}%
\pgfsetdash{}{0pt}%
\pgfpathmoveto{\pgfqpoint{1.971633in}{2.586781in}}%
\pgfpathlineto{\pgfqpoint{2.237924in}{2.752812in}}%
\pgfpathlineto{\pgfqpoint{1.971633in}{2.586781in}}%
\pgfpathclose%
\pgfusepath{fill}%
\end{pgfscope}%
\begin{pgfscope}%
\pgfpathrectangle{\pgfqpoint{0.017500in}{0.017500in}}{\pgfqpoint{6.218282in}{3.580372in}} %
\pgfusepath{clip}%
\pgfsetbuttcap%
\pgfsetroundjoin%
\definecolor{currentfill}{rgb}{0.805723,0.259813,0.230562}%
\pgfsetfillcolor{currentfill}%
\pgfsetlinewidth{0.000000pt}%
\definecolor{currentstroke}{rgb}{0.000000,0.000000,0.000000}%
\pgfsetstrokecolor{currentstroke}%
\pgfsetdash{}{0pt}%
\pgfpathmoveto{\pgfqpoint{1.971633in}{2.586781in}}%
\pgfpathlineto{\pgfqpoint{2.237924in}{2.752812in}}%
\pgfpathlineto{\pgfqpoint{1.971633in}{2.586781in}}%
\pgfpathclose%
\pgfusepath{fill}%
\end{pgfscope}%
\begin{pgfscope}%
\pgfpathrectangle{\pgfqpoint{0.017500in}{0.017500in}}{\pgfqpoint{6.218282in}{3.580372in}} %
\pgfusepath{clip}%
\pgfsetbuttcap%
\pgfsetroundjoin%
\definecolor{currentfill}{rgb}{0.805723,0.259813,0.230562}%
\pgfsetfillcolor{currentfill}%
\pgfsetlinewidth{0.000000pt}%
\definecolor{currentstroke}{rgb}{0.000000,0.000000,0.000000}%
\pgfsetstrokecolor{currentstroke}%
\pgfsetdash{}{0pt}%
\pgfpathmoveto{\pgfqpoint{1.971633in}{2.586781in}}%
\pgfpathlineto{\pgfqpoint{2.237924in}{2.752812in}}%
\pgfpathlineto{\pgfqpoint{1.971633in}{2.586781in}}%
\pgfpathclose%
\pgfusepath{fill}%
\end{pgfscope}%
\begin{pgfscope}%
\pgfpathrectangle{\pgfqpoint{0.017500in}{0.017500in}}{\pgfqpoint{6.218282in}{3.580372in}} %
\pgfusepath{clip}%
\pgfsetbuttcap%
\pgfsetroundjoin%
\definecolor{currentfill}{rgb}{0.805723,0.259813,0.230562}%
\pgfsetfillcolor{currentfill}%
\pgfsetlinewidth{0.000000pt}%
\definecolor{currentstroke}{rgb}{0.000000,0.000000,0.000000}%
\pgfsetstrokecolor{currentstroke}%
\pgfsetdash{}{0pt}%
\pgfpathmoveto{\pgfqpoint{1.971633in}{2.586781in}}%
\pgfpathlineto{\pgfqpoint{2.237924in}{2.752812in}}%
\pgfpathlineto{\pgfqpoint{1.971633in}{2.586781in}}%
\pgfpathclose%
\pgfusepath{fill}%
\end{pgfscope}%
\begin{pgfscope}%
\pgfpathrectangle{\pgfqpoint{0.017500in}{0.017500in}}{\pgfqpoint{6.218282in}{3.580372in}} %
\pgfusepath{clip}%
\pgfsetbuttcap%
\pgfsetroundjoin%
\definecolor{currentfill}{rgb}{0.805723,0.259813,0.230562}%
\pgfsetfillcolor{currentfill}%
\pgfsetlinewidth{0.000000pt}%
\definecolor{currentstroke}{rgb}{0.000000,0.000000,0.000000}%
\pgfsetstrokecolor{currentstroke}%
\pgfsetdash{}{0pt}%
\pgfpathmoveto{\pgfqpoint{1.971633in}{2.586781in}}%
\pgfpathlineto{\pgfqpoint{2.237924in}{2.752812in}}%
\pgfpathlineto{\pgfqpoint{1.971633in}{2.586781in}}%
\pgfpathclose%
\pgfusepath{fill}%
\end{pgfscope}%
\begin{pgfscope}%
\pgfpathrectangle{\pgfqpoint{0.017500in}{0.017500in}}{\pgfqpoint{6.218282in}{3.580372in}} %
\pgfusepath{clip}%
\pgfsetbuttcap%
\pgfsetroundjoin%
\definecolor{currentfill}{rgb}{0.805723,0.259813,0.230562}%
\pgfsetfillcolor{currentfill}%
\pgfsetlinewidth{0.000000pt}%
\definecolor{currentstroke}{rgb}{0.000000,0.000000,0.000000}%
\pgfsetstrokecolor{currentstroke}%
\pgfsetdash{}{0pt}%
\pgfpathmoveto{\pgfqpoint{1.971633in}{2.586781in}}%
\pgfpathlineto{\pgfqpoint{2.237924in}{2.752812in}}%
\pgfpathlineto{\pgfqpoint{1.971633in}{2.586781in}}%
\pgfpathclose%
\pgfusepath{fill}%
\end{pgfscope}%
\begin{pgfscope}%
\pgfpathrectangle{\pgfqpoint{0.017500in}{0.017500in}}{\pgfqpoint{6.218282in}{3.580372in}} %
\pgfusepath{clip}%
\pgfsetbuttcap%
\pgfsetroundjoin%
\definecolor{currentfill}{rgb}{0.805723,0.259813,0.230562}%
\pgfsetfillcolor{currentfill}%
\pgfsetlinewidth{0.000000pt}%
\definecolor{currentstroke}{rgb}{0.000000,0.000000,0.000000}%
\pgfsetstrokecolor{currentstroke}%
\pgfsetdash{}{0pt}%
\pgfpathmoveto{\pgfqpoint{1.971633in}{2.586781in}}%
\pgfpathlineto{\pgfqpoint{2.237924in}{2.752812in}}%
\pgfpathlineto{\pgfqpoint{1.971633in}{2.586781in}}%
\pgfpathclose%
\pgfusepath{fill}%
\end{pgfscope}%
\begin{pgfscope}%
\pgfpathrectangle{\pgfqpoint{0.017500in}{0.017500in}}{\pgfqpoint{6.218282in}{3.580372in}} %
\pgfusepath{clip}%
\pgfsetbuttcap%
\pgfsetroundjoin%
\definecolor{currentfill}{rgb}{0.805723,0.259813,0.230562}%
\pgfsetfillcolor{currentfill}%
\pgfsetlinewidth{0.000000pt}%
\definecolor{currentstroke}{rgb}{0.000000,0.000000,0.000000}%
\pgfsetstrokecolor{currentstroke}%
\pgfsetdash{}{0pt}%
\pgfpathmoveto{\pgfqpoint{1.971633in}{2.586781in}}%
\pgfpathlineto{\pgfqpoint{2.237924in}{2.752812in}}%
\pgfpathlineto{\pgfqpoint{1.971633in}{2.586781in}}%
\pgfpathclose%
\pgfusepath{fill}%
\end{pgfscope}%
\begin{pgfscope}%
\pgfpathrectangle{\pgfqpoint{0.017500in}{0.017500in}}{\pgfqpoint{6.218282in}{3.580372in}} %
\pgfusepath{clip}%
\pgfsetbuttcap%
\pgfsetroundjoin%
\definecolor{currentfill}{rgb}{0.805723,0.259813,0.230562}%
\pgfsetfillcolor{currentfill}%
\pgfsetlinewidth{0.000000pt}%
\definecolor{currentstroke}{rgb}{0.000000,0.000000,0.000000}%
\pgfsetstrokecolor{currentstroke}%
\pgfsetdash{}{0pt}%
\pgfpathmoveto{\pgfqpoint{1.971633in}{2.586781in}}%
\pgfpathlineto{\pgfqpoint{2.237924in}{2.752812in}}%
\pgfpathlineto{\pgfqpoint{1.971633in}{2.586781in}}%
\pgfpathclose%
\pgfusepath{fill}%
\end{pgfscope}%
\begin{pgfscope}%
\pgfpathrectangle{\pgfqpoint{0.017500in}{0.017500in}}{\pgfqpoint{6.218282in}{3.580372in}} %
\pgfusepath{clip}%
\pgfsetbuttcap%
\pgfsetroundjoin%
\definecolor{currentfill}{rgb}{0.805723,0.259813,0.230562}%
\pgfsetfillcolor{currentfill}%
\pgfsetlinewidth{0.000000pt}%
\definecolor{currentstroke}{rgb}{0.000000,0.000000,0.000000}%
\pgfsetstrokecolor{currentstroke}%
\pgfsetdash{}{0pt}%
\pgfpathmoveto{\pgfqpoint{1.971633in}{2.586781in}}%
\pgfpathlineto{\pgfqpoint{2.237924in}{2.752812in}}%
\pgfpathlineto{\pgfqpoint{1.971633in}{2.586781in}}%
\pgfpathclose%
\pgfusepath{fill}%
\end{pgfscope}%
\begin{pgfscope}%
\pgfpathrectangle{\pgfqpoint{0.017500in}{0.017500in}}{\pgfqpoint{6.218282in}{3.580372in}} %
\pgfusepath{clip}%
\pgfsetbuttcap%
\pgfsetroundjoin%
\definecolor{currentfill}{rgb}{0.805723,0.259813,0.230562}%
\pgfsetfillcolor{currentfill}%
\pgfsetlinewidth{0.000000pt}%
\definecolor{currentstroke}{rgb}{0.000000,0.000000,0.000000}%
\pgfsetstrokecolor{currentstroke}%
\pgfsetdash{}{0pt}%
\pgfpathmoveto{\pgfqpoint{1.971633in}{2.586781in}}%
\pgfpathlineto{\pgfqpoint{2.237924in}{2.752812in}}%
\pgfpathlineto{\pgfqpoint{1.971633in}{2.586781in}}%
\pgfpathclose%
\pgfusepath{fill}%
\end{pgfscope}%
\begin{pgfscope}%
\pgfpathrectangle{\pgfqpoint{0.017500in}{0.017500in}}{\pgfqpoint{6.218282in}{3.580372in}} %
\pgfusepath{clip}%
\pgfsetbuttcap%
\pgfsetroundjoin%
\definecolor{currentfill}{rgb}{0.805723,0.259813,0.230562}%
\pgfsetfillcolor{currentfill}%
\pgfsetlinewidth{0.000000pt}%
\definecolor{currentstroke}{rgb}{0.000000,0.000000,0.000000}%
\pgfsetstrokecolor{currentstroke}%
\pgfsetdash{}{0pt}%
\pgfpathmoveto{\pgfqpoint{1.971633in}{2.586781in}}%
\pgfpathlineto{\pgfqpoint{2.237924in}{2.752812in}}%
\pgfpathlineto{\pgfqpoint{1.971633in}{2.586781in}}%
\pgfpathclose%
\pgfusepath{fill}%
\end{pgfscope}%
\begin{pgfscope}%
\pgfpathrectangle{\pgfqpoint{0.017500in}{0.017500in}}{\pgfqpoint{6.218282in}{3.580372in}} %
\pgfusepath{clip}%
\pgfsetbuttcap%
\pgfsetroundjoin%
\definecolor{currentfill}{rgb}{0.805723,0.259813,0.230562}%
\pgfsetfillcolor{currentfill}%
\pgfsetlinewidth{0.000000pt}%
\definecolor{currentstroke}{rgb}{0.000000,0.000000,0.000000}%
\pgfsetstrokecolor{currentstroke}%
\pgfsetdash{}{0pt}%
\pgfpathmoveto{\pgfqpoint{1.971633in}{2.586781in}}%
\pgfpathlineto{\pgfqpoint{2.237924in}{2.752812in}}%
\pgfpathlineto{\pgfqpoint{1.971633in}{2.586781in}}%
\pgfpathclose%
\pgfusepath{fill}%
\end{pgfscope}%
\begin{pgfscope}%
\pgfpathrectangle{\pgfqpoint{0.017500in}{0.017500in}}{\pgfqpoint{6.218282in}{3.580372in}} %
\pgfusepath{clip}%
\pgfsetbuttcap%
\pgfsetroundjoin%
\definecolor{currentfill}{rgb}{0.805723,0.259813,0.230562}%
\pgfsetfillcolor{currentfill}%
\pgfsetlinewidth{0.000000pt}%
\definecolor{currentstroke}{rgb}{0.000000,0.000000,0.000000}%
\pgfsetstrokecolor{currentstroke}%
\pgfsetdash{}{0pt}%
\pgfpathmoveto{\pgfqpoint{1.971633in}{2.586781in}}%
\pgfpathlineto{\pgfqpoint{2.237924in}{2.752812in}}%
\pgfpathlineto{\pgfqpoint{1.971633in}{2.586781in}}%
\pgfpathclose%
\pgfusepath{fill}%
\end{pgfscope}%
\begin{pgfscope}%
\pgfpathrectangle{\pgfqpoint{0.017500in}{0.017500in}}{\pgfqpoint{6.218282in}{3.580372in}} %
\pgfusepath{clip}%
\pgfsetbuttcap%
\pgfsetroundjoin%
\definecolor{currentfill}{rgb}{0.805723,0.259813,0.230562}%
\pgfsetfillcolor{currentfill}%
\pgfsetlinewidth{0.000000pt}%
\definecolor{currentstroke}{rgb}{0.000000,0.000000,0.000000}%
\pgfsetstrokecolor{currentstroke}%
\pgfsetdash{}{0pt}%
\pgfpathmoveto{\pgfqpoint{1.971633in}{2.586781in}}%
\pgfpathlineto{\pgfqpoint{2.237924in}{2.752812in}}%
\pgfpathlineto{\pgfqpoint{1.971633in}{2.586781in}}%
\pgfpathclose%
\pgfusepath{fill}%
\end{pgfscope}%
\begin{pgfscope}%
\pgfpathrectangle{\pgfqpoint{0.017500in}{0.017500in}}{\pgfqpoint{6.218282in}{3.580372in}} %
\pgfusepath{clip}%
\pgfsetbuttcap%
\pgfsetroundjoin%
\definecolor{currentfill}{rgb}{0.805723,0.259813,0.230562}%
\pgfsetfillcolor{currentfill}%
\pgfsetlinewidth{0.000000pt}%
\definecolor{currentstroke}{rgb}{0.000000,0.000000,0.000000}%
\pgfsetstrokecolor{currentstroke}%
\pgfsetdash{}{0pt}%
\pgfpathmoveto{\pgfqpoint{1.971633in}{2.586781in}}%
\pgfpathlineto{\pgfqpoint{2.237924in}{2.752812in}}%
\pgfpathlineto{\pgfqpoint{1.971633in}{2.586781in}}%
\pgfpathclose%
\pgfusepath{fill}%
\end{pgfscope}%
\begin{pgfscope}%
\pgfpathrectangle{\pgfqpoint{0.017500in}{0.017500in}}{\pgfqpoint{6.218282in}{3.580372in}} %
\pgfusepath{clip}%
\pgfsetbuttcap%
\pgfsetroundjoin%
\definecolor{currentfill}{rgb}{0.805723,0.259813,0.230562}%
\pgfsetfillcolor{currentfill}%
\pgfsetlinewidth{0.000000pt}%
\definecolor{currentstroke}{rgb}{0.000000,0.000000,0.000000}%
\pgfsetstrokecolor{currentstroke}%
\pgfsetdash{}{0pt}%
\pgfpathmoveto{\pgfqpoint{1.971633in}{2.586781in}}%
\pgfpathlineto{\pgfqpoint{2.237924in}{2.752812in}}%
\pgfpathlineto{\pgfqpoint{1.971633in}{2.586781in}}%
\pgfpathclose%
\pgfusepath{fill}%
\end{pgfscope}%
\begin{pgfscope}%
\pgfpathrectangle{\pgfqpoint{0.017500in}{0.017500in}}{\pgfqpoint{6.218282in}{3.580372in}} %
\pgfusepath{clip}%
\pgfsetbuttcap%
\pgfsetroundjoin%
\definecolor{currentfill}{rgb}{0.805723,0.259813,0.230562}%
\pgfsetfillcolor{currentfill}%
\pgfsetlinewidth{0.000000pt}%
\definecolor{currentstroke}{rgb}{0.000000,0.000000,0.000000}%
\pgfsetstrokecolor{currentstroke}%
\pgfsetdash{}{0pt}%
\pgfpathmoveto{\pgfqpoint{1.971633in}{2.586781in}}%
\pgfpathlineto{\pgfqpoint{2.237924in}{2.752812in}}%
\pgfpathlineto{\pgfqpoint{1.971633in}{2.586781in}}%
\pgfpathclose%
\pgfusepath{fill}%
\end{pgfscope}%
\begin{pgfscope}%
\pgfpathrectangle{\pgfqpoint{0.017500in}{0.017500in}}{\pgfqpoint{6.218282in}{3.580372in}} %
\pgfusepath{clip}%
\pgfsetbuttcap%
\pgfsetroundjoin%
\definecolor{currentfill}{rgb}{0.343278,0.459354,0.884122}%
\pgfsetfillcolor{currentfill}%
\pgfsetlinewidth{0.000000pt}%
\definecolor{currentstroke}{rgb}{0.000000,0.000000,0.000000}%
\pgfsetstrokecolor{currentstroke}%
\pgfsetdash{}{0pt}%
\pgfpathmoveto{\pgfqpoint{1.486050in}{1.042638in}}%
\pgfpathlineto{\pgfqpoint{1.914830in}{0.967594in}}%
\pgfpathlineto{\pgfqpoint{2.143341in}{2.291288in}}%
\pgfpathlineto{\pgfqpoint{1.701778in}{2.375902in}}%
\pgfpathlineto{\pgfqpoint{1.486050in}{1.042638in}}%
\pgfpathclose%
\pgfusepath{fill}%
\end{pgfscope}%
\begin{pgfscope}%
\pgfpathrectangle{\pgfqpoint{0.017500in}{0.017500in}}{\pgfqpoint{6.218282in}{3.580372in}} %
\pgfusepath{clip}%
\pgfsetbuttcap%
\pgfsetroundjoin%
\definecolor{currentfill}{rgb}{0.343278,0.459354,0.884122}%
\pgfsetfillcolor{currentfill}%
\pgfsetlinewidth{0.000000pt}%
\definecolor{currentstroke}{rgb}{0.000000,0.000000,0.000000}%
\pgfsetstrokecolor{currentstroke}%
\pgfsetdash{}{0pt}%
\pgfpathmoveto{\pgfqpoint{1.486050in}{1.042638in}}%
\pgfpathlineto{\pgfqpoint{1.914830in}{0.967594in}}%
\pgfpathlineto{\pgfqpoint{2.143341in}{2.291288in}}%
\pgfpathlineto{\pgfqpoint{1.701778in}{2.375902in}}%
\pgfpathlineto{\pgfqpoint{1.486050in}{1.042638in}}%
\pgfpathclose%
\pgfusepath{fill}%
\end{pgfscope}%
\begin{pgfscope}%
\pgfpathrectangle{\pgfqpoint{0.017500in}{0.017500in}}{\pgfqpoint{6.218282in}{3.580372in}} %
\pgfusepath{clip}%
\pgfsetbuttcap%
\pgfsetroundjoin%
\definecolor{currentfill}{rgb}{0.343278,0.459354,0.884122}%
\pgfsetfillcolor{currentfill}%
\pgfsetlinewidth{0.000000pt}%
\definecolor{currentstroke}{rgb}{0.000000,0.000000,0.000000}%
\pgfsetstrokecolor{currentstroke}%
\pgfsetdash{}{0pt}%
\pgfpathmoveto{\pgfqpoint{1.486050in}{1.042638in}}%
\pgfpathlineto{\pgfqpoint{1.914830in}{0.967594in}}%
\pgfpathlineto{\pgfqpoint{2.143341in}{2.291288in}}%
\pgfpathlineto{\pgfqpoint{1.701778in}{2.375902in}}%
\pgfpathlineto{\pgfqpoint{1.486050in}{1.042638in}}%
\pgfpathclose%
\pgfusepath{fill}%
\end{pgfscope}%
\begin{pgfscope}%
\pgfpathrectangle{\pgfqpoint{0.017500in}{0.017500in}}{\pgfqpoint{6.218282in}{3.580372in}} %
\pgfusepath{clip}%
\pgfsetbuttcap%
\pgfsetroundjoin%
\definecolor{currentfill}{rgb}{0.343278,0.459354,0.884122}%
\pgfsetfillcolor{currentfill}%
\pgfsetlinewidth{0.000000pt}%
\definecolor{currentstroke}{rgb}{0.000000,0.000000,0.000000}%
\pgfsetstrokecolor{currentstroke}%
\pgfsetdash{}{0pt}%
\pgfpathmoveto{\pgfqpoint{1.486050in}{1.042638in}}%
\pgfpathlineto{\pgfqpoint{1.914830in}{0.967594in}}%
\pgfpathlineto{\pgfqpoint{2.143341in}{2.291288in}}%
\pgfpathlineto{\pgfqpoint{1.701778in}{2.375902in}}%
\pgfpathlineto{\pgfqpoint{1.486050in}{1.042638in}}%
\pgfpathclose%
\pgfusepath{fill}%
\end{pgfscope}%
\begin{pgfscope}%
\pgfpathrectangle{\pgfqpoint{0.017500in}{0.017500in}}{\pgfqpoint{6.218282in}{3.580372in}} %
\pgfusepath{clip}%
\pgfsetbuttcap%
\pgfsetroundjoin%
\definecolor{currentfill}{rgb}{0.343278,0.459354,0.884122}%
\pgfsetfillcolor{currentfill}%
\pgfsetlinewidth{0.000000pt}%
\definecolor{currentstroke}{rgb}{0.000000,0.000000,0.000000}%
\pgfsetstrokecolor{currentstroke}%
\pgfsetdash{}{0pt}%
\pgfpathmoveto{\pgfqpoint{1.486050in}{1.042638in}}%
\pgfpathlineto{\pgfqpoint{1.914830in}{0.967594in}}%
\pgfpathlineto{\pgfqpoint{2.143341in}{2.291288in}}%
\pgfpathlineto{\pgfqpoint{1.701778in}{2.375902in}}%
\pgfpathlineto{\pgfqpoint{1.486050in}{1.042638in}}%
\pgfpathclose%
\pgfusepath{fill}%
\end{pgfscope}%
\begin{pgfscope}%
\pgfpathrectangle{\pgfqpoint{0.017500in}{0.017500in}}{\pgfqpoint{6.218282in}{3.580372in}} %
\pgfusepath{clip}%
\pgfsetbuttcap%
\pgfsetroundjoin%
\definecolor{currentfill}{rgb}{0.343278,0.459354,0.884122}%
\pgfsetfillcolor{currentfill}%
\pgfsetlinewidth{0.000000pt}%
\definecolor{currentstroke}{rgb}{0.000000,0.000000,0.000000}%
\pgfsetstrokecolor{currentstroke}%
\pgfsetdash{}{0pt}%
\pgfpathmoveto{\pgfqpoint{1.486050in}{1.042638in}}%
\pgfpathlineto{\pgfqpoint{1.914830in}{0.967594in}}%
\pgfpathlineto{\pgfqpoint{2.143341in}{2.291288in}}%
\pgfpathlineto{\pgfqpoint{1.701778in}{2.375902in}}%
\pgfpathlineto{\pgfqpoint{1.486050in}{1.042638in}}%
\pgfpathclose%
\pgfusepath{fill}%
\end{pgfscope}%
\begin{pgfscope}%
\pgfpathrectangle{\pgfqpoint{0.017500in}{0.017500in}}{\pgfqpoint{6.218282in}{3.580372in}} %
\pgfusepath{clip}%
\pgfsetbuttcap%
\pgfsetroundjoin%
\definecolor{currentfill}{rgb}{0.800601,0.850358,0.930008}%
\pgfsetfillcolor{currentfill}%
\pgfsetlinewidth{0.000000pt}%
\definecolor{currentstroke}{rgb}{0.000000,0.000000,0.000000}%
\pgfsetstrokecolor{currentstroke}%
\pgfsetdash{}{0pt}%
\pgfpathmoveto{\pgfqpoint{4.063701in}{2.813820in}}%
\pgfpathlineto{\pgfqpoint{4.511410in}{2.753953in}}%
\pgfpathlineto{\pgfqpoint{3.236099in}{0.723497in}}%
\pgfpathlineto{\pgfqpoint{2.789950in}{0.798346in}}%
\pgfpathlineto{\pgfqpoint{4.063701in}{2.813820in}}%
\pgfpathclose%
\pgfusepath{fill}%
\end{pgfscope}%
\begin{pgfscope}%
\pgfpathrectangle{\pgfqpoint{0.017500in}{0.017500in}}{\pgfqpoint{6.218282in}{3.580372in}} %
\pgfusepath{clip}%
\pgfsetbuttcap%
\pgfsetroundjoin%
\definecolor{currentfill}{rgb}{0.800601,0.850358,0.930008}%
\pgfsetfillcolor{currentfill}%
\pgfsetlinewidth{0.000000pt}%
\definecolor{currentstroke}{rgb}{0.000000,0.000000,0.000000}%
\pgfsetstrokecolor{currentstroke}%
\pgfsetdash{}{0pt}%
\pgfpathmoveto{\pgfqpoint{4.063701in}{2.813820in}}%
\pgfpathlineto{\pgfqpoint{4.511410in}{2.753953in}}%
\pgfpathlineto{\pgfqpoint{3.236099in}{0.723497in}}%
\pgfpathlineto{\pgfqpoint{2.789950in}{0.798346in}}%
\pgfpathlineto{\pgfqpoint{4.063701in}{2.813820in}}%
\pgfpathclose%
\pgfusepath{fill}%
\end{pgfscope}%
\begin{pgfscope}%
\pgfpathrectangle{\pgfqpoint{0.017500in}{0.017500in}}{\pgfqpoint{6.218282in}{3.580372in}} %
\pgfusepath{clip}%
\pgfsetbuttcap%
\pgfsetroundjoin%
\definecolor{currentfill}{rgb}{0.800601,0.850358,0.930008}%
\pgfsetfillcolor{currentfill}%
\pgfsetlinewidth{0.000000pt}%
\definecolor{currentstroke}{rgb}{0.000000,0.000000,0.000000}%
\pgfsetstrokecolor{currentstroke}%
\pgfsetdash{}{0pt}%
\pgfpathmoveto{\pgfqpoint{4.063701in}{2.813820in}}%
\pgfpathlineto{\pgfqpoint{4.511410in}{2.753953in}}%
\pgfpathlineto{\pgfqpoint{3.236099in}{0.723497in}}%
\pgfpathlineto{\pgfqpoint{2.789950in}{0.798346in}}%
\pgfpathlineto{\pgfqpoint{4.063701in}{2.813820in}}%
\pgfpathclose%
\pgfusepath{fill}%
\end{pgfscope}%
\begin{pgfscope}%
\pgfpathrectangle{\pgfqpoint{0.017500in}{0.017500in}}{\pgfqpoint{6.218282in}{3.580372in}} %
\pgfusepath{clip}%
\pgfsetbuttcap%
\pgfsetroundjoin%
\definecolor{currentfill}{rgb}{0.800601,0.850358,0.930008}%
\pgfsetfillcolor{currentfill}%
\pgfsetlinewidth{0.000000pt}%
\definecolor{currentstroke}{rgb}{0.000000,0.000000,0.000000}%
\pgfsetstrokecolor{currentstroke}%
\pgfsetdash{}{0pt}%
\pgfpathmoveto{\pgfqpoint{4.063701in}{2.813820in}}%
\pgfpathlineto{\pgfqpoint{4.511410in}{2.753953in}}%
\pgfpathlineto{\pgfqpoint{3.236099in}{0.723497in}}%
\pgfpathlineto{\pgfqpoint{2.789950in}{0.798346in}}%
\pgfpathlineto{\pgfqpoint{4.063701in}{2.813820in}}%
\pgfpathclose%
\pgfusepath{fill}%
\end{pgfscope}%
\begin{pgfscope}%
\pgfpathrectangle{\pgfqpoint{0.017500in}{0.017500in}}{\pgfqpoint{6.218282in}{3.580372in}} %
\pgfusepath{clip}%
\pgfsetbuttcap%
\pgfsetroundjoin%
\definecolor{currentfill}{rgb}{0.800601,0.850358,0.930008}%
\pgfsetfillcolor{currentfill}%
\pgfsetlinewidth{0.000000pt}%
\definecolor{currentstroke}{rgb}{0.000000,0.000000,0.000000}%
\pgfsetstrokecolor{currentstroke}%
\pgfsetdash{}{0pt}%
\pgfpathmoveto{\pgfqpoint{4.063701in}{2.813820in}}%
\pgfpathlineto{\pgfqpoint{4.511410in}{2.753953in}}%
\pgfpathlineto{\pgfqpoint{3.236099in}{0.723497in}}%
\pgfpathlineto{\pgfqpoint{2.789950in}{0.798346in}}%
\pgfpathlineto{\pgfqpoint{4.063701in}{2.813820in}}%
\pgfpathclose%
\pgfusepath{fill}%
\end{pgfscope}%
\begin{pgfscope}%
\pgfpathrectangle{\pgfqpoint{0.017500in}{0.017500in}}{\pgfqpoint{6.218282in}{3.580372in}} %
\pgfusepath{clip}%
\pgfsetbuttcap%
\pgfsetroundjoin%
\definecolor{currentfill}{rgb}{0.729196,0.086679,0.167240}%
\pgfsetfillcolor{currentfill}%
\pgfsetlinewidth{0.000000pt}%
\definecolor{currentstroke}{rgb}{0.000000,0.000000,0.000000}%
\pgfsetstrokecolor{currentstroke}%
\pgfsetdash{}{0pt}%
\pgfpathmoveto{\pgfqpoint{3.816789in}{2.677291in}}%
\pgfpathlineto{\pgfqpoint{4.267628in}{2.613520in}}%
\pgfpathlineto{\pgfqpoint{4.511410in}{2.753953in}}%
\pgfpathlineto{\pgfqpoint{4.063701in}{2.813820in}}%
\pgfpathlineto{\pgfqpoint{3.816789in}{2.677291in}}%
\pgfpathclose%
\pgfusepath{fill}%
\end{pgfscope}%
\begin{pgfscope}%
\pgfpathrectangle{\pgfqpoint{0.017500in}{0.017500in}}{\pgfqpoint{6.218282in}{3.580372in}} %
\pgfusepath{clip}%
\pgfsetbuttcap%
\pgfsetroundjoin%
\definecolor{currentfill}{rgb}{0.729196,0.086679,0.167240}%
\pgfsetfillcolor{currentfill}%
\pgfsetlinewidth{0.000000pt}%
\definecolor{currentstroke}{rgb}{0.000000,0.000000,0.000000}%
\pgfsetstrokecolor{currentstroke}%
\pgfsetdash{}{0pt}%
\pgfpathmoveto{\pgfqpoint{3.816789in}{2.677291in}}%
\pgfpathlineto{\pgfqpoint{4.267628in}{2.613520in}}%
\pgfpathlineto{\pgfqpoint{4.511410in}{2.753953in}}%
\pgfpathlineto{\pgfqpoint{4.063701in}{2.813820in}}%
\pgfpathlineto{\pgfqpoint{3.816789in}{2.677291in}}%
\pgfpathclose%
\pgfusepath{fill}%
\end{pgfscope}%
\begin{pgfscope}%
\pgfpathrectangle{\pgfqpoint{0.017500in}{0.017500in}}{\pgfqpoint{6.218282in}{3.580372in}} %
\pgfusepath{clip}%
\pgfsetbuttcap%
\pgfsetroundjoin%
\definecolor{currentfill}{rgb}{0.729196,0.086679,0.167240}%
\pgfsetfillcolor{currentfill}%
\pgfsetlinewidth{0.000000pt}%
\definecolor{currentstroke}{rgb}{0.000000,0.000000,0.000000}%
\pgfsetstrokecolor{currentstroke}%
\pgfsetdash{}{0pt}%
\pgfpathmoveto{\pgfqpoint{3.816789in}{2.677291in}}%
\pgfpathlineto{\pgfqpoint{4.267628in}{2.613520in}}%
\pgfpathlineto{\pgfqpoint{4.511410in}{2.753953in}}%
\pgfpathlineto{\pgfqpoint{4.063701in}{2.813820in}}%
\pgfpathlineto{\pgfqpoint{3.816789in}{2.677291in}}%
\pgfpathclose%
\pgfusepath{fill}%
\end{pgfscope}%
\begin{pgfscope}%
\pgfpathrectangle{\pgfqpoint{0.017500in}{0.017500in}}{\pgfqpoint{6.218282in}{3.580372in}} %
\pgfusepath{clip}%
\pgfsetbuttcap%
\pgfsetroundjoin%
\definecolor{currentfill}{rgb}{0.729196,0.086679,0.167240}%
\pgfsetfillcolor{currentfill}%
\pgfsetlinewidth{0.000000pt}%
\definecolor{currentstroke}{rgb}{0.000000,0.000000,0.000000}%
\pgfsetstrokecolor{currentstroke}%
\pgfsetdash{}{0pt}%
\pgfpathmoveto{\pgfqpoint{3.816789in}{2.677291in}}%
\pgfpathlineto{\pgfqpoint{4.267628in}{2.613520in}}%
\pgfpathlineto{\pgfqpoint{4.511410in}{2.753953in}}%
\pgfpathlineto{\pgfqpoint{4.063701in}{2.813820in}}%
\pgfpathlineto{\pgfqpoint{3.816789in}{2.677291in}}%
\pgfpathclose%
\pgfusepath{fill}%
\end{pgfscope}%
\begin{pgfscope}%
\pgfpathrectangle{\pgfqpoint{0.017500in}{0.017500in}}{\pgfqpoint{6.218282in}{3.580372in}} %
\pgfusepath{clip}%
\pgfsetbuttcap%
\pgfsetroundjoin%
\definecolor{currentfill}{rgb}{0.729196,0.086679,0.167240}%
\pgfsetfillcolor{currentfill}%
\pgfsetlinewidth{0.000000pt}%
\definecolor{currentstroke}{rgb}{0.000000,0.000000,0.000000}%
\pgfsetstrokecolor{currentstroke}%
\pgfsetdash{}{0pt}%
\pgfpathmoveto{\pgfqpoint{3.816789in}{2.677291in}}%
\pgfpathlineto{\pgfqpoint{4.267628in}{2.613520in}}%
\pgfpathlineto{\pgfqpoint{4.511410in}{2.753953in}}%
\pgfpathlineto{\pgfqpoint{4.063701in}{2.813820in}}%
\pgfpathlineto{\pgfqpoint{3.816789in}{2.677291in}}%
\pgfpathclose%
\pgfusepath{fill}%
\end{pgfscope}%
\begin{pgfscope}%
\pgfpathrectangle{\pgfqpoint{0.017500in}{0.017500in}}{\pgfqpoint{6.218282in}{3.580372in}} %
\pgfusepath{clip}%
\pgfsetbuttcap%
\pgfsetroundjoin%
\definecolor{currentfill}{rgb}{0.729196,0.086679,0.167240}%
\pgfsetfillcolor{currentfill}%
\pgfsetlinewidth{0.000000pt}%
\definecolor{currentstroke}{rgb}{0.000000,0.000000,0.000000}%
\pgfsetstrokecolor{currentstroke}%
\pgfsetdash{}{0pt}%
\pgfpathmoveto{\pgfqpoint{3.816789in}{2.677291in}}%
\pgfpathlineto{\pgfqpoint{4.267628in}{2.613520in}}%
\pgfpathlineto{\pgfqpoint{4.511410in}{2.753953in}}%
\pgfpathlineto{\pgfqpoint{4.063701in}{2.813820in}}%
\pgfpathlineto{\pgfqpoint{3.816789in}{2.677291in}}%
\pgfpathclose%
\pgfusepath{fill}%
\end{pgfscope}%
\begin{pgfscope}%
\pgfpathrectangle{\pgfqpoint{0.017500in}{0.017500in}}{\pgfqpoint{6.218282in}{3.580372in}} %
\pgfusepath{clip}%
\pgfsetbuttcap%
\pgfsetroundjoin%
\definecolor{currentfill}{rgb}{0.800830,0.250829,0.225696}%
\pgfsetfillcolor{currentfill}%
\pgfsetlinewidth{0.000000pt}%
\definecolor{currentstroke}{rgb}{0.000000,0.000000,0.000000}%
\pgfsetstrokecolor{currentstroke}%
\pgfsetdash{}{0pt}%
\pgfpathmoveto{\pgfqpoint{1.971633in}{2.586781in}}%
\pgfpathlineto{\pgfqpoint{2.410872in}{2.508083in}}%
\pgfpathlineto{\pgfqpoint{2.674254in}{2.688576in}}%
\pgfpathlineto{\pgfqpoint{2.237924in}{2.752812in}}%
\pgfpathlineto{\pgfqpoint{1.971633in}{2.586781in}}%
\pgfpathclose%
\pgfusepath{fill}%
\end{pgfscope}%
\begin{pgfscope}%
\pgfpathrectangle{\pgfqpoint{0.017500in}{0.017500in}}{\pgfqpoint{6.218282in}{3.580372in}} %
\pgfusepath{clip}%
\pgfsetbuttcap%
\pgfsetroundjoin%
\definecolor{currentfill}{rgb}{0.800830,0.250829,0.225696}%
\pgfsetfillcolor{currentfill}%
\pgfsetlinewidth{0.000000pt}%
\definecolor{currentstroke}{rgb}{0.000000,0.000000,0.000000}%
\pgfsetstrokecolor{currentstroke}%
\pgfsetdash{}{0pt}%
\pgfpathmoveto{\pgfqpoint{1.971633in}{2.586781in}}%
\pgfpathlineto{\pgfqpoint{2.410872in}{2.508083in}}%
\pgfpathlineto{\pgfqpoint{2.674254in}{2.688576in}}%
\pgfpathlineto{\pgfqpoint{2.237924in}{2.752812in}}%
\pgfpathlineto{\pgfqpoint{1.971633in}{2.586781in}}%
\pgfpathclose%
\pgfusepath{fill}%
\end{pgfscope}%
\begin{pgfscope}%
\pgfpathrectangle{\pgfqpoint{0.017500in}{0.017500in}}{\pgfqpoint{6.218282in}{3.580372in}} %
\pgfusepath{clip}%
\pgfsetbuttcap%
\pgfsetroundjoin%
\definecolor{currentfill}{rgb}{0.800830,0.250829,0.225696}%
\pgfsetfillcolor{currentfill}%
\pgfsetlinewidth{0.000000pt}%
\definecolor{currentstroke}{rgb}{0.000000,0.000000,0.000000}%
\pgfsetstrokecolor{currentstroke}%
\pgfsetdash{}{0pt}%
\pgfpathmoveto{\pgfqpoint{1.971633in}{2.586781in}}%
\pgfpathlineto{\pgfqpoint{2.410872in}{2.508083in}}%
\pgfpathlineto{\pgfqpoint{2.674254in}{2.688576in}}%
\pgfpathlineto{\pgfqpoint{2.237924in}{2.752812in}}%
\pgfpathlineto{\pgfqpoint{1.971633in}{2.586781in}}%
\pgfpathclose%
\pgfusepath{fill}%
\end{pgfscope}%
\begin{pgfscope}%
\pgfpathrectangle{\pgfqpoint{0.017500in}{0.017500in}}{\pgfqpoint{6.218282in}{3.580372in}} %
\pgfusepath{clip}%
\pgfsetbuttcap%
\pgfsetroundjoin%
\definecolor{currentfill}{rgb}{0.800830,0.250829,0.225696}%
\pgfsetfillcolor{currentfill}%
\pgfsetlinewidth{0.000000pt}%
\definecolor{currentstroke}{rgb}{0.000000,0.000000,0.000000}%
\pgfsetstrokecolor{currentstroke}%
\pgfsetdash{}{0pt}%
\pgfpathmoveto{\pgfqpoint{1.971633in}{2.586781in}}%
\pgfpathlineto{\pgfqpoint{2.410872in}{2.508083in}}%
\pgfpathlineto{\pgfqpoint{2.674254in}{2.688576in}}%
\pgfpathlineto{\pgfqpoint{2.237924in}{2.752812in}}%
\pgfpathlineto{\pgfqpoint{1.971633in}{2.586781in}}%
\pgfpathclose%
\pgfusepath{fill}%
\end{pgfscope}%
\begin{pgfscope}%
\pgfpathrectangle{\pgfqpoint{0.017500in}{0.017500in}}{\pgfqpoint{6.218282in}{3.580372in}} %
\pgfusepath{clip}%
\pgfsetbuttcap%
\pgfsetroundjoin%
\definecolor{currentfill}{rgb}{0.800830,0.250829,0.225696}%
\pgfsetfillcolor{currentfill}%
\pgfsetlinewidth{0.000000pt}%
\definecolor{currentstroke}{rgb}{0.000000,0.000000,0.000000}%
\pgfsetstrokecolor{currentstroke}%
\pgfsetdash{}{0pt}%
\pgfpathmoveto{\pgfqpoint{1.971633in}{2.586781in}}%
\pgfpathlineto{\pgfqpoint{2.410872in}{2.508083in}}%
\pgfpathlineto{\pgfqpoint{2.674254in}{2.688576in}}%
\pgfpathlineto{\pgfqpoint{2.237924in}{2.752812in}}%
\pgfpathlineto{\pgfqpoint{1.971633in}{2.586781in}}%
\pgfpathclose%
\pgfusepath{fill}%
\end{pgfscope}%
\begin{pgfscope}%
\pgfpathrectangle{\pgfqpoint{0.017500in}{0.017500in}}{\pgfqpoint{6.218282in}{3.580372in}} %
\pgfusepath{clip}%
\pgfsetbuttcap%
\pgfsetroundjoin%
\definecolor{currentfill}{rgb}{0.800830,0.250829,0.225696}%
\pgfsetfillcolor{currentfill}%
\pgfsetlinewidth{0.000000pt}%
\definecolor{currentstroke}{rgb}{0.000000,0.000000,0.000000}%
\pgfsetstrokecolor{currentstroke}%
\pgfsetdash{}{0pt}%
\pgfpathmoveto{\pgfqpoint{1.971633in}{2.586781in}}%
\pgfpathlineto{\pgfqpoint{2.410872in}{2.508083in}}%
\pgfpathlineto{\pgfqpoint{2.674254in}{2.688576in}}%
\pgfpathlineto{\pgfqpoint{2.237924in}{2.752812in}}%
\pgfpathlineto{\pgfqpoint{1.971633in}{2.586781in}}%
\pgfpathclose%
\pgfusepath{fill}%
\end{pgfscope}%
\begin{pgfscope}%
\pgfpathrectangle{\pgfqpoint{0.017500in}{0.017500in}}{\pgfqpoint{6.218282in}{3.580372in}} %
\pgfusepath{clip}%
\pgfsetbuttcap%
\pgfsetroundjoin%
\definecolor{currentfill}{rgb}{0.243520,0.319189,0.771672}%
\pgfsetfillcolor{currentfill}%
\pgfsetlinewidth{0.000000pt}%
\definecolor{currentstroke}{rgb}{0.000000,0.000000,0.000000}%
\pgfsetstrokecolor{currentstroke}%
\pgfsetdash{}{0pt}%
\pgfpathmoveto{\pgfqpoint{1.914830in}{0.967594in}}%
\pgfpathlineto{\pgfqpoint{2.143341in}{2.291288in}}%
\pgfpathlineto{\pgfqpoint{1.914830in}{0.967594in}}%
\pgfpathclose%
\pgfusepath{fill}%
\end{pgfscope}%
\begin{pgfscope}%
\pgfpathrectangle{\pgfqpoint{0.017500in}{0.017500in}}{\pgfqpoint{6.218282in}{3.580372in}} %
\pgfusepath{clip}%
\pgfsetbuttcap%
\pgfsetroundjoin%
\definecolor{currentfill}{rgb}{0.243520,0.319189,0.771672}%
\pgfsetfillcolor{currentfill}%
\pgfsetlinewidth{0.000000pt}%
\definecolor{currentstroke}{rgb}{0.000000,0.000000,0.000000}%
\pgfsetstrokecolor{currentstroke}%
\pgfsetdash{}{0pt}%
\pgfpathmoveto{\pgfqpoint{1.914830in}{0.967594in}}%
\pgfpathlineto{\pgfqpoint{2.143341in}{2.291288in}}%
\pgfpathlineto{\pgfqpoint{1.914830in}{0.967594in}}%
\pgfpathclose%
\pgfusepath{fill}%
\end{pgfscope}%
\begin{pgfscope}%
\pgfpathrectangle{\pgfqpoint{0.017500in}{0.017500in}}{\pgfqpoint{6.218282in}{3.580372in}} %
\pgfusepath{clip}%
\pgfsetbuttcap%
\pgfsetroundjoin%
\definecolor{currentfill}{rgb}{0.243520,0.319189,0.771672}%
\pgfsetfillcolor{currentfill}%
\pgfsetlinewidth{0.000000pt}%
\definecolor{currentstroke}{rgb}{0.000000,0.000000,0.000000}%
\pgfsetstrokecolor{currentstroke}%
\pgfsetdash{}{0pt}%
\pgfpathmoveto{\pgfqpoint{1.914830in}{0.967594in}}%
\pgfpathlineto{\pgfqpoint{2.143341in}{2.291288in}}%
\pgfpathlineto{\pgfqpoint{1.914830in}{0.967594in}}%
\pgfpathclose%
\pgfusepath{fill}%
\end{pgfscope}%
\begin{pgfscope}%
\pgfpathrectangle{\pgfqpoint{0.017500in}{0.017500in}}{\pgfqpoint{6.218282in}{3.580372in}} %
\pgfusepath{clip}%
\pgfsetbuttcap%
\pgfsetroundjoin%
\definecolor{currentfill}{rgb}{0.243520,0.319189,0.771672}%
\pgfsetfillcolor{currentfill}%
\pgfsetlinewidth{0.000000pt}%
\definecolor{currentstroke}{rgb}{0.000000,0.000000,0.000000}%
\pgfsetstrokecolor{currentstroke}%
\pgfsetdash{}{0pt}%
\pgfpathmoveto{\pgfqpoint{1.914830in}{0.967594in}}%
\pgfpathlineto{\pgfqpoint{2.143341in}{2.291288in}}%
\pgfpathlineto{\pgfqpoint{1.914830in}{0.967594in}}%
\pgfpathclose%
\pgfusepath{fill}%
\end{pgfscope}%
\begin{pgfscope}%
\pgfpathrectangle{\pgfqpoint{0.017500in}{0.017500in}}{\pgfqpoint{6.218282in}{3.580372in}} %
\pgfusepath{clip}%
\pgfsetbuttcap%
\pgfsetroundjoin%
\definecolor{currentfill}{rgb}{0.243520,0.319189,0.771672}%
\pgfsetfillcolor{currentfill}%
\pgfsetlinewidth{0.000000pt}%
\definecolor{currentstroke}{rgb}{0.000000,0.000000,0.000000}%
\pgfsetstrokecolor{currentstroke}%
\pgfsetdash{}{0pt}%
\pgfpathmoveto{\pgfqpoint{1.914830in}{0.967594in}}%
\pgfpathlineto{\pgfqpoint{2.143341in}{2.291288in}}%
\pgfpathlineto{\pgfqpoint{1.914830in}{0.967594in}}%
\pgfpathclose%
\pgfusepath{fill}%
\end{pgfscope}%
\begin{pgfscope}%
\pgfpathrectangle{\pgfqpoint{0.017500in}{0.017500in}}{\pgfqpoint{6.218282in}{3.580372in}} %
\pgfusepath{clip}%
\pgfsetbuttcap%
\pgfsetroundjoin%
\definecolor{currentfill}{rgb}{0.243520,0.319189,0.771672}%
\pgfsetfillcolor{currentfill}%
\pgfsetlinewidth{0.000000pt}%
\definecolor{currentstroke}{rgb}{0.000000,0.000000,0.000000}%
\pgfsetstrokecolor{currentstroke}%
\pgfsetdash{}{0pt}%
\pgfpathmoveto{\pgfqpoint{1.914830in}{0.967594in}}%
\pgfpathlineto{\pgfqpoint{2.143341in}{2.291288in}}%
\pgfpathlineto{\pgfqpoint{1.914830in}{0.967594in}}%
\pgfpathclose%
\pgfusepath{fill}%
\end{pgfscope}%
\begin{pgfscope}%
\pgfpathrectangle{\pgfqpoint{0.017500in}{0.017500in}}{\pgfqpoint{6.218282in}{3.580372in}} %
\pgfusepath{clip}%
\pgfsetbuttcap%
\pgfsetroundjoin%
\definecolor{currentfill}{rgb}{0.243520,0.319189,0.771672}%
\pgfsetfillcolor{currentfill}%
\pgfsetlinewidth{0.000000pt}%
\definecolor{currentstroke}{rgb}{0.000000,0.000000,0.000000}%
\pgfsetstrokecolor{currentstroke}%
\pgfsetdash{}{0pt}%
\pgfpathmoveto{\pgfqpoint{1.914830in}{0.967594in}}%
\pgfpathlineto{\pgfqpoint{2.143341in}{2.291288in}}%
\pgfpathlineto{\pgfqpoint{1.914830in}{0.967594in}}%
\pgfpathclose%
\pgfusepath{fill}%
\end{pgfscope}%
\begin{pgfscope}%
\pgfpathrectangle{\pgfqpoint{0.017500in}{0.017500in}}{\pgfqpoint{6.218282in}{3.580372in}} %
\pgfusepath{clip}%
\pgfsetbuttcap%
\pgfsetroundjoin%
\definecolor{currentfill}{rgb}{0.243520,0.319189,0.771672}%
\pgfsetfillcolor{currentfill}%
\pgfsetlinewidth{0.000000pt}%
\definecolor{currentstroke}{rgb}{0.000000,0.000000,0.000000}%
\pgfsetstrokecolor{currentstroke}%
\pgfsetdash{}{0pt}%
\pgfpathmoveto{\pgfqpoint{1.914830in}{0.967594in}}%
\pgfpathlineto{\pgfqpoint{2.143341in}{2.291288in}}%
\pgfpathlineto{\pgfqpoint{1.914830in}{0.967594in}}%
\pgfpathclose%
\pgfusepath{fill}%
\end{pgfscope}%
\begin{pgfscope}%
\pgfpathrectangle{\pgfqpoint{0.017500in}{0.017500in}}{\pgfqpoint{6.218282in}{3.580372in}} %
\pgfusepath{clip}%
\pgfsetbuttcap%
\pgfsetroundjoin%
\definecolor{currentfill}{rgb}{0.243520,0.319189,0.771672}%
\pgfsetfillcolor{currentfill}%
\pgfsetlinewidth{0.000000pt}%
\definecolor{currentstroke}{rgb}{0.000000,0.000000,0.000000}%
\pgfsetstrokecolor{currentstroke}%
\pgfsetdash{}{0pt}%
\pgfpathmoveto{\pgfqpoint{1.914830in}{0.967594in}}%
\pgfpathlineto{\pgfqpoint{2.143341in}{2.291288in}}%
\pgfpathlineto{\pgfqpoint{1.914830in}{0.967594in}}%
\pgfpathclose%
\pgfusepath{fill}%
\end{pgfscope}%
\begin{pgfscope}%
\pgfpathrectangle{\pgfqpoint{0.017500in}{0.017500in}}{\pgfqpoint{6.218282in}{3.580372in}} %
\pgfusepath{clip}%
\pgfsetbuttcap%
\pgfsetroundjoin%
\definecolor{currentfill}{rgb}{0.243520,0.319189,0.771672}%
\pgfsetfillcolor{currentfill}%
\pgfsetlinewidth{0.000000pt}%
\definecolor{currentstroke}{rgb}{0.000000,0.000000,0.000000}%
\pgfsetstrokecolor{currentstroke}%
\pgfsetdash{}{0pt}%
\pgfpathmoveto{\pgfqpoint{1.914830in}{0.967594in}}%
\pgfpathlineto{\pgfqpoint{2.143341in}{2.291288in}}%
\pgfpathlineto{\pgfqpoint{1.914830in}{0.967594in}}%
\pgfpathclose%
\pgfusepath{fill}%
\end{pgfscope}%
\begin{pgfscope}%
\pgfpathrectangle{\pgfqpoint{0.017500in}{0.017500in}}{\pgfqpoint{6.218282in}{3.580372in}} %
\pgfusepath{clip}%
\pgfsetbuttcap%
\pgfsetroundjoin%
\definecolor{currentfill}{rgb}{0.243520,0.319189,0.771672}%
\pgfsetfillcolor{currentfill}%
\pgfsetlinewidth{0.000000pt}%
\definecolor{currentstroke}{rgb}{0.000000,0.000000,0.000000}%
\pgfsetstrokecolor{currentstroke}%
\pgfsetdash{}{0pt}%
\pgfpathmoveto{\pgfqpoint{1.914830in}{0.967594in}}%
\pgfpathlineto{\pgfqpoint{2.143341in}{2.291288in}}%
\pgfpathlineto{\pgfqpoint{1.914830in}{0.967594in}}%
\pgfpathclose%
\pgfusepath{fill}%
\end{pgfscope}%
\begin{pgfscope}%
\pgfpathrectangle{\pgfqpoint{0.017500in}{0.017500in}}{\pgfqpoint{6.218282in}{3.580372in}} %
\pgfusepath{clip}%
\pgfsetbuttcap%
\pgfsetroundjoin%
\definecolor{currentfill}{rgb}{0.243520,0.319189,0.771672}%
\pgfsetfillcolor{currentfill}%
\pgfsetlinewidth{0.000000pt}%
\definecolor{currentstroke}{rgb}{0.000000,0.000000,0.000000}%
\pgfsetstrokecolor{currentstroke}%
\pgfsetdash{}{0pt}%
\pgfpathmoveto{\pgfqpoint{1.914830in}{0.967594in}}%
\pgfpathlineto{\pgfqpoint{2.143341in}{2.291288in}}%
\pgfpathlineto{\pgfqpoint{1.914830in}{0.967594in}}%
\pgfpathclose%
\pgfusepath{fill}%
\end{pgfscope}%
\begin{pgfscope}%
\pgfpathrectangle{\pgfqpoint{0.017500in}{0.017500in}}{\pgfqpoint{6.218282in}{3.580372in}} %
\pgfusepath{clip}%
\pgfsetbuttcap%
\pgfsetroundjoin%
\definecolor{currentfill}{rgb}{0.243520,0.319189,0.771672}%
\pgfsetfillcolor{currentfill}%
\pgfsetlinewidth{0.000000pt}%
\definecolor{currentstroke}{rgb}{0.000000,0.000000,0.000000}%
\pgfsetstrokecolor{currentstroke}%
\pgfsetdash{}{0pt}%
\pgfpathmoveto{\pgfqpoint{1.914830in}{0.967594in}}%
\pgfpathlineto{\pgfqpoint{2.143341in}{2.291288in}}%
\pgfpathlineto{\pgfqpoint{1.914830in}{0.967594in}}%
\pgfpathclose%
\pgfusepath{fill}%
\end{pgfscope}%
\begin{pgfscope}%
\pgfpathrectangle{\pgfqpoint{0.017500in}{0.017500in}}{\pgfqpoint{6.218282in}{3.580372in}} %
\pgfusepath{clip}%
\pgfsetbuttcap%
\pgfsetroundjoin%
\definecolor{currentfill}{rgb}{0.243520,0.319189,0.771672}%
\pgfsetfillcolor{currentfill}%
\pgfsetlinewidth{0.000000pt}%
\definecolor{currentstroke}{rgb}{0.000000,0.000000,0.000000}%
\pgfsetstrokecolor{currentstroke}%
\pgfsetdash{}{0pt}%
\pgfpathmoveto{\pgfqpoint{1.914830in}{0.967594in}}%
\pgfpathlineto{\pgfqpoint{2.143341in}{2.291288in}}%
\pgfpathlineto{\pgfqpoint{1.914830in}{0.967594in}}%
\pgfpathclose%
\pgfusepath{fill}%
\end{pgfscope}%
\begin{pgfscope}%
\pgfpathrectangle{\pgfqpoint{0.017500in}{0.017500in}}{\pgfqpoint{6.218282in}{3.580372in}} %
\pgfusepath{clip}%
\pgfsetbuttcap%
\pgfsetroundjoin%
\definecolor{currentfill}{rgb}{0.243520,0.319189,0.771672}%
\pgfsetfillcolor{currentfill}%
\pgfsetlinewidth{0.000000pt}%
\definecolor{currentstroke}{rgb}{0.000000,0.000000,0.000000}%
\pgfsetstrokecolor{currentstroke}%
\pgfsetdash{}{0pt}%
\pgfpathmoveto{\pgfqpoint{1.914830in}{0.967594in}}%
\pgfpathlineto{\pgfqpoint{2.143341in}{2.291288in}}%
\pgfpathlineto{\pgfqpoint{1.914830in}{0.967594in}}%
\pgfpathclose%
\pgfusepath{fill}%
\end{pgfscope}%
\begin{pgfscope}%
\pgfpathrectangle{\pgfqpoint{0.017500in}{0.017500in}}{\pgfqpoint{6.218282in}{3.580372in}} %
\pgfusepath{clip}%
\pgfsetbuttcap%
\pgfsetroundjoin%
\definecolor{currentfill}{rgb}{0.243520,0.319189,0.771672}%
\pgfsetfillcolor{currentfill}%
\pgfsetlinewidth{0.000000pt}%
\definecolor{currentstroke}{rgb}{0.000000,0.000000,0.000000}%
\pgfsetstrokecolor{currentstroke}%
\pgfsetdash{}{0pt}%
\pgfpathmoveto{\pgfqpoint{1.914830in}{0.967594in}}%
\pgfpathlineto{\pgfqpoint{2.143341in}{2.291288in}}%
\pgfpathlineto{\pgfqpoint{1.914830in}{0.967594in}}%
\pgfpathclose%
\pgfusepath{fill}%
\end{pgfscope}%
\begin{pgfscope}%
\pgfpathrectangle{\pgfqpoint{0.017500in}{0.017500in}}{\pgfqpoint{6.218282in}{3.580372in}} %
\pgfusepath{clip}%
\pgfsetbuttcap%
\pgfsetroundjoin%
\definecolor{currentfill}{rgb}{0.243520,0.319189,0.771672}%
\pgfsetfillcolor{currentfill}%
\pgfsetlinewidth{0.000000pt}%
\definecolor{currentstroke}{rgb}{0.000000,0.000000,0.000000}%
\pgfsetstrokecolor{currentstroke}%
\pgfsetdash{}{0pt}%
\pgfpathmoveto{\pgfqpoint{1.914830in}{0.967594in}}%
\pgfpathlineto{\pgfqpoint{2.143341in}{2.291288in}}%
\pgfpathlineto{\pgfqpoint{1.914830in}{0.967594in}}%
\pgfpathclose%
\pgfusepath{fill}%
\end{pgfscope}%
\begin{pgfscope}%
\pgfpathrectangle{\pgfqpoint{0.017500in}{0.017500in}}{\pgfqpoint{6.218282in}{3.580372in}} %
\pgfusepath{clip}%
\pgfsetbuttcap%
\pgfsetroundjoin%
\definecolor{currentfill}{rgb}{0.243520,0.319189,0.771672}%
\pgfsetfillcolor{currentfill}%
\pgfsetlinewidth{0.000000pt}%
\definecolor{currentstroke}{rgb}{0.000000,0.000000,0.000000}%
\pgfsetstrokecolor{currentstroke}%
\pgfsetdash{}{0pt}%
\pgfpathmoveto{\pgfqpoint{1.914830in}{0.967594in}}%
\pgfpathlineto{\pgfqpoint{2.143341in}{2.291288in}}%
\pgfpathlineto{\pgfqpoint{1.914830in}{0.967594in}}%
\pgfpathclose%
\pgfusepath{fill}%
\end{pgfscope}%
\begin{pgfscope}%
\pgfpathrectangle{\pgfqpoint{0.017500in}{0.017500in}}{\pgfqpoint{6.218282in}{3.580372in}} %
\pgfusepath{clip}%
\pgfsetbuttcap%
\pgfsetroundjoin%
\definecolor{currentfill}{rgb}{0.243520,0.319189,0.771672}%
\pgfsetfillcolor{currentfill}%
\pgfsetlinewidth{0.000000pt}%
\definecolor{currentstroke}{rgb}{0.000000,0.000000,0.000000}%
\pgfsetstrokecolor{currentstroke}%
\pgfsetdash{}{0pt}%
\pgfpathmoveto{\pgfqpoint{1.914830in}{0.967594in}}%
\pgfpathlineto{\pgfqpoint{2.143341in}{2.291288in}}%
\pgfpathlineto{\pgfqpoint{1.914830in}{0.967594in}}%
\pgfpathclose%
\pgfusepath{fill}%
\end{pgfscope}%
\begin{pgfscope}%
\pgfpathrectangle{\pgfqpoint{0.017500in}{0.017500in}}{\pgfqpoint{6.218282in}{3.580372in}} %
\pgfusepath{clip}%
\pgfsetbuttcap%
\pgfsetroundjoin%
\definecolor{currentfill}{rgb}{0.243520,0.319189,0.771672}%
\pgfsetfillcolor{currentfill}%
\pgfsetlinewidth{0.000000pt}%
\definecolor{currentstroke}{rgb}{0.000000,0.000000,0.000000}%
\pgfsetstrokecolor{currentstroke}%
\pgfsetdash{}{0pt}%
\pgfpathmoveto{\pgfqpoint{1.914830in}{0.967594in}}%
\pgfpathlineto{\pgfqpoint{2.143341in}{2.291288in}}%
\pgfpathlineto{\pgfqpoint{1.914830in}{0.967594in}}%
\pgfpathclose%
\pgfusepath{fill}%
\end{pgfscope}%
\begin{pgfscope}%
\pgfpathrectangle{\pgfqpoint{0.017500in}{0.017500in}}{\pgfqpoint{6.218282in}{3.580372in}} %
\pgfusepath{clip}%
\pgfsetbuttcap%
\pgfsetroundjoin%
\definecolor{currentfill}{rgb}{0.243520,0.319189,0.771672}%
\pgfsetfillcolor{currentfill}%
\pgfsetlinewidth{0.000000pt}%
\definecolor{currentstroke}{rgb}{0.000000,0.000000,0.000000}%
\pgfsetstrokecolor{currentstroke}%
\pgfsetdash{}{0pt}%
\pgfpathmoveto{\pgfqpoint{1.914830in}{0.967594in}}%
\pgfpathlineto{\pgfqpoint{2.143341in}{2.291288in}}%
\pgfpathlineto{\pgfqpoint{1.914830in}{0.967594in}}%
\pgfpathclose%
\pgfusepath{fill}%
\end{pgfscope}%
\begin{pgfscope}%
\pgfpathrectangle{\pgfqpoint{0.017500in}{0.017500in}}{\pgfqpoint{6.218282in}{3.580372in}} %
\pgfusepath{clip}%
\pgfsetbuttcap%
\pgfsetroundjoin%
\definecolor{currentfill}{rgb}{0.243520,0.319189,0.771672}%
\pgfsetfillcolor{currentfill}%
\pgfsetlinewidth{0.000000pt}%
\definecolor{currentstroke}{rgb}{0.000000,0.000000,0.000000}%
\pgfsetstrokecolor{currentstroke}%
\pgfsetdash{}{0pt}%
\pgfpathmoveto{\pgfqpoint{1.914830in}{0.967594in}}%
\pgfpathlineto{\pgfqpoint{2.143341in}{2.291288in}}%
\pgfpathlineto{\pgfqpoint{1.914830in}{0.967594in}}%
\pgfpathclose%
\pgfusepath{fill}%
\end{pgfscope}%
\begin{pgfscope}%
\pgfpathrectangle{\pgfqpoint{0.017500in}{0.017500in}}{\pgfqpoint{6.218282in}{3.580372in}} %
\pgfusepath{clip}%
\pgfsetbuttcap%
\pgfsetroundjoin%
\definecolor{currentfill}{rgb}{0.243520,0.319189,0.771672}%
\pgfsetfillcolor{currentfill}%
\pgfsetlinewidth{0.000000pt}%
\definecolor{currentstroke}{rgb}{0.000000,0.000000,0.000000}%
\pgfsetstrokecolor{currentstroke}%
\pgfsetdash{}{0pt}%
\pgfpathmoveto{\pgfqpoint{1.914830in}{0.967594in}}%
\pgfpathlineto{\pgfqpoint{2.143341in}{2.291288in}}%
\pgfpathlineto{\pgfqpoint{1.914830in}{0.967594in}}%
\pgfpathclose%
\pgfusepath{fill}%
\end{pgfscope}%
\begin{pgfscope}%
\pgfpathrectangle{\pgfqpoint{0.017500in}{0.017500in}}{\pgfqpoint{6.218282in}{3.580372in}} %
\pgfusepath{clip}%
\pgfsetbuttcap%
\pgfsetroundjoin%
\definecolor{currentfill}{rgb}{0.243520,0.319189,0.771672}%
\pgfsetfillcolor{currentfill}%
\pgfsetlinewidth{0.000000pt}%
\definecolor{currentstroke}{rgb}{0.000000,0.000000,0.000000}%
\pgfsetstrokecolor{currentstroke}%
\pgfsetdash{}{0pt}%
\pgfpathmoveto{\pgfqpoint{1.914830in}{0.967594in}}%
\pgfpathlineto{\pgfqpoint{2.143341in}{2.291288in}}%
\pgfpathlineto{\pgfqpoint{1.914830in}{0.967594in}}%
\pgfpathclose%
\pgfusepath{fill}%
\end{pgfscope}%
\begin{pgfscope}%
\pgfpathrectangle{\pgfqpoint{0.017500in}{0.017500in}}{\pgfqpoint{6.218282in}{3.580372in}} %
\pgfusepath{clip}%
\pgfsetbuttcap%
\pgfsetroundjoin%
\definecolor{currentfill}{rgb}{0.243520,0.319189,0.771672}%
\pgfsetfillcolor{currentfill}%
\pgfsetlinewidth{0.000000pt}%
\definecolor{currentstroke}{rgb}{0.000000,0.000000,0.000000}%
\pgfsetstrokecolor{currentstroke}%
\pgfsetdash{}{0pt}%
\pgfpathmoveto{\pgfqpoint{1.914830in}{0.967594in}}%
\pgfpathlineto{\pgfqpoint{2.143341in}{2.291288in}}%
\pgfpathlineto{\pgfqpoint{1.914830in}{0.967594in}}%
\pgfpathclose%
\pgfusepath{fill}%
\end{pgfscope}%
\begin{pgfscope}%
\pgfpathrectangle{\pgfqpoint{0.017500in}{0.017500in}}{\pgfqpoint{6.218282in}{3.580372in}} %
\pgfusepath{clip}%
\pgfsetbuttcap%
\pgfsetroundjoin%
\definecolor{currentfill}{rgb}{0.243520,0.319189,0.771672}%
\pgfsetfillcolor{currentfill}%
\pgfsetlinewidth{0.000000pt}%
\definecolor{currentstroke}{rgb}{0.000000,0.000000,0.000000}%
\pgfsetstrokecolor{currentstroke}%
\pgfsetdash{}{0pt}%
\pgfpathmoveto{\pgfqpoint{1.914830in}{0.967594in}}%
\pgfpathlineto{\pgfqpoint{2.143341in}{2.291288in}}%
\pgfpathlineto{\pgfqpoint{1.914830in}{0.967594in}}%
\pgfpathclose%
\pgfusepath{fill}%
\end{pgfscope}%
\begin{pgfscope}%
\pgfpathrectangle{\pgfqpoint{0.017500in}{0.017500in}}{\pgfqpoint{6.218282in}{3.580372in}} %
\pgfusepath{clip}%
\pgfsetbuttcap%
\pgfsetroundjoin%
\definecolor{currentfill}{rgb}{0.243520,0.319189,0.771672}%
\pgfsetfillcolor{currentfill}%
\pgfsetlinewidth{0.000000pt}%
\definecolor{currentstroke}{rgb}{0.000000,0.000000,0.000000}%
\pgfsetstrokecolor{currentstroke}%
\pgfsetdash{}{0pt}%
\pgfpathmoveto{\pgfqpoint{1.914830in}{0.967594in}}%
\pgfpathlineto{\pgfqpoint{2.143341in}{2.291288in}}%
\pgfpathlineto{\pgfqpoint{1.914830in}{0.967594in}}%
\pgfpathclose%
\pgfusepath{fill}%
\end{pgfscope}%
\begin{pgfscope}%
\pgfpathrectangle{\pgfqpoint{0.017500in}{0.017500in}}{\pgfqpoint{6.218282in}{3.580372in}} %
\pgfusepath{clip}%
\pgfsetbuttcap%
\pgfsetroundjoin%
\definecolor{currentfill}{rgb}{0.243520,0.319189,0.771672}%
\pgfsetfillcolor{currentfill}%
\pgfsetlinewidth{0.000000pt}%
\definecolor{currentstroke}{rgb}{0.000000,0.000000,0.000000}%
\pgfsetstrokecolor{currentstroke}%
\pgfsetdash{}{0pt}%
\pgfpathmoveto{\pgfqpoint{1.914830in}{0.967594in}}%
\pgfpathlineto{\pgfqpoint{2.143341in}{2.291288in}}%
\pgfpathlineto{\pgfqpoint{1.914830in}{0.967594in}}%
\pgfpathclose%
\pgfusepath{fill}%
\end{pgfscope}%
\begin{pgfscope}%
\pgfpathrectangle{\pgfqpoint{0.017500in}{0.017500in}}{\pgfqpoint{6.218282in}{3.580372in}} %
\pgfusepath{clip}%
\pgfsetbuttcap%
\pgfsetroundjoin%
\definecolor{currentfill}{rgb}{0.243520,0.319189,0.771672}%
\pgfsetfillcolor{currentfill}%
\pgfsetlinewidth{0.000000pt}%
\definecolor{currentstroke}{rgb}{0.000000,0.000000,0.000000}%
\pgfsetstrokecolor{currentstroke}%
\pgfsetdash{}{0pt}%
\pgfpathmoveto{\pgfqpoint{1.914830in}{0.967594in}}%
\pgfpathlineto{\pgfqpoint{2.143341in}{2.291288in}}%
\pgfpathlineto{\pgfqpoint{1.914830in}{0.967594in}}%
\pgfpathclose%
\pgfusepath{fill}%
\end{pgfscope}%
\begin{pgfscope}%
\pgfpathrectangle{\pgfqpoint{0.017500in}{0.017500in}}{\pgfqpoint{6.218282in}{3.580372in}} %
\pgfusepath{clip}%
\pgfsetbuttcap%
\pgfsetroundjoin%
\definecolor{currentfill}{rgb}{0.243520,0.319189,0.771672}%
\pgfsetfillcolor{currentfill}%
\pgfsetlinewidth{0.000000pt}%
\definecolor{currentstroke}{rgb}{0.000000,0.000000,0.000000}%
\pgfsetstrokecolor{currentstroke}%
\pgfsetdash{}{0pt}%
\pgfpathmoveto{\pgfqpoint{1.914830in}{0.967594in}}%
\pgfpathlineto{\pgfqpoint{2.143341in}{2.291288in}}%
\pgfpathlineto{\pgfqpoint{1.914830in}{0.967594in}}%
\pgfpathclose%
\pgfusepath{fill}%
\end{pgfscope}%
\begin{pgfscope}%
\pgfpathrectangle{\pgfqpoint{0.017500in}{0.017500in}}{\pgfqpoint{6.218282in}{3.580372in}} %
\pgfusepath{clip}%
\pgfsetbuttcap%
\pgfsetroundjoin%
\definecolor{currentfill}{rgb}{0.758112,0.168122,0.188827}%
\pgfsetfillcolor{currentfill}%
\pgfsetlinewidth{0.000000pt}%
\definecolor{currentstroke}{rgb}{0.000000,0.000000,0.000000}%
\pgfsetstrokecolor{currentstroke}%
\pgfsetdash{}{0pt}%
\pgfpathmoveto{\pgfqpoint{3.116513in}{2.603812in}}%
\pgfpathlineto{\pgfqpoint{3.372093in}{2.761754in}}%
\pgfpathlineto{\pgfqpoint{3.116513in}{2.603812in}}%
\pgfpathclose%
\pgfusepath{fill}%
\end{pgfscope}%
\begin{pgfscope}%
\pgfpathrectangle{\pgfqpoint{0.017500in}{0.017500in}}{\pgfqpoint{6.218282in}{3.580372in}} %
\pgfusepath{clip}%
\pgfsetbuttcap%
\pgfsetroundjoin%
\definecolor{currentfill}{rgb}{0.758112,0.168122,0.188827}%
\pgfsetfillcolor{currentfill}%
\pgfsetlinewidth{0.000000pt}%
\definecolor{currentstroke}{rgb}{0.000000,0.000000,0.000000}%
\pgfsetstrokecolor{currentstroke}%
\pgfsetdash{}{0pt}%
\pgfpathmoveto{\pgfqpoint{3.116513in}{2.603812in}}%
\pgfpathlineto{\pgfqpoint{3.372093in}{2.761754in}}%
\pgfpathlineto{\pgfqpoint{3.116513in}{2.603812in}}%
\pgfpathclose%
\pgfusepath{fill}%
\end{pgfscope}%
\begin{pgfscope}%
\pgfpathrectangle{\pgfqpoint{0.017500in}{0.017500in}}{\pgfqpoint{6.218282in}{3.580372in}} %
\pgfusepath{clip}%
\pgfsetbuttcap%
\pgfsetroundjoin%
\definecolor{currentfill}{rgb}{0.758112,0.168122,0.188827}%
\pgfsetfillcolor{currentfill}%
\pgfsetlinewidth{0.000000pt}%
\definecolor{currentstroke}{rgb}{0.000000,0.000000,0.000000}%
\pgfsetstrokecolor{currentstroke}%
\pgfsetdash{}{0pt}%
\pgfpathmoveto{\pgfqpoint{3.116513in}{2.603812in}}%
\pgfpathlineto{\pgfqpoint{3.372093in}{2.761754in}}%
\pgfpathlineto{\pgfqpoint{3.116513in}{2.603812in}}%
\pgfpathclose%
\pgfusepath{fill}%
\end{pgfscope}%
\begin{pgfscope}%
\pgfpathrectangle{\pgfqpoint{0.017500in}{0.017500in}}{\pgfqpoint{6.218282in}{3.580372in}} %
\pgfusepath{clip}%
\pgfsetbuttcap%
\pgfsetroundjoin%
\definecolor{currentfill}{rgb}{0.758112,0.168122,0.188827}%
\pgfsetfillcolor{currentfill}%
\pgfsetlinewidth{0.000000pt}%
\definecolor{currentstroke}{rgb}{0.000000,0.000000,0.000000}%
\pgfsetstrokecolor{currentstroke}%
\pgfsetdash{}{0pt}%
\pgfpathmoveto{\pgfqpoint{3.116513in}{2.603812in}}%
\pgfpathlineto{\pgfqpoint{3.372093in}{2.761754in}}%
\pgfpathlineto{\pgfqpoint{3.116513in}{2.603812in}}%
\pgfpathclose%
\pgfusepath{fill}%
\end{pgfscope}%
\begin{pgfscope}%
\pgfpathrectangle{\pgfqpoint{0.017500in}{0.017500in}}{\pgfqpoint{6.218282in}{3.580372in}} %
\pgfusepath{clip}%
\pgfsetbuttcap%
\pgfsetroundjoin%
\definecolor{currentfill}{rgb}{0.758112,0.168122,0.188827}%
\pgfsetfillcolor{currentfill}%
\pgfsetlinewidth{0.000000pt}%
\definecolor{currentstroke}{rgb}{0.000000,0.000000,0.000000}%
\pgfsetstrokecolor{currentstroke}%
\pgfsetdash{}{0pt}%
\pgfpathmoveto{\pgfqpoint{3.116513in}{2.603812in}}%
\pgfpathlineto{\pgfqpoint{3.372093in}{2.761754in}}%
\pgfpathlineto{\pgfqpoint{3.116513in}{2.603812in}}%
\pgfpathclose%
\pgfusepath{fill}%
\end{pgfscope}%
\begin{pgfscope}%
\pgfpathrectangle{\pgfqpoint{0.017500in}{0.017500in}}{\pgfqpoint{6.218282in}{3.580372in}} %
\pgfusepath{clip}%
\pgfsetbuttcap%
\pgfsetroundjoin%
\definecolor{currentfill}{rgb}{0.758112,0.168122,0.188827}%
\pgfsetfillcolor{currentfill}%
\pgfsetlinewidth{0.000000pt}%
\definecolor{currentstroke}{rgb}{0.000000,0.000000,0.000000}%
\pgfsetstrokecolor{currentstroke}%
\pgfsetdash{}{0pt}%
\pgfpathmoveto{\pgfqpoint{3.116513in}{2.603812in}}%
\pgfpathlineto{\pgfqpoint{3.372093in}{2.761754in}}%
\pgfpathlineto{\pgfqpoint{3.116513in}{2.603812in}}%
\pgfpathclose%
\pgfusepath{fill}%
\end{pgfscope}%
\begin{pgfscope}%
\pgfpathrectangle{\pgfqpoint{0.017500in}{0.017500in}}{\pgfqpoint{6.218282in}{3.580372in}} %
\pgfusepath{clip}%
\pgfsetbuttcap%
\pgfsetroundjoin%
\definecolor{currentfill}{rgb}{0.758112,0.168122,0.188827}%
\pgfsetfillcolor{currentfill}%
\pgfsetlinewidth{0.000000pt}%
\definecolor{currentstroke}{rgb}{0.000000,0.000000,0.000000}%
\pgfsetstrokecolor{currentstroke}%
\pgfsetdash{}{0pt}%
\pgfpathmoveto{\pgfqpoint{3.116513in}{2.603812in}}%
\pgfpathlineto{\pgfqpoint{3.372093in}{2.761754in}}%
\pgfpathlineto{\pgfqpoint{3.116513in}{2.603812in}}%
\pgfpathclose%
\pgfusepath{fill}%
\end{pgfscope}%
\begin{pgfscope}%
\pgfpathrectangle{\pgfqpoint{0.017500in}{0.017500in}}{\pgfqpoint{6.218282in}{3.580372in}} %
\pgfusepath{clip}%
\pgfsetbuttcap%
\pgfsetroundjoin%
\definecolor{currentfill}{rgb}{0.758112,0.168122,0.188827}%
\pgfsetfillcolor{currentfill}%
\pgfsetlinewidth{0.000000pt}%
\definecolor{currentstroke}{rgb}{0.000000,0.000000,0.000000}%
\pgfsetstrokecolor{currentstroke}%
\pgfsetdash{}{0pt}%
\pgfpathmoveto{\pgfqpoint{3.116513in}{2.603812in}}%
\pgfpathlineto{\pgfqpoint{3.372093in}{2.761754in}}%
\pgfpathlineto{\pgfqpoint{3.116513in}{2.603812in}}%
\pgfpathclose%
\pgfusepath{fill}%
\end{pgfscope}%
\begin{pgfscope}%
\pgfpathrectangle{\pgfqpoint{0.017500in}{0.017500in}}{\pgfqpoint{6.218282in}{3.580372in}} %
\pgfusepath{clip}%
\pgfsetbuttcap%
\pgfsetroundjoin%
\definecolor{currentfill}{rgb}{0.758112,0.168122,0.188827}%
\pgfsetfillcolor{currentfill}%
\pgfsetlinewidth{0.000000pt}%
\definecolor{currentstroke}{rgb}{0.000000,0.000000,0.000000}%
\pgfsetstrokecolor{currentstroke}%
\pgfsetdash{}{0pt}%
\pgfpathmoveto{\pgfqpoint{3.116513in}{2.603812in}}%
\pgfpathlineto{\pgfqpoint{3.372093in}{2.761754in}}%
\pgfpathlineto{\pgfqpoint{3.116513in}{2.603812in}}%
\pgfpathclose%
\pgfusepath{fill}%
\end{pgfscope}%
\begin{pgfscope}%
\pgfpathrectangle{\pgfqpoint{0.017500in}{0.017500in}}{\pgfqpoint{6.218282in}{3.580372in}} %
\pgfusepath{clip}%
\pgfsetbuttcap%
\pgfsetroundjoin%
\definecolor{currentfill}{rgb}{0.758112,0.168122,0.188827}%
\pgfsetfillcolor{currentfill}%
\pgfsetlinewidth{0.000000pt}%
\definecolor{currentstroke}{rgb}{0.000000,0.000000,0.000000}%
\pgfsetstrokecolor{currentstroke}%
\pgfsetdash{}{0pt}%
\pgfpathmoveto{\pgfqpoint{3.116513in}{2.603812in}}%
\pgfpathlineto{\pgfqpoint{3.372093in}{2.761754in}}%
\pgfpathlineto{\pgfqpoint{3.116513in}{2.603812in}}%
\pgfpathclose%
\pgfusepath{fill}%
\end{pgfscope}%
\begin{pgfscope}%
\pgfpathrectangle{\pgfqpoint{0.017500in}{0.017500in}}{\pgfqpoint{6.218282in}{3.580372in}} %
\pgfusepath{clip}%
\pgfsetbuttcap%
\pgfsetroundjoin%
\definecolor{currentfill}{rgb}{0.758112,0.168122,0.188827}%
\pgfsetfillcolor{currentfill}%
\pgfsetlinewidth{0.000000pt}%
\definecolor{currentstroke}{rgb}{0.000000,0.000000,0.000000}%
\pgfsetstrokecolor{currentstroke}%
\pgfsetdash{}{0pt}%
\pgfpathmoveto{\pgfqpoint{3.116513in}{2.603812in}}%
\pgfpathlineto{\pgfqpoint{3.372093in}{2.761754in}}%
\pgfpathlineto{\pgfqpoint{3.116513in}{2.603812in}}%
\pgfpathclose%
\pgfusepath{fill}%
\end{pgfscope}%
\begin{pgfscope}%
\pgfpathrectangle{\pgfqpoint{0.017500in}{0.017500in}}{\pgfqpoint{6.218282in}{3.580372in}} %
\pgfusepath{clip}%
\pgfsetbuttcap%
\pgfsetroundjoin%
\definecolor{currentfill}{rgb}{0.758112,0.168122,0.188827}%
\pgfsetfillcolor{currentfill}%
\pgfsetlinewidth{0.000000pt}%
\definecolor{currentstroke}{rgb}{0.000000,0.000000,0.000000}%
\pgfsetstrokecolor{currentstroke}%
\pgfsetdash{}{0pt}%
\pgfpathmoveto{\pgfqpoint{3.116513in}{2.603812in}}%
\pgfpathlineto{\pgfqpoint{3.372093in}{2.761754in}}%
\pgfpathlineto{\pgfqpoint{3.116513in}{2.603812in}}%
\pgfpathclose%
\pgfusepath{fill}%
\end{pgfscope}%
\begin{pgfscope}%
\pgfpathrectangle{\pgfqpoint{0.017500in}{0.017500in}}{\pgfqpoint{6.218282in}{3.580372in}} %
\pgfusepath{clip}%
\pgfsetbuttcap%
\pgfsetroundjoin%
\definecolor{currentfill}{rgb}{0.758112,0.168122,0.188827}%
\pgfsetfillcolor{currentfill}%
\pgfsetlinewidth{0.000000pt}%
\definecolor{currentstroke}{rgb}{0.000000,0.000000,0.000000}%
\pgfsetstrokecolor{currentstroke}%
\pgfsetdash{}{0pt}%
\pgfpathmoveto{\pgfqpoint{3.116513in}{2.603812in}}%
\pgfpathlineto{\pgfqpoint{3.372093in}{2.761754in}}%
\pgfpathlineto{\pgfqpoint{3.116513in}{2.603812in}}%
\pgfpathclose%
\pgfusepath{fill}%
\end{pgfscope}%
\begin{pgfscope}%
\pgfpathrectangle{\pgfqpoint{0.017500in}{0.017500in}}{\pgfqpoint{6.218282in}{3.580372in}} %
\pgfusepath{clip}%
\pgfsetbuttcap%
\pgfsetroundjoin%
\definecolor{currentfill}{rgb}{0.758112,0.168122,0.188827}%
\pgfsetfillcolor{currentfill}%
\pgfsetlinewidth{0.000000pt}%
\definecolor{currentstroke}{rgb}{0.000000,0.000000,0.000000}%
\pgfsetstrokecolor{currentstroke}%
\pgfsetdash{}{0pt}%
\pgfpathmoveto{\pgfqpoint{3.116513in}{2.603812in}}%
\pgfpathlineto{\pgfqpoint{3.372093in}{2.761754in}}%
\pgfpathlineto{\pgfqpoint{3.116513in}{2.603812in}}%
\pgfpathclose%
\pgfusepath{fill}%
\end{pgfscope}%
\begin{pgfscope}%
\pgfpathrectangle{\pgfqpoint{0.017500in}{0.017500in}}{\pgfqpoint{6.218282in}{3.580372in}} %
\pgfusepath{clip}%
\pgfsetbuttcap%
\pgfsetroundjoin%
\definecolor{currentfill}{rgb}{0.758112,0.168122,0.188827}%
\pgfsetfillcolor{currentfill}%
\pgfsetlinewidth{0.000000pt}%
\definecolor{currentstroke}{rgb}{0.000000,0.000000,0.000000}%
\pgfsetstrokecolor{currentstroke}%
\pgfsetdash{}{0pt}%
\pgfpathmoveto{\pgfqpoint{3.116513in}{2.603812in}}%
\pgfpathlineto{\pgfqpoint{3.372093in}{2.761754in}}%
\pgfpathlineto{\pgfqpoint{3.116513in}{2.603812in}}%
\pgfpathclose%
\pgfusepath{fill}%
\end{pgfscope}%
\begin{pgfscope}%
\pgfpathrectangle{\pgfqpoint{0.017500in}{0.017500in}}{\pgfqpoint{6.218282in}{3.580372in}} %
\pgfusepath{clip}%
\pgfsetbuttcap%
\pgfsetroundjoin%
\definecolor{currentfill}{rgb}{0.758112,0.168122,0.188827}%
\pgfsetfillcolor{currentfill}%
\pgfsetlinewidth{0.000000pt}%
\definecolor{currentstroke}{rgb}{0.000000,0.000000,0.000000}%
\pgfsetstrokecolor{currentstroke}%
\pgfsetdash{}{0pt}%
\pgfpathmoveto{\pgfqpoint{3.116513in}{2.603812in}}%
\pgfpathlineto{\pgfqpoint{3.372093in}{2.761754in}}%
\pgfpathlineto{\pgfqpoint{3.116513in}{2.603812in}}%
\pgfpathclose%
\pgfusepath{fill}%
\end{pgfscope}%
\begin{pgfscope}%
\pgfpathrectangle{\pgfqpoint{0.017500in}{0.017500in}}{\pgfqpoint{6.218282in}{3.580372in}} %
\pgfusepath{clip}%
\pgfsetbuttcap%
\pgfsetroundjoin%
\definecolor{currentfill}{rgb}{0.758112,0.168122,0.188827}%
\pgfsetfillcolor{currentfill}%
\pgfsetlinewidth{0.000000pt}%
\definecolor{currentstroke}{rgb}{0.000000,0.000000,0.000000}%
\pgfsetstrokecolor{currentstroke}%
\pgfsetdash{}{0pt}%
\pgfpathmoveto{\pgfqpoint{3.116513in}{2.603812in}}%
\pgfpathlineto{\pgfqpoint{3.372093in}{2.761754in}}%
\pgfpathlineto{\pgfqpoint{3.116513in}{2.603812in}}%
\pgfpathclose%
\pgfusepath{fill}%
\end{pgfscope}%
\begin{pgfscope}%
\pgfpathrectangle{\pgfqpoint{0.017500in}{0.017500in}}{\pgfqpoint{6.218282in}{3.580372in}} %
\pgfusepath{clip}%
\pgfsetbuttcap%
\pgfsetroundjoin%
\definecolor{currentfill}{rgb}{0.758112,0.168122,0.188827}%
\pgfsetfillcolor{currentfill}%
\pgfsetlinewidth{0.000000pt}%
\definecolor{currentstroke}{rgb}{0.000000,0.000000,0.000000}%
\pgfsetstrokecolor{currentstroke}%
\pgfsetdash{}{0pt}%
\pgfpathmoveto{\pgfqpoint{3.116513in}{2.603812in}}%
\pgfpathlineto{\pgfqpoint{3.372093in}{2.761754in}}%
\pgfpathlineto{\pgfqpoint{3.116513in}{2.603812in}}%
\pgfpathclose%
\pgfusepath{fill}%
\end{pgfscope}%
\begin{pgfscope}%
\pgfpathrectangle{\pgfqpoint{0.017500in}{0.017500in}}{\pgfqpoint{6.218282in}{3.580372in}} %
\pgfusepath{clip}%
\pgfsetbuttcap%
\pgfsetroundjoin%
\definecolor{currentfill}{rgb}{0.758112,0.168122,0.188827}%
\pgfsetfillcolor{currentfill}%
\pgfsetlinewidth{0.000000pt}%
\definecolor{currentstroke}{rgb}{0.000000,0.000000,0.000000}%
\pgfsetstrokecolor{currentstroke}%
\pgfsetdash{}{0pt}%
\pgfpathmoveto{\pgfqpoint{3.116513in}{2.603812in}}%
\pgfpathlineto{\pgfqpoint{3.372093in}{2.761754in}}%
\pgfpathlineto{\pgfqpoint{3.116513in}{2.603812in}}%
\pgfpathclose%
\pgfusepath{fill}%
\end{pgfscope}%
\begin{pgfscope}%
\pgfpathrectangle{\pgfqpoint{0.017500in}{0.017500in}}{\pgfqpoint{6.218282in}{3.580372in}} %
\pgfusepath{clip}%
\pgfsetbuttcap%
\pgfsetroundjoin%
\definecolor{currentfill}{rgb}{0.758112,0.168122,0.188827}%
\pgfsetfillcolor{currentfill}%
\pgfsetlinewidth{0.000000pt}%
\definecolor{currentstroke}{rgb}{0.000000,0.000000,0.000000}%
\pgfsetstrokecolor{currentstroke}%
\pgfsetdash{}{0pt}%
\pgfpathmoveto{\pgfqpoint{3.116513in}{2.603812in}}%
\pgfpathlineto{\pgfqpoint{3.372093in}{2.761754in}}%
\pgfpathlineto{\pgfqpoint{3.116513in}{2.603812in}}%
\pgfpathclose%
\pgfusepath{fill}%
\end{pgfscope}%
\begin{pgfscope}%
\pgfpathrectangle{\pgfqpoint{0.017500in}{0.017500in}}{\pgfqpoint{6.218282in}{3.580372in}} %
\pgfusepath{clip}%
\pgfsetbuttcap%
\pgfsetroundjoin%
\definecolor{currentfill}{rgb}{0.758112,0.168122,0.188827}%
\pgfsetfillcolor{currentfill}%
\pgfsetlinewidth{0.000000pt}%
\definecolor{currentstroke}{rgb}{0.000000,0.000000,0.000000}%
\pgfsetstrokecolor{currentstroke}%
\pgfsetdash{}{0pt}%
\pgfpathmoveto{\pgfqpoint{3.116513in}{2.603812in}}%
\pgfpathlineto{\pgfqpoint{3.372093in}{2.761754in}}%
\pgfpathlineto{\pgfqpoint{3.116513in}{2.603812in}}%
\pgfpathclose%
\pgfusepath{fill}%
\end{pgfscope}%
\begin{pgfscope}%
\pgfpathrectangle{\pgfqpoint{0.017500in}{0.017500in}}{\pgfqpoint{6.218282in}{3.580372in}} %
\pgfusepath{clip}%
\pgfsetbuttcap%
\pgfsetroundjoin%
\definecolor{currentfill}{rgb}{0.758112,0.168122,0.188827}%
\pgfsetfillcolor{currentfill}%
\pgfsetlinewidth{0.000000pt}%
\definecolor{currentstroke}{rgb}{0.000000,0.000000,0.000000}%
\pgfsetstrokecolor{currentstroke}%
\pgfsetdash{}{0pt}%
\pgfpathmoveto{\pgfqpoint{3.116513in}{2.603812in}}%
\pgfpathlineto{\pgfqpoint{3.372093in}{2.761754in}}%
\pgfpathlineto{\pgfqpoint{3.116513in}{2.603812in}}%
\pgfpathclose%
\pgfusepath{fill}%
\end{pgfscope}%
\begin{pgfscope}%
\pgfpathrectangle{\pgfqpoint{0.017500in}{0.017500in}}{\pgfqpoint{6.218282in}{3.580372in}} %
\pgfusepath{clip}%
\pgfsetbuttcap%
\pgfsetroundjoin%
\definecolor{currentfill}{rgb}{0.758112,0.168122,0.188827}%
\pgfsetfillcolor{currentfill}%
\pgfsetlinewidth{0.000000pt}%
\definecolor{currentstroke}{rgb}{0.000000,0.000000,0.000000}%
\pgfsetstrokecolor{currentstroke}%
\pgfsetdash{}{0pt}%
\pgfpathmoveto{\pgfqpoint{3.116513in}{2.603812in}}%
\pgfpathlineto{\pgfqpoint{3.372093in}{2.761754in}}%
\pgfpathlineto{\pgfqpoint{3.116513in}{2.603812in}}%
\pgfpathclose%
\pgfusepath{fill}%
\end{pgfscope}%
\begin{pgfscope}%
\pgfpathrectangle{\pgfqpoint{0.017500in}{0.017500in}}{\pgfqpoint{6.218282in}{3.580372in}} %
\pgfusepath{clip}%
\pgfsetbuttcap%
\pgfsetroundjoin%
\definecolor{currentfill}{rgb}{0.758112,0.168122,0.188827}%
\pgfsetfillcolor{currentfill}%
\pgfsetlinewidth{0.000000pt}%
\definecolor{currentstroke}{rgb}{0.000000,0.000000,0.000000}%
\pgfsetstrokecolor{currentstroke}%
\pgfsetdash{}{0pt}%
\pgfpathmoveto{\pgfqpoint{3.116513in}{2.603812in}}%
\pgfpathlineto{\pgfqpoint{3.372093in}{2.761754in}}%
\pgfpathlineto{\pgfqpoint{3.116513in}{2.603812in}}%
\pgfpathclose%
\pgfusepath{fill}%
\end{pgfscope}%
\begin{pgfscope}%
\pgfpathrectangle{\pgfqpoint{0.017500in}{0.017500in}}{\pgfqpoint{6.218282in}{3.580372in}} %
\pgfusepath{clip}%
\pgfsetbuttcap%
\pgfsetroundjoin%
\definecolor{currentfill}{rgb}{0.758112,0.168122,0.188827}%
\pgfsetfillcolor{currentfill}%
\pgfsetlinewidth{0.000000pt}%
\definecolor{currentstroke}{rgb}{0.000000,0.000000,0.000000}%
\pgfsetstrokecolor{currentstroke}%
\pgfsetdash{}{0pt}%
\pgfpathmoveto{\pgfqpoint{3.116513in}{2.603812in}}%
\pgfpathlineto{\pgfqpoint{3.372093in}{2.761754in}}%
\pgfpathlineto{\pgfqpoint{3.116513in}{2.603812in}}%
\pgfpathclose%
\pgfusepath{fill}%
\end{pgfscope}%
\begin{pgfscope}%
\pgfpathrectangle{\pgfqpoint{0.017500in}{0.017500in}}{\pgfqpoint{6.218282in}{3.580372in}} %
\pgfusepath{clip}%
\pgfsetbuttcap%
\pgfsetroundjoin%
\definecolor{currentfill}{rgb}{0.758112,0.168122,0.188827}%
\pgfsetfillcolor{currentfill}%
\pgfsetlinewidth{0.000000pt}%
\definecolor{currentstroke}{rgb}{0.000000,0.000000,0.000000}%
\pgfsetstrokecolor{currentstroke}%
\pgfsetdash{}{0pt}%
\pgfpathmoveto{\pgfqpoint{3.116513in}{2.603812in}}%
\pgfpathlineto{\pgfqpoint{3.372093in}{2.761754in}}%
\pgfpathlineto{\pgfqpoint{3.116513in}{2.603812in}}%
\pgfpathclose%
\pgfusepath{fill}%
\end{pgfscope}%
\begin{pgfscope}%
\pgfpathrectangle{\pgfqpoint{0.017500in}{0.017500in}}{\pgfqpoint{6.218282in}{3.580372in}} %
\pgfusepath{clip}%
\pgfsetbuttcap%
\pgfsetroundjoin%
\definecolor{currentfill}{rgb}{0.758112,0.168122,0.188827}%
\pgfsetfillcolor{currentfill}%
\pgfsetlinewidth{0.000000pt}%
\definecolor{currentstroke}{rgb}{0.000000,0.000000,0.000000}%
\pgfsetstrokecolor{currentstroke}%
\pgfsetdash{}{0pt}%
\pgfpathmoveto{\pgfqpoint{3.116513in}{2.603812in}}%
\pgfpathlineto{\pgfqpoint{3.372093in}{2.761754in}}%
\pgfpathlineto{\pgfqpoint{3.116513in}{2.603812in}}%
\pgfpathclose%
\pgfusepath{fill}%
\end{pgfscope}%
\begin{pgfscope}%
\pgfpathrectangle{\pgfqpoint{0.017500in}{0.017500in}}{\pgfqpoint{6.218282in}{3.580372in}} %
\pgfusepath{clip}%
\pgfsetbuttcap%
\pgfsetroundjoin%
\definecolor{currentfill}{rgb}{0.758112,0.168122,0.188827}%
\pgfsetfillcolor{currentfill}%
\pgfsetlinewidth{0.000000pt}%
\definecolor{currentstroke}{rgb}{0.000000,0.000000,0.000000}%
\pgfsetstrokecolor{currentstroke}%
\pgfsetdash{}{0pt}%
\pgfpathmoveto{\pgfqpoint{3.116513in}{2.603812in}}%
\pgfpathlineto{\pgfqpoint{3.372093in}{2.761754in}}%
\pgfpathlineto{\pgfqpoint{3.116513in}{2.603812in}}%
\pgfpathclose%
\pgfusepath{fill}%
\end{pgfscope}%
\begin{pgfscope}%
\pgfpathrectangle{\pgfqpoint{0.017500in}{0.017500in}}{\pgfqpoint{6.218282in}{3.580372in}} %
\pgfusepath{clip}%
\pgfsetbuttcap%
\pgfsetroundjoin%
\definecolor{currentfill}{rgb}{0.758112,0.168122,0.188827}%
\pgfsetfillcolor{currentfill}%
\pgfsetlinewidth{0.000000pt}%
\definecolor{currentstroke}{rgb}{0.000000,0.000000,0.000000}%
\pgfsetstrokecolor{currentstroke}%
\pgfsetdash{}{0pt}%
\pgfpathmoveto{\pgfqpoint{3.116513in}{2.603812in}}%
\pgfpathlineto{\pgfqpoint{3.372093in}{2.761754in}}%
\pgfpathlineto{\pgfqpoint{3.116513in}{2.603812in}}%
\pgfpathclose%
\pgfusepath{fill}%
\end{pgfscope}%
\begin{pgfscope}%
\pgfpathrectangle{\pgfqpoint{0.017500in}{0.017500in}}{\pgfqpoint{6.218282in}{3.580372in}} %
\pgfusepath{clip}%
\pgfsetbuttcap%
\pgfsetroundjoin%
\definecolor{currentfill}{rgb}{0.758112,0.168122,0.188827}%
\pgfsetfillcolor{currentfill}%
\pgfsetlinewidth{0.000000pt}%
\definecolor{currentstroke}{rgb}{0.000000,0.000000,0.000000}%
\pgfsetstrokecolor{currentstroke}%
\pgfsetdash{}{0pt}%
\pgfpathmoveto{\pgfqpoint{3.116513in}{2.603812in}}%
\pgfpathlineto{\pgfqpoint{3.372093in}{2.761754in}}%
\pgfpathlineto{\pgfqpoint{3.116513in}{2.603812in}}%
\pgfpathclose%
\pgfusepath{fill}%
\end{pgfscope}%
\begin{pgfscope}%
\pgfpathrectangle{\pgfqpoint{0.017500in}{0.017500in}}{\pgfqpoint{6.218282in}{3.580372in}} %
\pgfusepath{clip}%
\pgfsetbuttcap%
\pgfsetroundjoin%
\definecolor{currentfill}{rgb}{0.902849,0.844796,0.811970}%
\pgfsetfillcolor{currentfill}%
\pgfsetlinewidth{0.000000pt}%
\definecolor{currentstroke}{rgb}{0.000000,0.000000,0.000000}%
\pgfsetstrokecolor{currentstroke}%
\pgfsetdash{}{0pt}%
\pgfpathmoveto{\pgfqpoint{4.511410in}{2.753953in}}%
\pgfpathlineto{\pgfqpoint{3.236099in}{0.723497in}}%
\pgfpathlineto{\pgfqpoint{4.511410in}{2.753953in}}%
\pgfpathclose%
\pgfusepath{fill}%
\end{pgfscope}%
\begin{pgfscope}%
\pgfpathrectangle{\pgfqpoint{0.017500in}{0.017500in}}{\pgfqpoint{6.218282in}{3.580372in}} %
\pgfusepath{clip}%
\pgfsetbuttcap%
\pgfsetroundjoin%
\definecolor{currentfill}{rgb}{0.902849,0.844796,0.811970}%
\pgfsetfillcolor{currentfill}%
\pgfsetlinewidth{0.000000pt}%
\definecolor{currentstroke}{rgb}{0.000000,0.000000,0.000000}%
\pgfsetstrokecolor{currentstroke}%
\pgfsetdash{}{0pt}%
\pgfpathmoveto{\pgfqpoint{4.511410in}{2.753953in}}%
\pgfpathlineto{\pgfqpoint{3.236099in}{0.723497in}}%
\pgfpathlineto{\pgfqpoint{4.511410in}{2.753953in}}%
\pgfpathclose%
\pgfusepath{fill}%
\end{pgfscope}%
\begin{pgfscope}%
\pgfpathrectangle{\pgfqpoint{0.017500in}{0.017500in}}{\pgfqpoint{6.218282in}{3.580372in}} %
\pgfusepath{clip}%
\pgfsetbuttcap%
\pgfsetroundjoin%
\definecolor{currentfill}{rgb}{0.902849,0.844796,0.811970}%
\pgfsetfillcolor{currentfill}%
\pgfsetlinewidth{0.000000pt}%
\definecolor{currentstroke}{rgb}{0.000000,0.000000,0.000000}%
\pgfsetstrokecolor{currentstroke}%
\pgfsetdash{}{0pt}%
\pgfpathmoveto{\pgfqpoint{4.511410in}{2.753953in}}%
\pgfpathlineto{\pgfqpoint{3.236099in}{0.723497in}}%
\pgfpathlineto{\pgfqpoint{4.511410in}{2.753953in}}%
\pgfpathclose%
\pgfusepath{fill}%
\end{pgfscope}%
\begin{pgfscope}%
\pgfpathrectangle{\pgfqpoint{0.017500in}{0.017500in}}{\pgfqpoint{6.218282in}{3.580372in}} %
\pgfusepath{clip}%
\pgfsetbuttcap%
\pgfsetroundjoin%
\definecolor{currentfill}{rgb}{0.902849,0.844796,0.811970}%
\pgfsetfillcolor{currentfill}%
\pgfsetlinewidth{0.000000pt}%
\definecolor{currentstroke}{rgb}{0.000000,0.000000,0.000000}%
\pgfsetstrokecolor{currentstroke}%
\pgfsetdash{}{0pt}%
\pgfpathmoveto{\pgfqpoint{4.511410in}{2.753953in}}%
\pgfpathlineto{\pgfqpoint{3.236099in}{0.723497in}}%
\pgfpathlineto{\pgfqpoint{4.511410in}{2.753953in}}%
\pgfpathclose%
\pgfusepath{fill}%
\end{pgfscope}%
\begin{pgfscope}%
\pgfpathrectangle{\pgfqpoint{0.017500in}{0.017500in}}{\pgfqpoint{6.218282in}{3.580372in}} %
\pgfusepath{clip}%
\pgfsetbuttcap%
\pgfsetroundjoin%
\definecolor{currentfill}{rgb}{0.902849,0.844796,0.811970}%
\pgfsetfillcolor{currentfill}%
\pgfsetlinewidth{0.000000pt}%
\definecolor{currentstroke}{rgb}{0.000000,0.000000,0.000000}%
\pgfsetstrokecolor{currentstroke}%
\pgfsetdash{}{0pt}%
\pgfpathmoveto{\pgfqpoint{4.511410in}{2.753953in}}%
\pgfpathlineto{\pgfqpoint{3.236099in}{0.723497in}}%
\pgfpathlineto{\pgfqpoint{4.511410in}{2.753953in}}%
\pgfpathclose%
\pgfusepath{fill}%
\end{pgfscope}%
\begin{pgfscope}%
\pgfpathrectangle{\pgfqpoint{0.017500in}{0.017500in}}{\pgfqpoint{6.218282in}{3.580372in}} %
\pgfusepath{clip}%
\pgfsetbuttcap%
\pgfsetroundjoin%
\definecolor{currentfill}{rgb}{0.902849,0.844796,0.811970}%
\pgfsetfillcolor{currentfill}%
\pgfsetlinewidth{0.000000pt}%
\definecolor{currentstroke}{rgb}{0.000000,0.000000,0.000000}%
\pgfsetstrokecolor{currentstroke}%
\pgfsetdash{}{0pt}%
\pgfpathmoveto{\pgfqpoint{4.511410in}{2.753953in}}%
\pgfpathlineto{\pgfqpoint{3.236099in}{0.723497in}}%
\pgfpathlineto{\pgfqpoint{4.511410in}{2.753953in}}%
\pgfpathclose%
\pgfusepath{fill}%
\end{pgfscope}%
\begin{pgfscope}%
\pgfpathrectangle{\pgfqpoint{0.017500in}{0.017500in}}{\pgfqpoint{6.218282in}{3.580372in}} %
\pgfusepath{clip}%
\pgfsetbuttcap%
\pgfsetroundjoin%
\definecolor{currentfill}{rgb}{0.902849,0.844796,0.811970}%
\pgfsetfillcolor{currentfill}%
\pgfsetlinewidth{0.000000pt}%
\definecolor{currentstroke}{rgb}{0.000000,0.000000,0.000000}%
\pgfsetstrokecolor{currentstroke}%
\pgfsetdash{}{0pt}%
\pgfpathmoveto{\pgfqpoint{4.511410in}{2.753953in}}%
\pgfpathlineto{\pgfqpoint{3.236099in}{0.723497in}}%
\pgfpathlineto{\pgfqpoint{4.511410in}{2.753953in}}%
\pgfpathclose%
\pgfusepath{fill}%
\end{pgfscope}%
\begin{pgfscope}%
\pgfpathrectangle{\pgfqpoint{0.017500in}{0.017500in}}{\pgfqpoint{6.218282in}{3.580372in}} %
\pgfusepath{clip}%
\pgfsetbuttcap%
\pgfsetroundjoin%
\definecolor{currentfill}{rgb}{0.902849,0.844796,0.811970}%
\pgfsetfillcolor{currentfill}%
\pgfsetlinewidth{0.000000pt}%
\definecolor{currentstroke}{rgb}{0.000000,0.000000,0.000000}%
\pgfsetstrokecolor{currentstroke}%
\pgfsetdash{}{0pt}%
\pgfpathmoveto{\pgfqpoint{4.511410in}{2.753953in}}%
\pgfpathlineto{\pgfqpoint{3.236099in}{0.723497in}}%
\pgfpathlineto{\pgfqpoint{4.511410in}{2.753953in}}%
\pgfpathclose%
\pgfusepath{fill}%
\end{pgfscope}%
\begin{pgfscope}%
\pgfpathrectangle{\pgfqpoint{0.017500in}{0.017500in}}{\pgfqpoint{6.218282in}{3.580372in}} %
\pgfusepath{clip}%
\pgfsetbuttcap%
\pgfsetroundjoin%
\definecolor{currentfill}{rgb}{0.902849,0.844796,0.811970}%
\pgfsetfillcolor{currentfill}%
\pgfsetlinewidth{0.000000pt}%
\definecolor{currentstroke}{rgb}{0.000000,0.000000,0.000000}%
\pgfsetstrokecolor{currentstroke}%
\pgfsetdash{}{0pt}%
\pgfpathmoveto{\pgfqpoint{4.511410in}{2.753953in}}%
\pgfpathlineto{\pgfqpoint{3.236099in}{0.723497in}}%
\pgfpathlineto{\pgfqpoint{4.511410in}{2.753953in}}%
\pgfpathclose%
\pgfusepath{fill}%
\end{pgfscope}%
\begin{pgfscope}%
\pgfpathrectangle{\pgfqpoint{0.017500in}{0.017500in}}{\pgfqpoint{6.218282in}{3.580372in}} %
\pgfusepath{clip}%
\pgfsetbuttcap%
\pgfsetroundjoin%
\definecolor{currentfill}{rgb}{0.902849,0.844796,0.811970}%
\pgfsetfillcolor{currentfill}%
\pgfsetlinewidth{0.000000pt}%
\definecolor{currentstroke}{rgb}{0.000000,0.000000,0.000000}%
\pgfsetstrokecolor{currentstroke}%
\pgfsetdash{}{0pt}%
\pgfpathmoveto{\pgfqpoint{4.511410in}{2.753953in}}%
\pgfpathlineto{\pgfqpoint{3.236099in}{0.723497in}}%
\pgfpathlineto{\pgfqpoint{4.511410in}{2.753953in}}%
\pgfpathclose%
\pgfusepath{fill}%
\end{pgfscope}%
\begin{pgfscope}%
\pgfpathrectangle{\pgfqpoint{0.017500in}{0.017500in}}{\pgfqpoint{6.218282in}{3.580372in}} %
\pgfusepath{clip}%
\pgfsetbuttcap%
\pgfsetroundjoin%
\definecolor{currentfill}{rgb}{0.902849,0.844796,0.811970}%
\pgfsetfillcolor{currentfill}%
\pgfsetlinewidth{0.000000pt}%
\definecolor{currentstroke}{rgb}{0.000000,0.000000,0.000000}%
\pgfsetstrokecolor{currentstroke}%
\pgfsetdash{}{0pt}%
\pgfpathmoveto{\pgfqpoint{4.511410in}{2.753953in}}%
\pgfpathlineto{\pgfqpoint{3.236099in}{0.723497in}}%
\pgfpathlineto{\pgfqpoint{4.511410in}{2.753953in}}%
\pgfpathclose%
\pgfusepath{fill}%
\end{pgfscope}%
\begin{pgfscope}%
\pgfpathrectangle{\pgfqpoint{0.017500in}{0.017500in}}{\pgfqpoint{6.218282in}{3.580372in}} %
\pgfusepath{clip}%
\pgfsetbuttcap%
\pgfsetroundjoin%
\definecolor{currentfill}{rgb}{0.902849,0.844796,0.811970}%
\pgfsetfillcolor{currentfill}%
\pgfsetlinewidth{0.000000pt}%
\definecolor{currentstroke}{rgb}{0.000000,0.000000,0.000000}%
\pgfsetstrokecolor{currentstroke}%
\pgfsetdash{}{0pt}%
\pgfpathmoveto{\pgfqpoint{4.511410in}{2.753953in}}%
\pgfpathlineto{\pgfqpoint{3.236099in}{0.723497in}}%
\pgfpathlineto{\pgfqpoint{4.511410in}{2.753953in}}%
\pgfpathclose%
\pgfusepath{fill}%
\end{pgfscope}%
\begin{pgfscope}%
\pgfpathrectangle{\pgfqpoint{0.017500in}{0.017500in}}{\pgfqpoint{6.218282in}{3.580372in}} %
\pgfusepath{clip}%
\pgfsetbuttcap%
\pgfsetroundjoin%
\definecolor{currentfill}{rgb}{0.902849,0.844796,0.811970}%
\pgfsetfillcolor{currentfill}%
\pgfsetlinewidth{0.000000pt}%
\definecolor{currentstroke}{rgb}{0.000000,0.000000,0.000000}%
\pgfsetstrokecolor{currentstroke}%
\pgfsetdash{}{0pt}%
\pgfpathmoveto{\pgfqpoint{4.511410in}{2.753953in}}%
\pgfpathlineto{\pgfqpoint{3.236099in}{0.723497in}}%
\pgfpathlineto{\pgfqpoint{4.511410in}{2.753953in}}%
\pgfpathclose%
\pgfusepath{fill}%
\end{pgfscope}%
\begin{pgfscope}%
\pgfpathrectangle{\pgfqpoint{0.017500in}{0.017500in}}{\pgfqpoint{6.218282in}{3.580372in}} %
\pgfusepath{clip}%
\pgfsetbuttcap%
\pgfsetroundjoin%
\definecolor{currentfill}{rgb}{0.902849,0.844796,0.811970}%
\pgfsetfillcolor{currentfill}%
\pgfsetlinewidth{0.000000pt}%
\definecolor{currentstroke}{rgb}{0.000000,0.000000,0.000000}%
\pgfsetstrokecolor{currentstroke}%
\pgfsetdash{}{0pt}%
\pgfpathmoveto{\pgfqpoint{4.511410in}{2.753953in}}%
\pgfpathlineto{\pgfqpoint{3.236099in}{0.723497in}}%
\pgfpathlineto{\pgfqpoint{4.511410in}{2.753953in}}%
\pgfpathclose%
\pgfusepath{fill}%
\end{pgfscope}%
\begin{pgfscope}%
\pgfpathrectangle{\pgfqpoint{0.017500in}{0.017500in}}{\pgfqpoint{6.218282in}{3.580372in}} %
\pgfusepath{clip}%
\pgfsetbuttcap%
\pgfsetroundjoin%
\definecolor{currentfill}{rgb}{0.902849,0.844796,0.811970}%
\pgfsetfillcolor{currentfill}%
\pgfsetlinewidth{0.000000pt}%
\definecolor{currentstroke}{rgb}{0.000000,0.000000,0.000000}%
\pgfsetstrokecolor{currentstroke}%
\pgfsetdash{}{0pt}%
\pgfpathmoveto{\pgfqpoint{4.511410in}{2.753953in}}%
\pgfpathlineto{\pgfqpoint{3.236099in}{0.723497in}}%
\pgfpathlineto{\pgfqpoint{4.511410in}{2.753953in}}%
\pgfpathclose%
\pgfusepath{fill}%
\end{pgfscope}%
\begin{pgfscope}%
\pgfpathrectangle{\pgfqpoint{0.017500in}{0.017500in}}{\pgfqpoint{6.218282in}{3.580372in}} %
\pgfusepath{clip}%
\pgfsetbuttcap%
\pgfsetroundjoin%
\definecolor{currentfill}{rgb}{0.902849,0.844796,0.811970}%
\pgfsetfillcolor{currentfill}%
\pgfsetlinewidth{0.000000pt}%
\definecolor{currentstroke}{rgb}{0.000000,0.000000,0.000000}%
\pgfsetstrokecolor{currentstroke}%
\pgfsetdash{}{0pt}%
\pgfpathmoveto{\pgfqpoint{4.511410in}{2.753953in}}%
\pgfpathlineto{\pgfqpoint{3.236099in}{0.723497in}}%
\pgfpathlineto{\pgfqpoint{4.511410in}{2.753953in}}%
\pgfpathclose%
\pgfusepath{fill}%
\end{pgfscope}%
\begin{pgfscope}%
\pgfpathrectangle{\pgfqpoint{0.017500in}{0.017500in}}{\pgfqpoint{6.218282in}{3.580372in}} %
\pgfusepath{clip}%
\pgfsetbuttcap%
\pgfsetroundjoin%
\definecolor{currentfill}{rgb}{0.902849,0.844796,0.811970}%
\pgfsetfillcolor{currentfill}%
\pgfsetlinewidth{0.000000pt}%
\definecolor{currentstroke}{rgb}{0.000000,0.000000,0.000000}%
\pgfsetstrokecolor{currentstroke}%
\pgfsetdash{}{0pt}%
\pgfpathmoveto{\pgfqpoint{4.511410in}{2.753953in}}%
\pgfpathlineto{\pgfqpoint{3.236099in}{0.723497in}}%
\pgfpathlineto{\pgfqpoint{4.511410in}{2.753953in}}%
\pgfpathclose%
\pgfusepath{fill}%
\end{pgfscope}%
\begin{pgfscope}%
\pgfpathrectangle{\pgfqpoint{0.017500in}{0.017500in}}{\pgfqpoint{6.218282in}{3.580372in}} %
\pgfusepath{clip}%
\pgfsetbuttcap%
\pgfsetroundjoin%
\definecolor{currentfill}{rgb}{0.902849,0.844796,0.811970}%
\pgfsetfillcolor{currentfill}%
\pgfsetlinewidth{0.000000pt}%
\definecolor{currentstroke}{rgb}{0.000000,0.000000,0.000000}%
\pgfsetstrokecolor{currentstroke}%
\pgfsetdash{}{0pt}%
\pgfpathmoveto{\pgfqpoint{4.511410in}{2.753953in}}%
\pgfpathlineto{\pgfqpoint{3.236099in}{0.723497in}}%
\pgfpathlineto{\pgfqpoint{4.511410in}{2.753953in}}%
\pgfpathclose%
\pgfusepath{fill}%
\end{pgfscope}%
\begin{pgfscope}%
\pgfpathrectangle{\pgfqpoint{0.017500in}{0.017500in}}{\pgfqpoint{6.218282in}{3.580372in}} %
\pgfusepath{clip}%
\pgfsetbuttcap%
\pgfsetroundjoin%
\definecolor{currentfill}{rgb}{0.902849,0.844796,0.811970}%
\pgfsetfillcolor{currentfill}%
\pgfsetlinewidth{0.000000pt}%
\definecolor{currentstroke}{rgb}{0.000000,0.000000,0.000000}%
\pgfsetstrokecolor{currentstroke}%
\pgfsetdash{}{0pt}%
\pgfpathmoveto{\pgfqpoint{4.511410in}{2.753953in}}%
\pgfpathlineto{\pgfqpoint{3.236099in}{0.723497in}}%
\pgfpathlineto{\pgfqpoint{4.511410in}{2.753953in}}%
\pgfpathclose%
\pgfusepath{fill}%
\end{pgfscope}%
\begin{pgfscope}%
\pgfpathrectangle{\pgfqpoint{0.017500in}{0.017500in}}{\pgfqpoint{6.218282in}{3.580372in}} %
\pgfusepath{clip}%
\pgfsetbuttcap%
\pgfsetroundjoin%
\definecolor{currentfill}{rgb}{0.902849,0.844796,0.811970}%
\pgfsetfillcolor{currentfill}%
\pgfsetlinewidth{0.000000pt}%
\definecolor{currentstroke}{rgb}{0.000000,0.000000,0.000000}%
\pgfsetstrokecolor{currentstroke}%
\pgfsetdash{}{0pt}%
\pgfpathmoveto{\pgfqpoint{4.511410in}{2.753953in}}%
\pgfpathlineto{\pgfqpoint{3.236099in}{0.723497in}}%
\pgfpathlineto{\pgfqpoint{4.511410in}{2.753953in}}%
\pgfpathclose%
\pgfusepath{fill}%
\end{pgfscope}%
\begin{pgfscope}%
\pgfpathrectangle{\pgfqpoint{0.017500in}{0.017500in}}{\pgfqpoint{6.218282in}{3.580372in}} %
\pgfusepath{clip}%
\pgfsetbuttcap%
\pgfsetroundjoin%
\definecolor{currentfill}{rgb}{0.902849,0.844796,0.811970}%
\pgfsetfillcolor{currentfill}%
\pgfsetlinewidth{0.000000pt}%
\definecolor{currentstroke}{rgb}{0.000000,0.000000,0.000000}%
\pgfsetstrokecolor{currentstroke}%
\pgfsetdash{}{0pt}%
\pgfpathmoveto{\pgfqpoint{4.511410in}{2.753953in}}%
\pgfpathlineto{\pgfqpoint{3.236099in}{0.723497in}}%
\pgfpathlineto{\pgfqpoint{4.511410in}{2.753953in}}%
\pgfpathclose%
\pgfusepath{fill}%
\end{pgfscope}%
\begin{pgfscope}%
\pgfpathrectangle{\pgfqpoint{0.017500in}{0.017500in}}{\pgfqpoint{6.218282in}{3.580372in}} %
\pgfusepath{clip}%
\pgfsetbuttcap%
\pgfsetroundjoin%
\definecolor{currentfill}{rgb}{0.902849,0.844796,0.811970}%
\pgfsetfillcolor{currentfill}%
\pgfsetlinewidth{0.000000pt}%
\definecolor{currentstroke}{rgb}{0.000000,0.000000,0.000000}%
\pgfsetstrokecolor{currentstroke}%
\pgfsetdash{}{0pt}%
\pgfpathmoveto{\pgfqpoint{4.511410in}{2.753953in}}%
\pgfpathlineto{\pgfqpoint{3.236099in}{0.723497in}}%
\pgfpathlineto{\pgfqpoint{4.511410in}{2.753953in}}%
\pgfpathclose%
\pgfusepath{fill}%
\end{pgfscope}%
\begin{pgfscope}%
\pgfpathrectangle{\pgfqpoint{0.017500in}{0.017500in}}{\pgfqpoint{6.218282in}{3.580372in}} %
\pgfusepath{clip}%
\pgfsetbuttcap%
\pgfsetroundjoin%
\definecolor{currentfill}{rgb}{0.902849,0.844796,0.811970}%
\pgfsetfillcolor{currentfill}%
\pgfsetlinewidth{0.000000pt}%
\definecolor{currentstroke}{rgb}{0.000000,0.000000,0.000000}%
\pgfsetstrokecolor{currentstroke}%
\pgfsetdash{}{0pt}%
\pgfpathmoveto{\pgfqpoint{4.511410in}{2.753953in}}%
\pgfpathlineto{\pgfqpoint{3.236099in}{0.723497in}}%
\pgfpathlineto{\pgfqpoint{4.511410in}{2.753953in}}%
\pgfpathclose%
\pgfusepath{fill}%
\end{pgfscope}%
\begin{pgfscope}%
\pgfpathrectangle{\pgfqpoint{0.017500in}{0.017500in}}{\pgfqpoint{6.218282in}{3.580372in}} %
\pgfusepath{clip}%
\pgfsetbuttcap%
\pgfsetroundjoin%
\definecolor{currentfill}{rgb}{0.902849,0.844796,0.811970}%
\pgfsetfillcolor{currentfill}%
\pgfsetlinewidth{0.000000pt}%
\definecolor{currentstroke}{rgb}{0.000000,0.000000,0.000000}%
\pgfsetstrokecolor{currentstroke}%
\pgfsetdash{}{0pt}%
\pgfpathmoveto{\pgfqpoint{4.511410in}{2.753953in}}%
\pgfpathlineto{\pgfqpoint{3.236099in}{0.723497in}}%
\pgfpathlineto{\pgfqpoint{4.511410in}{2.753953in}}%
\pgfpathclose%
\pgfusepath{fill}%
\end{pgfscope}%
\begin{pgfscope}%
\pgfpathrectangle{\pgfqpoint{0.017500in}{0.017500in}}{\pgfqpoint{6.218282in}{3.580372in}} %
\pgfusepath{clip}%
\pgfsetbuttcap%
\pgfsetroundjoin%
\definecolor{currentfill}{rgb}{0.902849,0.844796,0.811970}%
\pgfsetfillcolor{currentfill}%
\pgfsetlinewidth{0.000000pt}%
\definecolor{currentstroke}{rgb}{0.000000,0.000000,0.000000}%
\pgfsetstrokecolor{currentstroke}%
\pgfsetdash{}{0pt}%
\pgfpathmoveto{\pgfqpoint{4.511410in}{2.753953in}}%
\pgfpathlineto{\pgfqpoint{3.236099in}{0.723497in}}%
\pgfpathlineto{\pgfqpoint{4.511410in}{2.753953in}}%
\pgfpathclose%
\pgfusepath{fill}%
\end{pgfscope}%
\begin{pgfscope}%
\pgfpathrectangle{\pgfqpoint{0.017500in}{0.017500in}}{\pgfqpoint{6.218282in}{3.580372in}} %
\pgfusepath{clip}%
\pgfsetbuttcap%
\pgfsetroundjoin%
\definecolor{currentfill}{rgb}{0.758112,0.168122,0.188827}%
\pgfsetfillcolor{currentfill}%
\pgfsetlinewidth{0.000000pt}%
\definecolor{currentstroke}{rgb}{0.000000,0.000000,0.000000}%
\pgfsetstrokecolor{currentstroke}%
\pgfsetdash{}{0pt}%
\pgfpathmoveto{\pgfqpoint{3.116513in}{2.603812in}}%
\pgfpathlineto{\pgfqpoint{3.564492in}{2.541257in}}%
\pgfpathlineto{\pgfqpoint{3.816789in}{2.677291in}}%
\pgfpathlineto{\pgfqpoint{3.372093in}{2.761754in}}%
\pgfpathlineto{\pgfqpoint{3.116513in}{2.603812in}}%
\pgfpathclose%
\pgfusepath{fill}%
\end{pgfscope}%
\begin{pgfscope}%
\pgfpathrectangle{\pgfqpoint{0.017500in}{0.017500in}}{\pgfqpoint{6.218282in}{3.580372in}} %
\pgfusepath{clip}%
\pgfsetbuttcap%
\pgfsetroundjoin%
\definecolor{currentfill}{rgb}{0.758112,0.168122,0.188827}%
\pgfsetfillcolor{currentfill}%
\pgfsetlinewidth{0.000000pt}%
\definecolor{currentstroke}{rgb}{0.000000,0.000000,0.000000}%
\pgfsetstrokecolor{currentstroke}%
\pgfsetdash{}{0pt}%
\pgfpathmoveto{\pgfqpoint{3.116513in}{2.603812in}}%
\pgfpathlineto{\pgfqpoint{3.564492in}{2.541257in}}%
\pgfpathlineto{\pgfqpoint{3.816789in}{2.677291in}}%
\pgfpathlineto{\pgfqpoint{3.372093in}{2.761754in}}%
\pgfpathlineto{\pgfqpoint{3.116513in}{2.603812in}}%
\pgfpathclose%
\pgfusepath{fill}%
\end{pgfscope}%
\begin{pgfscope}%
\pgfpathrectangle{\pgfqpoint{0.017500in}{0.017500in}}{\pgfqpoint{6.218282in}{3.580372in}} %
\pgfusepath{clip}%
\pgfsetbuttcap%
\pgfsetroundjoin%
\definecolor{currentfill}{rgb}{0.758112,0.168122,0.188827}%
\pgfsetfillcolor{currentfill}%
\pgfsetlinewidth{0.000000pt}%
\definecolor{currentstroke}{rgb}{0.000000,0.000000,0.000000}%
\pgfsetstrokecolor{currentstroke}%
\pgfsetdash{}{0pt}%
\pgfpathmoveto{\pgfqpoint{3.116513in}{2.603812in}}%
\pgfpathlineto{\pgfqpoint{3.564492in}{2.541257in}}%
\pgfpathlineto{\pgfqpoint{3.816789in}{2.677291in}}%
\pgfpathlineto{\pgfqpoint{3.372093in}{2.761754in}}%
\pgfpathlineto{\pgfqpoint{3.116513in}{2.603812in}}%
\pgfpathclose%
\pgfusepath{fill}%
\end{pgfscope}%
\begin{pgfscope}%
\pgfpathrectangle{\pgfqpoint{0.017500in}{0.017500in}}{\pgfqpoint{6.218282in}{3.580372in}} %
\pgfusepath{clip}%
\pgfsetbuttcap%
\pgfsetroundjoin%
\definecolor{currentfill}{rgb}{0.758112,0.168122,0.188827}%
\pgfsetfillcolor{currentfill}%
\pgfsetlinewidth{0.000000pt}%
\definecolor{currentstroke}{rgb}{0.000000,0.000000,0.000000}%
\pgfsetstrokecolor{currentstroke}%
\pgfsetdash{}{0pt}%
\pgfpathmoveto{\pgfqpoint{3.116513in}{2.603812in}}%
\pgfpathlineto{\pgfqpoint{3.564492in}{2.541257in}}%
\pgfpathlineto{\pgfqpoint{3.816789in}{2.677291in}}%
\pgfpathlineto{\pgfqpoint{3.372093in}{2.761754in}}%
\pgfpathlineto{\pgfqpoint{3.116513in}{2.603812in}}%
\pgfpathclose%
\pgfusepath{fill}%
\end{pgfscope}%
\begin{pgfscope}%
\pgfpathrectangle{\pgfqpoint{0.017500in}{0.017500in}}{\pgfqpoint{6.218282in}{3.580372in}} %
\pgfusepath{clip}%
\pgfsetbuttcap%
\pgfsetroundjoin%
\definecolor{currentfill}{rgb}{0.758112,0.168122,0.188827}%
\pgfsetfillcolor{currentfill}%
\pgfsetlinewidth{0.000000pt}%
\definecolor{currentstroke}{rgb}{0.000000,0.000000,0.000000}%
\pgfsetstrokecolor{currentstroke}%
\pgfsetdash{}{0pt}%
\pgfpathmoveto{\pgfqpoint{3.116513in}{2.603812in}}%
\pgfpathlineto{\pgfqpoint{3.564492in}{2.541257in}}%
\pgfpathlineto{\pgfqpoint{3.816789in}{2.677291in}}%
\pgfpathlineto{\pgfqpoint{3.372093in}{2.761754in}}%
\pgfpathlineto{\pgfqpoint{3.116513in}{2.603812in}}%
\pgfpathclose%
\pgfusepath{fill}%
\end{pgfscope}%
\begin{pgfscope}%
\pgfpathrectangle{\pgfqpoint{0.017500in}{0.017500in}}{\pgfqpoint{6.218282in}{3.580372in}} %
\pgfusepath{clip}%
\pgfsetbuttcap%
\pgfsetroundjoin%
\definecolor{currentfill}{rgb}{0.758112,0.168122,0.188827}%
\pgfsetfillcolor{currentfill}%
\pgfsetlinewidth{0.000000pt}%
\definecolor{currentstroke}{rgb}{0.000000,0.000000,0.000000}%
\pgfsetstrokecolor{currentstroke}%
\pgfsetdash{}{0pt}%
\pgfpathmoveto{\pgfqpoint{3.116513in}{2.603812in}}%
\pgfpathlineto{\pgfqpoint{3.564492in}{2.541257in}}%
\pgfpathlineto{\pgfqpoint{3.816789in}{2.677291in}}%
\pgfpathlineto{\pgfqpoint{3.372093in}{2.761754in}}%
\pgfpathlineto{\pgfqpoint{3.116513in}{2.603812in}}%
\pgfpathclose%
\pgfusepath{fill}%
\end{pgfscope}%
\begin{pgfscope}%
\pgfpathrectangle{\pgfqpoint{0.017500in}{0.017500in}}{\pgfqpoint{6.218282in}{3.580372in}} %
\pgfusepath{clip}%
\pgfsetbuttcap%
\pgfsetroundjoin%
\definecolor{currentfill}{rgb}{0.729196,0.086679,0.167240}%
\pgfsetfillcolor{currentfill}%
\pgfsetlinewidth{0.000000pt}%
\definecolor{currentstroke}{rgb}{0.000000,0.000000,0.000000}%
\pgfsetstrokecolor{currentstroke}%
\pgfsetdash{}{0pt}%
\pgfpathmoveto{\pgfqpoint{4.267628in}{2.613520in}}%
\pgfpathlineto{\pgfqpoint{4.511410in}{2.753953in}}%
\pgfpathlineto{\pgfqpoint{4.267628in}{2.613520in}}%
\pgfpathclose%
\pgfusepath{fill}%
\end{pgfscope}%
\begin{pgfscope}%
\pgfpathrectangle{\pgfqpoint{0.017500in}{0.017500in}}{\pgfqpoint{6.218282in}{3.580372in}} %
\pgfusepath{clip}%
\pgfsetbuttcap%
\pgfsetroundjoin%
\definecolor{currentfill}{rgb}{0.729196,0.086679,0.167240}%
\pgfsetfillcolor{currentfill}%
\pgfsetlinewidth{0.000000pt}%
\definecolor{currentstroke}{rgb}{0.000000,0.000000,0.000000}%
\pgfsetstrokecolor{currentstroke}%
\pgfsetdash{}{0pt}%
\pgfpathmoveto{\pgfqpoint{4.267628in}{2.613520in}}%
\pgfpathlineto{\pgfqpoint{4.511410in}{2.753953in}}%
\pgfpathlineto{\pgfqpoint{4.267628in}{2.613520in}}%
\pgfpathclose%
\pgfusepath{fill}%
\end{pgfscope}%
\begin{pgfscope}%
\pgfpathrectangle{\pgfqpoint{0.017500in}{0.017500in}}{\pgfqpoint{6.218282in}{3.580372in}} %
\pgfusepath{clip}%
\pgfsetbuttcap%
\pgfsetroundjoin%
\definecolor{currentfill}{rgb}{0.729196,0.086679,0.167240}%
\pgfsetfillcolor{currentfill}%
\pgfsetlinewidth{0.000000pt}%
\definecolor{currentstroke}{rgb}{0.000000,0.000000,0.000000}%
\pgfsetstrokecolor{currentstroke}%
\pgfsetdash{}{0pt}%
\pgfpathmoveto{\pgfqpoint{4.267628in}{2.613520in}}%
\pgfpathlineto{\pgfqpoint{4.511410in}{2.753953in}}%
\pgfpathlineto{\pgfqpoint{4.267628in}{2.613520in}}%
\pgfpathclose%
\pgfusepath{fill}%
\end{pgfscope}%
\begin{pgfscope}%
\pgfpathrectangle{\pgfqpoint{0.017500in}{0.017500in}}{\pgfqpoint{6.218282in}{3.580372in}} %
\pgfusepath{clip}%
\pgfsetbuttcap%
\pgfsetroundjoin%
\definecolor{currentfill}{rgb}{0.729196,0.086679,0.167240}%
\pgfsetfillcolor{currentfill}%
\pgfsetlinewidth{0.000000pt}%
\definecolor{currentstroke}{rgb}{0.000000,0.000000,0.000000}%
\pgfsetstrokecolor{currentstroke}%
\pgfsetdash{}{0pt}%
\pgfpathmoveto{\pgfqpoint{4.267628in}{2.613520in}}%
\pgfpathlineto{\pgfqpoint{4.511410in}{2.753953in}}%
\pgfpathlineto{\pgfqpoint{4.267628in}{2.613520in}}%
\pgfpathclose%
\pgfusepath{fill}%
\end{pgfscope}%
\begin{pgfscope}%
\pgfpathrectangle{\pgfqpoint{0.017500in}{0.017500in}}{\pgfqpoint{6.218282in}{3.580372in}} %
\pgfusepath{clip}%
\pgfsetbuttcap%
\pgfsetroundjoin%
\definecolor{currentfill}{rgb}{0.729196,0.086679,0.167240}%
\pgfsetfillcolor{currentfill}%
\pgfsetlinewidth{0.000000pt}%
\definecolor{currentstroke}{rgb}{0.000000,0.000000,0.000000}%
\pgfsetstrokecolor{currentstroke}%
\pgfsetdash{}{0pt}%
\pgfpathmoveto{\pgfqpoint{4.267628in}{2.613520in}}%
\pgfpathlineto{\pgfqpoint{4.511410in}{2.753953in}}%
\pgfpathlineto{\pgfqpoint{4.267628in}{2.613520in}}%
\pgfpathclose%
\pgfusepath{fill}%
\end{pgfscope}%
\begin{pgfscope}%
\pgfpathrectangle{\pgfqpoint{0.017500in}{0.017500in}}{\pgfqpoint{6.218282in}{3.580372in}} %
\pgfusepath{clip}%
\pgfsetbuttcap%
\pgfsetroundjoin%
\definecolor{currentfill}{rgb}{0.729196,0.086679,0.167240}%
\pgfsetfillcolor{currentfill}%
\pgfsetlinewidth{0.000000pt}%
\definecolor{currentstroke}{rgb}{0.000000,0.000000,0.000000}%
\pgfsetstrokecolor{currentstroke}%
\pgfsetdash{}{0pt}%
\pgfpathmoveto{\pgfqpoint{4.267628in}{2.613520in}}%
\pgfpathlineto{\pgfqpoint{4.511410in}{2.753953in}}%
\pgfpathlineto{\pgfqpoint{4.267628in}{2.613520in}}%
\pgfpathclose%
\pgfusepath{fill}%
\end{pgfscope}%
\begin{pgfscope}%
\pgfpathrectangle{\pgfqpoint{0.017500in}{0.017500in}}{\pgfqpoint{6.218282in}{3.580372in}} %
\pgfusepath{clip}%
\pgfsetbuttcap%
\pgfsetroundjoin%
\definecolor{currentfill}{rgb}{0.729196,0.086679,0.167240}%
\pgfsetfillcolor{currentfill}%
\pgfsetlinewidth{0.000000pt}%
\definecolor{currentstroke}{rgb}{0.000000,0.000000,0.000000}%
\pgfsetstrokecolor{currentstroke}%
\pgfsetdash{}{0pt}%
\pgfpathmoveto{\pgfqpoint{4.267628in}{2.613520in}}%
\pgfpathlineto{\pgfqpoint{4.511410in}{2.753953in}}%
\pgfpathlineto{\pgfqpoint{4.267628in}{2.613520in}}%
\pgfpathclose%
\pgfusepath{fill}%
\end{pgfscope}%
\begin{pgfscope}%
\pgfpathrectangle{\pgfqpoint{0.017500in}{0.017500in}}{\pgfqpoint{6.218282in}{3.580372in}} %
\pgfusepath{clip}%
\pgfsetbuttcap%
\pgfsetroundjoin%
\definecolor{currentfill}{rgb}{0.729196,0.086679,0.167240}%
\pgfsetfillcolor{currentfill}%
\pgfsetlinewidth{0.000000pt}%
\definecolor{currentstroke}{rgb}{0.000000,0.000000,0.000000}%
\pgfsetstrokecolor{currentstroke}%
\pgfsetdash{}{0pt}%
\pgfpathmoveto{\pgfqpoint{4.267628in}{2.613520in}}%
\pgfpathlineto{\pgfqpoint{4.511410in}{2.753953in}}%
\pgfpathlineto{\pgfqpoint{4.267628in}{2.613520in}}%
\pgfpathclose%
\pgfusepath{fill}%
\end{pgfscope}%
\begin{pgfscope}%
\pgfpathrectangle{\pgfqpoint{0.017500in}{0.017500in}}{\pgfqpoint{6.218282in}{3.580372in}} %
\pgfusepath{clip}%
\pgfsetbuttcap%
\pgfsetroundjoin%
\definecolor{currentfill}{rgb}{0.729196,0.086679,0.167240}%
\pgfsetfillcolor{currentfill}%
\pgfsetlinewidth{0.000000pt}%
\definecolor{currentstroke}{rgb}{0.000000,0.000000,0.000000}%
\pgfsetstrokecolor{currentstroke}%
\pgfsetdash{}{0pt}%
\pgfpathmoveto{\pgfqpoint{4.267628in}{2.613520in}}%
\pgfpathlineto{\pgfqpoint{4.511410in}{2.753953in}}%
\pgfpathlineto{\pgfqpoint{4.267628in}{2.613520in}}%
\pgfpathclose%
\pgfusepath{fill}%
\end{pgfscope}%
\begin{pgfscope}%
\pgfpathrectangle{\pgfqpoint{0.017500in}{0.017500in}}{\pgfqpoint{6.218282in}{3.580372in}} %
\pgfusepath{clip}%
\pgfsetbuttcap%
\pgfsetroundjoin%
\definecolor{currentfill}{rgb}{0.729196,0.086679,0.167240}%
\pgfsetfillcolor{currentfill}%
\pgfsetlinewidth{0.000000pt}%
\definecolor{currentstroke}{rgb}{0.000000,0.000000,0.000000}%
\pgfsetstrokecolor{currentstroke}%
\pgfsetdash{}{0pt}%
\pgfpathmoveto{\pgfqpoint{4.267628in}{2.613520in}}%
\pgfpathlineto{\pgfqpoint{4.511410in}{2.753953in}}%
\pgfpathlineto{\pgfqpoint{4.267628in}{2.613520in}}%
\pgfpathclose%
\pgfusepath{fill}%
\end{pgfscope}%
\begin{pgfscope}%
\pgfpathrectangle{\pgfqpoint{0.017500in}{0.017500in}}{\pgfqpoint{6.218282in}{3.580372in}} %
\pgfusepath{clip}%
\pgfsetbuttcap%
\pgfsetroundjoin%
\definecolor{currentfill}{rgb}{0.729196,0.086679,0.167240}%
\pgfsetfillcolor{currentfill}%
\pgfsetlinewidth{0.000000pt}%
\definecolor{currentstroke}{rgb}{0.000000,0.000000,0.000000}%
\pgfsetstrokecolor{currentstroke}%
\pgfsetdash{}{0pt}%
\pgfpathmoveto{\pgfqpoint{4.267628in}{2.613520in}}%
\pgfpathlineto{\pgfqpoint{4.511410in}{2.753953in}}%
\pgfpathlineto{\pgfqpoint{4.267628in}{2.613520in}}%
\pgfpathclose%
\pgfusepath{fill}%
\end{pgfscope}%
\begin{pgfscope}%
\pgfpathrectangle{\pgfqpoint{0.017500in}{0.017500in}}{\pgfqpoint{6.218282in}{3.580372in}} %
\pgfusepath{clip}%
\pgfsetbuttcap%
\pgfsetroundjoin%
\definecolor{currentfill}{rgb}{0.729196,0.086679,0.167240}%
\pgfsetfillcolor{currentfill}%
\pgfsetlinewidth{0.000000pt}%
\definecolor{currentstroke}{rgb}{0.000000,0.000000,0.000000}%
\pgfsetstrokecolor{currentstroke}%
\pgfsetdash{}{0pt}%
\pgfpathmoveto{\pgfqpoint{4.267628in}{2.613520in}}%
\pgfpathlineto{\pgfqpoint{4.511410in}{2.753953in}}%
\pgfpathlineto{\pgfqpoint{4.267628in}{2.613520in}}%
\pgfpathclose%
\pgfusepath{fill}%
\end{pgfscope}%
\begin{pgfscope}%
\pgfpathrectangle{\pgfqpoint{0.017500in}{0.017500in}}{\pgfqpoint{6.218282in}{3.580372in}} %
\pgfusepath{clip}%
\pgfsetbuttcap%
\pgfsetroundjoin%
\definecolor{currentfill}{rgb}{0.729196,0.086679,0.167240}%
\pgfsetfillcolor{currentfill}%
\pgfsetlinewidth{0.000000pt}%
\definecolor{currentstroke}{rgb}{0.000000,0.000000,0.000000}%
\pgfsetstrokecolor{currentstroke}%
\pgfsetdash{}{0pt}%
\pgfpathmoveto{\pgfqpoint{4.267628in}{2.613520in}}%
\pgfpathlineto{\pgfqpoint{4.511410in}{2.753953in}}%
\pgfpathlineto{\pgfqpoint{4.267628in}{2.613520in}}%
\pgfpathclose%
\pgfusepath{fill}%
\end{pgfscope}%
\begin{pgfscope}%
\pgfpathrectangle{\pgfqpoint{0.017500in}{0.017500in}}{\pgfqpoint{6.218282in}{3.580372in}} %
\pgfusepath{clip}%
\pgfsetbuttcap%
\pgfsetroundjoin%
\definecolor{currentfill}{rgb}{0.729196,0.086679,0.167240}%
\pgfsetfillcolor{currentfill}%
\pgfsetlinewidth{0.000000pt}%
\definecolor{currentstroke}{rgb}{0.000000,0.000000,0.000000}%
\pgfsetstrokecolor{currentstroke}%
\pgfsetdash{}{0pt}%
\pgfpathmoveto{\pgfqpoint{4.267628in}{2.613520in}}%
\pgfpathlineto{\pgfqpoint{4.511410in}{2.753953in}}%
\pgfpathlineto{\pgfqpoint{4.267628in}{2.613520in}}%
\pgfpathclose%
\pgfusepath{fill}%
\end{pgfscope}%
\begin{pgfscope}%
\pgfpathrectangle{\pgfqpoint{0.017500in}{0.017500in}}{\pgfqpoint{6.218282in}{3.580372in}} %
\pgfusepath{clip}%
\pgfsetbuttcap%
\pgfsetroundjoin%
\definecolor{currentfill}{rgb}{0.729196,0.086679,0.167240}%
\pgfsetfillcolor{currentfill}%
\pgfsetlinewidth{0.000000pt}%
\definecolor{currentstroke}{rgb}{0.000000,0.000000,0.000000}%
\pgfsetstrokecolor{currentstroke}%
\pgfsetdash{}{0pt}%
\pgfpathmoveto{\pgfqpoint{4.267628in}{2.613520in}}%
\pgfpathlineto{\pgfqpoint{4.511410in}{2.753953in}}%
\pgfpathlineto{\pgfqpoint{4.267628in}{2.613520in}}%
\pgfpathclose%
\pgfusepath{fill}%
\end{pgfscope}%
\begin{pgfscope}%
\pgfpathrectangle{\pgfqpoint{0.017500in}{0.017500in}}{\pgfqpoint{6.218282in}{3.580372in}} %
\pgfusepath{clip}%
\pgfsetbuttcap%
\pgfsetroundjoin%
\definecolor{currentfill}{rgb}{0.729196,0.086679,0.167240}%
\pgfsetfillcolor{currentfill}%
\pgfsetlinewidth{0.000000pt}%
\definecolor{currentstroke}{rgb}{0.000000,0.000000,0.000000}%
\pgfsetstrokecolor{currentstroke}%
\pgfsetdash{}{0pt}%
\pgfpathmoveto{\pgfqpoint{4.267628in}{2.613520in}}%
\pgfpathlineto{\pgfqpoint{4.511410in}{2.753953in}}%
\pgfpathlineto{\pgfqpoint{4.267628in}{2.613520in}}%
\pgfpathclose%
\pgfusepath{fill}%
\end{pgfscope}%
\begin{pgfscope}%
\pgfpathrectangle{\pgfqpoint{0.017500in}{0.017500in}}{\pgfqpoint{6.218282in}{3.580372in}} %
\pgfusepath{clip}%
\pgfsetbuttcap%
\pgfsetroundjoin%
\definecolor{currentfill}{rgb}{0.729196,0.086679,0.167240}%
\pgfsetfillcolor{currentfill}%
\pgfsetlinewidth{0.000000pt}%
\definecolor{currentstroke}{rgb}{0.000000,0.000000,0.000000}%
\pgfsetstrokecolor{currentstroke}%
\pgfsetdash{}{0pt}%
\pgfpathmoveto{\pgfqpoint{4.267628in}{2.613520in}}%
\pgfpathlineto{\pgfqpoint{4.511410in}{2.753953in}}%
\pgfpathlineto{\pgfqpoint{4.267628in}{2.613520in}}%
\pgfpathclose%
\pgfusepath{fill}%
\end{pgfscope}%
\begin{pgfscope}%
\pgfpathrectangle{\pgfqpoint{0.017500in}{0.017500in}}{\pgfqpoint{6.218282in}{3.580372in}} %
\pgfusepath{clip}%
\pgfsetbuttcap%
\pgfsetroundjoin%
\definecolor{currentfill}{rgb}{0.729196,0.086679,0.167240}%
\pgfsetfillcolor{currentfill}%
\pgfsetlinewidth{0.000000pt}%
\definecolor{currentstroke}{rgb}{0.000000,0.000000,0.000000}%
\pgfsetstrokecolor{currentstroke}%
\pgfsetdash{}{0pt}%
\pgfpathmoveto{\pgfqpoint{4.267628in}{2.613520in}}%
\pgfpathlineto{\pgfqpoint{4.511410in}{2.753953in}}%
\pgfpathlineto{\pgfqpoint{4.267628in}{2.613520in}}%
\pgfpathclose%
\pgfusepath{fill}%
\end{pgfscope}%
\begin{pgfscope}%
\pgfpathrectangle{\pgfqpoint{0.017500in}{0.017500in}}{\pgfqpoint{6.218282in}{3.580372in}} %
\pgfusepath{clip}%
\pgfsetbuttcap%
\pgfsetroundjoin%
\definecolor{currentfill}{rgb}{0.729196,0.086679,0.167240}%
\pgfsetfillcolor{currentfill}%
\pgfsetlinewidth{0.000000pt}%
\definecolor{currentstroke}{rgb}{0.000000,0.000000,0.000000}%
\pgfsetstrokecolor{currentstroke}%
\pgfsetdash{}{0pt}%
\pgfpathmoveto{\pgfqpoint{4.267628in}{2.613520in}}%
\pgfpathlineto{\pgfqpoint{4.511410in}{2.753953in}}%
\pgfpathlineto{\pgfqpoint{4.267628in}{2.613520in}}%
\pgfpathclose%
\pgfusepath{fill}%
\end{pgfscope}%
\begin{pgfscope}%
\pgfpathrectangle{\pgfqpoint{0.017500in}{0.017500in}}{\pgfqpoint{6.218282in}{3.580372in}} %
\pgfusepath{clip}%
\pgfsetbuttcap%
\pgfsetroundjoin%
\definecolor{currentfill}{rgb}{0.729196,0.086679,0.167240}%
\pgfsetfillcolor{currentfill}%
\pgfsetlinewidth{0.000000pt}%
\definecolor{currentstroke}{rgb}{0.000000,0.000000,0.000000}%
\pgfsetstrokecolor{currentstroke}%
\pgfsetdash{}{0pt}%
\pgfpathmoveto{\pgfqpoint{4.267628in}{2.613520in}}%
\pgfpathlineto{\pgfqpoint{4.511410in}{2.753953in}}%
\pgfpathlineto{\pgfqpoint{4.267628in}{2.613520in}}%
\pgfpathclose%
\pgfusepath{fill}%
\end{pgfscope}%
\begin{pgfscope}%
\pgfpathrectangle{\pgfqpoint{0.017500in}{0.017500in}}{\pgfqpoint{6.218282in}{3.580372in}} %
\pgfusepath{clip}%
\pgfsetbuttcap%
\pgfsetroundjoin%
\definecolor{currentfill}{rgb}{0.729196,0.086679,0.167240}%
\pgfsetfillcolor{currentfill}%
\pgfsetlinewidth{0.000000pt}%
\definecolor{currentstroke}{rgb}{0.000000,0.000000,0.000000}%
\pgfsetstrokecolor{currentstroke}%
\pgfsetdash{}{0pt}%
\pgfpathmoveto{\pgfqpoint{4.267628in}{2.613520in}}%
\pgfpathlineto{\pgfqpoint{4.511410in}{2.753953in}}%
\pgfpathlineto{\pgfqpoint{4.267628in}{2.613520in}}%
\pgfpathclose%
\pgfusepath{fill}%
\end{pgfscope}%
\begin{pgfscope}%
\pgfpathrectangle{\pgfqpoint{0.017500in}{0.017500in}}{\pgfqpoint{6.218282in}{3.580372in}} %
\pgfusepath{clip}%
\pgfsetbuttcap%
\pgfsetroundjoin%
\definecolor{currentfill}{rgb}{0.729196,0.086679,0.167240}%
\pgfsetfillcolor{currentfill}%
\pgfsetlinewidth{0.000000pt}%
\definecolor{currentstroke}{rgb}{0.000000,0.000000,0.000000}%
\pgfsetstrokecolor{currentstroke}%
\pgfsetdash{}{0pt}%
\pgfpathmoveto{\pgfqpoint{4.267628in}{2.613520in}}%
\pgfpathlineto{\pgfqpoint{4.511410in}{2.753953in}}%
\pgfpathlineto{\pgfqpoint{4.267628in}{2.613520in}}%
\pgfpathclose%
\pgfusepath{fill}%
\end{pgfscope}%
\begin{pgfscope}%
\pgfpathrectangle{\pgfqpoint{0.017500in}{0.017500in}}{\pgfqpoint{6.218282in}{3.580372in}} %
\pgfusepath{clip}%
\pgfsetbuttcap%
\pgfsetroundjoin%
\definecolor{currentfill}{rgb}{0.729196,0.086679,0.167240}%
\pgfsetfillcolor{currentfill}%
\pgfsetlinewidth{0.000000pt}%
\definecolor{currentstroke}{rgb}{0.000000,0.000000,0.000000}%
\pgfsetstrokecolor{currentstroke}%
\pgfsetdash{}{0pt}%
\pgfpathmoveto{\pgfqpoint{4.267628in}{2.613520in}}%
\pgfpathlineto{\pgfqpoint{4.511410in}{2.753953in}}%
\pgfpathlineto{\pgfqpoint{4.267628in}{2.613520in}}%
\pgfpathclose%
\pgfusepath{fill}%
\end{pgfscope}%
\begin{pgfscope}%
\pgfpathrectangle{\pgfqpoint{0.017500in}{0.017500in}}{\pgfqpoint{6.218282in}{3.580372in}} %
\pgfusepath{clip}%
\pgfsetbuttcap%
\pgfsetroundjoin%
\definecolor{currentfill}{rgb}{0.729196,0.086679,0.167240}%
\pgfsetfillcolor{currentfill}%
\pgfsetlinewidth{0.000000pt}%
\definecolor{currentstroke}{rgb}{0.000000,0.000000,0.000000}%
\pgfsetstrokecolor{currentstroke}%
\pgfsetdash{}{0pt}%
\pgfpathmoveto{\pgfqpoint{4.267628in}{2.613520in}}%
\pgfpathlineto{\pgfqpoint{4.511410in}{2.753953in}}%
\pgfpathlineto{\pgfqpoint{4.267628in}{2.613520in}}%
\pgfpathclose%
\pgfusepath{fill}%
\end{pgfscope}%
\begin{pgfscope}%
\pgfpathrectangle{\pgfqpoint{0.017500in}{0.017500in}}{\pgfqpoint{6.218282in}{3.580372in}} %
\pgfusepath{clip}%
\pgfsetbuttcap%
\pgfsetroundjoin%
\definecolor{currentfill}{rgb}{0.729196,0.086679,0.167240}%
\pgfsetfillcolor{currentfill}%
\pgfsetlinewidth{0.000000pt}%
\definecolor{currentstroke}{rgb}{0.000000,0.000000,0.000000}%
\pgfsetstrokecolor{currentstroke}%
\pgfsetdash{}{0pt}%
\pgfpathmoveto{\pgfqpoint{4.267628in}{2.613520in}}%
\pgfpathlineto{\pgfqpoint{4.511410in}{2.753953in}}%
\pgfpathlineto{\pgfqpoint{4.267628in}{2.613520in}}%
\pgfpathclose%
\pgfusepath{fill}%
\end{pgfscope}%
\begin{pgfscope}%
\pgfpathrectangle{\pgfqpoint{0.017500in}{0.017500in}}{\pgfqpoint{6.218282in}{3.580372in}} %
\pgfusepath{clip}%
\pgfsetbuttcap%
\pgfsetroundjoin%
\definecolor{currentfill}{rgb}{0.729196,0.086679,0.167240}%
\pgfsetfillcolor{currentfill}%
\pgfsetlinewidth{0.000000pt}%
\definecolor{currentstroke}{rgb}{0.000000,0.000000,0.000000}%
\pgfsetstrokecolor{currentstroke}%
\pgfsetdash{}{0pt}%
\pgfpathmoveto{\pgfqpoint{4.267628in}{2.613520in}}%
\pgfpathlineto{\pgfqpoint{4.511410in}{2.753953in}}%
\pgfpathlineto{\pgfqpoint{4.267628in}{2.613520in}}%
\pgfpathclose%
\pgfusepath{fill}%
\end{pgfscope}%
\begin{pgfscope}%
\pgfpathrectangle{\pgfqpoint{0.017500in}{0.017500in}}{\pgfqpoint{6.218282in}{3.580372in}} %
\pgfusepath{clip}%
\pgfsetbuttcap%
\pgfsetroundjoin%
\definecolor{currentfill}{rgb}{0.729196,0.086679,0.167240}%
\pgfsetfillcolor{currentfill}%
\pgfsetlinewidth{0.000000pt}%
\definecolor{currentstroke}{rgb}{0.000000,0.000000,0.000000}%
\pgfsetstrokecolor{currentstroke}%
\pgfsetdash{}{0pt}%
\pgfpathmoveto{\pgfqpoint{4.267628in}{2.613520in}}%
\pgfpathlineto{\pgfqpoint{4.511410in}{2.753953in}}%
\pgfpathlineto{\pgfqpoint{4.267628in}{2.613520in}}%
\pgfpathclose%
\pgfusepath{fill}%
\end{pgfscope}%
\begin{pgfscope}%
\pgfpathrectangle{\pgfqpoint{0.017500in}{0.017500in}}{\pgfqpoint{6.218282in}{3.580372in}} %
\pgfusepath{clip}%
\pgfsetbuttcap%
\pgfsetroundjoin%
\definecolor{currentfill}{rgb}{0.729196,0.086679,0.167240}%
\pgfsetfillcolor{currentfill}%
\pgfsetlinewidth{0.000000pt}%
\definecolor{currentstroke}{rgb}{0.000000,0.000000,0.000000}%
\pgfsetstrokecolor{currentstroke}%
\pgfsetdash{}{0pt}%
\pgfpathmoveto{\pgfqpoint{4.267628in}{2.613520in}}%
\pgfpathlineto{\pgfqpoint{4.511410in}{2.753953in}}%
\pgfpathlineto{\pgfqpoint{4.267628in}{2.613520in}}%
\pgfpathclose%
\pgfusepath{fill}%
\end{pgfscope}%
\begin{pgfscope}%
\pgfpathrectangle{\pgfqpoint{0.017500in}{0.017500in}}{\pgfqpoint{6.218282in}{3.580372in}} %
\pgfusepath{clip}%
\pgfsetbuttcap%
\pgfsetroundjoin%
\definecolor{currentfill}{rgb}{0.729196,0.086679,0.167240}%
\pgfsetfillcolor{currentfill}%
\pgfsetlinewidth{0.000000pt}%
\definecolor{currentstroke}{rgb}{0.000000,0.000000,0.000000}%
\pgfsetstrokecolor{currentstroke}%
\pgfsetdash{}{0pt}%
\pgfpathmoveto{\pgfqpoint{4.267628in}{2.613520in}}%
\pgfpathlineto{\pgfqpoint{4.511410in}{2.753953in}}%
\pgfpathlineto{\pgfqpoint{4.267628in}{2.613520in}}%
\pgfpathclose%
\pgfusepath{fill}%
\end{pgfscope}%
\begin{pgfscope}%
\pgfpathrectangle{\pgfqpoint{0.017500in}{0.017500in}}{\pgfqpoint{6.218282in}{3.580372in}} %
\pgfusepath{clip}%
\pgfsetbuttcap%
\pgfsetroundjoin%
\definecolor{currentfill}{rgb}{0.729196,0.086679,0.167240}%
\pgfsetfillcolor{currentfill}%
\pgfsetlinewidth{0.000000pt}%
\definecolor{currentstroke}{rgb}{0.000000,0.000000,0.000000}%
\pgfsetstrokecolor{currentstroke}%
\pgfsetdash{}{0pt}%
\pgfpathmoveto{\pgfqpoint{4.267628in}{2.613520in}}%
\pgfpathlineto{\pgfqpoint{4.511410in}{2.753953in}}%
\pgfpathlineto{\pgfqpoint{4.267628in}{2.613520in}}%
\pgfpathclose%
\pgfusepath{fill}%
\end{pgfscope}%
\begin{pgfscope}%
\pgfpathrectangle{\pgfqpoint{0.017500in}{0.017500in}}{\pgfqpoint{6.218282in}{3.580372in}} %
\pgfusepath{clip}%
\pgfsetbuttcap%
\pgfsetroundjoin%
\definecolor{currentfill}{rgb}{0.810616,0.268797,0.235428}%
\pgfsetfillcolor{currentfill}%
\pgfsetlinewidth{0.000000pt}%
\definecolor{currentstroke}{rgb}{0.000000,0.000000,0.000000}%
\pgfsetstrokecolor{currentstroke}%
\pgfsetdash{}{0pt}%
\pgfpathmoveto{\pgfqpoint{2.410872in}{2.508083in}}%
\pgfpathlineto{\pgfqpoint{2.674254in}{2.688576in}}%
\pgfpathlineto{\pgfqpoint{2.410872in}{2.508083in}}%
\pgfpathclose%
\pgfusepath{fill}%
\end{pgfscope}%
\begin{pgfscope}%
\pgfpathrectangle{\pgfqpoint{0.017500in}{0.017500in}}{\pgfqpoint{6.218282in}{3.580372in}} %
\pgfusepath{clip}%
\pgfsetbuttcap%
\pgfsetroundjoin%
\definecolor{currentfill}{rgb}{0.810616,0.268797,0.235428}%
\pgfsetfillcolor{currentfill}%
\pgfsetlinewidth{0.000000pt}%
\definecolor{currentstroke}{rgb}{0.000000,0.000000,0.000000}%
\pgfsetstrokecolor{currentstroke}%
\pgfsetdash{}{0pt}%
\pgfpathmoveto{\pgfqpoint{2.410872in}{2.508083in}}%
\pgfpathlineto{\pgfqpoint{2.674254in}{2.688576in}}%
\pgfpathlineto{\pgfqpoint{2.410872in}{2.508083in}}%
\pgfpathclose%
\pgfusepath{fill}%
\end{pgfscope}%
\begin{pgfscope}%
\pgfpathrectangle{\pgfqpoint{0.017500in}{0.017500in}}{\pgfqpoint{6.218282in}{3.580372in}} %
\pgfusepath{clip}%
\pgfsetbuttcap%
\pgfsetroundjoin%
\definecolor{currentfill}{rgb}{0.810616,0.268797,0.235428}%
\pgfsetfillcolor{currentfill}%
\pgfsetlinewidth{0.000000pt}%
\definecolor{currentstroke}{rgb}{0.000000,0.000000,0.000000}%
\pgfsetstrokecolor{currentstroke}%
\pgfsetdash{}{0pt}%
\pgfpathmoveto{\pgfqpoint{2.410872in}{2.508083in}}%
\pgfpathlineto{\pgfqpoint{2.674254in}{2.688576in}}%
\pgfpathlineto{\pgfqpoint{2.410872in}{2.508083in}}%
\pgfpathclose%
\pgfusepath{fill}%
\end{pgfscope}%
\begin{pgfscope}%
\pgfpathrectangle{\pgfqpoint{0.017500in}{0.017500in}}{\pgfqpoint{6.218282in}{3.580372in}} %
\pgfusepath{clip}%
\pgfsetbuttcap%
\pgfsetroundjoin%
\definecolor{currentfill}{rgb}{0.810616,0.268797,0.235428}%
\pgfsetfillcolor{currentfill}%
\pgfsetlinewidth{0.000000pt}%
\definecolor{currentstroke}{rgb}{0.000000,0.000000,0.000000}%
\pgfsetstrokecolor{currentstroke}%
\pgfsetdash{}{0pt}%
\pgfpathmoveto{\pgfqpoint{2.410872in}{2.508083in}}%
\pgfpathlineto{\pgfqpoint{2.674254in}{2.688576in}}%
\pgfpathlineto{\pgfqpoint{2.410872in}{2.508083in}}%
\pgfpathclose%
\pgfusepath{fill}%
\end{pgfscope}%
\begin{pgfscope}%
\pgfpathrectangle{\pgfqpoint{0.017500in}{0.017500in}}{\pgfqpoint{6.218282in}{3.580372in}} %
\pgfusepath{clip}%
\pgfsetbuttcap%
\pgfsetroundjoin%
\definecolor{currentfill}{rgb}{0.810616,0.268797,0.235428}%
\pgfsetfillcolor{currentfill}%
\pgfsetlinewidth{0.000000pt}%
\definecolor{currentstroke}{rgb}{0.000000,0.000000,0.000000}%
\pgfsetstrokecolor{currentstroke}%
\pgfsetdash{}{0pt}%
\pgfpathmoveto{\pgfqpoint{2.410872in}{2.508083in}}%
\pgfpathlineto{\pgfqpoint{2.674254in}{2.688576in}}%
\pgfpathlineto{\pgfqpoint{2.410872in}{2.508083in}}%
\pgfpathclose%
\pgfusepath{fill}%
\end{pgfscope}%
\begin{pgfscope}%
\pgfpathrectangle{\pgfqpoint{0.017500in}{0.017500in}}{\pgfqpoint{6.218282in}{3.580372in}} %
\pgfusepath{clip}%
\pgfsetbuttcap%
\pgfsetroundjoin%
\definecolor{currentfill}{rgb}{0.810616,0.268797,0.235428}%
\pgfsetfillcolor{currentfill}%
\pgfsetlinewidth{0.000000pt}%
\definecolor{currentstroke}{rgb}{0.000000,0.000000,0.000000}%
\pgfsetstrokecolor{currentstroke}%
\pgfsetdash{}{0pt}%
\pgfpathmoveto{\pgfqpoint{2.410872in}{2.508083in}}%
\pgfpathlineto{\pgfqpoint{2.674254in}{2.688576in}}%
\pgfpathlineto{\pgfqpoint{2.410872in}{2.508083in}}%
\pgfpathclose%
\pgfusepath{fill}%
\end{pgfscope}%
\begin{pgfscope}%
\pgfpathrectangle{\pgfqpoint{0.017500in}{0.017500in}}{\pgfqpoint{6.218282in}{3.580372in}} %
\pgfusepath{clip}%
\pgfsetbuttcap%
\pgfsetroundjoin%
\definecolor{currentfill}{rgb}{0.810616,0.268797,0.235428}%
\pgfsetfillcolor{currentfill}%
\pgfsetlinewidth{0.000000pt}%
\definecolor{currentstroke}{rgb}{0.000000,0.000000,0.000000}%
\pgfsetstrokecolor{currentstroke}%
\pgfsetdash{}{0pt}%
\pgfpathmoveto{\pgfqpoint{2.410872in}{2.508083in}}%
\pgfpathlineto{\pgfqpoint{2.674254in}{2.688576in}}%
\pgfpathlineto{\pgfqpoint{2.410872in}{2.508083in}}%
\pgfpathclose%
\pgfusepath{fill}%
\end{pgfscope}%
\begin{pgfscope}%
\pgfpathrectangle{\pgfqpoint{0.017500in}{0.017500in}}{\pgfqpoint{6.218282in}{3.580372in}} %
\pgfusepath{clip}%
\pgfsetbuttcap%
\pgfsetroundjoin%
\definecolor{currentfill}{rgb}{0.810616,0.268797,0.235428}%
\pgfsetfillcolor{currentfill}%
\pgfsetlinewidth{0.000000pt}%
\definecolor{currentstroke}{rgb}{0.000000,0.000000,0.000000}%
\pgfsetstrokecolor{currentstroke}%
\pgfsetdash{}{0pt}%
\pgfpathmoveto{\pgfqpoint{2.410872in}{2.508083in}}%
\pgfpathlineto{\pgfqpoint{2.674254in}{2.688576in}}%
\pgfpathlineto{\pgfqpoint{2.410872in}{2.508083in}}%
\pgfpathclose%
\pgfusepath{fill}%
\end{pgfscope}%
\begin{pgfscope}%
\pgfpathrectangle{\pgfqpoint{0.017500in}{0.017500in}}{\pgfqpoint{6.218282in}{3.580372in}} %
\pgfusepath{clip}%
\pgfsetbuttcap%
\pgfsetroundjoin%
\definecolor{currentfill}{rgb}{0.810616,0.268797,0.235428}%
\pgfsetfillcolor{currentfill}%
\pgfsetlinewidth{0.000000pt}%
\definecolor{currentstroke}{rgb}{0.000000,0.000000,0.000000}%
\pgfsetstrokecolor{currentstroke}%
\pgfsetdash{}{0pt}%
\pgfpathmoveto{\pgfqpoint{2.410872in}{2.508083in}}%
\pgfpathlineto{\pgfqpoint{2.674254in}{2.688576in}}%
\pgfpathlineto{\pgfqpoint{2.410872in}{2.508083in}}%
\pgfpathclose%
\pgfusepath{fill}%
\end{pgfscope}%
\begin{pgfscope}%
\pgfpathrectangle{\pgfqpoint{0.017500in}{0.017500in}}{\pgfqpoint{6.218282in}{3.580372in}} %
\pgfusepath{clip}%
\pgfsetbuttcap%
\pgfsetroundjoin%
\definecolor{currentfill}{rgb}{0.810616,0.268797,0.235428}%
\pgfsetfillcolor{currentfill}%
\pgfsetlinewidth{0.000000pt}%
\definecolor{currentstroke}{rgb}{0.000000,0.000000,0.000000}%
\pgfsetstrokecolor{currentstroke}%
\pgfsetdash{}{0pt}%
\pgfpathmoveto{\pgfqpoint{2.410872in}{2.508083in}}%
\pgfpathlineto{\pgfqpoint{2.674254in}{2.688576in}}%
\pgfpathlineto{\pgfqpoint{2.410872in}{2.508083in}}%
\pgfpathclose%
\pgfusepath{fill}%
\end{pgfscope}%
\begin{pgfscope}%
\pgfpathrectangle{\pgfqpoint{0.017500in}{0.017500in}}{\pgfqpoint{6.218282in}{3.580372in}} %
\pgfusepath{clip}%
\pgfsetbuttcap%
\pgfsetroundjoin%
\definecolor{currentfill}{rgb}{0.810616,0.268797,0.235428}%
\pgfsetfillcolor{currentfill}%
\pgfsetlinewidth{0.000000pt}%
\definecolor{currentstroke}{rgb}{0.000000,0.000000,0.000000}%
\pgfsetstrokecolor{currentstroke}%
\pgfsetdash{}{0pt}%
\pgfpathmoveto{\pgfqpoint{2.410872in}{2.508083in}}%
\pgfpathlineto{\pgfqpoint{2.674254in}{2.688576in}}%
\pgfpathlineto{\pgfqpoint{2.410872in}{2.508083in}}%
\pgfpathclose%
\pgfusepath{fill}%
\end{pgfscope}%
\begin{pgfscope}%
\pgfpathrectangle{\pgfqpoint{0.017500in}{0.017500in}}{\pgfqpoint{6.218282in}{3.580372in}} %
\pgfusepath{clip}%
\pgfsetbuttcap%
\pgfsetroundjoin%
\definecolor{currentfill}{rgb}{0.810616,0.268797,0.235428}%
\pgfsetfillcolor{currentfill}%
\pgfsetlinewidth{0.000000pt}%
\definecolor{currentstroke}{rgb}{0.000000,0.000000,0.000000}%
\pgfsetstrokecolor{currentstroke}%
\pgfsetdash{}{0pt}%
\pgfpathmoveto{\pgfqpoint{2.410872in}{2.508083in}}%
\pgfpathlineto{\pgfqpoint{2.674254in}{2.688576in}}%
\pgfpathlineto{\pgfqpoint{2.410872in}{2.508083in}}%
\pgfpathclose%
\pgfusepath{fill}%
\end{pgfscope}%
\begin{pgfscope}%
\pgfpathrectangle{\pgfqpoint{0.017500in}{0.017500in}}{\pgfqpoint{6.218282in}{3.580372in}} %
\pgfusepath{clip}%
\pgfsetbuttcap%
\pgfsetroundjoin%
\definecolor{currentfill}{rgb}{0.810616,0.268797,0.235428}%
\pgfsetfillcolor{currentfill}%
\pgfsetlinewidth{0.000000pt}%
\definecolor{currentstroke}{rgb}{0.000000,0.000000,0.000000}%
\pgfsetstrokecolor{currentstroke}%
\pgfsetdash{}{0pt}%
\pgfpathmoveto{\pgfqpoint{2.410872in}{2.508083in}}%
\pgfpathlineto{\pgfqpoint{2.674254in}{2.688576in}}%
\pgfpathlineto{\pgfqpoint{2.410872in}{2.508083in}}%
\pgfpathclose%
\pgfusepath{fill}%
\end{pgfscope}%
\begin{pgfscope}%
\pgfpathrectangle{\pgfqpoint{0.017500in}{0.017500in}}{\pgfqpoint{6.218282in}{3.580372in}} %
\pgfusepath{clip}%
\pgfsetbuttcap%
\pgfsetroundjoin%
\definecolor{currentfill}{rgb}{0.810616,0.268797,0.235428}%
\pgfsetfillcolor{currentfill}%
\pgfsetlinewidth{0.000000pt}%
\definecolor{currentstroke}{rgb}{0.000000,0.000000,0.000000}%
\pgfsetstrokecolor{currentstroke}%
\pgfsetdash{}{0pt}%
\pgfpathmoveto{\pgfqpoint{2.410872in}{2.508083in}}%
\pgfpathlineto{\pgfqpoint{2.674254in}{2.688576in}}%
\pgfpathlineto{\pgfqpoint{2.410872in}{2.508083in}}%
\pgfpathclose%
\pgfusepath{fill}%
\end{pgfscope}%
\begin{pgfscope}%
\pgfpathrectangle{\pgfqpoint{0.017500in}{0.017500in}}{\pgfqpoint{6.218282in}{3.580372in}} %
\pgfusepath{clip}%
\pgfsetbuttcap%
\pgfsetroundjoin%
\definecolor{currentfill}{rgb}{0.810616,0.268797,0.235428}%
\pgfsetfillcolor{currentfill}%
\pgfsetlinewidth{0.000000pt}%
\definecolor{currentstroke}{rgb}{0.000000,0.000000,0.000000}%
\pgfsetstrokecolor{currentstroke}%
\pgfsetdash{}{0pt}%
\pgfpathmoveto{\pgfqpoint{2.410872in}{2.508083in}}%
\pgfpathlineto{\pgfqpoint{2.674254in}{2.688576in}}%
\pgfpathlineto{\pgfqpoint{2.410872in}{2.508083in}}%
\pgfpathclose%
\pgfusepath{fill}%
\end{pgfscope}%
\begin{pgfscope}%
\pgfpathrectangle{\pgfqpoint{0.017500in}{0.017500in}}{\pgfqpoint{6.218282in}{3.580372in}} %
\pgfusepath{clip}%
\pgfsetbuttcap%
\pgfsetroundjoin%
\definecolor{currentfill}{rgb}{0.810616,0.268797,0.235428}%
\pgfsetfillcolor{currentfill}%
\pgfsetlinewidth{0.000000pt}%
\definecolor{currentstroke}{rgb}{0.000000,0.000000,0.000000}%
\pgfsetstrokecolor{currentstroke}%
\pgfsetdash{}{0pt}%
\pgfpathmoveto{\pgfqpoint{2.410872in}{2.508083in}}%
\pgfpathlineto{\pgfqpoint{2.674254in}{2.688576in}}%
\pgfpathlineto{\pgfqpoint{2.410872in}{2.508083in}}%
\pgfpathclose%
\pgfusepath{fill}%
\end{pgfscope}%
\begin{pgfscope}%
\pgfpathrectangle{\pgfqpoint{0.017500in}{0.017500in}}{\pgfqpoint{6.218282in}{3.580372in}} %
\pgfusepath{clip}%
\pgfsetbuttcap%
\pgfsetroundjoin%
\definecolor{currentfill}{rgb}{0.810616,0.268797,0.235428}%
\pgfsetfillcolor{currentfill}%
\pgfsetlinewidth{0.000000pt}%
\definecolor{currentstroke}{rgb}{0.000000,0.000000,0.000000}%
\pgfsetstrokecolor{currentstroke}%
\pgfsetdash{}{0pt}%
\pgfpathmoveto{\pgfqpoint{2.410872in}{2.508083in}}%
\pgfpathlineto{\pgfqpoint{2.674254in}{2.688576in}}%
\pgfpathlineto{\pgfqpoint{2.410872in}{2.508083in}}%
\pgfpathclose%
\pgfusepath{fill}%
\end{pgfscope}%
\begin{pgfscope}%
\pgfpathrectangle{\pgfqpoint{0.017500in}{0.017500in}}{\pgfqpoint{6.218282in}{3.580372in}} %
\pgfusepath{clip}%
\pgfsetbuttcap%
\pgfsetroundjoin%
\definecolor{currentfill}{rgb}{0.810616,0.268797,0.235428}%
\pgfsetfillcolor{currentfill}%
\pgfsetlinewidth{0.000000pt}%
\definecolor{currentstroke}{rgb}{0.000000,0.000000,0.000000}%
\pgfsetstrokecolor{currentstroke}%
\pgfsetdash{}{0pt}%
\pgfpathmoveto{\pgfqpoint{2.410872in}{2.508083in}}%
\pgfpathlineto{\pgfqpoint{2.674254in}{2.688576in}}%
\pgfpathlineto{\pgfqpoint{2.410872in}{2.508083in}}%
\pgfpathclose%
\pgfusepath{fill}%
\end{pgfscope}%
\begin{pgfscope}%
\pgfpathrectangle{\pgfqpoint{0.017500in}{0.017500in}}{\pgfqpoint{6.218282in}{3.580372in}} %
\pgfusepath{clip}%
\pgfsetbuttcap%
\pgfsetroundjoin%
\definecolor{currentfill}{rgb}{0.810616,0.268797,0.235428}%
\pgfsetfillcolor{currentfill}%
\pgfsetlinewidth{0.000000pt}%
\definecolor{currentstroke}{rgb}{0.000000,0.000000,0.000000}%
\pgfsetstrokecolor{currentstroke}%
\pgfsetdash{}{0pt}%
\pgfpathmoveto{\pgfqpoint{2.410872in}{2.508083in}}%
\pgfpathlineto{\pgfqpoint{2.674254in}{2.688576in}}%
\pgfpathlineto{\pgfqpoint{2.410872in}{2.508083in}}%
\pgfpathclose%
\pgfusepath{fill}%
\end{pgfscope}%
\begin{pgfscope}%
\pgfpathrectangle{\pgfqpoint{0.017500in}{0.017500in}}{\pgfqpoint{6.218282in}{3.580372in}} %
\pgfusepath{clip}%
\pgfsetbuttcap%
\pgfsetroundjoin%
\definecolor{currentfill}{rgb}{0.810616,0.268797,0.235428}%
\pgfsetfillcolor{currentfill}%
\pgfsetlinewidth{0.000000pt}%
\definecolor{currentstroke}{rgb}{0.000000,0.000000,0.000000}%
\pgfsetstrokecolor{currentstroke}%
\pgfsetdash{}{0pt}%
\pgfpathmoveto{\pgfqpoint{2.410872in}{2.508083in}}%
\pgfpathlineto{\pgfqpoint{2.674254in}{2.688576in}}%
\pgfpathlineto{\pgfqpoint{2.410872in}{2.508083in}}%
\pgfpathclose%
\pgfusepath{fill}%
\end{pgfscope}%
\begin{pgfscope}%
\pgfpathrectangle{\pgfqpoint{0.017500in}{0.017500in}}{\pgfqpoint{6.218282in}{3.580372in}} %
\pgfusepath{clip}%
\pgfsetbuttcap%
\pgfsetroundjoin%
\definecolor{currentfill}{rgb}{0.810616,0.268797,0.235428}%
\pgfsetfillcolor{currentfill}%
\pgfsetlinewidth{0.000000pt}%
\definecolor{currentstroke}{rgb}{0.000000,0.000000,0.000000}%
\pgfsetstrokecolor{currentstroke}%
\pgfsetdash{}{0pt}%
\pgfpathmoveto{\pgfqpoint{2.410872in}{2.508083in}}%
\pgfpathlineto{\pgfqpoint{2.674254in}{2.688576in}}%
\pgfpathlineto{\pgfqpoint{2.410872in}{2.508083in}}%
\pgfpathclose%
\pgfusepath{fill}%
\end{pgfscope}%
\begin{pgfscope}%
\pgfpathrectangle{\pgfqpoint{0.017500in}{0.017500in}}{\pgfqpoint{6.218282in}{3.580372in}} %
\pgfusepath{clip}%
\pgfsetbuttcap%
\pgfsetroundjoin%
\definecolor{currentfill}{rgb}{0.810616,0.268797,0.235428}%
\pgfsetfillcolor{currentfill}%
\pgfsetlinewidth{0.000000pt}%
\definecolor{currentstroke}{rgb}{0.000000,0.000000,0.000000}%
\pgfsetstrokecolor{currentstroke}%
\pgfsetdash{}{0pt}%
\pgfpathmoveto{\pgfqpoint{2.410872in}{2.508083in}}%
\pgfpathlineto{\pgfqpoint{2.674254in}{2.688576in}}%
\pgfpathlineto{\pgfqpoint{2.410872in}{2.508083in}}%
\pgfpathclose%
\pgfusepath{fill}%
\end{pgfscope}%
\begin{pgfscope}%
\pgfpathrectangle{\pgfqpoint{0.017500in}{0.017500in}}{\pgfqpoint{6.218282in}{3.580372in}} %
\pgfusepath{clip}%
\pgfsetbuttcap%
\pgfsetroundjoin%
\definecolor{currentfill}{rgb}{0.810616,0.268797,0.235428}%
\pgfsetfillcolor{currentfill}%
\pgfsetlinewidth{0.000000pt}%
\definecolor{currentstroke}{rgb}{0.000000,0.000000,0.000000}%
\pgfsetstrokecolor{currentstroke}%
\pgfsetdash{}{0pt}%
\pgfpathmoveto{\pgfqpoint{2.410872in}{2.508083in}}%
\pgfpathlineto{\pgfqpoint{2.674254in}{2.688576in}}%
\pgfpathlineto{\pgfqpoint{2.410872in}{2.508083in}}%
\pgfpathclose%
\pgfusepath{fill}%
\end{pgfscope}%
\begin{pgfscope}%
\pgfpathrectangle{\pgfqpoint{0.017500in}{0.017500in}}{\pgfqpoint{6.218282in}{3.580372in}} %
\pgfusepath{clip}%
\pgfsetbuttcap%
\pgfsetroundjoin%
\definecolor{currentfill}{rgb}{0.810616,0.268797,0.235428}%
\pgfsetfillcolor{currentfill}%
\pgfsetlinewidth{0.000000pt}%
\definecolor{currentstroke}{rgb}{0.000000,0.000000,0.000000}%
\pgfsetstrokecolor{currentstroke}%
\pgfsetdash{}{0pt}%
\pgfpathmoveto{\pgfqpoint{2.410872in}{2.508083in}}%
\pgfpathlineto{\pgfqpoint{2.674254in}{2.688576in}}%
\pgfpathlineto{\pgfqpoint{2.410872in}{2.508083in}}%
\pgfpathclose%
\pgfusepath{fill}%
\end{pgfscope}%
\begin{pgfscope}%
\pgfpathrectangle{\pgfqpoint{0.017500in}{0.017500in}}{\pgfqpoint{6.218282in}{3.580372in}} %
\pgfusepath{clip}%
\pgfsetbuttcap%
\pgfsetroundjoin%
\definecolor{currentfill}{rgb}{0.810616,0.268797,0.235428}%
\pgfsetfillcolor{currentfill}%
\pgfsetlinewidth{0.000000pt}%
\definecolor{currentstroke}{rgb}{0.000000,0.000000,0.000000}%
\pgfsetstrokecolor{currentstroke}%
\pgfsetdash{}{0pt}%
\pgfpathmoveto{\pgfqpoint{2.410872in}{2.508083in}}%
\pgfpathlineto{\pgfqpoint{2.674254in}{2.688576in}}%
\pgfpathlineto{\pgfqpoint{2.410872in}{2.508083in}}%
\pgfpathclose%
\pgfusepath{fill}%
\end{pgfscope}%
\begin{pgfscope}%
\pgfpathrectangle{\pgfqpoint{0.017500in}{0.017500in}}{\pgfqpoint{6.218282in}{3.580372in}} %
\pgfusepath{clip}%
\pgfsetbuttcap%
\pgfsetroundjoin%
\definecolor{currentfill}{rgb}{0.810616,0.268797,0.235428}%
\pgfsetfillcolor{currentfill}%
\pgfsetlinewidth{0.000000pt}%
\definecolor{currentstroke}{rgb}{0.000000,0.000000,0.000000}%
\pgfsetstrokecolor{currentstroke}%
\pgfsetdash{}{0pt}%
\pgfpathmoveto{\pgfqpoint{2.410872in}{2.508083in}}%
\pgfpathlineto{\pgfqpoint{2.674254in}{2.688576in}}%
\pgfpathlineto{\pgfqpoint{2.410872in}{2.508083in}}%
\pgfpathclose%
\pgfusepath{fill}%
\end{pgfscope}%
\begin{pgfscope}%
\pgfpathrectangle{\pgfqpoint{0.017500in}{0.017500in}}{\pgfqpoint{6.218282in}{3.580372in}} %
\pgfusepath{clip}%
\pgfsetbuttcap%
\pgfsetroundjoin%
\definecolor{currentfill}{rgb}{0.810616,0.268797,0.235428}%
\pgfsetfillcolor{currentfill}%
\pgfsetlinewidth{0.000000pt}%
\definecolor{currentstroke}{rgb}{0.000000,0.000000,0.000000}%
\pgfsetstrokecolor{currentstroke}%
\pgfsetdash{}{0pt}%
\pgfpathmoveto{\pgfqpoint{2.410872in}{2.508083in}}%
\pgfpathlineto{\pgfqpoint{2.674254in}{2.688576in}}%
\pgfpathlineto{\pgfqpoint{2.410872in}{2.508083in}}%
\pgfpathclose%
\pgfusepath{fill}%
\end{pgfscope}%
\begin{pgfscope}%
\pgfpathrectangle{\pgfqpoint{0.017500in}{0.017500in}}{\pgfqpoint{6.218282in}{3.580372in}} %
\pgfusepath{clip}%
\pgfsetbuttcap%
\pgfsetroundjoin%
\definecolor{currentfill}{rgb}{0.810616,0.268797,0.235428}%
\pgfsetfillcolor{currentfill}%
\pgfsetlinewidth{0.000000pt}%
\definecolor{currentstroke}{rgb}{0.000000,0.000000,0.000000}%
\pgfsetstrokecolor{currentstroke}%
\pgfsetdash{}{0pt}%
\pgfpathmoveto{\pgfqpoint{2.410872in}{2.508083in}}%
\pgfpathlineto{\pgfqpoint{2.674254in}{2.688576in}}%
\pgfpathlineto{\pgfqpoint{2.410872in}{2.508083in}}%
\pgfpathclose%
\pgfusepath{fill}%
\end{pgfscope}%
\begin{pgfscope}%
\pgfpathrectangle{\pgfqpoint{0.017500in}{0.017500in}}{\pgfqpoint{6.218282in}{3.580372in}} %
\pgfusepath{clip}%
\pgfsetbuttcap%
\pgfsetroundjoin%
\definecolor{currentfill}{rgb}{0.810616,0.268797,0.235428}%
\pgfsetfillcolor{currentfill}%
\pgfsetlinewidth{0.000000pt}%
\definecolor{currentstroke}{rgb}{0.000000,0.000000,0.000000}%
\pgfsetstrokecolor{currentstroke}%
\pgfsetdash{}{0pt}%
\pgfpathmoveto{\pgfqpoint{2.410872in}{2.508083in}}%
\pgfpathlineto{\pgfqpoint{2.674254in}{2.688576in}}%
\pgfpathlineto{\pgfqpoint{2.410872in}{2.508083in}}%
\pgfpathclose%
\pgfusepath{fill}%
\end{pgfscope}%
\begin{pgfscope}%
\pgfpathrectangle{\pgfqpoint{0.017500in}{0.017500in}}{\pgfqpoint{6.218282in}{3.580372in}} %
\pgfusepath{clip}%
\pgfsetbuttcap%
\pgfsetroundjoin%
\definecolor{currentfill}{rgb}{0.810616,0.268797,0.235428}%
\pgfsetfillcolor{currentfill}%
\pgfsetlinewidth{0.000000pt}%
\definecolor{currentstroke}{rgb}{0.000000,0.000000,0.000000}%
\pgfsetstrokecolor{currentstroke}%
\pgfsetdash{}{0pt}%
\pgfpathmoveto{\pgfqpoint{2.410872in}{2.508083in}}%
\pgfpathlineto{\pgfqpoint{2.674254in}{2.688576in}}%
\pgfpathlineto{\pgfqpoint{2.410872in}{2.508083in}}%
\pgfpathclose%
\pgfusepath{fill}%
\end{pgfscope}%
\begin{pgfscope}%
\pgfpathrectangle{\pgfqpoint{0.017500in}{0.017500in}}{\pgfqpoint{6.218282in}{3.580372in}} %
\pgfusepath{clip}%
\pgfsetbuttcap%
\pgfsetroundjoin%
\definecolor{currentfill}{rgb}{0.333490,0.446265,0.874452}%
\pgfsetfillcolor{currentfill}%
\pgfsetlinewidth{0.000000pt}%
\definecolor{currentstroke}{rgb}{0.000000,0.000000,0.000000}%
\pgfsetstrokecolor{currentstroke}%
\pgfsetdash{}{0pt}%
\pgfpathmoveto{\pgfqpoint{1.914830in}{0.967594in}}%
\pgfpathlineto{\pgfqpoint{2.349853in}{0.872712in}}%
\pgfpathlineto{\pgfqpoint{2.590432in}{2.210709in}}%
\pgfpathlineto{\pgfqpoint{2.143341in}{2.291288in}}%
\pgfpathlineto{\pgfqpoint{1.914830in}{0.967594in}}%
\pgfpathclose%
\pgfusepath{fill}%
\end{pgfscope}%
\begin{pgfscope}%
\pgfpathrectangle{\pgfqpoint{0.017500in}{0.017500in}}{\pgfqpoint{6.218282in}{3.580372in}} %
\pgfusepath{clip}%
\pgfsetbuttcap%
\pgfsetroundjoin%
\definecolor{currentfill}{rgb}{0.333490,0.446265,0.874452}%
\pgfsetfillcolor{currentfill}%
\pgfsetlinewidth{0.000000pt}%
\definecolor{currentstroke}{rgb}{0.000000,0.000000,0.000000}%
\pgfsetstrokecolor{currentstroke}%
\pgfsetdash{}{0pt}%
\pgfpathmoveto{\pgfqpoint{1.914830in}{0.967594in}}%
\pgfpathlineto{\pgfqpoint{2.349853in}{0.872712in}}%
\pgfpathlineto{\pgfqpoint{2.590432in}{2.210709in}}%
\pgfpathlineto{\pgfqpoint{2.143341in}{2.291288in}}%
\pgfpathlineto{\pgfqpoint{1.914830in}{0.967594in}}%
\pgfpathclose%
\pgfusepath{fill}%
\end{pgfscope}%
\begin{pgfscope}%
\pgfpathrectangle{\pgfqpoint{0.017500in}{0.017500in}}{\pgfqpoint{6.218282in}{3.580372in}} %
\pgfusepath{clip}%
\pgfsetbuttcap%
\pgfsetroundjoin%
\definecolor{currentfill}{rgb}{0.333490,0.446265,0.874452}%
\pgfsetfillcolor{currentfill}%
\pgfsetlinewidth{0.000000pt}%
\definecolor{currentstroke}{rgb}{0.000000,0.000000,0.000000}%
\pgfsetstrokecolor{currentstroke}%
\pgfsetdash{}{0pt}%
\pgfpathmoveto{\pgfqpoint{1.914830in}{0.967594in}}%
\pgfpathlineto{\pgfqpoint{2.349853in}{0.872712in}}%
\pgfpathlineto{\pgfqpoint{2.590432in}{2.210709in}}%
\pgfpathlineto{\pgfqpoint{2.143341in}{2.291288in}}%
\pgfpathlineto{\pgfqpoint{1.914830in}{0.967594in}}%
\pgfpathclose%
\pgfusepath{fill}%
\end{pgfscope}%
\begin{pgfscope}%
\pgfpathrectangle{\pgfqpoint{0.017500in}{0.017500in}}{\pgfqpoint{6.218282in}{3.580372in}} %
\pgfusepath{clip}%
\pgfsetbuttcap%
\pgfsetroundjoin%
\definecolor{currentfill}{rgb}{0.333490,0.446265,0.874452}%
\pgfsetfillcolor{currentfill}%
\pgfsetlinewidth{0.000000pt}%
\definecolor{currentstroke}{rgb}{0.000000,0.000000,0.000000}%
\pgfsetstrokecolor{currentstroke}%
\pgfsetdash{}{0pt}%
\pgfpathmoveto{\pgfqpoint{1.914830in}{0.967594in}}%
\pgfpathlineto{\pgfqpoint{2.349853in}{0.872712in}}%
\pgfpathlineto{\pgfqpoint{2.590432in}{2.210709in}}%
\pgfpathlineto{\pgfqpoint{2.143341in}{2.291288in}}%
\pgfpathlineto{\pgfqpoint{1.914830in}{0.967594in}}%
\pgfpathclose%
\pgfusepath{fill}%
\end{pgfscope}%
\begin{pgfscope}%
\pgfpathrectangle{\pgfqpoint{0.017500in}{0.017500in}}{\pgfqpoint{6.218282in}{3.580372in}} %
\pgfusepath{clip}%
\pgfsetbuttcap%
\pgfsetroundjoin%
\definecolor{currentfill}{rgb}{0.333490,0.446265,0.874452}%
\pgfsetfillcolor{currentfill}%
\pgfsetlinewidth{0.000000pt}%
\definecolor{currentstroke}{rgb}{0.000000,0.000000,0.000000}%
\pgfsetstrokecolor{currentstroke}%
\pgfsetdash{}{0pt}%
\pgfpathmoveto{\pgfqpoint{1.914830in}{0.967594in}}%
\pgfpathlineto{\pgfqpoint{2.349853in}{0.872712in}}%
\pgfpathlineto{\pgfqpoint{2.590432in}{2.210709in}}%
\pgfpathlineto{\pgfqpoint{2.143341in}{2.291288in}}%
\pgfpathlineto{\pgfqpoint{1.914830in}{0.967594in}}%
\pgfpathclose%
\pgfusepath{fill}%
\end{pgfscope}%
\begin{pgfscope}%
\pgfpathrectangle{\pgfqpoint{0.017500in}{0.017500in}}{\pgfqpoint{6.218282in}{3.580372in}} %
\pgfusepath{clip}%
\pgfsetbuttcap%
\pgfsetroundjoin%
\definecolor{currentfill}{rgb}{0.333490,0.446265,0.874452}%
\pgfsetfillcolor{currentfill}%
\pgfsetlinewidth{0.000000pt}%
\definecolor{currentstroke}{rgb}{0.000000,0.000000,0.000000}%
\pgfsetstrokecolor{currentstroke}%
\pgfsetdash{}{0pt}%
\pgfpathmoveto{\pgfqpoint{1.914830in}{0.967594in}}%
\pgfpathlineto{\pgfqpoint{2.349853in}{0.872712in}}%
\pgfpathlineto{\pgfqpoint{2.590432in}{2.210709in}}%
\pgfpathlineto{\pgfqpoint{2.143341in}{2.291288in}}%
\pgfpathlineto{\pgfqpoint{1.914830in}{0.967594in}}%
\pgfpathclose%
\pgfusepath{fill}%
\end{pgfscope}%
\begin{pgfscope}%
\pgfpathrectangle{\pgfqpoint{0.017500in}{0.017500in}}{\pgfqpoint{6.218282in}{3.580372in}} %
\pgfusepath{clip}%
\pgfsetbuttcap%
\pgfsetroundjoin%
\definecolor{currentfill}{rgb}{0.804965,0.851666,0.926165}%
\pgfsetfillcolor{currentfill}%
\pgfsetlinewidth{0.000000pt}%
\definecolor{currentstroke}{rgb}{0.000000,0.000000,0.000000}%
\pgfsetstrokecolor{currentstroke}%
\pgfsetdash{}{0pt}%
\pgfpathmoveto{\pgfqpoint{4.511410in}{2.753953in}}%
\pgfpathlineto{\pgfqpoint{4.964811in}{2.684961in}}%
\pgfpathlineto{\pgfqpoint{3.688351in}{0.641693in}}%
\pgfpathlineto{\pgfqpoint{3.236099in}{0.723497in}}%
\pgfpathlineto{\pgfqpoint{4.511410in}{2.753953in}}%
\pgfpathclose%
\pgfusepath{fill}%
\end{pgfscope}%
\begin{pgfscope}%
\pgfpathrectangle{\pgfqpoint{0.017500in}{0.017500in}}{\pgfqpoint{6.218282in}{3.580372in}} %
\pgfusepath{clip}%
\pgfsetbuttcap%
\pgfsetroundjoin%
\definecolor{currentfill}{rgb}{0.804965,0.851666,0.926165}%
\pgfsetfillcolor{currentfill}%
\pgfsetlinewidth{0.000000pt}%
\definecolor{currentstroke}{rgb}{0.000000,0.000000,0.000000}%
\pgfsetstrokecolor{currentstroke}%
\pgfsetdash{}{0pt}%
\pgfpathmoveto{\pgfqpoint{4.511410in}{2.753953in}}%
\pgfpathlineto{\pgfqpoint{4.964811in}{2.684961in}}%
\pgfpathlineto{\pgfqpoint{3.688351in}{0.641693in}}%
\pgfpathlineto{\pgfqpoint{3.236099in}{0.723497in}}%
\pgfpathlineto{\pgfqpoint{4.511410in}{2.753953in}}%
\pgfpathclose%
\pgfusepath{fill}%
\end{pgfscope}%
\begin{pgfscope}%
\pgfpathrectangle{\pgfqpoint{0.017500in}{0.017500in}}{\pgfqpoint{6.218282in}{3.580372in}} %
\pgfusepath{clip}%
\pgfsetbuttcap%
\pgfsetroundjoin%
\definecolor{currentfill}{rgb}{0.804965,0.851666,0.926165}%
\pgfsetfillcolor{currentfill}%
\pgfsetlinewidth{0.000000pt}%
\definecolor{currentstroke}{rgb}{0.000000,0.000000,0.000000}%
\pgfsetstrokecolor{currentstroke}%
\pgfsetdash{}{0pt}%
\pgfpathmoveto{\pgfqpoint{4.511410in}{2.753953in}}%
\pgfpathlineto{\pgfqpoint{4.964811in}{2.684961in}}%
\pgfpathlineto{\pgfqpoint{3.688351in}{0.641693in}}%
\pgfpathlineto{\pgfqpoint{3.236099in}{0.723497in}}%
\pgfpathlineto{\pgfqpoint{4.511410in}{2.753953in}}%
\pgfpathclose%
\pgfusepath{fill}%
\end{pgfscope}%
\begin{pgfscope}%
\pgfpathrectangle{\pgfqpoint{0.017500in}{0.017500in}}{\pgfqpoint{6.218282in}{3.580372in}} %
\pgfusepath{clip}%
\pgfsetbuttcap%
\pgfsetroundjoin%
\definecolor{currentfill}{rgb}{0.804965,0.851666,0.926165}%
\pgfsetfillcolor{currentfill}%
\pgfsetlinewidth{0.000000pt}%
\definecolor{currentstroke}{rgb}{0.000000,0.000000,0.000000}%
\pgfsetstrokecolor{currentstroke}%
\pgfsetdash{}{0pt}%
\pgfpathmoveto{\pgfqpoint{4.511410in}{2.753953in}}%
\pgfpathlineto{\pgfqpoint{4.964811in}{2.684961in}}%
\pgfpathlineto{\pgfqpoint{3.688351in}{0.641693in}}%
\pgfpathlineto{\pgfqpoint{3.236099in}{0.723497in}}%
\pgfpathlineto{\pgfqpoint{4.511410in}{2.753953in}}%
\pgfpathclose%
\pgfusepath{fill}%
\end{pgfscope}%
\begin{pgfscope}%
\pgfpathrectangle{\pgfqpoint{0.017500in}{0.017500in}}{\pgfqpoint{6.218282in}{3.580372in}} %
\pgfusepath{clip}%
\pgfsetbuttcap%
\pgfsetroundjoin%
\definecolor{currentfill}{rgb}{0.804965,0.851666,0.926165}%
\pgfsetfillcolor{currentfill}%
\pgfsetlinewidth{0.000000pt}%
\definecolor{currentstroke}{rgb}{0.000000,0.000000,0.000000}%
\pgfsetstrokecolor{currentstroke}%
\pgfsetdash{}{0pt}%
\pgfpathmoveto{\pgfqpoint{4.511410in}{2.753953in}}%
\pgfpathlineto{\pgfqpoint{4.964811in}{2.684961in}}%
\pgfpathlineto{\pgfqpoint{3.688351in}{0.641693in}}%
\pgfpathlineto{\pgfqpoint{3.236099in}{0.723497in}}%
\pgfpathlineto{\pgfqpoint{4.511410in}{2.753953in}}%
\pgfpathclose%
\pgfusepath{fill}%
\end{pgfscope}%
\begin{pgfscope}%
\pgfpathrectangle{\pgfqpoint{0.017500in}{0.017500in}}{\pgfqpoint{6.218282in}{3.580372in}} %
\pgfusepath{clip}%
\pgfsetbuttcap%
\pgfsetroundjoin%
\definecolor{currentfill}{rgb}{0.723315,0.068898,0.162989}%
\pgfsetfillcolor{currentfill}%
\pgfsetlinewidth{0.000000pt}%
\definecolor{currentstroke}{rgb}{0.000000,0.000000,0.000000}%
\pgfsetstrokecolor{currentstroke}%
\pgfsetdash{}{0pt}%
\pgfpathmoveto{\pgfqpoint{4.267628in}{2.613520in}}%
\pgfpathlineto{\pgfqpoint{4.724651in}{2.549289in}}%
\pgfpathlineto{\pgfqpoint{4.964811in}{2.684961in}}%
\pgfpathlineto{\pgfqpoint{4.511410in}{2.753953in}}%
\pgfpathlineto{\pgfqpoint{4.267628in}{2.613520in}}%
\pgfpathclose%
\pgfusepath{fill}%
\end{pgfscope}%
\begin{pgfscope}%
\pgfpathrectangle{\pgfqpoint{0.017500in}{0.017500in}}{\pgfqpoint{6.218282in}{3.580372in}} %
\pgfusepath{clip}%
\pgfsetbuttcap%
\pgfsetroundjoin%
\definecolor{currentfill}{rgb}{0.723315,0.068898,0.162989}%
\pgfsetfillcolor{currentfill}%
\pgfsetlinewidth{0.000000pt}%
\definecolor{currentstroke}{rgb}{0.000000,0.000000,0.000000}%
\pgfsetstrokecolor{currentstroke}%
\pgfsetdash{}{0pt}%
\pgfpathmoveto{\pgfqpoint{4.267628in}{2.613520in}}%
\pgfpathlineto{\pgfqpoint{4.724651in}{2.549289in}}%
\pgfpathlineto{\pgfqpoint{4.964811in}{2.684961in}}%
\pgfpathlineto{\pgfqpoint{4.511410in}{2.753953in}}%
\pgfpathlineto{\pgfqpoint{4.267628in}{2.613520in}}%
\pgfpathclose%
\pgfusepath{fill}%
\end{pgfscope}%
\begin{pgfscope}%
\pgfpathrectangle{\pgfqpoint{0.017500in}{0.017500in}}{\pgfqpoint{6.218282in}{3.580372in}} %
\pgfusepath{clip}%
\pgfsetbuttcap%
\pgfsetroundjoin%
\definecolor{currentfill}{rgb}{0.723315,0.068898,0.162989}%
\pgfsetfillcolor{currentfill}%
\pgfsetlinewidth{0.000000pt}%
\definecolor{currentstroke}{rgb}{0.000000,0.000000,0.000000}%
\pgfsetstrokecolor{currentstroke}%
\pgfsetdash{}{0pt}%
\pgfpathmoveto{\pgfqpoint{4.267628in}{2.613520in}}%
\pgfpathlineto{\pgfqpoint{4.724651in}{2.549289in}}%
\pgfpathlineto{\pgfqpoint{4.964811in}{2.684961in}}%
\pgfpathlineto{\pgfqpoint{4.511410in}{2.753953in}}%
\pgfpathlineto{\pgfqpoint{4.267628in}{2.613520in}}%
\pgfpathclose%
\pgfusepath{fill}%
\end{pgfscope}%
\begin{pgfscope}%
\pgfpathrectangle{\pgfqpoint{0.017500in}{0.017500in}}{\pgfqpoint{6.218282in}{3.580372in}} %
\pgfusepath{clip}%
\pgfsetbuttcap%
\pgfsetroundjoin%
\definecolor{currentfill}{rgb}{0.723315,0.068898,0.162989}%
\pgfsetfillcolor{currentfill}%
\pgfsetlinewidth{0.000000pt}%
\definecolor{currentstroke}{rgb}{0.000000,0.000000,0.000000}%
\pgfsetstrokecolor{currentstroke}%
\pgfsetdash{}{0pt}%
\pgfpathmoveto{\pgfqpoint{4.267628in}{2.613520in}}%
\pgfpathlineto{\pgfqpoint{4.724651in}{2.549289in}}%
\pgfpathlineto{\pgfqpoint{4.964811in}{2.684961in}}%
\pgfpathlineto{\pgfqpoint{4.511410in}{2.753953in}}%
\pgfpathlineto{\pgfqpoint{4.267628in}{2.613520in}}%
\pgfpathclose%
\pgfusepath{fill}%
\end{pgfscope}%
\begin{pgfscope}%
\pgfpathrectangle{\pgfqpoint{0.017500in}{0.017500in}}{\pgfqpoint{6.218282in}{3.580372in}} %
\pgfusepath{clip}%
\pgfsetbuttcap%
\pgfsetroundjoin%
\definecolor{currentfill}{rgb}{0.723315,0.068898,0.162989}%
\pgfsetfillcolor{currentfill}%
\pgfsetlinewidth{0.000000pt}%
\definecolor{currentstroke}{rgb}{0.000000,0.000000,0.000000}%
\pgfsetstrokecolor{currentstroke}%
\pgfsetdash{}{0pt}%
\pgfpathmoveto{\pgfqpoint{4.267628in}{2.613520in}}%
\pgfpathlineto{\pgfqpoint{4.724651in}{2.549289in}}%
\pgfpathlineto{\pgfqpoint{4.964811in}{2.684961in}}%
\pgfpathlineto{\pgfqpoint{4.511410in}{2.753953in}}%
\pgfpathlineto{\pgfqpoint{4.267628in}{2.613520in}}%
\pgfpathclose%
\pgfusepath{fill}%
\end{pgfscope}%
\begin{pgfscope}%
\pgfpathrectangle{\pgfqpoint{0.017500in}{0.017500in}}{\pgfqpoint{6.218282in}{3.580372in}} %
\pgfusepath{clip}%
\pgfsetbuttcap%
\pgfsetroundjoin%
\definecolor{currentfill}{rgb}{0.723315,0.068898,0.162989}%
\pgfsetfillcolor{currentfill}%
\pgfsetlinewidth{0.000000pt}%
\definecolor{currentstroke}{rgb}{0.000000,0.000000,0.000000}%
\pgfsetstrokecolor{currentstroke}%
\pgfsetdash{}{0pt}%
\pgfpathmoveto{\pgfqpoint{4.267628in}{2.613520in}}%
\pgfpathlineto{\pgfqpoint{4.724651in}{2.549289in}}%
\pgfpathlineto{\pgfqpoint{4.964811in}{2.684961in}}%
\pgfpathlineto{\pgfqpoint{4.511410in}{2.753953in}}%
\pgfpathlineto{\pgfqpoint{4.267628in}{2.613520in}}%
\pgfpathclose%
\pgfusepath{fill}%
\end{pgfscope}%
\begin{pgfscope}%
\pgfpathrectangle{\pgfqpoint{0.017500in}{0.017500in}}{\pgfqpoint{6.218282in}{3.580372in}} %
\pgfusepath{clip}%
\pgfsetbuttcap%
\pgfsetroundjoin%
\definecolor{currentfill}{rgb}{0.805723,0.259813,0.230562}%
\pgfsetfillcolor{currentfill}%
\pgfsetlinewidth{0.000000pt}%
\definecolor{currentstroke}{rgb}{0.000000,0.000000,0.000000}%
\pgfsetstrokecolor{currentstroke}%
\pgfsetdash{}{0pt}%
\pgfpathmoveto{\pgfqpoint{2.410872in}{2.508083in}}%
\pgfpathlineto{\pgfqpoint{2.855513in}{2.451418in}}%
\pgfpathlineto{\pgfqpoint{3.116513in}{2.603812in}}%
\pgfpathlineto{\pgfqpoint{2.674254in}{2.688576in}}%
\pgfpathlineto{\pgfqpoint{2.410872in}{2.508083in}}%
\pgfpathclose%
\pgfusepath{fill}%
\end{pgfscope}%
\begin{pgfscope}%
\pgfpathrectangle{\pgfqpoint{0.017500in}{0.017500in}}{\pgfqpoint{6.218282in}{3.580372in}} %
\pgfusepath{clip}%
\pgfsetbuttcap%
\pgfsetroundjoin%
\definecolor{currentfill}{rgb}{0.805723,0.259813,0.230562}%
\pgfsetfillcolor{currentfill}%
\pgfsetlinewidth{0.000000pt}%
\definecolor{currentstroke}{rgb}{0.000000,0.000000,0.000000}%
\pgfsetstrokecolor{currentstroke}%
\pgfsetdash{}{0pt}%
\pgfpathmoveto{\pgfqpoint{2.410872in}{2.508083in}}%
\pgfpathlineto{\pgfqpoint{2.855513in}{2.451418in}}%
\pgfpathlineto{\pgfqpoint{3.116513in}{2.603812in}}%
\pgfpathlineto{\pgfqpoint{2.674254in}{2.688576in}}%
\pgfpathlineto{\pgfqpoint{2.410872in}{2.508083in}}%
\pgfpathclose%
\pgfusepath{fill}%
\end{pgfscope}%
\begin{pgfscope}%
\pgfpathrectangle{\pgfqpoint{0.017500in}{0.017500in}}{\pgfqpoint{6.218282in}{3.580372in}} %
\pgfusepath{clip}%
\pgfsetbuttcap%
\pgfsetroundjoin%
\definecolor{currentfill}{rgb}{0.805723,0.259813,0.230562}%
\pgfsetfillcolor{currentfill}%
\pgfsetlinewidth{0.000000pt}%
\definecolor{currentstroke}{rgb}{0.000000,0.000000,0.000000}%
\pgfsetstrokecolor{currentstroke}%
\pgfsetdash{}{0pt}%
\pgfpathmoveto{\pgfqpoint{2.410872in}{2.508083in}}%
\pgfpathlineto{\pgfqpoint{2.855513in}{2.451418in}}%
\pgfpathlineto{\pgfqpoint{3.116513in}{2.603812in}}%
\pgfpathlineto{\pgfqpoint{2.674254in}{2.688576in}}%
\pgfpathlineto{\pgfqpoint{2.410872in}{2.508083in}}%
\pgfpathclose%
\pgfusepath{fill}%
\end{pgfscope}%
\begin{pgfscope}%
\pgfpathrectangle{\pgfqpoint{0.017500in}{0.017500in}}{\pgfqpoint{6.218282in}{3.580372in}} %
\pgfusepath{clip}%
\pgfsetbuttcap%
\pgfsetroundjoin%
\definecolor{currentfill}{rgb}{0.805723,0.259813,0.230562}%
\pgfsetfillcolor{currentfill}%
\pgfsetlinewidth{0.000000pt}%
\definecolor{currentstroke}{rgb}{0.000000,0.000000,0.000000}%
\pgfsetstrokecolor{currentstroke}%
\pgfsetdash{}{0pt}%
\pgfpathmoveto{\pgfqpoint{2.410872in}{2.508083in}}%
\pgfpathlineto{\pgfqpoint{2.855513in}{2.451418in}}%
\pgfpathlineto{\pgfqpoint{3.116513in}{2.603812in}}%
\pgfpathlineto{\pgfqpoint{2.674254in}{2.688576in}}%
\pgfpathlineto{\pgfqpoint{2.410872in}{2.508083in}}%
\pgfpathclose%
\pgfusepath{fill}%
\end{pgfscope}%
\begin{pgfscope}%
\pgfpathrectangle{\pgfqpoint{0.017500in}{0.017500in}}{\pgfqpoint{6.218282in}{3.580372in}} %
\pgfusepath{clip}%
\pgfsetbuttcap%
\pgfsetroundjoin%
\definecolor{currentfill}{rgb}{0.805723,0.259813,0.230562}%
\pgfsetfillcolor{currentfill}%
\pgfsetlinewidth{0.000000pt}%
\definecolor{currentstroke}{rgb}{0.000000,0.000000,0.000000}%
\pgfsetstrokecolor{currentstroke}%
\pgfsetdash{}{0pt}%
\pgfpathmoveto{\pgfqpoint{2.410872in}{2.508083in}}%
\pgfpathlineto{\pgfqpoint{2.855513in}{2.451418in}}%
\pgfpathlineto{\pgfqpoint{3.116513in}{2.603812in}}%
\pgfpathlineto{\pgfqpoint{2.674254in}{2.688576in}}%
\pgfpathlineto{\pgfqpoint{2.410872in}{2.508083in}}%
\pgfpathclose%
\pgfusepath{fill}%
\end{pgfscope}%
\begin{pgfscope}%
\pgfpathrectangle{\pgfqpoint{0.017500in}{0.017500in}}{\pgfqpoint{6.218282in}{3.580372in}} %
\pgfusepath{clip}%
\pgfsetbuttcap%
\pgfsetroundjoin%
\definecolor{currentfill}{rgb}{0.805723,0.259813,0.230562}%
\pgfsetfillcolor{currentfill}%
\pgfsetlinewidth{0.000000pt}%
\definecolor{currentstroke}{rgb}{0.000000,0.000000,0.000000}%
\pgfsetstrokecolor{currentstroke}%
\pgfsetdash{}{0pt}%
\pgfpathmoveto{\pgfqpoint{2.410872in}{2.508083in}}%
\pgfpathlineto{\pgfqpoint{2.855513in}{2.451418in}}%
\pgfpathlineto{\pgfqpoint{3.116513in}{2.603812in}}%
\pgfpathlineto{\pgfqpoint{2.674254in}{2.688576in}}%
\pgfpathlineto{\pgfqpoint{2.410872in}{2.508083in}}%
\pgfpathclose%
\pgfusepath{fill}%
\end{pgfscope}%
\begin{pgfscope}%
\pgfpathrectangle{\pgfqpoint{0.017500in}{0.017500in}}{\pgfqpoint{6.218282in}{3.580372in}} %
\pgfusepath{clip}%
\pgfsetbuttcap%
\pgfsetroundjoin%
\definecolor{currentfill}{rgb}{0.229806,0.298718,0.753683}%
\pgfsetfillcolor{currentfill}%
\pgfsetlinewidth{0.000000pt}%
\definecolor{currentstroke}{rgb}{0.000000,0.000000,0.000000}%
\pgfsetstrokecolor{currentstroke}%
\pgfsetdash{}{0pt}%
\pgfpathmoveto{\pgfqpoint{2.349853in}{0.872712in}}%
\pgfpathlineto{\pgfqpoint{2.590432in}{2.210709in}}%
\pgfpathlineto{\pgfqpoint{2.349853in}{0.872712in}}%
\pgfpathclose%
\pgfusepath{fill}%
\end{pgfscope}%
\begin{pgfscope}%
\pgfpathrectangle{\pgfqpoint{0.017500in}{0.017500in}}{\pgfqpoint{6.218282in}{3.580372in}} %
\pgfusepath{clip}%
\pgfsetbuttcap%
\pgfsetroundjoin%
\definecolor{currentfill}{rgb}{0.229806,0.298718,0.753683}%
\pgfsetfillcolor{currentfill}%
\pgfsetlinewidth{0.000000pt}%
\definecolor{currentstroke}{rgb}{0.000000,0.000000,0.000000}%
\pgfsetstrokecolor{currentstroke}%
\pgfsetdash{}{0pt}%
\pgfpathmoveto{\pgfqpoint{2.349853in}{0.872712in}}%
\pgfpathlineto{\pgfqpoint{2.590432in}{2.210709in}}%
\pgfpathlineto{\pgfqpoint{2.349853in}{0.872712in}}%
\pgfpathclose%
\pgfusepath{fill}%
\end{pgfscope}%
\begin{pgfscope}%
\pgfpathrectangle{\pgfqpoint{0.017500in}{0.017500in}}{\pgfqpoint{6.218282in}{3.580372in}} %
\pgfusepath{clip}%
\pgfsetbuttcap%
\pgfsetroundjoin%
\definecolor{currentfill}{rgb}{0.229806,0.298718,0.753683}%
\pgfsetfillcolor{currentfill}%
\pgfsetlinewidth{0.000000pt}%
\definecolor{currentstroke}{rgb}{0.000000,0.000000,0.000000}%
\pgfsetstrokecolor{currentstroke}%
\pgfsetdash{}{0pt}%
\pgfpathmoveto{\pgfqpoint{2.349853in}{0.872712in}}%
\pgfpathlineto{\pgfqpoint{2.590432in}{2.210709in}}%
\pgfpathlineto{\pgfqpoint{2.349853in}{0.872712in}}%
\pgfpathclose%
\pgfusepath{fill}%
\end{pgfscope}%
\begin{pgfscope}%
\pgfpathrectangle{\pgfqpoint{0.017500in}{0.017500in}}{\pgfqpoint{6.218282in}{3.580372in}} %
\pgfusepath{clip}%
\pgfsetbuttcap%
\pgfsetroundjoin%
\definecolor{currentfill}{rgb}{0.229806,0.298718,0.753683}%
\pgfsetfillcolor{currentfill}%
\pgfsetlinewidth{0.000000pt}%
\definecolor{currentstroke}{rgb}{0.000000,0.000000,0.000000}%
\pgfsetstrokecolor{currentstroke}%
\pgfsetdash{}{0pt}%
\pgfpathmoveto{\pgfqpoint{2.349853in}{0.872712in}}%
\pgfpathlineto{\pgfqpoint{2.590432in}{2.210709in}}%
\pgfpathlineto{\pgfqpoint{2.349853in}{0.872712in}}%
\pgfpathclose%
\pgfusepath{fill}%
\end{pgfscope}%
\begin{pgfscope}%
\pgfpathrectangle{\pgfqpoint{0.017500in}{0.017500in}}{\pgfqpoint{6.218282in}{3.580372in}} %
\pgfusepath{clip}%
\pgfsetbuttcap%
\pgfsetroundjoin%
\definecolor{currentfill}{rgb}{0.229806,0.298718,0.753683}%
\pgfsetfillcolor{currentfill}%
\pgfsetlinewidth{0.000000pt}%
\definecolor{currentstroke}{rgb}{0.000000,0.000000,0.000000}%
\pgfsetstrokecolor{currentstroke}%
\pgfsetdash{}{0pt}%
\pgfpathmoveto{\pgfqpoint{2.349853in}{0.872712in}}%
\pgfpathlineto{\pgfqpoint{2.590432in}{2.210709in}}%
\pgfpathlineto{\pgfqpoint{2.349853in}{0.872712in}}%
\pgfpathclose%
\pgfusepath{fill}%
\end{pgfscope}%
\begin{pgfscope}%
\pgfpathrectangle{\pgfqpoint{0.017500in}{0.017500in}}{\pgfqpoint{6.218282in}{3.580372in}} %
\pgfusepath{clip}%
\pgfsetbuttcap%
\pgfsetroundjoin%
\definecolor{currentfill}{rgb}{0.229806,0.298718,0.753683}%
\pgfsetfillcolor{currentfill}%
\pgfsetlinewidth{0.000000pt}%
\definecolor{currentstroke}{rgb}{0.000000,0.000000,0.000000}%
\pgfsetstrokecolor{currentstroke}%
\pgfsetdash{}{0pt}%
\pgfpathmoveto{\pgfqpoint{2.349853in}{0.872712in}}%
\pgfpathlineto{\pgfqpoint{2.590432in}{2.210709in}}%
\pgfpathlineto{\pgfqpoint{2.349853in}{0.872712in}}%
\pgfpathclose%
\pgfusepath{fill}%
\end{pgfscope}%
\begin{pgfscope}%
\pgfpathrectangle{\pgfqpoint{0.017500in}{0.017500in}}{\pgfqpoint{6.218282in}{3.580372in}} %
\pgfusepath{clip}%
\pgfsetbuttcap%
\pgfsetroundjoin%
\definecolor{currentfill}{rgb}{0.229806,0.298718,0.753683}%
\pgfsetfillcolor{currentfill}%
\pgfsetlinewidth{0.000000pt}%
\definecolor{currentstroke}{rgb}{0.000000,0.000000,0.000000}%
\pgfsetstrokecolor{currentstroke}%
\pgfsetdash{}{0pt}%
\pgfpathmoveto{\pgfqpoint{2.349853in}{0.872712in}}%
\pgfpathlineto{\pgfqpoint{2.590432in}{2.210709in}}%
\pgfpathlineto{\pgfqpoint{2.349853in}{0.872712in}}%
\pgfpathclose%
\pgfusepath{fill}%
\end{pgfscope}%
\begin{pgfscope}%
\pgfpathrectangle{\pgfqpoint{0.017500in}{0.017500in}}{\pgfqpoint{6.218282in}{3.580372in}} %
\pgfusepath{clip}%
\pgfsetbuttcap%
\pgfsetroundjoin%
\definecolor{currentfill}{rgb}{0.229806,0.298718,0.753683}%
\pgfsetfillcolor{currentfill}%
\pgfsetlinewidth{0.000000pt}%
\definecolor{currentstroke}{rgb}{0.000000,0.000000,0.000000}%
\pgfsetstrokecolor{currentstroke}%
\pgfsetdash{}{0pt}%
\pgfpathmoveto{\pgfqpoint{2.349853in}{0.872712in}}%
\pgfpathlineto{\pgfqpoint{2.590432in}{2.210709in}}%
\pgfpathlineto{\pgfqpoint{2.349853in}{0.872712in}}%
\pgfpathclose%
\pgfusepath{fill}%
\end{pgfscope}%
\begin{pgfscope}%
\pgfpathrectangle{\pgfqpoint{0.017500in}{0.017500in}}{\pgfqpoint{6.218282in}{3.580372in}} %
\pgfusepath{clip}%
\pgfsetbuttcap%
\pgfsetroundjoin%
\definecolor{currentfill}{rgb}{0.229806,0.298718,0.753683}%
\pgfsetfillcolor{currentfill}%
\pgfsetlinewidth{0.000000pt}%
\definecolor{currentstroke}{rgb}{0.000000,0.000000,0.000000}%
\pgfsetstrokecolor{currentstroke}%
\pgfsetdash{}{0pt}%
\pgfpathmoveto{\pgfqpoint{2.349853in}{0.872712in}}%
\pgfpathlineto{\pgfqpoint{2.590432in}{2.210709in}}%
\pgfpathlineto{\pgfqpoint{2.349853in}{0.872712in}}%
\pgfpathclose%
\pgfusepath{fill}%
\end{pgfscope}%
\begin{pgfscope}%
\pgfpathrectangle{\pgfqpoint{0.017500in}{0.017500in}}{\pgfqpoint{6.218282in}{3.580372in}} %
\pgfusepath{clip}%
\pgfsetbuttcap%
\pgfsetroundjoin%
\definecolor{currentfill}{rgb}{0.229806,0.298718,0.753683}%
\pgfsetfillcolor{currentfill}%
\pgfsetlinewidth{0.000000pt}%
\definecolor{currentstroke}{rgb}{0.000000,0.000000,0.000000}%
\pgfsetstrokecolor{currentstroke}%
\pgfsetdash{}{0pt}%
\pgfpathmoveto{\pgfqpoint{2.349853in}{0.872712in}}%
\pgfpathlineto{\pgfqpoint{2.590432in}{2.210709in}}%
\pgfpathlineto{\pgfqpoint{2.349853in}{0.872712in}}%
\pgfpathclose%
\pgfusepath{fill}%
\end{pgfscope}%
\begin{pgfscope}%
\pgfpathrectangle{\pgfqpoint{0.017500in}{0.017500in}}{\pgfqpoint{6.218282in}{3.580372in}} %
\pgfusepath{clip}%
\pgfsetbuttcap%
\pgfsetroundjoin%
\definecolor{currentfill}{rgb}{0.229806,0.298718,0.753683}%
\pgfsetfillcolor{currentfill}%
\pgfsetlinewidth{0.000000pt}%
\definecolor{currentstroke}{rgb}{0.000000,0.000000,0.000000}%
\pgfsetstrokecolor{currentstroke}%
\pgfsetdash{}{0pt}%
\pgfpathmoveto{\pgfqpoint{2.349853in}{0.872712in}}%
\pgfpathlineto{\pgfqpoint{2.590432in}{2.210709in}}%
\pgfpathlineto{\pgfqpoint{2.349853in}{0.872712in}}%
\pgfpathclose%
\pgfusepath{fill}%
\end{pgfscope}%
\begin{pgfscope}%
\pgfpathrectangle{\pgfqpoint{0.017500in}{0.017500in}}{\pgfqpoint{6.218282in}{3.580372in}} %
\pgfusepath{clip}%
\pgfsetbuttcap%
\pgfsetroundjoin%
\definecolor{currentfill}{rgb}{0.229806,0.298718,0.753683}%
\pgfsetfillcolor{currentfill}%
\pgfsetlinewidth{0.000000pt}%
\definecolor{currentstroke}{rgb}{0.000000,0.000000,0.000000}%
\pgfsetstrokecolor{currentstroke}%
\pgfsetdash{}{0pt}%
\pgfpathmoveto{\pgfqpoint{2.349853in}{0.872712in}}%
\pgfpathlineto{\pgfqpoint{2.590432in}{2.210709in}}%
\pgfpathlineto{\pgfqpoint{2.349853in}{0.872712in}}%
\pgfpathclose%
\pgfusepath{fill}%
\end{pgfscope}%
\begin{pgfscope}%
\pgfpathrectangle{\pgfqpoint{0.017500in}{0.017500in}}{\pgfqpoint{6.218282in}{3.580372in}} %
\pgfusepath{clip}%
\pgfsetbuttcap%
\pgfsetroundjoin%
\definecolor{currentfill}{rgb}{0.229806,0.298718,0.753683}%
\pgfsetfillcolor{currentfill}%
\pgfsetlinewidth{0.000000pt}%
\definecolor{currentstroke}{rgb}{0.000000,0.000000,0.000000}%
\pgfsetstrokecolor{currentstroke}%
\pgfsetdash{}{0pt}%
\pgfpathmoveto{\pgfqpoint{2.349853in}{0.872712in}}%
\pgfpathlineto{\pgfqpoint{2.590432in}{2.210709in}}%
\pgfpathlineto{\pgfqpoint{2.349853in}{0.872712in}}%
\pgfpathclose%
\pgfusepath{fill}%
\end{pgfscope}%
\begin{pgfscope}%
\pgfpathrectangle{\pgfqpoint{0.017500in}{0.017500in}}{\pgfqpoint{6.218282in}{3.580372in}} %
\pgfusepath{clip}%
\pgfsetbuttcap%
\pgfsetroundjoin%
\definecolor{currentfill}{rgb}{0.229806,0.298718,0.753683}%
\pgfsetfillcolor{currentfill}%
\pgfsetlinewidth{0.000000pt}%
\definecolor{currentstroke}{rgb}{0.000000,0.000000,0.000000}%
\pgfsetstrokecolor{currentstroke}%
\pgfsetdash{}{0pt}%
\pgfpathmoveto{\pgfqpoint{2.349853in}{0.872712in}}%
\pgfpathlineto{\pgfqpoint{2.590432in}{2.210709in}}%
\pgfpathlineto{\pgfqpoint{2.349853in}{0.872712in}}%
\pgfpathclose%
\pgfusepath{fill}%
\end{pgfscope}%
\begin{pgfscope}%
\pgfpathrectangle{\pgfqpoint{0.017500in}{0.017500in}}{\pgfqpoint{6.218282in}{3.580372in}} %
\pgfusepath{clip}%
\pgfsetbuttcap%
\pgfsetroundjoin%
\definecolor{currentfill}{rgb}{0.229806,0.298718,0.753683}%
\pgfsetfillcolor{currentfill}%
\pgfsetlinewidth{0.000000pt}%
\definecolor{currentstroke}{rgb}{0.000000,0.000000,0.000000}%
\pgfsetstrokecolor{currentstroke}%
\pgfsetdash{}{0pt}%
\pgfpathmoveto{\pgfqpoint{2.349853in}{0.872712in}}%
\pgfpathlineto{\pgfqpoint{2.590432in}{2.210709in}}%
\pgfpathlineto{\pgfqpoint{2.349853in}{0.872712in}}%
\pgfpathclose%
\pgfusepath{fill}%
\end{pgfscope}%
\begin{pgfscope}%
\pgfpathrectangle{\pgfqpoint{0.017500in}{0.017500in}}{\pgfqpoint{6.218282in}{3.580372in}} %
\pgfusepath{clip}%
\pgfsetbuttcap%
\pgfsetroundjoin%
\definecolor{currentfill}{rgb}{0.229806,0.298718,0.753683}%
\pgfsetfillcolor{currentfill}%
\pgfsetlinewidth{0.000000pt}%
\definecolor{currentstroke}{rgb}{0.000000,0.000000,0.000000}%
\pgfsetstrokecolor{currentstroke}%
\pgfsetdash{}{0pt}%
\pgfpathmoveto{\pgfqpoint{2.349853in}{0.872712in}}%
\pgfpathlineto{\pgfqpoint{2.590432in}{2.210709in}}%
\pgfpathlineto{\pgfqpoint{2.349853in}{0.872712in}}%
\pgfpathclose%
\pgfusepath{fill}%
\end{pgfscope}%
\begin{pgfscope}%
\pgfpathrectangle{\pgfqpoint{0.017500in}{0.017500in}}{\pgfqpoint{6.218282in}{3.580372in}} %
\pgfusepath{clip}%
\pgfsetbuttcap%
\pgfsetroundjoin%
\definecolor{currentfill}{rgb}{0.229806,0.298718,0.753683}%
\pgfsetfillcolor{currentfill}%
\pgfsetlinewidth{0.000000pt}%
\definecolor{currentstroke}{rgb}{0.000000,0.000000,0.000000}%
\pgfsetstrokecolor{currentstroke}%
\pgfsetdash{}{0pt}%
\pgfpathmoveto{\pgfqpoint{2.349853in}{0.872712in}}%
\pgfpathlineto{\pgfqpoint{2.590432in}{2.210709in}}%
\pgfpathlineto{\pgfqpoint{2.349853in}{0.872712in}}%
\pgfpathclose%
\pgfusepath{fill}%
\end{pgfscope}%
\begin{pgfscope}%
\pgfpathrectangle{\pgfqpoint{0.017500in}{0.017500in}}{\pgfqpoint{6.218282in}{3.580372in}} %
\pgfusepath{clip}%
\pgfsetbuttcap%
\pgfsetroundjoin%
\definecolor{currentfill}{rgb}{0.229806,0.298718,0.753683}%
\pgfsetfillcolor{currentfill}%
\pgfsetlinewidth{0.000000pt}%
\definecolor{currentstroke}{rgb}{0.000000,0.000000,0.000000}%
\pgfsetstrokecolor{currentstroke}%
\pgfsetdash{}{0pt}%
\pgfpathmoveto{\pgfqpoint{2.349853in}{0.872712in}}%
\pgfpathlineto{\pgfqpoint{2.590432in}{2.210709in}}%
\pgfpathlineto{\pgfqpoint{2.349853in}{0.872712in}}%
\pgfpathclose%
\pgfusepath{fill}%
\end{pgfscope}%
\begin{pgfscope}%
\pgfpathrectangle{\pgfqpoint{0.017500in}{0.017500in}}{\pgfqpoint{6.218282in}{3.580372in}} %
\pgfusepath{clip}%
\pgfsetbuttcap%
\pgfsetroundjoin%
\definecolor{currentfill}{rgb}{0.229806,0.298718,0.753683}%
\pgfsetfillcolor{currentfill}%
\pgfsetlinewidth{0.000000pt}%
\definecolor{currentstroke}{rgb}{0.000000,0.000000,0.000000}%
\pgfsetstrokecolor{currentstroke}%
\pgfsetdash{}{0pt}%
\pgfpathmoveto{\pgfqpoint{2.349853in}{0.872712in}}%
\pgfpathlineto{\pgfqpoint{2.590432in}{2.210709in}}%
\pgfpathlineto{\pgfqpoint{2.349853in}{0.872712in}}%
\pgfpathclose%
\pgfusepath{fill}%
\end{pgfscope}%
\begin{pgfscope}%
\pgfpathrectangle{\pgfqpoint{0.017500in}{0.017500in}}{\pgfqpoint{6.218282in}{3.580372in}} %
\pgfusepath{clip}%
\pgfsetbuttcap%
\pgfsetroundjoin%
\definecolor{currentfill}{rgb}{0.229806,0.298718,0.753683}%
\pgfsetfillcolor{currentfill}%
\pgfsetlinewidth{0.000000pt}%
\definecolor{currentstroke}{rgb}{0.000000,0.000000,0.000000}%
\pgfsetstrokecolor{currentstroke}%
\pgfsetdash{}{0pt}%
\pgfpathmoveto{\pgfqpoint{2.349853in}{0.872712in}}%
\pgfpathlineto{\pgfqpoint{2.590432in}{2.210709in}}%
\pgfpathlineto{\pgfqpoint{2.349853in}{0.872712in}}%
\pgfpathclose%
\pgfusepath{fill}%
\end{pgfscope}%
\begin{pgfscope}%
\pgfpathrectangle{\pgfqpoint{0.017500in}{0.017500in}}{\pgfqpoint{6.218282in}{3.580372in}} %
\pgfusepath{clip}%
\pgfsetbuttcap%
\pgfsetroundjoin%
\definecolor{currentfill}{rgb}{0.229806,0.298718,0.753683}%
\pgfsetfillcolor{currentfill}%
\pgfsetlinewidth{0.000000pt}%
\definecolor{currentstroke}{rgb}{0.000000,0.000000,0.000000}%
\pgfsetstrokecolor{currentstroke}%
\pgfsetdash{}{0pt}%
\pgfpathmoveto{\pgfqpoint{2.349853in}{0.872712in}}%
\pgfpathlineto{\pgfqpoint{2.590432in}{2.210709in}}%
\pgfpathlineto{\pgfqpoint{2.349853in}{0.872712in}}%
\pgfpathclose%
\pgfusepath{fill}%
\end{pgfscope}%
\begin{pgfscope}%
\pgfpathrectangle{\pgfqpoint{0.017500in}{0.017500in}}{\pgfqpoint{6.218282in}{3.580372in}} %
\pgfusepath{clip}%
\pgfsetbuttcap%
\pgfsetroundjoin%
\definecolor{currentfill}{rgb}{0.229806,0.298718,0.753683}%
\pgfsetfillcolor{currentfill}%
\pgfsetlinewidth{0.000000pt}%
\definecolor{currentstroke}{rgb}{0.000000,0.000000,0.000000}%
\pgfsetstrokecolor{currentstroke}%
\pgfsetdash{}{0pt}%
\pgfpathmoveto{\pgfqpoint{2.349853in}{0.872712in}}%
\pgfpathlineto{\pgfqpoint{2.590432in}{2.210709in}}%
\pgfpathlineto{\pgfqpoint{2.349853in}{0.872712in}}%
\pgfpathclose%
\pgfusepath{fill}%
\end{pgfscope}%
\begin{pgfscope}%
\pgfpathrectangle{\pgfqpoint{0.017500in}{0.017500in}}{\pgfqpoint{6.218282in}{3.580372in}} %
\pgfusepath{clip}%
\pgfsetbuttcap%
\pgfsetroundjoin%
\definecolor{currentfill}{rgb}{0.229806,0.298718,0.753683}%
\pgfsetfillcolor{currentfill}%
\pgfsetlinewidth{0.000000pt}%
\definecolor{currentstroke}{rgb}{0.000000,0.000000,0.000000}%
\pgfsetstrokecolor{currentstroke}%
\pgfsetdash{}{0pt}%
\pgfpathmoveto{\pgfqpoint{2.349853in}{0.872712in}}%
\pgfpathlineto{\pgfqpoint{2.590432in}{2.210709in}}%
\pgfpathlineto{\pgfqpoint{2.349853in}{0.872712in}}%
\pgfpathclose%
\pgfusepath{fill}%
\end{pgfscope}%
\begin{pgfscope}%
\pgfpathrectangle{\pgfqpoint{0.017500in}{0.017500in}}{\pgfqpoint{6.218282in}{3.580372in}} %
\pgfusepath{clip}%
\pgfsetbuttcap%
\pgfsetroundjoin%
\definecolor{currentfill}{rgb}{0.229806,0.298718,0.753683}%
\pgfsetfillcolor{currentfill}%
\pgfsetlinewidth{0.000000pt}%
\definecolor{currentstroke}{rgb}{0.000000,0.000000,0.000000}%
\pgfsetstrokecolor{currentstroke}%
\pgfsetdash{}{0pt}%
\pgfpathmoveto{\pgfqpoint{2.349853in}{0.872712in}}%
\pgfpathlineto{\pgfqpoint{2.590432in}{2.210709in}}%
\pgfpathlineto{\pgfqpoint{2.349853in}{0.872712in}}%
\pgfpathclose%
\pgfusepath{fill}%
\end{pgfscope}%
\begin{pgfscope}%
\pgfpathrectangle{\pgfqpoint{0.017500in}{0.017500in}}{\pgfqpoint{6.218282in}{3.580372in}} %
\pgfusepath{clip}%
\pgfsetbuttcap%
\pgfsetroundjoin%
\definecolor{currentfill}{rgb}{0.229806,0.298718,0.753683}%
\pgfsetfillcolor{currentfill}%
\pgfsetlinewidth{0.000000pt}%
\definecolor{currentstroke}{rgb}{0.000000,0.000000,0.000000}%
\pgfsetstrokecolor{currentstroke}%
\pgfsetdash{}{0pt}%
\pgfpathmoveto{\pgfqpoint{2.349853in}{0.872712in}}%
\pgfpathlineto{\pgfqpoint{2.590432in}{2.210709in}}%
\pgfpathlineto{\pgfqpoint{2.349853in}{0.872712in}}%
\pgfpathclose%
\pgfusepath{fill}%
\end{pgfscope}%
\begin{pgfscope}%
\pgfpathrectangle{\pgfqpoint{0.017500in}{0.017500in}}{\pgfqpoint{6.218282in}{3.580372in}} %
\pgfusepath{clip}%
\pgfsetbuttcap%
\pgfsetroundjoin%
\definecolor{currentfill}{rgb}{0.229806,0.298718,0.753683}%
\pgfsetfillcolor{currentfill}%
\pgfsetlinewidth{0.000000pt}%
\definecolor{currentstroke}{rgb}{0.000000,0.000000,0.000000}%
\pgfsetstrokecolor{currentstroke}%
\pgfsetdash{}{0pt}%
\pgfpathmoveto{\pgfqpoint{2.349853in}{0.872712in}}%
\pgfpathlineto{\pgfqpoint{2.590432in}{2.210709in}}%
\pgfpathlineto{\pgfqpoint{2.349853in}{0.872712in}}%
\pgfpathclose%
\pgfusepath{fill}%
\end{pgfscope}%
\begin{pgfscope}%
\pgfpathrectangle{\pgfqpoint{0.017500in}{0.017500in}}{\pgfqpoint{6.218282in}{3.580372in}} %
\pgfusepath{clip}%
\pgfsetbuttcap%
\pgfsetroundjoin%
\definecolor{currentfill}{rgb}{0.229806,0.298718,0.753683}%
\pgfsetfillcolor{currentfill}%
\pgfsetlinewidth{0.000000pt}%
\definecolor{currentstroke}{rgb}{0.000000,0.000000,0.000000}%
\pgfsetstrokecolor{currentstroke}%
\pgfsetdash{}{0pt}%
\pgfpathmoveto{\pgfqpoint{2.349853in}{0.872712in}}%
\pgfpathlineto{\pgfqpoint{2.590432in}{2.210709in}}%
\pgfpathlineto{\pgfqpoint{2.349853in}{0.872712in}}%
\pgfpathclose%
\pgfusepath{fill}%
\end{pgfscope}%
\begin{pgfscope}%
\pgfpathrectangle{\pgfqpoint{0.017500in}{0.017500in}}{\pgfqpoint{6.218282in}{3.580372in}} %
\pgfusepath{clip}%
\pgfsetbuttcap%
\pgfsetroundjoin%
\definecolor{currentfill}{rgb}{0.229806,0.298718,0.753683}%
\pgfsetfillcolor{currentfill}%
\pgfsetlinewidth{0.000000pt}%
\definecolor{currentstroke}{rgb}{0.000000,0.000000,0.000000}%
\pgfsetstrokecolor{currentstroke}%
\pgfsetdash{}{0pt}%
\pgfpathmoveto{\pgfqpoint{2.349853in}{0.872712in}}%
\pgfpathlineto{\pgfqpoint{2.590432in}{2.210709in}}%
\pgfpathlineto{\pgfqpoint{2.349853in}{0.872712in}}%
\pgfpathclose%
\pgfusepath{fill}%
\end{pgfscope}%
\begin{pgfscope}%
\pgfpathrectangle{\pgfqpoint{0.017500in}{0.017500in}}{\pgfqpoint{6.218282in}{3.580372in}} %
\pgfusepath{clip}%
\pgfsetbuttcap%
\pgfsetroundjoin%
\definecolor{currentfill}{rgb}{0.229806,0.298718,0.753683}%
\pgfsetfillcolor{currentfill}%
\pgfsetlinewidth{0.000000pt}%
\definecolor{currentstroke}{rgb}{0.000000,0.000000,0.000000}%
\pgfsetstrokecolor{currentstroke}%
\pgfsetdash{}{0pt}%
\pgfpathmoveto{\pgfqpoint{2.349853in}{0.872712in}}%
\pgfpathlineto{\pgfqpoint{2.590432in}{2.210709in}}%
\pgfpathlineto{\pgfqpoint{2.349853in}{0.872712in}}%
\pgfpathclose%
\pgfusepath{fill}%
\end{pgfscope}%
\begin{pgfscope}%
\pgfpathrectangle{\pgfqpoint{0.017500in}{0.017500in}}{\pgfqpoint{6.218282in}{3.580372in}} %
\pgfusepath{clip}%
\pgfsetbuttcap%
\pgfsetroundjoin%
\definecolor{currentfill}{rgb}{0.229806,0.298718,0.753683}%
\pgfsetfillcolor{currentfill}%
\pgfsetlinewidth{0.000000pt}%
\definecolor{currentstroke}{rgb}{0.000000,0.000000,0.000000}%
\pgfsetstrokecolor{currentstroke}%
\pgfsetdash{}{0pt}%
\pgfpathmoveto{\pgfqpoint{2.349853in}{0.872712in}}%
\pgfpathlineto{\pgfqpoint{2.590432in}{2.210709in}}%
\pgfpathlineto{\pgfqpoint{2.349853in}{0.872712in}}%
\pgfpathclose%
\pgfusepath{fill}%
\end{pgfscope}%
\begin{pgfscope}%
\pgfpathrectangle{\pgfqpoint{0.017500in}{0.017500in}}{\pgfqpoint{6.218282in}{3.580372in}} %
\pgfusepath{clip}%
\pgfsetbuttcap%
\pgfsetroundjoin%
\definecolor{currentfill}{rgb}{0.888390,0.417703,0.327898}%
\pgfsetfillcolor{currentfill}%
\pgfsetlinewidth{0.000000pt}%
\definecolor{currentstroke}{rgb}{0.000000,0.000000,0.000000}%
\pgfsetstrokecolor{currentstroke}%
\pgfsetdash{}{0pt}%
\pgfpathmoveto{\pgfqpoint{1.701778in}{2.375902in}}%
\pgfpathlineto{\pgfqpoint{1.971633in}{2.586781in}}%
\pgfpathlineto{\pgfqpoint{1.701778in}{2.375902in}}%
\pgfpathclose%
\pgfusepath{fill}%
\end{pgfscope}%
\begin{pgfscope}%
\pgfpathrectangle{\pgfqpoint{0.017500in}{0.017500in}}{\pgfqpoint{6.218282in}{3.580372in}} %
\pgfusepath{clip}%
\pgfsetbuttcap%
\pgfsetroundjoin%
\definecolor{currentfill}{rgb}{0.888390,0.417703,0.327898}%
\pgfsetfillcolor{currentfill}%
\pgfsetlinewidth{0.000000pt}%
\definecolor{currentstroke}{rgb}{0.000000,0.000000,0.000000}%
\pgfsetstrokecolor{currentstroke}%
\pgfsetdash{}{0pt}%
\pgfpathmoveto{\pgfqpoint{1.701778in}{2.375902in}}%
\pgfpathlineto{\pgfqpoint{1.971633in}{2.586781in}}%
\pgfpathlineto{\pgfqpoint{1.701778in}{2.375902in}}%
\pgfpathclose%
\pgfusepath{fill}%
\end{pgfscope}%
\begin{pgfscope}%
\pgfpathrectangle{\pgfqpoint{0.017500in}{0.017500in}}{\pgfqpoint{6.218282in}{3.580372in}} %
\pgfusepath{clip}%
\pgfsetbuttcap%
\pgfsetroundjoin%
\definecolor{currentfill}{rgb}{0.888390,0.417703,0.327898}%
\pgfsetfillcolor{currentfill}%
\pgfsetlinewidth{0.000000pt}%
\definecolor{currentstroke}{rgb}{0.000000,0.000000,0.000000}%
\pgfsetstrokecolor{currentstroke}%
\pgfsetdash{}{0pt}%
\pgfpathmoveto{\pgfqpoint{1.701778in}{2.375902in}}%
\pgfpathlineto{\pgfqpoint{1.971633in}{2.586781in}}%
\pgfpathlineto{\pgfqpoint{1.701778in}{2.375902in}}%
\pgfpathclose%
\pgfusepath{fill}%
\end{pgfscope}%
\begin{pgfscope}%
\pgfpathrectangle{\pgfqpoint{0.017500in}{0.017500in}}{\pgfqpoint{6.218282in}{3.580372in}} %
\pgfusepath{clip}%
\pgfsetbuttcap%
\pgfsetroundjoin%
\definecolor{currentfill}{rgb}{0.888390,0.417703,0.327898}%
\pgfsetfillcolor{currentfill}%
\pgfsetlinewidth{0.000000pt}%
\definecolor{currentstroke}{rgb}{0.000000,0.000000,0.000000}%
\pgfsetstrokecolor{currentstroke}%
\pgfsetdash{}{0pt}%
\pgfpathmoveto{\pgfqpoint{1.701778in}{2.375902in}}%
\pgfpathlineto{\pgfqpoint{1.971633in}{2.586781in}}%
\pgfpathlineto{\pgfqpoint{1.701778in}{2.375902in}}%
\pgfpathclose%
\pgfusepath{fill}%
\end{pgfscope}%
\begin{pgfscope}%
\pgfpathrectangle{\pgfqpoint{0.017500in}{0.017500in}}{\pgfqpoint{6.218282in}{3.580372in}} %
\pgfusepath{clip}%
\pgfsetbuttcap%
\pgfsetroundjoin%
\definecolor{currentfill}{rgb}{0.888390,0.417703,0.327898}%
\pgfsetfillcolor{currentfill}%
\pgfsetlinewidth{0.000000pt}%
\definecolor{currentstroke}{rgb}{0.000000,0.000000,0.000000}%
\pgfsetstrokecolor{currentstroke}%
\pgfsetdash{}{0pt}%
\pgfpathmoveto{\pgfqpoint{1.701778in}{2.375902in}}%
\pgfpathlineto{\pgfqpoint{1.971633in}{2.586781in}}%
\pgfpathlineto{\pgfqpoint{1.701778in}{2.375902in}}%
\pgfpathclose%
\pgfusepath{fill}%
\end{pgfscope}%
\begin{pgfscope}%
\pgfpathrectangle{\pgfqpoint{0.017500in}{0.017500in}}{\pgfqpoint{6.218282in}{3.580372in}} %
\pgfusepath{clip}%
\pgfsetbuttcap%
\pgfsetroundjoin%
\definecolor{currentfill}{rgb}{0.888390,0.417703,0.327898}%
\pgfsetfillcolor{currentfill}%
\pgfsetlinewidth{0.000000pt}%
\definecolor{currentstroke}{rgb}{0.000000,0.000000,0.000000}%
\pgfsetstrokecolor{currentstroke}%
\pgfsetdash{}{0pt}%
\pgfpathmoveto{\pgfqpoint{1.701778in}{2.375902in}}%
\pgfpathlineto{\pgfqpoint{1.971633in}{2.586781in}}%
\pgfpathlineto{\pgfqpoint{1.701778in}{2.375902in}}%
\pgfpathclose%
\pgfusepath{fill}%
\end{pgfscope}%
\begin{pgfscope}%
\pgfpathrectangle{\pgfqpoint{0.017500in}{0.017500in}}{\pgfqpoint{6.218282in}{3.580372in}} %
\pgfusepath{clip}%
\pgfsetbuttcap%
\pgfsetroundjoin%
\definecolor{currentfill}{rgb}{0.888390,0.417703,0.327898}%
\pgfsetfillcolor{currentfill}%
\pgfsetlinewidth{0.000000pt}%
\definecolor{currentstroke}{rgb}{0.000000,0.000000,0.000000}%
\pgfsetstrokecolor{currentstroke}%
\pgfsetdash{}{0pt}%
\pgfpathmoveto{\pgfqpoint{1.701778in}{2.375902in}}%
\pgfpathlineto{\pgfqpoint{1.971633in}{2.586781in}}%
\pgfpathlineto{\pgfqpoint{1.701778in}{2.375902in}}%
\pgfpathclose%
\pgfusepath{fill}%
\end{pgfscope}%
\begin{pgfscope}%
\pgfpathrectangle{\pgfqpoint{0.017500in}{0.017500in}}{\pgfqpoint{6.218282in}{3.580372in}} %
\pgfusepath{clip}%
\pgfsetbuttcap%
\pgfsetroundjoin%
\definecolor{currentfill}{rgb}{0.888390,0.417703,0.327898}%
\pgfsetfillcolor{currentfill}%
\pgfsetlinewidth{0.000000pt}%
\definecolor{currentstroke}{rgb}{0.000000,0.000000,0.000000}%
\pgfsetstrokecolor{currentstroke}%
\pgfsetdash{}{0pt}%
\pgfpathmoveto{\pgfqpoint{1.701778in}{2.375902in}}%
\pgfpathlineto{\pgfqpoint{1.971633in}{2.586781in}}%
\pgfpathlineto{\pgfqpoint{1.701778in}{2.375902in}}%
\pgfpathclose%
\pgfusepath{fill}%
\end{pgfscope}%
\begin{pgfscope}%
\pgfpathrectangle{\pgfqpoint{0.017500in}{0.017500in}}{\pgfqpoint{6.218282in}{3.580372in}} %
\pgfusepath{clip}%
\pgfsetbuttcap%
\pgfsetroundjoin%
\definecolor{currentfill}{rgb}{0.888390,0.417703,0.327898}%
\pgfsetfillcolor{currentfill}%
\pgfsetlinewidth{0.000000pt}%
\definecolor{currentstroke}{rgb}{0.000000,0.000000,0.000000}%
\pgfsetstrokecolor{currentstroke}%
\pgfsetdash{}{0pt}%
\pgfpathmoveto{\pgfqpoint{1.701778in}{2.375902in}}%
\pgfpathlineto{\pgfqpoint{1.971633in}{2.586781in}}%
\pgfpathlineto{\pgfqpoint{1.701778in}{2.375902in}}%
\pgfpathclose%
\pgfusepath{fill}%
\end{pgfscope}%
\begin{pgfscope}%
\pgfpathrectangle{\pgfqpoint{0.017500in}{0.017500in}}{\pgfqpoint{6.218282in}{3.580372in}} %
\pgfusepath{clip}%
\pgfsetbuttcap%
\pgfsetroundjoin%
\definecolor{currentfill}{rgb}{0.888390,0.417703,0.327898}%
\pgfsetfillcolor{currentfill}%
\pgfsetlinewidth{0.000000pt}%
\definecolor{currentstroke}{rgb}{0.000000,0.000000,0.000000}%
\pgfsetstrokecolor{currentstroke}%
\pgfsetdash{}{0pt}%
\pgfpathmoveto{\pgfqpoint{1.701778in}{2.375902in}}%
\pgfpathlineto{\pgfqpoint{1.971633in}{2.586781in}}%
\pgfpathlineto{\pgfqpoint{1.701778in}{2.375902in}}%
\pgfpathclose%
\pgfusepath{fill}%
\end{pgfscope}%
\begin{pgfscope}%
\pgfpathrectangle{\pgfqpoint{0.017500in}{0.017500in}}{\pgfqpoint{6.218282in}{3.580372in}} %
\pgfusepath{clip}%
\pgfsetbuttcap%
\pgfsetroundjoin%
\definecolor{currentfill}{rgb}{0.888390,0.417703,0.327898}%
\pgfsetfillcolor{currentfill}%
\pgfsetlinewidth{0.000000pt}%
\definecolor{currentstroke}{rgb}{0.000000,0.000000,0.000000}%
\pgfsetstrokecolor{currentstroke}%
\pgfsetdash{}{0pt}%
\pgfpathmoveto{\pgfqpoint{1.701778in}{2.375902in}}%
\pgfpathlineto{\pgfqpoint{1.971633in}{2.586781in}}%
\pgfpathlineto{\pgfqpoint{1.701778in}{2.375902in}}%
\pgfpathclose%
\pgfusepath{fill}%
\end{pgfscope}%
\begin{pgfscope}%
\pgfpathrectangle{\pgfqpoint{0.017500in}{0.017500in}}{\pgfqpoint{6.218282in}{3.580372in}} %
\pgfusepath{clip}%
\pgfsetbuttcap%
\pgfsetroundjoin%
\definecolor{currentfill}{rgb}{0.888390,0.417703,0.327898}%
\pgfsetfillcolor{currentfill}%
\pgfsetlinewidth{0.000000pt}%
\definecolor{currentstroke}{rgb}{0.000000,0.000000,0.000000}%
\pgfsetstrokecolor{currentstroke}%
\pgfsetdash{}{0pt}%
\pgfpathmoveto{\pgfqpoint{1.701778in}{2.375902in}}%
\pgfpathlineto{\pgfqpoint{1.971633in}{2.586781in}}%
\pgfpathlineto{\pgfqpoint{1.701778in}{2.375902in}}%
\pgfpathclose%
\pgfusepath{fill}%
\end{pgfscope}%
\begin{pgfscope}%
\pgfpathrectangle{\pgfqpoint{0.017500in}{0.017500in}}{\pgfqpoint{6.218282in}{3.580372in}} %
\pgfusepath{clip}%
\pgfsetbuttcap%
\pgfsetroundjoin%
\definecolor{currentfill}{rgb}{0.888390,0.417703,0.327898}%
\pgfsetfillcolor{currentfill}%
\pgfsetlinewidth{0.000000pt}%
\definecolor{currentstroke}{rgb}{0.000000,0.000000,0.000000}%
\pgfsetstrokecolor{currentstroke}%
\pgfsetdash{}{0pt}%
\pgfpathmoveto{\pgfqpoint{1.701778in}{2.375902in}}%
\pgfpathlineto{\pgfqpoint{1.971633in}{2.586781in}}%
\pgfpathlineto{\pgfqpoint{1.701778in}{2.375902in}}%
\pgfpathclose%
\pgfusepath{fill}%
\end{pgfscope}%
\begin{pgfscope}%
\pgfpathrectangle{\pgfqpoint{0.017500in}{0.017500in}}{\pgfqpoint{6.218282in}{3.580372in}} %
\pgfusepath{clip}%
\pgfsetbuttcap%
\pgfsetroundjoin%
\definecolor{currentfill}{rgb}{0.888390,0.417703,0.327898}%
\pgfsetfillcolor{currentfill}%
\pgfsetlinewidth{0.000000pt}%
\definecolor{currentstroke}{rgb}{0.000000,0.000000,0.000000}%
\pgfsetstrokecolor{currentstroke}%
\pgfsetdash{}{0pt}%
\pgfpathmoveto{\pgfqpoint{1.701778in}{2.375902in}}%
\pgfpathlineto{\pgfqpoint{1.971633in}{2.586781in}}%
\pgfpathlineto{\pgfqpoint{1.701778in}{2.375902in}}%
\pgfpathclose%
\pgfusepath{fill}%
\end{pgfscope}%
\begin{pgfscope}%
\pgfpathrectangle{\pgfqpoint{0.017500in}{0.017500in}}{\pgfqpoint{6.218282in}{3.580372in}} %
\pgfusepath{clip}%
\pgfsetbuttcap%
\pgfsetroundjoin%
\definecolor{currentfill}{rgb}{0.888390,0.417703,0.327898}%
\pgfsetfillcolor{currentfill}%
\pgfsetlinewidth{0.000000pt}%
\definecolor{currentstroke}{rgb}{0.000000,0.000000,0.000000}%
\pgfsetstrokecolor{currentstroke}%
\pgfsetdash{}{0pt}%
\pgfpathmoveto{\pgfqpoint{1.701778in}{2.375902in}}%
\pgfpathlineto{\pgfqpoint{1.971633in}{2.586781in}}%
\pgfpathlineto{\pgfqpoint{1.701778in}{2.375902in}}%
\pgfpathclose%
\pgfusepath{fill}%
\end{pgfscope}%
\begin{pgfscope}%
\pgfpathrectangle{\pgfqpoint{0.017500in}{0.017500in}}{\pgfqpoint{6.218282in}{3.580372in}} %
\pgfusepath{clip}%
\pgfsetbuttcap%
\pgfsetroundjoin%
\definecolor{currentfill}{rgb}{0.888390,0.417703,0.327898}%
\pgfsetfillcolor{currentfill}%
\pgfsetlinewidth{0.000000pt}%
\definecolor{currentstroke}{rgb}{0.000000,0.000000,0.000000}%
\pgfsetstrokecolor{currentstroke}%
\pgfsetdash{}{0pt}%
\pgfpathmoveto{\pgfqpoint{1.701778in}{2.375902in}}%
\pgfpathlineto{\pgfqpoint{1.971633in}{2.586781in}}%
\pgfpathlineto{\pgfqpoint{1.701778in}{2.375902in}}%
\pgfpathclose%
\pgfusepath{fill}%
\end{pgfscope}%
\begin{pgfscope}%
\pgfpathrectangle{\pgfqpoint{0.017500in}{0.017500in}}{\pgfqpoint{6.218282in}{3.580372in}} %
\pgfusepath{clip}%
\pgfsetbuttcap%
\pgfsetroundjoin%
\definecolor{currentfill}{rgb}{0.888390,0.417703,0.327898}%
\pgfsetfillcolor{currentfill}%
\pgfsetlinewidth{0.000000pt}%
\definecolor{currentstroke}{rgb}{0.000000,0.000000,0.000000}%
\pgfsetstrokecolor{currentstroke}%
\pgfsetdash{}{0pt}%
\pgfpathmoveto{\pgfqpoint{1.701778in}{2.375902in}}%
\pgfpathlineto{\pgfqpoint{1.971633in}{2.586781in}}%
\pgfpathlineto{\pgfqpoint{1.701778in}{2.375902in}}%
\pgfpathclose%
\pgfusepath{fill}%
\end{pgfscope}%
\begin{pgfscope}%
\pgfpathrectangle{\pgfqpoint{0.017500in}{0.017500in}}{\pgfqpoint{6.218282in}{3.580372in}} %
\pgfusepath{clip}%
\pgfsetbuttcap%
\pgfsetroundjoin%
\definecolor{currentfill}{rgb}{0.888390,0.417703,0.327898}%
\pgfsetfillcolor{currentfill}%
\pgfsetlinewidth{0.000000pt}%
\definecolor{currentstroke}{rgb}{0.000000,0.000000,0.000000}%
\pgfsetstrokecolor{currentstroke}%
\pgfsetdash{}{0pt}%
\pgfpathmoveto{\pgfqpoint{1.701778in}{2.375902in}}%
\pgfpathlineto{\pgfqpoint{1.971633in}{2.586781in}}%
\pgfpathlineto{\pgfqpoint{1.701778in}{2.375902in}}%
\pgfpathclose%
\pgfusepath{fill}%
\end{pgfscope}%
\begin{pgfscope}%
\pgfpathrectangle{\pgfqpoint{0.017500in}{0.017500in}}{\pgfqpoint{6.218282in}{3.580372in}} %
\pgfusepath{clip}%
\pgfsetbuttcap%
\pgfsetroundjoin%
\definecolor{currentfill}{rgb}{0.888390,0.417703,0.327898}%
\pgfsetfillcolor{currentfill}%
\pgfsetlinewidth{0.000000pt}%
\definecolor{currentstroke}{rgb}{0.000000,0.000000,0.000000}%
\pgfsetstrokecolor{currentstroke}%
\pgfsetdash{}{0pt}%
\pgfpathmoveto{\pgfqpoint{1.701778in}{2.375902in}}%
\pgfpathlineto{\pgfqpoint{1.971633in}{2.586781in}}%
\pgfpathlineto{\pgfqpoint{1.701778in}{2.375902in}}%
\pgfpathclose%
\pgfusepath{fill}%
\end{pgfscope}%
\begin{pgfscope}%
\pgfpathrectangle{\pgfqpoint{0.017500in}{0.017500in}}{\pgfqpoint{6.218282in}{3.580372in}} %
\pgfusepath{clip}%
\pgfsetbuttcap%
\pgfsetroundjoin%
\definecolor{currentfill}{rgb}{0.888390,0.417703,0.327898}%
\pgfsetfillcolor{currentfill}%
\pgfsetlinewidth{0.000000pt}%
\definecolor{currentstroke}{rgb}{0.000000,0.000000,0.000000}%
\pgfsetstrokecolor{currentstroke}%
\pgfsetdash{}{0pt}%
\pgfpathmoveto{\pgfqpoint{1.701778in}{2.375902in}}%
\pgfpathlineto{\pgfqpoint{1.971633in}{2.586781in}}%
\pgfpathlineto{\pgfqpoint{1.701778in}{2.375902in}}%
\pgfpathclose%
\pgfusepath{fill}%
\end{pgfscope}%
\begin{pgfscope}%
\pgfpathrectangle{\pgfqpoint{0.017500in}{0.017500in}}{\pgfqpoint{6.218282in}{3.580372in}} %
\pgfusepath{clip}%
\pgfsetbuttcap%
\pgfsetroundjoin%
\definecolor{currentfill}{rgb}{0.888390,0.417703,0.327898}%
\pgfsetfillcolor{currentfill}%
\pgfsetlinewidth{0.000000pt}%
\definecolor{currentstroke}{rgb}{0.000000,0.000000,0.000000}%
\pgfsetstrokecolor{currentstroke}%
\pgfsetdash{}{0pt}%
\pgfpathmoveto{\pgfqpoint{1.701778in}{2.375902in}}%
\pgfpathlineto{\pgfqpoint{1.971633in}{2.586781in}}%
\pgfpathlineto{\pgfqpoint{1.701778in}{2.375902in}}%
\pgfpathclose%
\pgfusepath{fill}%
\end{pgfscope}%
\begin{pgfscope}%
\pgfpathrectangle{\pgfqpoint{0.017500in}{0.017500in}}{\pgfqpoint{6.218282in}{3.580372in}} %
\pgfusepath{clip}%
\pgfsetbuttcap%
\pgfsetroundjoin%
\definecolor{currentfill}{rgb}{0.888390,0.417703,0.327898}%
\pgfsetfillcolor{currentfill}%
\pgfsetlinewidth{0.000000pt}%
\definecolor{currentstroke}{rgb}{0.000000,0.000000,0.000000}%
\pgfsetstrokecolor{currentstroke}%
\pgfsetdash{}{0pt}%
\pgfpathmoveto{\pgfqpoint{1.701778in}{2.375902in}}%
\pgfpathlineto{\pgfqpoint{1.971633in}{2.586781in}}%
\pgfpathlineto{\pgfqpoint{1.701778in}{2.375902in}}%
\pgfpathclose%
\pgfusepath{fill}%
\end{pgfscope}%
\begin{pgfscope}%
\pgfpathrectangle{\pgfqpoint{0.017500in}{0.017500in}}{\pgfqpoint{6.218282in}{3.580372in}} %
\pgfusepath{clip}%
\pgfsetbuttcap%
\pgfsetroundjoin%
\definecolor{currentfill}{rgb}{0.888390,0.417703,0.327898}%
\pgfsetfillcolor{currentfill}%
\pgfsetlinewidth{0.000000pt}%
\definecolor{currentstroke}{rgb}{0.000000,0.000000,0.000000}%
\pgfsetstrokecolor{currentstroke}%
\pgfsetdash{}{0pt}%
\pgfpathmoveto{\pgfqpoint{1.701778in}{2.375902in}}%
\pgfpathlineto{\pgfqpoint{1.971633in}{2.586781in}}%
\pgfpathlineto{\pgfqpoint{1.701778in}{2.375902in}}%
\pgfpathclose%
\pgfusepath{fill}%
\end{pgfscope}%
\begin{pgfscope}%
\pgfpathrectangle{\pgfqpoint{0.017500in}{0.017500in}}{\pgfqpoint{6.218282in}{3.580372in}} %
\pgfusepath{clip}%
\pgfsetbuttcap%
\pgfsetroundjoin%
\definecolor{currentfill}{rgb}{0.888390,0.417703,0.327898}%
\pgfsetfillcolor{currentfill}%
\pgfsetlinewidth{0.000000pt}%
\definecolor{currentstroke}{rgb}{0.000000,0.000000,0.000000}%
\pgfsetstrokecolor{currentstroke}%
\pgfsetdash{}{0pt}%
\pgfpathmoveto{\pgfqpoint{1.701778in}{2.375902in}}%
\pgfpathlineto{\pgfqpoint{1.971633in}{2.586781in}}%
\pgfpathlineto{\pgfqpoint{1.701778in}{2.375902in}}%
\pgfpathclose%
\pgfusepath{fill}%
\end{pgfscope}%
\begin{pgfscope}%
\pgfpathrectangle{\pgfqpoint{0.017500in}{0.017500in}}{\pgfqpoint{6.218282in}{3.580372in}} %
\pgfusepath{clip}%
\pgfsetbuttcap%
\pgfsetroundjoin%
\definecolor{currentfill}{rgb}{0.888390,0.417703,0.327898}%
\pgfsetfillcolor{currentfill}%
\pgfsetlinewidth{0.000000pt}%
\definecolor{currentstroke}{rgb}{0.000000,0.000000,0.000000}%
\pgfsetstrokecolor{currentstroke}%
\pgfsetdash{}{0pt}%
\pgfpathmoveto{\pgfqpoint{1.701778in}{2.375902in}}%
\pgfpathlineto{\pgfqpoint{1.971633in}{2.586781in}}%
\pgfpathlineto{\pgfqpoint{1.701778in}{2.375902in}}%
\pgfpathclose%
\pgfusepath{fill}%
\end{pgfscope}%
\begin{pgfscope}%
\pgfpathrectangle{\pgfqpoint{0.017500in}{0.017500in}}{\pgfqpoint{6.218282in}{3.580372in}} %
\pgfusepath{clip}%
\pgfsetbuttcap%
\pgfsetroundjoin%
\definecolor{currentfill}{rgb}{0.888390,0.417703,0.327898}%
\pgfsetfillcolor{currentfill}%
\pgfsetlinewidth{0.000000pt}%
\definecolor{currentstroke}{rgb}{0.000000,0.000000,0.000000}%
\pgfsetstrokecolor{currentstroke}%
\pgfsetdash{}{0pt}%
\pgfpathmoveto{\pgfqpoint{1.701778in}{2.375902in}}%
\pgfpathlineto{\pgfqpoint{1.971633in}{2.586781in}}%
\pgfpathlineto{\pgfqpoint{1.701778in}{2.375902in}}%
\pgfpathclose%
\pgfusepath{fill}%
\end{pgfscope}%
\begin{pgfscope}%
\pgfpathrectangle{\pgfqpoint{0.017500in}{0.017500in}}{\pgfqpoint{6.218282in}{3.580372in}} %
\pgfusepath{clip}%
\pgfsetbuttcap%
\pgfsetroundjoin%
\definecolor{currentfill}{rgb}{0.888390,0.417703,0.327898}%
\pgfsetfillcolor{currentfill}%
\pgfsetlinewidth{0.000000pt}%
\definecolor{currentstroke}{rgb}{0.000000,0.000000,0.000000}%
\pgfsetstrokecolor{currentstroke}%
\pgfsetdash{}{0pt}%
\pgfpathmoveto{\pgfqpoint{1.701778in}{2.375902in}}%
\pgfpathlineto{\pgfqpoint{1.971633in}{2.586781in}}%
\pgfpathlineto{\pgfqpoint{1.701778in}{2.375902in}}%
\pgfpathclose%
\pgfusepath{fill}%
\end{pgfscope}%
\begin{pgfscope}%
\pgfpathrectangle{\pgfqpoint{0.017500in}{0.017500in}}{\pgfqpoint{6.218282in}{3.580372in}} %
\pgfusepath{clip}%
\pgfsetbuttcap%
\pgfsetroundjoin%
\definecolor{currentfill}{rgb}{0.888390,0.417703,0.327898}%
\pgfsetfillcolor{currentfill}%
\pgfsetlinewidth{0.000000pt}%
\definecolor{currentstroke}{rgb}{0.000000,0.000000,0.000000}%
\pgfsetstrokecolor{currentstroke}%
\pgfsetdash{}{0pt}%
\pgfpathmoveto{\pgfqpoint{1.701778in}{2.375902in}}%
\pgfpathlineto{\pgfqpoint{1.971633in}{2.586781in}}%
\pgfpathlineto{\pgfqpoint{1.701778in}{2.375902in}}%
\pgfpathclose%
\pgfusepath{fill}%
\end{pgfscope}%
\begin{pgfscope}%
\pgfpathrectangle{\pgfqpoint{0.017500in}{0.017500in}}{\pgfqpoint{6.218282in}{3.580372in}} %
\pgfusepath{clip}%
\pgfsetbuttcap%
\pgfsetroundjoin%
\definecolor{currentfill}{rgb}{0.888390,0.417703,0.327898}%
\pgfsetfillcolor{currentfill}%
\pgfsetlinewidth{0.000000pt}%
\definecolor{currentstroke}{rgb}{0.000000,0.000000,0.000000}%
\pgfsetstrokecolor{currentstroke}%
\pgfsetdash{}{0pt}%
\pgfpathmoveto{\pgfqpoint{1.701778in}{2.375902in}}%
\pgfpathlineto{\pgfqpoint{1.971633in}{2.586781in}}%
\pgfpathlineto{\pgfqpoint{1.701778in}{2.375902in}}%
\pgfpathclose%
\pgfusepath{fill}%
\end{pgfscope}%
\begin{pgfscope}%
\pgfpathrectangle{\pgfqpoint{0.017500in}{0.017500in}}{\pgfqpoint{6.218282in}{3.580372in}} %
\pgfusepath{clip}%
\pgfsetbuttcap%
\pgfsetroundjoin%
\definecolor{currentfill}{rgb}{0.888390,0.417703,0.327898}%
\pgfsetfillcolor{currentfill}%
\pgfsetlinewidth{0.000000pt}%
\definecolor{currentstroke}{rgb}{0.000000,0.000000,0.000000}%
\pgfsetstrokecolor{currentstroke}%
\pgfsetdash{}{0pt}%
\pgfpathmoveto{\pgfqpoint{1.701778in}{2.375902in}}%
\pgfpathlineto{\pgfqpoint{1.971633in}{2.586781in}}%
\pgfpathlineto{\pgfqpoint{1.701778in}{2.375902in}}%
\pgfpathclose%
\pgfusepath{fill}%
\end{pgfscope}%
\begin{pgfscope}%
\pgfpathrectangle{\pgfqpoint{0.017500in}{0.017500in}}{\pgfqpoint{6.218282in}{3.580372in}} %
\pgfusepath{clip}%
\pgfsetbuttcap%
\pgfsetroundjoin%
\definecolor{currentfill}{rgb}{0.763520,0.178667,0.193396}%
\pgfsetfillcolor{currentfill}%
\pgfsetlinewidth{0.000000pt}%
\definecolor{currentstroke}{rgb}{0.000000,0.000000,0.000000}%
\pgfsetstrokecolor{currentstroke}%
\pgfsetdash{}{0pt}%
\pgfpathmoveto{\pgfqpoint{3.564492in}{2.541257in}}%
\pgfpathlineto{\pgfqpoint{3.816789in}{2.677291in}}%
\pgfpathlineto{\pgfqpoint{3.564492in}{2.541257in}}%
\pgfpathclose%
\pgfusepath{fill}%
\end{pgfscope}%
\begin{pgfscope}%
\pgfpathrectangle{\pgfqpoint{0.017500in}{0.017500in}}{\pgfqpoint{6.218282in}{3.580372in}} %
\pgfusepath{clip}%
\pgfsetbuttcap%
\pgfsetroundjoin%
\definecolor{currentfill}{rgb}{0.763520,0.178667,0.193396}%
\pgfsetfillcolor{currentfill}%
\pgfsetlinewidth{0.000000pt}%
\definecolor{currentstroke}{rgb}{0.000000,0.000000,0.000000}%
\pgfsetstrokecolor{currentstroke}%
\pgfsetdash{}{0pt}%
\pgfpathmoveto{\pgfqpoint{3.564492in}{2.541257in}}%
\pgfpathlineto{\pgfqpoint{3.816789in}{2.677291in}}%
\pgfpathlineto{\pgfqpoint{3.564492in}{2.541257in}}%
\pgfpathclose%
\pgfusepath{fill}%
\end{pgfscope}%
\begin{pgfscope}%
\pgfpathrectangle{\pgfqpoint{0.017500in}{0.017500in}}{\pgfqpoint{6.218282in}{3.580372in}} %
\pgfusepath{clip}%
\pgfsetbuttcap%
\pgfsetroundjoin%
\definecolor{currentfill}{rgb}{0.763520,0.178667,0.193396}%
\pgfsetfillcolor{currentfill}%
\pgfsetlinewidth{0.000000pt}%
\definecolor{currentstroke}{rgb}{0.000000,0.000000,0.000000}%
\pgfsetstrokecolor{currentstroke}%
\pgfsetdash{}{0pt}%
\pgfpathmoveto{\pgfqpoint{3.564492in}{2.541257in}}%
\pgfpathlineto{\pgfqpoint{3.816789in}{2.677291in}}%
\pgfpathlineto{\pgfqpoint{3.564492in}{2.541257in}}%
\pgfpathclose%
\pgfusepath{fill}%
\end{pgfscope}%
\begin{pgfscope}%
\pgfpathrectangle{\pgfqpoint{0.017500in}{0.017500in}}{\pgfqpoint{6.218282in}{3.580372in}} %
\pgfusepath{clip}%
\pgfsetbuttcap%
\pgfsetroundjoin%
\definecolor{currentfill}{rgb}{0.763520,0.178667,0.193396}%
\pgfsetfillcolor{currentfill}%
\pgfsetlinewidth{0.000000pt}%
\definecolor{currentstroke}{rgb}{0.000000,0.000000,0.000000}%
\pgfsetstrokecolor{currentstroke}%
\pgfsetdash{}{0pt}%
\pgfpathmoveto{\pgfqpoint{3.564492in}{2.541257in}}%
\pgfpathlineto{\pgfqpoint{3.816789in}{2.677291in}}%
\pgfpathlineto{\pgfqpoint{3.564492in}{2.541257in}}%
\pgfpathclose%
\pgfusepath{fill}%
\end{pgfscope}%
\begin{pgfscope}%
\pgfpathrectangle{\pgfqpoint{0.017500in}{0.017500in}}{\pgfqpoint{6.218282in}{3.580372in}} %
\pgfusepath{clip}%
\pgfsetbuttcap%
\pgfsetroundjoin%
\definecolor{currentfill}{rgb}{0.763520,0.178667,0.193396}%
\pgfsetfillcolor{currentfill}%
\pgfsetlinewidth{0.000000pt}%
\definecolor{currentstroke}{rgb}{0.000000,0.000000,0.000000}%
\pgfsetstrokecolor{currentstroke}%
\pgfsetdash{}{0pt}%
\pgfpathmoveto{\pgfqpoint{3.564492in}{2.541257in}}%
\pgfpathlineto{\pgfqpoint{3.816789in}{2.677291in}}%
\pgfpathlineto{\pgfqpoint{3.564492in}{2.541257in}}%
\pgfpathclose%
\pgfusepath{fill}%
\end{pgfscope}%
\begin{pgfscope}%
\pgfpathrectangle{\pgfqpoint{0.017500in}{0.017500in}}{\pgfqpoint{6.218282in}{3.580372in}} %
\pgfusepath{clip}%
\pgfsetbuttcap%
\pgfsetroundjoin%
\definecolor{currentfill}{rgb}{0.763520,0.178667,0.193396}%
\pgfsetfillcolor{currentfill}%
\pgfsetlinewidth{0.000000pt}%
\definecolor{currentstroke}{rgb}{0.000000,0.000000,0.000000}%
\pgfsetstrokecolor{currentstroke}%
\pgfsetdash{}{0pt}%
\pgfpathmoveto{\pgfqpoint{3.564492in}{2.541257in}}%
\pgfpathlineto{\pgfqpoint{3.816789in}{2.677291in}}%
\pgfpathlineto{\pgfqpoint{3.564492in}{2.541257in}}%
\pgfpathclose%
\pgfusepath{fill}%
\end{pgfscope}%
\begin{pgfscope}%
\pgfpathrectangle{\pgfqpoint{0.017500in}{0.017500in}}{\pgfqpoint{6.218282in}{3.580372in}} %
\pgfusepath{clip}%
\pgfsetbuttcap%
\pgfsetroundjoin%
\definecolor{currentfill}{rgb}{0.763520,0.178667,0.193396}%
\pgfsetfillcolor{currentfill}%
\pgfsetlinewidth{0.000000pt}%
\definecolor{currentstroke}{rgb}{0.000000,0.000000,0.000000}%
\pgfsetstrokecolor{currentstroke}%
\pgfsetdash{}{0pt}%
\pgfpathmoveto{\pgfqpoint{3.564492in}{2.541257in}}%
\pgfpathlineto{\pgfqpoint{3.816789in}{2.677291in}}%
\pgfpathlineto{\pgfqpoint{3.564492in}{2.541257in}}%
\pgfpathclose%
\pgfusepath{fill}%
\end{pgfscope}%
\begin{pgfscope}%
\pgfpathrectangle{\pgfqpoint{0.017500in}{0.017500in}}{\pgfqpoint{6.218282in}{3.580372in}} %
\pgfusepath{clip}%
\pgfsetbuttcap%
\pgfsetroundjoin%
\definecolor{currentfill}{rgb}{0.763520,0.178667,0.193396}%
\pgfsetfillcolor{currentfill}%
\pgfsetlinewidth{0.000000pt}%
\definecolor{currentstroke}{rgb}{0.000000,0.000000,0.000000}%
\pgfsetstrokecolor{currentstroke}%
\pgfsetdash{}{0pt}%
\pgfpathmoveto{\pgfqpoint{3.564492in}{2.541257in}}%
\pgfpathlineto{\pgfqpoint{3.816789in}{2.677291in}}%
\pgfpathlineto{\pgfqpoint{3.564492in}{2.541257in}}%
\pgfpathclose%
\pgfusepath{fill}%
\end{pgfscope}%
\begin{pgfscope}%
\pgfpathrectangle{\pgfqpoint{0.017500in}{0.017500in}}{\pgfqpoint{6.218282in}{3.580372in}} %
\pgfusepath{clip}%
\pgfsetbuttcap%
\pgfsetroundjoin%
\definecolor{currentfill}{rgb}{0.763520,0.178667,0.193396}%
\pgfsetfillcolor{currentfill}%
\pgfsetlinewidth{0.000000pt}%
\definecolor{currentstroke}{rgb}{0.000000,0.000000,0.000000}%
\pgfsetstrokecolor{currentstroke}%
\pgfsetdash{}{0pt}%
\pgfpathmoveto{\pgfqpoint{3.564492in}{2.541257in}}%
\pgfpathlineto{\pgfqpoint{3.816789in}{2.677291in}}%
\pgfpathlineto{\pgfqpoint{3.564492in}{2.541257in}}%
\pgfpathclose%
\pgfusepath{fill}%
\end{pgfscope}%
\begin{pgfscope}%
\pgfpathrectangle{\pgfqpoint{0.017500in}{0.017500in}}{\pgfqpoint{6.218282in}{3.580372in}} %
\pgfusepath{clip}%
\pgfsetbuttcap%
\pgfsetroundjoin%
\definecolor{currentfill}{rgb}{0.763520,0.178667,0.193396}%
\pgfsetfillcolor{currentfill}%
\pgfsetlinewidth{0.000000pt}%
\definecolor{currentstroke}{rgb}{0.000000,0.000000,0.000000}%
\pgfsetstrokecolor{currentstroke}%
\pgfsetdash{}{0pt}%
\pgfpathmoveto{\pgfqpoint{3.564492in}{2.541257in}}%
\pgfpathlineto{\pgfqpoint{3.816789in}{2.677291in}}%
\pgfpathlineto{\pgfqpoint{3.564492in}{2.541257in}}%
\pgfpathclose%
\pgfusepath{fill}%
\end{pgfscope}%
\begin{pgfscope}%
\pgfpathrectangle{\pgfqpoint{0.017500in}{0.017500in}}{\pgfqpoint{6.218282in}{3.580372in}} %
\pgfusepath{clip}%
\pgfsetbuttcap%
\pgfsetroundjoin%
\definecolor{currentfill}{rgb}{0.763520,0.178667,0.193396}%
\pgfsetfillcolor{currentfill}%
\pgfsetlinewidth{0.000000pt}%
\definecolor{currentstroke}{rgb}{0.000000,0.000000,0.000000}%
\pgfsetstrokecolor{currentstroke}%
\pgfsetdash{}{0pt}%
\pgfpathmoveto{\pgfqpoint{3.564492in}{2.541257in}}%
\pgfpathlineto{\pgfqpoint{3.816789in}{2.677291in}}%
\pgfpathlineto{\pgfqpoint{3.564492in}{2.541257in}}%
\pgfpathclose%
\pgfusepath{fill}%
\end{pgfscope}%
\begin{pgfscope}%
\pgfpathrectangle{\pgfqpoint{0.017500in}{0.017500in}}{\pgfqpoint{6.218282in}{3.580372in}} %
\pgfusepath{clip}%
\pgfsetbuttcap%
\pgfsetroundjoin%
\definecolor{currentfill}{rgb}{0.763520,0.178667,0.193396}%
\pgfsetfillcolor{currentfill}%
\pgfsetlinewidth{0.000000pt}%
\definecolor{currentstroke}{rgb}{0.000000,0.000000,0.000000}%
\pgfsetstrokecolor{currentstroke}%
\pgfsetdash{}{0pt}%
\pgfpathmoveto{\pgfqpoint{3.564492in}{2.541257in}}%
\pgfpathlineto{\pgfqpoint{3.816789in}{2.677291in}}%
\pgfpathlineto{\pgfqpoint{3.564492in}{2.541257in}}%
\pgfpathclose%
\pgfusepath{fill}%
\end{pgfscope}%
\begin{pgfscope}%
\pgfpathrectangle{\pgfqpoint{0.017500in}{0.017500in}}{\pgfqpoint{6.218282in}{3.580372in}} %
\pgfusepath{clip}%
\pgfsetbuttcap%
\pgfsetroundjoin%
\definecolor{currentfill}{rgb}{0.763520,0.178667,0.193396}%
\pgfsetfillcolor{currentfill}%
\pgfsetlinewidth{0.000000pt}%
\definecolor{currentstroke}{rgb}{0.000000,0.000000,0.000000}%
\pgfsetstrokecolor{currentstroke}%
\pgfsetdash{}{0pt}%
\pgfpathmoveto{\pgfqpoint{3.564492in}{2.541257in}}%
\pgfpathlineto{\pgfqpoint{3.816789in}{2.677291in}}%
\pgfpathlineto{\pgfqpoint{3.564492in}{2.541257in}}%
\pgfpathclose%
\pgfusepath{fill}%
\end{pgfscope}%
\begin{pgfscope}%
\pgfpathrectangle{\pgfqpoint{0.017500in}{0.017500in}}{\pgfqpoint{6.218282in}{3.580372in}} %
\pgfusepath{clip}%
\pgfsetbuttcap%
\pgfsetroundjoin%
\definecolor{currentfill}{rgb}{0.763520,0.178667,0.193396}%
\pgfsetfillcolor{currentfill}%
\pgfsetlinewidth{0.000000pt}%
\definecolor{currentstroke}{rgb}{0.000000,0.000000,0.000000}%
\pgfsetstrokecolor{currentstroke}%
\pgfsetdash{}{0pt}%
\pgfpathmoveto{\pgfqpoint{3.564492in}{2.541257in}}%
\pgfpathlineto{\pgfqpoint{3.816789in}{2.677291in}}%
\pgfpathlineto{\pgfqpoint{3.564492in}{2.541257in}}%
\pgfpathclose%
\pgfusepath{fill}%
\end{pgfscope}%
\begin{pgfscope}%
\pgfpathrectangle{\pgfqpoint{0.017500in}{0.017500in}}{\pgfqpoint{6.218282in}{3.580372in}} %
\pgfusepath{clip}%
\pgfsetbuttcap%
\pgfsetroundjoin%
\definecolor{currentfill}{rgb}{0.763520,0.178667,0.193396}%
\pgfsetfillcolor{currentfill}%
\pgfsetlinewidth{0.000000pt}%
\definecolor{currentstroke}{rgb}{0.000000,0.000000,0.000000}%
\pgfsetstrokecolor{currentstroke}%
\pgfsetdash{}{0pt}%
\pgfpathmoveto{\pgfqpoint{3.564492in}{2.541257in}}%
\pgfpathlineto{\pgfqpoint{3.816789in}{2.677291in}}%
\pgfpathlineto{\pgfqpoint{3.564492in}{2.541257in}}%
\pgfpathclose%
\pgfusepath{fill}%
\end{pgfscope}%
\begin{pgfscope}%
\pgfpathrectangle{\pgfqpoint{0.017500in}{0.017500in}}{\pgfqpoint{6.218282in}{3.580372in}} %
\pgfusepath{clip}%
\pgfsetbuttcap%
\pgfsetroundjoin%
\definecolor{currentfill}{rgb}{0.763520,0.178667,0.193396}%
\pgfsetfillcolor{currentfill}%
\pgfsetlinewidth{0.000000pt}%
\definecolor{currentstroke}{rgb}{0.000000,0.000000,0.000000}%
\pgfsetstrokecolor{currentstroke}%
\pgfsetdash{}{0pt}%
\pgfpathmoveto{\pgfqpoint{3.564492in}{2.541257in}}%
\pgfpathlineto{\pgfqpoint{3.816789in}{2.677291in}}%
\pgfpathlineto{\pgfqpoint{3.564492in}{2.541257in}}%
\pgfpathclose%
\pgfusepath{fill}%
\end{pgfscope}%
\begin{pgfscope}%
\pgfpathrectangle{\pgfqpoint{0.017500in}{0.017500in}}{\pgfqpoint{6.218282in}{3.580372in}} %
\pgfusepath{clip}%
\pgfsetbuttcap%
\pgfsetroundjoin%
\definecolor{currentfill}{rgb}{0.763520,0.178667,0.193396}%
\pgfsetfillcolor{currentfill}%
\pgfsetlinewidth{0.000000pt}%
\definecolor{currentstroke}{rgb}{0.000000,0.000000,0.000000}%
\pgfsetstrokecolor{currentstroke}%
\pgfsetdash{}{0pt}%
\pgfpathmoveto{\pgfqpoint{3.564492in}{2.541257in}}%
\pgfpathlineto{\pgfqpoint{3.816789in}{2.677291in}}%
\pgfpathlineto{\pgfqpoint{3.564492in}{2.541257in}}%
\pgfpathclose%
\pgfusepath{fill}%
\end{pgfscope}%
\begin{pgfscope}%
\pgfpathrectangle{\pgfqpoint{0.017500in}{0.017500in}}{\pgfqpoint{6.218282in}{3.580372in}} %
\pgfusepath{clip}%
\pgfsetbuttcap%
\pgfsetroundjoin%
\definecolor{currentfill}{rgb}{0.763520,0.178667,0.193396}%
\pgfsetfillcolor{currentfill}%
\pgfsetlinewidth{0.000000pt}%
\definecolor{currentstroke}{rgb}{0.000000,0.000000,0.000000}%
\pgfsetstrokecolor{currentstroke}%
\pgfsetdash{}{0pt}%
\pgfpathmoveto{\pgfqpoint{3.564492in}{2.541257in}}%
\pgfpathlineto{\pgfqpoint{3.816789in}{2.677291in}}%
\pgfpathlineto{\pgfqpoint{3.564492in}{2.541257in}}%
\pgfpathclose%
\pgfusepath{fill}%
\end{pgfscope}%
\begin{pgfscope}%
\pgfpathrectangle{\pgfqpoint{0.017500in}{0.017500in}}{\pgfqpoint{6.218282in}{3.580372in}} %
\pgfusepath{clip}%
\pgfsetbuttcap%
\pgfsetroundjoin%
\definecolor{currentfill}{rgb}{0.763520,0.178667,0.193396}%
\pgfsetfillcolor{currentfill}%
\pgfsetlinewidth{0.000000pt}%
\definecolor{currentstroke}{rgb}{0.000000,0.000000,0.000000}%
\pgfsetstrokecolor{currentstroke}%
\pgfsetdash{}{0pt}%
\pgfpathmoveto{\pgfqpoint{3.564492in}{2.541257in}}%
\pgfpathlineto{\pgfqpoint{3.816789in}{2.677291in}}%
\pgfpathlineto{\pgfqpoint{3.564492in}{2.541257in}}%
\pgfpathclose%
\pgfusepath{fill}%
\end{pgfscope}%
\begin{pgfscope}%
\pgfpathrectangle{\pgfqpoint{0.017500in}{0.017500in}}{\pgfqpoint{6.218282in}{3.580372in}} %
\pgfusepath{clip}%
\pgfsetbuttcap%
\pgfsetroundjoin%
\definecolor{currentfill}{rgb}{0.763520,0.178667,0.193396}%
\pgfsetfillcolor{currentfill}%
\pgfsetlinewidth{0.000000pt}%
\definecolor{currentstroke}{rgb}{0.000000,0.000000,0.000000}%
\pgfsetstrokecolor{currentstroke}%
\pgfsetdash{}{0pt}%
\pgfpathmoveto{\pgfqpoint{3.564492in}{2.541257in}}%
\pgfpathlineto{\pgfqpoint{3.816789in}{2.677291in}}%
\pgfpathlineto{\pgfqpoint{3.564492in}{2.541257in}}%
\pgfpathclose%
\pgfusepath{fill}%
\end{pgfscope}%
\begin{pgfscope}%
\pgfpathrectangle{\pgfqpoint{0.017500in}{0.017500in}}{\pgfqpoint{6.218282in}{3.580372in}} %
\pgfusepath{clip}%
\pgfsetbuttcap%
\pgfsetroundjoin%
\definecolor{currentfill}{rgb}{0.763520,0.178667,0.193396}%
\pgfsetfillcolor{currentfill}%
\pgfsetlinewidth{0.000000pt}%
\definecolor{currentstroke}{rgb}{0.000000,0.000000,0.000000}%
\pgfsetstrokecolor{currentstroke}%
\pgfsetdash{}{0pt}%
\pgfpathmoveto{\pgfqpoint{3.564492in}{2.541257in}}%
\pgfpathlineto{\pgfqpoint{3.816789in}{2.677291in}}%
\pgfpathlineto{\pgfqpoint{3.564492in}{2.541257in}}%
\pgfpathclose%
\pgfusepath{fill}%
\end{pgfscope}%
\begin{pgfscope}%
\pgfpathrectangle{\pgfqpoint{0.017500in}{0.017500in}}{\pgfqpoint{6.218282in}{3.580372in}} %
\pgfusepath{clip}%
\pgfsetbuttcap%
\pgfsetroundjoin%
\definecolor{currentfill}{rgb}{0.763520,0.178667,0.193396}%
\pgfsetfillcolor{currentfill}%
\pgfsetlinewidth{0.000000pt}%
\definecolor{currentstroke}{rgb}{0.000000,0.000000,0.000000}%
\pgfsetstrokecolor{currentstroke}%
\pgfsetdash{}{0pt}%
\pgfpathmoveto{\pgfqpoint{3.564492in}{2.541257in}}%
\pgfpathlineto{\pgfqpoint{3.816789in}{2.677291in}}%
\pgfpathlineto{\pgfqpoint{3.564492in}{2.541257in}}%
\pgfpathclose%
\pgfusepath{fill}%
\end{pgfscope}%
\begin{pgfscope}%
\pgfpathrectangle{\pgfqpoint{0.017500in}{0.017500in}}{\pgfqpoint{6.218282in}{3.580372in}} %
\pgfusepath{clip}%
\pgfsetbuttcap%
\pgfsetroundjoin%
\definecolor{currentfill}{rgb}{0.763520,0.178667,0.193396}%
\pgfsetfillcolor{currentfill}%
\pgfsetlinewidth{0.000000pt}%
\definecolor{currentstroke}{rgb}{0.000000,0.000000,0.000000}%
\pgfsetstrokecolor{currentstroke}%
\pgfsetdash{}{0pt}%
\pgfpathmoveto{\pgfqpoint{3.564492in}{2.541257in}}%
\pgfpathlineto{\pgfqpoint{3.816789in}{2.677291in}}%
\pgfpathlineto{\pgfqpoint{3.564492in}{2.541257in}}%
\pgfpathclose%
\pgfusepath{fill}%
\end{pgfscope}%
\begin{pgfscope}%
\pgfpathrectangle{\pgfqpoint{0.017500in}{0.017500in}}{\pgfqpoint{6.218282in}{3.580372in}} %
\pgfusepath{clip}%
\pgfsetbuttcap%
\pgfsetroundjoin%
\definecolor{currentfill}{rgb}{0.763520,0.178667,0.193396}%
\pgfsetfillcolor{currentfill}%
\pgfsetlinewidth{0.000000pt}%
\definecolor{currentstroke}{rgb}{0.000000,0.000000,0.000000}%
\pgfsetstrokecolor{currentstroke}%
\pgfsetdash{}{0pt}%
\pgfpathmoveto{\pgfqpoint{3.564492in}{2.541257in}}%
\pgfpathlineto{\pgfqpoint{3.816789in}{2.677291in}}%
\pgfpathlineto{\pgfqpoint{3.564492in}{2.541257in}}%
\pgfpathclose%
\pgfusepath{fill}%
\end{pgfscope}%
\begin{pgfscope}%
\pgfpathrectangle{\pgfqpoint{0.017500in}{0.017500in}}{\pgfqpoint{6.218282in}{3.580372in}} %
\pgfusepath{clip}%
\pgfsetbuttcap%
\pgfsetroundjoin%
\definecolor{currentfill}{rgb}{0.763520,0.178667,0.193396}%
\pgfsetfillcolor{currentfill}%
\pgfsetlinewidth{0.000000pt}%
\definecolor{currentstroke}{rgb}{0.000000,0.000000,0.000000}%
\pgfsetstrokecolor{currentstroke}%
\pgfsetdash{}{0pt}%
\pgfpathmoveto{\pgfqpoint{3.564492in}{2.541257in}}%
\pgfpathlineto{\pgfqpoint{3.816789in}{2.677291in}}%
\pgfpathlineto{\pgfqpoint{3.564492in}{2.541257in}}%
\pgfpathclose%
\pgfusepath{fill}%
\end{pgfscope}%
\begin{pgfscope}%
\pgfpathrectangle{\pgfqpoint{0.017500in}{0.017500in}}{\pgfqpoint{6.218282in}{3.580372in}} %
\pgfusepath{clip}%
\pgfsetbuttcap%
\pgfsetroundjoin%
\definecolor{currentfill}{rgb}{0.763520,0.178667,0.193396}%
\pgfsetfillcolor{currentfill}%
\pgfsetlinewidth{0.000000pt}%
\definecolor{currentstroke}{rgb}{0.000000,0.000000,0.000000}%
\pgfsetstrokecolor{currentstroke}%
\pgfsetdash{}{0pt}%
\pgfpathmoveto{\pgfqpoint{3.564492in}{2.541257in}}%
\pgfpathlineto{\pgfqpoint{3.816789in}{2.677291in}}%
\pgfpathlineto{\pgfqpoint{3.564492in}{2.541257in}}%
\pgfpathclose%
\pgfusepath{fill}%
\end{pgfscope}%
\begin{pgfscope}%
\pgfpathrectangle{\pgfqpoint{0.017500in}{0.017500in}}{\pgfqpoint{6.218282in}{3.580372in}} %
\pgfusepath{clip}%
\pgfsetbuttcap%
\pgfsetroundjoin%
\definecolor{currentfill}{rgb}{0.763520,0.178667,0.193396}%
\pgfsetfillcolor{currentfill}%
\pgfsetlinewidth{0.000000pt}%
\definecolor{currentstroke}{rgb}{0.000000,0.000000,0.000000}%
\pgfsetstrokecolor{currentstroke}%
\pgfsetdash{}{0pt}%
\pgfpathmoveto{\pgfqpoint{3.564492in}{2.541257in}}%
\pgfpathlineto{\pgfqpoint{3.816789in}{2.677291in}}%
\pgfpathlineto{\pgfqpoint{3.564492in}{2.541257in}}%
\pgfpathclose%
\pgfusepath{fill}%
\end{pgfscope}%
\begin{pgfscope}%
\pgfpathrectangle{\pgfqpoint{0.017500in}{0.017500in}}{\pgfqpoint{6.218282in}{3.580372in}} %
\pgfusepath{clip}%
\pgfsetbuttcap%
\pgfsetroundjoin%
\definecolor{currentfill}{rgb}{0.763520,0.178667,0.193396}%
\pgfsetfillcolor{currentfill}%
\pgfsetlinewidth{0.000000pt}%
\definecolor{currentstroke}{rgb}{0.000000,0.000000,0.000000}%
\pgfsetstrokecolor{currentstroke}%
\pgfsetdash{}{0pt}%
\pgfpathmoveto{\pgfqpoint{3.564492in}{2.541257in}}%
\pgfpathlineto{\pgfqpoint{3.816789in}{2.677291in}}%
\pgfpathlineto{\pgfqpoint{3.564492in}{2.541257in}}%
\pgfpathclose%
\pgfusepath{fill}%
\end{pgfscope}%
\begin{pgfscope}%
\pgfpathrectangle{\pgfqpoint{0.017500in}{0.017500in}}{\pgfqpoint{6.218282in}{3.580372in}} %
\pgfusepath{clip}%
\pgfsetbuttcap%
\pgfsetroundjoin%
\definecolor{currentfill}{rgb}{0.763520,0.178667,0.193396}%
\pgfsetfillcolor{currentfill}%
\pgfsetlinewidth{0.000000pt}%
\definecolor{currentstroke}{rgb}{0.000000,0.000000,0.000000}%
\pgfsetstrokecolor{currentstroke}%
\pgfsetdash{}{0pt}%
\pgfpathmoveto{\pgfqpoint{3.564492in}{2.541257in}}%
\pgfpathlineto{\pgfqpoint{3.816789in}{2.677291in}}%
\pgfpathlineto{\pgfqpoint{3.564492in}{2.541257in}}%
\pgfpathclose%
\pgfusepath{fill}%
\end{pgfscope}%
\begin{pgfscope}%
\pgfpathrectangle{\pgfqpoint{0.017500in}{0.017500in}}{\pgfqpoint{6.218282in}{3.580372in}} %
\pgfusepath{clip}%
\pgfsetbuttcap%
\pgfsetroundjoin%
\definecolor{currentfill}{rgb}{0.763520,0.178667,0.193396}%
\pgfsetfillcolor{currentfill}%
\pgfsetlinewidth{0.000000pt}%
\definecolor{currentstroke}{rgb}{0.000000,0.000000,0.000000}%
\pgfsetstrokecolor{currentstroke}%
\pgfsetdash{}{0pt}%
\pgfpathmoveto{\pgfqpoint{3.564492in}{2.541257in}}%
\pgfpathlineto{\pgfqpoint{3.816789in}{2.677291in}}%
\pgfpathlineto{\pgfqpoint{3.564492in}{2.541257in}}%
\pgfpathclose%
\pgfusepath{fill}%
\end{pgfscope}%
\begin{pgfscope}%
\pgfpathrectangle{\pgfqpoint{0.017500in}{0.017500in}}{\pgfqpoint{6.218282in}{3.580372in}} %
\pgfusepath{clip}%
\pgfsetbuttcap%
\pgfsetroundjoin%
\definecolor{currentfill}{rgb}{0.899543,0.847500,0.817789}%
\pgfsetfillcolor{currentfill}%
\pgfsetlinewidth{0.000000pt}%
\definecolor{currentstroke}{rgb}{0.000000,0.000000,0.000000}%
\pgfsetstrokecolor{currentstroke}%
\pgfsetdash{}{0pt}%
\pgfpathmoveto{\pgfqpoint{4.964811in}{2.684961in}}%
\pgfpathlineto{\pgfqpoint{3.688351in}{0.641693in}}%
\pgfpathlineto{\pgfqpoint{4.964811in}{2.684961in}}%
\pgfpathclose%
\pgfusepath{fill}%
\end{pgfscope}%
\begin{pgfscope}%
\pgfpathrectangle{\pgfqpoint{0.017500in}{0.017500in}}{\pgfqpoint{6.218282in}{3.580372in}} %
\pgfusepath{clip}%
\pgfsetbuttcap%
\pgfsetroundjoin%
\definecolor{currentfill}{rgb}{0.899543,0.847500,0.817789}%
\pgfsetfillcolor{currentfill}%
\pgfsetlinewidth{0.000000pt}%
\definecolor{currentstroke}{rgb}{0.000000,0.000000,0.000000}%
\pgfsetstrokecolor{currentstroke}%
\pgfsetdash{}{0pt}%
\pgfpathmoveto{\pgfqpoint{4.964811in}{2.684961in}}%
\pgfpathlineto{\pgfqpoint{3.688351in}{0.641693in}}%
\pgfpathlineto{\pgfqpoint{4.964811in}{2.684961in}}%
\pgfpathclose%
\pgfusepath{fill}%
\end{pgfscope}%
\begin{pgfscope}%
\pgfpathrectangle{\pgfqpoint{0.017500in}{0.017500in}}{\pgfqpoint{6.218282in}{3.580372in}} %
\pgfusepath{clip}%
\pgfsetbuttcap%
\pgfsetroundjoin%
\definecolor{currentfill}{rgb}{0.899543,0.847500,0.817789}%
\pgfsetfillcolor{currentfill}%
\pgfsetlinewidth{0.000000pt}%
\definecolor{currentstroke}{rgb}{0.000000,0.000000,0.000000}%
\pgfsetstrokecolor{currentstroke}%
\pgfsetdash{}{0pt}%
\pgfpathmoveto{\pgfqpoint{4.964811in}{2.684961in}}%
\pgfpathlineto{\pgfqpoint{3.688351in}{0.641693in}}%
\pgfpathlineto{\pgfqpoint{4.964811in}{2.684961in}}%
\pgfpathclose%
\pgfusepath{fill}%
\end{pgfscope}%
\begin{pgfscope}%
\pgfpathrectangle{\pgfqpoint{0.017500in}{0.017500in}}{\pgfqpoint{6.218282in}{3.580372in}} %
\pgfusepath{clip}%
\pgfsetbuttcap%
\pgfsetroundjoin%
\definecolor{currentfill}{rgb}{0.899543,0.847500,0.817789}%
\pgfsetfillcolor{currentfill}%
\pgfsetlinewidth{0.000000pt}%
\definecolor{currentstroke}{rgb}{0.000000,0.000000,0.000000}%
\pgfsetstrokecolor{currentstroke}%
\pgfsetdash{}{0pt}%
\pgfpathmoveto{\pgfqpoint{4.964811in}{2.684961in}}%
\pgfpathlineto{\pgfqpoint{3.688351in}{0.641693in}}%
\pgfpathlineto{\pgfqpoint{4.964811in}{2.684961in}}%
\pgfpathclose%
\pgfusepath{fill}%
\end{pgfscope}%
\begin{pgfscope}%
\pgfpathrectangle{\pgfqpoint{0.017500in}{0.017500in}}{\pgfqpoint{6.218282in}{3.580372in}} %
\pgfusepath{clip}%
\pgfsetbuttcap%
\pgfsetroundjoin%
\definecolor{currentfill}{rgb}{0.899543,0.847500,0.817789}%
\pgfsetfillcolor{currentfill}%
\pgfsetlinewidth{0.000000pt}%
\definecolor{currentstroke}{rgb}{0.000000,0.000000,0.000000}%
\pgfsetstrokecolor{currentstroke}%
\pgfsetdash{}{0pt}%
\pgfpathmoveto{\pgfqpoint{4.964811in}{2.684961in}}%
\pgfpathlineto{\pgfqpoint{3.688351in}{0.641693in}}%
\pgfpathlineto{\pgfqpoint{4.964811in}{2.684961in}}%
\pgfpathclose%
\pgfusepath{fill}%
\end{pgfscope}%
\begin{pgfscope}%
\pgfpathrectangle{\pgfqpoint{0.017500in}{0.017500in}}{\pgfqpoint{6.218282in}{3.580372in}} %
\pgfusepath{clip}%
\pgfsetbuttcap%
\pgfsetroundjoin%
\definecolor{currentfill}{rgb}{0.899543,0.847500,0.817789}%
\pgfsetfillcolor{currentfill}%
\pgfsetlinewidth{0.000000pt}%
\definecolor{currentstroke}{rgb}{0.000000,0.000000,0.000000}%
\pgfsetstrokecolor{currentstroke}%
\pgfsetdash{}{0pt}%
\pgfpathmoveto{\pgfqpoint{4.964811in}{2.684961in}}%
\pgfpathlineto{\pgfqpoint{3.688351in}{0.641693in}}%
\pgfpathlineto{\pgfqpoint{4.964811in}{2.684961in}}%
\pgfpathclose%
\pgfusepath{fill}%
\end{pgfscope}%
\begin{pgfscope}%
\pgfpathrectangle{\pgfqpoint{0.017500in}{0.017500in}}{\pgfqpoint{6.218282in}{3.580372in}} %
\pgfusepath{clip}%
\pgfsetbuttcap%
\pgfsetroundjoin%
\definecolor{currentfill}{rgb}{0.899543,0.847500,0.817789}%
\pgfsetfillcolor{currentfill}%
\pgfsetlinewidth{0.000000pt}%
\definecolor{currentstroke}{rgb}{0.000000,0.000000,0.000000}%
\pgfsetstrokecolor{currentstroke}%
\pgfsetdash{}{0pt}%
\pgfpathmoveto{\pgfqpoint{4.964811in}{2.684961in}}%
\pgfpathlineto{\pgfqpoint{3.688351in}{0.641693in}}%
\pgfpathlineto{\pgfqpoint{4.964811in}{2.684961in}}%
\pgfpathclose%
\pgfusepath{fill}%
\end{pgfscope}%
\begin{pgfscope}%
\pgfpathrectangle{\pgfqpoint{0.017500in}{0.017500in}}{\pgfqpoint{6.218282in}{3.580372in}} %
\pgfusepath{clip}%
\pgfsetbuttcap%
\pgfsetroundjoin%
\definecolor{currentfill}{rgb}{0.899543,0.847500,0.817789}%
\pgfsetfillcolor{currentfill}%
\pgfsetlinewidth{0.000000pt}%
\definecolor{currentstroke}{rgb}{0.000000,0.000000,0.000000}%
\pgfsetstrokecolor{currentstroke}%
\pgfsetdash{}{0pt}%
\pgfpathmoveto{\pgfqpoint{4.964811in}{2.684961in}}%
\pgfpathlineto{\pgfqpoint{3.688351in}{0.641693in}}%
\pgfpathlineto{\pgfqpoint{4.964811in}{2.684961in}}%
\pgfpathclose%
\pgfusepath{fill}%
\end{pgfscope}%
\begin{pgfscope}%
\pgfpathrectangle{\pgfqpoint{0.017500in}{0.017500in}}{\pgfqpoint{6.218282in}{3.580372in}} %
\pgfusepath{clip}%
\pgfsetbuttcap%
\pgfsetroundjoin%
\definecolor{currentfill}{rgb}{0.899543,0.847500,0.817789}%
\pgfsetfillcolor{currentfill}%
\pgfsetlinewidth{0.000000pt}%
\definecolor{currentstroke}{rgb}{0.000000,0.000000,0.000000}%
\pgfsetstrokecolor{currentstroke}%
\pgfsetdash{}{0pt}%
\pgfpathmoveto{\pgfqpoint{4.964811in}{2.684961in}}%
\pgfpathlineto{\pgfqpoint{3.688351in}{0.641693in}}%
\pgfpathlineto{\pgfqpoint{4.964811in}{2.684961in}}%
\pgfpathclose%
\pgfusepath{fill}%
\end{pgfscope}%
\begin{pgfscope}%
\pgfpathrectangle{\pgfqpoint{0.017500in}{0.017500in}}{\pgfqpoint{6.218282in}{3.580372in}} %
\pgfusepath{clip}%
\pgfsetbuttcap%
\pgfsetroundjoin%
\definecolor{currentfill}{rgb}{0.899543,0.847500,0.817789}%
\pgfsetfillcolor{currentfill}%
\pgfsetlinewidth{0.000000pt}%
\definecolor{currentstroke}{rgb}{0.000000,0.000000,0.000000}%
\pgfsetstrokecolor{currentstroke}%
\pgfsetdash{}{0pt}%
\pgfpathmoveto{\pgfqpoint{4.964811in}{2.684961in}}%
\pgfpathlineto{\pgfqpoint{3.688351in}{0.641693in}}%
\pgfpathlineto{\pgfqpoint{4.964811in}{2.684961in}}%
\pgfpathclose%
\pgfusepath{fill}%
\end{pgfscope}%
\begin{pgfscope}%
\pgfpathrectangle{\pgfqpoint{0.017500in}{0.017500in}}{\pgfqpoint{6.218282in}{3.580372in}} %
\pgfusepath{clip}%
\pgfsetbuttcap%
\pgfsetroundjoin%
\definecolor{currentfill}{rgb}{0.899543,0.847500,0.817789}%
\pgfsetfillcolor{currentfill}%
\pgfsetlinewidth{0.000000pt}%
\definecolor{currentstroke}{rgb}{0.000000,0.000000,0.000000}%
\pgfsetstrokecolor{currentstroke}%
\pgfsetdash{}{0pt}%
\pgfpathmoveto{\pgfqpoint{4.964811in}{2.684961in}}%
\pgfpathlineto{\pgfqpoint{3.688351in}{0.641693in}}%
\pgfpathlineto{\pgfqpoint{4.964811in}{2.684961in}}%
\pgfpathclose%
\pgfusepath{fill}%
\end{pgfscope}%
\begin{pgfscope}%
\pgfpathrectangle{\pgfqpoint{0.017500in}{0.017500in}}{\pgfqpoint{6.218282in}{3.580372in}} %
\pgfusepath{clip}%
\pgfsetbuttcap%
\pgfsetroundjoin%
\definecolor{currentfill}{rgb}{0.899543,0.847500,0.817789}%
\pgfsetfillcolor{currentfill}%
\pgfsetlinewidth{0.000000pt}%
\definecolor{currentstroke}{rgb}{0.000000,0.000000,0.000000}%
\pgfsetstrokecolor{currentstroke}%
\pgfsetdash{}{0pt}%
\pgfpathmoveto{\pgfqpoint{4.964811in}{2.684961in}}%
\pgfpathlineto{\pgfqpoint{3.688351in}{0.641693in}}%
\pgfpathlineto{\pgfqpoint{4.964811in}{2.684961in}}%
\pgfpathclose%
\pgfusepath{fill}%
\end{pgfscope}%
\begin{pgfscope}%
\pgfpathrectangle{\pgfqpoint{0.017500in}{0.017500in}}{\pgfqpoint{6.218282in}{3.580372in}} %
\pgfusepath{clip}%
\pgfsetbuttcap%
\pgfsetroundjoin%
\definecolor{currentfill}{rgb}{0.899543,0.847500,0.817789}%
\pgfsetfillcolor{currentfill}%
\pgfsetlinewidth{0.000000pt}%
\definecolor{currentstroke}{rgb}{0.000000,0.000000,0.000000}%
\pgfsetstrokecolor{currentstroke}%
\pgfsetdash{}{0pt}%
\pgfpathmoveto{\pgfqpoint{4.964811in}{2.684961in}}%
\pgfpathlineto{\pgfqpoint{3.688351in}{0.641693in}}%
\pgfpathlineto{\pgfqpoint{4.964811in}{2.684961in}}%
\pgfpathclose%
\pgfusepath{fill}%
\end{pgfscope}%
\begin{pgfscope}%
\pgfpathrectangle{\pgfqpoint{0.017500in}{0.017500in}}{\pgfqpoint{6.218282in}{3.580372in}} %
\pgfusepath{clip}%
\pgfsetbuttcap%
\pgfsetroundjoin%
\definecolor{currentfill}{rgb}{0.899543,0.847500,0.817789}%
\pgfsetfillcolor{currentfill}%
\pgfsetlinewidth{0.000000pt}%
\definecolor{currentstroke}{rgb}{0.000000,0.000000,0.000000}%
\pgfsetstrokecolor{currentstroke}%
\pgfsetdash{}{0pt}%
\pgfpathmoveto{\pgfqpoint{4.964811in}{2.684961in}}%
\pgfpathlineto{\pgfqpoint{3.688351in}{0.641693in}}%
\pgfpathlineto{\pgfqpoint{4.964811in}{2.684961in}}%
\pgfpathclose%
\pgfusepath{fill}%
\end{pgfscope}%
\begin{pgfscope}%
\pgfpathrectangle{\pgfqpoint{0.017500in}{0.017500in}}{\pgfqpoint{6.218282in}{3.580372in}} %
\pgfusepath{clip}%
\pgfsetbuttcap%
\pgfsetroundjoin%
\definecolor{currentfill}{rgb}{0.899543,0.847500,0.817789}%
\pgfsetfillcolor{currentfill}%
\pgfsetlinewidth{0.000000pt}%
\definecolor{currentstroke}{rgb}{0.000000,0.000000,0.000000}%
\pgfsetstrokecolor{currentstroke}%
\pgfsetdash{}{0pt}%
\pgfpathmoveto{\pgfqpoint{4.964811in}{2.684961in}}%
\pgfpathlineto{\pgfqpoint{3.688351in}{0.641693in}}%
\pgfpathlineto{\pgfqpoint{4.964811in}{2.684961in}}%
\pgfpathclose%
\pgfusepath{fill}%
\end{pgfscope}%
\begin{pgfscope}%
\pgfpathrectangle{\pgfqpoint{0.017500in}{0.017500in}}{\pgfqpoint{6.218282in}{3.580372in}} %
\pgfusepath{clip}%
\pgfsetbuttcap%
\pgfsetroundjoin%
\definecolor{currentfill}{rgb}{0.899543,0.847500,0.817789}%
\pgfsetfillcolor{currentfill}%
\pgfsetlinewidth{0.000000pt}%
\definecolor{currentstroke}{rgb}{0.000000,0.000000,0.000000}%
\pgfsetstrokecolor{currentstroke}%
\pgfsetdash{}{0pt}%
\pgfpathmoveto{\pgfqpoint{4.964811in}{2.684961in}}%
\pgfpathlineto{\pgfqpoint{3.688351in}{0.641693in}}%
\pgfpathlineto{\pgfqpoint{4.964811in}{2.684961in}}%
\pgfpathclose%
\pgfusepath{fill}%
\end{pgfscope}%
\begin{pgfscope}%
\pgfpathrectangle{\pgfqpoint{0.017500in}{0.017500in}}{\pgfqpoint{6.218282in}{3.580372in}} %
\pgfusepath{clip}%
\pgfsetbuttcap%
\pgfsetroundjoin%
\definecolor{currentfill}{rgb}{0.899543,0.847500,0.817789}%
\pgfsetfillcolor{currentfill}%
\pgfsetlinewidth{0.000000pt}%
\definecolor{currentstroke}{rgb}{0.000000,0.000000,0.000000}%
\pgfsetstrokecolor{currentstroke}%
\pgfsetdash{}{0pt}%
\pgfpathmoveto{\pgfqpoint{4.964811in}{2.684961in}}%
\pgfpathlineto{\pgfqpoint{3.688351in}{0.641693in}}%
\pgfpathlineto{\pgfqpoint{4.964811in}{2.684961in}}%
\pgfpathclose%
\pgfusepath{fill}%
\end{pgfscope}%
\begin{pgfscope}%
\pgfpathrectangle{\pgfqpoint{0.017500in}{0.017500in}}{\pgfqpoint{6.218282in}{3.580372in}} %
\pgfusepath{clip}%
\pgfsetbuttcap%
\pgfsetroundjoin%
\definecolor{currentfill}{rgb}{0.899543,0.847500,0.817789}%
\pgfsetfillcolor{currentfill}%
\pgfsetlinewidth{0.000000pt}%
\definecolor{currentstroke}{rgb}{0.000000,0.000000,0.000000}%
\pgfsetstrokecolor{currentstroke}%
\pgfsetdash{}{0pt}%
\pgfpathmoveto{\pgfqpoint{4.964811in}{2.684961in}}%
\pgfpathlineto{\pgfqpoint{3.688351in}{0.641693in}}%
\pgfpathlineto{\pgfqpoint{4.964811in}{2.684961in}}%
\pgfpathclose%
\pgfusepath{fill}%
\end{pgfscope}%
\begin{pgfscope}%
\pgfpathrectangle{\pgfqpoint{0.017500in}{0.017500in}}{\pgfqpoint{6.218282in}{3.580372in}} %
\pgfusepath{clip}%
\pgfsetbuttcap%
\pgfsetroundjoin%
\definecolor{currentfill}{rgb}{0.899543,0.847500,0.817789}%
\pgfsetfillcolor{currentfill}%
\pgfsetlinewidth{0.000000pt}%
\definecolor{currentstroke}{rgb}{0.000000,0.000000,0.000000}%
\pgfsetstrokecolor{currentstroke}%
\pgfsetdash{}{0pt}%
\pgfpathmoveto{\pgfqpoint{4.964811in}{2.684961in}}%
\pgfpathlineto{\pgfqpoint{3.688351in}{0.641693in}}%
\pgfpathlineto{\pgfqpoint{4.964811in}{2.684961in}}%
\pgfpathclose%
\pgfusepath{fill}%
\end{pgfscope}%
\begin{pgfscope}%
\pgfpathrectangle{\pgfqpoint{0.017500in}{0.017500in}}{\pgfqpoint{6.218282in}{3.580372in}} %
\pgfusepath{clip}%
\pgfsetbuttcap%
\pgfsetroundjoin%
\definecolor{currentfill}{rgb}{0.899543,0.847500,0.817789}%
\pgfsetfillcolor{currentfill}%
\pgfsetlinewidth{0.000000pt}%
\definecolor{currentstroke}{rgb}{0.000000,0.000000,0.000000}%
\pgfsetstrokecolor{currentstroke}%
\pgfsetdash{}{0pt}%
\pgfpathmoveto{\pgfqpoint{4.964811in}{2.684961in}}%
\pgfpathlineto{\pgfqpoint{3.688351in}{0.641693in}}%
\pgfpathlineto{\pgfqpoint{4.964811in}{2.684961in}}%
\pgfpathclose%
\pgfusepath{fill}%
\end{pgfscope}%
\begin{pgfscope}%
\pgfpathrectangle{\pgfqpoint{0.017500in}{0.017500in}}{\pgfqpoint{6.218282in}{3.580372in}} %
\pgfusepath{clip}%
\pgfsetbuttcap%
\pgfsetroundjoin%
\definecolor{currentfill}{rgb}{0.899543,0.847500,0.817789}%
\pgfsetfillcolor{currentfill}%
\pgfsetlinewidth{0.000000pt}%
\definecolor{currentstroke}{rgb}{0.000000,0.000000,0.000000}%
\pgfsetstrokecolor{currentstroke}%
\pgfsetdash{}{0pt}%
\pgfpathmoveto{\pgfqpoint{4.964811in}{2.684961in}}%
\pgfpathlineto{\pgfqpoint{3.688351in}{0.641693in}}%
\pgfpathlineto{\pgfqpoint{4.964811in}{2.684961in}}%
\pgfpathclose%
\pgfusepath{fill}%
\end{pgfscope}%
\begin{pgfscope}%
\pgfpathrectangle{\pgfqpoint{0.017500in}{0.017500in}}{\pgfqpoint{6.218282in}{3.580372in}} %
\pgfusepath{clip}%
\pgfsetbuttcap%
\pgfsetroundjoin%
\definecolor{currentfill}{rgb}{0.899543,0.847500,0.817789}%
\pgfsetfillcolor{currentfill}%
\pgfsetlinewidth{0.000000pt}%
\definecolor{currentstroke}{rgb}{0.000000,0.000000,0.000000}%
\pgfsetstrokecolor{currentstroke}%
\pgfsetdash{}{0pt}%
\pgfpathmoveto{\pgfqpoint{4.964811in}{2.684961in}}%
\pgfpathlineto{\pgfqpoint{3.688351in}{0.641693in}}%
\pgfpathlineto{\pgfqpoint{4.964811in}{2.684961in}}%
\pgfpathclose%
\pgfusepath{fill}%
\end{pgfscope}%
\begin{pgfscope}%
\pgfpathrectangle{\pgfqpoint{0.017500in}{0.017500in}}{\pgfqpoint{6.218282in}{3.580372in}} %
\pgfusepath{clip}%
\pgfsetbuttcap%
\pgfsetroundjoin%
\definecolor{currentfill}{rgb}{0.899543,0.847500,0.817789}%
\pgfsetfillcolor{currentfill}%
\pgfsetlinewidth{0.000000pt}%
\definecolor{currentstroke}{rgb}{0.000000,0.000000,0.000000}%
\pgfsetstrokecolor{currentstroke}%
\pgfsetdash{}{0pt}%
\pgfpathmoveto{\pgfqpoint{4.964811in}{2.684961in}}%
\pgfpathlineto{\pgfqpoint{3.688351in}{0.641693in}}%
\pgfpathlineto{\pgfqpoint{4.964811in}{2.684961in}}%
\pgfpathclose%
\pgfusepath{fill}%
\end{pgfscope}%
\begin{pgfscope}%
\pgfpathrectangle{\pgfqpoint{0.017500in}{0.017500in}}{\pgfqpoint{6.218282in}{3.580372in}} %
\pgfusepath{clip}%
\pgfsetbuttcap%
\pgfsetroundjoin%
\definecolor{currentfill}{rgb}{0.899543,0.847500,0.817789}%
\pgfsetfillcolor{currentfill}%
\pgfsetlinewidth{0.000000pt}%
\definecolor{currentstroke}{rgb}{0.000000,0.000000,0.000000}%
\pgfsetstrokecolor{currentstroke}%
\pgfsetdash{}{0pt}%
\pgfpathmoveto{\pgfqpoint{4.964811in}{2.684961in}}%
\pgfpathlineto{\pgfqpoint{3.688351in}{0.641693in}}%
\pgfpathlineto{\pgfqpoint{4.964811in}{2.684961in}}%
\pgfpathclose%
\pgfusepath{fill}%
\end{pgfscope}%
\begin{pgfscope}%
\pgfpathrectangle{\pgfqpoint{0.017500in}{0.017500in}}{\pgfqpoint{6.218282in}{3.580372in}} %
\pgfusepath{clip}%
\pgfsetbuttcap%
\pgfsetroundjoin%
\definecolor{currentfill}{rgb}{0.899543,0.847500,0.817789}%
\pgfsetfillcolor{currentfill}%
\pgfsetlinewidth{0.000000pt}%
\definecolor{currentstroke}{rgb}{0.000000,0.000000,0.000000}%
\pgfsetstrokecolor{currentstroke}%
\pgfsetdash{}{0pt}%
\pgfpathmoveto{\pgfqpoint{4.964811in}{2.684961in}}%
\pgfpathlineto{\pgfqpoint{3.688351in}{0.641693in}}%
\pgfpathlineto{\pgfqpoint{4.964811in}{2.684961in}}%
\pgfpathclose%
\pgfusepath{fill}%
\end{pgfscope}%
\begin{pgfscope}%
\pgfpathrectangle{\pgfqpoint{0.017500in}{0.017500in}}{\pgfqpoint{6.218282in}{3.580372in}} %
\pgfusepath{clip}%
\pgfsetbuttcap%
\pgfsetroundjoin%
\definecolor{currentfill}{rgb}{0.884643,0.410017,0.322507}%
\pgfsetfillcolor{currentfill}%
\pgfsetlinewidth{0.000000pt}%
\definecolor{currentstroke}{rgb}{0.000000,0.000000,0.000000}%
\pgfsetstrokecolor{currentstroke}%
\pgfsetdash{}{0pt}%
\pgfpathmoveto{\pgfqpoint{1.701778in}{2.375902in}}%
\pgfpathlineto{\pgfqpoint{2.143341in}{2.291288in}}%
\pgfpathlineto{\pgfqpoint{2.410872in}{2.508083in}}%
\pgfpathlineto{\pgfqpoint{1.971633in}{2.586781in}}%
\pgfpathlineto{\pgfqpoint{1.701778in}{2.375902in}}%
\pgfpathclose%
\pgfusepath{fill}%
\end{pgfscope}%
\begin{pgfscope}%
\pgfpathrectangle{\pgfqpoint{0.017500in}{0.017500in}}{\pgfqpoint{6.218282in}{3.580372in}} %
\pgfusepath{clip}%
\pgfsetbuttcap%
\pgfsetroundjoin%
\definecolor{currentfill}{rgb}{0.884643,0.410017,0.322507}%
\pgfsetfillcolor{currentfill}%
\pgfsetlinewidth{0.000000pt}%
\definecolor{currentstroke}{rgb}{0.000000,0.000000,0.000000}%
\pgfsetstrokecolor{currentstroke}%
\pgfsetdash{}{0pt}%
\pgfpathmoveto{\pgfqpoint{1.701778in}{2.375902in}}%
\pgfpathlineto{\pgfqpoint{2.143341in}{2.291288in}}%
\pgfpathlineto{\pgfqpoint{2.410872in}{2.508083in}}%
\pgfpathlineto{\pgfqpoint{1.971633in}{2.586781in}}%
\pgfpathlineto{\pgfqpoint{1.701778in}{2.375902in}}%
\pgfpathclose%
\pgfusepath{fill}%
\end{pgfscope}%
\begin{pgfscope}%
\pgfpathrectangle{\pgfqpoint{0.017500in}{0.017500in}}{\pgfqpoint{6.218282in}{3.580372in}} %
\pgfusepath{clip}%
\pgfsetbuttcap%
\pgfsetroundjoin%
\definecolor{currentfill}{rgb}{0.884643,0.410017,0.322507}%
\pgfsetfillcolor{currentfill}%
\pgfsetlinewidth{0.000000pt}%
\definecolor{currentstroke}{rgb}{0.000000,0.000000,0.000000}%
\pgfsetstrokecolor{currentstroke}%
\pgfsetdash{}{0pt}%
\pgfpathmoveto{\pgfqpoint{1.701778in}{2.375902in}}%
\pgfpathlineto{\pgfqpoint{2.143341in}{2.291288in}}%
\pgfpathlineto{\pgfqpoint{2.410872in}{2.508083in}}%
\pgfpathlineto{\pgfqpoint{1.971633in}{2.586781in}}%
\pgfpathlineto{\pgfqpoint{1.701778in}{2.375902in}}%
\pgfpathclose%
\pgfusepath{fill}%
\end{pgfscope}%
\begin{pgfscope}%
\pgfpathrectangle{\pgfqpoint{0.017500in}{0.017500in}}{\pgfqpoint{6.218282in}{3.580372in}} %
\pgfusepath{clip}%
\pgfsetbuttcap%
\pgfsetroundjoin%
\definecolor{currentfill}{rgb}{0.884643,0.410017,0.322507}%
\pgfsetfillcolor{currentfill}%
\pgfsetlinewidth{0.000000pt}%
\definecolor{currentstroke}{rgb}{0.000000,0.000000,0.000000}%
\pgfsetstrokecolor{currentstroke}%
\pgfsetdash{}{0pt}%
\pgfpathmoveto{\pgfqpoint{1.701778in}{2.375902in}}%
\pgfpathlineto{\pgfqpoint{2.143341in}{2.291288in}}%
\pgfpathlineto{\pgfqpoint{2.410872in}{2.508083in}}%
\pgfpathlineto{\pgfqpoint{1.971633in}{2.586781in}}%
\pgfpathlineto{\pgfqpoint{1.701778in}{2.375902in}}%
\pgfpathclose%
\pgfusepath{fill}%
\end{pgfscope}%
\begin{pgfscope}%
\pgfpathrectangle{\pgfqpoint{0.017500in}{0.017500in}}{\pgfqpoint{6.218282in}{3.580372in}} %
\pgfusepath{clip}%
\pgfsetbuttcap%
\pgfsetroundjoin%
\definecolor{currentfill}{rgb}{0.884643,0.410017,0.322507}%
\pgfsetfillcolor{currentfill}%
\pgfsetlinewidth{0.000000pt}%
\definecolor{currentstroke}{rgb}{0.000000,0.000000,0.000000}%
\pgfsetstrokecolor{currentstroke}%
\pgfsetdash{}{0pt}%
\pgfpathmoveto{\pgfqpoint{1.701778in}{2.375902in}}%
\pgfpathlineto{\pgfqpoint{2.143341in}{2.291288in}}%
\pgfpathlineto{\pgfqpoint{2.410872in}{2.508083in}}%
\pgfpathlineto{\pgfqpoint{1.971633in}{2.586781in}}%
\pgfpathlineto{\pgfqpoint{1.701778in}{2.375902in}}%
\pgfpathclose%
\pgfusepath{fill}%
\end{pgfscope}%
\begin{pgfscope}%
\pgfpathrectangle{\pgfqpoint{0.017500in}{0.017500in}}{\pgfqpoint{6.218282in}{3.580372in}} %
\pgfusepath{clip}%
\pgfsetbuttcap%
\pgfsetroundjoin%
\definecolor{currentfill}{rgb}{0.884643,0.410017,0.322507}%
\pgfsetfillcolor{currentfill}%
\pgfsetlinewidth{0.000000pt}%
\definecolor{currentstroke}{rgb}{0.000000,0.000000,0.000000}%
\pgfsetstrokecolor{currentstroke}%
\pgfsetdash{}{0pt}%
\pgfpathmoveto{\pgfqpoint{1.701778in}{2.375902in}}%
\pgfpathlineto{\pgfqpoint{2.143341in}{2.291288in}}%
\pgfpathlineto{\pgfqpoint{2.410872in}{2.508083in}}%
\pgfpathlineto{\pgfqpoint{1.971633in}{2.586781in}}%
\pgfpathlineto{\pgfqpoint{1.701778in}{2.375902in}}%
\pgfpathclose%
\pgfusepath{fill}%
\end{pgfscope}%
\begin{pgfscope}%
\pgfpathrectangle{\pgfqpoint{0.017500in}{0.017500in}}{\pgfqpoint{6.218282in}{3.580372in}} %
\pgfusepath{clip}%
\pgfsetbuttcap%
\pgfsetroundjoin%
\definecolor{currentfill}{rgb}{0.758112,0.168122,0.188827}%
\pgfsetfillcolor{currentfill}%
\pgfsetlinewidth{0.000000pt}%
\definecolor{currentstroke}{rgb}{0.000000,0.000000,0.000000}%
\pgfsetstrokecolor{currentstroke}%
\pgfsetdash{}{0pt}%
\pgfpathmoveto{\pgfqpoint{3.564492in}{2.541257in}}%
\pgfpathlineto{\pgfqpoint{4.018447in}{2.468167in}}%
\pgfpathlineto{\pgfqpoint{4.267628in}{2.613520in}}%
\pgfpathlineto{\pgfqpoint{3.816789in}{2.677291in}}%
\pgfpathlineto{\pgfqpoint{3.564492in}{2.541257in}}%
\pgfpathclose%
\pgfusepath{fill}%
\end{pgfscope}%
\begin{pgfscope}%
\pgfpathrectangle{\pgfqpoint{0.017500in}{0.017500in}}{\pgfqpoint{6.218282in}{3.580372in}} %
\pgfusepath{clip}%
\pgfsetbuttcap%
\pgfsetroundjoin%
\definecolor{currentfill}{rgb}{0.758112,0.168122,0.188827}%
\pgfsetfillcolor{currentfill}%
\pgfsetlinewidth{0.000000pt}%
\definecolor{currentstroke}{rgb}{0.000000,0.000000,0.000000}%
\pgfsetstrokecolor{currentstroke}%
\pgfsetdash{}{0pt}%
\pgfpathmoveto{\pgfqpoint{3.564492in}{2.541257in}}%
\pgfpathlineto{\pgfqpoint{4.018447in}{2.468167in}}%
\pgfpathlineto{\pgfqpoint{4.267628in}{2.613520in}}%
\pgfpathlineto{\pgfqpoint{3.816789in}{2.677291in}}%
\pgfpathlineto{\pgfqpoint{3.564492in}{2.541257in}}%
\pgfpathclose%
\pgfusepath{fill}%
\end{pgfscope}%
\begin{pgfscope}%
\pgfpathrectangle{\pgfqpoint{0.017500in}{0.017500in}}{\pgfqpoint{6.218282in}{3.580372in}} %
\pgfusepath{clip}%
\pgfsetbuttcap%
\pgfsetroundjoin%
\definecolor{currentfill}{rgb}{0.758112,0.168122,0.188827}%
\pgfsetfillcolor{currentfill}%
\pgfsetlinewidth{0.000000pt}%
\definecolor{currentstroke}{rgb}{0.000000,0.000000,0.000000}%
\pgfsetstrokecolor{currentstroke}%
\pgfsetdash{}{0pt}%
\pgfpathmoveto{\pgfqpoint{3.564492in}{2.541257in}}%
\pgfpathlineto{\pgfqpoint{4.018447in}{2.468167in}}%
\pgfpathlineto{\pgfqpoint{4.267628in}{2.613520in}}%
\pgfpathlineto{\pgfqpoint{3.816789in}{2.677291in}}%
\pgfpathlineto{\pgfqpoint{3.564492in}{2.541257in}}%
\pgfpathclose%
\pgfusepath{fill}%
\end{pgfscope}%
\begin{pgfscope}%
\pgfpathrectangle{\pgfqpoint{0.017500in}{0.017500in}}{\pgfqpoint{6.218282in}{3.580372in}} %
\pgfusepath{clip}%
\pgfsetbuttcap%
\pgfsetroundjoin%
\definecolor{currentfill}{rgb}{0.758112,0.168122,0.188827}%
\pgfsetfillcolor{currentfill}%
\pgfsetlinewidth{0.000000pt}%
\definecolor{currentstroke}{rgb}{0.000000,0.000000,0.000000}%
\pgfsetstrokecolor{currentstroke}%
\pgfsetdash{}{0pt}%
\pgfpathmoveto{\pgfqpoint{3.564492in}{2.541257in}}%
\pgfpathlineto{\pgfqpoint{4.018447in}{2.468167in}}%
\pgfpathlineto{\pgfqpoint{4.267628in}{2.613520in}}%
\pgfpathlineto{\pgfqpoint{3.816789in}{2.677291in}}%
\pgfpathlineto{\pgfqpoint{3.564492in}{2.541257in}}%
\pgfpathclose%
\pgfusepath{fill}%
\end{pgfscope}%
\begin{pgfscope}%
\pgfpathrectangle{\pgfqpoint{0.017500in}{0.017500in}}{\pgfqpoint{6.218282in}{3.580372in}} %
\pgfusepath{clip}%
\pgfsetbuttcap%
\pgfsetroundjoin%
\definecolor{currentfill}{rgb}{0.758112,0.168122,0.188827}%
\pgfsetfillcolor{currentfill}%
\pgfsetlinewidth{0.000000pt}%
\definecolor{currentstroke}{rgb}{0.000000,0.000000,0.000000}%
\pgfsetstrokecolor{currentstroke}%
\pgfsetdash{}{0pt}%
\pgfpathmoveto{\pgfqpoint{3.564492in}{2.541257in}}%
\pgfpathlineto{\pgfqpoint{4.018447in}{2.468167in}}%
\pgfpathlineto{\pgfqpoint{4.267628in}{2.613520in}}%
\pgfpathlineto{\pgfqpoint{3.816789in}{2.677291in}}%
\pgfpathlineto{\pgfqpoint{3.564492in}{2.541257in}}%
\pgfpathclose%
\pgfusepath{fill}%
\end{pgfscope}%
\begin{pgfscope}%
\pgfpathrectangle{\pgfqpoint{0.017500in}{0.017500in}}{\pgfqpoint{6.218282in}{3.580372in}} %
\pgfusepath{clip}%
\pgfsetbuttcap%
\pgfsetroundjoin%
\definecolor{currentfill}{rgb}{0.758112,0.168122,0.188827}%
\pgfsetfillcolor{currentfill}%
\pgfsetlinewidth{0.000000pt}%
\definecolor{currentstroke}{rgb}{0.000000,0.000000,0.000000}%
\pgfsetstrokecolor{currentstroke}%
\pgfsetdash{}{0pt}%
\pgfpathmoveto{\pgfqpoint{3.564492in}{2.541257in}}%
\pgfpathlineto{\pgfqpoint{4.018447in}{2.468167in}}%
\pgfpathlineto{\pgfqpoint{4.267628in}{2.613520in}}%
\pgfpathlineto{\pgfqpoint{3.816789in}{2.677291in}}%
\pgfpathlineto{\pgfqpoint{3.564492in}{2.541257in}}%
\pgfpathclose%
\pgfusepath{fill}%
\end{pgfscope}%
\begin{pgfscope}%
\pgfpathrectangle{\pgfqpoint{0.017500in}{0.017500in}}{\pgfqpoint{6.218282in}{3.580372in}} %
\pgfusepath{clip}%
\pgfsetbuttcap%
\pgfsetroundjoin%
\definecolor{currentfill}{rgb}{0.723315,0.068898,0.162989}%
\pgfsetfillcolor{currentfill}%
\pgfsetlinewidth{0.000000pt}%
\definecolor{currentstroke}{rgb}{0.000000,0.000000,0.000000}%
\pgfsetstrokecolor{currentstroke}%
\pgfsetdash{}{0pt}%
\pgfpathmoveto{\pgfqpoint{4.724651in}{2.549289in}}%
\pgfpathlineto{\pgfqpoint{4.964811in}{2.684961in}}%
\pgfpathlineto{\pgfqpoint{4.724651in}{2.549289in}}%
\pgfpathclose%
\pgfusepath{fill}%
\end{pgfscope}%
\begin{pgfscope}%
\pgfpathrectangle{\pgfqpoint{0.017500in}{0.017500in}}{\pgfqpoint{6.218282in}{3.580372in}} %
\pgfusepath{clip}%
\pgfsetbuttcap%
\pgfsetroundjoin%
\definecolor{currentfill}{rgb}{0.723315,0.068898,0.162989}%
\pgfsetfillcolor{currentfill}%
\pgfsetlinewidth{0.000000pt}%
\definecolor{currentstroke}{rgb}{0.000000,0.000000,0.000000}%
\pgfsetstrokecolor{currentstroke}%
\pgfsetdash{}{0pt}%
\pgfpathmoveto{\pgfqpoint{4.724651in}{2.549289in}}%
\pgfpathlineto{\pgfqpoint{4.964811in}{2.684961in}}%
\pgfpathlineto{\pgfqpoint{4.724651in}{2.549289in}}%
\pgfpathclose%
\pgfusepath{fill}%
\end{pgfscope}%
\begin{pgfscope}%
\pgfpathrectangle{\pgfqpoint{0.017500in}{0.017500in}}{\pgfqpoint{6.218282in}{3.580372in}} %
\pgfusepath{clip}%
\pgfsetbuttcap%
\pgfsetroundjoin%
\definecolor{currentfill}{rgb}{0.723315,0.068898,0.162989}%
\pgfsetfillcolor{currentfill}%
\pgfsetlinewidth{0.000000pt}%
\definecolor{currentstroke}{rgb}{0.000000,0.000000,0.000000}%
\pgfsetstrokecolor{currentstroke}%
\pgfsetdash{}{0pt}%
\pgfpathmoveto{\pgfqpoint{4.724651in}{2.549289in}}%
\pgfpathlineto{\pgfqpoint{4.964811in}{2.684961in}}%
\pgfpathlineto{\pgfqpoint{4.724651in}{2.549289in}}%
\pgfpathclose%
\pgfusepath{fill}%
\end{pgfscope}%
\begin{pgfscope}%
\pgfpathrectangle{\pgfqpoint{0.017500in}{0.017500in}}{\pgfqpoint{6.218282in}{3.580372in}} %
\pgfusepath{clip}%
\pgfsetbuttcap%
\pgfsetroundjoin%
\definecolor{currentfill}{rgb}{0.723315,0.068898,0.162989}%
\pgfsetfillcolor{currentfill}%
\pgfsetlinewidth{0.000000pt}%
\definecolor{currentstroke}{rgb}{0.000000,0.000000,0.000000}%
\pgfsetstrokecolor{currentstroke}%
\pgfsetdash{}{0pt}%
\pgfpathmoveto{\pgfqpoint{4.724651in}{2.549289in}}%
\pgfpathlineto{\pgfqpoint{4.964811in}{2.684961in}}%
\pgfpathlineto{\pgfqpoint{4.724651in}{2.549289in}}%
\pgfpathclose%
\pgfusepath{fill}%
\end{pgfscope}%
\begin{pgfscope}%
\pgfpathrectangle{\pgfqpoint{0.017500in}{0.017500in}}{\pgfqpoint{6.218282in}{3.580372in}} %
\pgfusepath{clip}%
\pgfsetbuttcap%
\pgfsetroundjoin%
\definecolor{currentfill}{rgb}{0.723315,0.068898,0.162989}%
\pgfsetfillcolor{currentfill}%
\pgfsetlinewidth{0.000000pt}%
\definecolor{currentstroke}{rgb}{0.000000,0.000000,0.000000}%
\pgfsetstrokecolor{currentstroke}%
\pgfsetdash{}{0pt}%
\pgfpathmoveto{\pgfqpoint{4.724651in}{2.549289in}}%
\pgfpathlineto{\pgfqpoint{4.964811in}{2.684961in}}%
\pgfpathlineto{\pgfqpoint{4.724651in}{2.549289in}}%
\pgfpathclose%
\pgfusepath{fill}%
\end{pgfscope}%
\begin{pgfscope}%
\pgfpathrectangle{\pgfqpoint{0.017500in}{0.017500in}}{\pgfqpoint{6.218282in}{3.580372in}} %
\pgfusepath{clip}%
\pgfsetbuttcap%
\pgfsetroundjoin%
\definecolor{currentfill}{rgb}{0.723315,0.068898,0.162989}%
\pgfsetfillcolor{currentfill}%
\pgfsetlinewidth{0.000000pt}%
\definecolor{currentstroke}{rgb}{0.000000,0.000000,0.000000}%
\pgfsetstrokecolor{currentstroke}%
\pgfsetdash{}{0pt}%
\pgfpathmoveto{\pgfqpoint{4.724651in}{2.549289in}}%
\pgfpathlineto{\pgfqpoint{4.964811in}{2.684961in}}%
\pgfpathlineto{\pgfqpoint{4.724651in}{2.549289in}}%
\pgfpathclose%
\pgfusepath{fill}%
\end{pgfscope}%
\begin{pgfscope}%
\pgfpathrectangle{\pgfqpoint{0.017500in}{0.017500in}}{\pgfqpoint{6.218282in}{3.580372in}} %
\pgfusepath{clip}%
\pgfsetbuttcap%
\pgfsetroundjoin%
\definecolor{currentfill}{rgb}{0.723315,0.068898,0.162989}%
\pgfsetfillcolor{currentfill}%
\pgfsetlinewidth{0.000000pt}%
\definecolor{currentstroke}{rgb}{0.000000,0.000000,0.000000}%
\pgfsetstrokecolor{currentstroke}%
\pgfsetdash{}{0pt}%
\pgfpathmoveto{\pgfqpoint{4.724651in}{2.549289in}}%
\pgfpathlineto{\pgfqpoint{4.964811in}{2.684961in}}%
\pgfpathlineto{\pgfqpoint{4.724651in}{2.549289in}}%
\pgfpathclose%
\pgfusepath{fill}%
\end{pgfscope}%
\begin{pgfscope}%
\pgfpathrectangle{\pgfqpoint{0.017500in}{0.017500in}}{\pgfqpoint{6.218282in}{3.580372in}} %
\pgfusepath{clip}%
\pgfsetbuttcap%
\pgfsetroundjoin%
\definecolor{currentfill}{rgb}{0.723315,0.068898,0.162989}%
\pgfsetfillcolor{currentfill}%
\pgfsetlinewidth{0.000000pt}%
\definecolor{currentstroke}{rgb}{0.000000,0.000000,0.000000}%
\pgfsetstrokecolor{currentstroke}%
\pgfsetdash{}{0pt}%
\pgfpathmoveto{\pgfqpoint{4.724651in}{2.549289in}}%
\pgfpathlineto{\pgfqpoint{4.964811in}{2.684961in}}%
\pgfpathlineto{\pgfqpoint{4.724651in}{2.549289in}}%
\pgfpathclose%
\pgfusepath{fill}%
\end{pgfscope}%
\begin{pgfscope}%
\pgfpathrectangle{\pgfqpoint{0.017500in}{0.017500in}}{\pgfqpoint{6.218282in}{3.580372in}} %
\pgfusepath{clip}%
\pgfsetbuttcap%
\pgfsetroundjoin%
\definecolor{currentfill}{rgb}{0.723315,0.068898,0.162989}%
\pgfsetfillcolor{currentfill}%
\pgfsetlinewidth{0.000000pt}%
\definecolor{currentstroke}{rgb}{0.000000,0.000000,0.000000}%
\pgfsetstrokecolor{currentstroke}%
\pgfsetdash{}{0pt}%
\pgfpathmoveto{\pgfqpoint{4.724651in}{2.549289in}}%
\pgfpathlineto{\pgfqpoint{4.964811in}{2.684961in}}%
\pgfpathlineto{\pgfqpoint{4.724651in}{2.549289in}}%
\pgfpathclose%
\pgfusepath{fill}%
\end{pgfscope}%
\begin{pgfscope}%
\pgfpathrectangle{\pgfqpoint{0.017500in}{0.017500in}}{\pgfqpoint{6.218282in}{3.580372in}} %
\pgfusepath{clip}%
\pgfsetbuttcap%
\pgfsetroundjoin%
\definecolor{currentfill}{rgb}{0.723315,0.068898,0.162989}%
\pgfsetfillcolor{currentfill}%
\pgfsetlinewidth{0.000000pt}%
\definecolor{currentstroke}{rgb}{0.000000,0.000000,0.000000}%
\pgfsetstrokecolor{currentstroke}%
\pgfsetdash{}{0pt}%
\pgfpathmoveto{\pgfqpoint{4.724651in}{2.549289in}}%
\pgfpathlineto{\pgfqpoint{4.964811in}{2.684961in}}%
\pgfpathlineto{\pgfqpoint{4.724651in}{2.549289in}}%
\pgfpathclose%
\pgfusepath{fill}%
\end{pgfscope}%
\begin{pgfscope}%
\pgfpathrectangle{\pgfqpoint{0.017500in}{0.017500in}}{\pgfqpoint{6.218282in}{3.580372in}} %
\pgfusepath{clip}%
\pgfsetbuttcap%
\pgfsetroundjoin%
\definecolor{currentfill}{rgb}{0.723315,0.068898,0.162989}%
\pgfsetfillcolor{currentfill}%
\pgfsetlinewidth{0.000000pt}%
\definecolor{currentstroke}{rgb}{0.000000,0.000000,0.000000}%
\pgfsetstrokecolor{currentstroke}%
\pgfsetdash{}{0pt}%
\pgfpathmoveto{\pgfqpoint{4.724651in}{2.549289in}}%
\pgfpathlineto{\pgfqpoint{4.964811in}{2.684961in}}%
\pgfpathlineto{\pgfqpoint{4.724651in}{2.549289in}}%
\pgfpathclose%
\pgfusepath{fill}%
\end{pgfscope}%
\begin{pgfscope}%
\pgfpathrectangle{\pgfqpoint{0.017500in}{0.017500in}}{\pgfqpoint{6.218282in}{3.580372in}} %
\pgfusepath{clip}%
\pgfsetbuttcap%
\pgfsetroundjoin%
\definecolor{currentfill}{rgb}{0.723315,0.068898,0.162989}%
\pgfsetfillcolor{currentfill}%
\pgfsetlinewidth{0.000000pt}%
\definecolor{currentstroke}{rgb}{0.000000,0.000000,0.000000}%
\pgfsetstrokecolor{currentstroke}%
\pgfsetdash{}{0pt}%
\pgfpathmoveto{\pgfqpoint{4.724651in}{2.549289in}}%
\pgfpathlineto{\pgfqpoint{4.964811in}{2.684961in}}%
\pgfpathlineto{\pgfqpoint{4.724651in}{2.549289in}}%
\pgfpathclose%
\pgfusepath{fill}%
\end{pgfscope}%
\begin{pgfscope}%
\pgfpathrectangle{\pgfqpoint{0.017500in}{0.017500in}}{\pgfqpoint{6.218282in}{3.580372in}} %
\pgfusepath{clip}%
\pgfsetbuttcap%
\pgfsetroundjoin%
\definecolor{currentfill}{rgb}{0.723315,0.068898,0.162989}%
\pgfsetfillcolor{currentfill}%
\pgfsetlinewidth{0.000000pt}%
\definecolor{currentstroke}{rgb}{0.000000,0.000000,0.000000}%
\pgfsetstrokecolor{currentstroke}%
\pgfsetdash{}{0pt}%
\pgfpathmoveto{\pgfqpoint{4.724651in}{2.549289in}}%
\pgfpathlineto{\pgfqpoint{4.964811in}{2.684961in}}%
\pgfpathlineto{\pgfqpoint{4.724651in}{2.549289in}}%
\pgfpathclose%
\pgfusepath{fill}%
\end{pgfscope}%
\begin{pgfscope}%
\pgfpathrectangle{\pgfqpoint{0.017500in}{0.017500in}}{\pgfqpoint{6.218282in}{3.580372in}} %
\pgfusepath{clip}%
\pgfsetbuttcap%
\pgfsetroundjoin%
\definecolor{currentfill}{rgb}{0.723315,0.068898,0.162989}%
\pgfsetfillcolor{currentfill}%
\pgfsetlinewidth{0.000000pt}%
\definecolor{currentstroke}{rgb}{0.000000,0.000000,0.000000}%
\pgfsetstrokecolor{currentstroke}%
\pgfsetdash{}{0pt}%
\pgfpathmoveto{\pgfqpoint{4.724651in}{2.549289in}}%
\pgfpathlineto{\pgfqpoint{4.964811in}{2.684961in}}%
\pgfpathlineto{\pgfqpoint{4.724651in}{2.549289in}}%
\pgfpathclose%
\pgfusepath{fill}%
\end{pgfscope}%
\begin{pgfscope}%
\pgfpathrectangle{\pgfqpoint{0.017500in}{0.017500in}}{\pgfqpoint{6.218282in}{3.580372in}} %
\pgfusepath{clip}%
\pgfsetbuttcap%
\pgfsetroundjoin%
\definecolor{currentfill}{rgb}{0.723315,0.068898,0.162989}%
\pgfsetfillcolor{currentfill}%
\pgfsetlinewidth{0.000000pt}%
\definecolor{currentstroke}{rgb}{0.000000,0.000000,0.000000}%
\pgfsetstrokecolor{currentstroke}%
\pgfsetdash{}{0pt}%
\pgfpathmoveto{\pgfqpoint{4.724651in}{2.549289in}}%
\pgfpathlineto{\pgfqpoint{4.964811in}{2.684961in}}%
\pgfpathlineto{\pgfqpoint{4.724651in}{2.549289in}}%
\pgfpathclose%
\pgfusepath{fill}%
\end{pgfscope}%
\begin{pgfscope}%
\pgfpathrectangle{\pgfqpoint{0.017500in}{0.017500in}}{\pgfqpoint{6.218282in}{3.580372in}} %
\pgfusepath{clip}%
\pgfsetbuttcap%
\pgfsetroundjoin%
\definecolor{currentfill}{rgb}{0.723315,0.068898,0.162989}%
\pgfsetfillcolor{currentfill}%
\pgfsetlinewidth{0.000000pt}%
\definecolor{currentstroke}{rgb}{0.000000,0.000000,0.000000}%
\pgfsetstrokecolor{currentstroke}%
\pgfsetdash{}{0pt}%
\pgfpathmoveto{\pgfqpoint{4.724651in}{2.549289in}}%
\pgfpathlineto{\pgfqpoint{4.964811in}{2.684961in}}%
\pgfpathlineto{\pgfqpoint{4.724651in}{2.549289in}}%
\pgfpathclose%
\pgfusepath{fill}%
\end{pgfscope}%
\begin{pgfscope}%
\pgfpathrectangle{\pgfqpoint{0.017500in}{0.017500in}}{\pgfqpoint{6.218282in}{3.580372in}} %
\pgfusepath{clip}%
\pgfsetbuttcap%
\pgfsetroundjoin%
\definecolor{currentfill}{rgb}{0.723315,0.068898,0.162989}%
\pgfsetfillcolor{currentfill}%
\pgfsetlinewidth{0.000000pt}%
\definecolor{currentstroke}{rgb}{0.000000,0.000000,0.000000}%
\pgfsetstrokecolor{currentstroke}%
\pgfsetdash{}{0pt}%
\pgfpathmoveto{\pgfqpoint{4.724651in}{2.549289in}}%
\pgfpathlineto{\pgfqpoint{4.964811in}{2.684961in}}%
\pgfpathlineto{\pgfqpoint{4.724651in}{2.549289in}}%
\pgfpathclose%
\pgfusepath{fill}%
\end{pgfscope}%
\begin{pgfscope}%
\pgfpathrectangle{\pgfqpoint{0.017500in}{0.017500in}}{\pgfqpoint{6.218282in}{3.580372in}} %
\pgfusepath{clip}%
\pgfsetbuttcap%
\pgfsetroundjoin%
\definecolor{currentfill}{rgb}{0.723315,0.068898,0.162989}%
\pgfsetfillcolor{currentfill}%
\pgfsetlinewidth{0.000000pt}%
\definecolor{currentstroke}{rgb}{0.000000,0.000000,0.000000}%
\pgfsetstrokecolor{currentstroke}%
\pgfsetdash{}{0pt}%
\pgfpathmoveto{\pgfqpoint{4.724651in}{2.549289in}}%
\pgfpathlineto{\pgfqpoint{4.964811in}{2.684961in}}%
\pgfpathlineto{\pgfqpoint{4.724651in}{2.549289in}}%
\pgfpathclose%
\pgfusepath{fill}%
\end{pgfscope}%
\begin{pgfscope}%
\pgfpathrectangle{\pgfqpoint{0.017500in}{0.017500in}}{\pgfqpoint{6.218282in}{3.580372in}} %
\pgfusepath{clip}%
\pgfsetbuttcap%
\pgfsetroundjoin%
\definecolor{currentfill}{rgb}{0.723315,0.068898,0.162989}%
\pgfsetfillcolor{currentfill}%
\pgfsetlinewidth{0.000000pt}%
\definecolor{currentstroke}{rgb}{0.000000,0.000000,0.000000}%
\pgfsetstrokecolor{currentstroke}%
\pgfsetdash{}{0pt}%
\pgfpathmoveto{\pgfqpoint{4.724651in}{2.549289in}}%
\pgfpathlineto{\pgfqpoint{4.964811in}{2.684961in}}%
\pgfpathlineto{\pgfqpoint{4.724651in}{2.549289in}}%
\pgfpathclose%
\pgfusepath{fill}%
\end{pgfscope}%
\begin{pgfscope}%
\pgfpathrectangle{\pgfqpoint{0.017500in}{0.017500in}}{\pgfqpoint{6.218282in}{3.580372in}} %
\pgfusepath{clip}%
\pgfsetbuttcap%
\pgfsetroundjoin%
\definecolor{currentfill}{rgb}{0.723315,0.068898,0.162989}%
\pgfsetfillcolor{currentfill}%
\pgfsetlinewidth{0.000000pt}%
\definecolor{currentstroke}{rgb}{0.000000,0.000000,0.000000}%
\pgfsetstrokecolor{currentstroke}%
\pgfsetdash{}{0pt}%
\pgfpathmoveto{\pgfqpoint{4.724651in}{2.549289in}}%
\pgfpathlineto{\pgfqpoint{4.964811in}{2.684961in}}%
\pgfpathlineto{\pgfqpoint{4.724651in}{2.549289in}}%
\pgfpathclose%
\pgfusepath{fill}%
\end{pgfscope}%
\begin{pgfscope}%
\pgfpathrectangle{\pgfqpoint{0.017500in}{0.017500in}}{\pgfqpoint{6.218282in}{3.580372in}} %
\pgfusepath{clip}%
\pgfsetbuttcap%
\pgfsetroundjoin%
\definecolor{currentfill}{rgb}{0.723315,0.068898,0.162989}%
\pgfsetfillcolor{currentfill}%
\pgfsetlinewidth{0.000000pt}%
\definecolor{currentstroke}{rgb}{0.000000,0.000000,0.000000}%
\pgfsetstrokecolor{currentstroke}%
\pgfsetdash{}{0pt}%
\pgfpathmoveto{\pgfqpoint{4.724651in}{2.549289in}}%
\pgfpathlineto{\pgfqpoint{4.964811in}{2.684961in}}%
\pgfpathlineto{\pgfqpoint{4.724651in}{2.549289in}}%
\pgfpathclose%
\pgfusepath{fill}%
\end{pgfscope}%
\begin{pgfscope}%
\pgfpathrectangle{\pgfqpoint{0.017500in}{0.017500in}}{\pgfqpoint{6.218282in}{3.580372in}} %
\pgfusepath{clip}%
\pgfsetbuttcap%
\pgfsetroundjoin%
\definecolor{currentfill}{rgb}{0.723315,0.068898,0.162989}%
\pgfsetfillcolor{currentfill}%
\pgfsetlinewidth{0.000000pt}%
\definecolor{currentstroke}{rgb}{0.000000,0.000000,0.000000}%
\pgfsetstrokecolor{currentstroke}%
\pgfsetdash{}{0pt}%
\pgfpathmoveto{\pgfqpoint{4.724651in}{2.549289in}}%
\pgfpathlineto{\pgfqpoint{4.964811in}{2.684961in}}%
\pgfpathlineto{\pgfqpoint{4.724651in}{2.549289in}}%
\pgfpathclose%
\pgfusepath{fill}%
\end{pgfscope}%
\begin{pgfscope}%
\pgfpathrectangle{\pgfqpoint{0.017500in}{0.017500in}}{\pgfqpoint{6.218282in}{3.580372in}} %
\pgfusepath{clip}%
\pgfsetbuttcap%
\pgfsetroundjoin%
\definecolor{currentfill}{rgb}{0.723315,0.068898,0.162989}%
\pgfsetfillcolor{currentfill}%
\pgfsetlinewidth{0.000000pt}%
\definecolor{currentstroke}{rgb}{0.000000,0.000000,0.000000}%
\pgfsetstrokecolor{currentstroke}%
\pgfsetdash{}{0pt}%
\pgfpathmoveto{\pgfqpoint{4.724651in}{2.549289in}}%
\pgfpathlineto{\pgfqpoint{4.964811in}{2.684961in}}%
\pgfpathlineto{\pgfqpoint{4.724651in}{2.549289in}}%
\pgfpathclose%
\pgfusepath{fill}%
\end{pgfscope}%
\begin{pgfscope}%
\pgfpathrectangle{\pgfqpoint{0.017500in}{0.017500in}}{\pgfqpoint{6.218282in}{3.580372in}} %
\pgfusepath{clip}%
\pgfsetbuttcap%
\pgfsetroundjoin%
\definecolor{currentfill}{rgb}{0.723315,0.068898,0.162989}%
\pgfsetfillcolor{currentfill}%
\pgfsetlinewidth{0.000000pt}%
\definecolor{currentstroke}{rgb}{0.000000,0.000000,0.000000}%
\pgfsetstrokecolor{currentstroke}%
\pgfsetdash{}{0pt}%
\pgfpathmoveto{\pgfqpoint{4.724651in}{2.549289in}}%
\pgfpathlineto{\pgfqpoint{4.964811in}{2.684961in}}%
\pgfpathlineto{\pgfqpoint{4.724651in}{2.549289in}}%
\pgfpathclose%
\pgfusepath{fill}%
\end{pgfscope}%
\begin{pgfscope}%
\pgfpathrectangle{\pgfqpoint{0.017500in}{0.017500in}}{\pgfqpoint{6.218282in}{3.580372in}} %
\pgfusepath{clip}%
\pgfsetbuttcap%
\pgfsetroundjoin%
\definecolor{currentfill}{rgb}{0.723315,0.068898,0.162989}%
\pgfsetfillcolor{currentfill}%
\pgfsetlinewidth{0.000000pt}%
\definecolor{currentstroke}{rgb}{0.000000,0.000000,0.000000}%
\pgfsetstrokecolor{currentstroke}%
\pgfsetdash{}{0pt}%
\pgfpathmoveto{\pgfqpoint{4.724651in}{2.549289in}}%
\pgfpathlineto{\pgfqpoint{4.964811in}{2.684961in}}%
\pgfpathlineto{\pgfqpoint{4.724651in}{2.549289in}}%
\pgfpathclose%
\pgfusepath{fill}%
\end{pgfscope}%
\begin{pgfscope}%
\pgfpathrectangle{\pgfqpoint{0.017500in}{0.017500in}}{\pgfqpoint{6.218282in}{3.580372in}} %
\pgfusepath{clip}%
\pgfsetbuttcap%
\pgfsetroundjoin%
\definecolor{currentfill}{rgb}{0.723315,0.068898,0.162989}%
\pgfsetfillcolor{currentfill}%
\pgfsetlinewidth{0.000000pt}%
\definecolor{currentstroke}{rgb}{0.000000,0.000000,0.000000}%
\pgfsetstrokecolor{currentstroke}%
\pgfsetdash{}{0pt}%
\pgfpathmoveto{\pgfqpoint{4.724651in}{2.549289in}}%
\pgfpathlineto{\pgfqpoint{4.964811in}{2.684961in}}%
\pgfpathlineto{\pgfqpoint{4.724651in}{2.549289in}}%
\pgfpathclose%
\pgfusepath{fill}%
\end{pgfscope}%
\begin{pgfscope}%
\pgfpathrectangle{\pgfqpoint{0.017500in}{0.017500in}}{\pgfqpoint{6.218282in}{3.580372in}} %
\pgfusepath{clip}%
\pgfsetbuttcap%
\pgfsetroundjoin%
\definecolor{currentfill}{rgb}{0.723315,0.068898,0.162989}%
\pgfsetfillcolor{currentfill}%
\pgfsetlinewidth{0.000000pt}%
\definecolor{currentstroke}{rgb}{0.000000,0.000000,0.000000}%
\pgfsetstrokecolor{currentstroke}%
\pgfsetdash{}{0pt}%
\pgfpathmoveto{\pgfqpoint{4.724651in}{2.549289in}}%
\pgfpathlineto{\pgfqpoint{4.964811in}{2.684961in}}%
\pgfpathlineto{\pgfqpoint{4.724651in}{2.549289in}}%
\pgfpathclose%
\pgfusepath{fill}%
\end{pgfscope}%
\begin{pgfscope}%
\pgfpathrectangle{\pgfqpoint{0.017500in}{0.017500in}}{\pgfqpoint{6.218282in}{3.580372in}} %
\pgfusepath{clip}%
\pgfsetbuttcap%
\pgfsetroundjoin%
\definecolor{currentfill}{rgb}{0.723315,0.068898,0.162989}%
\pgfsetfillcolor{currentfill}%
\pgfsetlinewidth{0.000000pt}%
\definecolor{currentstroke}{rgb}{0.000000,0.000000,0.000000}%
\pgfsetstrokecolor{currentstroke}%
\pgfsetdash{}{0pt}%
\pgfpathmoveto{\pgfqpoint{4.724651in}{2.549289in}}%
\pgfpathlineto{\pgfqpoint{4.964811in}{2.684961in}}%
\pgfpathlineto{\pgfqpoint{4.724651in}{2.549289in}}%
\pgfpathclose%
\pgfusepath{fill}%
\end{pgfscope}%
\begin{pgfscope}%
\pgfpathrectangle{\pgfqpoint{0.017500in}{0.017500in}}{\pgfqpoint{6.218282in}{3.580372in}} %
\pgfusepath{clip}%
\pgfsetbuttcap%
\pgfsetroundjoin%
\definecolor{currentfill}{rgb}{0.723315,0.068898,0.162989}%
\pgfsetfillcolor{currentfill}%
\pgfsetlinewidth{0.000000pt}%
\definecolor{currentstroke}{rgb}{0.000000,0.000000,0.000000}%
\pgfsetstrokecolor{currentstroke}%
\pgfsetdash{}{0pt}%
\pgfpathmoveto{\pgfqpoint{4.724651in}{2.549289in}}%
\pgfpathlineto{\pgfqpoint{4.964811in}{2.684961in}}%
\pgfpathlineto{\pgfqpoint{4.724651in}{2.549289in}}%
\pgfpathclose%
\pgfusepath{fill}%
\end{pgfscope}%
\begin{pgfscope}%
\pgfpathrectangle{\pgfqpoint{0.017500in}{0.017500in}}{\pgfqpoint{6.218282in}{3.580372in}} %
\pgfusepath{clip}%
\pgfsetbuttcap%
\pgfsetroundjoin%
\definecolor{currentfill}{rgb}{0.723315,0.068898,0.162989}%
\pgfsetfillcolor{currentfill}%
\pgfsetlinewidth{0.000000pt}%
\definecolor{currentstroke}{rgb}{0.000000,0.000000,0.000000}%
\pgfsetstrokecolor{currentstroke}%
\pgfsetdash{}{0pt}%
\pgfpathmoveto{\pgfqpoint{4.724651in}{2.549289in}}%
\pgfpathlineto{\pgfqpoint{4.964811in}{2.684961in}}%
\pgfpathlineto{\pgfqpoint{4.724651in}{2.549289in}}%
\pgfpathclose%
\pgfusepath{fill}%
\end{pgfscope}%
\begin{pgfscope}%
\pgfpathrectangle{\pgfqpoint{0.017500in}{0.017500in}}{\pgfqpoint{6.218282in}{3.580372in}} %
\pgfusepath{clip}%
\pgfsetbuttcap%
\pgfsetroundjoin%
\definecolor{currentfill}{rgb}{0.328604,0.439712,0.869587}%
\pgfsetfillcolor{currentfill}%
\pgfsetlinewidth{0.000000pt}%
\definecolor{currentstroke}{rgb}{0.000000,0.000000,0.000000}%
\pgfsetstrokecolor{currentstroke}%
\pgfsetdash{}{0pt}%
\pgfpathmoveto{\pgfqpoint{2.349853in}{0.872712in}}%
\pgfpathlineto{\pgfqpoint{2.789950in}{0.798346in}}%
\pgfpathlineto{\pgfqpoint{3.043326in}{2.141782in}}%
\pgfpathlineto{\pgfqpoint{2.590432in}{2.210709in}}%
\pgfpathlineto{\pgfqpoint{2.349853in}{0.872712in}}%
\pgfpathclose%
\pgfusepath{fill}%
\end{pgfscope}%
\begin{pgfscope}%
\pgfpathrectangle{\pgfqpoint{0.017500in}{0.017500in}}{\pgfqpoint{6.218282in}{3.580372in}} %
\pgfusepath{clip}%
\pgfsetbuttcap%
\pgfsetroundjoin%
\definecolor{currentfill}{rgb}{0.328604,0.439712,0.869587}%
\pgfsetfillcolor{currentfill}%
\pgfsetlinewidth{0.000000pt}%
\definecolor{currentstroke}{rgb}{0.000000,0.000000,0.000000}%
\pgfsetstrokecolor{currentstroke}%
\pgfsetdash{}{0pt}%
\pgfpathmoveto{\pgfqpoint{2.349853in}{0.872712in}}%
\pgfpathlineto{\pgfqpoint{2.789950in}{0.798346in}}%
\pgfpathlineto{\pgfqpoint{3.043326in}{2.141782in}}%
\pgfpathlineto{\pgfqpoint{2.590432in}{2.210709in}}%
\pgfpathlineto{\pgfqpoint{2.349853in}{0.872712in}}%
\pgfpathclose%
\pgfusepath{fill}%
\end{pgfscope}%
\begin{pgfscope}%
\pgfpathrectangle{\pgfqpoint{0.017500in}{0.017500in}}{\pgfqpoint{6.218282in}{3.580372in}} %
\pgfusepath{clip}%
\pgfsetbuttcap%
\pgfsetroundjoin%
\definecolor{currentfill}{rgb}{0.328604,0.439712,0.869587}%
\pgfsetfillcolor{currentfill}%
\pgfsetlinewidth{0.000000pt}%
\definecolor{currentstroke}{rgb}{0.000000,0.000000,0.000000}%
\pgfsetstrokecolor{currentstroke}%
\pgfsetdash{}{0pt}%
\pgfpathmoveto{\pgfqpoint{2.349853in}{0.872712in}}%
\pgfpathlineto{\pgfqpoint{2.789950in}{0.798346in}}%
\pgfpathlineto{\pgfqpoint{3.043326in}{2.141782in}}%
\pgfpathlineto{\pgfqpoint{2.590432in}{2.210709in}}%
\pgfpathlineto{\pgfqpoint{2.349853in}{0.872712in}}%
\pgfpathclose%
\pgfusepath{fill}%
\end{pgfscope}%
\begin{pgfscope}%
\pgfpathrectangle{\pgfqpoint{0.017500in}{0.017500in}}{\pgfqpoint{6.218282in}{3.580372in}} %
\pgfusepath{clip}%
\pgfsetbuttcap%
\pgfsetroundjoin%
\definecolor{currentfill}{rgb}{0.328604,0.439712,0.869587}%
\pgfsetfillcolor{currentfill}%
\pgfsetlinewidth{0.000000pt}%
\definecolor{currentstroke}{rgb}{0.000000,0.000000,0.000000}%
\pgfsetstrokecolor{currentstroke}%
\pgfsetdash{}{0pt}%
\pgfpathmoveto{\pgfqpoint{2.349853in}{0.872712in}}%
\pgfpathlineto{\pgfqpoint{2.789950in}{0.798346in}}%
\pgfpathlineto{\pgfqpoint{3.043326in}{2.141782in}}%
\pgfpathlineto{\pgfqpoint{2.590432in}{2.210709in}}%
\pgfpathlineto{\pgfqpoint{2.349853in}{0.872712in}}%
\pgfpathclose%
\pgfusepath{fill}%
\end{pgfscope}%
\begin{pgfscope}%
\pgfpathrectangle{\pgfqpoint{0.017500in}{0.017500in}}{\pgfqpoint{6.218282in}{3.580372in}} %
\pgfusepath{clip}%
\pgfsetbuttcap%
\pgfsetroundjoin%
\definecolor{currentfill}{rgb}{0.328604,0.439712,0.869587}%
\pgfsetfillcolor{currentfill}%
\pgfsetlinewidth{0.000000pt}%
\definecolor{currentstroke}{rgb}{0.000000,0.000000,0.000000}%
\pgfsetstrokecolor{currentstroke}%
\pgfsetdash{}{0pt}%
\pgfpathmoveto{\pgfqpoint{2.349853in}{0.872712in}}%
\pgfpathlineto{\pgfqpoint{2.789950in}{0.798346in}}%
\pgfpathlineto{\pgfqpoint{3.043326in}{2.141782in}}%
\pgfpathlineto{\pgfqpoint{2.590432in}{2.210709in}}%
\pgfpathlineto{\pgfqpoint{2.349853in}{0.872712in}}%
\pgfpathclose%
\pgfusepath{fill}%
\end{pgfscope}%
\begin{pgfscope}%
\pgfpathrectangle{\pgfqpoint{0.017500in}{0.017500in}}{\pgfqpoint{6.218282in}{3.580372in}} %
\pgfusepath{clip}%
\pgfsetbuttcap%
\pgfsetroundjoin%
\definecolor{currentfill}{rgb}{0.328604,0.439712,0.869587}%
\pgfsetfillcolor{currentfill}%
\pgfsetlinewidth{0.000000pt}%
\definecolor{currentstroke}{rgb}{0.000000,0.000000,0.000000}%
\pgfsetstrokecolor{currentstroke}%
\pgfsetdash{}{0pt}%
\pgfpathmoveto{\pgfqpoint{2.349853in}{0.872712in}}%
\pgfpathlineto{\pgfqpoint{2.789950in}{0.798346in}}%
\pgfpathlineto{\pgfqpoint{3.043326in}{2.141782in}}%
\pgfpathlineto{\pgfqpoint{2.590432in}{2.210709in}}%
\pgfpathlineto{\pgfqpoint{2.349853in}{0.872712in}}%
\pgfpathclose%
\pgfusepath{fill}%
\end{pgfscope}%
\begin{pgfscope}%
\pgfpathrectangle{\pgfqpoint{0.017500in}{0.017500in}}{\pgfqpoint{6.218282in}{3.580372in}} %
\pgfusepath{clip}%
\pgfsetbuttcap%
\pgfsetroundjoin%
\definecolor{currentfill}{rgb}{0.805723,0.259813,0.230562}%
\pgfsetfillcolor{currentfill}%
\pgfsetlinewidth{0.000000pt}%
\definecolor{currentstroke}{rgb}{0.000000,0.000000,0.000000}%
\pgfsetstrokecolor{currentstroke}%
\pgfsetdash{}{0pt}%
\pgfpathmoveto{\pgfqpoint{2.855513in}{2.451418in}}%
\pgfpathlineto{\pgfqpoint{3.116513in}{2.603812in}}%
\pgfpathlineto{\pgfqpoint{2.855513in}{2.451418in}}%
\pgfpathclose%
\pgfusepath{fill}%
\end{pgfscope}%
\begin{pgfscope}%
\pgfpathrectangle{\pgfqpoint{0.017500in}{0.017500in}}{\pgfqpoint{6.218282in}{3.580372in}} %
\pgfusepath{clip}%
\pgfsetbuttcap%
\pgfsetroundjoin%
\definecolor{currentfill}{rgb}{0.805723,0.259813,0.230562}%
\pgfsetfillcolor{currentfill}%
\pgfsetlinewidth{0.000000pt}%
\definecolor{currentstroke}{rgb}{0.000000,0.000000,0.000000}%
\pgfsetstrokecolor{currentstroke}%
\pgfsetdash{}{0pt}%
\pgfpathmoveto{\pgfqpoint{2.855513in}{2.451418in}}%
\pgfpathlineto{\pgfqpoint{3.116513in}{2.603812in}}%
\pgfpathlineto{\pgfqpoint{2.855513in}{2.451418in}}%
\pgfpathclose%
\pgfusepath{fill}%
\end{pgfscope}%
\begin{pgfscope}%
\pgfpathrectangle{\pgfqpoint{0.017500in}{0.017500in}}{\pgfqpoint{6.218282in}{3.580372in}} %
\pgfusepath{clip}%
\pgfsetbuttcap%
\pgfsetroundjoin%
\definecolor{currentfill}{rgb}{0.805723,0.259813,0.230562}%
\pgfsetfillcolor{currentfill}%
\pgfsetlinewidth{0.000000pt}%
\definecolor{currentstroke}{rgb}{0.000000,0.000000,0.000000}%
\pgfsetstrokecolor{currentstroke}%
\pgfsetdash{}{0pt}%
\pgfpathmoveto{\pgfqpoint{2.855513in}{2.451418in}}%
\pgfpathlineto{\pgfqpoint{3.116513in}{2.603812in}}%
\pgfpathlineto{\pgfqpoint{2.855513in}{2.451418in}}%
\pgfpathclose%
\pgfusepath{fill}%
\end{pgfscope}%
\begin{pgfscope}%
\pgfpathrectangle{\pgfqpoint{0.017500in}{0.017500in}}{\pgfqpoint{6.218282in}{3.580372in}} %
\pgfusepath{clip}%
\pgfsetbuttcap%
\pgfsetroundjoin%
\definecolor{currentfill}{rgb}{0.805723,0.259813,0.230562}%
\pgfsetfillcolor{currentfill}%
\pgfsetlinewidth{0.000000pt}%
\definecolor{currentstroke}{rgb}{0.000000,0.000000,0.000000}%
\pgfsetstrokecolor{currentstroke}%
\pgfsetdash{}{0pt}%
\pgfpathmoveto{\pgfqpoint{2.855513in}{2.451418in}}%
\pgfpathlineto{\pgfqpoint{3.116513in}{2.603812in}}%
\pgfpathlineto{\pgfqpoint{2.855513in}{2.451418in}}%
\pgfpathclose%
\pgfusepath{fill}%
\end{pgfscope}%
\begin{pgfscope}%
\pgfpathrectangle{\pgfqpoint{0.017500in}{0.017500in}}{\pgfqpoint{6.218282in}{3.580372in}} %
\pgfusepath{clip}%
\pgfsetbuttcap%
\pgfsetroundjoin%
\definecolor{currentfill}{rgb}{0.805723,0.259813,0.230562}%
\pgfsetfillcolor{currentfill}%
\pgfsetlinewidth{0.000000pt}%
\definecolor{currentstroke}{rgb}{0.000000,0.000000,0.000000}%
\pgfsetstrokecolor{currentstroke}%
\pgfsetdash{}{0pt}%
\pgfpathmoveto{\pgfqpoint{2.855513in}{2.451418in}}%
\pgfpathlineto{\pgfqpoint{3.116513in}{2.603812in}}%
\pgfpathlineto{\pgfqpoint{2.855513in}{2.451418in}}%
\pgfpathclose%
\pgfusepath{fill}%
\end{pgfscope}%
\begin{pgfscope}%
\pgfpathrectangle{\pgfqpoint{0.017500in}{0.017500in}}{\pgfqpoint{6.218282in}{3.580372in}} %
\pgfusepath{clip}%
\pgfsetbuttcap%
\pgfsetroundjoin%
\definecolor{currentfill}{rgb}{0.805723,0.259813,0.230562}%
\pgfsetfillcolor{currentfill}%
\pgfsetlinewidth{0.000000pt}%
\definecolor{currentstroke}{rgb}{0.000000,0.000000,0.000000}%
\pgfsetstrokecolor{currentstroke}%
\pgfsetdash{}{0pt}%
\pgfpathmoveto{\pgfqpoint{2.855513in}{2.451418in}}%
\pgfpathlineto{\pgfqpoint{3.116513in}{2.603812in}}%
\pgfpathlineto{\pgfqpoint{2.855513in}{2.451418in}}%
\pgfpathclose%
\pgfusepath{fill}%
\end{pgfscope}%
\begin{pgfscope}%
\pgfpathrectangle{\pgfqpoint{0.017500in}{0.017500in}}{\pgfqpoint{6.218282in}{3.580372in}} %
\pgfusepath{clip}%
\pgfsetbuttcap%
\pgfsetroundjoin%
\definecolor{currentfill}{rgb}{0.805723,0.259813,0.230562}%
\pgfsetfillcolor{currentfill}%
\pgfsetlinewidth{0.000000pt}%
\definecolor{currentstroke}{rgb}{0.000000,0.000000,0.000000}%
\pgfsetstrokecolor{currentstroke}%
\pgfsetdash{}{0pt}%
\pgfpathmoveto{\pgfqpoint{2.855513in}{2.451418in}}%
\pgfpathlineto{\pgfqpoint{3.116513in}{2.603812in}}%
\pgfpathlineto{\pgfqpoint{2.855513in}{2.451418in}}%
\pgfpathclose%
\pgfusepath{fill}%
\end{pgfscope}%
\begin{pgfscope}%
\pgfpathrectangle{\pgfqpoint{0.017500in}{0.017500in}}{\pgfqpoint{6.218282in}{3.580372in}} %
\pgfusepath{clip}%
\pgfsetbuttcap%
\pgfsetroundjoin%
\definecolor{currentfill}{rgb}{0.805723,0.259813,0.230562}%
\pgfsetfillcolor{currentfill}%
\pgfsetlinewidth{0.000000pt}%
\definecolor{currentstroke}{rgb}{0.000000,0.000000,0.000000}%
\pgfsetstrokecolor{currentstroke}%
\pgfsetdash{}{0pt}%
\pgfpathmoveto{\pgfqpoint{2.855513in}{2.451418in}}%
\pgfpathlineto{\pgfqpoint{3.116513in}{2.603812in}}%
\pgfpathlineto{\pgfqpoint{2.855513in}{2.451418in}}%
\pgfpathclose%
\pgfusepath{fill}%
\end{pgfscope}%
\begin{pgfscope}%
\pgfpathrectangle{\pgfqpoint{0.017500in}{0.017500in}}{\pgfqpoint{6.218282in}{3.580372in}} %
\pgfusepath{clip}%
\pgfsetbuttcap%
\pgfsetroundjoin%
\definecolor{currentfill}{rgb}{0.805723,0.259813,0.230562}%
\pgfsetfillcolor{currentfill}%
\pgfsetlinewidth{0.000000pt}%
\definecolor{currentstroke}{rgb}{0.000000,0.000000,0.000000}%
\pgfsetstrokecolor{currentstroke}%
\pgfsetdash{}{0pt}%
\pgfpathmoveto{\pgfqpoint{2.855513in}{2.451418in}}%
\pgfpathlineto{\pgfqpoint{3.116513in}{2.603812in}}%
\pgfpathlineto{\pgfqpoint{2.855513in}{2.451418in}}%
\pgfpathclose%
\pgfusepath{fill}%
\end{pgfscope}%
\begin{pgfscope}%
\pgfpathrectangle{\pgfqpoint{0.017500in}{0.017500in}}{\pgfqpoint{6.218282in}{3.580372in}} %
\pgfusepath{clip}%
\pgfsetbuttcap%
\pgfsetroundjoin%
\definecolor{currentfill}{rgb}{0.805723,0.259813,0.230562}%
\pgfsetfillcolor{currentfill}%
\pgfsetlinewidth{0.000000pt}%
\definecolor{currentstroke}{rgb}{0.000000,0.000000,0.000000}%
\pgfsetstrokecolor{currentstroke}%
\pgfsetdash{}{0pt}%
\pgfpathmoveto{\pgfqpoint{2.855513in}{2.451418in}}%
\pgfpathlineto{\pgfqpoint{3.116513in}{2.603812in}}%
\pgfpathlineto{\pgfqpoint{2.855513in}{2.451418in}}%
\pgfpathclose%
\pgfusepath{fill}%
\end{pgfscope}%
\begin{pgfscope}%
\pgfpathrectangle{\pgfqpoint{0.017500in}{0.017500in}}{\pgfqpoint{6.218282in}{3.580372in}} %
\pgfusepath{clip}%
\pgfsetbuttcap%
\pgfsetroundjoin%
\definecolor{currentfill}{rgb}{0.805723,0.259813,0.230562}%
\pgfsetfillcolor{currentfill}%
\pgfsetlinewidth{0.000000pt}%
\definecolor{currentstroke}{rgb}{0.000000,0.000000,0.000000}%
\pgfsetstrokecolor{currentstroke}%
\pgfsetdash{}{0pt}%
\pgfpathmoveto{\pgfqpoint{2.855513in}{2.451418in}}%
\pgfpathlineto{\pgfqpoint{3.116513in}{2.603812in}}%
\pgfpathlineto{\pgfqpoint{2.855513in}{2.451418in}}%
\pgfpathclose%
\pgfusepath{fill}%
\end{pgfscope}%
\begin{pgfscope}%
\pgfpathrectangle{\pgfqpoint{0.017500in}{0.017500in}}{\pgfqpoint{6.218282in}{3.580372in}} %
\pgfusepath{clip}%
\pgfsetbuttcap%
\pgfsetroundjoin%
\definecolor{currentfill}{rgb}{0.805723,0.259813,0.230562}%
\pgfsetfillcolor{currentfill}%
\pgfsetlinewidth{0.000000pt}%
\definecolor{currentstroke}{rgb}{0.000000,0.000000,0.000000}%
\pgfsetstrokecolor{currentstroke}%
\pgfsetdash{}{0pt}%
\pgfpathmoveto{\pgfqpoint{2.855513in}{2.451418in}}%
\pgfpathlineto{\pgfqpoint{3.116513in}{2.603812in}}%
\pgfpathlineto{\pgfqpoint{2.855513in}{2.451418in}}%
\pgfpathclose%
\pgfusepath{fill}%
\end{pgfscope}%
\begin{pgfscope}%
\pgfpathrectangle{\pgfqpoint{0.017500in}{0.017500in}}{\pgfqpoint{6.218282in}{3.580372in}} %
\pgfusepath{clip}%
\pgfsetbuttcap%
\pgfsetroundjoin%
\definecolor{currentfill}{rgb}{0.805723,0.259813,0.230562}%
\pgfsetfillcolor{currentfill}%
\pgfsetlinewidth{0.000000pt}%
\definecolor{currentstroke}{rgb}{0.000000,0.000000,0.000000}%
\pgfsetstrokecolor{currentstroke}%
\pgfsetdash{}{0pt}%
\pgfpathmoveto{\pgfqpoint{2.855513in}{2.451418in}}%
\pgfpathlineto{\pgfqpoint{3.116513in}{2.603812in}}%
\pgfpathlineto{\pgfqpoint{2.855513in}{2.451418in}}%
\pgfpathclose%
\pgfusepath{fill}%
\end{pgfscope}%
\begin{pgfscope}%
\pgfpathrectangle{\pgfqpoint{0.017500in}{0.017500in}}{\pgfqpoint{6.218282in}{3.580372in}} %
\pgfusepath{clip}%
\pgfsetbuttcap%
\pgfsetroundjoin%
\definecolor{currentfill}{rgb}{0.805723,0.259813,0.230562}%
\pgfsetfillcolor{currentfill}%
\pgfsetlinewidth{0.000000pt}%
\definecolor{currentstroke}{rgb}{0.000000,0.000000,0.000000}%
\pgfsetstrokecolor{currentstroke}%
\pgfsetdash{}{0pt}%
\pgfpathmoveto{\pgfqpoint{2.855513in}{2.451418in}}%
\pgfpathlineto{\pgfqpoint{3.116513in}{2.603812in}}%
\pgfpathlineto{\pgfqpoint{2.855513in}{2.451418in}}%
\pgfpathclose%
\pgfusepath{fill}%
\end{pgfscope}%
\begin{pgfscope}%
\pgfpathrectangle{\pgfqpoint{0.017500in}{0.017500in}}{\pgfqpoint{6.218282in}{3.580372in}} %
\pgfusepath{clip}%
\pgfsetbuttcap%
\pgfsetroundjoin%
\definecolor{currentfill}{rgb}{0.805723,0.259813,0.230562}%
\pgfsetfillcolor{currentfill}%
\pgfsetlinewidth{0.000000pt}%
\definecolor{currentstroke}{rgb}{0.000000,0.000000,0.000000}%
\pgfsetstrokecolor{currentstroke}%
\pgfsetdash{}{0pt}%
\pgfpathmoveto{\pgfqpoint{2.855513in}{2.451418in}}%
\pgfpathlineto{\pgfqpoint{3.116513in}{2.603812in}}%
\pgfpathlineto{\pgfqpoint{2.855513in}{2.451418in}}%
\pgfpathclose%
\pgfusepath{fill}%
\end{pgfscope}%
\begin{pgfscope}%
\pgfpathrectangle{\pgfqpoint{0.017500in}{0.017500in}}{\pgfqpoint{6.218282in}{3.580372in}} %
\pgfusepath{clip}%
\pgfsetbuttcap%
\pgfsetroundjoin%
\definecolor{currentfill}{rgb}{0.805723,0.259813,0.230562}%
\pgfsetfillcolor{currentfill}%
\pgfsetlinewidth{0.000000pt}%
\definecolor{currentstroke}{rgb}{0.000000,0.000000,0.000000}%
\pgfsetstrokecolor{currentstroke}%
\pgfsetdash{}{0pt}%
\pgfpathmoveto{\pgfqpoint{2.855513in}{2.451418in}}%
\pgfpathlineto{\pgfqpoint{3.116513in}{2.603812in}}%
\pgfpathlineto{\pgfqpoint{2.855513in}{2.451418in}}%
\pgfpathclose%
\pgfusepath{fill}%
\end{pgfscope}%
\begin{pgfscope}%
\pgfpathrectangle{\pgfqpoint{0.017500in}{0.017500in}}{\pgfqpoint{6.218282in}{3.580372in}} %
\pgfusepath{clip}%
\pgfsetbuttcap%
\pgfsetroundjoin%
\definecolor{currentfill}{rgb}{0.805723,0.259813,0.230562}%
\pgfsetfillcolor{currentfill}%
\pgfsetlinewidth{0.000000pt}%
\definecolor{currentstroke}{rgb}{0.000000,0.000000,0.000000}%
\pgfsetstrokecolor{currentstroke}%
\pgfsetdash{}{0pt}%
\pgfpathmoveto{\pgfqpoint{2.855513in}{2.451418in}}%
\pgfpathlineto{\pgfqpoint{3.116513in}{2.603812in}}%
\pgfpathlineto{\pgfqpoint{2.855513in}{2.451418in}}%
\pgfpathclose%
\pgfusepath{fill}%
\end{pgfscope}%
\begin{pgfscope}%
\pgfpathrectangle{\pgfqpoint{0.017500in}{0.017500in}}{\pgfqpoint{6.218282in}{3.580372in}} %
\pgfusepath{clip}%
\pgfsetbuttcap%
\pgfsetroundjoin%
\definecolor{currentfill}{rgb}{0.805723,0.259813,0.230562}%
\pgfsetfillcolor{currentfill}%
\pgfsetlinewidth{0.000000pt}%
\definecolor{currentstroke}{rgb}{0.000000,0.000000,0.000000}%
\pgfsetstrokecolor{currentstroke}%
\pgfsetdash{}{0pt}%
\pgfpathmoveto{\pgfqpoint{2.855513in}{2.451418in}}%
\pgfpathlineto{\pgfqpoint{3.116513in}{2.603812in}}%
\pgfpathlineto{\pgfqpoint{2.855513in}{2.451418in}}%
\pgfpathclose%
\pgfusepath{fill}%
\end{pgfscope}%
\begin{pgfscope}%
\pgfpathrectangle{\pgfqpoint{0.017500in}{0.017500in}}{\pgfqpoint{6.218282in}{3.580372in}} %
\pgfusepath{clip}%
\pgfsetbuttcap%
\pgfsetroundjoin%
\definecolor{currentfill}{rgb}{0.805723,0.259813,0.230562}%
\pgfsetfillcolor{currentfill}%
\pgfsetlinewidth{0.000000pt}%
\definecolor{currentstroke}{rgb}{0.000000,0.000000,0.000000}%
\pgfsetstrokecolor{currentstroke}%
\pgfsetdash{}{0pt}%
\pgfpathmoveto{\pgfqpoint{2.855513in}{2.451418in}}%
\pgfpathlineto{\pgfqpoint{3.116513in}{2.603812in}}%
\pgfpathlineto{\pgfqpoint{2.855513in}{2.451418in}}%
\pgfpathclose%
\pgfusepath{fill}%
\end{pgfscope}%
\begin{pgfscope}%
\pgfpathrectangle{\pgfqpoint{0.017500in}{0.017500in}}{\pgfqpoint{6.218282in}{3.580372in}} %
\pgfusepath{clip}%
\pgfsetbuttcap%
\pgfsetroundjoin%
\definecolor{currentfill}{rgb}{0.805723,0.259813,0.230562}%
\pgfsetfillcolor{currentfill}%
\pgfsetlinewidth{0.000000pt}%
\definecolor{currentstroke}{rgb}{0.000000,0.000000,0.000000}%
\pgfsetstrokecolor{currentstroke}%
\pgfsetdash{}{0pt}%
\pgfpathmoveto{\pgfqpoint{2.855513in}{2.451418in}}%
\pgfpathlineto{\pgfqpoint{3.116513in}{2.603812in}}%
\pgfpathlineto{\pgfqpoint{2.855513in}{2.451418in}}%
\pgfpathclose%
\pgfusepath{fill}%
\end{pgfscope}%
\begin{pgfscope}%
\pgfpathrectangle{\pgfqpoint{0.017500in}{0.017500in}}{\pgfqpoint{6.218282in}{3.580372in}} %
\pgfusepath{clip}%
\pgfsetbuttcap%
\pgfsetroundjoin%
\definecolor{currentfill}{rgb}{0.805723,0.259813,0.230562}%
\pgfsetfillcolor{currentfill}%
\pgfsetlinewidth{0.000000pt}%
\definecolor{currentstroke}{rgb}{0.000000,0.000000,0.000000}%
\pgfsetstrokecolor{currentstroke}%
\pgfsetdash{}{0pt}%
\pgfpathmoveto{\pgfqpoint{2.855513in}{2.451418in}}%
\pgfpathlineto{\pgfqpoint{3.116513in}{2.603812in}}%
\pgfpathlineto{\pgfqpoint{2.855513in}{2.451418in}}%
\pgfpathclose%
\pgfusepath{fill}%
\end{pgfscope}%
\begin{pgfscope}%
\pgfpathrectangle{\pgfqpoint{0.017500in}{0.017500in}}{\pgfqpoint{6.218282in}{3.580372in}} %
\pgfusepath{clip}%
\pgfsetbuttcap%
\pgfsetroundjoin%
\definecolor{currentfill}{rgb}{0.805723,0.259813,0.230562}%
\pgfsetfillcolor{currentfill}%
\pgfsetlinewidth{0.000000pt}%
\definecolor{currentstroke}{rgb}{0.000000,0.000000,0.000000}%
\pgfsetstrokecolor{currentstroke}%
\pgfsetdash{}{0pt}%
\pgfpathmoveto{\pgfqpoint{2.855513in}{2.451418in}}%
\pgfpathlineto{\pgfqpoint{3.116513in}{2.603812in}}%
\pgfpathlineto{\pgfqpoint{2.855513in}{2.451418in}}%
\pgfpathclose%
\pgfusepath{fill}%
\end{pgfscope}%
\begin{pgfscope}%
\pgfpathrectangle{\pgfqpoint{0.017500in}{0.017500in}}{\pgfqpoint{6.218282in}{3.580372in}} %
\pgfusepath{clip}%
\pgfsetbuttcap%
\pgfsetroundjoin%
\definecolor{currentfill}{rgb}{0.805723,0.259813,0.230562}%
\pgfsetfillcolor{currentfill}%
\pgfsetlinewidth{0.000000pt}%
\definecolor{currentstroke}{rgb}{0.000000,0.000000,0.000000}%
\pgfsetstrokecolor{currentstroke}%
\pgfsetdash{}{0pt}%
\pgfpathmoveto{\pgfqpoint{2.855513in}{2.451418in}}%
\pgfpathlineto{\pgfqpoint{3.116513in}{2.603812in}}%
\pgfpathlineto{\pgfqpoint{2.855513in}{2.451418in}}%
\pgfpathclose%
\pgfusepath{fill}%
\end{pgfscope}%
\begin{pgfscope}%
\pgfpathrectangle{\pgfqpoint{0.017500in}{0.017500in}}{\pgfqpoint{6.218282in}{3.580372in}} %
\pgfusepath{clip}%
\pgfsetbuttcap%
\pgfsetroundjoin%
\definecolor{currentfill}{rgb}{0.805723,0.259813,0.230562}%
\pgfsetfillcolor{currentfill}%
\pgfsetlinewidth{0.000000pt}%
\definecolor{currentstroke}{rgb}{0.000000,0.000000,0.000000}%
\pgfsetstrokecolor{currentstroke}%
\pgfsetdash{}{0pt}%
\pgfpathmoveto{\pgfqpoint{2.855513in}{2.451418in}}%
\pgfpathlineto{\pgfqpoint{3.116513in}{2.603812in}}%
\pgfpathlineto{\pgfqpoint{2.855513in}{2.451418in}}%
\pgfpathclose%
\pgfusepath{fill}%
\end{pgfscope}%
\begin{pgfscope}%
\pgfpathrectangle{\pgfqpoint{0.017500in}{0.017500in}}{\pgfqpoint{6.218282in}{3.580372in}} %
\pgfusepath{clip}%
\pgfsetbuttcap%
\pgfsetroundjoin%
\definecolor{currentfill}{rgb}{0.805723,0.259813,0.230562}%
\pgfsetfillcolor{currentfill}%
\pgfsetlinewidth{0.000000pt}%
\definecolor{currentstroke}{rgb}{0.000000,0.000000,0.000000}%
\pgfsetstrokecolor{currentstroke}%
\pgfsetdash{}{0pt}%
\pgfpathmoveto{\pgfqpoint{2.855513in}{2.451418in}}%
\pgfpathlineto{\pgfqpoint{3.116513in}{2.603812in}}%
\pgfpathlineto{\pgfqpoint{2.855513in}{2.451418in}}%
\pgfpathclose%
\pgfusepath{fill}%
\end{pgfscope}%
\begin{pgfscope}%
\pgfpathrectangle{\pgfqpoint{0.017500in}{0.017500in}}{\pgfqpoint{6.218282in}{3.580372in}} %
\pgfusepath{clip}%
\pgfsetbuttcap%
\pgfsetroundjoin%
\definecolor{currentfill}{rgb}{0.805723,0.259813,0.230562}%
\pgfsetfillcolor{currentfill}%
\pgfsetlinewidth{0.000000pt}%
\definecolor{currentstroke}{rgb}{0.000000,0.000000,0.000000}%
\pgfsetstrokecolor{currentstroke}%
\pgfsetdash{}{0pt}%
\pgfpathmoveto{\pgfqpoint{2.855513in}{2.451418in}}%
\pgfpathlineto{\pgfqpoint{3.116513in}{2.603812in}}%
\pgfpathlineto{\pgfqpoint{2.855513in}{2.451418in}}%
\pgfpathclose%
\pgfusepath{fill}%
\end{pgfscope}%
\begin{pgfscope}%
\pgfpathrectangle{\pgfqpoint{0.017500in}{0.017500in}}{\pgfqpoint{6.218282in}{3.580372in}} %
\pgfusepath{clip}%
\pgfsetbuttcap%
\pgfsetroundjoin%
\definecolor{currentfill}{rgb}{0.805723,0.259813,0.230562}%
\pgfsetfillcolor{currentfill}%
\pgfsetlinewidth{0.000000pt}%
\definecolor{currentstroke}{rgb}{0.000000,0.000000,0.000000}%
\pgfsetstrokecolor{currentstroke}%
\pgfsetdash{}{0pt}%
\pgfpathmoveto{\pgfqpoint{2.855513in}{2.451418in}}%
\pgfpathlineto{\pgfqpoint{3.116513in}{2.603812in}}%
\pgfpathlineto{\pgfqpoint{2.855513in}{2.451418in}}%
\pgfpathclose%
\pgfusepath{fill}%
\end{pgfscope}%
\begin{pgfscope}%
\pgfpathrectangle{\pgfqpoint{0.017500in}{0.017500in}}{\pgfqpoint{6.218282in}{3.580372in}} %
\pgfusepath{clip}%
\pgfsetbuttcap%
\pgfsetroundjoin%
\definecolor{currentfill}{rgb}{0.805723,0.259813,0.230562}%
\pgfsetfillcolor{currentfill}%
\pgfsetlinewidth{0.000000pt}%
\definecolor{currentstroke}{rgb}{0.000000,0.000000,0.000000}%
\pgfsetstrokecolor{currentstroke}%
\pgfsetdash{}{0pt}%
\pgfpathmoveto{\pgfqpoint{2.855513in}{2.451418in}}%
\pgfpathlineto{\pgfqpoint{3.116513in}{2.603812in}}%
\pgfpathlineto{\pgfqpoint{2.855513in}{2.451418in}}%
\pgfpathclose%
\pgfusepath{fill}%
\end{pgfscope}%
\begin{pgfscope}%
\pgfpathrectangle{\pgfqpoint{0.017500in}{0.017500in}}{\pgfqpoint{6.218282in}{3.580372in}} %
\pgfusepath{clip}%
\pgfsetbuttcap%
\pgfsetroundjoin%
\definecolor{currentfill}{rgb}{0.805723,0.259813,0.230562}%
\pgfsetfillcolor{currentfill}%
\pgfsetlinewidth{0.000000pt}%
\definecolor{currentstroke}{rgb}{0.000000,0.000000,0.000000}%
\pgfsetstrokecolor{currentstroke}%
\pgfsetdash{}{0pt}%
\pgfpathmoveto{\pgfqpoint{2.855513in}{2.451418in}}%
\pgfpathlineto{\pgfqpoint{3.116513in}{2.603812in}}%
\pgfpathlineto{\pgfqpoint{2.855513in}{2.451418in}}%
\pgfpathclose%
\pgfusepath{fill}%
\end{pgfscope}%
\begin{pgfscope}%
\pgfpathrectangle{\pgfqpoint{0.017500in}{0.017500in}}{\pgfqpoint{6.218282in}{3.580372in}} %
\pgfusepath{clip}%
\pgfsetbuttcap%
\pgfsetroundjoin%
\definecolor{currentfill}{rgb}{0.805723,0.259813,0.230562}%
\pgfsetfillcolor{currentfill}%
\pgfsetlinewidth{0.000000pt}%
\definecolor{currentstroke}{rgb}{0.000000,0.000000,0.000000}%
\pgfsetstrokecolor{currentstroke}%
\pgfsetdash{}{0pt}%
\pgfpathmoveto{\pgfqpoint{2.855513in}{2.451418in}}%
\pgfpathlineto{\pgfqpoint{3.116513in}{2.603812in}}%
\pgfpathlineto{\pgfqpoint{2.855513in}{2.451418in}}%
\pgfpathclose%
\pgfusepath{fill}%
\end{pgfscope}%
\begin{pgfscope}%
\pgfpathrectangle{\pgfqpoint{0.017500in}{0.017500in}}{\pgfqpoint{6.218282in}{3.580372in}} %
\pgfusepath{clip}%
\pgfsetbuttcap%
\pgfsetroundjoin%
\definecolor{currentfill}{rgb}{0.800830,0.250829,0.225696}%
\pgfsetfillcolor{currentfill}%
\pgfsetlinewidth{0.000000pt}%
\definecolor{currentstroke}{rgb}{0.000000,0.000000,0.000000}%
\pgfsetstrokecolor{currentstroke}%
\pgfsetdash{}{0pt}%
\pgfpathmoveto{\pgfqpoint{2.855513in}{2.451418in}}%
\pgfpathlineto{\pgfqpoint{3.306485in}{2.382499in}}%
\pgfpathlineto{\pgfqpoint{3.564492in}{2.541257in}}%
\pgfpathlineto{\pgfqpoint{3.116513in}{2.603812in}}%
\pgfpathlineto{\pgfqpoint{2.855513in}{2.451418in}}%
\pgfpathclose%
\pgfusepath{fill}%
\end{pgfscope}%
\begin{pgfscope}%
\pgfpathrectangle{\pgfqpoint{0.017500in}{0.017500in}}{\pgfqpoint{6.218282in}{3.580372in}} %
\pgfusepath{clip}%
\pgfsetbuttcap%
\pgfsetroundjoin%
\definecolor{currentfill}{rgb}{0.800830,0.250829,0.225696}%
\pgfsetfillcolor{currentfill}%
\pgfsetlinewidth{0.000000pt}%
\definecolor{currentstroke}{rgb}{0.000000,0.000000,0.000000}%
\pgfsetstrokecolor{currentstroke}%
\pgfsetdash{}{0pt}%
\pgfpathmoveto{\pgfqpoint{2.855513in}{2.451418in}}%
\pgfpathlineto{\pgfqpoint{3.306485in}{2.382499in}}%
\pgfpathlineto{\pgfqpoint{3.564492in}{2.541257in}}%
\pgfpathlineto{\pgfqpoint{3.116513in}{2.603812in}}%
\pgfpathlineto{\pgfqpoint{2.855513in}{2.451418in}}%
\pgfpathclose%
\pgfusepath{fill}%
\end{pgfscope}%
\begin{pgfscope}%
\pgfpathrectangle{\pgfqpoint{0.017500in}{0.017500in}}{\pgfqpoint{6.218282in}{3.580372in}} %
\pgfusepath{clip}%
\pgfsetbuttcap%
\pgfsetroundjoin%
\definecolor{currentfill}{rgb}{0.800830,0.250829,0.225696}%
\pgfsetfillcolor{currentfill}%
\pgfsetlinewidth{0.000000pt}%
\definecolor{currentstroke}{rgb}{0.000000,0.000000,0.000000}%
\pgfsetstrokecolor{currentstroke}%
\pgfsetdash{}{0pt}%
\pgfpathmoveto{\pgfqpoint{2.855513in}{2.451418in}}%
\pgfpathlineto{\pgfqpoint{3.306485in}{2.382499in}}%
\pgfpathlineto{\pgfqpoint{3.564492in}{2.541257in}}%
\pgfpathlineto{\pgfqpoint{3.116513in}{2.603812in}}%
\pgfpathlineto{\pgfqpoint{2.855513in}{2.451418in}}%
\pgfpathclose%
\pgfusepath{fill}%
\end{pgfscope}%
\begin{pgfscope}%
\pgfpathrectangle{\pgfqpoint{0.017500in}{0.017500in}}{\pgfqpoint{6.218282in}{3.580372in}} %
\pgfusepath{clip}%
\pgfsetbuttcap%
\pgfsetroundjoin%
\definecolor{currentfill}{rgb}{0.800830,0.250829,0.225696}%
\pgfsetfillcolor{currentfill}%
\pgfsetlinewidth{0.000000pt}%
\definecolor{currentstroke}{rgb}{0.000000,0.000000,0.000000}%
\pgfsetstrokecolor{currentstroke}%
\pgfsetdash{}{0pt}%
\pgfpathmoveto{\pgfqpoint{2.855513in}{2.451418in}}%
\pgfpathlineto{\pgfqpoint{3.306485in}{2.382499in}}%
\pgfpathlineto{\pgfqpoint{3.564492in}{2.541257in}}%
\pgfpathlineto{\pgfqpoint{3.116513in}{2.603812in}}%
\pgfpathlineto{\pgfqpoint{2.855513in}{2.451418in}}%
\pgfpathclose%
\pgfusepath{fill}%
\end{pgfscope}%
\begin{pgfscope}%
\pgfpathrectangle{\pgfqpoint{0.017500in}{0.017500in}}{\pgfqpoint{6.218282in}{3.580372in}} %
\pgfusepath{clip}%
\pgfsetbuttcap%
\pgfsetroundjoin%
\definecolor{currentfill}{rgb}{0.800830,0.250829,0.225696}%
\pgfsetfillcolor{currentfill}%
\pgfsetlinewidth{0.000000pt}%
\definecolor{currentstroke}{rgb}{0.000000,0.000000,0.000000}%
\pgfsetstrokecolor{currentstroke}%
\pgfsetdash{}{0pt}%
\pgfpathmoveto{\pgfqpoint{2.855513in}{2.451418in}}%
\pgfpathlineto{\pgfqpoint{3.306485in}{2.382499in}}%
\pgfpathlineto{\pgfqpoint{3.564492in}{2.541257in}}%
\pgfpathlineto{\pgfqpoint{3.116513in}{2.603812in}}%
\pgfpathlineto{\pgfqpoint{2.855513in}{2.451418in}}%
\pgfpathclose%
\pgfusepath{fill}%
\end{pgfscope}%
\begin{pgfscope}%
\pgfpathrectangle{\pgfqpoint{0.017500in}{0.017500in}}{\pgfqpoint{6.218282in}{3.580372in}} %
\pgfusepath{clip}%
\pgfsetbuttcap%
\pgfsetroundjoin%
\definecolor{currentfill}{rgb}{0.800830,0.250829,0.225696}%
\pgfsetfillcolor{currentfill}%
\pgfsetlinewidth{0.000000pt}%
\definecolor{currentstroke}{rgb}{0.000000,0.000000,0.000000}%
\pgfsetstrokecolor{currentstroke}%
\pgfsetdash{}{0pt}%
\pgfpathmoveto{\pgfqpoint{2.855513in}{2.451418in}}%
\pgfpathlineto{\pgfqpoint{3.306485in}{2.382499in}}%
\pgfpathlineto{\pgfqpoint{3.564492in}{2.541257in}}%
\pgfpathlineto{\pgfqpoint{3.116513in}{2.603812in}}%
\pgfpathlineto{\pgfqpoint{2.855513in}{2.451418in}}%
\pgfpathclose%
\pgfusepath{fill}%
\end{pgfscope}%
\begin{pgfscope}%
\pgfpathrectangle{\pgfqpoint{0.017500in}{0.017500in}}{\pgfqpoint{6.218282in}{3.580372in}} %
\pgfusepath{clip}%
\pgfsetbuttcap%
\pgfsetroundjoin%
\definecolor{currentfill}{rgb}{0.234377,0.305542,0.759680}%
\pgfsetfillcolor{currentfill}%
\pgfsetlinewidth{0.000000pt}%
\definecolor{currentstroke}{rgb}{0.000000,0.000000,0.000000}%
\pgfsetstrokecolor{currentstroke}%
\pgfsetdash{}{0pt}%
\pgfpathmoveto{\pgfqpoint{2.789950in}{0.798346in}}%
\pgfpathlineto{\pgfqpoint{3.043326in}{2.141782in}}%
\pgfpathlineto{\pgfqpoint{2.789950in}{0.798346in}}%
\pgfpathclose%
\pgfusepath{fill}%
\end{pgfscope}%
\begin{pgfscope}%
\pgfpathrectangle{\pgfqpoint{0.017500in}{0.017500in}}{\pgfqpoint{6.218282in}{3.580372in}} %
\pgfusepath{clip}%
\pgfsetbuttcap%
\pgfsetroundjoin%
\definecolor{currentfill}{rgb}{0.234377,0.305542,0.759680}%
\pgfsetfillcolor{currentfill}%
\pgfsetlinewidth{0.000000pt}%
\definecolor{currentstroke}{rgb}{0.000000,0.000000,0.000000}%
\pgfsetstrokecolor{currentstroke}%
\pgfsetdash{}{0pt}%
\pgfpathmoveto{\pgfqpoint{2.789950in}{0.798346in}}%
\pgfpathlineto{\pgfqpoint{3.043326in}{2.141782in}}%
\pgfpathlineto{\pgfqpoint{2.789950in}{0.798346in}}%
\pgfpathclose%
\pgfusepath{fill}%
\end{pgfscope}%
\begin{pgfscope}%
\pgfpathrectangle{\pgfqpoint{0.017500in}{0.017500in}}{\pgfqpoint{6.218282in}{3.580372in}} %
\pgfusepath{clip}%
\pgfsetbuttcap%
\pgfsetroundjoin%
\definecolor{currentfill}{rgb}{0.234377,0.305542,0.759680}%
\pgfsetfillcolor{currentfill}%
\pgfsetlinewidth{0.000000pt}%
\definecolor{currentstroke}{rgb}{0.000000,0.000000,0.000000}%
\pgfsetstrokecolor{currentstroke}%
\pgfsetdash{}{0pt}%
\pgfpathmoveto{\pgfqpoint{2.789950in}{0.798346in}}%
\pgfpathlineto{\pgfqpoint{3.043326in}{2.141782in}}%
\pgfpathlineto{\pgfqpoint{2.789950in}{0.798346in}}%
\pgfpathclose%
\pgfusepath{fill}%
\end{pgfscope}%
\begin{pgfscope}%
\pgfpathrectangle{\pgfqpoint{0.017500in}{0.017500in}}{\pgfqpoint{6.218282in}{3.580372in}} %
\pgfusepath{clip}%
\pgfsetbuttcap%
\pgfsetroundjoin%
\definecolor{currentfill}{rgb}{0.234377,0.305542,0.759680}%
\pgfsetfillcolor{currentfill}%
\pgfsetlinewidth{0.000000pt}%
\definecolor{currentstroke}{rgb}{0.000000,0.000000,0.000000}%
\pgfsetstrokecolor{currentstroke}%
\pgfsetdash{}{0pt}%
\pgfpathmoveto{\pgfqpoint{2.789950in}{0.798346in}}%
\pgfpathlineto{\pgfqpoint{3.043326in}{2.141782in}}%
\pgfpathlineto{\pgfqpoint{2.789950in}{0.798346in}}%
\pgfpathclose%
\pgfusepath{fill}%
\end{pgfscope}%
\begin{pgfscope}%
\pgfpathrectangle{\pgfqpoint{0.017500in}{0.017500in}}{\pgfqpoint{6.218282in}{3.580372in}} %
\pgfusepath{clip}%
\pgfsetbuttcap%
\pgfsetroundjoin%
\definecolor{currentfill}{rgb}{0.234377,0.305542,0.759680}%
\pgfsetfillcolor{currentfill}%
\pgfsetlinewidth{0.000000pt}%
\definecolor{currentstroke}{rgb}{0.000000,0.000000,0.000000}%
\pgfsetstrokecolor{currentstroke}%
\pgfsetdash{}{0pt}%
\pgfpathmoveto{\pgfqpoint{2.789950in}{0.798346in}}%
\pgfpathlineto{\pgfqpoint{3.043326in}{2.141782in}}%
\pgfpathlineto{\pgfqpoint{2.789950in}{0.798346in}}%
\pgfpathclose%
\pgfusepath{fill}%
\end{pgfscope}%
\begin{pgfscope}%
\pgfpathrectangle{\pgfqpoint{0.017500in}{0.017500in}}{\pgfqpoint{6.218282in}{3.580372in}} %
\pgfusepath{clip}%
\pgfsetbuttcap%
\pgfsetroundjoin%
\definecolor{currentfill}{rgb}{0.234377,0.305542,0.759680}%
\pgfsetfillcolor{currentfill}%
\pgfsetlinewidth{0.000000pt}%
\definecolor{currentstroke}{rgb}{0.000000,0.000000,0.000000}%
\pgfsetstrokecolor{currentstroke}%
\pgfsetdash{}{0pt}%
\pgfpathmoveto{\pgfqpoint{2.789950in}{0.798346in}}%
\pgfpathlineto{\pgfqpoint{3.043326in}{2.141782in}}%
\pgfpathlineto{\pgfqpoint{2.789950in}{0.798346in}}%
\pgfpathclose%
\pgfusepath{fill}%
\end{pgfscope}%
\begin{pgfscope}%
\pgfpathrectangle{\pgfqpoint{0.017500in}{0.017500in}}{\pgfqpoint{6.218282in}{3.580372in}} %
\pgfusepath{clip}%
\pgfsetbuttcap%
\pgfsetroundjoin%
\definecolor{currentfill}{rgb}{0.234377,0.305542,0.759680}%
\pgfsetfillcolor{currentfill}%
\pgfsetlinewidth{0.000000pt}%
\definecolor{currentstroke}{rgb}{0.000000,0.000000,0.000000}%
\pgfsetstrokecolor{currentstroke}%
\pgfsetdash{}{0pt}%
\pgfpathmoveto{\pgfqpoint{2.789950in}{0.798346in}}%
\pgfpathlineto{\pgfqpoint{3.043326in}{2.141782in}}%
\pgfpathlineto{\pgfqpoint{2.789950in}{0.798346in}}%
\pgfpathclose%
\pgfusepath{fill}%
\end{pgfscope}%
\begin{pgfscope}%
\pgfpathrectangle{\pgfqpoint{0.017500in}{0.017500in}}{\pgfqpoint{6.218282in}{3.580372in}} %
\pgfusepath{clip}%
\pgfsetbuttcap%
\pgfsetroundjoin%
\definecolor{currentfill}{rgb}{0.234377,0.305542,0.759680}%
\pgfsetfillcolor{currentfill}%
\pgfsetlinewidth{0.000000pt}%
\definecolor{currentstroke}{rgb}{0.000000,0.000000,0.000000}%
\pgfsetstrokecolor{currentstroke}%
\pgfsetdash{}{0pt}%
\pgfpathmoveto{\pgfqpoint{2.789950in}{0.798346in}}%
\pgfpathlineto{\pgfqpoint{3.043326in}{2.141782in}}%
\pgfpathlineto{\pgfqpoint{2.789950in}{0.798346in}}%
\pgfpathclose%
\pgfusepath{fill}%
\end{pgfscope}%
\begin{pgfscope}%
\pgfpathrectangle{\pgfqpoint{0.017500in}{0.017500in}}{\pgfqpoint{6.218282in}{3.580372in}} %
\pgfusepath{clip}%
\pgfsetbuttcap%
\pgfsetroundjoin%
\definecolor{currentfill}{rgb}{0.234377,0.305542,0.759680}%
\pgfsetfillcolor{currentfill}%
\pgfsetlinewidth{0.000000pt}%
\definecolor{currentstroke}{rgb}{0.000000,0.000000,0.000000}%
\pgfsetstrokecolor{currentstroke}%
\pgfsetdash{}{0pt}%
\pgfpathmoveto{\pgfqpoint{2.789950in}{0.798346in}}%
\pgfpathlineto{\pgfqpoint{3.043326in}{2.141782in}}%
\pgfpathlineto{\pgfqpoint{2.789950in}{0.798346in}}%
\pgfpathclose%
\pgfusepath{fill}%
\end{pgfscope}%
\begin{pgfscope}%
\pgfpathrectangle{\pgfqpoint{0.017500in}{0.017500in}}{\pgfqpoint{6.218282in}{3.580372in}} %
\pgfusepath{clip}%
\pgfsetbuttcap%
\pgfsetroundjoin%
\definecolor{currentfill}{rgb}{0.234377,0.305542,0.759680}%
\pgfsetfillcolor{currentfill}%
\pgfsetlinewidth{0.000000pt}%
\definecolor{currentstroke}{rgb}{0.000000,0.000000,0.000000}%
\pgfsetstrokecolor{currentstroke}%
\pgfsetdash{}{0pt}%
\pgfpathmoveto{\pgfqpoint{2.789950in}{0.798346in}}%
\pgfpathlineto{\pgfqpoint{3.043326in}{2.141782in}}%
\pgfpathlineto{\pgfqpoint{2.789950in}{0.798346in}}%
\pgfpathclose%
\pgfusepath{fill}%
\end{pgfscope}%
\begin{pgfscope}%
\pgfpathrectangle{\pgfqpoint{0.017500in}{0.017500in}}{\pgfqpoint{6.218282in}{3.580372in}} %
\pgfusepath{clip}%
\pgfsetbuttcap%
\pgfsetroundjoin%
\definecolor{currentfill}{rgb}{0.234377,0.305542,0.759680}%
\pgfsetfillcolor{currentfill}%
\pgfsetlinewidth{0.000000pt}%
\definecolor{currentstroke}{rgb}{0.000000,0.000000,0.000000}%
\pgfsetstrokecolor{currentstroke}%
\pgfsetdash{}{0pt}%
\pgfpathmoveto{\pgfqpoint{2.789950in}{0.798346in}}%
\pgfpathlineto{\pgfqpoint{3.043326in}{2.141782in}}%
\pgfpathlineto{\pgfqpoint{2.789950in}{0.798346in}}%
\pgfpathclose%
\pgfusepath{fill}%
\end{pgfscope}%
\begin{pgfscope}%
\pgfpathrectangle{\pgfqpoint{0.017500in}{0.017500in}}{\pgfqpoint{6.218282in}{3.580372in}} %
\pgfusepath{clip}%
\pgfsetbuttcap%
\pgfsetroundjoin%
\definecolor{currentfill}{rgb}{0.234377,0.305542,0.759680}%
\pgfsetfillcolor{currentfill}%
\pgfsetlinewidth{0.000000pt}%
\definecolor{currentstroke}{rgb}{0.000000,0.000000,0.000000}%
\pgfsetstrokecolor{currentstroke}%
\pgfsetdash{}{0pt}%
\pgfpathmoveto{\pgfqpoint{2.789950in}{0.798346in}}%
\pgfpathlineto{\pgfqpoint{3.043326in}{2.141782in}}%
\pgfpathlineto{\pgfqpoint{2.789950in}{0.798346in}}%
\pgfpathclose%
\pgfusepath{fill}%
\end{pgfscope}%
\begin{pgfscope}%
\pgfpathrectangle{\pgfqpoint{0.017500in}{0.017500in}}{\pgfqpoint{6.218282in}{3.580372in}} %
\pgfusepath{clip}%
\pgfsetbuttcap%
\pgfsetroundjoin%
\definecolor{currentfill}{rgb}{0.234377,0.305542,0.759680}%
\pgfsetfillcolor{currentfill}%
\pgfsetlinewidth{0.000000pt}%
\definecolor{currentstroke}{rgb}{0.000000,0.000000,0.000000}%
\pgfsetstrokecolor{currentstroke}%
\pgfsetdash{}{0pt}%
\pgfpathmoveto{\pgfqpoint{2.789950in}{0.798346in}}%
\pgfpathlineto{\pgfqpoint{3.043326in}{2.141782in}}%
\pgfpathlineto{\pgfqpoint{2.789950in}{0.798346in}}%
\pgfpathclose%
\pgfusepath{fill}%
\end{pgfscope}%
\begin{pgfscope}%
\pgfpathrectangle{\pgfqpoint{0.017500in}{0.017500in}}{\pgfqpoint{6.218282in}{3.580372in}} %
\pgfusepath{clip}%
\pgfsetbuttcap%
\pgfsetroundjoin%
\definecolor{currentfill}{rgb}{0.234377,0.305542,0.759680}%
\pgfsetfillcolor{currentfill}%
\pgfsetlinewidth{0.000000pt}%
\definecolor{currentstroke}{rgb}{0.000000,0.000000,0.000000}%
\pgfsetstrokecolor{currentstroke}%
\pgfsetdash{}{0pt}%
\pgfpathmoveto{\pgfqpoint{2.789950in}{0.798346in}}%
\pgfpathlineto{\pgfqpoint{3.043326in}{2.141782in}}%
\pgfpathlineto{\pgfqpoint{2.789950in}{0.798346in}}%
\pgfpathclose%
\pgfusepath{fill}%
\end{pgfscope}%
\begin{pgfscope}%
\pgfpathrectangle{\pgfqpoint{0.017500in}{0.017500in}}{\pgfqpoint{6.218282in}{3.580372in}} %
\pgfusepath{clip}%
\pgfsetbuttcap%
\pgfsetroundjoin%
\definecolor{currentfill}{rgb}{0.234377,0.305542,0.759680}%
\pgfsetfillcolor{currentfill}%
\pgfsetlinewidth{0.000000pt}%
\definecolor{currentstroke}{rgb}{0.000000,0.000000,0.000000}%
\pgfsetstrokecolor{currentstroke}%
\pgfsetdash{}{0pt}%
\pgfpathmoveto{\pgfqpoint{2.789950in}{0.798346in}}%
\pgfpathlineto{\pgfqpoint{3.043326in}{2.141782in}}%
\pgfpathlineto{\pgfqpoint{2.789950in}{0.798346in}}%
\pgfpathclose%
\pgfusepath{fill}%
\end{pgfscope}%
\begin{pgfscope}%
\pgfpathrectangle{\pgfqpoint{0.017500in}{0.017500in}}{\pgfqpoint{6.218282in}{3.580372in}} %
\pgfusepath{clip}%
\pgfsetbuttcap%
\pgfsetroundjoin%
\definecolor{currentfill}{rgb}{0.234377,0.305542,0.759680}%
\pgfsetfillcolor{currentfill}%
\pgfsetlinewidth{0.000000pt}%
\definecolor{currentstroke}{rgb}{0.000000,0.000000,0.000000}%
\pgfsetstrokecolor{currentstroke}%
\pgfsetdash{}{0pt}%
\pgfpathmoveto{\pgfqpoint{2.789950in}{0.798346in}}%
\pgfpathlineto{\pgfqpoint{3.043326in}{2.141782in}}%
\pgfpathlineto{\pgfqpoint{2.789950in}{0.798346in}}%
\pgfpathclose%
\pgfusepath{fill}%
\end{pgfscope}%
\begin{pgfscope}%
\pgfpathrectangle{\pgfqpoint{0.017500in}{0.017500in}}{\pgfqpoint{6.218282in}{3.580372in}} %
\pgfusepath{clip}%
\pgfsetbuttcap%
\pgfsetroundjoin%
\definecolor{currentfill}{rgb}{0.234377,0.305542,0.759680}%
\pgfsetfillcolor{currentfill}%
\pgfsetlinewidth{0.000000pt}%
\definecolor{currentstroke}{rgb}{0.000000,0.000000,0.000000}%
\pgfsetstrokecolor{currentstroke}%
\pgfsetdash{}{0pt}%
\pgfpathmoveto{\pgfqpoint{2.789950in}{0.798346in}}%
\pgfpathlineto{\pgfqpoint{3.043326in}{2.141782in}}%
\pgfpathlineto{\pgfqpoint{2.789950in}{0.798346in}}%
\pgfpathclose%
\pgfusepath{fill}%
\end{pgfscope}%
\begin{pgfscope}%
\pgfpathrectangle{\pgfqpoint{0.017500in}{0.017500in}}{\pgfqpoint{6.218282in}{3.580372in}} %
\pgfusepath{clip}%
\pgfsetbuttcap%
\pgfsetroundjoin%
\definecolor{currentfill}{rgb}{0.234377,0.305542,0.759680}%
\pgfsetfillcolor{currentfill}%
\pgfsetlinewidth{0.000000pt}%
\definecolor{currentstroke}{rgb}{0.000000,0.000000,0.000000}%
\pgfsetstrokecolor{currentstroke}%
\pgfsetdash{}{0pt}%
\pgfpathmoveto{\pgfqpoint{2.789950in}{0.798346in}}%
\pgfpathlineto{\pgfqpoint{3.043326in}{2.141782in}}%
\pgfpathlineto{\pgfqpoint{2.789950in}{0.798346in}}%
\pgfpathclose%
\pgfusepath{fill}%
\end{pgfscope}%
\begin{pgfscope}%
\pgfpathrectangle{\pgfqpoint{0.017500in}{0.017500in}}{\pgfqpoint{6.218282in}{3.580372in}} %
\pgfusepath{clip}%
\pgfsetbuttcap%
\pgfsetroundjoin%
\definecolor{currentfill}{rgb}{0.234377,0.305542,0.759680}%
\pgfsetfillcolor{currentfill}%
\pgfsetlinewidth{0.000000pt}%
\definecolor{currentstroke}{rgb}{0.000000,0.000000,0.000000}%
\pgfsetstrokecolor{currentstroke}%
\pgfsetdash{}{0pt}%
\pgfpathmoveto{\pgfqpoint{2.789950in}{0.798346in}}%
\pgfpathlineto{\pgfqpoint{3.043326in}{2.141782in}}%
\pgfpathlineto{\pgfqpoint{2.789950in}{0.798346in}}%
\pgfpathclose%
\pgfusepath{fill}%
\end{pgfscope}%
\begin{pgfscope}%
\pgfpathrectangle{\pgfqpoint{0.017500in}{0.017500in}}{\pgfqpoint{6.218282in}{3.580372in}} %
\pgfusepath{clip}%
\pgfsetbuttcap%
\pgfsetroundjoin%
\definecolor{currentfill}{rgb}{0.234377,0.305542,0.759680}%
\pgfsetfillcolor{currentfill}%
\pgfsetlinewidth{0.000000pt}%
\definecolor{currentstroke}{rgb}{0.000000,0.000000,0.000000}%
\pgfsetstrokecolor{currentstroke}%
\pgfsetdash{}{0pt}%
\pgfpathmoveto{\pgfqpoint{2.789950in}{0.798346in}}%
\pgfpathlineto{\pgfqpoint{3.043326in}{2.141782in}}%
\pgfpathlineto{\pgfqpoint{2.789950in}{0.798346in}}%
\pgfpathclose%
\pgfusepath{fill}%
\end{pgfscope}%
\begin{pgfscope}%
\pgfpathrectangle{\pgfqpoint{0.017500in}{0.017500in}}{\pgfqpoint{6.218282in}{3.580372in}} %
\pgfusepath{clip}%
\pgfsetbuttcap%
\pgfsetroundjoin%
\definecolor{currentfill}{rgb}{0.234377,0.305542,0.759680}%
\pgfsetfillcolor{currentfill}%
\pgfsetlinewidth{0.000000pt}%
\definecolor{currentstroke}{rgb}{0.000000,0.000000,0.000000}%
\pgfsetstrokecolor{currentstroke}%
\pgfsetdash{}{0pt}%
\pgfpathmoveto{\pgfqpoint{2.789950in}{0.798346in}}%
\pgfpathlineto{\pgfqpoint{3.043326in}{2.141782in}}%
\pgfpathlineto{\pgfqpoint{2.789950in}{0.798346in}}%
\pgfpathclose%
\pgfusepath{fill}%
\end{pgfscope}%
\begin{pgfscope}%
\pgfpathrectangle{\pgfqpoint{0.017500in}{0.017500in}}{\pgfqpoint{6.218282in}{3.580372in}} %
\pgfusepath{clip}%
\pgfsetbuttcap%
\pgfsetroundjoin%
\definecolor{currentfill}{rgb}{0.234377,0.305542,0.759680}%
\pgfsetfillcolor{currentfill}%
\pgfsetlinewidth{0.000000pt}%
\definecolor{currentstroke}{rgb}{0.000000,0.000000,0.000000}%
\pgfsetstrokecolor{currentstroke}%
\pgfsetdash{}{0pt}%
\pgfpathmoveto{\pgfqpoint{2.789950in}{0.798346in}}%
\pgfpathlineto{\pgfqpoint{3.043326in}{2.141782in}}%
\pgfpathlineto{\pgfqpoint{2.789950in}{0.798346in}}%
\pgfpathclose%
\pgfusepath{fill}%
\end{pgfscope}%
\begin{pgfscope}%
\pgfpathrectangle{\pgfqpoint{0.017500in}{0.017500in}}{\pgfqpoint{6.218282in}{3.580372in}} %
\pgfusepath{clip}%
\pgfsetbuttcap%
\pgfsetroundjoin%
\definecolor{currentfill}{rgb}{0.234377,0.305542,0.759680}%
\pgfsetfillcolor{currentfill}%
\pgfsetlinewidth{0.000000pt}%
\definecolor{currentstroke}{rgb}{0.000000,0.000000,0.000000}%
\pgfsetstrokecolor{currentstroke}%
\pgfsetdash{}{0pt}%
\pgfpathmoveto{\pgfqpoint{2.789950in}{0.798346in}}%
\pgfpathlineto{\pgfqpoint{3.043326in}{2.141782in}}%
\pgfpathlineto{\pgfqpoint{2.789950in}{0.798346in}}%
\pgfpathclose%
\pgfusepath{fill}%
\end{pgfscope}%
\begin{pgfscope}%
\pgfpathrectangle{\pgfqpoint{0.017500in}{0.017500in}}{\pgfqpoint{6.218282in}{3.580372in}} %
\pgfusepath{clip}%
\pgfsetbuttcap%
\pgfsetroundjoin%
\definecolor{currentfill}{rgb}{0.234377,0.305542,0.759680}%
\pgfsetfillcolor{currentfill}%
\pgfsetlinewidth{0.000000pt}%
\definecolor{currentstroke}{rgb}{0.000000,0.000000,0.000000}%
\pgfsetstrokecolor{currentstroke}%
\pgfsetdash{}{0pt}%
\pgfpathmoveto{\pgfqpoint{2.789950in}{0.798346in}}%
\pgfpathlineto{\pgfqpoint{3.043326in}{2.141782in}}%
\pgfpathlineto{\pgfqpoint{2.789950in}{0.798346in}}%
\pgfpathclose%
\pgfusepath{fill}%
\end{pgfscope}%
\begin{pgfscope}%
\pgfpathrectangle{\pgfqpoint{0.017500in}{0.017500in}}{\pgfqpoint{6.218282in}{3.580372in}} %
\pgfusepath{clip}%
\pgfsetbuttcap%
\pgfsetroundjoin%
\definecolor{currentfill}{rgb}{0.234377,0.305542,0.759680}%
\pgfsetfillcolor{currentfill}%
\pgfsetlinewidth{0.000000pt}%
\definecolor{currentstroke}{rgb}{0.000000,0.000000,0.000000}%
\pgfsetstrokecolor{currentstroke}%
\pgfsetdash{}{0pt}%
\pgfpathmoveto{\pgfqpoint{2.789950in}{0.798346in}}%
\pgfpathlineto{\pgfqpoint{3.043326in}{2.141782in}}%
\pgfpathlineto{\pgfqpoint{2.789950in}{0.798346in}}%
\pgfpathclose%
\pgfusepath{fill}%
\end{pgfscope}%
\begin{pgfscope}%
\pgfpathrectangle{\pgfqpoint{0.017500in}{0.017500in}}{\pgfqpoint{6.218282in}{3.580372in}} %
\pgfusepath{clip}%
\pgfsetbuttcap%
\pgfsetroundjoin%
\definecolor{currentfill}{rgb}{0.234377,0.305542,0.759680}%
\pgfsetfillcolor{currentfill}%
\pgfsetlinewidth{0.000000pt}%
\definecolor{currentstroke}{rgb}{0.000000,0.000000,0.000000}%
\pgfsetstrokecolor{currentstroke}%
\pgfsetdash{}{0pt}%
\pgfpathmoveto{\pgfqpoint{2.789950in}{0.798346in}}%
\pgfpathlineto{\pgfqpoint{3.043326in}{2.141782in}}%
\pgfpathlineto{\pgfqpoint{2.789950in}{0.798346in}}%
\pgfpathclose%
\pgfusepath{fill}%
\end{pgfscope}%
\begin{pgfscope}%
\pgfpathrectangle{\pgfqpoint{0.017500in}{0.017500in}}{\pgfqpoint{6.218282in}{3.580372in}} %
\pgfusepath{clip}%
\pgfsetbuttcap%
\pgfsetroundjoin%
\definecolor{currentfill}{rgb}{0.234377,0.305542,0.759680}%
\pgfsetfillcolor{currentfill}%
\pgfsetlinewidth{0.000000pt}%
\definecolor{currentstroke}{rgb}{0.000000,0.000000,0.000000}%
\pgfsetstrokecolor{currentstroke}%
\pgfsetdash{}{0pt}%
\pgfpathmoveto{\pgfqpoint{2.789950in}{0.798346in}}%
\pgfpathlineto{\pgfqpoint{3.043326in}{2.141782in}}%
\pgfpathlineto{\pgfqpoint{2.789950in}{0.798346in}}%
\pgfpathclose%
\pgfusepath{fill}%
\end{pgfscope}%
\begin{pgfscope}%
\pgfpathrectangle{\pgfqpoint{0.017500in}{0.017500in}}{\pgfqpoint{6.218282in}{3.580372in}} %
\pgfusepath{clip}%
\pgfsetbuttcap%
\pgfsetroundjoin%
\definecolor{currentfill}{rgb}{0.234377,0.305542,0.759680}%
\pgfsetfillcolor{currentfill}%
\pgfsetlinewidth{0.000000pt}%
\definecolor{currentstroke}{rgb}{0.000000,0.000000,0.000000}%
\pgfsetstrokecolor{currentstroke}%
\pgfsetdash{}{0pt}%
\pgfpathmoveto{\pgfqpoint{2.789950in}{0.798346in}}%
\pgfpathlineto{\pgfqpoint{3.043326in}{2.141782in}}%
\pgfpathlineto{\pgfqpoint{2.789950in}{0.798346in}}%
\pgfpathclose%
\pgfusepath{fill}%
\end{pgfscope}%
\begin{pgfscope}%
\pgfpathrectangle{\pgfqpoint{0.017500in}{0.017500in}}{\pgfqpoint{6.218282in}{3.580372in}} %
\pgfusepath{clip}%
\pgfsetbuttcap%
\pgfsetroundjoin%
\definecolor{currentfill}{rgb}{0.234377,0.305542,0.759680}%
\pgfsetfillcolor{currentfill}%
\pgfsetlinewidth{0.000000pt}%
\definecolor{currentstroke}{rgb}{0.000000,0.000000,0.000000}%
\pgfsetstrokecolor{currentstroke}%
\pgfsetdash{}{0pt}%
\pgfpathmoveto{\pgfqpoint{2.789950in}{0.798346in}}%
\pgfpathlineto{\pgfqpoint{3.043326in}{2.141782in}}%
\pgfpathlineto{\pgfqpoint{2.789950in}{0.798346in}}%
\pgfpathclose%
\pgfusepath{fill}%
\end{pgfscope}%
\begin{pgfscope}%
\pgfpathrectangle{\pgfqpoint{0.017500in}{0.017500in}}{\pgfqpoint{6.218282in}{3.580372in}} %
\pgfusepath{clip}%
\pgfsetbuttcap%
\pgfsetroundjoin%
\definecolor{currentfill}{rgb}{0.234377,0.305542,0.759680}%
\pgfsetfillcolor{currentfill}%
\pgfsetlinewidth{0.000000pt}%
\definecolor{currentstroke}{rgb}{0.000000,0.000000,0.000000}%
\pgfsetstrokecolor{currentstroke}%
\pgfsetdash{}{0pt}%
\pgfpathmoveto{\pgfqpoint{2.789950in}{0.798346in}}%
\pgfpathlineto{\pgfqpoint{3.043326in}{2.141782in}}%
\pgfpathlineto{\pgfqpoint{2.789950in}{0.798346in}}%
\pgfpathclose%
\pgfusepath{fill}%
\end{pgfscope}%
\begin{pgfscope}%
\pgfpathrectangle{\pgfqpoint{0.017500in}{0.017500in}}{\pgfqpoint{6.218282in}{3.580372in}} %
\pgfusepath{clip}%
\pgfsetbuttcap%
\pgfsetroundjoin%
\definecolor{currentfill}{rgb}{0.899534,0.440692,0.344107}%
\pgfsetfillcolor{currentfill}%
\pgfsetlinewidth{0.000000pt}%
\definecolor{currentstroke}{rgb}{0.000000,0.000000,0.000000}%
\pgfsetstrokecolor{currentstroke}%
\pgfsetdash{}{0pt}%
\pgfpathmoveto{\pgfqpoint{2.143341in}{2.291288in}}%
\pgfpathlineto{\pgfqpoint{2.410872in}{2.508083in}}%
\pgfpathlineto{\pgfqpoint{2.143341in}{2.291288in}}%
\pgfpathclose%
\pgfusepath{fill}%
\end{pgfscope}%
\begin{pgfscope}%
\pgfpathrectangle{\pgfqpoint{0.017500in}{0.017500in}}{\pgfqpoint{6.218282in}{3.580372in}} %
\pgfusepath{clip}%
\pgfsetbuttcap%
\pgfsetroundjoin%
\definecolor{currentfill}{rgb}{0.899534,0.440692,0.344107}%
\pgfsetfillcolor{currentfill}%
\pgfsetlinewidth{0.000000pt}%
\definecolor{currentstroke}{rgb}{0.000000,0.000000,0.000000}%
\pgfsetstrokecolor{currentstroke}%
\pgfsetdash{}{0pt}%
\pgfpathmoveto{\pgfqpoint{2.143341in}{2.291288in}}%
\pgfpathlineto{\pgfqpoint{2.410872in}{2.508083in}}%
\pgfpathlineto{\pgfqpoint{2.143341in}{2.291288in}}%
\pgfpathclose%
\pgfusepath{fill}%
\end{pgfscope}%
\begin{pgfscope}%
\pgfpathrectangle{\pgfqpoint{0.017500in}{0.017500in}}{\pgfqpoint{6.218282in}{3.580372in}} %
\pgfusepath{clip}%
\pgfsetbuttcap%
\pgfsetroundjoin%
\definecolor{currentfill}{rgb}{0.899534,0.440692,0.344107}%
\pgfsetfillcolor{currentfill}%
\pgfsetlinewidth{0.000000pt}%
\definecolor{currentstroke}{rgb}{0.000000,0.000000,0.000000}%
\pgfsetstrokecolor{currentstroke}%
\pgfsetdash{}{0pt}%
\pgfpathmoveto{\pgfqpoint{2.143341in}{2.291288in}}%
\pgfpathlineto{\pgfqpoint{2.410872in}{2.508083in}}%
\pgfpathlineto{\pgfqpoint{2.143341in}{2.291288in}}%
\pgfpathclose%
\pgfusepath{fill}%
\end{pgfscope}%
\begin{pgfscope}%
\pgfpathrectangle{\pgfqpoint{0.017500in}{0.017500in}}{\pgfqpoint{6.218282in}{3.580372in}} %
\pgfusepath{clip}%
\pgfsetbuttcap%
\pgfsetroundjoin%
\definecolor{currentfill}{rgb}{0.899534,0.440692,0.344107}%
\pgfsetfillcolor{currentfill}%
\pgfsetlinewidth{0.000000pt}%
\definecolor{currentstroke}{rgb}{0.000000,0.000000,0.000000}%
\pgfsetstrokecolor{currentstroke}%
\pgfsetdash{}{0pt}%
\pgfpathmoveto{\pgfqpoint{2.143341in}{2.291288in}}%
\pgfpathlineto{\pgfqpoint{2.410872in}{2.508083in}}%
\pgfpathlineto{\pgfqpoint{2.143341in}{2.291288in}}%
\pgfpathclose%
\pgfusepath{fill}%
\end{pgfscope}%
\begin{pgfscope}%
\pgfpathrectangle{\pgfqpoint{0.017500in}{0.017500in}}{\pgfqpoint{6.218282in}{3.580372in}} %
\pgfusepath{clip}%
\pgfsetbuttcap%
\pgfsetroundjoin%
\definecolor{currentfill}{rgb}{0.899534,0.440692,0.344107}%
\pgfsetfillcolor{currentfill}%
\pgfsetlinewidth{0.000000pt}%
\definecolor{currentstroke}{rgb}{0.000000,0.000000,0.000000}%
\pgfsetstrokecolor{currentstroke}%
\pgfsetdash{}{0pt}%
\pgfpathmoveto{\pgfqpoint{2.143341in}{2.291288in}}%
\pgfpathlineto{\pgfqpoint{2.410872in}{2.508083in}}%
\pgfpathlineto{\pgfqpoint{2.143341in}{2.291288in}}%
\pgfpathclose%
\pgfusepath{fill}%
\end{pgfscope}%
\begin{pgfscope}%
\pgfpathrectangle{\pgfqpoint{0.017500in}{0.017500in}}{\pgfqpoint{6.218282in}{3.580372in}} %
\pgfusepath{clip}%
\pgfsetbuttcap%
\pgfsetroundjoin%
\definecolor{currentfill}{rgb}{0.899534,0.440692,0.344107}%
\pgfsetfillcolor{currentfill}%
\pgfsetlinewidth{0.000000pt}%
\definecolor{currentstroke}{rgb}{0.000000,0.000000,0.000000}%
\pgfsetstrokecolor{currentstroke}%
\pgfsetdash{}{0pt}%
\pgfpathmoveto{\pgfqpoint{2.143341in}{2.291288in}}%
\pgfpathlineto{\pgfqpoint{2.410872in}{2.508083in}}%
\pgfpathlineto{\pgfqpoint{2.143341in}{2.291288in}}%
\pgfpathclose%
\pgfusepath{fill}%
\end{pgfscope}%
\begin{pgfscope}%
\pgfpathrectangle{\pgfqpoint{0.017500in}{0.017500in}}{\pgfqpoint{6.218282in}{3.580372in}} %
\pgfusepath{clip}%
\pgfsetbuttcap%
\pgfsetroundjoin%
\definecolor{currentfill}{rgb}{0.899534,0.440692,0.344107}%
\pgfsetfillcolor{currentfill}%
\pgfsetlinewidth{0.000000pt}%
\definecolor{currentstroke}{rgb}{0.000000,0.000000,0.000000}%
\pgfsetstrokecolor{currentstroke}%
\pgfsetdash{}{0pt}%
\pgfpathmoveto{\pgfqpoint{2.143341in}{2.291288in}}%
\pgfpathlineto{\pgfqpoint{2.410872in}{2.508083in}}%
\pgfpathlineto{\pgfqpoint{2.143341in}{2.291288in}}%
\pgfpathclose%
\pgfusepath{fill}%
\end{pgfscope}%
\begin{pgfscope}%
\pgfpathrectangle{\pgfqpoint{0.017500in}{0.017500in}}{\pgfqpoint{6.218282in}{3.580372in}} %
\pgfusepath{clip}%
\pgfsetbuttcap%
\pgfsetroundjoin%
\definecolor{currentfill}{rgb}{0.899534,0.440692,0.344107}%
\pgfsetfillcolor{currentfill}%
\pgfsetlinewidth{0.000000pt}%
\definecolor{currentstroke}{rgb}{0.000000,0.000000,0.000000}%
\pgfsetstrokecolor{currentstroke}%
\pgfsetdash{}{0pt}%
\pgfpathmoveto{\pgfqpoint{2.143341in}{2.291288in}}%
\pgfpathlineto{\pgfqpoint{2.410872in}{2.508083in}}%
\pgfpathlineto{\pgfqpoint{2.143341in}{2.291288in}}%
\pgfpathclose%
\pgfusepath{fill}%
\end{pgfscope}%
\begin{pgfscope}%
\pgfpathrectangle{\pgfqpoint{0.017500in}{0.017500in}}{\pgfqpoint{6.218282in}{3.580372in}} %
\pgfusepath{clip}%
\pgfsetbuttcap%
\pgfsetroundjoin%
\definecolor{currentfill}{rgb}{0.899534,0.440692,0.344107}%
\pgfsetfillcolor{currentfill}%
\pgfsetlinewidth{0.000000pt}%
\definecolor{currentstroke}{rgb}{0.000000,0.000000,0.000000}%
\pgfsetstrokecolor{currentstroke}%
\pgfsetdash{}{0pt}%
\pgfpathmoveto{\pgfqpoint{2.143341in}{2.291288in}}%
\pgfpathlineto{\pgfqpoint{2.410872in}{2.508083in}}%
\pgfpathlineto{\pgfqpoint{2.143341in}{2.291288in}}%
\pgfpathclose%
\pgfusepath{fill}%
\end{pgfscope}%
\begin{pgfscope}%
\pgfpathrectangle{\pgfqpoint{0.017500in}{0.017500in}}{\pgfqpoint{6.218282in}{3.580372in}} %
\pgfusepath{clip}%
\pgfsetbuttcap%
\pgfsetroundjoin%
\definecolor{currentfill}{rgb}{0.899534,0.440692,0.344107}%
\pgfsetfillcolor{currentfill}%
\pgfsetlinewidth{0.000000pt}%
\definecolor{currentstroke}{rgb}{0.000000,0.000000,0.000000}%
\pgfsetstrokecolor{currentstroke}%
\pgfsetdash{}{0pt}%
\pgfpathmoveto{\pgfqpoint{2.143341in}{2.291288in}}%
\pgfpathlineto{\pgfqpoint{2.410872in}{2.508083in}}%
\pgfpathlineto{\pgfqpoint{2.143341in}{2.291288in}}%
\pgfpathclose%
\pgfusepath{fill}%
\end{pgfscope}%
\begin{pgfscope}%
\pgfpathrectangle{\pgfqpoint{0.017500in}{0.017500in}}{\pgfqpoint{6.218282in}{3.580372in}} %
\pgfusepath{clip}%
\pgfsetbuttcap%
\pgfsetroundjoin%
\definecolor{currentfill}{rgb}{0.899534,0.440692,0.344107}%
\pgfsetfillcolor{currentfill}%
\pgfsetlinewidth{0.000000pt}%
\definecolor{currentstroke}{rgb}{0.000000,0.000000,0.000000}%
\pgfsetstrokecolor{currentstroke}%
\pgfsetdash{}{0pt}%
\pgfpathmoveto{\pgfqpoint{2.143341in}{2.291288in}}%
\pgfpathlineto{\pgfqpoint{2.410872in}{2.508083in}}%
\pgfpathlineto{\pgfqpoint{2.143341in}{2.291288in}}%
\pgfpathclose%
\pgfusepath{fill}%
\end{pgfscope}%
\begin{pgfscope}%
\pgfpathrectangle{\pgfqpoint{0.017500in}{0.017500in}}{\pgfqpoint{6.218282in}{3.580372in}} %
\pgfusepath{clip}%
\pgfsetbuttcap%
\pgfsetroundjoin%
\definecolor{currentfill}{rgb}{0.899534,0.440692,0.344107}%
\pgfsetfillcolor{currentfill}%
\pgfsetlinewidth{0.000000pt}%
\definecolor{currentstroke}{rgb}{0.000000,0.000000,0.000000}%
\pgfsetstrokecolor{currentstroke}%
\pgfsetdash{}{0pt}%
\pgfpathmoveto{\pgfqpoint{2.143341in}{2.291288in}}%
\pgfpathlineto{\pgfqpoint{2.410872in}{2.508083in}}%
\pgfpathlineto{\pgfqpoint{2.143341in}{2.291288in}}%
\pgfpathclose%
\pgfusepath{fill}%
\end{pgfscope}%
\begin{pgfscope}%
\pgfpathrectangle{\pgfqpoint{0.017500in}{0.017500in}}{\pgfqpoint{6.218282in}{3.580372in}} %
\pgfusepath{clip}%
\pgfsetbuttcap%
\pgfsetroundjoin%
\definecolor{currentfill}{rgb}{0.899534,0.440692,0.344107}%
\pgfsetfillcolor{currentfill}%
\pgfsetlinewidth{0.000000pt}%
\definecolor{currentstroke}{rgb}{0.000000,0.000000,0.000000}%
\pgfsetstrokecolor{currentstroke}%
\pgfsetdash{}{0pt}%
\pgfpathmoveto{\pgfqpoint{2.143341in}{2.291288in}}%
\pgfpathlineto{\pgfqpoint{2.410872in}{2.508083in}}%
\pgfpathlineto{\pgfqpoint{2.143341in}{2.291288in}}%
\pgfpathclose%
\pgfusepath{fill}%
\end{pgfscope}%
\begin{pgfscope}%
\pgfpathrectangle{\pgfqpoint{0.017500in}{0.017500in}}{\pgfqpoint{6.218282in}{3.580372in}} %
\pgfusepath{clip}%
\pgfsetbuttcap%
\pgfsetroundjoin%
\definecolor{currentfill}{rgb}{0.899534,0.440692,0.344107}%
\pgfsetfillcolor{currentfill}%
\pgfsetlinewidth{0.000000pt}%
\definecolor{currentstroke}{rgb}{0.000000,0.000000,0.000000}%
\pgfsetstrokecolor{currentstroke}%
\pgfsetdash{}{0pt}%
\pgfpathmoveto{\pgfqpoint{2.143341in}{2.291288in}}%
\pgfpathlineto{\pgfqpoint{2.410872in}{2.508083in}}%
\pgfpathlineto{\pgfqpoint{2.143341in}{2.291288in}}%
\pgfpathclose%
\pgfusepath{fill}%
\end{pgfscope}%
\begin{pgfscope}%
\pgfpathrectangle{\pgfqpoint{0.017500in}{0.017500in}}{\pgfqpoint{6.218282in}{3.580372in}} %
\pgfusepath{clip}%
\pgfsetbuttcap%
\pgfsetroundjoin%
\definecolor{currentfill}{rgb}{0.899534,0.440692,0.344107}%
\pgfsetfillcolor{currentfill}%
\pgfsetlinewidth{0.000000pt}%
\definecolor{currentstroke}{rgb}{0.000000,0.000000,0.000000}%
\pgfsetstrokecolor{currentstroke}%
\pgfsetdash{}{0pt}%
\pgfpathmoveto{\pgfqpoint{2.143341in}{2.291288in}}%
\pgfpathlineto{\pgfqpoint{2.410872in}{2.508083in}}%
\pgfpathlineto{\pgfqpoint{2.143341in}{2.291288in}}%
\pgfpathclose%
\pgfusepath{fill}%
\end{pgfscope}%
\begin{pgfscope}%
\pgfpathrectangle{\pgfqpoint{0.017500in}{0.017500in}}{\pgfqpoint{6.218282in}{3.580372in}} %
\pgfusepath{clip}%
\pgfsetbuttcap%
\pgfsetroundjoin%
\definecolor{currentfill}{rgb}{0.899534,0.440692,0.344107}%
\pgfsetfillcolor{currentfill}%
\pgfsetlinewidth{0.000000pt}%
\definecolor{currentstroke}{rgb}{0.000000,0.000000,0.000000}%
\pgfsetstrokecolor{currentstroke}%
\pgfsetdash{}{0pt}%
\pgfpathmoveto{\pgfqpoint{2.143341in}{2.291288in}}%
\pgfpathlineto{\pgfqpoint{2.410872in}{2.508083in}}%
\pgfpathlineto{\pgfqpoint{2.143341in}{2.291288in}}%
\pgfpathclose%
\pgfusepath{fill}%
\end{pgfscope}%
\begin{pgfscope}%
\pgfpathrectangle{\pgfqpoint{0.017500in}{0.017500in}}{\pgfqpoint{6.218282in}{3.580372in}} %
\pgfusepath{clip}%
\pgfsetbuttcap%
\pgfsetroundjoin%
\definecolor{currentfill}{rgb}{0.899534,0.440692,0.344107}%
\pgfsetfillcolor{currentfill}%
\pgfsetlinewidth{0.000000pt}%
\definecolor{currentstroke}{rgb}{0.000000,0.000000,0.000000}%
\pgfsetstrokecolor{currentstroke}%
\pgfsetdash{}{0pt}%
\pgfpathmoveto{\pgfqpoint{2.143341in}{2.291288in}}%
\pgfpathlineto{\pgfqpoint{2.410872in}{2.508083in}}%
\pgfpathlineto{\pgfqpoint{2.143341in}{2.291288in}}%
\pgfpathclose%
\pgfusepath{fill}%
\end{pgfscope}%
\begin{pgfscope}%
\pgfpathrectangle{\pgfqpoint{0.017500in}{0.017500in}}{\pgfqpoint{6.218282in}{3.580372in}} %
\pgfusepath{clip}%
\pgfsetbuttcap%
\pgfsetroundjoin%
\definecolor{currentfill}{rgb}{0.899534,0.440692,0.344107}%
\pgfsetfillcolor{currentfill}%
\pgfsetlinewidth{0.000000pt}%
\definecolor{currentstroke}{rgb}{0.000000,0.000000,0.000000}%
\pgfsetstrokecolor{currentstroke}%
\pgfsetdash{}{0pt}%
\pgfpathmoveto{\pgfqpoint{2.143341in}{2.291288in}}%
\pgfpathlineto{\pgfqpoint{2.410872in}{2.508083in}}%
\pgfpathlineto{\pgfqpoint{2.143341in}{2.291288in}}%
\pgfpathclose%
\pgfusepath{fill}%
\end{pgfscope}%
\begin{pgfscope}%
\pgfpathrectangle{\pgfqpoint{0.017500in}{0.017500in}}{\pgfqpoint{6.218282in}{3.580372in}} %
\pgfusepath{clip}%
\pgfsetbuttcap%
\pgfsetroundjoin%
\definecolor{currentfill}{rgb}{0.899534,0.440692,0.344107}%
\pgfsetfillcolor{currentfill}%
\pgfsetlinewidth{0.000000pt}%
\definecolor{currentstroke}{rgb}{0.000000,0.000000,0.000000}%
\pgfsetstrokecolor{currentstroke}%
\pgfsetdash{}{0pt}%
\pgfpathmoveto{\pgfqpoint{2.143341in}{2.291288in}}%
\pgfpathlineto{\pgfqpoint{2.410872in}{2.508083in}}%
\pgfpathlineto{\pgfqpoint{2.143341in}{2.291288in}}%
\pgfpathclose%
\pgfusepath{fill}%
\end{pgfscope}%
\begin{pgfscope}%
\pgfpathrectangle{\pgfqpoint{0.017500in}{0.017500in}}{\pgfqpoint{6.218282in}{3.580372in}} %
\pgfusepath{clip}%
\pgfsetbuttcap%
\pgfsetroundjoin%
\definecolor{currentfill}{rgb}{0.899534,0.440692,0.344107}%
\pgfsetfillcolor{currentfill}%
\pgfsetlinewidth{0.000000pt}%
\definecolor{currentstroke}{rgb}{0.000000,0.000000,0.000000}%
\pgfsetstrokecolor{currentstroke}%
\pgfsetdash{}{0pt}%
\pgfpathmoveto{\pgfqpoint{2.143341in}{2.291288in}}%
\pgfpathlineto{\pgfqpoint{2.410872in}{2.508083in}}%
\pgfpathlineto{\pgfqpoint{2.143341in}{2.291288in}}%
\pgfpathclose%
\pgfusepath{fill}%
\end{pgfscope}%
\begin{pgfscope}%
\pgfpathrectangle{\pgfqpoint{0.017500in}{0.017500in}}{\pgfqpoint{6.218282in}{3.580372in}} %
\pgfusepath{clip}%
\pgfsetbuttcap%
\pgfsetroundjoin%
\definecolor{currentfill}{rgb}{0.899534,0.440692,0.344107}%
\pgfsetfillcolor{currentfill}%
\pgfsetlinewidth{0.000000pt}%
\definecolor{currentstroke}{rgb}{0.000000,0.000000,0.000000}%
\pgfsetstrokecolor{currentstroke}%
\pgfsetdash{}{0pt}%
\pgfpathmoveto{\pgfqpoint{2.143341in}{2.291288in}}%
\pgfpathlineto{\pgfqpoint{2.410872in}{2.508083in}}%
\pgfpathlineto{\pgfqpoint{2.143341in}{2.291288in}}%
\pgfpathclose%
\pgfusepath{fill}%
\end{pgfscope}%
\begin{pgfscope}%
\pgfpathrectangle{\pgfqpoint{0.017500in}{0.017500in}}{\pgfqpoint{6.218282in}{3.580372in}} %
\pgfusepath{clip}%
\pgfsetbuttcap%
\pgfsetroundjoin%
\definecolor{currentfill}{rgb}{0.899534,0.440692,0.344107}%
\pgfsetfillcolor{currentfill}%
\pgfsetlinewidth{0.000000pt}%
\definecolor{currentstroke}{rgb}{0.000000,0.000000,0.000000}%
\pgfsetstrokecolor{currentstroke}%
\pgfsetdash{}{0pt}%
\pgfpathmoveto{\pgfqpoint{2.143341in}{2.291288in}}%
\pgfpathlineto{\pgfqpoint{2.410872in}{2.508083in}}%
\pgfpathlineto{\pgfqpoint{2.143341in}{2.291288in}}%
\pgfpathclose%
\pgfusepath{fill}%
\end{pgfscope}%
\begin{pgfscope}%
\pgfpathrectangle{\pgfqpoint{0.017500in}{0.017500in}}{\pgfqpoint{6.218282in}{3.580372in}} %
\pgfusepath{clip}%
\pgfsetbuttcap%
\pgfsetroundjoin%
\definecolor{currentfill}{rgb}{0.899534,0.440692,0.344107}%
\pgfsetfillcolor{currentfill}%
\pgfsetlinewidth{0.000000pt}%
\definecolor{currentstroke}{rgb}{0.000000,0.000000,0.000000}%
\pgfsetstrokecolor{currentstroke}%
\pgfsetdash{}{0pt}%
\pgfpathmoveto{\pgfqpoint{2.143341in}{2.291288in}}%
\pgfpathlineto{\pgfqpoint{2.410872in}{2.508083in}}%
\pgfpathlineto{\pgfqpoint{2.143341in}{2.291288in}}%
\pgfpathclose%
\pgfusepath{fill}%
\end{pgfscope}%
\begin{pgfscope}%
\pgfpathrectangle{\pgfqpoint{0.017500in}{0.017500in}}{\pgfqpoint{6.218282in}{3.580372in}} %
\pgfusepath{clip}%
\pgfsetbuttcap%
\pgfsetroundjoin%
\definecolor{currentfill}{rgb}{0.899534,0.440692,0.344107}%
\pgfsetfillcolor{currentfill}%
\pgfsetlinewidth{0.000000pt}%
\definecolor{currentstroke}{rgb}{0.000000,0.000000,0.000000}%
\pgfsetstrokecolor{currentstroke}%
\pgfsetdash{}{0pt}%
\pgfpathmoveto{\pgfqpoint{2.143341in}{2.291288in}}%
\pgfpathlineto{\pgfqpoint{2.410872in}{2.508083in}}%
\pgfpathlineto{\pgfqpoint{2.143341in}{2.291288in}}%
\pgfpathclose%
\pgfusepath{fill}%
\end{pgfscope}%
\begin{pgfscope}%
\pgfpathrectangle{\pgfqpoint{0.017500in}{0.017500in}}{\pgfqpoint{6.218282in}{3.580372in}} %
\pgfusepath{clip}%
\pgfsetbuttcap%
\pgfsetroundjoin%
\definecolor{currentfill}{rgb}{0.899534,0.440692,0.344107}%
\pgfsetfillcolor{currentfill}%
\pgfsetlinewidth{0.000000pt}%
\definecolor{currentstroke}{rgb}{0.000000,0.000000,0.000000}%
\pgfsetstrokecolor{currentstroke}%
\pgfsetdash{}{0pt}%
\pgfpathmoveto{\pgfqpoint{2.143341in}{2.291288in}}%
\pgfpathlineto{\pgfqpoint{2.410872in}{2.508083in}}%
\pgfpathlineto{\pgfqpoint{2.143341in}{2.291288in}}%
\pgfpathclose%
\pgfusepath{fill}%
\end{pgfscope}%
\begin{pgfscope}%
\pgfpathrectangle{\pgfqpoint{0.017500in}{0.017500in}}{\pgfqpoint{6.218282in}{3.580372in}} %
\pgfusepath{clip}%
\pgfsetbuttcap%
\pgfsetroundjoin%
\definecolor{currentfill}{rgb}{0.899534,0.440692,0.344107}%
\pgfsetfillcolor{currentfill}%
\pgfsetlinewidth{0.000000pt}%
\definecolor{currentstroke}{rgb}{0.000000,0.000000,0.000000}%
\pgfsetstrokecolor{currentstroke}%
\pgfsetdash{}{0pt}%
\pgfpathmoveto{\pgfqpoint{2.143341in}{2.291288in}}%
\pgfpathlineto{\pgfqpoint{2.410872in}{2.508083in}}%
\pgfpathlineto{\pgfqpoint{2.143341in}{2.291288in}}%
\pgfpathclose%
\pgfusepath{fill}%
\end{pgfscope}%
\begin{pgfscope}%
\pgfpathrectangle{\pgfqpoint{0.017500in}{0.017500in}}{\pgfqpoint{6.218282in}{3.580372in}} %
\pgfusepath{clip}%
\pgfsetbuttcap%
\pgfsetroundjoin%
\definecolor{currentfill}{rgb}{0.899534,0.440692,0.344107}%
\pgfsetfillcolor{currentfill}%
\pgfsetlinewidth{0.000000pt}%
\definecolor{currentstroke}{rgb}{0.000000,0.000000,0.000000}%
\pgfsetstrokecolor{currentstroke}%
\pgfsetdash{}{0pt}%
\pgfpathmoveto{\pgfqpoint{2.143341in}{2.291288in}}%
\pgfpathlineto{\pgfqpoint{2.410872in}{2.508083in}}%
\pgfpathlineto{\pgfqpoint{2.143341in}{2.291288in}}%
\pgfpathclose%
\pgfusepath{fill}%
\end{pgfscope}%
\begin{pgfscope}%
\pgfpathrectangle{\pgfqpoint{0.017500in}{0.017500in}}{\pgfqpoint{6.218282in}{3.580372in}} %
\pgfusepath{clip}%
\pgfsetbuttcap%
\pgfsetroundjoin%
\definecolor{currentfill}{rgb}{0.899534,0.440692,0.344107}%
\pgfsetfillcolor{currentfill}%
\pgfsetlinewidth{0.000000pt}%
\definecolor{currentstroke}{rgb}{0.000000,0.000000,0.000000}%
\pgfsetstrokecolor{currentstroke}%
\pgfsetdash{}{0pt}%
\pgfpathmoveto{\pgfqpoint{2.143341in}{2.291288in}}%
\pgfpathlineto{\pgfqpoint{2.410872in}{2.508083in}}%
\pgfpathlineto{\pgfqpoint{2.143341in}{2.291288in}}%
\pgfpathclose%
\pgfusepath{fill}%
\end{pgfscope}%
\begin{pgfscope}%
\pgfpathrectangle{\pgfqpoint{0.017500in}{0.017500in}}{\pgfqpoint{6.218282in}{3.580372in}} %
\pgfusepath{clip}%
\pgfsetbuttcap%
\pgfsetroundjoin%
\definecolor{currentfill}{rgb}{0.899534,0.440692,0.344107}%
\pgfsetfillcolor{currentfill}%
\pgfsetlinewidth{0.000000pt}%
\definecolor{currentstroke}{rgb}{0.000000,0.000000,0.000000}%
\pgfsetstrokecolor{currentstroke}%
\pgfsetdash{}{0pt}%
\pgfpathmoveto{\pgfqpoint{2.143341in}{2.291288in}}%
\pgfpathlineto{\pgfqpoint{2.410872in}{2.508083in}}%
\pgfpathlineto{\pgfqpoint{2.143341in}{2.291288in}}%
\pgfpathclose%
\pgfusepath{fill}%
\end{pgfscope}%
\begin{pgfscope}%
\pgfpathrectangle{\pgfqpoint{0.017500in}{0.017500in}}{\pgfqpoint{6.218282in}{3.580372in}} %
\pgfusepath{clip}%
\pgfsetbuttcap%
\pgfsetroundjoin%
\definecolor{currentfill}{rgb}{0.899534,0.440692,0.344107}%
\pgfsetfillcolor{currentfill}%
\pgfsetlinewidth{0.000000pt}%
\definecolor{currentstroke}{rgb}{0.000000,0.000000,0.000000}%
\pgfsetstrokecolor{currentstroke}%
\pgfsetdash{}{0pt}%
\pgfpathmoveto{\pgfqpoint{2.143341in}{2.291288in}}%
\pgfpathlineto{\pgfqpoint{2.410872in}{2.508083in}}%
\pgfpathlineto{\pgfqpoint{2.143341in}{2.291288in}}%
\pgfpathclose%
\pgfusepath{fill}%
\end{pgfscope}%
\begin{pgfscope}%
\pgfpathrectangle{\pgfqpoint{0.017500in}{0.017500in}}{\pgfqpoint{6.218282in}{3.580372in}} %
\pgfusepath{clip}%
\pgfsetbuttcap%
\pgfsetroundjoin%
\definecolor{currentfill}{rgb}{0.763520,0.178667,0.193396}%
\pgfsetfillcolor{currentfill}%
\pgfsetlinewidth{0.000000pt}%
\definecolor{currentstroke}{rgb}{0.000000,0.000000,0.000000}%
\pgfsetstrokecolor{currentstroke}%
\pgfsetdash{}{0pt}%
\pgfpathmoveto{\pgfqpoint{4.018447in}{2.468167in}}%
\pgfpathlineto{\pgfqpoint{4.267628in}{2.613520in}}%
\pgfpathlineto{\pgfqpoint{4.018447in}{2.468167in}}%
\pgfpathclose%
\pgfusepath{fill}%
\end{pgfscope}%
\begin{pgfscope}%
\pgfpathrectangle{\pgfqpoint{0.017500in}{0.017500in}}{\pgfqpoint{6.218282in}{3.580372in}} %
\pgfusepath{clip}%
\pgfsetbuttcap%
\pgfsetroundjoin%
\definecolor{currentfill}{rgb}{0.763520,0.178667,0.193396}%
\pgfsetfillcolor{currentfill}%
\pgfsetlinewidth{0.000000pt}%
\definecolor{currentstroke}{rgb}{0.000000,0.000000,0.000000}%
\pgfsetstrokecolor{currentstroke}%
\pgfsetdash{}{0pt}%
\pgfpathmoveto{\pgfqpoint{4.018447in}{2.468167in}}%
\pgfpathlineto{\pgfqpoint{4.267628in}{2.613520in}}%
\pgfpathlineto{\pgfqpoint{4.018447in}{2.468167in}}%
\pgfpathclose%
\pgfusepath{fill}%
\end{pgfscope}%
\begin{pgfscope}%
\pgfpathrectangle{\pgfqpoint{0.017500in}{0.017500in}}{\pgfqpoint{6.218282in}{3.580372in}} %
\pgfusepath{clip}%
\pgfsetbuttcap%
\pgfsetroundjoin%
\definecolor{currentfill}{rgb}{0.763520,0.178667,0.193396}%
\pgfsetfillcolor{currentfill}%
\pgfsetlinewidth{0.000000pt}%
\definecolor{currentstroke}{rgb}{0.000000,0.000000,0.000000}%
\pgfsetstrokecolor{currentstroke}%
\pgfsetdash{}{0pt}%
\pgfpathmoveto{\pgfqpoint{4.018447in}{2.468167in}}%
\pgfpathlineto{\pgfqpoint{4.267628in}{2.613520in}}%
\pgfpathlineto{\pgfqpoint{4.018447in}{2.468167in}}%
\pgfpathclose%
\pgfusepath{fill}%
\end{pgfscope}%
\begin{pgfscope}%
\pgfpathrectangle{\pgfqpoint{0.017500in}{0.017500in}}{\pgfqpoint{6.218282in}{3.580372in}} %
\pgfusepath{clip}%
\pgfsetbuttcap%
\pgfsetroundjoin%
\definecolor{currentfill}{rgb}{0.763520,0.178667,0.193396}%
\pgfsetfillcolor{currentfill}%
\pgfsetlinewidth{0.000000pt}%
\definecolor{currentstroke}{rgb}{0.000000,0.000000,0.000000}%
\pgfsetstrokecolor{currentstroke}%
\pgfsetdash{}{0pt}%
\pgfpathmoveto{\pgfqpoint{4.018447in}{2.468167in}}%
\pgfpathlineto{\pgfqpoint{4.267628in}{2.613520in}}%
\pgfpathlineto{\pgfqpoint{4.018447in}{2.468167in}}%
\pgfpathclose%
\pgfusepath{fill}%
\end{pgfscope}%
\begin{pgfscope}%
\pgfpathrectangle{\pgfqpoint{0.017500in}{0.017500in}}{\pgfqpoint{6.218282in}{3.580372in}} %
\pgfusepath{clip}%
\pgfsetbuttcap%
\pgfsetroundjoin%
\definecolor{currentfill}{rgb}{0.763520,0.178667,0.193396}%
\pgfsetfillcolor{currentfill}%
\pgfsetlinewidth{0.000000pt}%
\definecolor{currentstroke}{rgb}{0.000000,0.000000,0.000000}%
\pgfsetstrokecolor{currentstroke}%
\pgfsetdash{}{0pt}%
\pgfpathmoveto{\pgfqpoint{4.018447in}{2.468167in}}%
\pgfpathlineto{\pgfqpoint{4.267628in}{2.613520in}}%
\pgfpathlineto{\pgfqpoint{4.018447in}{2.468167in}}%
\pgfpathclose%
\pgfusepath{fill}%
\end{pgfscope}%
\begin{pgfscope}%
\pgfpathrectangle{\pgfqpoint{0.017500in}{0.017500in}}{\pgfqpoint{6.218282in}{3.580372in}} %
\pgfusepath{clip}%
\pgfsetbuttcap%
\pgfsetroundjoin%
\definecolor{currentfill}{rgb}{0.763520,0.178667,0.193396}%
\pgfsetfillcolor{currentfill}%
\pgfsetlinewidth{0.000000pt}%
\definecolor{currentstroke}{rgb}{0.000000,0.000000,0.000000}%
\pgfsetstrokecolor{currentstroke}%
\pgfsetdash{}{0pt}%
\pgfpathmoveto{\pgfqpoint{4.018447in}{2.468167in}}%
\pgfpathlineto{\pgfqpoint{4.267628in}{2.613520in}}%
\pgfpathlineto{\pgfqpoint{4.018447in}{2.468167in}}%
\pgfpathclose%
\pgfusepath{fill}%
\end{pgfscope}%
\begin{pgfscope}%
\pgfpathrectangle{\pgfqpoint{0.017500in}{0.017500in}}{\pgfqpoint{6.218282in}{3.580372in}} %
\pgfusepath{clip}%
\pgfsetbuttcap%
\pgfsetroundjoin%
\definecolor{currentfill}{rgb}{0.763520,0.178667,0.193396}%
\pgfsetfillcolor{currentfill}%
\pgfsetlinewidth{0.000000pt}%
\definecolor{currentstroke}{rgb}{0.000000,0.000000,0.000000}%
\pgfsetstrokecolor{currentstroke}%
\pgfsetdash{}{0pt}%
\pgfpathmoveto{\pgfqpoint{4.018447in}{2.468167in}}%
\pgfpathlineto{\pgfqpoint{4.267628in}{2.613520in}}%
\pgfpathlineto{\pgfqpoint{4.018447in}{2.468167in}}%
\pgfpathclose%
\pgfusepath{fill}%
\end{pgfscope}%
\begin{pgfscope}%
\pgfpathrectangle{\pgfqpoint{0.017500in}{0.017500in}}{\pgfqpoint{6.218282in}{3.580372in}} %
\pgfusepath{clip}%
\pgfsetbuttcap%
\pgfsetroundjoin%
\definecolor{currentfill}{rgb}{0.763520,0.178667,0.193396}%
\pgfsetfillcolor{currentfill}%
\pgfsetlinewidth{0.000000pt}%
\definecolor{currentstroke}{rgb}{0.000000,0.000000,0.000000}%
\pgfsetstrokecolor{currentstroke}%
\pgfsetdash{}{0pt}%
\pgfpathmoveto{\pgfqpoint{4.018447in}{2.468167in}}%
\pgfpathlineto{\pgfqpoint{4.267628in}{2.613520in}}%
\pgfpathlineto{\pgfqpoint{4.018447in}{2.468167in}}%
\pgfpathclose%
\pgfusepath{fill}%
\end{pgfscope}%
\begin{pgfscope}%
\pgfpathrectangle{\pgfqpoint{0.017500in}{0.017500in}}{\pgfqpoint{6.218282in}{3.580372in}} %
\pgfusepath{clip}%
\pgfsetbuttcap%
\pgfsetroundjoin%
\definecolor{currentfill}{rgb}{0.763520,0.178667,0.193396}%
\pgfsetfillcolor{currentfill}%
\pgfsetlinewidth{0.000000pt}%
\definecolor{currentstroke}{rgb}{0.000000,0.000000,0.000000}%
\pgfsetstrokecolor{currentstroke}%
\pgfsetdash{}{0pt}%
\pgfpathmoveto{\pgfqpoint{4.018447in}{2.468167in}}%
\pgfpathlineto{\pgfqpoint{4.267628in}{2.613520in}}%
\pgfpathlineto{\pgfqpoint{4.018447in}{2.468167in}}%
\pgfpathclose%
\pgfusepath{fill}%
\end{pgfscope}%
\begin{pgfscope}%
\pgfpathrectangle{\pgfqpoint{0.017500in}{0.017500in}}{\pgfqpoint{6.218282in}{3.580372in}} %
\pgfusepath{clip}%
\pgfsetbuttcap%
\pgfsetroundjoin%
\definecolor{currentfill}{rgb}{0.763520,0.178667,0.193396}%
\pgfsetfillcolor{currentfill}%
\pgfsetlinewidth{0.000000pt}%
\definecolor{currentstroke}{rgb}{0.000000,0.000000,0.000000}%
\pgfsetstrokecolor{currentstroke}%
\pgfsetdash{}{0pt}%
\pgfpathmoveto{\pgfqpoint{4.018447in}{2.468167in}}%
\pgfpathlineto{\pgfqpoint{4.267628in}{2.613520in}}%
\pgfpathlineto{\pgfqpoint{4.018447in}{2.468167in}}%
\pgfpathclose%
\pgfusepath{fill}%
\end{pgfscope}%
\begin{pgfscope}%
\pgfpathrectangle{\pgfqpoint{0.017500in}{0.017500in}}{\pgfqpoint{6.218282in}{3.580372in}} %
\pgfusepath{clip}%
\pgfsetbuttcap%
\pgfsetroundjoin%
\definecolor{currentfill}{rgb}{0.763520,0.178667,0.193396}%
\pgfsetfillcolor{currentfill}%
\pgfsetlinewidth{0.000000pt}%
\definecolor{currentstroke}{rgb}{0.000000,0.000000,0.000000}%
\pgfsetstrokecolor{currentstroke}%
\pgfsetdash{}{0pt}%
\pgfpathmoveto{\pgfqpoint{4.018447in}{2.468167in}}%
\pgfpathlineto{\pgfqpoint{4.267628in}{2.613520in}}%
\pgfpathlineto{\pgfqpoint{4.018447in}{2.468167in}}%
\pgfpathclose%
\pgfusepath{fill}%
\end{pgfscope}%
\begin{pgfscope}%
\pgfpathrectangle{\pgfqpoint{0.017500in}{0.017500in}}{\pgfqpoint{6.218282in}{3.580372in}} %
\pgfusepath{clip}%
\pgfsetbuttcap%
\pgfsetroundjoin%
\definecolor{currentfill}{rgb}{0.763520,0.178667,0.193396}%
\pgfsetfillcolor{currentfill}%
\pgfsetlinewidth{0.000000pt}%
\definecolor{currentstroke}{rgb}{0.000000,0.000000,0.000000}%
\pgfsetstrokecolor{currentstroke}%
\pgfsetdash{}{0pt}%
\pgfpathmoveto{\pgfqpoint{4.018447in}{2.468167in}}%
\pgfpathlineto{\pgfqpoint{4.267628in}{2.613520in}}%
\pgfpathlineto{\pgfqpoint{4.018447in}{2.468167in}}%
\pgfpathclose%
\pgfusepath{fill}%
\end{pgfscope}%
\begin{pgfscope}%
\pgfpathrectangle{\pgfqpoint{0.017500in}{0.017500in}}{\pgfqpoint{6.218282in}{3.580372in}} %
\pgfusepath{clip}%
\pgfsetbuttcap%
\pgfsetroundjoin%
\definecolor{currentfill}{rgb}{0.763520,0.178667,0.193396}%
\pgfsetfillcolor{currentfill}%
\pgfsetlinewidth{0.000000pt}%
\definecolor{currentstroke}{rgb}{0.000000,0.000000,0.000000}%
\pgfsetstrokecolor{currentstroke}%
\pgfsetdash{}{0pt}%
\pgfpathmoveto{\pgfqpoint{4.018447in}{2.468167in}}%
\pgfpathlineto{\pgfqpoint{4.267628in}{2.613520in}}%
\pgfpathlineto{\pgfqpoint{4.018447in}{2.468167in}}%
\pgfpathclose%
\pgfusepath{fill}%
\end{pgfscope}%
\begin{pgfscope}%
\pgfpathrectangle{\pgfqpoint{0.017500in}{0.017500in}}{\pgfqpoint{6.218282in}{3.580372in}} %
\pgfusepath{clip}%
\pgfsetbuttcap%
\pgfsetroundjoin%
\definecolor{currentfill}{rgb}{0.763520,0.178667,0.193396}%
\pgfsetfillcolor{currentfill}%
\pgfsetlinewidth{0.000000pt}%
\definecolor{currentstroke}{rgb}{0.000000,0.000000,0.000000}%
\pgfsetstrokecolor{currentstroke}%
\pgfsetdash{}{0pt}%
\pgfpathmoveto{\pgfqpoint{4.018447in}{2.468167in}}%
\pgfpathlineto{\pgfqpoint{4.267628in}{2.613520in}}%
\pgfpathlineto{\pgfqpoint{4.018447in}{2.468167in}}%
\pgfpathclose%
\pgfusepath{fill}%
\end{pgfscope}%
\begin{pgfscope}%
\pgfpathrectangle{\pgfqpoint{0.017500in}{0.017500in}}{\pgfqpoint{6.218282in}{3.580372in}} %
\pgfusepath{clip}%
\pgfsetbuttcap%
\pgfsetroundjoin%
\definecolor{currentfill}{rgb}{0.763520,0.178667,0.193396}%
\pgfsetfillcolor{currentfill}%
\pgfsetlinewidth{0.000000pt}%
\definecolor{currentstroke}{rgb}{0.000000,0.000000,0.000000}%
\pgfsetstrokecolor{currentstroke}%
\pgfsetdash{}{0pt}%
\pgfpathmoveto{\pgfqpoint{4.018447in}{2.468167in}}%
\pgfpathlineto{\pgfqpoint{4.267628in}{2.613520in}}%
\pgfpathlineto{\pgfqpoint{4.018447in}{2.468167in}}%
\pgfpathclose%
\pgfusepath{fill}%
\end{pgfscope}%
\begin{pgfscope}%
\pgfpathrectangle{\pgfqpoint{0.017500in}{0.017500in}}{\pgfqpoint{6.218282in}{3.580372in}} %
\pgfusepath{clip}%
\pgfsetbuttcap%
\pgfsetroundjoin%
\definecolor{currentfill}{rgb}{0.763520,0.178667,0.193396}%
\pgfsetfillcolor{currentfill}%
\pgfsetlinewidth{0.000000pt}%
\definecolor{currentstroke}{rgb}{0.000000,0.000000,0.000000}%
\pgfsetstrokecolor{currentstroke}%
\pgfsetdash{}{0pt}%
\pgfpathmoveto{\pgfqpoint{4.018447in}{2.468167in}}%
\pgfpathlineto{\pgfqpoint{4.267628in}{2.613520in}}%
\pgfpathlineto{\pgfqpoint{4.018447in}{2.468167in}}%
\pgfpathclose%
\pgfusepath{fill}%
\end{pgfscope}%
\begin{pgfscope}%
\pgfpathrectangle{\pgfqpoint{0.017500in}{0.017500in}}{\pgfqpoint{6.218282in}{3.580372in}} %
\pgfusepath{clip}%
\pgfsetbuttcap%
\pgfsetroundjoin%
\definecolor{currentfill}{rgb}{0.763520,0.178667,0.193396}%
\pgfsetfillcolor{currentfill}%
\pgfsetlinewidth{0.000000pt}%
\definecolor{currentstroke}{rgb}{0.000000,0.000000,0.000000}%
\pgfsetstrokecolor{currentstroke}%
\pgfsetdash{}{0pt}%
\pgfpathmoveto{\pgfqpoint{4.018447in}{2.468167in}}%
\pgfpathlineto{\pgfqpoint{4.267628in}{2.613520in}}%
\pgfpathlineto{\pgfqpoint{4.018447in}{2.468167in}}%
\pgfpathclose%
\pgfusepath{fill}%
\end{pgfscope}%
\begin{pgfscope}%
\pgfpathrectangle{\pgfqpoint{0.017500in}{0.017500in}}{\pgfqpoint{6.218282in}{3.580372in}} %
\pgfusepath{clip}%
\pgfsetbuttcap%
\pgfsetroundjoin%
\definecolor{currentfill}{rgb}{0.763520,0.178667,0.193396}%
\pgfsetfillcolor{currentfill}%
\pgfsetlinewidth{0.000000pt}%
\definecolor{currentstroke}{rgb}{0.000000,0.000000,0.000000}%
\pgfsetstrokecolor{currentstroke}%
\pgfsetdash{}{0pt}%
\pgfpathmoveto{\pgfqpoint{4.018447in}{2.468167in}}%
\pgfpathlineto{\pgfqpoint{4.267628in}{2.613520in}}%
\pgfpathlineto{\pgfqpoint{4.018447in}{2.468167in}}%
\pgfpathclose%
\pgfusepath{fill}%
\end{pgfscope}%
\begin{pgfscope}%
\pgfpathrectangle{\pgfqpoint{0.017500in}{0.017500in}}{\pgfqpoint{6.218282in}{3.580372in}} %
\pgfusepath{clip}%
\pgfsetbuttcap%
\pgfsetroundjoin%
\definecolor{currentfill}{rgb}{0.763520,0.178667,0.193396}%
\pgfsetfillcolor{currentfill}%
\pgfsetlinewidth{0.000000pt}%
\definecolor{currentstroke}{rgb}{0.000000,0.000000,0.000000}%
\pgfsetstrokecolor{currentstroke}%
\pgfsetdash{}{0pt}%
\pgfpathmoveto{\pgfqpoint{4.018447in}{2.468167in}}%
\pgfpathlineto{\pgfqpoint{4.267628in}{2.613520in}}%
\pgfpathlineto{\pgfqpoint{4.018447in}{2.468167in}}%
\pgfpathclose%
\pgfusepath{fill}%
\end{pgfscope}%
\begin{pgfscope}%
\pgfpathrectangle{\pgfqpoint{0.017500in}{0.017500in}}{\pgfqpoint{6.218282in}{3.580372in}} %
\pgfusepath{clip}%
\pgfsetbuttcap%
\pgfsetroundjoin%
\definecolor{currentfill}{rgb}{0.763520,0.178667,0.193396}%
\pgfsetfillcolor{currentfill}%
\pgfsetlinewidth{0.000000pt}%
\definecolor{currentstroke}{rgb}{0.000000,0.000000,0.000000}%
\pgfsetstrokecolor{currentstroke}%
\pgfsetdash{}{0pt}%
\pgfpathmoveto{\pgfqpoint{4.018447in}{2.468167in}}%
\pgfpathlineto{\pgfqpoint{4.267628in}{2.613520in}}%
\pgfpathlineto{\pgfqpoint{4.018447in}{2.468167in}}%
\pgfpathclose%
\pgfusepath{fill}%
\end{pgfscope}%
\begin{pgfscope}%
\pgfpathrectangle{\pgfqpoint{0.017500in}{0.017500in}}{\pgfqpoint{6.218282in}{3.580372in}} %
\pgfusepath{clip}%
\pgfsetbuttcap%
\pgfsetroundjoin%
\definecolor{currentfill}{rgb}{0.763520,0.178667,0.193396}%
\pgfsetfillcolor{currentfill}%
\pgfsetlinewidth{0.000000pt}%
\definecolor{currentstroke}{rgb}{0.000000,0.000000,0.000000}%
\pgfsetstrokecolor{currentstroke}%
\pgfsetdash{}{0pt}%
\pgfpathmoveto{\pgfqpoint{4.018447in}{2.468167in}}%
\pgfpathlineto{\pgfqpoint{4.267628in}{2.613520in}}%
\pgfpathlineto{\pgfqpoint{4.018447in}{2.468167in}}%
\pgfpathclose%
\pgfusepath{fill}%
\end{pgfscope}%
\begin{pgfscope}%
\pgfpathrectangle{\pgfqpoint{0.017500in}{0.017500in}}{\pgfqpoint{6.218282in}{3.580372in}} %
\pgfusepath{clip}%
\pgfsetbuttcap%
\pgfsetroundjoin%
\definecolor{currentfill}{rgb}{0.763520,0.178667,0.193396}%
\pgfsetfillcolor{currentfill}%
\pgfsetlinewidth{0.000000pt}%
\definecolor{currentstroke}{rgb}{0.000000,0.000000,0.000000}%
\pgfsetstrokecolor{currentstroke}%
\pgfsetdash{}{0pt}%
\pgfpathmoveto{\pgfqpoint{4.018447in}{2.468167in}}%
\pgfpathlineto{\pgfqpoint{4.267628in}{2.613520in}}%
\pgfpathlineto{\pgfqpoint{4.018447in}{2.468167in}}%
\pgfpathclose%
\pgfusepath{fill}%
\end{pgfscope}%
\begin{pgfscope}%
\pgfpathrectangle{\pgfqpoint{0.017500in}{0.017500in}}{\pgfqpoint{6.218282in}{3.580372in}} %
\pgfusepath{clip}%
\pgfsetbuttcap%
\pgfsetroundjoin%
\definecolor{currentfill}{rgb}{0.763520,0.178667,0.193396}%
\pgfsetfillcolor{currentfill}%
\pgfsetlinewidth{0.000000pt}%
\definecolor{currentstroke}{rgb}{0.000000,0.000000,0.000000}%
\pgfsetstrokecolor{currentstroke}%
\pgfsetdash{}{0pt}%
\pgfpathmoveto{\pgfqpoint{4.018447in}{2.468167in}}%
\pgfpathlineto{\pgfqpoint{4.267628in}{2.613520in}}%
\pgfpathlineto{\pgfqpoint{4.018447in}{2.468167in}}%
\pgfpathclose%
\pgfusepath{fill}%
\end{pgfscope}%
\begin{pgfscope}%
\pgfpathrectangle{\pgfqpoint{0.017500in}{0.017500in}}{\pgfqpoint{6.218282in}{3.580372in}} %
\pgfusepath{clip}%
\pgfsetbuttcap%
\pgfsetroundjoin%
\definecolor{currentfill}{rgb}{0.763520,0.178667,0.193396}%
\pgfsetfillcolor{currentfill}%
\pgfsetlinewidth{0.000000pt}%
\definecolor{currentstroke}{rgb}{0.000000,0.000000,0.000000}%
\pgfsetstrokecolor{currentstroke}%
\pgfsetdash{}{0pt}%
\pgfpathmoveto{\pgfqpoint{4.018447in}{2.468167in}}%
\pgfpathlineto{\pgfqpoint{4.267628in}{2.613520in}}%
\pgfpathlineto{\pgfqpoint{4.018447in}{2.468167in}}%
\pgfpathclose%
\pgfusepath{fill}%
\end{pgfscope}%
\begin{pgfscope}%
\pgfpathrectangle{\pgfqpoint{0.017500in}{0.017500in}}{\pgfqpoint{6.218282in}{3.580372in}} %
\pgfusepath{clip}%
\pgfsetbuttcap%
\pgfsetroundjoin%
\definecolor{currentfill}{rgb}{0.763520,0.178667,0.193396}%
\pgfsetfillcolor{currentfill}%
\pgfsetlinewidth{0.000000pt}%
\definecolor{currentstroke}{rgb}{0.000000,0.000000,0.000000}%
\pgfsetstrokecolor{currentstroke}%
\pgfsetdash{}{0pt}%
\pgfpathmoveto{\pgfqpoint{4.018447in}{2.468167in}}%
\pgfpathlineto{\pgfqpoint{4.267628in}{2.613520in}}%
\pgfpathlineto{\pgfqpoint{4.018447in}{2.468167in}}%
\pgfpathclose%
\pgfusepath{fill}%
\end{pgfscope}%
\begin{pgfscope}%
\pgfpathrectangle{\pgfqpoint{0.017500in}{0.017500in}}{\pgfqpoint{6.218282in}{3.580372in}} %
\pgfusepath{clip}%
\pgfsetbuttcap%
\pgfsetroundjoin%
\definecolor{currentfill}{rgb}{0.763520,0.178667,0.193396}%
\pgfsetfillcolor{currentfill}%
\pgfsetlinewidth{0.000000pt}%
\definecolor{currentstroke}{rgb}{0.000000,0.000000,0.000000}%
\pgfsetstrokecolor{currentstroke}%
\pgfsetdash{}{0pt}%
\pgfpathmoveto{\pgfqpoint{4.018447in}{2.468167in}}%
\pgfpathlineto{\pgfqpoint{4.267628in}{2.613520in}}%
\pgfpathlineto{\pgfqpoint{4.018447in}{2.468167in}}%
\pgfpathclose%
\pgfusepath{fill}%
\end{pgfscope}%
\begin{pgfscope}%
\pgfpathrectangle{\pgfqpoint{0.017500in}{0.017500in}}{\pgfqpoint{6.218282in}{3.580372in}} %
\pgfusepath{clip}%
\pgfsetbuttcap%
\pgfsetroundjoin%
\definecolor{currentfill}{rgb}{0.763520,0.178667,0.193396}%
\pgfsetfillcolor{currentfill}%
\pgfsetlinewidth{0.000000pt}%
\definecolor{currentstroke}{rgb}{0.000000,0.000000,0.000000}%
\pgfsetstrokecolor{currentstroke}%
\pgfsetdash{}{0pt}%
\pgfpathmoveto{\pgfqpoint{4.018447in}{2.468167in}}%
\pgfpathlineto{\pgfqpoint{4.267628in}{2.613520in}}%
\pgfpathlineto{\pgfqpoint{4.018447in}{2.468167in}}%
\pgfpathclose%
\pgfusepath{fill}%
\end{pgfscope}%
\begin{pgfscope}%
\pgfpathrectangle{\pgfqpoint{0.017500in}{0.017500in}}{\pgfqpoint{6.218282in}{3.580372in}} %
\pgfusepath{clip}%
\pgfsetbuttcap%
\pgfsetroundjoin%
\definecolor{currentfill}{rgb}{0.763520,0.178667,0.193396}%
\pgfsetfillcolor{currentfill}%
\pgfsetlinewidth{0.000000pt}%
\definecolor{currentstroke}{rgb}{0.000000,0.000000,0.000000}%
\pgfsetstrokecolor{currentstroke}%
\pgfsetdash{}{0pt}%
\pgfpathmoveto{\pgfqpoint{4.018447in}{2.468167in}}%
\pgfpathlineto{\pgfqpoint{4.267628in}{2.613520in}}%
\pgfpathlineto{\pgfqpoint{4.018447in}{2.468167in}}%
\pgfpathclose%
\pgfusepath{fill}%
\end{pgfscope}%
\begin{pgfscope}%
\pgfpathrectangle{\pgfqpoint{0.017500in}{0.017500in}}{\pgfqpoint{6.218282in}{3.580372in}} %
\pgfusepath{clip}%
\pgfsetbuttcap%
\pgfsetroundjoin%
\definecolor{currentfill}{rgb}{0.763520,0.178667,0.193396}%
\pgfsetfillcolor{currentfill}%
\pgfsetlinewidth{0.000000pt}%
\definecolor{currentstroke}{rgb}{0.000000,0.000000,0.000000}%
\pgfsetstrokecolor{currentstroke}%
\pgfsetdash{}{0pt}%
\pgfpathmoveto{\pgfqpoint{4.018447in}{2.468167in}}%
\pgfpathlineto{\pgfqpoint{4.267628in}{2.613520in}}%
\pgfpathlineto{\pgfqpoint{4.018447in}{2.468167in}}%
\pgfpathclose%
\pgfusepath{fill}%
\end{pgfscope}%
\begin{pgfscope}%
\pgfpathrectangle{\pgfqpoint{0.017500in}{0.017500in}}{\pgfqpoint{6.218282in}{3.580372in}} %
\pgfusepath{clip}%
\pgfsetbuttcap%
\pgfsetroundjoin%
\definecolor{currentfill}{rgb}{0.763520,0.178667,0.193396}%
\pgfsetfillcolor{currentfill}%
\pgfsetlinewidth{0.000000pt}%
\definecolor{currentstroke}{rgb}{0.000000,0.000000,0.000000}%
\pgfsetstrokecolor{currentstroke}%
\pgfsetdash{}{0pt}%
\pgfpathmoveto{\pgfqpoint{4.018447in}{2.468167in}}%
\pgfpathlineto{\pgfqpoint{4.267628in}{2.613520in}}%
\pgfpathlineto{\pgfqpoint{4.018447in}{2.468167in}}%
\pgfpathclose%
\pgfusepath{fill}%
\end{pgfscope}%
\begin{pgfscope}%
\pgfpathrectangle{\pgfqpoint{0.017500in}{0.017500in}}{\pgfqpoint{6.218282in}{3.580372in}} %
\pgfusepath{clip}%
\pgfsetbuttcap%
\pgfsetroundjoin%
\definecolor{currentfill}{rgb}{0.892138,0.425389,0.333289}%
\pgfsetfillcolor{currentfill}%
\pgfsetlinewidth{0.000000pt}%
\definecolor{currentstroke}{rgb}{0.000000,0.000000,0.000000}%
\pgfsetstrokecolor{currentstroke}%
\pgfsetdash{}{0pt}%
\pgfpathmoveto{\pgfqpoint{2.143341in}{2.291288in}}%
\pgfpathlineto{\pgfqpoint{2.590432in}{2.210709in}}%
\pgfpathlineto{\pgfqpoint{2.855513in}{2.451418in}}%
\pgfpathlineto{\pgfqpoint{2.410872in}{2.508083in}}%
\pgfpathlineto{\pgfqpoint{2.143341in}{2.291288in}}%
\pgfpathclose%
\pgfusepath{fill}%
\end{pgfscope}%
\begin{pgfscope}%
\pgfpathrectangle{\pgfqpoint{0.017500in}{0.017500in}}{\pgfqpoint{6.218282in}{3.580372in}} %
\pgfusepath{clip}%
\pgfsetbuttcap%
\pgfsetroundjoin%
\definecolor{currentfill}{rgb}{0.892138,0.425389,0.333289}%
\pgfsetfillcolor{currentfill}%
\pgfsetlinewidth{0.000000pt}%
\definecolor{currentstroke}{rgb}{0.000000,0.000000,0.000000}%
\pgfsetstrokecolor{currentstroke}%
\pgfsetdash{}{0pt}%
\pgfpathmoveto{\pgfqpoint{2.143341in}{2.291288in}}%
\pgfpathlineto{\pgfqpoint{2.590432in}{2.210709in}}%
\pgfpathlineto{\pgfqpoint{2.855513in}{2.451418in}}%
\pgfpathlineto{\pgfqpoint{2.410872in}{2.508083in}}%
\pgfpathlineto{\pgfqpoint{2.143341in}{2.291288in}}%
\pgfpathclose%
\pgfusepath{fill}%
\end{pgfscope}%
\begin{pgfscope}%
\pgfpathrectangle{\pgfqpoint{0.017500in}{0.017500in}}{\pgfqpoint{6.218282in}{3.580372in}} %
\pgfusepath{clip}%
\pgfsetbuttcap%
\pgfsetroundjoin%
\definecolor{currentfill}{rgb}{0.892138,0.425389,0.333289}%
\pgfsetfillcolor{currentfill}%
\pgfsetlinewidth{0.000000pt}%
\definecolor{currentstroke}{rgb}{0.000000,0.000000,0.000000}%
\pgfsetstrokecolor{currentstroke}%
\pgfsetdash{}{0pt}%
\pgfpathmoveto{\pgfqpoint{2.143341in}{2.291288in}}%
\pgfpathlineto{\pgfqpoint{2.590432in}{2.210709in}}%
\pgfpathlineto{\pgfqpoint{2.855513in}{2.451418in}}%
\pgfpathlineto{\pgfqpoint{2.410872in}{2.508083in}}%
\pgfpathlineto{\pgfqpoint{2.143341in}{2.291288in}}%
\pgfpathclose%
\pgfusepath{fill}%
\end{pgfscope}%
\begin{pgfscope}%
\pgfpathrectangle{\pgfqpoint{0.017500in}{0.017500in}}{\pgfqpoint{6.218282in}{3.580372in}} %
\pgfusepath{clip}%
\pgfsetbuttcap%
\pgfsetroundjoin%
\definecolor{currentfill}{rgb}{0.892138,0.425389,0.333289}%
\pgfsetfillcolor{currentfill}%
\pgfsetlinewidth{0.000000pt}%
\definecolor{currentstroke}{rgb}{0.000000,0.000000,0.000000}%
\pgfsetstrokecolor{currentstroke}%
\pgfsetdash{}{0pt}%
\pgfpathmoveto{\pgfqpoint{2.143341in}{2.291288in}}%
\pgfpathlineto{\pgfqpoint{2.590432in}{2.210709in}}%
\pgfpathlineto{\pgfqpoint{2.855513in}{2.451418in}}%
\pgfpathlineto{\pgfqpoint{2.410872in}{2.508083in}}%
\pgfpathlineto{\pgfqpoint{2.143341in}{2.291288in}}%
\pgfpathclose%
\pgfusepath{fill}%
\end{pgfscope}%
\begin{pgfscope}%
\pgfpathrectangle{\pgfqpoint{0.017500in}{0.017500in}}{\pgfqpoint{6.218282in}{3.580372in}} %
\pgfusepath{clip}%
\pgfsetbuttcap%
\pgfsetroundjoin%
\definecolor{currentfill}{rgb}{0.892138,0.425389,0.333289}%
\pgfsetfillcolor{currentfill}%
\pgfsetlinewidth{0.000000pt}%
\definecolor{currentstroke}{rgb}{0.000000,0.000000,0.000000}%
\pgfsetstrokecolor{currentstroke}%
\pgfsetdash{}{0pt}%
\pgfpathmoveto{\pgfqpoint{2.143341in}{2.291288in}}%
\pgfpathlineto{\pgfqpoint{2.590432in}{2.210709in}}%
\pgfpathlineto{\pgfqpoint{2.855513in}{2.451418in}}%
\pgfpathlineto{\pgfqpoint{2.410872in}{2.508083in}}%
\pgfpathlineto{\pgfqpoint{2.143341in}{2.291288in}}%
\pgfpathclose%
\pgfusepath{fill}%
\end{pgfscope}%
\begin{pgfscope}%
\pgfpathrectangle{\pgfqpoint{0.017500in}{0.017500in}}{\pgfqpoint{6.218282in}{3.580372in}} %
\pgfusepath{clip}%
\pgfsetbuttcap%
\pgfsetroundjoin%
\definecolor{currentfill}{rgb}{0.892138,0.425389,0.333289}%
\pgfsetfillcolor{currentfill}%
\pgfsetlinewidth{0.000000pt}%
\definecolor{currentstroke}{rgb}{0.000000,0.000000,0.000000}%
\pgfsetstrokecolor{currentstroke}%
\pgfsetdash{}{0pt}%
\pgfpathmoveto{\pgfqpoint{2.143341in}{2.291288in}}%
\pgfpathlineto{\pgfqpoint{2.590432in}{2.210709in}}%
\pgfpathlineto{\pgfqpoint{2.855513in}{2.451418in}}%
\pgfpathlineto{\pgfqpoint{2.410872in}{2.508083in}}%
\pgfpathlineto{\pgfqpoint{2.143341in}{2.291288in}}%
\pgfpathclose%
\pgfusepath{fill}%
\end{pgfscope}%
\begin{pgfscope}%
\pgfpathrectangle{\pgfqpoint{0.017500in}{0.017500in}}{\pgfqpoint{6.218282in}{3.580372in}} %
\pgfusepath{clip}%
\pgfsetbuttcap%
\pgfsetroundjoin%
\definecolor{currentfill}{rgb}{0.758112,0.168122,0.188827}%
\pgfsetfillcolor{currentfill}%
\pgfsetlinewidth{0.000000pt}%
\definecolor{currentstroke}{rgb}{0.000000,0.000000,0.000000}%
\pgfsetstrokecolor{currentstroke}%
\pgfsetdash{}{0pt}%
\pgfpathmoveto{\pgfqpoint{4.018447in}{2.468167in}}%
\pgfpathlineto{\pgfqpoint{4.478468in}{2.393972in}}%
\pgfpathlineto{\pgfqpoint{4.724651in}{2.549289in}}%
\pgfpathlineto{\pgfqpoint{4.267628in}{2.613520in}}%
\pgfpathlineto{\pgfqpoint{4.018447in}{2.468167in}}%
\pgfpathclose%
\pgfusepath{fill}%
\end{pgfscope}%
\begin{pgfscope}%
\pgfpathrectangle{\pgfqpoint{0.017500in}{0.017500in}}{\pgfqpoint{6.218282in}{3.580372in}} %
\pgfusepath{clip}%
\pgfsetbuttcap%
\pgfsetroundjoin%
\definecolor{currentfill}{rgb}{0.758112,0.168122,0.188827}%
\pgfsetfillcolor{currentfill}%
\pgfsetlinewidth{0.000000pt}%
\definecolor{currentstroke}{rgb}{0.000000,0.000000,0.000000}%
\pgfsetstrokecolor{currentstroke}%
\pgfsetdash{}{0pt}%
\pgfpathmoveto{\pgfqpoint{4.018447in}{2.468167in}}%
\pgfpathlineto{\pgfqpoint{4.478468in}{2.393972in}}%
\pgfpathlineto{\pgfqpoint{4.724651in}{2.549289in}}%
\pgfpathlineto{\pgfqpoint{4.267628in}{2.613520in}}%
\pgfpathlineto{\pgfqpoint{4.018447in}{2.468167in}}%
\pgfpathclose%
\pgfusepath{fill}%
\end{pgfscope}%
\begin{pgfscope}%
\pgfpathrectangle{\pgfqpoint{0.017500in}{0.017500in}}{\pgfqpoint{6.218282in}{3.580372in}} %
\pgfusepath{clip}%
\pgfsetbuttcap%
\pgfsetroundjoin%
\definecolor{currentfill}{rgb}{0.758112,0.168122,0.188827}%
\pgfsetfillcolor{currentfill}%
\pgfsetlinewidth{0.000000pt}%
\definecolor{currentstroke}{rgb}{0.000000,0.000000,0.000000}%
\pgfsetstrokecolor{currentstroke}%
\pgfsetdash{}{0pt}%
\pgfpathmoveto{\pgfqpoint{4.018447in}{2.468167in}}%
\pgfpathlineto{\pgfqpoint{4.478468in}{2.393972in}}%
\pgfpathlineto{\pgfqpoint{4.724651in}{2.549289in}}%
\pgfpathlineto{\pgfqpoint{4.267628in}{2.613520in}}%
\pgfpathlineto{\pgfqpoint{4.018447in}{2.468167in}}%
\pgfpathclose%
\pgfusepath{fill}%
\end{pgfscope}%
\begin{pgfscope}%
\pgfpathrectangle{\pgfqpoint{0.017500in}{0.017500in}}{\pgfqpoint{6.218282in}{3.580372in}} %
\pgfusepath{clip}%
\pgfsetbuttcap%
\pgfsetroundjoin%
\definecolor{currentfill}{rgb}{0.758112,0.168122,0.188827}%
\pgfsetfillcolor{currentfill}%
\pgfsetlinewidth{0.000000pt}%
\definecolor{currentstroke}{rgb}{0.000000,0.000000,0.000000}%
\pgfsetstrokecolor{currentstroke}%
\pgfsetdash{}{0pt}%
\pgfpathmoveto{\pgfqpoint{4.018447in}{2.468167in}}%
\pgfpathlineto{\pgfqpoint{4.478468in}{2.393972in}}%
\pgfpathlineto{\pgfqpoint{4.724651in}{2.549289in}}%
\pgfpathlineto{\pgfqpoint{4.267628in}{2.613520in}}%
\pgfpathlineto{\pgfqpoint{4.018447in}{2.468167in}}%
\pgfpathclose%
\pgfusepath{fill}%
\end{pgfscope}%
\begin{pgfscope}%
\pgfpathrectangle{\pgfqpoint{0.017500in}{0.017500in}}{\pgfqpoint{6.218282in}{3.580372in}} %
\pgfusepath{clip}%
\pgfsetbuttcap%
\pgfsetroundjoin%
\definecolor{currentfill}{rgb}{0.758112,0.168122,0.188827}%
\pgfsetfillcolor{currentfill}%
\pgfsetlinewidth{0.000000pt}%
\definecolor{currentstroke}{rgb}{0.000000,0.000000,0.000000}%
\pgfsetstrokecolor{currentstroke}%
\pgfsetdash{}{0pt}%
\pgfpathmoveto{\pgfqpoint{4.018447in}{2.468167in}}%
\pgfpathlineto{\pgfqpoint{4.478468in}{2.393972in}}%
\pgfpathlineto{\pgfqpoint{4.724651in}{2.549289in}}%
\pgfpathlineto{\pgfqpoint{4.267628in}{2.613520in}}%
\pgfpathlineto{\pgfqpoint{4.018447in}{2.468167in}}%
\pgfpathclose%
\pgfusepath{fill}%
\end{pgfscope}%
\begin{pgfscope}%
\pgfpathrectangle{\pgfqpoint{0.017500in}{0.017500in}}{\pgfqpoint{6.218282in}{3.580372in}} %
\pgfusepath{clip}%
\pgfsetbuttcap%
\pgfsetroundjoin%
\definecolor{currentfill}{rgb}{0.758112,0.168122,0.188827}%
\pgfsetfillcolor{currentfill}%
\pgfsetlinewidth{0.000000pt}%
\definecolor{currentstroke}{rgb}{0.000000,0.000000,0.000000}%
\pgfsetstrokecolor{currentstroke}%
\pgfsetdash{}{0pt}%
\pgfpathmoveto{\pgfqpoint{4.018447in}{2.468167in}}%
\pgfpathlineto{\pgfqpoint{4.478468in}{2.393972in}}%
\pgfpathlineto{\pgfqpoint{4.724651in}{2.549289in}}%
\pgfpathlineto{\pgfqpoint{4.267628in}{2.613520in}}%
\pgfpathlineto{\pgfqpoint{4.018447in}{2.468167in}}%
\pgfpathclose%
\pgfusepath{fill}%
\end{pgfscope}%
\begin{pgfscope}%
\pgfpathrectangle{\pgfqpoint{0.017500in}{0.017500in}}{\pgfqpoint{6.218282in}{3.580372in}} %
\pgfusepath{clip}%
\pgfsetbuttcap%
\pgfsetroundjoin%
\definecolor{currentfill}{rgb}{0.333490,0.446265,0.874452}%
\pgfsetfillcolor{currentfill}%
\pgfsetlinewidth{0.000000pt}%
\definecolor{currentstroke}{rgb}{0.000000,0.000000,0.000000}%
\pgfsetstrokecolor{currentstroke}%
\pgfsetdash{}{0pt}%
\pgfpathmoveto{\pgfqpoint{2.789950in}{0.798346in}}%
\pgfpathlineto{\pgfqpoint{3.236099in}{0.723497in}}%
\pgfpathlineto{\pgfqpoint{3.502561in}{2.073396in}}%
\pgfpathlineto{\pgfqpoint{3.043326in}{2.141782in}}%
\pgfpathlineto{\pgfqpoint{2.789950in}{0.798346in}}%
\pgfpathclose%
\pgfusepath{fill}%
\end{pgfscope}%
\begin{pgfscope}%
\pgfpathrectangle{\pgfqpoint{0.017500in}{0.017500in}}{\pgfqpoint{6.218282in}{3.580372in}} %
\pgfusepath{clip}%
\pgfsetbuttcap%
\pgfsetroundjoin%
\definecolor{currentfill}{rgb}{0.333490,0.446265,0.874452}%
\pgfsetfillcolor{currentfill}%
\pgfsetlinewidth{0.000000pt}%
\definecolor{currentstroke}{rgb}{0.000000,0.000000,0.000000}%
\pgfsetstrokecolor{currentstroke}%
\pgfsetdash{}{0pt}%
\pgfpathmoveto{\pgfqpoint{2.789950in}{0.798346in}}%
\pgfpathlineto{\pgfqpoint{3.236099in}{0.723497in}}%
\pgfpathlineto{\pgfqpoint{3.502561in}{2.073396in}}%
\pgfpathlineto{\pgfqpoint{3.043326in}{2.141782in}}%
\pgfpathlineto{\pgfqpoint{2.789950in}{0.798346in}}%
\pgfpathclose%
\pgfusepath{fill}%
\end{pgfscope}%
\begin{pgfscope}%
\pgfpathrectangle{\pgfqpoint{0.017500in}{0.017500in}}{\pgfqpoint{6.218282in}{3.580372in}} %
\pgfusepath{clip}%
\pgfsetbuttcap%
\pgfsetroundjoin%
\definecolor{currentfill}{rgb}{0.333490,0.446265,0.874452}%
\pgfsetfillcolor{currentfill}%
\pgfsetlinewidth{0.000000pt}%
\definecolor{currentstroke}{rgb}{0.000000,0.000000,0.000000}%
\pgfsetstrokecolor{currentstroke}%
\pgfsetdash{}{0pt}%
\pgfpathmoveto{\pgfqpoint{2.789950in}{0.798346in}}%
\pgfpathlineto{\pgfqpoint{3.236099in}{0.723497in}}%
\pgfpathlineto{\pgfqpoint{3.502561in}{2.073396in}}%
\pgfpathlineto{\pgfqpoint{3.043326in}{2.141782in}}%
\pgfpathlineto{\pgfqpoint{2.789950in}{0.798346in}}%
\pgfpathclose%
\pgfusepath{fill}%
\end{pgfscope}%
\begin{pgfscope}%
\pgfpathrectangle{\pgfqpoint{0.017500in}{0.017500in}}{\pgfqpoint{6.218282in}{3.580372in}} %
\pgfusepath{clip}%
\pgfsetbuttcap%
\pgfsetroundjoin%
\definecolor{currentfill}{rgb}{0.333490,0.446265,0.874452}%
\pgfsetfillcolor{currentfill}%
\pgfsetlinewidth{0.000000pt}%
\definecolor{currentstroke}{rgb}{0.000000,0.000000,0.000000}%
\pgfsetstrokecolor{currentstroke}%
\pgfsetdash{}{0pt}%
\pgfpathmoveto{\pgfqpoint{2.789950in}{0.798346in}}%
\pgfpathlineto{\pgfqpoint{3.236099in}{0.723497in}}%
\pgfpathlineto{\pgfqpoint{3.502561in}{2.073396in}}%
\pgfpathlineto{\pgfqpoint{3.043326in}{2.141782in}}%
\pgfpathlineto{\pgfqpoint{2.789950in}{0.798346in}}%
\pgfpathclose%
\pgfusepath{fill}%
\end{pgfscope}%
\begin{pgfscope}%
\pgfpathrectangle{\pgfqpoint{0.017500in}{0.017500in}}{\pgfqpoint{6.218282in}{3.580372in}} %
\pgfusepath{clip}%
\pgfsetbuttcap%
\pgfsetroundjoin%
\definecolor{currentfill}{rgb}{0.333490,0.446265,0.874452}%
\pgfsetfillcolor{currentfill}%
\pgfsetlinewidth{0.000000pt}%
\definecolor{currentstroke}{rgb}{0.000000,0.000000,0.000000}%
\pgfsetstrokecolor{currentstroke}%
\pgfsetdash{}{0pt}%
\pgfpathmoveto{\pgfqpoint{2.789950in}{0.798346in}}%
\pgfpathlineto{\pgfqpoint{3.236099in}{0.723497in}}%
\pgfpathlineto{\pgfqpoint{3.502561in}{2.073396in}}%
\pgfpathlineto{\pgfqpoint{3.043326in}{2.141782in}}%
\pgfpathlineto{\pgfqpoint{2.789950in}{0.798346in}}%
\pgfpathclose%
\pgfusepath{fill}%
\end{pgfscope}%
\begin{pgfscope}%
\pgfpathrectangle{\pgfqpoint{0.017500in}{0.017500in}}{\pgfqpoint{6.218282in}{3.580372in}} %
\pgfusepath{clip}%
\pgfsetbuttcap%
\pgfsetroundjoin%
\definecolor{currentfill}{rgb}{0.333490,0.446265,0.874452}%
\pgfsetfillcolor{currentfill}%
\pgfsetlinewidth{0.000000pt}%
\definecolor{currentstroke}{rgb}{0.000000,0.000000,0.000000}%
\pgfsetstrokecolor{currentstroke}%
\pgfsetdash{}{0pt}%
\pgfpathmoveto{\pgfqpoint{2.789950in}{0.798346in}}%
\pgfpathlineto{\pgfqpoint{3.236099in}{0.723497in}}%
\pgfpathlineto{\pgfqpoint{3.502561in}{2.073396in}}%
\pgfpathlineto{\pgfqpoint{3.043326in}{2.141782in}}%
\pgfpathlineto{\pgfqpoint{2.789950in}{0.798346in}}%
\pgfpathclose%
\pgfusepath{fill}%
\end{pgfscope}%
\begin{pgfscope}%
\pgfpathrectangle{\pgfqpoint{0.017500in}{0.017500in}}{\pgfqpoint{6.218282in}{3.580372in}} %
\pgfusepath{clip}%
\pgfsetbuttcap%
\pgfsetroundjoin%
\definecolor{currentfill}{rgb}{0.800830,0.250829,0.225696}%
\pgfsetfillcolor{currentfill}%
\pgfsetlinewidth{0.000000pt}%
\definecolor{currentstroke}{rgb}{0.000000,0.000000,0.000000}%
\pgfsetstrokecolor{currentstroke}%
\pgfsetdash{}{0pt}%
\pgfpathmoveto{\pgfqpoint{3.306485in}{2.382499in}}%
\pgfpathlineto{\pgfqpoint{3.564492in}{2.541257in}}%
\pgfpathlineto{\pgfqpoint{3.306485in}{2.382499in}}%
\pgfpathclose%
\pgfusepath{fill}%
\end{pgfscope}%
\begin{pgfscope}%
\pgfpathrectangle{\pgfqpoint{0.017500in}{0.017500in}}{\pgfqpoint{6.218282in}{3.580372in}} %
\pgfusepath{clip}%
\pgfsetbuttcap%
\pgfsetroundjoin%
\definecolor{currentfill}{rgb}{0.800830,0.250829,0.225696}%
\pgfsetfillcolor{currentfill}%
\pgfsetlinewidth{0.000000pt}%
\definecolor{currentstroke}{rgb}{0.000000,0.000000,0.000000}%
\pgfsetstrokecolor{currentstroke}%
\pgfsetdash{}{0pt}%
\pgfpathmoveto{\pgfqpoint{3.306485in}{2.382499in}}%
\pgfpathlineto{\pgfqpoint{3.564492in}{2.541257in}}%
\pgfpathlineto{\pgfqpoint{3.306485in}{2.382499in}}%
\pgfpathclose%
\pgfusepath{fill}%
\end{pgfscope}%
\begin{pgfscope}%
\pgfpathrectangle{\pgfqpoint{0.017500in}{0.017500in}}{\pgfqpoint{6.218282in}{3.580372in}} %
\pgfusepath{clip}%
\pgfsetbuttcap%
\pgfsetroundjoin%
\definecolor{currentfill}{rgb}{0.800830,0.250829,0.225696}%
\pgfsetfillcolor{currentfill}%
\pgfsetlinewidth{0.000000pt}%
\definecolor{currentstroke}{rgb}{0.000000,0.000000,0.000000}%
\pgfsetstrokecolor{currentstroke}%
\pgfsetdash{}{0pt}%
\pgfpathmoveto{\pgfqpoint{3.306485in}{2.382499in}}%
\pgfpathlineto{\pgfqpoint{3.564492in}{2.541257in}}%
\pgfpathlineto{\pgfqpoint{3.306485in}{2.382499in}}%
\pgfpathclose%
\pgfusepath{fill}%
\end{pgfscope}%
\begin{pgfscope}%
\pgfpathrectangle{\pgfqpoint{0.017500in}{0.017500in}}{\pgfqpoint{6.218282in}{3.580372in}} %
\pgfusepath{clip}%
\pgfsetbuttcap%
\pgfsetroundjoin%
\definecolor{currentfill}{rgb}{0.800830,0.250829,0.225696}%
\pgfsetfillcolor{currentfill}%
\pgfsetlinewidth{0.000000pt}%
\definecolor{currentstroke}{rgb}{0.000000,0.000000,0.000000}%
\pgfsetstrokecolor{currentstroke}%
\pgfsetdash{}{0pt}%
\pgfpathmoveto{\pgfqpoint{3.306485in}{2.382499in}}%
\pgfpathlineto{\pgfqpoint{3.564492in}{2.541257in}}%
\pgfpathlineto{\pgfqpoint{3.306485in}{2.382499in}}%
\pgfpathclose%
\pgfusepath{fill}%
\end{pgfscope}%
\begin{pgfscope}%
\pgfpathrectangle{\pgfqpoint{0.017500in}{0.017500in}}{\pgfqpoint{6.218282in}{3.580372in}} %
\pgfusepath{clip}%
\pgfsetbuttcap%
\pgfsetroundjoin%
\definecolor{currentfill}{rgb}{0.800830,0.250829,0.225696}%
\pgfsetfillcolor{currentfill}%
\pgfsetlinewidth{0.000000pt}%
\definecolor{currentstroke}{rgb}{0.000000,0.000000,0.000000}%
\pgfsetstrokecolor{currentstroke}%
\pgfsetdash{}{0pt}%
\pgfpathmoveto{\pgfqpoint{3.306485in}{2.382499in}}%
\pgfpathlineto{\pgfqpoint{3.564492in}{2.541257in}}%
\pgfpathlineto{\pgfqpoint{3.306485in}{2.382499in}}%
\pgfpathclose%
\pgfusepath{fill}%
\end{pgfscope}%
\begin{pgfscope}%
\pgfpathrectangle{\pgfqpoint{0.017500in}{0.017500in}}{\pgfqpoint{6.218282in}{3.580372in}} %
\pgfusepath{clip}%
\pgfsetbuttcap%
\pgfsetroundjoin%
\definecolor{currentfill}{rgb}{0.800830,0.250829,0.225696}%
\pgfsetfillcolor{currentfill}%
\pgfsetlinewidth{0.000000pt}%
\definecolor{currentstroke}{rgb}{0.000000,0.000000,0.000000}%
\pgfsetstrokecolor{currentstroke}%
\pgfsetdash{}{0pt}%
\pgfpathmoveto{\pgfqpoint{3.306485in}{2.382499in}}%
\pgfpathlineto{\pgfqpoint{3.564492in}{2.541257in}}%
\pgfpathlineto{\pgfqpoint{3.306485in}{2.382499in}}%
\pgfpathclose%
\pgfusepath{fill}%
\end{pgfscope}%
\begin{pgfscope}%
\pgfpathrectangle{\pgfqpoint{0.017500in}{0.017500in}}{\pgfqpoint{6.218282in}{3.580372in}} %
\pgfusepath{clip}%
\pgfsetbuttcap%
\pgfsetroundjoin%
\definecolor{currentfill}{rgb}{0.800830,0.250829,0.225696}%
\pgfsetfillcolor{currentfill}%
\pgfsetlinewidth{0.000000pt}%
\definecolor{currentstroke}{rgb}{0.000000,0.000000,0.000000}%
\pgfsetstrokecolor{currentstroke}%
\pgfsetdash{}{0pt}%
\pgfpathmoveto{\pgfqpoint{3.306485in}{2.382499in}}%
\pgfpathlineto{\pgfqpoint{3.564492in}{2.541257in}}%
\pgfpathlineto{\pgfqpoint{3.306485in}{2.382499in}}%
\pgfpathclose%
\pgfusepath{fill}%
\end{pgfscope}%
\begin{pgfscope}%
\pgfpathrectangle{\pgfqpoint{0.017500in}{0.017500in}}{\pgfqpoint{6.218282in}{3.580372in}} %
\pgfusepath{clip}%
\pgfsetbuttcap%
\pgfsetroundjoin%
\definecolor{currentfill}{rgb}{0.800830,0.250829,0.225696}%
\pgfsetfillcolor{currentfill}%
\pgfsetlinewidth{0.000000pt}%
\definecolor{currentstroke}{rgb}{0.000000,0.000000,0.000000}%
\pgfsetstrokecolor{currentstroke}%
\pgfsetdash{}{0pt}%
\pgfpathmoveto{\pgfqpoint{3.306485in}{2.382499in}}%
\pgfpathlineto{\pgfqpoint{3.564492in}{2.541257in}}%
\pgfpathlineto{\pgfqpoint{3.306485in}{2.382499in}}%
\pgfpathclose%
\pgfusepath{fill}%
\end{pgfscope}%
\begin{pgfscope}%
\pgfpathrectangle{\pgfqpoint{0.017500in}{0.017500in}}{\pgfqpoint{6.218282in}{3.580372in}} %
\pgfusepath{clip}%
\pgfsetbuttcap%
\pgfsetroundjoin%
\definecolor{currentfill}{rgb}{0.800830,0.250829,0.225696}%
\pgfsetfillcolor{currentfill}%
\pgfsetlinewidth{0.000000pt}%
\definecolor{currentstroke}{rgb}{0.000000,0.000000,0.000000}%
\pgfsetstrokecolor{currentstroke}%
\pgfsetdash{}{0pt}%
\pgfpathmoveto{\pgfqpoint{3.306485in}{2.382499in}}%
\pgfpathlineto{\pgfqpoint{3.564492in}{2.541257in}}%
\pgfpathlineto{\pgfqpoint{3.306485in}{2.382499in}}%
\pgfpathclose%
\pgfusepath{fill}%
\end{pgfscope}%
\begin{pgfscope}%
\pgfpathrectangle{\pgfqpoint{0.017500in}{0.017500in}}{\pgfqpoint{6.218282in}{3.580372in}} %
\pgfusepath{clip}%
\pgfsetbuttcap%
\pgfsetroundjoin%
\definecolor{currentfill}{rgb}{0.800830,0.250829,0.225696}%
\pgfsetfillcolor{currentfill}%
\pgfsetlinewidth{0.000000pt}%
\definecolor{currentstroke}{rgb}{0.000000,0.000000,0.000000}%
\pgfsetstrokecolor{currentstroke}%
\pgfsetdash{}{0pt}%
\pgfpathmoveto{\pgfqpoint{3.306485in}{2.382499in}}%
\pgfpathlineto{\pgfqpoint{3.564492in}{2.541257in}}%
\pgfpathlineto{\pgfqpoint{3.306485in}{2.382499in}}%
\pgfpathclose%
\pgfusepath{fill}%
\end{pgfscope}%
\begin{pgfscope}%
\pgfpathrectangle{\pgfqpoint{0.017500in}{0.017500in}}{\pgfqpoint{6.218282in}{3.580372in}} %
\pgfusepath{clip}%
\pgfsetbuttcap%
\pgfsetroundjoin%
\definecolor{currentfill}{rgb}{0.800830,0.250829,0.225696}%
\pgfsetfillcolor{currentfill}%
\pgfsetlinewidth{0.000000pt}%
\definecolor{currentstroke}{rgb}{0.000000,0.000000,0.000000}%
\pgfsetstrokecolor{currentstroke}%
\pgfsetdash{}{0pt}%
\pgfpathmoveto{\pgfqpoint{3.306485in}{2.382499in}}%
\pgfpathlineto{\pgfqpoint{3.564492in}{2.541257in}}%
\pgfpathlineto{\pgfqpoint{3.306485in}{2.382499in}}%
\pgfpathclose%
\pgfusepath{fill}%
\end{pgfscope}%
\begin{pgfscope}%
\pgfpathrectangle{\pgfqpoint{0.017500in}{0.017500in}}{\pgfqpoint{6.218282in}{3.580372in}} %
\pgfusepath{clip}%
\pgfsetbuttcap%
\pgfsetroundjoin%
\definecolor{currentfill}{rgb}{0.800830,0.250829,0.225696}%
\pgfsetfillcolor{currentfill}%
\pgfsetlinewidth{0.000000pt}%
\definecolor{currentstroke}{rgb}{0.000000,0.000000,0.000000}%
\pgfsetstrokecolor{currentstroke}%
\pgfsetdash{}{0pt}%
\pgfpathmoveto{\pgfqpoint{3.306485in}{2.382499in}}%
\pgfpathlineto{\pgfqpoint{3.564492in}{2.541257in}}%
\pgfpathlineto{\pgfqpoint{3.306485in}{2.382499in}}%
\pgfpathclose%
\pgfusepath{fill}%
\end{pgfscope}%
\begin{pgfscope}%
\pgfpathrectangle{\pgfqpoint{0.017500in}{0.017500in}}{\pgfqpoint{6.218282in}{3.580372in}} %
\pgfusepath{clip}%
\pgfsetbuttcap%
\pgfsetroundjoin%
\definecolor{currentfill}{rgb}{0.800830,0.250829,0.225696}%
\pgfsetfillcolor{currentfill}%
\pgfsetlinewidth{0.000000pt}%
\definecolor{currentstroke}{rgb}{0.000000,0.000000,0.000000}%
\pgfsetstrokecolor{currentstroke}%
\pgfsetdash{}{0pt}%
\pgfpathmoveto{\pgfqpoint{3.306485in}{2.382499in}}%
\pgfpathlineto{\pgfqpoint{3.564492in}{2.541257in}}%
\pgfpathlineto{\pgfqpoint{3.306485in}{2.382499in}}%
\pgfpathclose%
\pgfusepath{fill}%
\end{pgfscope}%
\begin{pgfscope}%
\pgfpathrectangle{\pgfqpoint{0.017500in}{0.017500in}}{\pgfqpoint{6.218282in}{3.580372in}} %
\pgfusepath{clip}%
\pgfsetbuttcap%
\pgfsetroundjoin%
\definecolor{currentfill}{rgb}{0.800830,0.250829,0.225696}%
\pgfsetfillcolor{currentfill}%
\pgfsetlinewidth{0.000000pt}%
\definecolor{currentstroke}{rgb}{0.000000,0.000000,0.000000}%
\pgfsetstrokecolor{currentstroke}%
\pgfsetdash{}{0pt}%
\pgfpathmoveto{\pgfqpoint{3.306485in}{2.382499in}}%
\pgfpathlineto{\pgfqpoint{3.564492in}{2.541257in}}%
\pgfpathlineto{\pgfqpoint{3.306485in}{2.382499in}}%
\pgfpathclose%
\pgfusepath{fill}%
\end{pgfscope}%
\begin{pgfscope}%
\pgfpathrectangle{\pgfqpoint{0.017500in}{0.017500in}}{\pgfqpoint{6.218282in}{3.580372in}} %
\pgfusepath{clip}%
\pgfsetbuttcap%
\pgfsetroundjoin%
\definecolor{currentfill}{rgb}{0.800830,0.250829,0.225696}%
\pgfsetfillcolor{currentfill}%
\pgfsetlinewidth{0.000000pt}%
\definecolor{currentstroke}{rgb}{0.000000,0.000000,0.000000}%
\pgfsetstrokecolor{currentstroke}%
\pgfsetdash{}{0pt}%
\pgfpathmoveto{\pgfqpoint{3.306485in}{2.382499in}}%
\pgfpathlineto{\pgfqpoint{3.564492in}{2.541257in}}%
\pgfpathlineto{\pgfqpoint{3.306485in}{2.382499in}}%
\pgfpathclose%
\pgfusepath{fill}%
\end{pgfscope}%
\begin{pgfscope}%
\pgfpathrectangle{\pgfqpoint{0.017500in}{0.017500in}}{\pgfqpoint{6.218282in}{3.580372in}} %
\pgfusepath{clip}%
\pgfsetbuttcap%
\pgfsetroundjoin%
\definecolor{currentfill}{rgb}{0.800830,0.250829,0.225696}%
\pgfsetfillcolor{currentfill}%
\pgfsetlinewidth{0.000000pt}%
\definecolor{currentstroke}{rgb}{0.000000,0.000000,0.000000}%
\pgfsetstrokecolor{currentstroke}%
\pgfsetdash{}{0pt}%
\pgfpathmoveto{\pgfqpoint{3.306485in}{2.382499in}}%
\pgfpathlineto{\pgfqpoint{3.564492in}{2.541257in}}%
\pgfpathlineto{\pgfqpoint{3.306485in}{2.382499in}}%
\pgfpathclose%
\pgfusepath{fill}%
\end{pgfscope}%
\begin{pgfscope}%
\pgfpathrectangle{\pgfqpoint{0.017500in}{0.017500in}}{\pgfqpoint{6.218282in}{3.580372in}} %
\pgfusepath{clip}%
\pgfsetbuttcap%
\pgfsetroundjoin%
\definecolor{currentfill}{rgb}{0.800830,0.250829,0.225696}%
\pgfsetfillcolor{currentfill}%
\pgfsetlinewidth{0.000000pt}%
\definecolor{currentstroke}{rgb}{0.000000,0.000000,0.000000}%
\pgfsetstrokecolor{currentstroke}%
\pgfsetdash{}{0pt}%
\pgfpathmoveto{\pgfqpoint{3.306485in}{2.382499in}}%
\pgfpathlineto{\pgfqpoint{3.564492in}{2.541257in}}%
\pgfpathlineto{\pgfqpoint{3.306485in}{2.382499in}}%
\pgfpathclose%
\pgfusepath{fill}%
\end{pgfscope}%
\begin{pgfscope}%
\pgfpathrectangle{\pgfqpoint{0.017500in}{0.017500in}}{\pgfqpoint{6.218282in}{3.580372in}} %
\pgfusepath{clip}%
\pgfsetbuttcap%
\pgfsetroundjoin%
\definecolor{currentfill}{rgb}{0.800830,0.250829,0.225696}%
\pgfsetfillcolor{currentfill}%
\pgfsetlinewidth{0.000000pt}%
\definecolor{currentstroke}{rgb}{0.000000,0.000000,0.000000}%
\pgfsetstrokecolor{currentstroke}%
\pgfsetdash{}{0pt}%
\pgfpathmoveto{\pgfqpoint{3.306485in}{2.382499in}}%
\pgfpathlineto{\pgfqpoint{3.564492in}{2.541257in}}%
\pgfpathlineto{\pgfqpoint{3.306485in}{2.382499in}}%
\pgfpathclose%
\pgfusepath{fill}%
\end{pgfscope}%
\begin{pgfscope}%
\pgfpathrectangle{\pgfqpoint{0.017500in}{0.017500in}}{\pgfqpoint{6.218282in}{3.580372in}} %
\pgfusepath{clip}%
\pgfsetbuttcap%
\pgfsetroundjoin%
\definecolor{currentfill}{rgb}{0.800830,0.250829,0.225696}%
\pgfsetfillcolor{currentfill}%
\pgfsetlinewidth{0.000000pt}%
\definecolor{currentstroke}{rgb}{0.000000,0.000000,0.000000}%
\pgfsetstrokecolor{currentstroke}%
\pgfsetdash{}{0pt}%
\pgfpathmoveto{\pgfqpoint{3.306485in}{2.382499in}}%
\pgfpathlineto{\pgfqpoint{3.564492in}{2.541257in}}%
\pgfpathlineto{\pgfqpoint{3.306485in}{2.382499in}}%
\pgfpathclose%
\pgfusepath{fill}%
\end{pgfscope}%
\begin{pgfscope}%
\pgfpathrectangle{\pgfqpoint{0.017500in}{0.017500in}}{\pgfqpoint{6.218282in}{3.580372in}} %
\pgfusepath{clip}%
\pgfsetbuttcap%
\pgfsetroundjoin%
\definecolor{currentfill}{rgb}{0.800830,0.250829,0.225696}%
\pgfsetfillcolor{currentfill}%
\pgfsetlinewidth{0.000000pt}%
\definecolor{currentstroke}{rgb}{0.000000,0.000000,0.000000}%
\pgfsetstrokecolor{currentstroke}%
\pgfsetdash{}{0pt}%
\pgfpathmoveto{\pgfqpoint{3.306485in}{2.382499in}}%
\pgfpathlineto{\pgfqpoint{3.564492in}{2.541257in}}%
\pgfpathlineto{\pgfqpoint{3.306485in}{2.382499in}}%
\pgfpathclose%
\pgfusepath{fill}%
\end{pgfscope}%
\begin{pgfscope}%
\pgfpathrectangle{\pgfqpoint{0.017500in}{0.017500in}}{\pgfqpoint{6.218282in}{3.580372in}} %
\pgfusepath{clip}%
\pgfsetbuttcap%
\pgfsetroundjoin%
\definecolor{currentfill}{rgb}{0.800830,0.250829,0.225696}%
\pgfsetfillcolor{currentfill}%
\pgfsetlinewidth{0.000000pt}%
\definecolor{currentstroke}{rgb}{0.000000,0.000000,0.000000}%
\pgfsetstrokecolor{currentstroke}%
\pgfsetdash{}{0pt}%
\pgfpathmoveto{\pgfqpoint{3.306485in}{2.382499in}}%
\pgfpathlineto{\pgfqpoint{3.564492in}{2.541257in}}%
\pgfpathlineto{\pgfqpoint{3.306485in}{2.382499in}}%
\pgfpathclose%
\pgfusepath{fill}%
\end{pgfscope}%
\begin{pgfscope}%
\pgfpathrectangle{\pgfqpoint{0.017500in}{0.017500in}}{\pgfqpoint{6.218282in}{3.580372in}} %
\pgfusepath{clip}%
\pgfsetbuttcap%
\pgfsetroundjoin%
\definecolor{currentfill}{rgb}{0.800830,0.250829,0.225696}%
\pgfsetfillcolor{currentfill}%
\pgfsetlinewidth{0.000000pt}%
\definecolor{currentstroke}{rgb}{0.000000,0.000000,0.000000}%
\pgfsetstrokecolor{currentstroke}%
\pgfsetdash{}{0pt}%
\pgfpathmoveto{\pgfqpoint{3.306485in}{2.382499in}}%
\pgfpathlineto{\pgfqpoint{3.564492in}{2.541257in}}%
\pgfpathlineto{\pgfqpoint{3.306485in}{2.382499in}}%
\pgfpathclose%
\pgfusepath{fill}%
\end{pgfscope}%
\begin{pgfscope}%
\pgfpathrectangle{\pgfqpoint{0.017500in}{0.017500in}}{\pgfqpoint{6.218282in}{3.580372in}} %
\pgfusepath{clip}%
\pgfsetbuttcap%
\pgfsetroundjoin%
\definecolor{currentfill}{rgb}{0.800830,0.250829,0.225696}%
\pgfsetfillcolor{currentfill}%
\pgfsetlinewidth{0.000000pt}%
\definecolor{currentstroke}{rgb}{0.000000,0.000000,0.000000}%
\pgfsetstrokecolor{currentstroke}%
\pgfsetdash{}{0pt}%
\pgfpathmoveto{\pgfqpoint{3.306485in}{2.382499in}}%
\pgfpathlineto{\pgfqpoint{3.564492in}{2.541257in}}%
\pgfpathlineto{\pgfqpoint{3.306485in}{2.382499in}}%
\pgfpathclose%
\pgfusepath{fill}%
\end{pgfscope}%
\begin{pgfscope}%
\pgfpathrectangle{\pgfqpoint{0.017500in}{0.017500in}}{\pgfqpoint{6.218282in}{3.580372in}} %
\pgfusepath{clip}%
\pgfsetbuttcap%
\pgfsetroundjoin%
\definecolor{currentfill}{rgb}{0.800830,0.250829,0.225696}%
\pgfsetfillcolor{currentfill}%
\pgfsetlinewidth{0.000000pt}%
\definecolor{currentstroke}{rgb}{0.000000,0.000000,0.000000}%
\pgfsetstrokecolor{currentstroke}%
\pgfsetdash{}{0pt}%
\pgfpathmoveto{\pgfqpoint{3.306485in}{2.382499in}}%
\pgfpathlineto{\pgfqpoint{3.564492in}{2.541257in}}%
\pgfpathlineto{\pgfqpoint{3.306485in}{2.382499in}}%
\pgfpathclose%
\pgfusepath{fill}%
\end{pgfscope}%
\begin{pgfscope}%
\pgfpathrectangle{\pgfqpoint{0.017500in}{0.017500in}}{\pgfqpoint{6.218282in}{3.580372in}} %
\pgfusepath{clip}%
\pgfsetbuttcap%
\pgfsetroundjoin%
\definecolor{currentfill}{rgb}{0.800830,0.250829,0.225696}%
\pgfsetfillcolor{currentfill}%
\pgfsetlinewidth{0.000000pt}%
\definecolor{currentstroke}{rgb}{0.000000,0.000000,0.000000}%
\pgfsetstrokecolor{currentstroke}%
\pgfsetdash{}{0pt}%
\pgfpathmoveto{\pgfqpoint{3.306485in}{2.382499in}}%
\pgfpathlineto{\pgfqpoint{3.564492in}{2.541257in}}%
\pgfpathlineto{\pgfqpoint{3.306485in}{2.382499in}}%
\pgfpathclose%
\pgfusepath{fill}%
\end{pgfscope}%
\begin{pgfscope}%
\pgfpathrectangle{\pgfqpoint{0.017500in}{0.017500in}}{\pgfqpoint{6.218282in}{3.580372in}} %
\pgfusepath{clip}%
\pgfsetbuttcap%
\pgfsetroundjoin%
\definecolor{currentfill}{rgb}{0.800830,0.250829,0.225696}%
\pgfsetfillcolor{currentfill}%
\pgfsetlinewidth{0.000000pt}%
\definecolor{currentstroke}{rgb}{0.000000,0.000000,0.000000}%
\pgfsetstrokecolor{currentstroke}%
\pgfsetdash{}{0pt}%
\pgfpathmoveto{\pgfqpoint{3.306485in}{2.382499in}}%
\pgfpathlineto{\pgfqpoint{3.564492in}{2.541257in}}%
\pgfpathlineto{\pgfqpoint{3.306485in}{2.382499in}}%
\pgfpathclose%
\pgfusepath{fill}%
\end{pgfscope}%
\begin{pgfscope}%
\pgfpathrectangle{\pgfqpoint{0.017500in}{0.017500in}}{\pgfqpoint{6.218282in}{3.580372in}} %
\pgfusepath{clip}%
\pgfsetbuttcap%
\pgfsetroundjoin%
\definecolor{currentfill}{rgb}{0.800830,0.250829,0.225696}%
\pgfsetfillcolor{currentfill}%
\pgfsetlinewidth{0.000000pt}%
\definecolor{currentstroke}{rgb}{0.000000,0.000000,0.000000}%
\pgfsetstrokecolor{currentstroke}%
\pgfsetdash{}{0pt}%
\pgfpathmoveto{\pgfqpoint{3.306485in}{2.382499in}}%
\pgfpathlineto{\pgfqpoint{3.564492in}{2.541257in}}%
\pgfpathlineto{\pgfqpoint{3.306485in}{2.382499in}}%
\pgfpathclose%
\pgfusepath{fill}%
\end{pgfscope}%
\begin{pgfscope}%
\pgfpathrectangle{\pgfqpoint{0.017500in}{0.017500in}}{\pgfqpoint{6.218282in}{3.580372in}} %
\pgfusepath{clip}%
\pgfsetbuttcap%
\pgfsetroundjoin%
\definecolor{currentfill}{rgb}{0.800830,0.250829,0.225696}%
\pgfsetfillcolor{currentfill}%
\pgfsetlinewidth{0.000000pt}%
\definecolor{currentstroke}{rgb}{0.000000,0.000000,0.000000}%
\pgfsetstrokecolor{currentstroke}%
\pgfsetdash{}{0pt}%
\pgfpathmoveto{\pgfqpoint{3.306485in}{2.382499in}}%
\pgfpathlineto{\pgfqpoint{3.564492in}{2.541257in}}%
\pgfpathlineto{\pgfqpoint{3.306485in}{2.382499in}}%
\pgfpathclose%
\pgfusepath{fill}%
\end{pgfscope}%
\begin{pgfscope}%
\pgfpathrectangle{\pgfqpoint{0.017500in}{0.017500in}}{\pgfqpoint{6.218282in}{3.580372in}} %
\pgfusepath{clip}%
\pgfsetbuttcap%
\pgfsetroundjoin%
\definecolor{currentfill}{rgb}{0.800830,0.250829,0.225696}%
\pgfsetfillcolor{currentfill}%
\pgfsetlinewidth{0.000000pt}%
\definecolor{currentstroke}{rgb}{0.000000,0.000000,0.000000}%
\pgfsetstrokecolor{currentstroke}%
\pgfsetdash{}{0pt}%
\pgfpathmoveto{\pgfqpoint{3.306485in}{2.382499in}}%
\pgfpathlineto{\pgfqpoint{3.564492in}{2.541257in}}%
\pgfpathlineto{\pgfqpoint{3.306485in}{2.382499in}}%
\pgfpathclose%
\pgfusepath{fill}%
\end{pgfscope}%
\begin{pgfscope}%
\pgfpathrectangle{\pgfqpoint{0.017500in}{0.017500in}}{\pgfqpoint{6.218282in}{3.580372in}} %
\pgfusepath{clip}%
\pgfsetbuttcap%
\pgfsetroundjoin%
\definecolor{currentfill}{rgb}{0.800830,0.250829,0.225696}%
\pgfsetfillcolor{currentfill}%
\pgfsetlinewidth{0.000000pt}%
\definecolor{currentstroke}{rgb}{0.000000,0.000000,0.000000}%
\pgfsetstrokecolor{currentstroke}%
\pgfsetdash{}{0pt}%
\pgfpathmoveto{\pgfqpoint{3.306485in}{2.382499in}}%
\pgfpathlineto{\pgfqpoint{3.564492in}{2.541257in}}%
\pgfpathlineto{\pgfqpoint{3.306485in}{2.382499in}}%
\pgfpathclose%
\pgfusepath{fill}%
\end{pgfscope}%
\begin{pgfscope}%
\pgfpathrectangle{\pgfqpoint{0.017500in}{0.017500in}}{\pgfqpoint{6.218282in}{3.580372in}} %
\pgfusepath{clip}%
\pgfsetbuttcap%
\pgfsetroundjoin%
\definecolor{currentfill}{rgb}{0.800830,0.250829,0.225696}%
\pgfsetfillcolor{currentfill}%
\pgfsetlinewidth{0.000000pt}%
\definecolor{currentstroke}{rgb}{0.000000,0.000000,0.000000}%
\pgfsetstrokecolor{currentstroke}%
\pgfsetdash{}{0pt}%
\pgfpathmoveto{\pgfqpoint{3.306485in}{2.382499in}}%
\pgfpathlineto{\pgfqpoint{3.763398in}{2.297457in}}%
\pgfpathlineto{\pgfqpoint{4.018447in}{2.468167in}}%
\pgfpathlineto{\pgfqpoint{3.564492in}{2.541257in}}%
\pgfpathlineto{\pgfqpoint{3.306485in}{2.382499in}}%
\pgfpathclose%
\pgfusepath{fill}%
\end{pgfscope}%
\begin{pgfscope}%
\pgfpathrectangle{\pgfqpoint{0.017500in}{0.017500in}}{\pgfqpoint{6.218282in}{3.580372in}} %
\pgfusepath{clip}%
\pgfsetbuttcap%
\pgfsetroundjoin%
\definecolor{currentfill}{rgb}{0.800830,0.250829,0.225696}%
\pgfsetfillcolor{currentfill}%
\pgfsetlinewidth{0.000000pt}%
\definecolor{currentstroke}{rgb}{0.000000,0.000000,0.000000}%
\pgfsetstrokecolor{currentstroke}%
\pgfsetdash{}{0pt}%
\pgfpathmoveto{\pgfqpoint{3.306485in}{2.382499in}}%
\pgfpathlineto{\pgfqpoint{3.763398in}{2.297457in}}%
\pgfpathlineto{\pgfqpoint{4.018447in}{2.468167in}}%
\pgfpathlineto{\pgfqpoint{3.564492in}{2.541257in}}%
\pgfpathlineto{\pgfqpoint{3.306485in}{2.382499in}}%
\pgfpathclose%
\pgfusepath{fill}%
\end{pgfscope}%
\begin{pgfscope}%
\pgfpathrectangle{\pgfqpoint{0.017500in}{0.017500in}}{\pgfqpoint{6.218282in}{3.580372in}} %
\pgfusepath{clip}%
\pgfsetbuttcap%
\pgfsetroundjoin%
\definecolor{currentfill}{rgb}{0.800830,0.250829,0.225696}%
\pgfsetfillcolor{currentfill}%
\pgfsetlinewidth{0.000000pt}%
\definecolor{currentstroke}{rgb}{0.000000,0.000000,0.000000}%
\pgfsetstrokecolor{currentstroke}%
\pgfsetdash{}{0pt}%
\pgfpathmoveto{\pgfqpoint{3.306485in}{2.382499in}}%
\pgfpathlineto{\pgfqpoint{3.763398in}{2.297457in}}%
\pgfpathlineto{\pgfqpoint{4.018447in}{2.468167in}}%
\pgfpathlineto{\pgfqpoint{3.564492in}{2.541257in}}%
\pgfpathlineto{\pgfqpoint{3.306485in}{2.382499in}}%
\pgfpathclose%
\pgfusepath{fill}%
\end{pgfscope}%
\begin{pgfscope}%
\pgfpathrectangle{\pgfqpoint{0.017500in}{0.017500in}}{\pgfqpoint{6.218282in}{3.580372in}} %
\pgfusepath{clip}%
\pgfsetbuttcap%
\pgfsetroundjoin%
\definecolor{currentfill}{rgb}{0.800830,0.250829,0.225696}%
\pgfsetfillcolor{currentfill}%
\pgfsetlinewidth{0.000000pt}%
\definecolor{currentstroke}{rgb}{0.000000,0.000000,0.000000}%
\pgfsetstrokecolor{currentstroke}%
\pgfsetdash{}{0pt}%
\pgfpathmoveto{\pgfqpoint{3.306485in}{2.382499in}}%
\pgfpathlineto{\pgfqpoint{3.763398in}{2.297457in}}%
\pgfpathlineto{\pgfqpoint{4.018447in}{2.468167in}}%
\pgfpathlineto{\pgfqpoint{3.564492in}{2.541257in}}%
\pgfpathlineto{\pgfqpoint{3.306485in}{2.382499in}}%
\pgfpathclose%
\pgfusepath{fill}%
\end{pgfscope}%
\begin{pgfscope}%
\pgfpathrectangle{\pgfqpoint{0.017500in}{0.017500in}}{\pgfqpoint{6.218282in}{3.580372in}} %
\pgfusepath{clip}%
\pgfsetbuttcap%
\pgfsetroundjoin%
\definecolor{currentfill}{rgb}{0.800830,0.250829,0.225696}%
\pgfsetfillcolor{currentfill}%
\pgfsetlinewidth{0.000000pt}%
\definecolor{currentstroke}{rgb}{0.000000,0.000000,0.000000}%
\pgfsetstrokecolor{currentstroke}%
\pgfsetdash{}{0pt}%
\pgfpathmoveto{\pgfqpoint{3.306485in}{2.382499in}}%
\pgfpathlineto{\pgfqpoint{3.763398in}{2.297457in}}%
\pgfpathlineto{\pgfqpoint{4.018447in}{2.468167in}}%
\pgfpathlineto{\pgfqpoint{3.564492in}{2.541257in}}%
\pgfpathlineto{\pgfqpoint{3.306485in}{2.382499in}}%
\pgfpathclose%
\pgfusepath{fill}%
\end{pgfscope}%
\begin{pgfscope}%
\pgfpathrectangle{\pgfqpoint{0.017500in}{0.017500in}}{\pgfqpoint{6.218282in}{3.580372in}} %
\pgfusepath{clip}%
\pgfsetbuttcap%
\pgfsetroundjoin%
\definecolor{currentfill}{rgb}{0.800830,0.250829,0.225696}%
\pgfsetfillcolor{currentfill}%
\pgfsetlinewidth{0.000000pt}%
\definecolor{currentstroke}{rgb}{0.000000,0.000000,0.000000}%
\pgfsetstrokecolor{currentstroke}%
\pgfsetdash{}{0pt}%
\pgfpathmoveto{\pgfqpoint{3.306485in}{2.382499in}}%
\pgfpathlineto{\pgfqpoint{3.763398in}{2.297457in}}%
\pgfpathlineto{\pgfqpoint{4.018447in}{2.468167in}}%
\pgfpathlineto{\pgfqpoint{3.564492in}{2.541257in}}%
\pgfpathlineto{\pgfqpoint{3.306485in}{2.382499in}}%
\pgfpathclose%
\pgfusepath{fill}%
\end{pgfscope}%
\begin{pgfscope}%
\pgfpathrectangle{\pgfqpoint{0.017500in}{0.017500in}}{\pgfqpoint{6.218282in}{3.580372in}} %
\pgfusepath{clip}%
\pgfsetbuttcap%
\pgfsetroundjoin%
\definecolor{currentfill}{rgb}{0.899534,0.440692,0.344107}%
\pgfsetfillcolor{currentfill}%
\pgfsetlinewidth{0.000000pt}%
\definecolor{currentstroke}{rgb}{0.000000,0.000000,0.000000}%
\pgfsetstrokecolor{currentstroke}%
\pgfsetdash{}{0pt}%
\pgfpathmoveto{\pgfqpoint{2.590432in}{2.210709in}}%
\pgfpathlineto{\pgfqpoint{2.855513in}{2.451418in}}%
\pgfpathlineto{\pgfqpoint{2.590432in}{2.210709in}}%
\pgfpathclose%
\pgfusepath{fill}%
\end{pgfscope}%
\begin{pgfscope}%
\pgfpathrectangle{\pgfqpoint{0.017500in}{0.017500in}}{\pgfqpoint{6.218282in}{3.580372in}} %
\pgfusepath{clip}%
\pgfsetbuttcap%
\pgfsetroundjoin%
\definecolor{currentfill}{rgb}{0.899534,0.440692,0.344107}%
\pgfsetfillcolor{currentfill}%
\pgfsetlinewidth{0.000000pt}%
\definecolor{currentstroke}{rgb}{0.000000,0.000000,0.000000}%
\pgfsetstrokecolor{currentstroke}%
\pgfsetdash{}{0pt}%
\pgfpathmoveto{\pgfqpoint{2.590432in}{2.210709in}}%
\pgfpathlineto{\pgfqpoint{2.855513in}{2.451418in}}%
\pgfpathlineto{\pgfqpoint{2.590432in}{2.210709in}}%
\pgfpathclose%
\pgfusepath{fill}%
\end{pgfscope}%
\begin{pgfscope}%
\pgfpathrectangle{\pgfqpoint{0.017500in}{0.017500in}}{\pgfqpoint{6.218282in}{3.580372in}} %
\pgfusepath{clip}%
\pgfsetbuttcap%
\pgfsetroundjoin%
\definecolor{currentfill}{rgb}{0.899534,0.440692,0.344107}%
\pgfsetfillcolor{currentfill}%
\pgfsetlinewidth{0.000000pt}%
\definecolor{currentstroke}{rgb}{0.000000,0.000000,0.000000}%
\pgfsetstrokecolor{currentstroke}%
\pgfsetdash{}{0pt}%
\pgfpathmoveto{\pgfqpoint{2.590432in}{2.210709in}}%
\pgfpathlineto{\pgfqpoint{2.855513in}{2.451418in}}%
\pgfpathlineto{\pgfqpoint{2.590432in}{2.210709in}}%
\pgfpathclose%
\pgfusepath{fill}%
\end{pgfscope}%
\begin{pgfscope}%
\pgfpathrectangle{\pgfqpoint{0.017500in}{0.017500in}}{\pgfqpoint{6.218282in}{3.580372in}} %
\pgfusepath{clip}%
\pgfsetbuttcap%
\pgfsetroundjoin%
\definecolor{currentfill}{rgb}{0.899534,0.440692,0.344107}%
\pgfsetfillcolor{currentfill}%
\pgfsetlinewidth{0.000000pt}%
\definecolor{currentstroke}{rgb}{0.000000,0.000000,0.000000}%
\pgfsetstrokecolor{currentstroke}%
\pgfsetdash{}{0pt}%
\pgfpathmoveto{\pgfqpoint{2.590432in}{2.210709in}}%
\pgfpathlineto{\pgfqpoint{2.855513in}{2.451418in}}%
\pgfpathlineto{\pgfqpoint{2.590432in}{2.210709in}}%
\pgfpathclose%
\pgfusepath{fill}%
\end{pgfscope}%
\begin{pgfscope}%
\pgfpathrectangle{\pgfqpoint{0.017500in}{0.017500in}}{\pgfqpoint{6.218282in}{3.580372in}} %
\pgfusepath{clip}%
\pgfsetbuttcap%
\pgfsetroundjoin%
\definecolor{currentfill}{rgb}{0.899534,0.440692,0.344107}%
\pgfsetfillcolor{currentfill}%
\pgfsetlinewidth{0.000000pt}%
\definecolor{currentstroke}{rgb}{0.000000,0.000000,0.000000}%
\pgfsetstrokecolor{currentstroke}%
\pgfsetdash{}{0pt}%
\pgfpathmoveto{\pgfqpoint{2.590432in}{2.210709in}}%
\pgfpathlineto{\pgfqpoint{2.855513in}{2.451418in}}%
\pgfpathlineto{\pgfqpoint{2.590432in}{2.210709in}}%
\pgfpathclose%
\pgfusepath{fill}%
\end{pgfscope}%
\begin{pgfscope}%
\pgfpathrectangle{\pgfqpoint{0.017500in}{0.017500in}}{\pgfqpoint{6.218282in}{3.580372in}} %
\pgfusepath{clip}%
\pgfsetbuttcap%
\pgfsetroundjoin%
\definecolor{currentfill}{rgb}{0.899534,0.440692,0.344107}%
\pgfsetfillcolor{currentfill}%
\pgfsetlinewidth{0.000000pt}%
\definecolor{currentstroke}{rgb}{0.000000,0.000000,0.000000}%
\pgfsetstrokecolor{currentstroke}%
\pgfsetdash{}{0pt}%
\pgfpathmoveto{\pgfqpoint{2.590432in}{2.210709in}}%
\pgfpathlineto{\pgfqpoint{2.855513in}{2.451418in}}%
\pgfpathlineto{\pgfqpoint{2.590432in}{2.210709in}}%
\pgfpathclose%
\pgfusepath{fill}%
\end{pgfscope}%
\begin{pgfscope}%
\pgfpathrectangle{\pgfqpoint{0.017500in}{0.017500in}}{\pgfqpoint{6.218282in}{3.580372in}} %
\pgfusepath{clip}%
\pgfsetbuttcap%
\pgfsetroundjoin%
\definecolor{currentfill}{rgb}{0.899534,0.440692,0.344107}%
\pgfsetfillcolor{currentfill}%
\pgfsetlinewidth{0.000000pt}%
\definecolor{currentstroke}{rgb}{0.000000,0.000000,0.000000}%
\pgfsetstrokecolor{currentstroke}%
\pgfsetdash{}{0pt}%
\pgfpathmoveto{\pgfqpoint{2.590432in}{2.210709in}}%
\pgfpathlineto{\pgfqpoint{2.855513in}{2.451418in}}%
\pgfpathlineto{\pgfqpoint{2.590432in}{2.210709in}}%
\pgfpathclose%
\pgfusepath{fill}%
\end{pgfscope}%
\begin{pgfscope}%
\pgfpathrectangle{\pgfqpoint{0.017500in}{0.017500in}}{\pgfqpoint{6.218282in}{3.580372in}} %
\pgfusepath{clip}%
\pgfsetbuttcap%
\pgfsetroundjoin%
\definecolor{currentfill}{rgb}{0.899534,0.440692,0.344107}%
\pgfsetfillcolor{currentfill}%
\pgfsetlinewidth{0.000000pt}%
\definecolor{currentstroke}{rgb}{0.000000,0.000000,0.000000}%
\pgfsetstrokecolor{currentstroke}%
\pgfsetdash{}{0pt}%
\pgfpathmoveto{\pgfqpoint{2.590432in}{2.210709in}}%
\pgfpathlineto{\pgfqpoint{2.855513in}{2.451418in}}%
\pgfpathlineto{\pgfqpoint{2.590432in}{2.210709in}}%
\pgfpathclose%
\pgfusepath{fill}%
\end{pgfscope}%
\begin{pgfscope}%
\pgfpathrectangle{\pgfqpoint{0.017500in}{0.017500in}}{\pgfqpoint{6.218282in}{3.580372in}} %
\pgfusepath{clip}%
\pgfsetbuttcap%
\pgfsetroundjoin%
\definecolor{currentfill}{rgb}{0.899534,0.440692,0.344107}%
\pgfsetfillcolor{currentfill}%
\pgfsetlinewidth{0.000000pt}%
\definecolor{currentstroke}{rgb}{0.000000,0.000000,0.000000}%
\pgfsetstrokecolor{currentstroke}%
\pgfsetdash{}{0pt}%
\pgfpathmoveto{\pgfqpoint{2.590432in}{2.210709in}}%
\pgfpathlineto{\pgfqpoint{2.855513in}{2.451418in}}%
\pgfpathlineto{\pgfqpoint{2.590432in}{2.210709in}}%
\pgfpathclose%
\pgfusepath{fill}%
\end{pgfscope}%
\begin{pgfscope}%
\pgfpathrectangle{\pgfqpoint{0.017500in}{0.017500in}}{\pgfqpoint{6.218282in}{3.580372in}} %
\pgfusepath{clip}%
\pgfsetbuttcap%
\pgfsetroundjoin%
\definecolor{currentfill}{rgb}{0.899534,0.440692,0.344107}%
\pgfsetfillcolor{currentfill}%
\pgfsetlinewidth{0.000000pt}%
\definecolor{currentstroke}{rgb}{0.000000,0.000000,0.000000}%
\pgfsetstrokecolor{currentstroke}%
\pgfsetdash{}{0pt}%
\pgfpathmoveto{\pgfqpoint{2.590432in}{2.210709in}}%
\pgfpathlineto{\pgfqpoint{2.855513in}{2.451418in}}%
\pgfpathlineto{\pgfqpoint{2.590432in}{2.210709in}}%
\pgfpathclose%
\pgfusepath{fill}%
\end{pgfscope}%
\begin{pgfscope}%
\pgfpathrectangle{\pgfqpoint{0.017500in}{0.017500in}}{\pgfqpoint{6.218282in}{3.580372in}} %
\pgfusepath{clip}%
\pgfsetbuttcap%
\pgfsetroundjoin%
\definecolor{currentfill}{rgb}{0.899534,0.440692,0.344107}%
\pgfsetfillcolor{currentfill}%
\pgfsetlinewidth{0.000000pt}%
\definecolor{currentstroke}{rgb}{0.000000,0.000000,0.000000}%
\pgfsetstrokecolor{currentstroke}%
\pgfsetdash{}{0pt}%
\pgfpathmoveto{\pgfqpoint{2.590432in}{2.210709in}}%
\pgfpathlineto{\pgfqpoint{2.855513in}{2.451418in}}%
\pgfpathlineto{\pgfqpoint{2.590432in}{2.210709in}}%
\pgfpathclose%
\pgfusepath{fill}%
\end{pgfscope}%
\begin{pgfscope}%
\pgfpathrectangle{\pgfqpoint{0.017500in}{0.017500in}}{\pgfqpoint{6.218282in}{3.580372in}} %
\pgfusepath{clip}%
\pgfsetbuttcap%
\pgfsetroundjoin%
\definecolor{currentfill}{rgb}{0.899534,0.440692,0.344107}%
\pgfsetfillcolor{currentfill}%
\pgfsetlinewidth{0.000000pt}%
\definecolor{currentstroke}{rgb}{0.000000,0.000000,0.000000}%
\pgfsetstrokecolor{currentstroke}%
\pgfsetdash{}{0pt}%
\pgfpathmoveto{\pgfqpoint{2.590432in}{2.210709in}}%
\pgfpathlineto{\pgfqpoint{2.855513in}{2.451418in}}%
\pgfpathlineto{\pgfqpoint{2.590432in}{2.210709in}}%
\pgfpathclose%
\pgfusepath{fill}%
\end{pgfscope}%
\begin{pgfscope}%
\pgfpathrectangle{\pgfqpoint{0.017500in}{0.017500in}}{\pgfqpoint{6.218282in}{3.580372in}} %
\pgfusepath{clip}%
\pgfsetbuttcap%
\pgfsetroundjoin%
\definecolor{currentfill}{rgb}{0.899534,0.440692,0.344107}%
\pgfsetfillcolor{currentfill}%
\pgfsetlinewidth{0.000000pt}%
\definecolor{currentstroke}{rgb}{0.000000,0.000000,0.000000}%
\pgfsetstrokecolor{currentstroke}%
\pgfsetdash{}{0pt}%
\pgfpathmoveto{\pgfqpoint{2.590432in}{2.210709in}}%
\pgfpathlineto{\pgfqpoint{2.855513in}{2.451418in}}%
\pgfpathlineto{\pgfqpoint{2.590432in}{2.210709in}}%
\pgfpathclose%
\pgfusepath{fill}%
\end{pgfscope}%
\begin{pgfscope}%
\pgfpathrectangle{\pgfqpoint{0.017500in}{0.017500in}}{\pgfqpoint{6.218282in}{3.580372in}} %
\pgfusepath{clip}%
\pgfsetbuttcap%
\pgfsetroundjoin%
\definecolor{currentfill}{rgb}{0.899534,0.440692,0.344107}%
\pgfsetfillcolor{currentfill}%
\pgfsetlinewidth{0.000000pt}%
\definecolor{currentstroke}{rgb}{0.000000,0.000000,0.000000}%
\pgfsetstrokecolor{currentstroke}%
\pgfsetdash{}{0pt}%
\pgfpathmoveto{\pgfqpoint{2.590432in}{2.210709in}}%
\pgfpathlineto{\pgfqpoint{2.855513in}{2.451418in}}%
\pgfpathlineto{\pgfqpoint{2.590432in}{2.210709in}}%
\pgfpathclose%
\pgfusepath{fill}%
\end{pgfscope}%
\begin{pgfscope}%
\pgfpathrectangle{\pgfqpoint{0.017500in}{0.017500in}}{\pgfqpoint{6.218282in}{3.580372in}} %
\pgfusepath{clip}%
\pgfsetbuttcap%
\pgfsetroundjoin%
\definecolor{currentfill}{rgb}{0.899534,0.440692,0.344107}%
\pgfsetfillcolor{currentfill}%
\pgfsetlinewidth{0.000000pt}%
\definecolor{currentstroke}{rgb}{0.000000,0.000000,0.000000}%
\pgfsetstrokecolor{currentstroke}%
\pgfsetdash{}{0pt}%
\pgfpathmoveto{\pgfqpoint{2.590432in}{2.210709in}}%
\pgfpathlineto{\pgfqpoint{2.855513in}{2.451418in}}%
\pgfpathlineto{\pgfqpoint{2.590432in}{2.210709in}}%
\pgfpathclose%
\pgfusepath{fill}%
\end{pgfscope}%
\begin{pgfscope}%
\pgfpathrectangle{\pgfqpoint{0.017500in}{0.017500in}}{\pgfqpoint{6.218282in}{3.580372in}} %
\pgfusepath{clip}%
\pgfsetbuttcap%
\pgfsetroundjoin%
\definecolor{currentfill}{rgb}{0.899534,0.440692,0.344107}%
\pgfsetfillcolor{currentfill}%
\pgfsetlinewidth{0.000000pt}%
\definecolor{currentstroke}{rgb}{0.000000,0.000000,0.000000}%
\pgfsetstrokecolor{currentstroke}%
\pgfsetdash{}{0pt}%
\pgfpathmoveto{\pgfqpoint{2.590432in}{2.210709in}}%
\pgfpathlineto{\pgfqpoint{2.855513in}{2.451418in}}%
\pgfpathlineto{\pgfqpoint{2.590432in}{2.210709in}}%
\pgfpathclose%
\pgfusepath{fill}%
\end{pgfscope}%
\begin{pgfscope}%
\pgfpathrectangle{\pgfqpoint{0.017500in}{0.017500in}}{\pgfqpoint{6.218282in}{3.580372in}} %
\pgfusepath{clip}%
\pgfsetbuttcap%
\pgfsetroundjoin%
\definecolor{currentfill}{rgb}{0.899534,0.440692,0.344107}%
\pgfsetfillcolor{currentfill}%
\pgfsetlinewidth{0.000000pt}%
\definecolor{currentstroke}{rgb}{0.000000,0.000000,0.000000}%
\pgfsetstrokecolor{currentstroke}%
\pgfsetdash{}{0pt}%
\pgfpathmoveto{\pgfqpoint{2.590432in}{2.210709in}}%
\pgfpathlineto{\pgfqpoint{2.855513in}{2.451418in}}%
\pgfpathlineto{\pgfqpoint{2.590432in}{2.210709in}}%
\pgfpathclose%
\pgfusepath{fill}%
\end{pgfscope}%
\begin{pgfscope}%
\pgfpathrectangle{\pgfqpoint{0.017500in}{0.017500in}}{\pgfqpoint{6.218282in}{3.580372in}} %
\pgfusepath{clip}%
\pgfsetbuttcap%
\pgfsetroundjoin%
\definecolor{currentfill}{rgb}{0.899534,0.440692,0.344107}%
\pgfsetfillcolor{currentfill}%
\pgfsetlinewidth{0.000000pt}%
\definecolor{currentstroke}{rgb}{0.000000,0.000000,0.000000}%
\pgfsetstrokecolor{currentstroke}%
\pgfsetdash{}{0pt}%
\pgfpathmoveto{\pgfqpoint{2.590432in}{2.210709in}}%
\pgfpathlineto{\pgfqpoint{2.855513in}{2.451418in}}%
\pgfpathlineto{\pgfqpoint{2.590432in}{2.210709in}}%
\pgfpathclose%
\pgfusepath{fill}%
\end{pgfscope}%
\begin{pgfscope}%
\pgfpathrectangle{\pgfqpoint{0.017500in}{0.017500in}}{\pgfqpoint{6.218282in}{3.580372in}} %
\pgfusepath{clip}%
\pgfsetbuttcap%
\pgfsetroundjoin%
\definecolor{currentfill}{rgb}{0.899534,0.440692,0.344107}%
\pgfsetfillcolor{currentfill}%
\pgfsetlinewidth{0.000000pt}%
\definecolor{currentstroke}{rgb}{0.000000,0.000000,0.000000}%
\pgfsetstrokecolor{currentstroke}%
\pgfsetdash{}{0pt}%
\pgfpathmoveto{\pgfqpoint{2.590432in}{2.210709in}}%
\pgfpathlineto{\pgfqpoint{2.855513in}{2.451418in}}%
\pgfpathlineto{\pgfqpoint{2.590432in}{2.210709in}}%
\pgfpathclose%
\pgfusepath{fill}%
\end{pgfscope}%
\begin{pgfscope}%
\pgfpathrectangle{\pgfqpoint{0.017500in}{0.017500in}}{\pgfqpoint{6.218282in}{3.580372in}} %
\pgfusepath{clip}%
\pgfsetbuttcap%
\pgfsetroundjoin%
\definecolor{currentfill}{rgb}{0.899534,0.440692,0.344107}%
\pgfsetfillcolor{currentfill}%
\pgfsetlinewidth{0.000000pt}%
\definecolor{currentstroke}{rgb}{0.000000,0.000000,0.000000}%
\pgfsetstrokecolor{currentstroke}%
\pgfsetdash{}{0pt}%
\pgfpathmoveto{\pgfqpoint{2.590432in}{2.210709in}}%
\pgfpathlineto{\pgfqpoint{2.855513in}{2.451418in}}%
\pgfpathlineto{\pgfqpoint{2.590432in}{2.210709in}}%
\pgfpathclose%
\pgfusepath{fill}%
\end{pgfscope}%
\begin{pgfscope}%
\pgfpathrectangle{\pgfqpoint{0.017500in}{0.017500in}}{\pgfqpoint{6.218282in}{3.580372in}} %
\pgfusepath{clip}%
\pgfsetbuttcap%
\pgfsetroundjoin%
\definecolor{currentfill}{rgb}{0.899534,0.440692,0.344107}%
\pgfsetfillcolor{currentfill}%
\pgfsetlinewidth{0.000000pt}%
\definecolor{currentstroke}{rgb}{0.000000,0.000000,0.000000}%
\pgfsetstrokecolor{currentstroke}%
\pgfsetdash{}{0pt}%
\pgfpathmoveto{\pgfqpoint{2.590432in}{2.210709in}}%
\pgfpathlineto{\pgfqpoint{2.855513in}{2.451418in}}%
\pgfpathlineto{\pgfqpoint{2.590432in}{2.210709in}}%
\pgfpathclose%
\pgfusepath{fill}%
\end{pgfscope}%
\begin{pgfscope}%
\pgfpathrectangle{\pgfqpoint{0.017500in}{0.017500in}}{\pgfqpoint{6.218282in}{3.580372in}} %
\pgfusepath{clip}%
\pgfsetbuttcap%
\pgfsetroundjoin%
\definecolor{currentfill}{rgb}{0.899534,0.440692,0.344107}%
\pgfsetfillcolor{currentfill}%
\pgfsetlinewidth{0.000000pt}%
\definecolor{currentstroke}{rgb}{0.000000,0.000000,0.000000}%
\pgfsetstrokecolor{currentstroke}%
\pgfsetdash{}{0pt}%
\pgfpathmoveto{\pgfqpoint{2.590432in}{2.210709in}}%
\pgfpathlineto{\pgfqpoint{2.855513in}{2.451418in}}%
\pgfpathlineto{\pgfqpoint{2.590432in}{2.210709in}}%
\pgfpathclose%
\pgfusepath{fill}%
\end{pgfscope}%
\begin{pgfscope}%
\pgfpathrectangle{\pgfqpoint{0.017500in}{0.017500in}}{\pgfqpoint{6.218282in}{3.580372in}} %
\pgfusepath{clip}%
\pgfsetbuttcap%
\pgfsetroundjoin%
\definecolor{currentfill}{rgb}{0.899534,0.440692,0.344107}%
\pgfsetfillcolor{currentfill}%
\pgfsetlinewidth{0.000000pt}%
\definecolor{currentstroke}{rgb}{0.000000,0.000000,0.000000}%
\pgfsetstrokecolor{currentstroke}%
\pgfsetdash{}{0pt}%
\pgfpathmoveto{\pgfqpoint{2.590432in}{2.210709in}}%
\pgfpathlineto{\pgfqpoint{2.855513in}{2.451418in}}%
\pgfpathlineto{\pgfqpoint{2.590432in}{2.210709in}}%
\pgfpathclose%
\pgfusepath{fill}%
\end{pgfscope}%
\begin{pgfscope}%
\pgfpathrectangle{\pgfqpoint{0.017500in}{0.017500in}}{\pgfqpoint{6.218282in}{3.580372in}} %
\pgfusepath{clip}%
\pgfsetbuttcap%
\pgfsetroundjoin%
\definecolor{currentfill}{rgb}{0.899534,0.440692,0.344107}%
\pgfsetfillcolor{currentfill}%
\pgfsetlinewidth{0.000000pt}%
\definecolor{currentstroke}{rgb}{0.000000,0.000000,0.000000}%
\pgfsetstrokecolor{currentstroke}%
\pgfsetdash{}{0pt}%
\pgfpathmoveto{\pgfqpoint{2.590432in}{2.210709in}}%
\pgfpathlineto{\pgfqpoint{2.855513in}{2.451418in}}%
\pgfpathlineto{\pgfqpoint{2.590432in}{2.210709in}}%
\pgfpathclose%
\pgfusepath{fill}%
\end{pgfscope}%
\begin{pgfscope}%
\pgfpathrectangle{\pgfqpoint{0.017500in}{0.017500in}}{\pgfqpoint{6.218282in}{3.580372in}} %
\pgfusepath{clip}%
\pgfsetbuttcap%
\pgfsetroundjoin%
\definecolor{currentfill}{rgb}{0.899534,0.440692,0.344107}%
\pgfsetfillcolor{currentfill}%
\pgfsetlinewidth{0.000000pt}%
\definecolor{currentstroke}{rgb}{0.000000,0.000000,0.000000}%
\pgfsetstrokecolor{currentstroke}%
\pgfsetdash{}{0pt}%
\pgfpathmoveto{\pgfqpoint{2.590432in}{2.210709in}}%
\pgfpathlineto{\pgfqpoint{2.855513in}{2.451418in}}%
\pgfpathlineto{\pgfqpoint{2.590432in}{2.210709in}}%
\pgfpathclose%
\pgfusepath{fill}%
\end{pgfscope}%
\begin{pgfscope}%
\pgfpathrectangle{\pgfqpoint{0.017500in}{0.017500in}}{\pgfqpoint{6.218282in}{3.580372in}} %
\pgfusepath{clip}%
\pgfsetbuttcap%
\pgfsetroundjoin%
\definecolor{currentfill}{rgb}{0.899534,0.440692,0.344107}%
\pgfsetfillcolor{currentfill}%
\pgfsetlinewidth{0.000000pt}%
\definecolor{currentstroke}{rgb}{0.000000,0.000000,0.000000}%
\pgfsetstrokecolor{currentstroke}%
\pgfsetdash{}{0pt}%
\pgfpathmoveto{\pgfqpoint{2.590432in}{2.210709in}}%
\pgfpathlineto{\pgfqpoint{2.855513in}{2.451418in}}%
\pgfpathlineto{\pgfqpoint{2.590432in}{2.210709in}}%
\pgfpathclose%
\pgfusepath{fill}%
\end{pgfscope}%
\begin{pgfscope}%
\pgfpathrectangle{\pgfqpoint{0.017500in}{0.017500in}}{\pgfqpoint{6.218282in}{3.580372in}} %
\pgfusepath{clip}%
\pgfsetbuttcap%
\pgfsetroundjoin%
\definecolor{currentfill}{rgb}{0.899534,0.440692,0.344107}%
\pgfsetfillcolor{currentfill}%
\pgfsetlinewidth{0.000000pt}%
\definecolor{currentstroke}{rgb}{0.000000,0.000000,0.000000}%
\pgfsetstrokecolor{currentstroke}%
\pgfsetdash{}{0pt}%
\pgfpathmoveto{\pgfqpoint{2.590432in}{2.210709in}}%
\pgfpathlineto{\pgfqpoint{2.855513in}{2.451418in}}%
\pgfpathlineto{\pgfqpoint{2.590432in}{2.210709in}}%
\pgfpathclose%
\pgfusepath{fill}%
\end{pgfscope}%
\begin{pgfscope}%
\pgfpathrectangle{\pgfqpoint{0.017500in}{0.017500in}}{\pgfqpoint{6.218282in}{3.580372in}} %
\pgfusepath{clip}%
\pgfsetbuttcap%
\pgfsetroundjoin%
\definecolor{currentfill}{rgb}{0.899534,0.440692,0.344107}%
\pgfsetfillcolor{currentfill}%
\pgfsetlinewidth{0.000000pt}%
\definecolor{currentstroke}{rgb}{0.000000,0.000000,0.000000}%
\pgfsetstrokecolor{currentstroke}%
\pgfsetdash{}{0pt}%
\pgfpathmoveto{\pgfqpoint{2.590432in}{2.210709in}}%
\pgfpathlineto{\pgfqpoint{2.855513in}{2.451418in}}%
\pgfpathlineto{\pgfqpoint{2.590432in}{2.210709in}}%
\pgfpathclose%
\pgfusepath{fill}%
\end{pgfscope}%
\begin{pgfscope}%
\pgfpathrectangle{\pgfqpoint{0.017500in}{0.017500in}}{\pgfqpoint{6.218282in}{3.580372in}} %
\pgfusepath{clip}%
\pgfsetbuttcap%
\pgfsetroundjoin%
\definecolor{currentfill}{rgb}{0.899534,0.440692,0.344107}%
\pgfsetfillcolor{currentfill}%
\pgfsetlinewidth{0.000000pt}%
\definecolor{currentstroke}{rgb}{0.000000,0.000000,0.000000}%
\pgfsetstrokecolor{currentstroke}%
\pgfsetdash{}{0pt}%
\pgfpathmoveto{\pgfqpoint{2.590432in}{2.210709in}}%
\pgfpathlineto{\pgfqpoint{2.855513in}{2.451418in}}%
\pgfpathlineto{\pgfqpoint{2.590432in}{2.210709in}}%
\pgfpathclose%
\pgfusepath{fill}%
\end{pgfscope}%
\begin{pgfscope}%
\pgfpathrectangle{\pgfqpoint{0.017500in}{0.017500in}}{\pgfqpoint{6.218282in}{3.580372in}} %
\pgfusepath{clip}%
\pgfsetbuttcap%
\pgfsetroundjoin%
\definecolor{currentfill}{rgb}{0.899534,0.440692,0.344107}%
\pgfsetfillcolor{currentfill}%
\pgfsetlinewidth{0.000000pt}%
\definecolor{currentstroke}{rgb}{0.000000,0.000000,0.000000}%
\pgfsetstrokecolor{currentstroke}%
\pgfsetdash{}{0pt}%
\pgfpathmoveto{\pgfqpoint{2.590432in}{2.210709in}}%
\pgfpathlineto{\pgfqpoint{2.855513in}{2.451418in}}%
\pgfpathlineto{\pgfqpoint{2.590432in}{2.210709in}}%
\pgfpathclose%
\pgfusepath{fill}%
\end{pgfscope}%
\begin{pgfscope}%
\pgfpathrectangle{\pgfqpoint{0.017500in}{0.017500in}}{\pgfqpoint{6.218282in}{3.580372in}} %
\pgfusepath{clip}%
\pgfsetbuttcap%
\pgfsetroundjoin%
\definecolor{currentfill}{rgb}{0.243520,0.319189,0.771672}%
\pgfsetfillcolor{currentfill}%
\pgfsetlinewidth{0.000000pt}%
\definecolor{currentstroke}{rgb}{0.000000,0.000000,0.000000}%
\pgfsetstrokecolor{currentstroke}%
\pgfsetdash{}{0pt}%
\pgfpathmoveto{\pgfqpoint{3.236099in}{0.723497in}}%
\pgfpathlineto{\pgfqpoint{3.502561in}{2.073396in}}%
\pgfpathlineto{\pgfqpoint{3.236099in}{0.723497in}}%
\pgfpathclose%
\pgfusepath{fill}%
\end{pgfscope}%
\begin{pgfscope}%
\pgfpathrectangle{\pgfqpoint{0.017500in}{0.017500in}}{\pgfqpoint{6.218282in}{3.580372in}} %
\pgfusepath{clip}%
\pgfsetbuttcap%
\pgfsetroundjoin%
\definecolor{currentfill}{rgb}{0.243520,0.319189,0.771672}%
\pgfsetfillcolor{currentfill}%
\pgfsetlinewidth{0.000000pt}%
\definecolor{currentstroke}{rgb}{0.000000,0.000000,0.000000}%
\pgfsetstrokecolor{currentstroke}%
\pgfsetdash{}{0pt}%
\pgfpathmoveto{\pgfqpoint{3.236099in}{0.723497in}}%
\pgfpathlineto{\pgfqpoint{3.502561in}{2.073396in}}%
\pgfpathlineto{\pgfqpoint{3.236099in}{0.723497in}}%
\pgfpathclose%
\pgfusepath{fill}%
\end{pgfscope}%
\begin{pgfscope}%
\pgfpathrectangle{\pgfqpoint{0.017500in}{0.017500in}}{\pgfqpoint{6.218282in}{3.580372in}} %
\pgfusepath{clip}%
\pgfsetbuttcap%
\pgfsetroundjoin%
\definecolor{currentfill}{rgb}{0.243520,0.319189,0.771672}%
\pgfsetfillcolor{currentfill}%
\pgfsetlinewidth{0.000000pt}%
\definecolor{currentstroke}{rgb}{0.000000,0.000000,0.000000}%
\pgfsetstrokecolor{currentstroke}%
\pgfsetdash{}{0pt}%
\pgfpathmoveto{\pgfqpoint{3.236099in}{0.723497in}}%
\pgfpathlineto{\pgfqpoint{3.502561in}{2.073396in}}%
\pgfpathlineto{\pgfqpoint{3.236099in}{0.723497in}}%
\pgfpathclose%
\pgfusepath{fill}%
\end{pgfscope}%
\begin{pgfscope}%
\pgfpathrectangle{\pgfqpoint{0.017500in}{0.017500in}}{\pgfqpoint{6.218282in}{3.580372in}} %
\pgfusepath{clip}%
\pgfsetbuttcap%
\pgfsetroundjoin%
\definecolor{currentfill}{rgb}{0.243520,0.319189,0.771672}%
\pgfsetfillcolor{currentfill}%
\pgfsetlinewidth{0.000000pt}%
\definecolor{currentstroke}{rgb}{0.000000,0.000000,0.000000}%
\pgfsetstrokecolor{currentstroke}%
\pgfsetdash{}{0pt}%
\pgfpathmoveto{\pgfqpoint{3.236099in}{0.723497in}}%
\pgfpathlineto{\pgfqpoint{3.502561in}{2.073396in}}%
\pgfpathlineto{\pgfqpoint{3.236099in}{0.723497in}}%
\pgfpathclose%
\pgfusepath{fill}%
\end{pgfscope}%
\begin{pgfscope}%
\pgfpathrectangle{\pgfqpoint{0.017500in}{0.017500in}}{\pgfqpoint{6.218282in}{3.580372in}} %
\pgfusepath{clip}%
\pgfsetbuttcap%
\pgfsetroundjoin%
\definecolor{currentfill}{rgb}{0.243520,0.319189,0.771672}%
\pgfsetfillcolor{currentfill}%
\pgfsetlinewidth{0.000000pt}%
\definecolor{currentstroke}{rgb}{0.000000,0.000000,0.000000}%
\pgfsetstrokecolor{currentstroke}%
\pgfsetdash{}{0pt}%
\pgfpathmoveto{\pgfqpoint{3.236099in}{0.723497in}}%
\pgfpathlineto{\pgfqpoint{3.502561in}{2.073396in}}%
\pgfpathlineto{\pgfqpoint{3.236099in}{0.723497in}}%
\pgfpathclose%
\pgfusepath{fill}%
\end{pgfscope}%
\begin{pgfscope}%
\pgfpathrectangle{\pgfqpoint{0.017500in}{0.017500in}}{\pgfqpoint{6.218282in}{3.580372in}} %
\pgfusepath{clip}%
\pgfsetbuttcap%
\pgfsetroundjoin%
\definecolor{currentfill}{rgb}{0.243520,0.319189,0.771672}%
\pgfsetfillcolor{currentfill}%
\pgfsetlinewidth{0.000000pt}%
\definecolor{currentstroke}{rgb}{0.000000,0.000000,0.000000}%
\pgfsetstrokecolor{currentstroke}%
\pgfsetdash{}{0pt}%
\pgfpathmoveto{\pgfqpoint{3.236099in}{0.723497in}}%
\pgfpathlineto{\pgfqpoint{3.502561in}{2.073396in}}%
\pgfpathlineto{\pgfqpoint{3.236099in}{0.723497in}}%
\pgfpathclose%
\pgfusepath{fill}%
\end{pgfscope}%
\begin{pgfscope}%
\pgfpathrectangle{\pgfqpoint{0.017500in}{0.017500in}}{\pgfqpoint{6.218282in}{3.580372in}} %
\pgfusepath{clip}%
\pgfsetbuttcap%
\pgfsetroundjoin%
\definecolor{currentfill}{rgb}{0.243520,0.319189,0.771672}%
\pgfsetfillcolor{currentfill}%
\pgfsetlinewidth{0.000000pt}%
\definecolor{currentstroke}{rgb}{0.000000,0.000000,0.000000}%
\pgfsetstrokecolor{currentstroke}%
\pgfsetdash{}{0pt}%
\pgfpathmoveto{\pgfqpoint{3.236099in}{0.723497in}}%
\pgfpathlineto{\pgfqpoint{3.502561in}{2.073396in}}%
\pgfpathlineto{\pgfqpoint{3.236099in}{0.723497in}}%
\pgfpathclose%
\pgfusepath{fill}%
\end{pgfscope}%
\begin{pgfscope}%
\pgfpathrectangle{\pgfqpoint{0.017500in}{0.017500in}}{\pgfqpoint{6.218282in}{3.580372in}} %
\pgfusepath{clip}%
\pgfsetbuttcap%
\pgfsetroundjoin%
\definecolor{currentfill}{rgb}{0.243520,0.319189,0.771672}%
\pgfsetfillcolor{currentfill}%
\pgfsetlinewidth{0.000000pt}%
\definecolor{currentstroke}{rgb}{0.000000,0.000000,0.000000}%
\pgfsetstrokecolor{currentstroke}%
\pgfsetdash{}{0pt}%
\pgfpathmoveto{\pgfqpoint{3.236099in}{0.723497in}}%
\pgfpathlineto{\pgfqpoint{3.502561in}{2.073396in}}%
\pgfpathlineto{\pgfqpoint{3.236099in}{0.723497in}}%
\pgfpathclose%
\pgfusepath{fill}%
\end{pgfscope}%
\begin{pgfscope}%
\pgfpathrectangle{\pgfqpoint{0.017500in}{0.017500in}}{\pgfqpoint{6.218282in}{3.580372in}} %
\pgfusepath{clip}%
\pgfsetbuttcap%
\pgfsetroundjoin%
\definecolor{currentfill}{rgb}{0.243520,0.319189,0.771672}%
\pgfsetfillcolor{currentfill}%
\pgfsetlinewidth{0.000000pt}%
\definecolor{currentstroke}{rgb}{0.000000,0.000000,0.000000}%
\pgfsetstrokecolor{currentstroke}%
\pgfsetdash{}{0pt}%
\pgfpathmoveto{\pgfqpoint{3.236099in}{0.723497in}}%
\pgfpathlineto{\pgfqpoint{3.502561in}{2.073396in}}%
\pgfpathlineto{\pgfqpoint{3.236099in}{0.723497in}}%
\pgfpathclose%
\pgfusepath{fill}%
\end{pgfscope}%
\begin{pgfscope}%
\pgfpathrectangle{\pgfqpoint{0.017500in}{0.017500in}}{\pgfqpoint{6.218282in}{3.580372in}} %
\pgfusepath{clip}%
\pgfsetbuttcap%
\pgfsetroundjoin%
\definecolor{currentfill}{rgb}{0.243520,0.319189,0.771672}%
\pgfsetfillcolor{currentfill}%
\pgfsetlinewidth{0.000000pt}%
\definecolor{currentstroke}{rgb}{0.000000,0.000000,0.000000}%
\pgfsetstrokecolor{currentstroke}%
\pgfsetdash{}{0pt}%
\pgfpathmoveto{\pgfqpoint{3.236099in}{0.723497in}}%
\pgfpathlineto{\pgfqpoint{3.502561in}{2.073396in}}%
\pgfpathlineto{\pgfqpoint{3.236099in}{0.723497in}}%
\pgfpathclose%
\pgfusepath{fill}%
\end{pgfscope}%
\begin{pgfscope}%
\pgfpathrectangle{\pgfqpoint{0.017500in}{0.017500in}}{\pgfqpoint{6.218282in}{3.580372in}} %
\pgfusepath{clip}%
\pgfsetbuttcap%
\pgfsetroundjoin%
\definecolor{currentfill}{rgb}{0.243520,0.319189,0.771672}%
\pgfsetfillcolor{currentfill}%
\pgfsetlinewidth{0.000000pt}%
\definecolor{currentstroke}{rgb}{0.000000,0.000000,0.000000}%
\pgfsetstrokecolor{currentstroke}%
\pgfsetdash{}{0pt}%
\pgfpathmoveto{\pgfqpoint{3.236099in}{0.723497in}}%
\pgfpathlineto{\pgfqpoint{3.502561in}{2.073396in}}%
\pgfpathlineto{\pgfqpoint{3.236099in}{0.723497in}}%
\pgfpathclose%
\pgfusepath{fill}%
\end{pgfscope}%
\begin{pgfscope}%
\pgfpathrectangle{\pgfqpoint{0.017500in}{0.017500in}}{\pgfqpoint{6.218282in}{3.580372in}} %
\pgfusepath{clip}%
\pgfsetbuttcap%
\pgfsetroundjoin%
\definecolor{currentfill}{rgb}{0.243520,0.319189,0.771672}%
\pgfsetfillcolor{currentfill}%
\pgfsetlinewidth{0.000000pt}%
\definecolor{currentstroke}{rgb}{0.000000,0.000000,0.000000}%
\pgfsetstrokecolor{currentstroke}%
\pgfsetdash{}{0pt}%
\pgfpathmoveto{\pgfqpoint{3.236099in}{0.723497in}}%
\pgfpathlineto{\pgfqpoint{3.502561in}{2.073396in}}%
\pgfpathlineto{\pgfqpoint{3.236099in}{0.723497in}}%
\pgfpathclose%
\pgfusepath{fill}%
\end{pgfscope}%
\begin{pgfscope}%
\pgfpathrectangle{\pgfqpoint{0.017500in}{0.017500in}}{\pgfqpoint{6.218282in}{3.580372in}} %
\pgfusepath{clip}%
\pgfsetbuttcap%
\pgfsetroundjoin%
\definecolor{currentfill}{rgb}{0.243520,0.319189,0.771672}%
\pgfsetfillcolor{currentfill}%
\pgfsetlinewidth{0.000000pt}%
\definecolor{currentstroke}{rgb}{0.000000,0.000000,0.000000}%
\pgfsetstrokecolor{currentstroke}%
\pgfsetdash{}{0pt}%
\pgfpathmoveto{\pgfqpoint{3.236099in}{0.723497in}}%
\pgfpathlineto{\pgfqpoint{3.502561in}{2.073396in}}%
\pgfpathlineto{\pgfqpoint{3.236099in}{0.723497in}}%
\pgfpathclose%
\pgfusepath{fill}%
\end{pgfscope}%
\begin{pgfscope}%
\pgfpathrectangle{\pgfqpoint{0.017500in}{0.017500in}}{\pgfqpoint{6.218282in}{3.580372in}} %
\pgfusepath{clip}%
\pgfsetbuttcap%
\pgfsetroundjoin%
\definecolor{currentfill}{rgb}{0.243520,0.319189,0.771672}%
\pgfsetfillcolor{currentfill}%
\pgfsetlinewidth{0.000000pt}%
\definecolor{currentstroke}{rgb}{0.000000,0.000000,0.000000}%
\pgfsetstrokecolor{currentstroke}%
\pgfsetdash{}{0pt}%
\pgfpathmoveto{\pgfqpoint{3.236099in}{0.723497in}}%
\pgfpathlineto{\pgfqpoint{3.502561in}{2.073396in}}%
\pgfpathlineto{\pgfqpoint{3.236099in}{0.723497in}}%
\pgfpathclose%
\pgfusepath{fill}%
\end{pgfscope}%
\begin{pgfscope}%
\pgfpathrectangle{\pgfqpoint{0.017500in}{0.017500in}}{\pgfqpoint{6.218282in}{3.580372in}} %
\pgfusepath{clip}%
\pgfsetbuttcap%
\pgfsetroundjoin%
\definecolor{currentfill}{rgb}{0.243520,0.319189,0.771672}%
\pgfsetfillcolor{currentfill}%
\pgfsetlinewidth{0.000000pt}%
\definecolor{currentstroke}{rgb}{0.000000,0.000000,0.000000}%
\pgfsetstrokecolor{currentstroke}%
\pgfsetdash{}{0pt}%
\pgfpathmoveto{\pgfqpoint{3.236099in}{0.723497in}}%
\pgfpathlineto{\pgfqpoint{3.502561in}{2.073396in}}%
\pgfpathlineto{\pgfqpoint{3.236099in}{0.723497in}}%
\pgfpathclose%
\pgfusepath{fill}%
\end{pgfscope}%
\begin{pgfscope}%
\pgfpathrectangle{\pgfqpoint{0.017500in}{0.017500in}}{\pgfqpoint{6.218282in}{3.580372in}} %
\pgfusepath{clip}%
\pgfsetbuttcap%
\pgfsetroundjoin%
\definecolor{currentfill}{rgb}{0.243520,0.319189,0.771672}%
\pgfsetfillcolor{currentfill}%
\pgfsetlinewidth{0.000000pt}%
\definecolor{currentstroke}{rgb}{0.000000,0.000000,0.000000}%
\pgfsetstrokecolor{currentstroke}%
\pgfsetdash{}{0pt}%
\pgfpathmoveto{\pgfqpoint{3.236099in}{0.723497in}}%
\pgfpathlineto{\pgfqpoint{3.502561in}{2.073396in}}%
\pgfpathlineto{\pgfqpoint{3.236099in}{0.723497in}}%
\pgfpathclose%
\pgfusepath{fill}%
\end{pgfscope}%
\begin{pgfscope}%
\pgfpathrectangle{\pgfqpoint{0.017500in}{0.017500in}}{\pgfqpoint{6.218282in}{3.580372in}} %
\pgfusepath{clip}%
\pgfsetbuttcap%
\pgfsetroundjoin%
\definecolor{currentfill}{rgb}{0.243520,0.319189,0.771672}%
\pgfsetfillcolor{currentfill}%
\pgfsetlinewidth{0.000000pt}%
\definecolor{currentstroke}{rgb}{0.000000,0.000000,0.000000}%
\pgfsetstrokecolor{currentstroke}%
\pgfsetdash{}{0pt}%
\pgfpathmoveto{\pgfqpoint{3.236099in}{0.723497in}}%
\pgfpathlineto{\pgfqpoint{3.502561in}{2.073396in}}%
\pgfpathlineto{\pgfqpoint{3.236099in}{0.723497in}}%
\pgfpathclose%
\pgfusepath{fill}%
\end{pgfscope}%
\begin{pgfscope}%
\pgfpathrectangle{\pgfqpoint{0.017500in}{0.017500in}}{\pgfqpoint{6.218282in}{3.580372in}} %
\pgfusepath{clip}%
\pgfsetbuttcap%
\pgfsetroundjoin%
\definecolor{currentfill}{rgb}{0.243520,0.319189,0.771672}%
\pgfsetfillcolor{currentfill}%
\pgfsetlinewidth{0.000000pt}%
\definecolor{currentstroke}{rgb}{0.000000,0.000000,0.000000}%
\pgfsetstrokecolor{currentstroke}%
\pgfsetdash{}{0pt}%
\pgfpathmoveto{\pgfqpoint{3.236099in}{0.723497in}}%
\pgfpathlineto{\pgfqpoint{3.502561in}{2.073396in}}%
\pgfpathlineto{\pgfqpoint{3.236099in}{0.723497in}}%
\pgfpathclose%
\pgfusepath{fill}%
\end{pgfscope}%
\begin{pgfscope}%
\pgfpathrectangle{\pgfqpoint{0.017500in}{0.017500in}}{\pgfqpoint{6.218282in}{3.580372in}} %
\pgfusepath{clip}%
\pgfsetbuttcap%
\pgfsetroundjoin%
\definecolor{currentfill}{rgb}{0.243520,0.319189,0.771672}%
\pgfsetfillcolor{currentfill}%
\pgfsetlinewidth{0.000000pt}%
\definecolor{currentstroke}{rgb}{0.000000,0.000000,0.000000}%
\pgfsetstrokecolor{currentstroke}%
\pgfsetdash{}{0pt}%
\pgfpathmoveto{\pgfqpoint{3.236099in}{0.723497in}}%
\pgfpathlineto{\pgfqpoint{3.502561in}{2.073396in}}%
\pgfpathlineto{\pgfqpoint{3.236099in}{0.723497in}}%
\pgfpathclose%
\pgfusepath{fill}%
\end{pgfscope}%
\begin{pgfscope}%
\pgfpathrectangle{\pgfqpoint{0.017500in}{0.017500in}}{\pgfqpoint{6.218282in}{3.580372in}} %
\pgfusepath{clip}%
\pgfsetbuttcap%
\pgfsetroundjoin%
\definecolor{currentfill}{rgb}{0.243520,0.319189,0.771672}%
\pgfsetfillcolor{currentfill}%
\pgfsetlinewidth{0.000000pt}%
\definecolor{currentstroke}{rgb}{0.000000,0.000000,0.000000}%
\pgfsetstrokecolor{currentstroke}%
\pgfsetdash{}{0pt}%
\pgfpathmoveto{\pgfqpoint{3.236099in}{0.723497in}}%
\pgfpathlineto{\pgfqpoint{3.502561in}{2.073396in}}%
\pgfpathlineto{\pgfqpoint{3.236099in}{0.723497in}}%
\pgfpathclose%
\pgfusepath{fill}%
\end{pgfscope}%
\begin{pgfscope}%
\pgfpathrectangle{\pgfqpoint{0.017500in}{0.017500in}}{\pgfqpoint{6.218282in}{3.580372in}} %
\pgfusepath{clip}%
\pgfsetbuttcap%
\pgfsetroundjoin%
\definecolor{currentfill}{rgb}{0.243520,0.319189,0.771672}%
\pgfsetfillcolor{currentfill}%
\pgfsetlinewidth{0.000000pt}%
\definecolor{currentstroke}{rgb}{0.000000,0.000000,0.000000}%
\pgfsetstrokecolor{currentstroke}%
\pgfsetdash{}{0pt}%
\pgfpathmoveto{\pgfqpoint{3.236099in}{0.723497in}}%
\pgfpathlineto{\pgfqpoint{3.502561in}{2.073396in}}%
\pgfpathlineto{\pgfqpoint{3.236099in}{0.723497in}}%
\pgfpathclose%
\pgfusepath{fill}%
\end{pgfscope}%
\begin{pgfscope}%
\pgfpathrectangle{\pgfqpoint{0.017500in}{0.017500in}}{\pgfqpoint{6.218282in}{3.580372in}} %
\pgfusepath{clip}%
\pgfsetbuttcap%
\pgfsetroundjoin%
\definecolor{currentfill}{rgb}{0.243520,0.319189,0.771672}%
\pgfsetfillcolor{currentfill}%
\pgfsetlinewidth{0.000000pt}%
\definecolor{currentstroke}{rgb}{0.000000,0.000000,0.000000}%
\pgfsetstrokecolor{currentstroke}%
\pgfsetdash{}{0pt}%
\pgfpathmoveto{\pgfqpoint{3.236099in}{0.723497in}}%
\pgfpathlineto{\pgfqpoint{3.502561in}{2.073396in}}%
\pgfpathlineto{\pgfqpoint{3.236099in}{0.723497in}}%
\pgfpathclose%
\pgfusepath{fill}%
\end{pgfscope}%
\begin{pgfscope}%
\pgfpathrectangle{\pgfqpoint{0.017500in}{0.017500in}}{\pgfqpoint{6.218282in}{3.580372in}} %
\pgfusepath{clip}%
\pgfsetbuttcap%
\pgfsetroundjoin%
\definecolor{currentfill}{rgb}{0.243520,0.319189,0.771672}%
\pgfsetfillcolor{currentfill}%
\pgfsetlinewidth{0.000000pt}%
\definecolor{currentstroke}{rgb}{0.000000,0.000000,0.000000}%
\pgfsetstrokecolor{currentstroke}%
\pgfsetdash{}{0pt}%
\pgfpathmoveto{\pgfqpoint{3.236099in}{0.723497in}}%
\pgfpathlineto{\pgfqpoint{3.502561in}{2.073396in}}%
\pgfpathlineto{\pgfqpoint{3.236099in}{0.723497in}}%
\pgfpathclose%
\pgfusepath{fill}%
\end{pgfscope}%
\begin{pgfscope}%
\pgfpathrectangle{\pgfqpoint{0.017500in}{0.017500in}}{\pgfqpoint{6.218282in}{3.580372in}} %
\pgfusepath{clip}%
\pgfsetbuttcap%
\pgfsetroundjoin%
\definecolor{currentfill}{rgb}{0.243520,0.319189,0.771672}%
\pgfsetfillcolor{currentfill}%
\pgfsetlinewidth{0.000000pt}%
\definecolor{currentstroke}{rgb}{0.000000,0.000000,0.000000}%
\pgfsetstrokecolor{currentstroke}%
\pgfsetdash{}{0pt}%
\pgfpathmoveto{\pgfqpoint{3.236099in}{0.723497in}}%
\pgfpathlineto{\pgfqpoint{3.502561in}{2.073396in}}%
\pgfpathlineto{\pgfqpoint{3.236099in}{0.723497in}}%
\pgfpathclose%
\pgfusepath{fill}%
\end{pgfscope}%
\begin{pgfscope}%
\pgfpathrectangle{\pgfqpoint{0.017500in}{0.017500in}}{\pgfqpoint{6.218282in}{3.580372in}} %
\pgfusepath{clip}%
\pgfsetbuttcap%
\pgfsetroundjoin%
\definecolor{currentfill}{rgb}{0.243520,0.319189,0.771672}%
\pgfsetfillcolor{currentfill}%
\pgfsetlinewidth{0.000000pt}%
\definecolor{currentstroke}{rgb}{0.000000,0.000000,0.000000}%
\pgfsetstrokecolor{currentstroke}%
\pgfsetdash{}{0pt}%
\pgfpathmoveto{\pgfqpoint{3.236099in}{0.723497in}}%
\pgfpathlineto{\pgfqpoint{3.502561in}{2.073396in}}%
\pgfpathlineto{\pgfqpoint{3.236099in}{0.723497in}}%
\pgfpathclose%
\pgfusepath{fill}%
\end{pgfscope}%
\begin{pgfscope}%
\pgfpathrectangle{\pgfqpoint{0.017500in}{0.017500in}}{\pgfqpoint{6.218282in}{3.580372in}} %
\pgfusepath{clip}%
\pgfsetbuttcap%
\pgfsetroundjoin%
\definecolor{currentfill}{rgb}{0.243520,0.319189,0.771672}%
\pgfsetfillcolor{currentfill}%
\pgfsetlinewidth{0.000000pt}%
\definecolor{currentstroke}{rgb}{0.000000,0.000000,0.000000}%
\pgfsetstrokecolor{currentstroke}%
\pgfsetdash{}{0pt}%
\pgfpathmoveto{\pgfqpoint{3.236099in}{0.723497in}}%
\pgfpathlineto{\pgfqpoint{3.502561in}{2.073396in}}%
\pgfpathlineto{\pgfqpoint{3.236099in}{0.723497in}}%
\pgfpathclose%
\pgfusepath{fill}%
\end{pgfscope}%
\begin{pgfscope}%
\pgfpathrectangle{\pgfqpoint{0.017500in}{0.017500in}}{\pgfqpoint{6.218282in}{3.580372in}} %
\pgfusepath{clip}%
\pgfsetbuttcap%
\pgfsetroundjoin%
\definecolor{currentfill}{rgb}{0.243520,0.319189,0.771672}%
\pgfsetfillcolor{currentfill}%
\pgfsetlinewidth{0.000000pt}%
\definecolor{currentstroke}{rgb}{0.000000,0.000000,0.000000}%
\pgfsetstrokecolor{currentstroke}%
\pgfsetdash{}{0pt}%
\pgfpathmoveto{\pgfqpoint{3.236099in}{0.723497in}}%
\pgfpathlineto{\pgfqpoint{3.502561in}{2.073396in}}%
\pgfpathlineto{\pgfqpoint{3.236099in}{0.723497in}}%
\pgfpathclose%
\pgfusepath{fill}%
\end{pgfscope}%
\begin{pgfscope}%
\pgfpathrectangle{\pgfqpoint{0.017500in}{0.017500in}}{\pgfqpoint{6.218282in}{3.580372in}} %
\pgfusepath{clip}%
\pgfsetbuttcap%
\pgfsetroundjoin%
\definecolor{currentfill}{rgb}{0.243520,0.319189,0.771672}%
\pgfsetfillcolor{currentfill}%
\pgfsetlinewidth{0.000000pt}%
\definecolor{currentstroke}{rgb}{0.000000,0.000000,0.000000}%
\pgfsetstrokecolor{currentstroke}%
\pgfsetdash{}{0pt}%
\pgfpathmoveto{\pgfqpoint{3.236099in}{0.723497in}}%
\pgfpathlineto{\pgfqpoint{3.502561in}{2.073396in}}%
\pgfpathlineto{\pgfqpoint{3.236099in}{0.723497in}}%
\pgfpathclose%
\pgfusepath{fill}%
\end{pgfscope}%
\begin{pgfscope}%
\pgfpathrectangle{\pgfqpoint{0.017500in}{0.017500in}}{\pgfqpoint{6.218282in}{3.580372in}} %
\pgfusepath{clip}%
\pgfsetbuttcap%
\pgfsetroundjoin%
\definecolor{currentfill}{rgb}{0.243520,0.319189,0.771672}%
\pgfsetfillcolor{currentfill}%
\pgfsetlinewidth{0.000000pt}%
\definecolor{currentstroke}{rgb}{0.000000,0.000000,0.000000}%
\pgfsetstrokecolor{currentstroke}%
\pgfsetdash{}{0pt}%
\pgfpathmoveto{\pgfqpoint{3.236099in}{0.723497in}}%
\pgfpathlineto{\pgfqpoint{3.502561in}{2.073396in}}%
\pgfpathlineto{\pgfqpoint{3.236099in}{0.723497in}}%
\pgfpathclose%
\pgfusepath{fill}%
\end{pgfscope}%
\begin{pgfscope}%
\pgfpathrectangle{\pgfqpoint{0.017500in}{0.017500in}}{\pgfqpoint{6.218282in}{3.580372in}} %
\pgfusepath{clip}%
\pgfsetbuttcap%
\pgfsetroundjoin%
\definecolor{currentfill}{rgb}{0.243520,0.319189,0.771672}%
\pgfsetfillcolor{currentfill}%
\pgfsetlinewidth{0.000000pt}%
\definecolor{currentstroke}{rgb}{0.000000,0.000000,0.000000}%
\pgfsetstrokecolor{currentstroke}%
\pgfsetdash{}{0pt}%
\pgfpathmoveto{\pgfqpoint{3.236099in}{0.723497in}}%
\pgfpathlineto{\pgfqpoint{3.502561in}{2.073396in}}%
\pgfpathlineto{\pgfqpoint{3.236099in}{0.723497in}}%
\pgfpathclose%
\pgfusepath{fill}%
\end{pgfscope}%
\begin{pgfscope}%
\pgfpathrectangle{\pgfqpoint{0.017500in}{0.017500in}}{\pgfqpoint{6.218282in}{3.580372in}} %
\pgfusepath{clip}%
\pgfsetbuttcap%
\pgfsetroundjoin%
\definecolor{currentfill}{rgb}{0.763520,0.178667,0.193396}%
\pgfsetfillcolor{currentfill}%
\pgfsetlinewidth{0.000000pt}%
\definecolor{currentstroke}{rgb}{0.000000,0.000000,0.000000}%
\pgfsetstrokecolor{currentstroke}%
\pgfsetdash{}{0pt}%
\pgfpathmoveto{\pgfqpoint{4.478468in}{2.393972in}}%
\pgfpathlineto{\pgfqpoint{4.724651in}{2.549289in}}%
\pgfpathlineto{\pgfqpoint{4.478468in}{2.393972in}}%
\pgfpathclose%
\pgfusepath{fill}%
\end{pgfscope}%
\begin{pgfscope}%
\pgfpathrectangle{\pgfqpoint{0.017500in}{0.017500in}}{\pgfqpoint{6.218282in}{3.580372in}} %
\pgfusepath{clip}%
\pgfsetbuttcap%
\pgfsetroundjoin%
\definecolor{currentfill}{rgb}{0.763520,0.178667,0.193396}%
\pgfsetfillcolor{currentfill}%
\pgfsetlinewidth{0.000000pt}%
\definecolor{currentstroke}{rgb}{0.000000,0.000000,0.000000}%
\pgfsetstrokecolor{currentstroke}%
\pgfsetdash{}{0pt}%
\pgfpathmoveto{\pgfqpoint{4.478468in}{2.393972in}}%
\pgfpathlineto{\pgfqpoint{4.724651in}{2.549289in}}%
\pgfpathlineto{\pgfqpoint{4.478468in}{2.393972in}}%
\pgfpathclose%
\pgfusepath{fill}%
\end{pgfscope}%
\begin{pgfscope}%
\pgfpathrectangle{\pgfqpoint{0.017500in}{0.017500in}}{\pgfqpoint{6.218282in}{3.580372in}} %
\pgfusepath{clip}%
\pgfsetbuttcap%
\pgfsetroundjoin%
\definecolor{currentfill}{rgb}{0.763520,0.178667,0.193396}%
\pgfsetfillcolor{currentfill}%
\pgfsetlinewidth{0.000000pt}%
\definecolor{currentstroke}{rgb}{0.000000,0.000000,0.000000}%
\pgfsetstrokecolor{currentstroke}%
\pgfsetdash{}{0pt}%
\pgfpathmoveto{\pgfqpoint{4.478468in}{2.393972in}}%
\pgfpathlineto{\pgfqpoint{4.724651in}{2.549289in}}%
\pgfpathlineto{\pgfqpoint{4.478468in}{2.393972in}}%
\pgfpathclose%
\pgfusepath{fill}%
\end{pgfscope}%
\begin{pgfscope}%
\pgfpathrectangle{\pgfqpoint{0.017500in}{0.017500in}}{\pgfqpoint{6.218282in}{3.580372in}} %
\pgfusepath{clip}%
\pgfsetbuttcap%
\pgfsetroundjoin%
\definecolor{currentfill}{rgb}{0.763520,0.178667,0.193396}%
\pgfsetfillcolor{currentfill}%
\pgfsetlinewidth{0.000000pt}%
\definecolor{currentstroke}{rgb}{0.000000,0.000000,0.000000}%
\pgfsetstrokecolor{currentstroke}%
\pgfsetdash{}{0pt}%
\pgfpathmoveto{\pgfqpoint{4.478468in}{2.393972in}}%
\pgfpathlineto{\pgfqpoint{4.724651in}{2.549289in}}%
\pgfpathlineto{\pgfqpoint{4.478468in}{2.393972in}}%
\pgfpathclose%
\pgfusepath{fill}%
\end{pgfscope}%
\begin{pgfscope}%
\pgfpathrectangle{\pgfqpoint{0.017500in}{0.017500in}}{\pgfqpoint{6.218282in}{3.580372in}} %
\pgfusepath{clip}%
\pgfsetbuttcap%
\pgfsetroundjoin%
\definecolor{currentfill}{rgb}{0.763520,0.178667,0.193396}%
\pgfsetfillcolor{currentfill}%
\pgfsetlinewidth{0.000000pt}%
\definecolor{currentstroke}{rgb}{0.000000,0.000000,0.000000}%
\pgfsetstrokecolor{currentstroke}%
\pgfsetdash{}{0pt}%
\pgfpathmoveto{\pgfqpoint{4.478468in}{2.393972in}}%
\pgfpathlineto{\pgfqpoint{4.724651in}{2.549289in}}%
\pgfpathlineto{\pgfqpoint{4.478468in}{2.393972in}}%
\pgfpathclose%
\pgfusepath{fill}%
\end{pgfscope}%
\begin{pgfscope}%
\pgfpathrectangle{\pgfqpoint{0.017500in}{0.017500in}}{\pgfqpoint{6.218282in}{3.580372in}} %
\pgfusepath{clip}%
\pgfsetbuttcap%
\pgfsetroundjoin%
\definecolor{currentfill}{rgb}{0.763520,0.178667,0.193396}%
\pgfsetfillcolor{currentfill}%
\pgfsetlinewidth{0.000000pt}%
\definecolor{currentstroke}{rgb}{0.000000,0.000000,0.000000}%
\pgfsetstrokecolor{currentstroke}%
\pgfsetdash{}{0pt}%
\pgfpathmoveto{\pgfqpoint{4.478468in}{2.393972in}}%
\pgfpathlineto{\pgfqpoint{4.724651in}{2.549289in}}%
\pgfpathlineto{\pgfqpoint{4.478468in}{2.393972in}}%
\pgfpathclose%
\pgfusepath{fill}%
\end{pgfscope}%
\begin{pgfscope}%
\pgfpathrectangle{\pgfqpoint{0.017500in}{0.017500in}}{\pgfqpoint{6.218282in}{3.580372in}} %
\pgfusepath{clip}%
\pgfsetbuttcap%
\pgfsetroundjoin%
\definecolor{currentfill}{rgb}{0.763520,0.178667,0.193396}%
\pgfsetfillcolor{currentfill}%
\pgfsetlinewidth{0.000000pt}%
\definecolor{currentstroke}{rgb}{0.000000,0.000000,0.000000}%
\pgfsetstrokecolor{currentstroke}%
\pgfsetdash{}{0pt}%
\pgfpathmoveto{\pgfqpoint{4.478468in}{2.393972in}}%
\pgfpathlineto{\pgfqpoint{4.724651in}{2.549289in}}%
\pgfpathlineto{\pgfqpoint{4.478468in}{2.393972in}}%
\pgfpathclose%
\pgfusepath{fill}%
\end{pgfscope}%
\begin{pgfscope}%
\pgfpathrectangle{\pgfqpoint{0.017500in}{0.017500in}}{\pgfqpoint{6.218282in}{3.580372in}} %
\pgfusepath{clip}%
\pgfsetbuttcap%
\pgfsetroundjoin%
\definecolor{currentfill}{rgb}{0.763520,0.178667,0.193396}%
\pgfsetfillcolor{currentfill}%
\pgfsetlinewidth{0.000000pt}%
\definecolor{currentstroke}{rgb}{0.000000,0.000000,0.000000}%
\pgfsetstrokecolor{currentstroke}%
\pgfsetdash{}{0pt}%
\pgfpathmoveto{\pgfqpoint{4.478468in}{2.393972in}}%
\pgfpathlineto{\pgfqpoint{4.724651in}{2.549289in}}%
\pgfpathlineto{\pgfqpoint{4.478468in}{2.393972in}}%
\pgfpathclose%
\pgfusepath{fill}%
\end{pgfscope}%
\begin{pgfscope}%
\pgfpathrectangle{\pgfqpoint{0.017500in}{0.017500in}}{\pgfqpoint{6.218282in}{3.580372in}} %
\pgfusepath{clip}%
\pgfsetbuttcap%
\pgfsetroundjoin%
\definecolor{currentfill}{rgb}{0.763520,0.178667,0.193396}%
\pgfsetfillcolor{currentfill}%
\pgfsetlinewidth{0.000000pt}%
\definecolor{currentstroke}{rgb}{0.000000,0.000000,0.000000}%
\pgfsetstrokecolor{currentstroke}%
\pgfsetdash{}{0pt}%
\pgfpathmoveto{\pgfqpoint{4.478468in}{2.393972in}}%
\pgfpathlineto{\pgfqpoint{4.724651in}{2.549289in}}%
\pgfpathlineto{\pgfqpoint{4.478468in}{2.393972in}}%
\pgfpathclose%
\pgfusepath{fill}%
\end{pgfscope}%
\begin{pgfscope}%
\pgfpathrectangle{\pgfqpoint{0.017500in}{0.017500in}}{\pgfqpoint{6.218282in}{3.580372in}} %
\pgfusepath{clip}%
\pgfsetbuttcap%
\pgfsetroundjoin%
\definecolor{currentfill}{rgb}{0.763520,0.178667,0.193396}%
\pgfsetfillcolor{currentfill}%
\pgfsetlinewidth{0.000000pt}%
\definecolor{currentstroke}{rgb}{0.000000,0.000000,0.000000}%
\pgfsetstrokecolor{currentstroke}%
\pgfsetdash{}{0pt}%
\pgfpathmoveto{\pgfqpoint{4.478468in}{2.393972in}}%
\pgfpathlineto{\pgfqpoint{4.724651in}{2.549289in}}%
\pgfpathlineto{\pgfqpoint{4.478468in}{2.393972in}}%
\pgfpathclose%
\pgfusepath{fill}%
\end{pgfscope}%
\begin{pgfscope}%
\pgfpathrectangle{\pgfqpoint{0.017500in}{0.017500in}}{\pgfqpoint{6.218282in}{3.580372in}} %
\pgfusepath{clip}%
\pgfsetbuttcap%
\pgfsetroundjoin%
\definecolor{currentfill}{rgb}{0.763520,0.178667,0.193396}%
\pgfsetfillcolor{currentfill}%
\pgfsetlinewidth{0.000000pt}%
\definecolor{currentstroke}{rgb}{0.000000,0.000000,0.000000}%
\pgfsetstrokecolor{currentstroke}%
\pgfsetdash{}{0pt}%
\pgfpathmoveto{\pgfqpoint{4.478468in}{2.393972in}}%
\pgfpathlineto{\pgfqpoint{4.724651in}{2.549289in}}%
\pgfpathlineto{\pgfqpoint{4.478468in}{2.393972in}}%
\pgfpathclose%
\pgfusepath{fill}%
\end{pgfscope}%
\begin{pgfscope}%
\pgfpathrectangle{\pgfqpoint{0.017500in}{0.017500in}}{\pgfqpoint{6.218282in}{3.580372in}} %
\pgfusepath{clip}%
\pgfsetbuttcap%
\pgfsetroundjoin%
\definecolor{currentfill}{rgb}{0.763520,0.178667,0.193396}%
\pgfsetfillcolor{currentfill}%
\pgfsetlinewidth{0.000000pt}%
\definecolor{currentstroke}{rgb}{0.000000,0.000000,0.000000}%
\pgfsetstrokecolor{currentstroke}%
\pgfsetdash{}{0pt}%
\pgfpathmoveto{\pgfqpoint{4.478468in}{2.393972in}}%
\pgfpathlineto{\pgfqpoint{4.724651in}{2.549289in}}%
\pgfpathlineto{\pgfqpoint{4.478468in}{2.393972in}}%
\pgfpathclose%
\pgfusepath{fill}%
\end{pgfscope}%
\begin{pgfscope}%
\pgfpathrectangle{\pgfqpoint{0.017500in}{0.017500in}}{\pgfqpoint{6.218282in}{3.580372in}} %
\pgfusepath{clip}%
\pgfsetbuttcap%
\pgfsetroundjoin%
\definecolor{currentfill}{rgb}{0.763520,0.178667,0.193396}%
\pgfsetfillcolor{currentfill}%
\pgfsetlinewidth{0.000000pt}%
\definecolor{currentstroke}{rgb}{0.000000,0.000000,0.000000}%
\pgfsetstrokecolor{currentstroke}%
\pgfsetdash{}{0pt}%
\pgfpathmoveto{\pgfqpoint{4.478468in}{2.393972in}}%
\pgfpathlineto{\pgfqpoint{4.724651in}{2.549289in}}%
\pgfpathlineto{\pgfqpoint{4.478468in}{2.393972in}}%
\pgfpathclose%
\pgfusepath{fill}%
\end{pgfscope}%
\begin{pgfscope}%
\pgfpathrectangle{\pgfqpoint{0.017500in}{0.017500in}}{\pgfqpoint{6.218282in}{3.580372in}} %
\pgfusepath{clip}%
\pgfsetbuttcap%
\pgfsetroundjoin%
\definecolor{currentfill}{rgb}{0.763520,0.178667,0.193396}%
\pgfsetfillcolor{currentfill}%
\pgfsetlinewidth{0.000000pt}%
\definecolor{currentstroke}{rgb}{0.000000,0.000000,0.000000}%
\pgfsetstrokecolor{currentstroke}%
\pgfsetdash{}{0pt}%
\pgfpathmoveto{\pgfqpoint{4.478468in}{2.393972in}}%
\pgfpathlineto{\pgfqpoint{4.724651in}{2.549289in}}%
\pgfpathlineto{\pgfqpoint{4.478468in}{2.393972in}}%
\pgfpathclose%
\pgfusepath{fill}%
\end{pgfscope}%
\begin{pgfscope}%
\pgfpathrectangle{\pgfqpoint{0.017500in}{0.017500in}}{\pgfqpoint{6.218282in}{3.580372in}} %
\pgfusepath{clip}%
\pgfsetbuttcap%
\pgfsetroundjoin%
\definecolor{currentfill}{rgb}{0.763520,0.178667,0.193396}%
\pgfsetfillcolor{currentfill}%
\pgfsetlinewidth{0.000000pt}%
\definecolor{currentstroke}{rgb}{0.000000,0.000000,0.000000}%
\pgfsetstrokecolor{currentstroke}%
\pgfsetdash{}{0pt}%
\pgfpathmoveto{\pgfqpoint{4.478468in}{2.393972in}}%
\pgfpathlineto{\pgfqpoint{4.724651in}{2.549289in}}%
\pgfpathlineto{\pgfqpoint{4.478468in}{2.393972in}}%
\pgfpathclose%
\pgfusepath{fill}%
\end{pgfscope}%
\begin{pgfscope}%
\pgfpathrectangle{\pgfqpoint{0.017500in}{0.017500in}}{\pgfqpoint{6.218282in}{3.580372in}} %
\pgfusepath{clip}%
\pgfsetbuttcap%
\pgfsetroundjoin%
\definecolor{currentfill}{rgb}{0.763520,0.178667,0.193396}%
\pgfsetfillcolor{currentfill}%
\pgfsetlinewidth{0.000000pt}%
\definecolor{currentstroke}{rgb}{0.000000,0.000000,0.000000}%
\pgfsetstrokecolor{currentstroke}%
\pgfsetdash{}{0pt}%
\pgfpathmoveto{\pgfqpoint{4.478468in}{2.393972in}}%
\pgfpathlineto{\pgfqpoint{4.724651in}{2.549289in}}%
\pgfpathlineto{\pgfqpoint{4.478468in}{2.393972in}}%
\pgfpathclose%
\pgfusepath{fill}%
\end{pgfscope}%
\begin{pgfscope}%
\pgfpathrectangle{\pgfqpoint{0.017500in}{0.017500in}}{\pgfqpoint{6.218282in}{3.580372in}} %
\pgfusepath{clip}%
\pgfsetbuttcap%
\pgfsetroundjoin%
\definecolor{currentfill}{rgb}{0.763520,0.178667,0.193396}%
\pgfsetfillcolor{currentfill}%
\pgfsetlinewidth{0.000000pt}%
\definecolor{currentstroke}{rgb}{0.000000,0.000000,0.000000}%
\pgfsetstrokecolor{currentstroke}%
\pgfsetdash{}{0pt}%
\pgfpathmoveto{\pgfqpoint{4.478468in}{2.393972in}}%
\pgfpathlineto{\pgfqpoint{4.724651in}{2.549289in}}%
\pgfpathlineto{\pgfqpoint{4.478468in}{2.393972in}}%
\pgfpathclose%
\pgfusepath{fill}%
\end{pgfscope}%
\begin{pgfscope}%
\pgfpathrectangle{\pgfqpoint{0.017500in}{0.017500in}}{\pgfqpoint{6.218282in}{3.580372in}} %
\pgfusepath{clip}%
\pgfsetbuttcap%
\pgfsetroundjoin%
\definecolor{currentfill}{rgb}{0.763520,0.178667,0.193396}%
\pgfsetfillcolor{currentfill}%
\pgfsetlinewidth{0.000000pt}%
\definecolor{currentstroke}{rgb}{0.000000,0.000000,0.000000}%
\pgfsetstrokecolor{currentstroke}%
\pgfsetdash{}{0pt}%
\pgfpathmoveto{\pgfqpoint{4.478468in}{2.393972in}}%
\pgfpathlineto{\pgfqpoint{4.724651in}{2.549289in}}%
\pgfpathlineto{\pgfqpoint{4.478468in}{2.393972in}}%
\pgfpathclose%
\pgfusepath{fill}%
\end{pgfscope}%
\begin{pgfscope}%
\pgfpathrectangle{\pgfqpoint{0.017500in}{0.017500in}}{\pgfqpoint{6.218282in}{3.580372in}} %
\pgfusepath{clip}%
\pgfsetbuttcap%
\pgfsetroundjoin%
\definecolor{currentfill}{rgb}{0.763520,0.178667,0.193396}%
\pgfsetfillcolor{currentfill}%
\pgfsetlinewidth{0.000000pt}%
\definecolor{currentstroke}{rgb}{0.000000,0.000000,0.000000}%
\pgfsetstrokecolor{currentstroke}%
\pgfsetdash{}{0pt}%
\pgfpathmoveto{\pgfqpoint{4.478468in}{2.393972in}}%
\pgfpathlineto{\pgfqpoint{4.724651in}{2.549289in}}%
\pgfpathlineto{\pgfqpoint{4.478468in}{2.393972in}}%
\pgfpathclose%
\pgfusepath{fill}%
\end{pgfscope}%
\begin{pgfscope}%
\pgfpathrectangle{\pgfqpoint{0.017500in}{0.017500in}}{\pgfqpoint{6.218282in}{3.580372in}} %
\pgfusepath{clip}%
\pgfsetbuttcap%
\pgfsetroundjoin%
\definecolor{currentfill}{rgb}{0.763520,0.178667,0.193396}%
\pgfsetfillcolor{currentfill}%
\pgfsetlinewidth{0.000000pt}%
\definecolor{currentstroke}{rgb}{0.000000,0.000000,0.000000}%
\pgfsetstrokecolor{currentstroke}%
\pgfsetdash{}{0pt}%
\pgfpathmoveto{\pgfqpoint{4.478468in}{2.393972in}}%
\pgfpathlineto{\pgfqpoint{4.724651in}{2.549289in}}%
\pgfpathlineto{\pgfqpoint{4.478468in}{2.393972in}}%
\pgfpathclose%
\pgfusepath{fill}%
\end{pgfscope}%
\begin{pgfscope}%
\pgfpathrectangle{\pgfqpoint{0.017500in}{0.017500in}}{\pgfqpoint{6.218282in}{3.580372in}} %
\pgfusepath{clip}%
\pgfsetbuttcap%
\pgfsetroundjoin%
\definecolor{currentfill}{rgb}{0.763520,0.178667,0.193396}%
\pgfsetfillcolor{currentfill}%
\pgfsetlinewidth{0.000000pt}%
\definecolor{currentstroke}{rgb}{0.000000,0.000000,0.000000}%
\pgfsetstrokecolor{currentstroke}%
\pgfsetdash{}{0pt}%
\pgfpathmoveto{\pgfqpoint{4.478468in}{2.393972in}}%
\pgfpathlineto{\pgfqpoint{4.724651in}{2.549289in}}%
\pgfpathlineto{\pgfqpoint{4.478468in}{2.393972in}}%
\pgfpathclose%
\pgfusepath{fill}%
\end{pgfscope}%
\begin{pgfscope}%
\pgfpathrectangle{\pgfqpoint{0.017500in}{0.017500in}}{\pgfqpoint{6.218282in}{3.580372in}} %
\pgfusepath{clip}%
\pgfsetbuttcap%
\pgfsetroundjoin%
\definecolor{currentfill}{rgb}{0.763520,0.178667,0.193396}%
\pgfsetfillcolor{currentfill}%
\pgfsetlinewidth{0.000000pt}%
\definecolor{currentstroke}{rgb}{0.000000,0.000000,0.000000}%
\pgfsetstrokecolor{currentstroke}%
\pgfsetdash{}{0pt}%
\pgfpathmoveto{\pgfqpoint{4.478468in}{2.393972in}}%
\pgfpathlineto{\pgfqpoint{4.724651in}{2.549289in}}%
\pgfpathlineto{\pgfqpoint{4.478468in}{2.393972in}}%
\pgfpathclose%
\pgfusepath{fill}%
\end{pgfscope}%
\begin{pgfscope}%
\pgfpathrectangle{\pgfqpoint{0.017500in}{0.017500in}}{\pgfqpoint{6.218282in}{3.580372in}} %
\pgfusepath{clip}%
\pgfsetbuttcap%
\pgfsetroundjoin%
\definecolor{currentfill}{rgb}{0.763520,0.178667,0.193396}%
\pgfsetfillcolor{currentfill}%
\pgfsetlinewidth{0.000000pt}%
\definecolor{currentstroke}{rgb}{0.000000,0.000000,0.000000}%
\pgfsetstrokecolor{currentstroke}%
\pgfsetdash{}{0pt}%
\pgfpathmoveto{\pgfqpoint{4.478468in}{2.393972in}}%
\pgfpathlineto{\pgfqpoint{4.724651in}{2.549289in}}%
\pgfpathlineto{\pgfqpoint{4.478468in}{2.393972in}}%
\pgfpathclose%
\pgfusepath{fill}%
\end{pgfscope}%
\begin{pgfscope}%
\pgfpathrectangle{\pgfqpoint{0.017500in}{0.017500in}}{\pgfqpoint{6.218282in}{3.580372in}} %
\pgfusepath{clip}%
\pgfsetbuttcap%
\pgfsetroundjoin%
\definecolor{currentfill}{rgb}{0.763520,0.178667,0.193396}%
\pgfsetfillcolor{currentfill}%
\pgfsetlinewidth{0.000000pt}%
\definecolor{currentstroke}{rgb}{0.000000,0.000000,0.000000}%
\pgfsetstrokecolor{currentstroke}%
\pgfsetdash{}{0pt}%
\pgfpathmoveto{\pgfqpoint{4.478468in}{2.393972in}}%
\pgfpathlineto{\pgfqpoint{4.724651in}{2.549289in}}%
\pgfpathlineto{\pgfqpoint{4.478468in}{2.393972in}}%
\pgfpathclose%
\pgfusepath{fill}%
\end{pgfscope}%
\begin{pgfscope}%
\pgfpathrectangle{\pgfqpoint{0.017500in}{0.017500in}}{\pgfqpoint{6.218282in}{3.580372in}} %
\pgfusepath{clip}%
\pgfsetbuttcap%
\pgfsetroundjoin%
\definecolor{currentfill}{rgb}{0.763520,0.178667,0.193396}%
\pgfsetfillcolor{currentfill}%
\pgfsetlinewidth{0.000000pt}%
\definecolor{currentstroke}{rgb}{0.000000,0.000000,0.000000}%
\pgfsetstrokecolor{currentstroke}%
\pgfsetdash{}{0pt}%
\pgfpathmoveto{\pgfqpoint{4.478468in}{2.393972in}}%
\pgfpathlineto{\pgfqpoint{4.724651in}{2.549289in}}%
\pgfpathlineto{\pgfqpoint{4.478468in}{2.393972in}}%
\pgfpathclose%
\pgfusepath{fill}%
\end{pgfscope}%
\begin{pgfscope}%
\pgfpathrectangle{\pgfqpoint{0.017500in}{0.017500in}}{\pgfqpoint{6.218282in}{3.580372in}} %
\pgfusepath{clip}%
\pgfsetbuttcap%
\pgfsetroundjoin%
\definecolor{currentfill}{rgb}{0.763520,0.178667,0.193396}%
\pgfsetfillcolor{currentfill}%
\pgfsetlinewidth{0.000000pt}%
\definecolor{currentstroke}{rgb}{0.000000,0.000000,0.000000}%
\pgfsetstrokecolor{currentstroke}%
\pgfsetdash{}{0pt}%
\pgfpathmoveto{\pgfqpoint{4.478468in}{2.393972in}}%
\pgfpathlineto{\pgfqpoint{4.724651in}{2.549289in}}%
\pgfpathlineto{\pgfqpoint{4.478468in}{2.393972in}}%
\pgfpathclose%
\pgfusepath{fill}%
\end{pgfscope}%
\begin{pgfscope}%
\pgfpathrectangle{\pgfqpoint{0.017500in}{0.017500in}}{\pgfqpoint{6.218282in}{3.580372in}} %
\pgfusepath{clip}%
\pgfsetbuttcap%
\pgfsetroundjoin%
\definecolor{currentfill}{rgb}{0.763520,0.178667,0.193396}%
\pgfsetfillcolor{currentfill}%
\pgfsetlinewidth{0.000000pt}%
\definecolor{currentstroke}{rgb}{0.000000,0.000000,0.000000}%
\pgfsetstrokecolor{currentstroke}%
\pgfsetdash{}{0pt}%
\pgfpathmoveto{\pgfqpoint{4.478468in}{2.393972in}}%
\pgfpathlineto{\pgfqpoint{4.724651in}{2.549289in}}%
\pgfpathlineto{\pgfqpoint{4.478468in}{2.393972in}}%
\pgfpathclose%
\pgfusepath{fill}%
\end{pgfscope}%
\begin{pgfscope}%
\pgfpathrectangle{\pgfqpoint{0.017500in}{0.017500in}}{\pgfqpoint{6.218282in}{3.580372in}} %
\pgfusepath{clip}%
\pgfsetbuttcap%
\pgfsetroundjoin%
\definecolor{currentfill}{rgb}{0.763520,0.178667,0.193396}%
\pgfsetfillcolor{currentfill}%
\pgfsetlinewidth{0.000000pt}%
\definecolor{currentstroke}{rgb}{0.000000,0.000000,0.000000}%
\pgfsetstrokecolor{currentstroke}%
\pgfsetdash{}{0pt}%
\pgfpathmoveto{\pgfqpoint{4.478468in}{2.393972in}}%
\pgfpathlineto{\pgfqpoint{4.724651in}{2.549289in}}%
\pgfpathlineto{\pgfqpoint{4.478468in}{2.393972in}}%
\pgfpathclose%
\pgfusepath{fill}%
\end{pgfscope}%
\begin{pgfscope}%
\pgfpathrectangle{\pgfqpoint{0.017500in}{0.017500in}}{\pgfqpoint{6.218282in}{3.580372in}} %
\pgfusepath{clip}%
\pgfsetbuttcap%
\pgfsetroundjoin%
\definecolor{currentfill}{rgb}{0.763520,0.178667,0.193396}%
\pgfsetfillcolor{currentfill}%
\pgfsetlinewidth{0.000000pt}%
\definecolor{currentstroke}{rgb}{0.000000,0.000000,0.000000}%
\pgfsetstrokecolor{currentstroke}%
\pgfsetdash{}{0pt}%
\pgfpathmoveto{\pgfqpoint{4.478468in}{2.393972in}}%
\pgfpathlineto{\pgfqpoint{4.724651in}{2.549289in}}%
\pgfpathlineto{\pgfqpoint{4.478468in}{2.393972in}}%
\pgfpathclose%
\pgfusepath{fill}%
\end{pgfscope}%
\begin{pgfscope}%
\pgfpathrectangle{\pgfqpoint{0.017500in}{0.017500in}}{\pgfqpoint{6.218282in}{3.580372in}} %
\pgfusepath{clip}%
\pgfsetbuttcap%
\pgfsetroundjoin%
\definecolor{currentfill}{rgb}{0.763520,0.178667,0.193396}%
\pgfsetfillcolor{currentfill}%
\pgfsetlinewidth{0.000000pt}%
\definecolor{currentstroke}{rgb}{0.000000,0.000000,0.000000}%
\pgfsetstrokecolor{currentstroke}%
\pgfsetdash{}{0pt}%
\pgfpathmoveto{\pgfqpoint{4.478468in}{2.393972in}}%
\pgfpathlineto{\pgfqpoint{4.724651in}{2.549289in}}%
\pgfpathlineto{\pgfqpoint{4.478468in}{2.393972in}}%
\pgfpathclose%
\pgfusepath{fill}%
\end{pgfscope}%
\begin{pgfscope}%
\pgfpathrectangle{\pgfqpoint{0.017500in}{0.017500in}}{\pgfqpoint{6.218282in}{3.580372in}} %
\pgfusepath{clip}%
\pgfsetbuttcap%
\pgfsetroundjoin%
\definecolor{currentfill}{rgb}{0.888390,0.417703,0.327898}%
\pgfsetfillcolor{currentfill}%
\pgfsetlinewidth{0.000000pt}%
\definecolor{currentstroke}{rgb}{0.000000,0.000000,0.000000}%
\pgfsetstrokecolor{currentstroke}%
\pgfsetdash{}{0pt}%
\pgfpathmoveto{\pgfqpoint{2.590432in}{2.210709in}}%
\pgfpathlineto{\pgfqpoint{3.043326in}{2.141782in}}%
\pgfpathlineto{\pgfqpoint{3.306485in}{2.382499in}}%
\pgfpathlineto{\pgfqpoint{2.855513in}{2.451418in}}%
\pgfpathlineto{\pgfqpoint{2.590432in}{2.210709in}}%
\pgfpathclose%
\pgfusepath{fill}%
\end{pgfscope}%
\begin{pgfscope}%
\pgfpathrectangle{\pgfqpoint{0.017500in}{0.017500in}}{\pgfqpoint{6.218282in}{3.580372in}} %
\pgfusepath{clip}%
\pgfsetbuttcap%
\pgfsetroundjoin%
\definecolor{currentfill}{rgb}{0.888390,0.417703,0.327898}%
\pgfsetfillcolor{currentfill}%
\pgfsetlinewidth{0.000000pt}%
\definecolor{currentstroke}{rgb}{0.000000,0.000000,0.000000}%
\pgfsetstrokecolor{currentstroke}%
\pgfsetdash{}{0pt}%
\pgfpathmoveto{\pgfqpoint{2.590432in}{2.210709in}}%
\pgfpathlineto{\pgfqpoint{3.043326in}{2.141782in}}%
\pgfpathlineto{\pgfqpoint{3.306485in}{2.382499in}}%
\pgfpathlineto{\pgfqpoint{2.855513in}{2.451418in}}%
\pgfpathlineto{\pgfqpoint{2.590432in}{2.210709in}}%
\pgfpathclose%
\pgfusepath{fill}%
\end{pgfscope}%
\begin{pgfscope}%
\pgfpathrectangle{\pgfqpoint{0.017500in}{0.017500in}}{\pgfqpoint{6.218282in}{3.580372in}} %
\pgfusepath{clip}%
\pgfsetbuttcap%
\pgfsetroundjoin%
\definecolor{currentfill}{rgb}{0.888390,0.417703,0.327898}%
\pgfsetfillcolor{currentfill}%
\pgfsetlinewidth{0.000000pt}%
\definecolor{currentstroke}{rgb}{0.000000,0.000000,0.000000}%
\pgfsetstrokecolor{currentstroke}%
\pgfsetdash{}{0pt}%
\pgfpathmoveto{\pgfqpoint{2.590432in}{2.210709in}}%
\pgfpathlineto{\pgfqpoint{3.043326in}{2.141782in}}%
\pgfpathlineto{\pgfqpoint{3.306485in}{2.382499in}}%
\pgfpathlineto{\pgfqpoint{2.855513in}{2.451418in}}%
\pgfpathlineto{\pgfqpoint{2.590432in}{2.210709in}}%
\pgfpathclose%
\pgfusepath{fill}%
\end{pgfscope}%
\begin{pgfscope}%
\pgfpathrectangle{\pgfqpoint{0.017500in}{0.017500in}}{\pgfqpoint{6.218282in}{3.580372in}} %
\pgfusepath{clip}%
\pgfsetbuttcap%
\pgfsetroundjoin%
\definecolor{currentfill}{rgb}{0.888390,0.417703,0.327898}%
\pgfsetfillcolor{currentfill}%
\pgfsetlinewidth{0.000000pt}%
\definecolor{currentstroke}{rgb}{0.000000,0.000000,0.000000}%
\pgfsetstrokecolor{currentstroke}%
\pgfsetdash{}{0pt}%
\pgfpathmoveto{\pgfqpoint{2.590432in}{2.210709in}}%
\pgfpathlineto{\pgfqpoint{3.043326in}{2.141782in}}%
\pgfpathlineto{\pgfqpoint{3.306485in}{2.382499in}}%
\pgfpathlineto{\pgfqpoint{2.855513in}{2.451418in}}%
\pgfpathlineto{\pgfqpoint{2.590432in}{2.210709in}}%
\pgfpathclose%
\pgfusepath{fill}%
\end{pgfscope}%
\begin{pgfscope}%
\pgfpathrectangle{\pgfqpoint{0.017500in}{0.017500in}}{\pgfqpoint{6.218282in}{3.580372in}} %
\pgfusepath{clip}%
\pgfsetbuttcap%
\pgfsetroundjoin%
\definecolor{currentfill}{rgb}{0.888390,0.417703,0.327898}%
\pgfsetfillcolor{currentfill}%
\pgfsetlinewidth{0.000000pt}%
\definecolor{currentstroke}{rgb}{0.000000,0.000000,0.000000}%
\pgfsetstrokecolor{currentstroke}%
\pgfsetdash{}{0pt}%
\pgfpathmoveto{\pgfqpoint{2.590432in}{2.210709in}}%
\pgfpathlineto{\pgfqpoint{3.043326in}{2.141782in}}%
\pgfpathlineto{\pgfqpoint{3.306485in}{2.382499in}}%
\pgfpathlineto{\pgfqpoint{2.855513in}{2.451418in}}%
\pgfpathlineto{\pgfqpoint{2.590432in}{2.210709in}}%
\pgfpathclose%
\pgfusepath{fill}%
\end{pgfscope}%
\begin{pgfscope}%
\pgfpathrectangle{\pgfqpoint{0.017500in}{0.017500in}}{\pgfqpoint{6.218282in}{3.580372in}} %
\pgfusepath{clip}%
\pgfsetbuttcap%
\pgfsetroundjoin%
\definecolor{currentfill}{rgb}{0.888390,0.417703,0.327898}%
\pgfsetfillcolor{currentfill}%
\pgfsetlinewidth{0.000000pt}%
\definecolor{currentstroke}{rgb}{0.000000,0.000000,0.000000}%
\pgfsetstrokecolor{currentstroke}%
\pgfsetdash{}{0pt}%
\pgfpathmoveto{\pgfqpoint{2.590432in}{2.210709in}}%
\pgfpathlineto{\pgfqpoint{3.043326in}{2.141782in}}%
\pgfpathlineto{\pgfqpoint{3.306485in}{2.382499in}}%
\pgfpathlineto{\pgfqpoint{2.855513in}{2.451418in}}%
\pgfpathlineto{\pgfqpoint{2.590432in}{2.210709in}}%
\pgfpathclose%
\pgfusepath{fill}%
\end{pgfscope}%
\begin{pgfscope}%
\pgfpathrectangle{\pgfqpoint{0.017500in}{0.017500in}}{\pgfqpoint{6.218282in}{3.580372in}} %
\pgfusepath{clip}%
\pgfsetbuttcap%
\pgfsetroundjoin%
\definecolor{currentfill}{rgb}{0.810616,0.268797,0.235428}%
\pgfsetfillcolor{currentfill}%
\pgfsetlinewidth{0.000000pt}%
\definecolor{currentstroke}{rgb}{0.000000,0.000000,0.000000}%
\pgfsetstrokecolor{currentstroke}%
\pgfsetdash{}{0pt}%
\pgfpathmoveto{\pgfqpoint{3.763398in}{2.297457in}}%
\pgfpathlineto{\pgfqpoint{4.018447in}{2.468167in}}%
\pgfpathlineto{\pgfqpoint{3.763398in}{2.297457in}}%
\pgfpathclose%
\pgfusepath{fill}%
\end{pgfscope}%
\begin{pgfscope}%
\pgfpathrectangle{\pgfqpoint{0.017500in}{0.017500in}}{\pgfqpoint{6.218282in}{3.580372in}} %
\pgfusepath{clip}%
\pgfsetbuttcap%
\pgfsetroundjoin%
\definecolor{currentfill}{rgb}{0.810616,0.268797,0.235428}%
\pgfsetfillcolor{currentfill}%
\pgfsetlinewidth{0.000000pt}%
\definecolor{currentstroke}{rgb}{0.000000,0.000000,0.000000}%
\pgfsetstrokecolor{currentstroke}%
\pgfsetdash{}{0pt}%
\pgfpathmoveto{\pgfqpoint{3.763398in}{2.297457in}}%
\pgfpathlineto{\pgfqpoint{4.018447in}{2.468167in}}%
\pgfpathlineto{\pgfqpoint{3.763398in}{2.297457in}}%
\pgfpathclose%
\pgfusepath{fill}%
\end{pgfscope}%
\begin{pgfscope}%
\pgfpathrectangle{\pgfqpoint{0.017500in}{0.017500in}}{\pgfqpoint{6.218282in}{3.580372in}} %
\pgfusepath{clip}%
\pgfsetbuttcap%
\pgfsetroundjoin%
\definecolor{currentfill}{rgb}{0.810616,0.268797,0.235428}%
\pgfsetfillcolor{currentfill}%
\pgfsetlinewidth{0.000000pt}%
\definecolor{currentstroke}{rgb}{0.000000,0.000000,0.000000}%
\pgfsetstrokecolor{currentstroke}%
\pgfsetdash{}{0pt}%
\pgfpathmoveto{\pgfqpoint{3.763398in}{2.297457in}}%
\pgfpathlineto{\pgfqpoint{4.018447in}{2.468167in}}%
\pgfpathlineto{\pgfqpoint{3.763398in}{2.297457in}}%
\pgfpathclose%
\pgfusepath{fill}%
\end{pgfscope}%
\begin{pgfscope}%
\pgfpathrectangle{\pgfqpoint{0.017500in}{0.017500in}}{\pgfqpoint{6.218282in}{3.580372in}} %
\pgfusepath{clip}%
\pgfsetbuttcap%
\pgfsetroundjoin%
\definecolor{currentfill}{rgb}{0.810616,0.268797,0.235428}%
\pgfsetfillcolor{currentfill}%
\pgfsetlinewidth{0.000000pt}%
\definecolor{currentstroke}{rgb}{0.000000,0.000000,0.000000}%
\pgfsetstrokecolor{currentstroke}%
\pgfsetdash{}{0pt}%
\pgfpathmoveto{\pgfqpoint{3.763398in}{2.297457in}}%
\pgfpathlineto{\pgfqpoint{4.018447in}{2.468167in}}%
\pgfpathlineto{\pgfqpoint{3.763398in}{2.297457in}}%
\pgfpathclose%
\pgfusepath{fill}%
\end{pgfscope}%
\begin{pgfscope}%
\pgfpathrectangle{\pgfqpoint{0.017500in}{0.017500in}}{\pgfqpoint{6.218282in}{3.580372in}} %
\pgfusepath{clip}%
\pgfsetbuttcap%
\pgfsetroundjoin%
\definecolor{currentfill}{rgb}{0.810616,0.268797,0.235428}%
\pgfsetfillcolor{currentfill}%
\pgfsetlinewidth{0.000000pt}%
\definecolor{currentstroke}{rgb}{0.000000,0.000000,0.000000}%
\pgfsetstrokecolor{currentstroke}%
\pgfsetdash{}{0pt}%
\pgfpathmoveto{\pgfqpoint{3.763398in}{2.297457in}}%
\pgfpathlineto{\pgfqpoint{4.018447in}{2.468167in}}%
\pgfpathlineto{\pgfqpoint{3.763398in}{2.297457in}}%
\pgfpathclose%
\pgfusepath{fill}%
\end{pgfscope}%
\begin{pgfscope}%
\pgfpathrectangle{\pgfqpoint{0.017500in}{0.017500in}}{\pgfqpoint{6.218282in}{3.580372in}} %
\pgfusepath{clip}%
\pgfsetbuttcap%
\pgfsetroundjoin%
\definecolor{currentfill}{rgb}{0.810616,0.268797,0.235428}%
\pgfsetfillcolor{currentfill}%
\pgfsetlinewidth{0.000000pt}%
\definecolor{currentstroke}{rgb}{0.000000,0.000000,0.000000}%
\pgfsetstrokecolor{currentstroke}%
\pgfsetdash{}{0pt}%
\pgfpathmoveto{\pgfqpoint{3.763398in}{2.297457in}}%
\pgfpathlineto{\pgfqpoint{4.018447in}{2.468167in}}%
\pgfpathlineto{\pgfqpoint{3.763398in}{2.297457in}}%
\pgfpathclose%
\pgfusepath{fill}%
\end{pgfscope}%
\begin{pgfscope}%
\pgfpathrectangle{\pgfqpoint{0.017500in}{0.017500in}}{\pgfqpoint{6.218282in}{3.580372in}} %
\pgfusepath{clip}%
\pgfsetbuttcap%
\pgfsetroundjoin%
\definecolor{currentfill}{rgb}{0.810616,0.268797,0.235428}%
\pgfsetfillcolor{currentfill}%
\pgfsetlinewidth{0.000000pt}%
\definecolor{currentstroke}{rgb}{0.000000,0.000000,0.000000}%
\pgfsetstrokecolor{currentstroke}%
\pgfsetdash{}{0pt}%
\pgfpathmoveto{\pgfqpoint{3.763398in}{2.297457in}}%
\pgfpathlineto{\pgfqpoint{4.018447in}{2.468167in}}%
\pgfpathlineto{\pgfqpoint{3.763398in}{2.297457in}}%
\pgfpathclose%
\pgfusepath{fill}%
\end{pgfscope}%
\begin{pgfscope}%
\pgfpathrectangle{\pgfqpoint{0.017500in}{0.017500in}}{\pgfqpoint{6.218282in}{3.580372in}} %
\pgfusepath{clip}%
\pgfsetbuttcap%
\pgfsetroundjoin%
\definecolor{currentfill}{rgb}{0.810616,0.268797,0.235428}%
\pgfsetfillcolor{currentfill}%
\pgfsetlinewidth{0.000000pt}%
\definecolor{currentstroke}{rgb}{0.000000,0.000000,0.000000}%
\pgfsetstrokecolor{currentstroke}%
\pgfsetdash{}{0pt}%
\pgfpathmoveto{\pgfqpoint{3.763398in}{2.297457in}}%
\pgfpathlineto{\pgfqpoint{4.018447in}{2.468167in}}%
\pgfpathlineto{\pgfqpoint{3.763398in}{2.297457in}}%
\pgfpathclose%
\pgfusepath{fill}%
\end{pgfscope}%
\begin{pgfscope}%
\pgfpathrectangle{\pgfqpoint{0.017500in}{0.017500in}}{\pgfqpoint{6.218282in}{3.580372in}} %
\pgfusepath{clip}%
\pgfsetbuttcap%
\pgfsetroundjoin%
\definecolor{currentfill}{rgb}{0.810616,0.268797,0.235428}%
\pgfsetfillcolor{currentfill}%
\pgfsetlinewidth{0.000000pt}%
\definecolor{currentstroke}{rgb}{0.000000,0.000000,0.000000}%
\pgfsetstrokecolor{currentstroke}%
\pgfsetdash{}{0pt}%
\pgfpathmoveto{\pgfqpoint{3.763398in}{2.297457in}}%
\pgfpathlineto{\pgfqpoint{4.018447in}{2.468167in}}%
\pgfpathlineto{\pgfqpoint{3.763398in}{2.297457in}}%
\pgfpathclose%
\pgfusepath{fill}%
\end{pgfscope}%
\begin{pgfscope}%
\pgfpathrectangle{\pgfqpoint{0.017500in}{0.017500in}}{\pgfqpoint{6.218282in}{3.580372in}} %
\pgfusepath{clip}%
\pgfsetbuttcap%
\pgfsetroundjoin%
\definecolor{currentfill}{rgb}{0.810616,0.268797,0.235428}%
\pgfsetfillcolor{currentfill}%
\pgfsetlinewidth{0.000000pt}%
\definecolor{currentstroke}{rgb}{0.000000,0.000000,0.000000}%
\pgfsetstrokecolor{currentstroke}%
\pgfsetdash{}{0pt}%
\pgfpathmoveto{\pgfqpoint{3.763398in}{2.297457in}}%
\pgfpathlineto{\pgfqpoint{4.018447in}{2.468167in}}%
\pgfpathlineto{\pgfqpoint{3.763398in}{2.297457in}}%
\pgfpathclose%
\pgfusepath{fill}%
\end{pgfscope}%
\begin{pgfscope}%
\pgfpathrectangle{\pgfqpoint{0.017500in}{0.017500in}}{\pgfqpoint{6.218282in}{3.580372in}} %
\pgfusepath{clip}%
\pgfsetbuttcap%
\pgfsetroundjoin%
\definecolor{currentfill}{rgb}{0.810616,0.268797,0.235428}%
\pgfsetfillcolor{currentfill}%
\pgfsetlinewidth{0.000000pt}%
\definecolor{currentstroke}{rgb}{0.000000,0.000000,0.000000}%
\pgfsetstrokecolor{currentstroke}%
\pgfsetdash{}{0pt}%
\pgfpathmoveto{\pgfqpoint{3.763398in}{2.297457in}}%
\pgfpathlineto{\pgfqpoint{4.018447in}{2.468167in}}%
\pgfpathlineto{\pgfqpoint{3.763398in}{2.297457in}}%
\pgfpathclose%
\pgfusepath{fill}%
\end{pgfscope}%
\begin{pgfscope}%
\pgfpathrectangle{\pgfqpoint{0.017500in}{0.017500in}}{\pgfqpoint{6.218282in}{3.580372in}} %
\pgfusepath{clip}%
\pgfsetbuttcap%
\pgfsetroundjoin%
\definecolor{currentfill}{rgb}{0.810616,0.268797,0.235428}%
\pgfsetfillcolor{currentfill}%
\pgfsetlinewidth{0.000000pt}%
\definecolor{currentstroke}{rgb}{0.000000,0.000000,0.000000}%
\pgfsetstrokecolor{currentstroke}%
\pgfsetdash{}{0pt}%
\pgfpathmoveto{\pgfqpoint{3.763398in}{2.297457in}}%
\pgfpathlineto{\pgfqpoint{4.018447in}{2.468167in}}%
\pgfpathlineto{\pgfqpoint{3.763398in}{2.297457in}}%
\pgfpathclose%
\pgfusepath{fill}%
\end{pgfscope}%
\begin{pgfscope}%
\pgfpathrectangle{\pgfqpoint{0.017500in}{0.017500in}}{\pgfqpoint{6.218282in}{3.580372in}} %
\pgfusepath{clip}%
\pgfsetbuttcap%
\pgfsetroundjoin%
\definecolor{currentfill}{rgb}{0.810616,0.268797,0.235428}%
\pgfsetfillcolor{currentfill}%
\pgfsetlinewidth{0.000000pt}%
\definecolor{currentstroke}{rgb}{0.000000,0.000000,0.000000}%
\pgfsetstrokecolor{currentstroke}%
\pgfsetdash{}{0pt}%
\pgfpathmoveto{\pgfqpoint{3.763398in}{2.297457in}}%
\pgfpathlineto{\pgfqpoint{4.018447in}{2.468167in}}%
\pgfpathlineto{\pgfqpoint{3.763398in}{2.297457in}}%
\pgfpathclose%
\pgfusepath{fill}%
\end{pgfscope}%
\begin{pgfscope}%
\pgfpathrectangle{\pgfqpoint{0.017500in}{0.017500in}}{\pgfqpoint{6.218282in}{3.580372in}} %
\pgfusepath{clip}%
\pgfsetbuttcap%
\pgfsetroundjoin%
\definecolor{currentfill}{rgb}{0.810616,0.268797,0.235428}%
\pgfsetfillcolor{currentfill}%
\pgfsetlinewidth{0.000000pt}%
\definecolor{currentstroke}{rgb}{0.000000,0.000000,0.000000}%
\pgfsetstrokecolor{currentstroke}%
\pgfsetdash{}{0pt}%
\pgfpathmoveto{\pgfqpoint{3.763398in}{2.297457in}}%
\pgfpathlineto{\pgfqpoint{4.018447in}{2.468167in}}%
\pgfpathlineto{\pgfqpoint{3.763398in}{2.297457in}}%
\pgfpathclose%
\pgfusepath{fill}%
\end{pgfscope}%
\begin{pgfscope}%
\pgfpathrectangle{\pgfqpoint{0.017500in}{0.017500in}}{\pgfqpoint{6.218282in}{3.580372in}} %
\pgfusepath{clip}%
\pgfsetbuttcap%
\pgfsetroundjoin%
\definecolor{currentfill}{rgb}{0.810616,0.268797,0.235428}%
\pgfsetfillcolor{currentfill}%
\pgfsetlinewidth{0.000000pt}%
\definecolor{currentstroke}{rgb}{0.000000,0.000000,0.000000}%
\pgfsetstrokecolor{currentstroke}%
\pgfsetdash{}{0pt}%
\pgfpathmoveto{\pgfqpoint{3.763398in}{2.297457in}}%
\pgfpathlineto{\pgfqpoint{4.018447in}{2.468167in}}%
\pgfpathlineto{\pgfqpoint{3.763398in}{2.297457in}}%
\pgfpathclose%
\pgfusepath{fill}%
\end{pgfscope}%
\begin{pgfscope}%
\pgfpathrectangle{\pgfqpoint{0.017500in}{0.017500in}}{\pgfqpoint{6.218282in}{3.580372in}} %
\pgfusepath{clip}%
\pgfsetbuttcap%
\pgfsetroundjoin%
\definecolor{currentfill}{rgb}{0.810616,0.268797,0.235428}%
\pgfsetfillcolor{currentfill}%
\pgfsetlinewidth{0.000000pt}%
\definecolor{currentstroke}{rgb}{0.000000,0.000000,0.000000}%
\pgfsetstrokecolor{currentstroke}%
\pgfsetdash{}{0pt}%
\pgfpathmoveto{\pgfqpoint{3.763398in}{2.297457in}}%
\pgfpathlineto{\pgfqpoint{4.018447in}{2.468167in}}%
\pgfpathlineto{\pgfqpoint{3.763398in}{2.297457in}}%
\pgfpathclose%
\pgfusepath{fill}%
\end{pgfscope}%
\begin{pgfscope}%
\pgfpathrectangle{\pgfqpoint{0.017500in}{0.017500in}}{\pgfqpoint{6.218282in}{3.580372in}} %
\pgfusepath{clip}%
\pgfsetbuttcap%
\pgfsetroundjoin%
\definecolor{currentfill}{rgb}{0.810616,0.268797,0.235428}%
\pgfsetfillcolor{currentfill}%
\pgfsetlinewidth{0.000000pt}%
\definecolor{currentstroke}{rgb}{0.000000,0.000000,0.000000}%
\pgfsetstrokecolor{currentstroke}%
\pgfsetdash{}{0pt}%
\pgfpathmoveto{\pgfqpoint{3.763398in}{2.297457in}}%
\pgfpathlineto{\pgfqpoint{4.018447in}{2.468167in}}%
\pgfpathlineto{\pgfqpoint{3.763398in}{2.297457in}}%
\pgfpathclose%
\pgfusepath{fill}%
\end{pgfscope}%
\begin{pgfscope}%
\pgfpathrectangle{\pgfqpoint{0.017500in}{0.017500in}}{\pgfqpoint{6.218282in}{3.580372in}} %
\pgfusepath{clip}%
\pgfsetbuttcap%
\pgfsetroundjoin%
\definecolor{currentfill}{rgb}{0.810616,0.268797,0.235428}%
\pgfsetfillcolor{currentfill}%
\pgfsetlinewidth{0.000000pt}%
\definecolor{currentstroke}{rgb}{0.000000,0.000000,0.000000}%
\pgfsetstrokecolor{currentstroke}%
\pgfsetdash{}{0pt}%
\pgfpathmoveto{\pgfqpoint{3.763398in}{2.297457in}}%
\pgfpathlineto{\pgfqpoint{4.018447in}{2.468167in}}%
\pgfpathlineto{\pgfqpoint{3.763398in}{2.297457in}}%
\pgfpathclose%
\pgfusepath{fill}%
\end{pgfscope}%
\begin{pgfscope}%
\pgfpathrectangle{\pgfqpoint{0.017500in}{0.017500in}}{\pgfqpoint{6.218282in}{3.580372in}} %
\pgfusepath{clip}%
\pgfsetbuttcap%
\pgfsetroundjoin%
\definecolor{currentfill}{rgb}{0.810616,0.268797,0.235428}%
\pgfsetfillcolor{currentfill}%
\pgfsetlinewidth{0.000000pt}%
\definecolor{currentstroke}{rgb}{0.000000,0.000000,0.000000}%
\pgfsetstrokecolor{currentstroke}%
\pgfsetdash{}{0pt}%
\pgfpathmoveto{\pgfqpoint{3.763398in}{2.297457in}}%
\pgfpathlineto{\pgfqpoint{4.018447in}{2.468167in}}%
\pgfpathlineto{\pgfqpoint{3.763398in}{2.297457in}}%
\pgfpathclose%
\pgfusepath{fill}%
\end{pgfscope}%
\begin{pgfscope}%
\pgfpathrectangle{\pgfqpoint{0.017500in}{0.017500in}}{\pgfqpoint{6.218282in}{3.580372in}} %
\pgfusepath{clip}%
\pgfsetbuttcap%
\pgfsetroundjoin%
\definecolor{currentfill}{rgb}{0.810616,0.268797,0.235428}%
\pgfsetfillcolor{currentfill}%
\pgfsetlinewidth{0.000000pt}%
\definecolor{currentstroke}{rgb}{0.000000,0.000000,0.000000}%
\pgfsetstrokecolor{currentstroke}%
\pgfsetdash{}{0pt}%
\pgfpathmoveto{\pgfqpoint{3.763398in}{2.297457in}}%
\pgfpathlineto{\pgfqpoint{4.018447in}{2.468167in}}%
\pgfpathlineto{\pgfqpoint{3.763398in}{2.297457in}}%
\pgfpathclose%
\pgfusepath{fill}%
\end{pgfscope}%
\begin{pgfscope}%
\pgfpathrectangle{\pgfqpoint{0.017500in}{0.017500in}}{\pgfqpoint{6.218282in}{3.580372in}} %
\pgfusepath{clip}%
\pgfsetbuttcap%
\pgfsetroundjoin%
\definecolor{currentfill}{rgb}{0.810616,0.268797,0.235428}%
\pgfsetfillcolor{currentfill}%
\pgfsetlinewidth{0.000000pt}%
\definecolor{currentstroke}{rgb}{0.000000,0.000000,0.000000}%
\pgfsetstrokecolor{currentstroke}%
\pgfsetdash{}{0pt}%
\pgfpathmoveto{\pgfqpoint{3.763398in}{2.297457in}}%
\pgfpathlineto{\pgfqpoint{4.018447in}{2.468167in}}%
\pgfpathlineto{\pgfqpoint{3.763398in}{2.297457in}}%
\pgfpathclose%
\pgfusepath{fill}%
\end{pgfscope}%
\begin{pgfscope}%
\pgfpathrectangle{\pgfqpoint{0.017500in}{0.017500in}}{\pgfqpoint{6.218282in}{3.580372in}} %
\pgfusepath{clip}%
\pgfsetbuttcap%
\pgfsetroundjoin%
\definecolor{currentfill}{rgb}{0.810616,0.268797,0.235428}%
\pgfsetfillcolor{currentfill}%
\pgfsetlinewidth{0.000000pt}%
\definecolor{currentstroke}{rgb}{0.000000,0.000000,0.000000}%
\pgfsetstrokecolor{currentstroke}%
\pgfsetdash{}{0pt}%
\pgfpathmoveto{\pgfqpoint{3.763398in}{2.297457in}}%
\pgfpathlineto{\pgfqpoint{4.018447in}{2.468167in}}%
\pgfpathlineto{\pgfqpoint{3.763398in}{2.297457in}}%
\pgfpathclose%
\pgfusepath{fill}%
\end{pgfscope}%
\begin{pgfscope}%
\pgfpathrectangle{\pgfqpoint{0.017500in}{0.017500in}}{\pgfqpoint{6.218282in}{3.580372in}} %
\pgfusepath{clip}%
\pgfsetbuttcap%
\pgfsetroundjoin%
\definecolor{currentfill}{rgb}{0.810616,0.268797,0.235428}%
\pgfsetfillcolor{currentfill}%
\pgfsetlinewidth{0.000000pt}%
\definecolor{currentstroke}{rgb}{0.000000,0.000000,0.000000}%
\pgfsetstrokecolor{currentstroke}%
\pgfsetdash{}{0pt}%
\pgfpathmoveto{\pgfqpoint{3.763398in}{2.297457in}}%
\pgfpathlineto{\pgfqpoint{4.018447in}{2.468167in}}%
\pgfpathlineto{\pgfqpoint{3.763398in}{2.297457in}}%
\pgfpathclose%
\pgfusepath{fill}%
\end{pgfscope}%
\begin{pgfscope}%
\pgfpathrectangle{\pgfqpoint{0.017500in}{0.017500in}}{\pgfqpoint{6.218282in}{3.580372in}} %
\pgfusepath{clip}%
\pgfsetbuttcap%
\pgfsetroundjoin%
\definecolor{currentfill}{rgb}{0.810616,0.268797,0.235428}%
\pgfsetfillcolor{currentfill}%
\pgfsetlinewidth{0.000000pt}%
\definecolor{currentstroke}{rgb}{0.000000,0.000000,0.000000}%
\pgfsetstrokecolor{currentstroke}%
\pgfsetdash{}{0pt}%
\pgfpathmoveto{\pgfqpoint{3.763398in}{2.297457in}}%
\pgfpathlineto{\pgfqpoint{4.018447in}{2.468167in}}%
\pgfpathlineto{\pgfqpoint{3.763398in}{2.297457in}}%
\pgfpathclose%
\pgfusepath{fill}%
\end{pgfscope}%
\begin{pgfscope}%
\pgfpathrectangle{\pgfqpoint{0.017500in}{0.017500in}}{\pgfqpoint{6.218282in}{3.580372in}} %
\pgfusepath{clip}%
\pgfsetbuttcap%
\pgfsetroundjoin%
\definecolor{currentfill}{rgb}{0.810616,0.268797,0.235428}%
\pgfsetfillcolor{currentfill}%
\pgfsetlinewidth{0.000000pt}%
\definecolor{currentstroke}{rgb}{0.000000,0.000000,0.000000}%
\pgfsetstrokecolor{currentstroke}%
\pgfsetdash{}{0pt}%
\pgfpathmoveto{\pgfqpoint{3.763398in}{2.297457in}}%
\pgfpathlineto{\pgfqpoint{4.018447in}{2.468167in}}%
\pgfpathlineto{\pgfqpoint{3.763398in}{2.297457in}}%
\pgfpathclose%
\pgfusepath{fill}%
\end{pgfscope}%
\begin{pgfscope}%
\pgfpathrectangle{\pgfqpoint{0.017500in}{0.017500in}}{\pgfqpoint{6.218282in}{3.580372in}} %
\pgfusepath{clip}%
\pgfsetbuttcap%
\pgfsetroundjoin%
\definecolor{currentfill}{rgb}{0.810616,0.268797,0.235428}%
\pgfsetfillcolor{currentfill}%
\pgfsetlinewidth{0.000000pt}%
\definecolor{currentstroke}{rgb}{0.000000,0.000000,0.000000}%
\pgfsetstrokecolor{currentstroke}%
\pgfsetdash{}{0pt}%
\pgfpathmoveto{\pgfqpoint{3.763398in}{2.297457in}}%
\pgfpathlineto{\pgfqpoint{4.018447in}{2.468167in}}%
\pgfpathlineto{\pgfqpoint{3.763398in}{2.297457in}}%
\pgfpathclose%
\pgfusepath{fill}%
\end{pgfscope}%
\begin{pgfscope}%
\pgfpathrectangle{\pgfqpoint{0.017500in}{0.017500in}}{\pgfqpoint{6.218282in}{3.580372in}} %
\pgfusepath{clip}%
\pgfsetbuttcap%
\pgfsetroundjoin%
\definecolor{currentfill}{rgb}{0.810616,0.268797,0.235428}%
\pgfsetfillcolor{currentfill}%
\pgfsetlinewidth{0.000000pt}%
\definecolor{currentstroke}{rgb}{0.000000,0.000000,0.000000}%
\pgfsetstrokecolor{currentstroke}%
\pgfsetdash{}{0pt}%
\pgfpathmoveto{\pgfqpoint{3.763398in}{2.297457in}}%
\pgfpathlineto{\pgfqpoint{4.018447in}{2.468167in}}%
\pgfpathlineto{\pgfqpoint{3.763398in}{2.297457in}}%
\pgfpathclose%
\pgfusepath{fill}%
\end{pgfscope}%
\begin{pgfscope}%
\pgfpathrectangle{\pgfqpoint{0.017500in}{0.017500in}}{\pgfqpoint{6.218282in}{3.580372in}} %
\pgfusepath{clip}%
\pgfsetbuttcap%
\pgfsetroundjoin%
\definecolor{currentfill}{rgb}{0.810616,0.268797,0.235428}%
\pgfsetfillcolor{currentfill}%
\pgfsetlinewidth{0.000000pt}%
\definecolor{currentstroke}{rgb}{0.000000,0.000000,0.000000}%
\pgfsetstrokecolor{currentstroke}%
\pgfsetdash{}{0pt}%
\pgfpathmoveto{\pgfqpoint{3.763398in}{2.297457in}}%
\pgfpathlineto{\pgfqpoint{4.018447in}{2.468167in}}%
\pgfpathlineto{\pgfqpoint{3.763398in}{2.297457in}}%
\pgfpathclose%
\pgfusepath{fill}%
\end{pgfscope}%
\begin{pgfscope}%
\pgfpathrectangle{\pgfqpoint{0.017500in}{0.017500in}}{\pgfqpoint{6.218282in}{3.580372in}} %
\pgfusepath{clip}%
\pgfsetbuttcap%
\pgfsetroundjoin%
\definecolor{currentfill}{rgb}{0.810616,0.268797,0.235428}%
\pgfsetfillcolor{currentfill}%
\pgfsetlinewidth{0.000000pt}%
\definecolor{currentstroke}{rgb}{0.000000,0.000000,0.000000}%
\pgfsetstrokecolor{currentstroke}%
\pgfsetdash{}{0pt}%
\pgfpathmoveto{\pgfqpoint{3.763398in}{2.297457in}}%
\pgfpathlineto{\pgfqpoint{4.018447in}{2.468167in}}%
\pgfpathlineto{\pgfqpoint{3.763398in}{2.297457in}}%
\pgfpathclose%
\pgfusepath{fill}%
\end{pgfscope}%
\begin{pgfscope}%
\pgfpathrectangle{\pgfqpoint{0.017500in}{0.017500in}}{\pgfqpoint{6.218282in}{3.580372in}} %
\pgfusepath{clip}%
\pgfsetbuttcap%
\pgfsetroundjoin%
\definecolor{currentfill}{rgb}{0.810616,0.268797,0.235428}%
\pgfsetfillcolor{currentfill}%
\pgfsetlinewidth{0.000000pt}%
\definecolor{currentstroke}{rgb}{0.000000,0.000000,0.000000}%
\pgfsetstrokecolor{currentstroke}%
\pgfsetdash{}{0pt}%
\pgfpathmoveto{\pgfqpoint{3.763398in}{2.297457in}}%
\pgfpathlineto{\pgfqpoint{4.018447in}{2.468167in}}%
\pgfpathlineto{\pgfqpoint{3.763398in}{2.297457in}}%
\pgfpathclose%
\pgfusepath{fill}%
\end{pgfscope}%
\begin{pgfscope}%
\pgfpathrectangle{\pgfqpoint{0.017500in}{0.017500in}}{\pgfqpoint{6.218282in}{3.580372in}} %
\pgfusepath{clip}%
\pgfsetbuttcap%
\pgfsetroundjoin%
\definecolor{currentfill}{rgb}{0.338377,0.452819,0.879317}%
\pgfsetfillcolor{currentfill}%
\pgfsetlinewidth{0.000000pt}%
\definecolor{currentstroke}{rgb}{0.000000,0.000000,0.000000}%
\pgfsetstrokecolor{currentstroke}%
\pgfsetdash{}{0pt}%
\pgfpathmoveto{\pgfqpoint{3.236099in}{0.723497in}}%
\pgfpathlineto{\pgfqpoint{3.688351in}{0.641693in}}%
\pgfpathlineto{\pgfqpoint{3.968076in}{1.994042in}}%
\pgfpathlineto{\pgfqpoint{3.502561in}{2.073396in}}%
\pgfpathlineto{\pgfqpoint{3.236099in}{0.723497in}}%
\pgfpathclose%
\pgfusepath{fill}%
\end{pgfscope}%
\begin{pgfscope}%
\pgfpathrectangle{\pgfqpoint{0.017500in}{0.017500in}}{\pgfqpoint{6.218282in}{3.580372in}} %
\pgfusepath{clip}%
\pgfsetbuttcap%
\pgfsetroundjoin%
\definecolor{currentfill}{rgb}{0.338377,0.452819,0.879317}%
\pgfsetfillcolor{currentfill}%
\pgfsetlinewidth{0.000000pt}%
\definecolor{currentstroke}{rgb}{0.000000,0.000000,0.000000}%
\pgfsetstrokecolor{currentstroke}%
\pgfsetdash{}{0pt}%
\pgfpathmoveto{\pgfqpoint{3.236099in}{0.723497in}}%
\pgfpathlineto{\pgfqpoint{3.688351in}{0.641693in}}%
\pgfpathlineto{\pgfqpoint{3.968076in}{1.994042in}}%
\pgfpathlineto{\pgfqpoint{3.502561in}{2.073396in}}%
\pgfpathlineto{\pgfqpoint{3.236099in}{0.723497in}}%
\pgfpathclose%
\pgfusepath{fill}%
\end{pgfscope}%
\begin{pgfscope}%
\pgfpathrectangle{\pgfqpoint{0.017500in}{0.017500in}}{\pgfqpoint{6.218282in}{3.580372in}} %
\pgfusepath{clip}%
\pgfsetbuttcap%
\pgfsetroundjoin%
\definecolor{currentfill}{rgb}{0.338377,0.452819,0.879317}%
\pgfsetfillcolor{currentfill}%
\pgfsetlinewidth{0.000000pt}%
\definecolor{currentstroke}{rgb}{0.000000,0.000000,0.000000}%
\pgfsetstrokecolor{currentstroke}%
\pgfsetdash{}{0pt}%
\pgfpathmoveto{\pgfqpoint{3.236099in}{0.723497in}}%
\pgfpathlineto{\pgfqpoint{3.688351in}{0.641693in}}%
\pgfpathlineto{\pgfqpoint{3.968076in}{1.994042in}}%
\pgfpathlineto{\pgfqpoint{3.502561in}{2.073396in}}%
\pgfpathlineto{\pgfqpoint{3.236099in}{0.723497in}}%
\pgfpathclose%
\pgfusepath{fill}%
\end{pgfscope}%
\begin{pgfscope}%
\pgfpathrectangle{\pgfqpoint{0.017500in}{0.017500in}}{\pgfqpoint{6.218282in}{3.580372in}} %
\pgfusepath{clip}%
\pgfsetbuttcap%
\pgfsetroundjoin%
\definecolor{currentfill}{rgb}{0.338377,0.452819,0.879317}%
\pgfsetfillcolor{currentfill}%
\pgfsetlinewidth{0.000000pt}%
\definecolor{currentstroke}{rgb}{0.000000,0.000000,0.000000}%
\pgfsetstrokecolor{currentstroke}%
\pgfsetdash{}{0pt}%
\pgfpathmoveto{\pgfqpoint{3.236099in}{0.723497in}}%
\pgfpathlineto{\pgfqpoint{3.688351in}{0.641693in}}%
\pgfpathlineto{\pgfqpoint{3.968076in}{1.994042in}}%
\pgfpathlineto{\pgfqpoint{3.502561in}{2.073396in}}%
\pgfpathlineto{\pgfqpoint{3.236099in}{0.723497in}}%
\pgfpathclose%
\pgfusepath{fill}%
\end{pgfscope}%
\begin{pgfscope}%
\pgfpathrectangle{\pgfqpoint{0.017500in}{0.017500in}}{\pgfqpoint{6.218282in}{3.580372in}} %
\pgfusepath{clip}%
\pgfsetbuttcap%
\pgfsetroundjoin%
\definecolor{currentfill}{rgb}{0.338377,0.452819,0.879317}%
\pgfsetfillcolor{currentfill}%
\pgfsetlinewidth{0.000000pt}%
\definecolor{currentstroke}{rgb}{0.000000,0.000000,0.000000}%
\pgfsetstrokecolor{currentstroke}%
\pgfsetdash{}{0pt}%
\pgfpathmoveto{\pgfqpoint{3.236099in}{0.723497in}}%
\pgfpathlineto{\pgfqpoint{3.688351in}{0.641693in}}%
\pgfpathlineto{\pgfqpoint{3.968076in}{1.994042in}}%
\pgfpathlineto{\pgfqpoint{3.502561in}{2.073396in}}%
\pgfpathlineto{\pgfqpoint{3.236099in}{0.723497in}}%
\pgfpathclose%
\pgfusepath{fill}%
\end{pgfscope}%
\begin{pgfscope}%
\pgfpathrectangle{\pgfqpoint{0.017500in}{0.017500in}}{\pgfqpoint{6.218282in}{3.580372in}} %
\pgfusepath{clip}%
\pgfsetbuttcap%
\pgfsetroundjoin%
\definecolor{currentfill}{rgb}{0.338377,0.452819,0.879317}%
\pgfsetfillcolor{currentfill}%
\pgfsetlinewidth{0.000000pt}%
\definecolor{currentstroke}{rgb}{0.000000,0.000000,0.000000}%
\pgfsetstrokecolor{currentstroke}%
\pgfsetdash{}{0pt}%
\pgfpathmoveto{\pgfqpoint{3.236099in}{0.723497in}}%
\pgfpathlineto{\pgfqpoint{3.688351in}{0.641693in}}%
\pgfpathlineto{\pgfqpoint{3.968076in}{1.994042in}}%
\pgfpathlineto{\pgfqpoint{3.502561in}{2.073396in}}%
\pgfpathlineto{\pgfqpoint{3.236099in}{0.723497in}}%
\pgfpathclose%
\pgfusepath{fill}%
\end{pgfscope}%
\begin{pgfscope}%
\pgfpathrectangle{\pgfqpoint{0.017500in}{0.017500in}}{\pgfqpoint{6.218282in}{3.580372in}} %
\pgfusepath{clip}%
\pgfsetbuttcap%
\pgfsetroundjoin%
\definecolor{currentfill}{rgb}{0.805723,0.259813,0.230562}%
\pgfsetfillcolor{currentfill}%
\pgfsetlinewidth{0.000000pt}%
\definecolor{currentstroke}{rgb}{0.000000,0.000000,0.000000}%
\pgfsetstrokecolor{currentstroke}%
\pgfsetdash{}{0pt}%
\pgfpathmoveto{\pgfqpoint{3.763398in}{2.297457in}}%
\pgfpathlineto{\pgfqpoint{4.226975in}{2.236848in}}%
\pgfpathlineto{\pgfqpoint{4.478468in}{2.393972in}}%
\pgfpathlineto{\pgfqpoint{4.018447in}{2.468167in}}%
\pgfpathlineto{\pgfqpoint{3.763398in}{2.297457in}}%
\pgfpathclose%
\pgfusepath{fill}%
\end{pgfscope}%
\begin{pgfscope}%
\pgfpathrectangle{\pgfqpoint{0.017500in}{0.017500in}}{\pgfqpoint{6.218282in}{3.580372in}} %
\pgfusepath{clip}%
\pgfsetbuttcap%
\pgfsetroundjoin%
\definecolor{currentfill}{rgb}{0.805723,0.259813,0.230562}%
\pgfsetfillcolor{currentfill}%
\pgfsetlinewidth{0.000000pt}%
\definecolor{currentstroke}{rgb}{0.000000,0.000000,0.000000}%
\pgfsetstrokecolor{currentstroke}%
\pgfsetdash{}{0pt}%
\pgfpathmoveto{\pgfqpoint{3.763398in}{2.297457in}}%
\pgfpathlineto{\pgfqpoint{4.226975in}{2.236848in}}%
\pgfpathlineto{\pgfqpoint{4.478468in}{2.393972in}}%
\pgfpathlineto{\pgfqpoint{4.018447in}{2.468167in}}%
\pgfpathlineto{\pgfqpoint{3.763398in}{2.297457in}}%
\pgfpathclose%
\pgfusepath{fill}%
\end{pgfscope}%
\begin{pgfscope}%
\pgfpathrectangle{\pgfqpoint{0.017500in}{0.017500in}}{\pgfqpoint{6.218282in}{3.580372in}} %
\pgfusepath{clip}%
\pgfsetbuttcap%
\pgfsetroundjoin%
\definecolor{currentfill}{rgb}{0.805723,0.259813,0.230562}%
\pgfsetfillcolor{currentfill}%
\pgfsetlinewidth{0.000000pt}%
\definecolor{currentstroke}{rgb}{0.000000,0.000000,0.000000}%
\pgfsetstrokecolor{currentstroke}%
\pgfsetdash{}{0pt}%
\pgfpathmoveto{\pgfqpoint{3.763398in}{2.297457in}}%
\pgfpathlineto{\pgfqpoint{4.226975in}{2.236848in}}%
\pgfpathlineto{\pgfqpoint{4.478468in}{2.393972in}}%
\pgfpathlineto{\pgfqpoint{4.018447in}{2.468167in}}%
\pgfpathlineto{\pgfqpoint{3.763398in}{2.297457in}}%
\pgfpathclose%
\pgfusepath{fill}%
\end{pgfscope}%
\begin{pgfscope}%
\pgfpathrectangle{\pgfqpoint{0.017500in}{0.017500in}}{\pgfqpoint{6.218282in}{3.580372in}} %
\pgfusepath{clip}%
\pgfsetbuttcap%
\pgfsetroundjoin%
\definecolor{currentfill}{rgb}{0.805723,0.259813,0.230562}%
\pgfsetfillcolor{currentfill}%
\pgfsetlinewidth{0.000000pt}%
\definecolor{currentstroke}{rgb}{0.000000,0.000000,0.000000}%
\pgfsetstrokecolor{currentstroke}%
\pgfsetdash{}{0pt}%
\pgfpathmoveto{\pgfqpoint{3.763398in}{2.297457in}}%
\pgfpathlineto{\pgfqpoint{4.226975in}{2.236848in}}%
\pgfpathlineto{\pgfqpoint{4.478468in}{2.393972in}}%
\pgfpathlineto{\pgfqpoint{4.018447in}{2.468167in}}%
\pgfpathlineto{\pgfqpoint{3.763398in}{2.297457in}}%
\pgfpathclose%
\pgfusepath{fill}%
\end{pgfscope}%
\begin{pgfscope}%
\pgfpathrectangle{\pgfqpoint{0.017500in}{0.017500in}}{\pgfqpoint{6.218282in}{3.580372in}} %
\pgfusepath{clip}%
\pgfsetbuttcap%
\pgfsetroundjoin%
\definecolor{currentfill}{rgb}{0.805723,0.259813,0.230562}%
\pgfsetfillcolor{currentfill}%
\pgfsetlinewidth{0.000000pt}%
\definecolor{currentstroke}{rgb}{0.000000,0.000000,0.000000}%
\pgfsetstrokecolor{currentstroke}%
\pgfsetdash{}{0pt}%
\pgfpathmoveto{\pgfqpoint{3.763398in}{2.297457in}}%
\pgfpathlineto{\pgfqpoint{4.226975in}{2.236848in}}%
\pgfpathlineto{\pgfqpoint{4.478468in}{2.393972in}}%
\pgfpathlineto{\pgfqpoint{4.018447in}{2.468167in}}%
\pgfpathlineto{\pgfqpoint{3.763398in}{2.297457in}}%
\pgfpathclose%
\pgfusepath{fill}%
\end{pgfscope}%
\begin{pgfscope}%
\pgfpathrectangle{\pgfqpoint{0.017500in}{0.017500in}}{\pgfqpoint{6.218282in}{3.580372in}} %
\pgfusepath{clip}%
\pgfsetbuttcap%
\pgfsetroundjoin%
\definecolor{currentfill}{rgb}{0.805723,0.259813,0.230562}%
\pgfsetfillcolor{currentfill}%
\pgfsetlinewidth{0.000000pt}%
\definecolor{currentstroke}{rgb}{0.000000,0.000000,0.000000}%
\pgfsetstrokecolor{currentstroke}%
\pgfsetdash{}{0pt}%
\pgfpathmoveto{\pgfqpoint{3.763398in}{2.297457in}}%
\pgfpathlineto{\pgfqpoint{4.226975in}{2.236848in}}%
\pgfpathlineto{\pgfqpoint{4.478468in}{2.393972in}}%
\pgfpathlineto{\pgfqpoint{4.018447in}{2.468167in}}%
\pgfpathlineto{\pgfqpoint{3.763398in}{2.297457in}}%
\pgfpathclose%
\pgfusepath{fill}%
\end{pgfscope}%
\begin{pgfscope}%
\pgfpathrectangle{\pgfqpoint{0.017500in}{0.017500in}}{\pgfqpoint{6.218282in}{3.580372in}} %
\pgfusepath{clip}%
\pgfsetbuttcap%
\pgfsetroundjoin%
\definecolor{currentfill}{rgb}{0.243520,0.319189,0.771672}%
\pgfsetfillcolor{currentfill}%
\pgfsetlinewidth{0.000000pt}%
\definecolor{currentstroke}{rgb}{0.000000,0.000000,0.000000}%
\pgfsetstrokecolor{currentstroke}%
\pgfsetdash{}{0pt}%
\pgfpathmoveto{\pgfqpoint{3.688351in}{0.641693in}}%
\pgfpathlineto{\pgfqpoint{3.968076in}{1.994042in}}%
\pgfpathlineto{\pgfqpoint{3.688351in}{0.641693in}}%
\pgfpathclose%
\pgfusepath{fill}%
\end{pgfscope}%
\begin{pgfscope}%
\pgfpathrectangle{\pgfqpoint{0.017500in}{0.017500in}}{\pgfqpoint{6.218282in}{3.580372in}} %
\pgfusepath{clip}%
\pgfsetbuttcap%
\pgfsetroundjoin%
\definecolor{currentfill}{rgb}{0.243520,0.319189,0.771672}%
\pgfsetfillcolor{currentfill}%
\pgfsetlinewidth{0.000000pt}%
\definecolor{currentstroke}{rgb}{0.000000,0.000000,0.000000}%
\pgfsetstrokecolor{currentstroke}%
\pgfsetdash{}{0pt}%
\pgfpathmoveto{\pgfqpoint{3.688351in}{0.641693in}}%
\pgfpathlineto{\pgfqpoint{3.968076in}{1.994042in}}%
\pgfpathlineto{\pgfqpoint{3.688351in}{0.641693in}}%
\pgfpathclose%
\pgfusepath{fill}%
\end{pgfscope}%
\begin{pgfscope}%
\pgfpathrectangle{\pgfqpoint{0.017500in}{0.017500in}}{\pgfqpoint{6.218282in}{3.580372in}} %
\pgfusepath{clip}%
\pgfsetbuttcap%
\pgfsetroundjoin%
\definecolor{currentfill}{rgb}{0.243520,0.319189,0.771672}%
\pgfsetfillcolor{currentfill}%
\pgfsetlinewidth{0.000000pt}%
\definecolor{currentstroke}{rgb}{0.000000,0.000000,0.000000}%
\pgfsetstrokecolor{currentstroke}%
\pgfsetdash{}{0pt}%
\pgfpathmoveto{\pgfqpoint{3.688351in}{0.641693in}}%
\pgfpathlineto{\pgfqpoint{3.968076in}{1.994042in}}%
\pgfpathlineto{\pgfqpoint{3.688351in}{0.641693in}}%
\pgfpathclose%
\pgfusepath{fill}%
\end{pgfscope}%
\begin{pgfscope}%
\pgfpathrectangle{\pgfqpoint{0.017500in}{0.017500in}}{\pgfqpoint{6.218282in}{3.580372in}} %
\pgfusepath{clip}%
\pgfsetbuttcap%
\pgfsetroundjoin%
\definecolor{currentfill}{rgb}{0.243520,0.319189,0.771672}%
\pgfsetfillcolor{currentfill}%
\pgfsetlinewidth{0.000000pt}%
\definecolor{currentstroke}{rgb}{0.000000,0.000000,0.000000}%
\pgfsetstrokecolor{currentstroke}%
\pgfsetdash{}{0pt}%
\pgfpathmoveto{\pgfqpoint{3.688351in}{0.641693in}}%
\pgfpathlineto{\pgfqpoint{3.968076in}{1.994042in}}%
\pgfpathlineto{\pgfqpoint{3.688351in}{0.641693in}}%
\pgfpathclose%
\pgfusepath{fill}%
\end{pgfscope}%
\begin{pgfscope}%
\pgfpathrectangle{\pgfqpoint{0.017500in}{0.017500in}}{\pgfqpoint{6.218282in}{3.580372in}} %
\pgfusepath{clip}%
\pgfsetbuttcap%
\pgfsetroundjoin%
\definecolor{currentfill}{rgb}{0.243520,0.319189,0.771672}%
\pgfsetfillcolor{currentfill}%
\pgfsetlinewidth{0.000000pt}%
\definecolor{currentstroke}{rgb}{0.000000,0.000000,0.000000}%
\pgfsetstrokecolor{currentstroke}%
\pgfsetdash{}{0pt}%
\pgfpathmoveto{\pgfqpoint{3.688351in}{0.641693in}}%
\pgfpathlineto{\pgfqpoint{3.968076in}{1.994042in}}%
\pgfpathlineto{\pgfqpoint{3.688351in}{0.641693in}}%
\pgfpathclose%
\pgfusepath{fill}%
\end{pgfscope}%
\begin{pgfscope}%
\pgfpathrectangle{\pgfqpoint{0.017500in}{0.017500in}}{\pgfqpoint{6.218282in}{3.580372in}} %
\pgfusepath{clip}%
\pgfsetbuttcap%
\pgfsetroundjoin%
\definecolor{currentfill}{rgb}{0.243520,0.319189,0.771672}%
\pgfsetfillcolor{currentfill}%
\pgfsetlinewidth{0.000000pt}%
\definecolor{currentstroke}{rgb}{0.000000,0.000000,0.000000}%
\pgfsetstrokecolor{currentstroke}%
\pgfsetdash{}{0pt}%
\pgfpathmoveto{\pgfqpoint{3.688351in}{0.641693in}}%
\pgfpathlineto{\pgfqpoint{3.968076in}{1.994042in}}%
\pgfpathlineto{\pgfqpoint{3.688351in}{0.641693in}}%
\pgfpathclose%
\pgfusepath{fill}%
\end{pgfscope}%
\begin{pgfscope}%
\pgfpathrectangle{\pgfqpoint{0.017500in}{0.017500in}}{\pgfqpoint{6.218282in}{3.580372in}} %
\pgfusepath{clip}%
\pgfsetbuttcap%
\pgfsetroundjoin%
\definecolor{currentfill}{rgb}{0.243520,0.319189,0.771672}%
\pgfsetfillcolor{currentfill}%
\pgfsetlinewidth{0.000000pt}%
\definecolor{currentstroke}{rgb}{0.000000,0.000000,0.000000}%
\pgfsetstrokecolor{currentstroke}%
\pgfsetdash{}{0pt}%
\pgfpathmoveto{\pgfqpoint{3.688351in}{0.641693in}}%
\pgfpathlineto{\pgfqpoint{3.968076in}{1.994042in}}%
\pgfpathlineto{\pgfqpoint{3.688351in}{0.641693in}}%
\pgfpathclose%
\pgfusepath{fill}%
\end{pgfscope}%
\begin{pgfscope}%
\pgfpathrectangle{\pgfqpoint{0.017500in}{0.017500in}}{\pgfqpoint{6.218282in}{3.580372in}} %
\pgfusepath{clip}%
\pgfsetbuttcap%
\pgfsetroundjoin%
\definecolor{currentfill}{rgb}{0.243520,0.319189,0.771672}%
\pgfsetfillcolor{currentfill}%
\pgfsetlinewidth{0.000000pt}%
\definecolor{currentstroke}{rgb}{0.000000,0.000000,0.000000}%
\pgfsetstrokecolor{currentstroke}%
\pgfsetdash{}{0pt}%
\pgfpathmoveto{\pgfqpoint{3.688351in}{0.641693in}}%
\pgfpathlineto{\pgfqpoint{3.968076in}{1.994042in}}%
\pgfpathlineto{\pgfqpoint{3.688351in}{0.641693in}}%
\pgfpathclose%
\pgfusepath{fill}%
\end{pgfscope}%
\begin{pgfscope}%
\pgfpathrectangle{\pgfqpoint{0.017500in}{0.017500in}}{\pgfqpoint{6.218282in}{3.580372in}} %
\pgfusepath{clip}%
\pgfsetbuttcap%
\pgfsetroundjoin%
\definecolor{currentfill}{rgb}{0.243520,0.319189,0.771672}%
\pgfsetfillcolor{currentfill}%
\pgfsetlinewidth{0.000000pt}%
\definecolor{currentstroke}{rgb}{0.000000,0.000000,0.000000}%
\pgfsetstrokecolor{currentstroke}%
\pgfsetdash{}{0pt}%
\pgfpathmoveto{\pgfqpoint{3.688351in}{0.641693in}}%
\pgfpathlineto{\pgfqpoint{3.968076in}{1.994042in}}%
\pgfpathlineto{\pgfqpoint{3.688351in}{0.641693in}}%
\pgfpathclose%
\pgfusepath{fill}%
\end{pgfscope}%
\begin{pgfscope}%
\pgfpathrectangle{\pgfqpoint{0.017500in}{0.017500in}}{\pgfqpoint{6.218282in}{3.580372in}} %
\pgfusepath{clip}%
\pgfsetbuttcap%
\pgfsetroundjoin%
\definecolor{currentfill}{rgb}{0.243520,0.319189,0.771672}%
\pgfsetfillcolor{currentfill}%
\pgfsetlinewidth{0.000000pt}%
\definecolor{currentstroke}{rgb}{0.000000,0.000000,0.000000}%
\pgfsetstrokecolor{currentstroke}%
\pgfsetdash{}{0pt}%
\pgfpathmoveto{\pgfqpoint{3.688351in}{0.641693in}}%
\pgfpathlineto{\pgfqpoint{3.968076in}{1.994042in}}%
\pgfpathlineto{\pgfqpoint{3.688351in}{0.641693in}}%
\pgfpathclose%
\pgfusepath{fill}%
\end{pgfscope}%
\begin{pgfscope}%
\pgfpathrectangle{\pgfqpoint{0.017500in}{0.017500in}}{\pgfqpoint{6.218282in}{3.580372in}} %
\pgfusepath{clip}%
\pgfsetbuttcap%
\pgfsetroundjoin%
\definecolor{currentfill}{rgb}{0.243520,0.319189,0.771672}%
\pgfsetfillcolor{currentfill}%
\pgfsetlinewidth{0.000000pt}%
\definecolor{currentstroke}{rgb}{0.000000,0.000000,0.000000}%
\pgfsetstrokecolor{currentstroke}%
\pgfsetdash{}{0pt}%
\pgfpathmoveto{\pgfqpoint{3.688351in}{0.641693in}}%
\pgfpathlineto{\pgfqpoint{3.968076in}{1.994042in}}%
\pgfpathlineto{\pgfqpoint{3.688351in}{0.641693in}}%
\pgfpathclose%
\pgfusepath{fill}%
\end{pgfscope}%
\begin{pgfscope}%
\pgfpathrectangle{\pgfqpoint{0.017500in}{0.017500in}}{\pgfqpoint{6.218282in}{3.580372in}} %
\pgfusepath{clip}%
\pgfsetbuttcap%
\pgfsetroundjoin%
\definecolor{currentfill}{rgb}{0.243520,0.319189,0.771672}%
\pgfsetfillcolor{currentfill}%
\pgfsetlinewidth{0.000000pt}%
\definecolor{currentstroke}{rgb}{0.000000,0.000000,0.000000}%
\pgfsetstrokecolor{currentstroke}%
\pgfsetdash{}{0pt}%
\pgfpathmoveto{\pgfqpoint{3.688351in}{0.641693in}}%
\pgfpathlineto{\pgfqpoint{3.968076in}{1.994042in}}%
\pgfpathlineto{\pgfqpoint{3.688351in}{0.641693in}}%
\pgfpathclose%
\pgfusepath{fill}%
\end{pgfscope}%
\begin{pgfscope}%
\pgfpathrectangle{\pgfqpoint{0.017500in}{0.017500in}}{\pgfqpoint{6.218282in}{3.580372in}} %
\pgfusepath{clip}%
\pgfsetbuttcap%
\pgfsetroundjoin%
\definecolor{currentfill}{rgb}{0.243520,0.319189,0.771672}%
\pgfsetfillcolor{currentfill}%
\pgfsetlinewidth{0.000000pt}%
\definecolor{currentstroke}{rgb}{0.000000,0.000000,0.000000}%
\pgfsetstrokecolor{currentstroke}%
\pgfsetdash{}{0pt}%
\pgfpathmoveto{\pgfqpoint{3.688351in}{0.641693in}}%
\pgfpathlineto{\pgfqpoint{3.968076in}{1.994042in}}%
\pgfpathlineto{\pgfqpoint{3.688351in}{0.641693in}}%
\pgfpathclose%
\pgfusepath{fill}%
\end{pgfscope}%
\begin{pgfscope}%
\pgfpathrectangle{\pgfqpoint{0.017500in}{0.017500in}}{\pgfqpoint{6.218282in}{3.580372in}} %
\pgfusepath{clip}%
\pgfsetbuttcap%
\pgfsetroundjoin%
\definecolor{currentfill}{rgb}{0.243520,0.319189,0.771672}%
\pgfsetfillcolor{currentfill}%
\pgfsetlinewidth{0.000000pt}%
\definecolor{currentstroke}{rgb}{0.000000,0.000000,0.000000}%
\pgfsetstrokecolor{currentstroke}%
\pgfsetdash{}{0pt}%
\pgfpathmoveto{\pgfqpoint{3.688351in}{0.641693in}}%
\pgfpathlineto{\pgfqpoint{3.968076in}{1.994042in}}%
\pgfpathlineto{\pgfqpoint{3.688351in}{0.641693in}}%
\pgfpathclose%
\pgfusepath{fill}%
\end{pgfscope}%
\begin{pgfscope}%
\pgfpathrectangle{\pgfqpoint{0.017500in}{0.017500in}}{\pgfqpoint{6.218282in}{3.580372in}} %
\pgfusepath{clip}%
\pgfsetbuttcap%
\pgfsetroundjoin%
\definecolor{currentfill}{rgb}{0.243520,0.319189,0.771672}%
\pgfsetfillcolor{currentfill}%
\pgfsetlinewidth{0.000000pt}%
\definecolor{currentstroke}{rgb}{0.000000,0.000000,0.000000}%
\pgfsetstrokecolor{currentstroke}%
\pgfsetdash{}{0pt}%
\pgfpathmoveto{\pgfqpoint{3.688351in}{0.641693in}}%
\pgfpathlineto{\pgfqpoint{3.968076in}{1.994042in}}%
\pgfpathlineto{\pgfqpoint{3.688351in}{0.641693in}}%
\pgfpathclose%
\pgfusepath{fill}%
\end{pgfscope}%
\begin{pgfscope}%
\pgfpathrectangle{\pgfqpoint{0.017500in}{0.017500in}}{\pgfqpoint{6.218282in}{3.580372in}} %
\pgfusepath{clip}%
\pgfsetbuttcap%
\pgfsetroundjoin%
\definecolor{currentfill}{rgb}{0.243520,0.319189,0.771672}%
\pgfsetfillcolor{currentfill}%
\pgfsetlinewidth{0.000000pt}%
\definecolor{currentstroke}{rgb}{0.000000,0.000000,0.000000}%
\pgfsetstrokecolor{currentstroke}%
\pgfsetdash{}{0pt}%
\pgfpathmoveto{\pgfqpoint{3.688351in}{0.641693in}}%
\pgfpathlineto{\pgfqpoint{3.968076in}{1.994042in}}%
\pgfpathlineto{\pgfqpoint{3.688351in}{0.641693in}}%
\pgfpathclose%
\pgfusepath{fill}%
\end{pgfscope}%
\begin{pgfscope}%
\pgfpathrectangle{\pgfqpoint{0.017500in}{0.017500in}}{\pgfqpoint{6.218282in}{3.580372in}} %
\pgfusepath{clip}%
\pgfsetbuttcap%
\pgfsetroundjoin%
\definecolor{currentfill}{rgb}{0.243520,0.319189,0.771672}%
\pgfsetfillcolor{currentfill}%
\pgfsetlinewidth{0.000000pt}%
\definecolor{currentstroke}{rgb}{0.000000,0.000000,0.000000}%
\pgfsetstrokecolor{currentstroke}%
\pgfsetdash{}{0pt}%
\pgfpathmoveto{\pgfqpoint{3.688351in}{0.641693in}}%
\pgfpathlineto{\pgfqpoint{3.968076in}{1.994042in}}%
\pgfpathlineto{\pgfqpoint{3.688351in}{0.641693in}}%
\pgfpathclose%
\pgfusepath{fill}%
\end{pgfscope}%
\begin{pgfscope}%
\pgfpathrectangle{\pgfqpoint{0.017500in}{0.017500in}}{\pgfqpoint{6.218282in}{3.580372in}} %
\pgfusepath{clip}%
\pgfsetbuttcap%
\pgfsetroundjoin%
\definecolor{currentfill}{rgb}{0.243520,0.319189,0.771672}%
\pgfsetfillcolor{currentfill}%
\pgfsetlinewidth{0.000000pt}%
\definecolor{currentstroke}{rgb}{0.000000,0.000000,0.000000}%
\pgfsetstrokecolor{currentstroke}%
\pgfsetdash{}{0pt}%
\pgfpathmoveto{\pgfqpoint{3.688351in}{0.641693in}}%
\pgfpathlineto{\pgfqpoint{3.968076in}{1.994042in}}%
\pgfpathlineto{\pgfqpoint{3.688351in}{0.641693in}}%
\pgfpathclose%
\pgfusepath{fill}%
\end{pgfscope}%
\begin{pgfscope}%
\pgfpathrectangle{\pgfqpoint{0.017500in}{0.017500in}}{\pgfqpoint{6.218282in}{3.580372in}} %
\pgfusepath{clip}%
\pgfsetbuttcap%
\pgfsetroundjoin%
\definecolor{currentfill}{rgb}{0.243520,0.319189,0.771672}%
\pgfsetfillcolor{currentfill}%
\pgfsetlinewidth{0.000000pt}%
\definecolor{currentstroke}{rgb}{0.000000,0.000000,0.000000}%
\pgfsetstrokecolor{currentstroke}%
\pgfsetdash{}{0pt}%
\pgfpathmoveto{\pgfqpoint{3.688351in}{0.641693in}}%
\pgfpathlineto{\pgfqpoint{3.968076in}{1.994042in}}%
\pgfpathlineto{\pgfqpoint{3.688351in}{0.641693in}}%
\pgfpathclose%
\pgfusepath{fill}%
\end{pgfscope}%
\begin{pgfscope}%
\pgfpathrectangle{\pgfqpoint{0.017500in}{0.017500in}}{\pgfqpoint{6.218282in}{3.580372in}} %
\pgfusepath{clip}%
\pgfsetbuttcap%
\pgfsetroundjoin%
\definecolor{currentfill}{rgb}{0.243520,0.319189,0.771672}%
\pgfsetfillcolor{currentfill}%
\pgfsetlinewidth{0.000000pt}%
\definecolor{currentstroke}{rgb}{0.000000,0.000000,0.000000}%
\pgfsetstrokecolor{currentstroke}%
\pgfsetdash{}{0pt}%
\pgfpathmoveto{\pgfqpoint{3.688351in}{0.641693in}}%
\pgfpathlineto{\pgfqpoint{3.968076in}{1.994042in}}%
\pgfpathlineto{\pgfqpoint{3.688351in}{0.641693in}}%
\pgfpathclose%
\pgfusepath{fill}%
\end{pgfscope}%
\begin{pgfscope}%
\pgfpathrectangle{\pgfqpoint{0.017500in}{0.017500in}}{\pgfqpoint{6.218282in}{3.580372in}} %
\pgfusepath{clip}%
\pgfsetbuttcap%
\pgfsetroundjoin%
\definecolor{currentfill}{rgb}{0.243520,0.319189,0.771672}%
\pgfsetfillcolor{currentfill}%
\pgfsetlinewidth{0.000000pt}%
\definecolor{currentstroke}{rgb}{0.000000,0.000000,0.000000}%
\pgfsetstrokecolor{currentstroke}%
\pgfsetdash{}{0pt}%
\pgfpathmoveto{\pgfqpoint{3.688351in}{0.641693in}}%
\pgfpathlineto{\pgfqpoint{3.968076in}{1.994042in}}%
\pgfpathlineto{\pgfqpoint{3.688351in}{0.641693in}}%
\pgfpathclose%
\pgfusepath{fill}%
\end{pgfscope}%
\begin{pgfscope}%
\pgfpathrectangle{\pgfqpoint{0.017500in}{0.017500in}}{\pgfqpoint{6.218282in}{3.580372in}} %
\pgfusepath{clip}%
\pgfsetbuttcap%
\pgfsetroundjoin%
\definecolor{currentfill}{rgb}{0.243520,0.319189,0.771672}%
\pgfsetfillcolor{currentfill}%
\pgfsetlinewidth{0.000000pt}%
\definecolor{currentstroke}{rgb}{0.000000,0.000000,0.000000}%
\pgfsetstrokecolor{currentstroke}%
\pgfsetdash{}{0pt}%
\pgfpathmoveto{\pgfqpoint{3.688351in}{0.641693in}}%
\pgfpathlineto{\pgfqpoint{3.968076in}{1.994042in}}%
\pgfpathlineto{\pgfqpoint{3.688351in}{0.641693in}}%
\pgfpathclose%
\pgfusepath{fill}%
\end{pgfscope}%
\begin{pgfscope}%
\pgfpathrectangle{\pgfqpoint{0.017500in}{0.017500in}}{\pgfqpoint{6.218282in}{3.580372in}} %
\pgfusepath{clip}%
\pgfsetbuttcap%
\pgfsetroundjoin%
\definecolor{currentfill}{rgb}{0.243520,0.319189,0.771672}%
\pgfsetfillcolor{currentfill}%
\pgfsetlinewidth{0.000000pt}%
\definecolor{currentstroke}{rgb}{0.000000,0.000000,0.000000}%
\pgfsetstrokecolor{currentstroke}%
\pgfsetdash{}{0pt}%
\pgfpathmoveto{\pgfqpoint{3.688351in}{0.641693in}}%
\pgfpathlineto{\pgfqpoint{3.968076in}{1.994042in}}%
\pgfpathlineto{\pgfqpoint{3.688351in}{0.641693in}}%
\pgfpathclose%
\pgfusepath{fill}%
\end{pgfscope}%
\begin{pgfscope}%
\pgfpathrectangle{\pgfqpoint{0.017500in}{0.017500in}}{\pgfqpoint{6.218282in}{3.580372in}} %
\pgfusepath{clip}%
\pgfsetbuttcap%
\pgfsetroundjoin%
\definecolor{currentfill}{rgb}{0.243520,0.319189,0.771672}%
\pgfsetfillcolor{currentfill}%
\pgfsetlinewidth{0.000000pt}%
\definecolor{currentstroke}{rgb}{0.000000,0.000000,0.000000}%
\pgfsetstrokecolor{currentstroke}%
\pgfsetdash{}{0pt}%
\pgfpathmoveto{\pgfqpoint{3.688351in}{0.641693in}}%
\pgfpathlineto{\pgfqpoint{3.968076in}{1.994042in}}%
\pgfpathlineto{\pgfqpoint{3.688351in}{0.641693in}}%
\pgfpathclose%
\pgfusepath{fill}%
\end{pgfscope}%
\begin{pgfscope}%
\pgfpathrectangle{\pgfqpoint{0.017500in}{0.017500in}}{\pgfqpoint{6.218282in}{3.580372in}} %
\pgfusepath{clip}%
\pgfsetbuttcap%
\pgfsetroundjoin%
\definecolor{currentfill}{rgb}{0.243520,0.319189,0.771672}%
\pgfsetfillcolor{currentfill}%
\pgfsetlinewidth{0.000000pt}%
\definecolor{currentstroke}{rgb}{0.000000,0.000000,0.000000}%
\pgfsetstrokecolor{currentstroke}%
\pgfsetdash{}{0pt}%
\pgfpathmoveto{\pgfqpoint{3.688351in}{0.641693in}}%
\pgfpathlineto{\pgfqpoint{3.968076in}{1.994042in}}%
\pgfpathlineto{\pgfqpoint{3.688351in}{0.641693in}}%
\pgfpathclose%
\pgfusepath{fill}%
\end{pgfscope}%
\begin{pgfscope}%
\pgfpathrectangle{\pgfqpoint{0.017500in}{0.017500in}}{\pgfqpoint{6.218282in}{3.580372in}} %
\pgfusepath{clip}%
\pgfsetbuttcap%
\pgfsetroundjoin%
\definecolor{currentfill}{rgb}{0.243520,0.319189,0.771672}%
\pgfsetfillcolor{currentfill}%
\pgfsetlinewidth{0.000000pt}%
\definecolor{currentstroke}{rgb}{0.000000,0.000000,0.000000}%
\pgfsetstrokecolor{currentstroke}%
\pgfsetdash{}{0pt}%
\pgfpathmoveto{\pgfqpoint{3.688351in}{0.641693in}}%
\pgfpathlineto{\pgfqpoint{3.968076in}{1.994042in}}%
\pgfpathlineto{\pgfqpoint{3.688351in}{0.641693in}}%
\pgfpathclose%
\pgfusepath{fill}%
\end{pgfscope}%
\begin{pgfscope}%
\pgfpathrectangle{\pgfqpoint{0.017500in}{0.017500in}}{\pgfqpoint{6.218282in}{3.580372in}} %
\pgfusepath{clip}%
\pgfsetbuttcap%
\pgfsetroundjoin%
\definecolor{currentfill}{rgb}{0.243520,0.319189,0.771672}%
\pgfsetfillcolor{currentfill}%
\pgfsetlinewidth{0.000000pt}%
\definecolor{currentstroke}{rgb}{0.000000,0.000000,0.000000}%
\pgfsetstrokecolor{currentstroke}%
\pgfsetdash{}{0pt}%
\pgfpathmoveto{\pgfqpoint{3.688351in}{0.641693in}}%
\pgfpathlineto{\pgfqpoint{3.968076in}{1.994042in}}%
\pgfpathlineto{\pgfqpoint{3.688351in}{0.641693in}}%
\pgfpathclose%
\pgfusepath{fill}%
\end{pgfscope}%
\begin{pgfscope}%
\pgfpathrectangle{\pgfqpoint{0.017500in}{0.017500in}}{\pgfqpoint{6.218282in}{3.580372in}} %
\pgfusepath{clip}%
\pgfsetbuttcap%
\pgfsetroundjoin%
\definecolor{currentfill}{rgb}{0.243520,0.319189,0.771672}%
\pgfsetfillcolor{currentfill}%
\pgfsetlinewidth{0.000000pt}%
\definecolor{currentstroke}{rgb}{0.000000,0.000000,0.000000}%
\pgfsetstrokecolor{currentstroke}%
\pgfsetdash{}{0pt}%
\pgfpathmoveto{\pgfqpoint{3.688351in}{0.641693in}}%
\pgfpathlineto{\pgfqpoint{3.968076in}{1.994042in}}%
\pgfpathlineto{\pgfqpoint{3.688351in}{0.641693in}}%
\pgfpathclose%
\pgfusepath{fill}%
\end{pgfscope}%
\begin{pgfscope}%
\pgfpathrectangle{\pgfqpoint{0.017500in}{0.017500in}}{\pgfqpoint{6.218282in}{3.580372in}} %
\pgfusepath{clip}%
\pgfsetbuttcap%
\pgfsetroundjoin%
\definecolor{currentfill}{rgb}{0.243520,0.319189,0.771672}%
\pgfsetfillcolor{currentfill}%
\pgfsetlinewidth{0.000000pt}%
\definecolor{currentstroke}{rgb}{0.000000,0.000000,0.000000}%
\pgfsetstrokecolor{currentstroke}%
\pgfsetdash{}{0pt}%
\pgfpathmoveto{\pgfqpoint{3.688351in}{0.641693in}}%
\pgfpathlineto{\pgfqpoint{3.968076in}{1.994042in}}%
\pgfpathlineto{\pgfqpoint{3.688351in}{0.641693in}}%
\pgfpathclose%
\pgfusepath{fill}%
\end{pgfscope}%
\begin{pgfscope}%
\pgfpathrectangle{\pgfqpoint{0.017500in}{0.017500in}}{\pgfqpoint{6.218282in}{3.580372in}} %
\pgfusepath{clip}%
\pgfsetbuttcap%
\pgfsetroundjoin%
\definecolor{currentfill}{rgb}{0.243520,0.319189,0.771672}%
\pgfsetfillcolor{currentfill}%
\pgfsetlinewidth{0.000000pt}%
\definecolor{currentstroke}{rgb}{0.000000,0.000000,0.000000}%
\pgfsetstrokecolor{currentstroke}%
\pgfsetdash{}{0pt}%
\pgfpathmoveto{\pgfqpoint{3.688351in}{0.641693in}}%
\pgfpathlineto{\pgfqpoint{3.968076in}{1.994042in}}%
\pgfpathlineto{\pgfqpoint{3.688351in}{0.641693in}}%
\pgfpathclose%
\pgfusepath{fill}%
\end{pgfscope}%
\begin{pgfscope}%
\pgfpathrectangle{\pgfqpoint{0.017500in}{0.017500in}}{\pgfqpoint{6.218282in}{3.580372in}} %
\pgfusepath{clip}%
\pgfsetbuttcap%
\pgfsetroundjoin%
\definecolor{currentfill}{rgb}{0.895885,0.433075,0.338681}%
\pgfsetfillcolor{currentfill}%
\pgfsetlinewidth{0.000000pt}%
\definecolor{currentstroke}{rgb}{0.000000,0.000000,0.000000}%
\pgfsetstrokecolor{currentstroke}%
\pgfsetdash{}{0pt}%
\pgfpathmoveto{\pgfqpoint{3.043326in}{2.141782in}}%
\pgfpathlineto{\pgfqpoint{3.306485in}{2.382499in}}%
\pgfpathlineto{\pgfqpoint{3.043326in}{2.141782in}}%
\pgfpathclose%
\pgfusepath{fill}%
\end{pgfscope}%
\begin{pgfscope}%
\pgfpathrectangle{\pgfqpoint{0.017500in}{0.017500in}}{\pgfqpoint{6.218282in}{3.580372in}} %
\pgfusepath{clip}%
\pgfsetbuttcap%
\pgfsetroundjoin%
\definecolor{currentfill}{rgb}{0.895885,0.433075,0.338681}%
\pgfsetfillcolor{currentfill}%
\pgfsetlinewidth{0.000000pt}%
\definecolor{currentstroke}{rgb}{0.000000,0.000000,0.000000}%
\pgfsetstrokecolor{currentstroke}%
\pgfsetdash{}{0pt}%
\pgfpathmoveto{\pgfqpoint{3.043326in}{2.141782in}}%
\pgfpathlineto{\pgfqpoint{3.306485in}{2.382499in}}%
\pgfpathlineto{\pgfqpoint{3.043326in}{2.141782in}}%
\pgfpathclose%
\pgfusepath{fill}%
\end{pgfscope}%
\begin{pgfscope}%
\pgfpathrectangle{\pgfqpoint{0.017500in}{0.017500in}}{\pgfqpoint{6.218282in}{3.580372in}} %
\pgfusepath{clip}%
\pgfsetbuttcap%
\pgfsetroundjoin%
\definecolor{currentfill}{rgb}{0.895885,0.433075,0.338681}%
\pgfsetfillcolor{currentfill}%
\pgfsetlinewidth{0.000000pt}%
\definecolor{currentstroke}{rgb}{0.000000,0.000000,0.000000}%
\pgfsetstrokecolor{currentstroke}%
\pgfsetdash{}{0pt}%
\pgfpathmoveto{\pgfqpoint{3.043326in}{2.141782in}}%
\pgfpathlineto{\pgfqpoint{3.306485in}{2.382499in}}%
\pgfpathlineto{\pgfqpoint{3.043326in}{2.141782in}}%
\pgfpathclose%
\pgfusepath{fill}%
\end{pgfscope}%
\begin{pgfscope}%
\pgfpathrectangle{\pgfqpoint{0.017500in}{0.017500in}}{\pgfqpoint{6.218282in}{3.580372in}} %
\pgfusepath{clip}%
\pgfsetbuttcap%
\pgfsetroundjoin%
\definecolor{currentfill}{rgb}{0.895885,0.433075,0.338681}%
\pgfsetfillcolor{currentfill}%
\pgfsetlinewidth{0.000000pt}%
\definecolor{currentstroke}{rgb}{0.000000,0.000000,0.000000}%
\pgfsetstrokecolor{currentstroke}%
\pgfsetdash{}{0pt}%
\pgfpathmoveto{\pgfqpoint{3.043326in}{2.141782in}}%
\pgfpathlineto{\pgfqpoint{3.306485in}{2.382499in}}%
\pgfpathlineto{\pgfqpoint{3.043326in}{2.141782in}}%
\pgfpathclose%
\pgfusepath{fill}%
\end{pgfscope}%
\begin{pgfscope}%
\pgfpathrectangle{\pgfqpoint{0.017500in}{0.017500in}}{\pgfqpoint{6.218282in}{3.580372in}} %
\pgfusepath{clip}%
\pgfsetbuttcap%
\pgfsetroundjoin%
\definecolor{currentfill}{rgb}{0.895885,0.433075,0.338681}%
\pgfsetfillcolor{currentfill}%
\pgfsetlinewidth{0.000000pt}%
\definecolor{currentstroke}{rgb}{0.000000,0.000000,0.000000}%
\pgfsetstrokecolor{currentstroke}%
\pgfsetdash{}{0pt}%
\pgfpathmoveto{\pgfqpoint{3.043326in}{2.141782in}}%
\pgfpathlineto{\pgfqpoint{3.306485in}{2.382499in}}%
\pgfpathlineto{\pgfqpoint{3.043326in}{2.141782in}}%
\pgfpathclose%
\pgfusepath{fill}%
\end{pgfscope}%
\begin{pgfscope}%
\pgfpathrectangle{\pgfqpoint{0.017500in}{0.017500in}}{\pgfqpoint{6.218282in}{3.580372in}} %
\pgfusepath{clip}%
\pgfsetbuttcap%
\pgfsetroundjoin%
\definecolor{currentfill}{rgb}{0.895885,0.433075,0.338681}%
\pgfsetfillcolor{currentfill}%
\pgfsetlinewidth{0.000000pt}%
\definecolor{currentstroke}{rgb}{0.000000,0.000000,0.000000}%
\pgfsetstrokecolor{currentstroke}%
\pgfsetdash{}{0pt}%
\pgfpathmoveto{\pgfqpoint{3.043326in}{2.141782in}}%
\pgfpathlineto{\pgfqpoint{3.306485in}{2.382499in}}%
\pgfpathlineto{\pgfqpoint{3.043326in}{2.141782in}}%
\pgfpathclose%
\pgfusepath{fill}%
\end{pgfscope}%
\begin{pgfscope}%
\pgfpathrectangle{\pgfqpoint{0.017500in}{0.017500in}}{\pgfqpoint{6.218282in}{3.580372in}} %
\pgfusepath{clip}%
\pgfsetbuttcap%
\pgfsetroundjoin%
\definecolor{currentfill}{rgb}{0.895885,0.433075,0.338681}%
\pgfsetfillcolor{currentfill}%
\pgfsetlinewidth{0.000000pt}%
\definecolor{currentstroke}{rgb}{0.000000,0.000000,0.000000}%
\pgfsetstrokecolor{currentstroke}%
\pgfsetdash{}{0pt}%
\pgfpathmoveto{\pgfqpoint{3.043326in}{2.141782in}}%
\pgfpathlineto{\pgfqpoint{3.306485in}{2.382499in}}%
\pgfpathlineto{\pgfqpoint{3.043326in}{2.141782in}}%
\pgfpathclose%
\pgfusepath{fill}%
\end{pgfscope}%
\begin{pgfscope}%
\pgfpathrectangle{\pgfqpoint{0.017500in}{0.017500in}}{\pgfqpoint{6.218282in}{3.580372in}} %
\pgfusepath{clip}%
\pgfsetbuttcap%
\pgfsetroundjoin%
\definecolor{currentfill}{rgb}{0.895885,0.433075,0.338681}%
\pgfsetfillcolor{currentfill}%
\pgfsetlinewidth{0.000000pt}%
\definecolor{currentstroke}{rgb}{0.000000,0.000000,0.000000}%
\pgfsetstrokecolor{currentstroke}%
\pgfsetdash{}{0pt}%
\pgfpathmoveto{\pgfqpoint{3.043326in}{2.141782in}}%
\pgfpathlineto{\pgfqpoint{3.306485in}{2.382499in}}%
\pgfpathlineto{\pgfqpoint{3.043326in}{2.141782in}}%
\pgfpathclose%
\pgfusepath{fill}%
\end{pgfscope}%
\begin{pgfscope}%
\pgfpathrectangle{\pgfqpoint{0.017500in}{0.017500in}}{\pgfqpoint{6.218282in}{3.580372in}} %
\pgfusepath{clip}%
\pgfsetbuttcap%
\pgfsetroundjoin%
\definecolor{currentfill}{rgb}{0.895885,0.433075,0.338681}%
\pgfsetfillcolor{currentfill}%
\pgfsetlinewidth{0.000000pt}%
\definecolor{currentstroke}{rgb}{0.000000,0.000000,0.000000}%
\pgfsetstrokecolor{currentstroke}%
\pgfsetdash{}{0pt}%
\pgfpathmoveto{\pgfqpoint{3.043326in}{2.141782in}}%
\pgfpathlineto{\pgfqpoint{3.306485in}{2.382499in}}%
\pgfpathlineto{\pgfqpoint{3.043326in}{2.141782in}}%
\pgfpathclose%
\pgfusepath{fill}%
\end{pgfscope}%
\begin{pgfscope}%
\pgfpathrectangle{\pgfqpoint{0.017500in}{0.017500in}}{\pgfqpoint{6.218282in}{3.580372in}} %
\pgfusepath{clip}%
\pgfsetbuttcap%
\pgfsetroundjoin%
\definecolor{currentfill}{rgb}{0.895885,0.433075,0.338681}%
\pgfsetfillcolor{currentfill}%
\pgfsetlinewidth{0.000000pt}%
\definecolor{currentstroke}{rgb}{0.000000,0.000000,0.000000}%
\pgfsetstrokecolor{currentstroke}%
\pgfsetdash{}{0pt}%
\pgfpathmoveto{\pgfqpoint{3.043326in}{2.141782in}}%
\pgfpathlineto{\pgfqpoint{3.306485in}{2.382499in}}%
\pgfpathlineto{\pgfqpoint{3.043326in}{2.141782in}}%
\pgfpathclose%
\pgfusepath{fill}%
\end{pgfscope}%
\begin{pgfscope}%
\pgfpathrectangle{\pgfqpoint{0.017500in}{0.017500in}}{\pgfqpoint{6.218282in}{3.580372in}} %
\pgfusepath{clip}%
\pgfsetbuttcap%
\pgfsetroundjoin%
\definecolor{currentfill}{rgb}{0.895885,0.433075,0.338681}%
\pgfsetfillcolor{currentfill}%
\pgfsetlinewidth{0.000000pt}%
\definecolor{currentstroke}{rgb}{0.000000,0.000000,0.000000}%
\pgfsetstrokecolor{currentstroke}%
\pgfsetdash{}{0pt}%
\pgfpathmoveto{\pgfqpoint{3.043326in}{2.141782in}}%
\pgfpathlineto{\pgfqpoint{3.306485in}{2.382499in}}%
\pgfpathlineto{\pgfqpoint{3.043326in}{2.141782in}}%
\pgfpathclose%
\pgfusepath{fill}%
\end{pgfscope}%
\begin{pgfscope}%
\pgfpathrectangle{\pgfqpoint{0.017500in}{0.017500in}}{\pgfqpoint{6.218282in}{3.580372in}} %
\pgfusepath{clip}%
\pgfsetbuttcap%
\pgfsetroundjoin%
\definecolor{currentfill}{rgb}{0.895885,0.433075,0.338681}%
\pgfsetfillcolor{currentfill}%
\pgfsetlinewidth{0.000000pt}%
\definecolor{currentstroke}{rgb}{0.000000,0.000000,0.000000}%
\pgfsetstrokecolor{currentstroke}%
\pgfsetdash{}{0pt}%
\pgfpathmoveto{\pgfqpoint{3.043326in}{2.141782in}}%
\pgfpathlineto{\pgfqpoint{3.306485in}{2.382499in}}%
\pgfpathlineto{\pgfqpoint{3.043326in}{2.141782in}}%
\pgfpathclose%
\pgfusepath{fill}%
\end{pgfscope}%
\begin{pgfscope}%
\pgfpathrectangle{\pgfqpoint{0.017500in}{0.017500in}}{\pgfqpoint{6.218282in}{3.580372in}} %
\pgfusepath{clip}%
\pgfsetbuttcap%
\pgfsetroundjoin%
\definecolor{currentfill}{rgb}{0.895885,0.433075,0.338681}%
\pgfsetfillcolor{currentfill}%
\pgfsetlinewidth{0.000000pt}%
\definecolor{currentstroke}{rgb}{0.000000,0.000000,0.000000}%
\pgfsetstrokecolor{currentstroke}%
\pgfsetdash{}{0pt}%
\pgfpathmoveto{\pgfqpoint{3.043326in}{2.141782in}}%
\pgfpathlineto{\pgfqpoint{3.306485in}{2.382499in}}%
\pgfpathlineto{\pgfqpoint{3.043326in}{2.141782in}}%
\pgfpathclose%
\pgfusepath{fill}%
\end{pgfscope}%
\begin{pgfscope}%
\pgfpathrectangle{\pgfqpoint{0.017500in}{0.017500in}}{\pgfqpoint{6.218282in}{3.580372in}} %
\pgfusepath{clip}%
\pgfsetbuttcap%
\pgfsetroundjoin%
\definecolor{currentfill}{rgb}{0.895885,0.433075,0.338681}%
\pgfsetfillcolor{currentfill}%
\pgfsetlinewidth{0.000000pt}%
\definecolor{currentstroke}{rgb}{0.000000,0.000000,0.000000}%
\pgfsetstrokecolor{currentstroke}%
\pgfsetdash{}{0pt}%
\pgfpathmoveto{\pgfqpoint{3.043326in}{2.141782in}}%
\pgfpathlineto{\pgfqpoint{3.306485in}{2.382499in}}%
\pgfpathlineto{\pgfqpoint{3.043326in}{2.141782in}}%
\pgfpathclose%
\pgfusepath{fill}%
\end{pgfscope}%
\begin{pgfscope}%
\pgfpathrectangle{\pgfqpoint{0.017500in}{0.017500in}}{\pgfqpoint{6.218282in}{3.580372in}} %
\pgfusepath{clip}%
\pgfsetbuttcap%
\pgfsetroundjoin%
\definecolor{currentfill}{rgb}{0.895885,0.433075,0.338681}%
\pgfsetfillcolor{currentfill}%
\pgfsetlinewidth{0.000000pt}%
\definecolor{currentstroke}{rgb}{0.000000,0.000000,0.000000}%
\pgfsetstrokecolor{currentstroke}%
\pgfsetdash{}{0pt}%
\pgfpathmoveto{\pgfqpoint{3.043326in}{2.141782in}}%
\pgfpathlineto{\pgfqpoint{3.306485in}{2.382499in}}%
\pgfpathlineto{\pgfqpoint{3.043326in}{2.141782in}}%
\pgfpathclose%
\pgfusepath{fill}%
\end{pgfscope}%
\begin{pgfscope}%
\pgfpathrectangle{\pgfqpoint{0.017500in}{0.017500in}}{\pgfqpoint{6.218282in}{3.580372in}} %
\pgfusepath{clip}%
\pgfsetbuttcap%
\pgfsetroundjoin%
\definecolor{currentfill}{rgb}{0.895885,0.433075,0.338681}%
\pgfsetfillcolor{currentfill}%
\pgfsetlinewidth{0.000000pt}%
\definecolor{currentstroke}{rgb}{0.000000,0.000000,0.000000}%
\pgfsetstrokecolor{currentstroke}%
\pgfsetdash{}{0pt}%
\pgfpathmoveto{\pgfqpoint{3.043326in}{2.141782in}}%
\pgfpathlineto{\pgfqpoint{3.306485in}{2.382499in}}%
\pgfpathlineto{\pgfqpoint{3.043326in}{2.141782in}}%
\pgfpathclose%
\pgfusepath{fill}%
\end{pgfscope}%
\begin{pgfscope}%
\pgfpathrectangle{\pgfqpoint{0.017500in}{0.017500in}}{\pgfqpoint{6.218282in}{3.580372in}} %
\pgfusepath{clip}%
\pgfsetbuttcap%
\pgfsetroundjoin%
\definecolor{currentfill}{rgb}{0.895885,0.433075,0.338681}%
\pgfsetfillcolor{currentfill}%
\pgfsetlinewidth{0.000000pt}%
\definecolor{currentstroke}{rgb}{0.000000,0.000000,0.000000}%
\pgfsetstrokecolor{currentstroke}%
\pgfsetdash{}{0pt}%
\pgfpathmoveto{\pgfqpoint{3.043326in}{2.141782in}}%
\pgfpathlineto{\pgfqpoint{3.306485in}{2.382499in}}%
\pgfpathlineto{\pgfqpoint{3.043326in}{2.141782in}}%
\pgfpathclose%
\pgfusepath{fill}%
\end{pgfscope}%
\begin{pgfscope}%
\pgfpathrectangle{\pgfqpoint{0.017500in}{0.017500in}}{\pgfqpoint{6.218282in}{3.580372in}} %
\pgfusepath{clip}%
\pgfsetbuttcap%
\pgfsetroundjoin%
\definecolor{currentfill}{rgb}{0.895885,0.433075,0.338681}%
\pgfsetfillcolor{currentfill}%
\pgfsetlinewidth{0.000000pt}%
\definecolor{currentstroke}{rgb}{0.000000,0.000000,0.000000}%
\pgfsetstrokecolor{currentstroke}%
\pgfsetdash{}{0pt}%
\pgfpathmoveto{\pgfqpoint{3.043326in}{2.141782in}}%
\pgfpathlineto{\pgfqpoint{3.306485in}{2.382499in}}%
\pgfpathlineto{\pgfqpoint{3.043326in}{2.141782in}}%
\pgfpathclose%
\pgfusepath{fill}%
\end{pgfscope}%
\begin{pgfscope}%
\pgfpathrectangle{\pgfqpoint{0.017500in}{0.017500in}}{\pgfqpoint{6.218282in}{3.580372in}} %
\pgfusepath{clip}%
\pgfsetbuttcap%
\pgfsetroundjoin%
\definecolor{currentfill}{rgb}{0.895885,0.433075,0.338681}%
\pgfsetfillcolor{currentfill}%
\pgfsetlinewidth{0.000000pt}%
\definecolor{currentstroke}{rgb}{0.000000,0.000000,0.000000}%
\pgfsetstrokecolor{currentstroke}%
\pgfsetdash{}{0pt}%
\pgfpathmoveto{\pgfqpoint{3.043326in}{2.141782in}}%
\pgfpathlineto{\pgfqpoint{3.306485in}{2.382499in}}%
\pgfpathlineto{\pgfqpoint{3.043326in}{2.141782in}}%
\pgfpathclose%
\pgfusepath{fill}%
\end{pgfscope}%
\begin{pgfscope}%
\pgfpathrectangle{\pgfqpoint{0.017500in}{0.017500in}}{\pgfqpoint{6.218282in}{3.580372in}} %
\pgfusepath{clip}%
\pgfsetbuttcap%
\pgfsetroundjoin%
\definecolor{currentfill}{rgb}{0.895885,0.433075,0.338681}%
\pgfsetfillcolor{currentfill}%
\pgfsetlinewidth{0.000000pt}%
\definecolor{currentstroke}{rgb}{0.000000,0.000000,0.000000}%
\pgfsetstrokecolor{currentstroke}%
\pgfsetdash{}{0pt}%
\pgfpathmoveto{\pgfqpoint{3.043326in}{2.141782in}}%
\pgfpathlineto{\pgfqpoint{3.306485in}{2.382499in}}%
\pgfpathlineto{\pgfqpoint{3.043326in}{2.141782in}}%
\pgfpathclose%
\pgfusepath{fill}%
\end{pgfscope}%
\begin{pgfscope}%
\pgfpathrectangle{\pgfqpoint{0.017500in}{0.017500in}}{\pgfqpoint{6.218282in}{3.580372in}} %
\pgfusepath{clip}%
\pgfsetbuttcap%
\pgfsetroundjoin%
\definecolor{currentfill}{rgb}{0.895885,0.433075,0.338681}%
\pgfsetfillcolor{currentfill}%
\pgfsetlinewidth{0.000000pt}%
\definecolor{currentstroke}{rgb}{0.000000,0.000000,0.000000}%
\pgfsetstrokecolor{currentstroke}%
\pgfsetdash{}{0pt}%
\pgfpathmoveto{\pgfqpoint{3.043326in}{2.141782in}}%
\pgfpathlineto{\pgfqpoint{3.306485in}{2.382499in}}%
\pgfpathlineto{\pgfqpoint{3.043326in}{2.141782in}}%
\pgfpathclose%
\pgfusepath{fill}%
\end{pgfscope}%
\begin{pgfscope}%
\pgfpathrectangle{\pgfqpoint{0.017500in}{0.017500in}}{\pgfqpoint{6.218282in}{3.580372in}} %
\pgfusepath{clip}%
\pgfsetbuttcap%
\pgfsetroundjoin%
\definecolor{currentfill}{rgb}{0.895885,0.433075,0.338681}%
\pgfsetfillcolor{currentfill}%
\pgfsetlinewidth{0.000000pt}%
\definecolor{currentstroke}{rgb}{0.000000,0.000000,0.000000}%
\pgfsetstrokecolor{currentstroke}%
\pgfsetdash{}{0pt}%
\pgfpathmoveto{\pgfqpoint{3.043326in}{2.141782in}}%
\pgfpathlineto{\pgfqpoint{3.306485in}{2.382499in}}%
\pgfpathlineto{\pgfqpoint{3.043326in}{2.141782in}}%
\pgfpathclose%
\pgfusepath{fill}%
\end{pgfscope}%
\begin{pgfscope}%
\pgfpathrectangle{\pgfqpoint{0.017500in}{0.017500in}}{\pgfqpoint{6.218282in}{3.580372in}} %
\pgfusepath{clip}%
\pgfsetbuttcap%
\pgfsetroundjoin%
\definecolor{currentfill}{rgb}{0.895885,0.433075,0.338681}%
\pgfsetfillcolor{currentfill}%
\pgfsetlinewidth{0.000000pt}%
\definecolor{currentstroke}{rgb}{0.000000,0.000000,0.000000}%
\pgfsetstrokecolor{currentstroke}%
\pgfsetdash{}{0pt}%
\pgfpathmoveto{\pgfqpoint{3.043326in}{2.141782in}}%
\pgfpathlineto{\pgfqpoint{3.306485in}{2.382499in}}%
\pgfpathlineto{\pgfqpoint{3.043326in}{2.141782in}}%
\pgfpathclose%
\pgfusepath{fill}%
\end{pgfscope}%
\begin{pgfscope}%
\pgfpathrectangle{\pgfqpoint{0.017500in}{0.017500in}}{\pgfqpoint{6.218282in}{3.580372in}} %
\pgfusepath{clip}%
\pgfsetbuttcap%
\pgfsetroundjoin%
\definecolor{currentfill}{rgb}{0.895885,0.433075,0.338681}%
\pgfsetfillcolor{currentfill}%
\pgfsetlinewidth{0.000000pt}%
\definecolor{currentstroke}{rgb}{0.000000,0.000000,0.000000}%
\pgfsetstrokecolor{currentstroke}%
\pgfsetdash{}{0pt}%
\pgfpathmoveto{\pgfqpoint{3.043326in}{2.141782in}}%
\pgfpathlineto{\pgfqpoint{3.306485in}{2.382499in}}%
\pgfpathlineto{\pgfqpoint{3.043326in}{2.141782in}}%
\pgfpathclose%
\pgfusepath{fill}%
\end{pgfscope}%
\begin{pgfscope}%
\pgfpathrectangle{\pgfqpoint{0.017500in}{0.017500in}}{\pgfqpoint{6.218282in}{3.580372in}} %
\pgfusepath{clip}%
\pgfsetbuttcap%
\pgfsetroundjoin%
\definecolor{currentfill}{rgb}{0.895885,0.433075,0.338681}%
\pgfsetfillcolor{currentfill}%
\pgfsetlinewidth{0.000000pt}%
\definecolor{currentstroke}{rgb}{0.000000,0.000000,0.000000}%
\pgfsetstrokecolor{currentstroke}%
\pgfsetdash{}{0pt}%
\pgfpathmoveto{\pgfqpoint{3.043326in}{2.141782in}}%
\pgfpathlineto{\pgfqpoint{3.306485in}{2.382499in}}%
\pgfpathlineto{\pgfqpoint{3.043326in}{2.141782in}}%
\pgfpathclose%
\pgfusepath{fill}%
\end{pgfscope}%
\begin{pgfscope}%
\pgfpathrectangle{\pgfqpoint{0.017500in}{0.017500in}}{\pgfqpoint{6.218282in}{3.580372in}} %
\pgfusepath{clip}%
\pgfsetbuttcap%
\pgfsetroundjoin%
\definecolor{currentfill}{rgb}{0.895885,0.433075,0.338681}%
\pgfsetfillcolor{currentfill}%
\pgfsetlinewidth{0.000000pt}%
\definecolor{currentstroke}{rgb}{0.000000,0.000000,0.000000}%
\pgfsetstrokecolor{currentstroke}%
\pgfsetdash{}{0pt}%
\pgfpathmoveto{\pgfqpoint{3.043326in}{2.141782in}}%
\pgfpathlineto{\pgfqpoint{3.306485in}{2.382499in}}%
\pgfpathlineto{\pgfqpoint{3.043326in}{2.141782in}}%
\pgfpathclose%
\pgfusepath{fill}%
\end{pgfscope}%
\begin{pgfscope}%
\pgfpathrectangle{\pgfqpoint{0.017500in}{0.017500in}}{\pgfqpoint{6.218282in}{3.580372in}} %
\pgfusepath{clip}%
\pgfsetbuttcap%
\pgfsetroundjoin%
\definecolor{currentfill}{rgb}{0.895885,0.433075,0.338681}%
\pgfsetfillcolor{currentfill}%
\pgfsetlinewidth{0.000000pt}%
\definecolor{currentstroke}{rgb}{0.000000,0.000000,0.000000}%
\pgfsetstrokecolor{currentstroke}%
\pgfsetdash{}{0pt}%
\pgfpathmoveto{\pgfqpoint{3.043326in}{2.141782in}}%
\pgfpathlineto{\pgfqpoint{3.306485in}{2.382499in}}%
\pgfpathlineto{\pgfqpoint{3.043326in}{2.141782in}}%
\pgfpathclose%
\pgfusepath{fill}%
\end{pgfscope}%
\begin{pgfscope}%
\pgfpathrectangle{\pgfqpoint{0.017500in}{0.017500in}}{\pgfqpoint{6.218282in}{3.580372in}} %
\pgfusepath{clip}%
\pgfsetbuttcap%
\pgfsetroundjoin%
\definecolor{currentfill}{rgb}{0.895885,0.433075,0.338681}%
\pgfsetfillcolor{currentfill}%
\pgfsetlinewidth{0.000000pt}%
\definecolor{currentstroke}{rgb}{0.000000,0.000000,0.000000}%
\pgfsetstrokecolor{currentstroke}%
\pgfsetdash{}{0pt}%
\pgfpathmoveto{\pgfqpoint{3.043326in}{2.141782in}}%
\pgfpathlineto{\pgfqpoint{3.306485in}{2.382499in}}%
\pgfpathlineto{\pgfqpoint{3.043326in}{2.141782in}}%
\pgfpathclose%
\pgfusepath{fill}%
\end{pgfscope}%
\begin{pgfscope}%
\pgfpathrectangle{\pgfqpoint{0.017500in}{0.017500in}}{\pgfqpoint{6.218282in}{3.580372in}} %
\pgfusepath{clip}%
\pgfsetbuttcap%
\pgfsetroundjoin%
\definecolor{currentfill}{rgb}{0.895885,0.433075,0.338681}%
\pgfsetfillcolor{currentfill}%
\pgfsetlinewidth{0.000000pt}%
\definecolor{currentstroke}{rgb}{0.000000,0.000000,0.000000}%
\pgfsetstrokecolor{currentstroke}%
\pgfsetdash{}{0pt}%
\pgfpathmoveto{\pgfqpoint{3.043326in}{2.141782in}}%
\pgfpathlineto{\pgfqpoint{3.306485in}{2.382499in}}%
\pgfpathlineto{\pgfqpoint{3.043326in}{2.141782in}}%
\pgfpathclose%
\pgfusepath{fill}%
\end{pgfscope}%
\begin{pgfscope}%
\pgfpathrectangle{\pgfqpoint{0.017500in}{0.017500in}}{\pgfqpoint{6.218282in}{3.580372in}} %
\pgfusepath{clip}%
\pgfsetbuttcap%
\pgfsetroundjoin%
\definecolor{currentfill}{rgb}{0.895885,0.433075,0.338681}%
\pgfsetfillcolor{currentfill}%
\pgfsetlinewidth{0.000000pt}%
\definecolor{currentstroke}{rgb}{0.000000,0.000000,0.000000}%
\pgfsetstrokecolor{currentstroke}%
\pgfsetdash{}{0pt}%
\pgfpathmoveto{\pgfqpoint{3.043326in}{2.141782in}}%
\pgfpathlineto{\pgfqpoint{3.306485in}{2.382499in}}%
\pgfpathlineto{\pgfqpoint{3.043326in}{2.141782in}}%
\pgfpathclose%
\pgfusepath{fill}%
\end{pgfscope}%
\begin{pgfscope}%
\pgfpathrectangle{\pgfqpoint{0.017500in}{0.017500in}}{\pgfqpoint{6.218282in}{3.580372in}} %
\pgfusepath{clip}%
\pgfsetbuttcap%
\pgfsetroundjoin%
\definecolor{currentfill}{rgb}{0.888390,0.417703,0.327898}%
\pgfsetfillcolor{currentfill}%
\pgfsetlinewidth{0.000000pt}%
\definecolor{currentstroke}{rgb}{0.000000,0.000000,0.000000}%
\pgfsetstrokecolor{currentstroke}%
\pgfsetdash{}{0pt}%
\pgfpathmoveto{\pgfqpoint{3.043326in}{2.141782in}}%
\pgfpathlineto{\pgfqpoint{3.502561in}{2.073396in}}%
\pgfpathlineto{\pgfqpoint{3.763398in}{2.297457in}}%
\pgfpathlineto{\pgfqpoint{3.306485in}{2.382499in}}%
\pgfpathlineto{\pgfqpoint{3.043326in}{2.141782in}}%
\pgfpathclose%
\pgfusepath{fill}%
\end{pgfscope}%
\begin{pgfscope}%
\pgfpathrectangle{\pgfqpoint{0.017500in}{0.017500in}}{\pgfqpoint{6.218282in}{3.580372in}} %
\pgfusepath{clip}%
\pgfsetbuttcap%
\pgfsetroundjoin%
\definecolor{currentfill}{rgb}{0.888390,0.417703,0.327898}%
\pgfsetfillcolor{currentfill}%
\pgfsetlinewidth{0.000000pt}%
\definecolor{currentstroke}{rgb}{0.000000,0.000000,0.000000}%
\pgfsetstrokecolor{currentstroke}%
\pgfsetdash{}{0pt}%
\pgfpathmoveto{\pgfqpoint{3.043326in}{2.141782in}}%
\pgfpathlineto{\pgfqpoint{3.502561in}{2.073396in}}%
\pgfpathlineto{\pgfqpoint{3.763398in}{2.297457in}}%
\pgfpathlineto{\pgfqpoint{3.306485in}{2.382499in}}%
\pgfpathlineto{\pgfqpoint{3.043326in}{2.141782in}}%
\pgfpathclose%
\pgfusepath{fill}%
\end{pgfscope}%
\begin{pgfscope}%
\pgfpathrectangle{\pgfqpoint{0.017500in}{0.017500in}}{\pgfqpoint{6.218282in}{3.580372in}} %
\pgfusepath{clip}%
\pgfsetbuttcap%
\pgfsetroundjoin%
\definecolor{currentfill}{rgb}{0.888390,0.417703,0.327898}%
\pgfsetfillcolor{currentfill}%
\pgfsetlinewidth{0.000000pt}%
\definecolor{currentstroke}{rgb}{0.000000,0.000000,0.000000}%
\pgfsetstrokecolor{currentstroke}%
\pgfsetdash{}{0pt}%
\pgfpathmoveto{\pgfqpoint{3.043326in}{2.141782in}}%
\pgfpathlineto{\pgfqpoint{3.502561in}{2.073396in}}%
\pgfpathlineto{\pgfqpoint{3.763398in}{2.297457in}}%
\pgfpathlineto{\pgfqpoint{3.306485in}{2.382499in}}%
\pgfpathlineto{\pgfqpoint{3.043326in}{2.141782in}}%
\pgfpathclose%
\pgfusepath{fill}%
\end{pgfscope}%
\begin{pgfscope}%
\pgfpathrectangle{\pgfqpoint{0.017500in}{0.017500in}}{\pgfqpoint{6.218282in}{3.580372in}} %
\pgfusepath{clip}%
\pgfsetbuttcap%
\pgfsetroundjoin%
\definecolor{currentfill}{rgb}{0.888390,0.417703,0.327898}%
\pgfsetfillcolor{currentfill}%
\pgfsetlinewidth{0.000000pt}%
\definecolor{currentstroke}{rgb}{0.000000,0.000000,0.000000}%
\pgfsetstrokecolor{currentstroke}%
\pgfsetdash{}{0pt}%
\pgfpathmoveto{\pgfqpoint{3.043326in}{2.141782in}}%
\pgfpathlineto{\pgfqpoint{3.502561in}{2.073396in}}%
\pgfpathlineto{\pgfqpoint{3.763398in}{2.297457in}}%
\pgfpathlineto{\pgfqpoint{3.306485in}{2.382499in}}%
\pgfpathlineto{\pgfqpoint{3.043326in}{2.141782in}}%
\pgfpathclose%
\pgfusepath{fill}%
\end{pgfscope}%
\begin{pgfscope}%
\pgfpathrectangle{\pgfqpoint{0.017500in}{0.017500in}}{\pgfqpoint{6.218282in}{3.580372in}} %
\pgfusepath{clip}%
\pgfsetbuttcap%
\pgfsetroundjoin%
\definecolor{currentfill}{rgb}{0.888390,0.417703,0.327898}%
\pgfsetfillcolor{currentfill}%
\pgfsetlinewidth{0.000000pt}%
\definecolor{currentstroke}{rgb}{0.000000,0.000000,0.000000}%
\pgfsetstrokecolor{currentstroke}%
\pgfsetdash{}{0pt}%
\pgfpathmoveto{\pgfqpoint{3.043326in}{2.141782in}}%
\pgfpathlineto{\pgfqpoint{3.502561in}{2.073396in}}%
\pgfpathlineto{\pgfqpoint{3.763398in}{2.297457in}}%
\pgfpathlineto{\pgfqpoint{3.306485in}{2.382499in}}%
\pgfpathlineto{\pgfqpoint{3.043326in}{2.141782in}}%
\pgfpathclose%
\pgfusepath{fill}%
\end{pgfscope}%
\begin{pgfscope}%
\pgfpathrectangle{\pgfqpoint{0.017500in}{0.017500in}}{\pgfqpoint{6.218282in}{3.580372in}} %
\pgfusepath{clip}%
\pgfsetbuttcap%
\pgfsetroundjoin%
\definecolor{currentfill}{rgb}{0.888390,0.417703,0.327898}%
\pgfsetfillcolor{currentfill}%
\pgfsetlinewidth{0.000000pt}%
\definecolor{currentstroke}{rgb}{0.000000,0.000000,0.000000}%
\pgfsetstrokecolor{currentstroke}%
\pgfsetdash{}{0pt}%
\pgfpathmoveto{\pgfqpoint{3.043326in}{2.141782in}}%
\pgfpathlineto{\pgfqpoint{3.502561in}{2.073396in}}%
\pgfpathlineto{\pgfqpoint{3.763398in}{2.297457in}}%
\pgfpathlineto{\pgfqpoint{3.306485in}{2.382499in}}%
\pgfpathlineto{\pgfqpoint{3.043326in}{2.141782in}}%
\pgfpathclose%
\pgfusepath{fill}%
\end{pgfscope}%
\begin{pgfscope}%
\pgfpathrectangle{\pgfqpoint{0.017500in}{0.017500in}}{\pgfqpoint{6.218282in}{3.580372in}} %
\pgfusepath{clip}%
\pgfsetbuttcap%
\pgfsetroundjoin%
\definecolor{currentfill}{rgb}{0.800830,0.250829,0.225696}%
\pgfsetfillcolor{currentfill}%
\pgfsetlinewidth{0.000000pt}%
\definecolor{currentstroke}{rgb}{0.000000,0.000000,0.000000}%
\pgfsetstrokecolor{currentstroke}%
\pgfsetdash{}{0pt}%
\pgfpathmoveto{\pgfqpoint{4.226975in}{2.236848in}}%
\pgfpathlineto{\pgfqpoint{4.478468in}{2.393972in}}%
\pgfpathlineto{\pgfqpoint{4.226975in}{2.236848in}}%
\pgfpathclose%
\pgfusepath{fill}%
\end{pgfscope}%
\begin{pgfscope}%
\pgfpathrectangle{\pgfqpoint{0.017500in}{0.017500in}}{\pgfqpoint{6.218282in}{3.580372in}} %
\pgfusepath{clip}%
\pgfsetbuttcap%
\pgfsetroundjoin%
\definecolor{currentfill}{rgb}{0.800830,0.250829,0.225696}%
\pgfsetfillcolor{currentfill}%
\pgfsetlinewidth{0.000000pt}%
\definecolor{currentstroke}{rgb}{0.000000,0.000000,0.000000}%
\pgfsetstrokecolor{currentstroke}%
\pgfsetdash{}{0pt}%
\pgfpathmoveto{\pgfqpoint{4.226975in}{2.236848in}}%
\pgfpathlineto{\pgfqpoint{4.478468in}{2.393972in}}%
\pgfpathlineto{\pgfqpoint{4.226975in}{2.236848in}}%
\pgfpathclose%
\pgfusepath{fill}%
\end{pgfscope}%
\begin{pgfscope}%
\pgfpathrectangle{\pgfqpoint{0.017500in}{0.017500in}}{\pgfqpoint{6.218282in}{3.580372in}} %
\pgfusepath{clip}%
\pgfsetbuttcap%
\pgfsetroundjoin%
\definecolor{currentfill}{rgb}{0.800830,0.250829,0.225696}%
\pgfsetfillcolor{currentfill}%
\pgfsetlinewidth{0.000000pt}%
\definecolor{currentstroke}{rgb}{0.000000,0.000000,0.000000}%
\pgfsetstrokecolor{currentstroke}%
\pgfsetdash{}{0pt}%
\pgfpathmoveto{\pgfqpoint{4.226975in}{2.236848in}}%
\pgfpathlineto{\pgfqpoint{4.478468in}{2.393972in}}%
\pgfpathlineto{\pgfqpoint{4.226975in}{2.236848in}}%
\pgfpathclose%
\pgfusepath{fill}%
\end{pgfscope}%
\begin{pgfscope}%
\pgfpathrectangle{\pgfqpoint{0.017500in}{0.017500in}}{\pgfqpoint{6.218282in}{3.580372in}} %
\pgfusepath{clip}%
\pgfsetbuttcap%
\pgfsetroundjoin%
\definecolor{currentfill}{rgb}{0.800830,0.250829,0.225696}%
\pgfsetfillcolor{currentfill}%
\pgfsetlinewidth{0.000000pt}%
\definecolor{currentstroke}{rgb}{0.000000,0.000000,0.000000}%
\pgfsetstrokecolor{currentstroke}%
\pgfsetdash{}{0pt}%
\pgfpathmoveto{\pgfqpoint{4.226975in}{2.236848in}}%
\pgfpathlineto{\pgfqpoint{4.478468in}{2.393972in}}%
\pgfpathlineto{\pgfqpoint{4.226975in}{2.236848in}}%
\pgfpathclose%
\pgfusepath{fill}%
\end{pgfscope}%
\begin{pgfscope}%
\pgfpathrectangle{\pgfqpoint{0.017500in}{0.017500in}}{\pgfqpoint{6.218282in}{3.580372in}} %
\pgfusepath{clip}%
\pgfsetbuttcap%
\pgfsetroundjoin%
\definecolor{currentfill}{rgb}{0.800830,0.250829,0.225696}%
\pgfsetfillcolor{currentfill}%
\pgfsetlinewidth{0.000000pt}%
\definecolor{currentstroke}{rgb}{0.000000,0.000000,0.000000}%
\pgfsetstrokecolor{currentstroke}%
\pgfsetdash{}{0pt}%
\pgfpathmoveto{\pgfqpoint{4.226975in}{2.236848in}}%
\pgfpathlineto{\pgfqpoint{4.478468in}{2.393972in}}%
\pgfpathlineto{\pgfqpoint{4.226975in}{2.236848in}}%
\pgfpathclose%
\pgfusepath{fill}%
\end{pgfscope}%
\begin{pgfscope}%
\pgfpathrectangle{\pgfqpoint{0.017500in}{0.017500in}}{\pgfqpoint{6.218282in}{3.580372in}} %
\pgfusepath{clip}%
\pgfsetbuttcap%
\pgfsetroundjoin%
\definecolor{currentfill}{rgb}{0.800830,0.250829,0.225696}%
\pgfsetfillcolor{currentfill}%
\pgfsetlinewidth{0.000000pt}%
\definecolor{currentstroke}{rgb}{0.000000,0.000000,0.000000}%
\pgfsetstrokecolor{currentstroke}%
\pgfsetdash{}{0pt}%
\pgfpathmoveto{\pgfqpoint{4.226975in}{2.236848in}}%
\pgfpathlineto{\pgfqpoint{4.478468in}{2.393972in}}%
\pgfpathlineto{\pgfqpoint{4.226975in}{2.236848in}}%
\pgfpathclose%
\pgfusepath{fill}%
\end{pgfscope}%
\begin{pgfscope}%
\pgfpathrectangle{\pgfqpoint{0.017500in}{0.017500in}}{\pgfqpoint{6.218282in}{3.580372in}} %
\pgfusepath{clip}%
\pgfsetbuttcap%
\pgfsetroundjoin%
\definecolor{currentfill}{rgb}{0.800830,0.250829,0.225696}%
\pgfsetfillcolor{currentfill}%
\pgfsetlinewidth{0.000000pt}%
\definecolor{currentstroke}{rgb}{0.000000,0.000000,0.000000}%
\pgfsetstrokecolor{currentstroke}%
\pgfsetdash{}{0pt}%
\pgfpathmoveto{\pgfqpoint{4.226975in}{2.236848in}}%
\pgfpathlineto{\pgfqpoint{4.478468in}{2.393972in}}%
\pgfpathlineto{\pgfqpoint{4.226975in}{2.236848in}}%
\pgfpathclose%
\pgfusepath{fill}%
\end{pgfscope}%
\begin{pgfscope}%
\pgfpathrectangle{\pgfqpoint{0.017500in}{0.017500in}}{\pgfqpoint{6.218282in}{3.580372in}} %
\pgfusepath{clip}%
\pgfsetbuttcap%
\pgfsetroundjoin%
\definecolor{currentfill}{rgb}{0.800830,0.250829,0.225696}%
\pgfsetfillcolor{currentfill}%
\pgfsetlinewidth{0.000000pt}%
\definecolor{currentstroke}{rgb}{0.000000,0.000000,0.000000}%
\pgfsetstrokecolor{currentstroke}%
\pgfsetdash{}{0pt}%
\pgfpathmoveto{\pgfqpoint{4.226975in}{2.236848in}}%
\pgfpathlineto{\pgfqpoint{4.478468in}{2.393972in}}%
\pgfpathlineto{\pgfqpoint{4.226975in}{2.236848in}}%
\pgfpathclose%
\pgfusepath{fill}%
\end{pgfscope}%
\begin{pgfscope}%
\pgfpathrectangle{\pgfqpoint{0.017500in}{0.017500in}}{\pgfqpoint{6.218282in}{3.580372in}} %
\pgfusepath{clip}%
\pgfsetbuttcap%
\pgfsetroundjoin%
\definecolor{currentfill}{rgb}{0.800830,0.250829,0.225696}%
\pgfsetfillcolor{currentfill}%
\pgfsetlinewidth{0.000000pt}%
\definecolor{currentstroke}{rgb}{0.000000,0.000000,0.000000}%
\pgfsetstrokecolor{currentstroke}%
\pgfsetdash{}{0pt}%
\pgfpathmoveto{\pgfqpoint{4.226975in}{2.236848in}}%
\pgfpathlineto{\pgfqpoint{4.478468in}{2.393972in}}%
\pgfpathlineto{\pgfqpoint{4.226975in}{2.236848in}}%
\pgfpathclose%
\pgfusepath{fill}%
\end{pgfscope}%
\begin{pgfscope}%
\pgfpathrectangle{\pgfqpoint{0.017500in}{0.017500in}}{\pgfqpoint{6.218282in}{3.580372in}} %
\pgfusepath{clip}%
\pgfsetbuttcap%
\pgfsetroundjoin%
\definecolor{currentfill}{rgb}{0.800830,0.250829,0.225696}%
\pgfsetfillcolor{currentfill}%
\pgfsetlinewidth{0.000000pt}%
\definecolor{currentstroke}{rgb}{0.000000,0.000000,0.000000}%
\pgfsetstrokecolor{currentstroke}%
\pgfsetdash{}{0pt}%
\pgfpathmoveto{\pgfqpoint{4.226975in}{2.236848in}}%
\pgfpathlineto{\pgfqpoint{4.478468in}{2.393972in}}%
\pgfpathlineto{\pgfqpoint{4.226975in}{2.236848in}}%
\pgfpathclose%
\pgfusepath{fill}%
\end{pgfscope}%
\begin{pgfscope}%
\pgfpathrectangle{\pgfqpoint{0.017500in}{0.017500in}}{\pgfqpoint{6.218282in}{3.580372in}} %
\pgfusepath{clip}%
\pgfsetbuttcap%
\pgfsetroundjoin%
\definecolor{currentfill}{rgb}{0.800830,0.250829,0.225696}%
\pgfsetfillcolor{currentfill}%
\pgfsetlinewidth{0.000000pt}%
\definecolor{currentstroke}{rgb}{0.000000,0.000000,0.000000}%
\pgfsetstrokecolor{currentstroke}%
\pgfsetdash{}{0pt}%
\pgfpathmoveto{\pgfqpoint{4.226975in}{2.236848in}}%
\pgfpathlineto{\pgfqpoint{4.478468in}{2.393972in}}%
\pgfpathlineto{\pgfqpoint{4.226975in}{2.236848in}}%
\pgfpathclose%
\pgfusepath{fill}%
\end{pgfscope}%
\begin{pgfscope}%
\pgfpathrectangle{\pgfqpoint{0.017500in}{0.017500in}}{\pgfqpoint{6.218282in}{3.580372in}} %
\pgfusepath{clip}%
\pgfsetbuttcap%
\pgfsetroundjoin%
\definecolor{currentfill}{rgb}{0.800830,0.250829,0.225696}%
\pgfsetfillcolor{currentfill}%
\pgfsetlinewidth{0.000000pt}%
\definecolor{currentstroke}{rgb}{0.000000,0.000000,0.000000}%
\pgfsetstrokecolor{currentstroke}%
\pgfsetdash{}{0pt}%
\pgfpathmoveto{\pgfqpoint{4.226975in}{2.236848in}}%
\pgfpathlineto{\pgfqpoint{4.478468in}{2.393972in}}%
\pgfpathlineto{\pgfqpoint{4.226975in}{2.236848in}}%
\pgfpathclose%
\pgfusepath{fill}%
\end{pgfscope}%
\begin{pgfscope}%
\pgfpathrectangle{\pgfqpoint{0.017500in}{0.017500in}}{\pgfqpoint{6.218282in}{3.580372in}} %
\pgfusepath{clip}%
\pgfsetbuttcap%
\pgfsetroundjoin%
\definecolor{currentfill}{rgb}{0.800830,0.250829,0.225696}%
\pgfsetfillcolor{currentfill}%
\pgfsetlinewidth{0.000000pt}%
\definecolor{currentstroke}{rgb}{0.000000,0.000000,0.000000}%
\pgfsetstrokecolor{currentstroke}%
\pgfsetdash{}{0pt}%
\pgfpathmoveto{\pgfqpoint{4.226975in}{2.236848in}}%
\pgfpathlineto{\pgfqpoint{4.478468in}{2.393972in}}%
\pgfpathlineto{\pgfqpoint{4.226975in}{2.236848in}}%
\pgfpathclose%
\pgfusepath{fill}%
\end{pgfscope}%
\begin{pgfscope}%
\pgfpathrectangle{\pgfqpoint{0.017500in}{0.017500in}}{\pgfqpoint{6.218282in}{3.580372in}} %
\pgfusepath{clip}%
\pgfsetbuttcap%
\pgfsetroundjoin%
\definecolor{currentfill}{rgb}{0.800830,0.250829,0.225696}%
\pgfsetfillcolor{currentfill}%
\pgfsetlinewidth{0.000000pt}%
\definecolor{currentstroke}{rgb}{0.000000,0.000000,0.000000}%
\pgfsetstrokecolor{currentstroke}%
\pgfsetdash{}{0pt}%
\pgfpathmoveto{\pgfqpoint{4.226975in}{2.236848in}}%
\pgfpathlineto{\pgfqpoint{4.478468in}{2.393972in}}%
\pgfpathlineto{\pgfqpoint{4.226975in}{2.236848in}}%
\pgfpathclose%
\pgfusepath{fill}%
\end{pgfscope}%
\begin{pgfscope}%
\pgfpathrectangle{\pgfqpoint{0.017500in}{0.017500in}}{\pgfqpoint{6.218282in}{3.580372in}} %
\pgfusepath{clip}%
\pgfsetbuttcap%
\pgfsetroundjoin%
\definecolor{currentfill}{rgb}{0.800830,0.250829,0.225696}%
\pgfsetfillcolor{currentfill}%
\pgfsetlinewidth{0.000000pt}%
\definecolor{currentstroke}{rgb}{0.000000,0.000000,0.000000}%
\pgfsetstrokecolor{currentstroke}%
\pgfsetdash{}{0pt}%
\pgfpathmoveto{\pgfqpoint{4.226975in}{2.236848in}}%
\pgfpathlineto{\pgfqpoint{4.478468in}{2.393972in}}%
\pgfpathlineto{\pgfqpoint{4.226975in}{2.236848in}}%
\pgfpathclose%
\pgfusepath{fill}%
\end{pgfscope}%
\begin{pgfscope}%
\pgfpathrectangle{\pgfqpoint{0.017500in}{0.017500in}}{\pgfqpoint{6.218282in}{3.580372in}} %
\pgfusepath{clip}%
\pgfsetbuttcap%
\pgfsetroundjoin%
\definecolor{currentfill}{rgb}{0.800830,0.250829,0.225696}%
\pgfsetfillcolor{currentfill}%
\pgfsetlinewidth{0.000000pt}%
\definecolor{currentstroke}{rgb}{0.000000,0.000000,0.000000}%
\pgfsetstrokecolor{currentstroke}%
\pgfsetdash{}{0pt}%
\pgfpathmoveto{\pgfqpoint{4.226975in}{2.236848in}}%
\pgfpathlineto{\pgfqpoint{4.478468in}{2.393972in}}%
\pgfpathlineto{\pgfqpoint{4.226975in}{2.236848in}}%
\pgfpathclose%
\pgfusepath{fill}%
\end{pgfscope}%
\begin{pgfscope}%
\pgfpathrectangle{\pgfqpoint{0.017500in}{0.017500in}}{\pgfqpoint{6.218282in}{3.580372in}} %
\pgfusepath{clip}%
\pgfsetbuttcap%
\pgfsetroundjoin%
\definecolor{currentfill}{rgb}{0.800830,0.250829,0.225696}%
\pgfsetfillcolor{currentfill}%
\pgfsetlinewidth{0.000000pt}%
\definecolor{currentstroke}{rgb}{0.000000,0.000000,0.000000}%
\pgfsetstrokecolor{currentstroke}%
\pgfsetdash{}{0pt}%
\pgfpathmoveto{\pgfqpoint{4.226975in}{2.236848in}}%
\pgfpathlineto{\pgfqpoint{4.478468in}{2.393972in}}%
\pgfpathlineto{\pgfqpoint{4.226975in}{2.236848in}}%
\pgfpathclose%
\pgfusepath{fill}%
\end{pgfscope}%
\begin{pgfscope}%
\pgfpathrectangle{\pgfqpoint{0.017500in}{0.017500in}}{\pgfqpoint{6.218282in}{3.580372in}} %
\pgfusepath{clip}%
\pgfsetbuttcap%
\pgfsetroundjoin%
\definecolor{currentfill}{rgb}{0.800830,0.250829,0.225696}%
\pgfsetfillcolor{currentfill}%
\pgfsetlinewidth{0.000000pt}%
\definecolor{currentstroke}{rgb}{0.000000,0.000000,0.000000}%
\pgfsetstrokecolor{currentstroke}%
\pgfsetdash{}{0pt}%
\pgfpathmoveto{\pgfqpoint{4.226975in}{2.236848in}}%
\pgfpathlineto{\pgfqpoint{4.478468in}{2.393972in}}%
\pgfpathlineto{\pgfqpoint{4.226975in}{2.236848in}}%
\pgfpathclose%
\pgfusepath{fill}%
\end{pgfscope}%
\begin{pgfscope}%
\pgfpathrectangle{\pgfqpoint{0.017500in}{0.017500in}}{\pgfqpoint{6.218282in}{3.580372in}} %
\pgfusepath{clip}%
\pgfsetbuttcap%
\pgfsetroundjoin%
\definecolor{currentfill}{rgb}{0.800830,0.250829,0.225696}%
\pgfsetfillcolor{currentfill}%
\pgfsetlinewidth{0.000000pt}%
\definecolor{currentstroke}{rgb}{0.000000,0.000000,0.000000}%
\pgfsetstrokecolor{currentstroke}%
\pgfsetdash{}{0pt}%
\pgfpathmoveto{\pgfqpoint{4.226975in}{2.236848in}}%
\pgfpathlineto{\pgfqpoint{4.478468in}{2.393972in}}%
\pgfpathlineto{\pgfqpoint{4.226975in}{2.236848in}}%
\pgfpathclose%
\pgfusepath{fill}%
\end{pgfscope}%
\begin{pgfscope}%
\pgfpathrectangle{\pgfqpoint{0.017500in}{0.017500in}}{\pgfqpoint{6.218282in}{3.580372in}} %
\pgfusepath{clip}%
\pgfsetbuttcap%
\pgfsetroundjoin%
\definecolor{currentfill}{rgb}{0.800830,0.250829,0.225696}%
\pgfsetfillcolor{currentfill}%
\pgfsetlinewidth{0.000000pt}%
\definecolor{currentstroke}{rgb}{0.000000,0.000000,0.000000}%
\pgfsetstrokecolor{currentstroke}%
\pgfsetdash{}{0pt}%
\pgfpathmoveto{\pgfqpoint{4.226975in}{2.236848in}}%
\pgfpathlineto{\pgfqpoint{4.478468in}{2.393972in}}%
\pgfpathlineto{\pgfqpoint{4.226975in}{2.236848in}}%
\pgfpathclose%
\pgfusepath{fill}%
\end{pgfscope}%
\begin{pgfscope}%
\pgfpathrectangle{\pgfqpoint{0.017500in}{0.017500in}}{\pgfqpoint{6.218282in}{3.580372in}} %
\pgfusepath{clip}%
\pgfsetbuttcap%
\pgfsetroundjoin%
\definecolor{currentfill}{rgb}{0.800830,0.250829,0.225696}%
\pgfsetfillcolor{currentfill}%
\pgfsetlinewidth{0.000000pt}%
\definecolor{currentstroke}{rgb}{0.000000,0.000000,0.000000}%
\pgfsetstrokecolor{currentstroke}%
\pgfsetdash{}{0pt}%
\pgfpathmoveto{\pgfqpoint{4.226975in}{2.236848in}}%
\pgfpathlineto{\pgfqpoint{4.478468in}{2.393972in}}%
\pgfpathlineto{\pgfqpoint{4.226975in}{2.236848in}}%
\pgfpathclose%
\pgfusepath{fill}%
\end{pgfscope}%
\begin{pgfscope}%
\pgfpathrectangle{\pgfqpoint{0.017500in}{0.017500in}}{\pgfqpoint{6.218282in}{3.580372in}} %
\pgfusepath{clip}%
\pgfsetbuttcap%
\pgfsetroundjoin%
\definecolor{currentfill}{rgb}{0.800830,0.250829,0.225696}%
\pgfsetfillcolor{currentfill}%
\pgfsetlinewidth{0.000000pt}%
\definecolor{currentstroke}{rgb}{0.000000,0.000000,0.000000}%
\pgfsetstrokecolor{currentstroke}%
\pgfsetdash{}{0pt}%
\pgfpathmoveto{\pgfqpoint{4.226975in}{2.236848in}}%
\pgfpathlineto{\pgfqpoint{4.478468in}{2.393972in}}%
\pgfpathlineto{\pgfqpoint{4.226975in}{2.236848in}}%
\pgfpathclose%
\pgfusepath{fill}%
\end{pgfscope}%
\begin{pgfscope}%
\pgfpathrectangle{\pgfqpoint{0.017500in}{0.017500in}}{\pgfqpoint{6.218282in}{3.580372in}} %
\pgfusepath{clip}%
\pgfsetbuttcap%
\pgfsetroundjoin%
\definecolor{currentfill}{rgb}{0.800830,0.250829,0.225696}%
\pgfsetfillcolor{currentfill}%
\pgfsetlinewidth{0.000000pt}%
\definecolor{currentstroke}{rgb}{0.000000,0.000000,0.000000}%
\pgfsetstrokecolor{currentstroke}%
\pgfsetdash{}{0pt}%
\pgfpathmoveto{\pgfqpoint{4.226975in}{2.236848in}}%
\pgfpathlineto{\pgfqpoint{4.478468in}{2.393972in}}%
\pgfpathlineto{\pgfqpoint{4.226975in}{2.236848in}}%
\pgfpathclose%
\pgfusepath{fill}%
\end{pgfscope}%
\begin{pgfscope}%
\pgfpathrectangle{\pgfqpoint{0.017500in}{0.017500in}}{\pgfqpoint{6.218282in}{3.580372in}} %
\pgfusepath{clip}%
\pgfsetbuttcap%
\pgfsetroundjoin%
\definecolor{currentfill}{rgb}{0.800830,0.250829,0.225696}%
\pgfsetfillcolor{currentfill}%
\pgfsetlinewidth{0.000000pt}%
\definecolor{currentstroke}{rgb}{0.000000,0.000000,0.000000}%
\pgfsetstrokecolor{currentstroke}%
\pgfsetdash{}{0pt}%
\pgfpathmoveto{\pgfqpoint{4.226975in}{2.236848in}}%
\pgfpathlineto{\pgfqpoint{4.478468in}{2.393972in}}%
\pgfpathlineto{\pgfqpoint{4.226975in}{2.236848in}}%
\pgfpathclose%
\pgfusepath{fill}%
\end{pgfscope}%
\begin{pgfscope}%
\pgfpathrectangle{\pgfqpoint{0.017500in}{0.017500in}}{\pgfqpoint{6.218282in}{3.580372in}} %
\pgfusepath{clip}%
\pgfsetbuttcap%
\pgfsetroundjoin%
\definecolor{currentfill}{rgb}{0.800830,0.250829,0.225696}%
\pgfsetfillcolor{currentfill}%
\pgfsetlinewidth{0.000000pt}%
\definecolor{currentstroke}{rgb}{0.000000,0.000000,0.000000}%
\pgfsetstrokecolor{currentstroke}%
\pgfsetdash{}{0pt}%
\pgfpathmoveto{\pgfqpoint{4.226975in}{2.236848in}}%
\pgfpathlineto{\pgfqpoint{4.478468in}{2.393972in}}%
\pgfpathlineto{\pgfqpoint{4.226975in}{2.236848in}}%
\pgfpathclose%
\pgfusepath{fill}%
\end{pgfscope}%
\begin{pgfscope}%
\pgfpathrectangle{\pgfqpoint{0.017500in}{0.017500in}}{\pgfqpoint{6.218282in}{3.580372in}} %
\pgfusepath{clip}%
\pgfsetbuttcap%
\pgfsetroundjoin%
\definecolor{currentfill}{rgb}{0.800830,0.250829,0.225696}%
\pgfsetfillcolor{currentfill}%
\pgfsetlinewidth{0.000000pt}%
\definecolor{currentstroke}{rgb}{0.000000,0.000000,0.000000}%
\pgfsetstrokecolor{currentstroke}%
\pgfsetdash{}{0pt}%
\pgfpathmoveto{\pgfqpoint{4.226975in}{2.236848in}}%
\pgfpathlineto{\pgfqpoint{4.478468in}{2.393972in}}%
\pgfpathlineto{\pgfqpoint{4.226975in}{2.236848in}}%
\pgfpathclose%
\pgfusepath{fill}%
\end{pgfscope}%
\begin{pgfscope}%
\pgfpathrectangle{\pgfqpoint{0.017500in}{0.017500in}}{\pgfqpoint{6.218282in}{3.580372in}} %
\pgfusepath{clip}%
\pgfsetbuttcap%
\pgfsetroundjoin%
\definecolor{currentfill}{rgb}{0.800830,0.250829,0.225696}%
\pgfsetfillcolor{currentfill}%
\pgfsetlinewidth{0.000000pt}%
\definecolor{currentstroke}{rgb}{0.000000,0.000000,0.000000}%
\pgfsetstrokecolor{currentstroke}%
\pgfsetdash{}{0pt}%
\pgfpathmoveto{\pgfqpoint{4.226975in}{2.236848in}}%
\pgfpathlineto{\pgfqpoint{4.478468in}{2.393972in}}%
\pgfpathlineto{\pgfqpoint{4.226975in}{2.236848in}}%
\pgfpathclose%
\pgfusepath{fill}%
\end{pgfscope}%
\begin{pgfscope}%
\pgfpathrectangle{\pgfqpoint{0.017500in}{0.017500in}}{\pgfqpoint{6.218282in}{3.580372in}} %
\pgfusepath{clip}%
\pgfsetbuttcap%
\pgfsetroundjoin%
\definecolor{currentfill}{rgb}{0.800830,0.250829,0.225696}%
\pgfsetfillcolor{currentfill}%
\pgfsetlinewidth{0.000000pt}%
\definecolor{currentstroke}{rgb}{0.000000,0.000000,0.000000}%
\pgfsetstrokecolor{currentstroke}%
\pgfsetdash{}{0pt}%
\pgfpathmoveto{\pgfqpoint{4.226975in}{2.236848in}}%
\pgfpathlineto{\pgfqpoint{4.478468in}{2.393972in}}%
\pgfpathlineto{\pgfqpoint{4.226975in}{2.236848in}}%
\pgfpathclose%
\pgfusepath{fill}%
\end{pgfscope}%
\begin{pgfscope}%
\pgfpathrectangle{\pgfqpoint{0.017500in}{0.017500in}}{\pgfqpoint{6.218282in}{3.580372in}} %
\pgfusepath{clip}%
\pgfsetbuttcap%
\pgfsetroundjoin%
\definecolor{currentfill}{rgb}{0.800830,0.250829,0.225696}%
\pgfsetfillcolor{currentfill}%
\pgfsetlinewidth{0.000000pt}%
\definecolor{currentstroke}{rgb}{0.000000,0.000000,0.000000}%
\pgfsetstrokecolor{currentstroke}%
\pgfsetdash{}{0pt}%
\pgfpathmoveto{\pgfqpoint{4.226975in}{2.236848in}}%
\pgfpathlineto{\pgfqpoint{4.478468in}{2.393972in}}%
\pgfpathlineto{\pgfqpoint{4.226975in}{2.236848in}}%
\pgfpathclose%
\pgfusepath{fill}%
\end{pgfscope}%
\begin{pgfscope}%
\pgfpathrectangle{\pgfqpoint{0.017500in}{0.017500in}}{\pgfqpoint{6.218282in}{3.580372in}} %
\pgfusepath{clip}%
\pgfsetbuttcap%
\pgfsetroundjoin%
\definecolor{currentfill}{rgb}{0.800830,0.250829,0.225696}%
\pgfsetfillcolor{currentfill}%
\pgfsetlinewidth{0.000000pt}%
\definecolor{currentstroke}{rgb}{0.000000,0.000000,0.000000}%
\pgfsetstrokecolor{currentstroke}%
\pgfsetdash{}{0pt}%
\pgfpathmoveto{\pgfqpoint{4.226975in}{2.236848in}}%
\pgfpathlineto{\pgfqpoint{4.478468in}{2.393972in}}%
\pgfpathlineto{\pgfqpoint{4.226975in}{2.236848in}}%
\pgfpathclose%
\pgfusepath{fill}%
\end{pgfscope}%
\begin{pgfscope}%
\pgfpathrectangle{\pgfqpoint{0.017500in}{0.017500in}}{\pgfqpoint{6.218282in}{3.580372in}} %
\pgfusepath{clip}%
\pgfsetbuttcap%
\pgfsetroundjoin%
\definecolor{currentfill}{rgb}{0.895885,0.433075,0.338681}%
\pgfsetfillcolor{currentfill}%
\pgfsetlinewidth{0.000000pt}%
\definecolor{currentstroke}{rgb}{0.000000,0.000000,0.000000}%
\pgfsetstrokecolor{currentstroke}%
\pgfsetdash{}{0pt}%
\pgfpathmoveto{\pgfqpoint{3.502561in}{2.073396in}}%
\pgfpathlineto{\pgfqpoint{3.763398in}{2.297457in}}%
\pgfpathlineto{\pgfqpoint{3.502561in}{2.073396in}}%
\pgfpathclose%
\pgfusepath{fill}%
\end{pgfscope}%
\begin{pgfscope}%
\pgfpathrectangle{\pgfqpoint{0.017500in}{0.017500in}}{\pgfqpoint{6.218282in}{3.580372in}} %
\pgfusepath{clip}%
\pgfsetbuttcap%
\pgfsetroundjoin%
\definecolor{currentfill}{rgb}{0.895885,0.433075,0.338681}%
\pgfsetfillcolor{currentfill}%
\pgfsetlinewidth{0.000000pt}%
\definecolor{currentstroke}{rgb}{0.000000,0.000000,0.000000}%
\pgfsetstrokecolor{currentstroke}%
\pgfsetdash{}{0pt}%
\pgfpathmoveto{\pgfqpoint{3.502561in}{2.073396in}}%
\pgfpathlineto{\pgfqpoint{3.763398in}{2.297457in}}%
\pgfpathlineto{\pgfqpoint{3.502561in}{2.073396in}}%
\pgfpathclose%
\pgfusepath{fill}%
\end{pgfscope}%
\begin{pgfscope}%
\pgfpathrectangle{\pgfqpoint{0.017500in}{0.017500in}}{\pgfqpoint{6.218282in}{3.580372in}} %
\pgfusepath{clip}%
\pgfsetbuttcap%
\pgfsetroundjoin%
\definecolor{currentfill}{rgb}{0.895885,0.433075,0.338681}%
\pgfsetfillcolor{currentfill}%
\pgfsetlinewidth{0.000000pt}%
\definecolor{currentstroke}{rgb}{0.000000,0.000000,0.000000}%
\pgfsetstrokecolor{currentstroke}%
\pgfsetdash{}{0pt}%
\pgfpathmoveto{\pgfqpoint{3.502561in}{2.073396in}}%
\pgfpathlineto{\pgfqpoint{3.763398in}{2.297457in}}%
\pgfpathlineto{\pgfqpoint{3.502561in}{2.073396in}}%
\pgfpathclose%
\pgfusepath{fill}%
\end{pgfscope}%
\begin{pgfscope}%
\pgfpathrectangle{\pgfqpoint{0.017500in}{0.017500in}}{\pgfqpoint{6.218282in}{3.580372in}} %
\pgfusepath{clip}%
\pgfsetbuttcap%
\pgfsetroundjoin%
\definecolor{currentfill}{rgb}{0.895885,0.433075,0.338681}%
\pgfsetfillcolor{currentfill}%
\pgfsetlinewidth{0.000000pt}%
\definecolor{currentstroke}{rgb}{0.000000,0.000000,0.000000}%
\pgfsetstrokecolor{currentstroke}%
\pgfsetdash{}{0pt}%
\pgfpathmoveto{\pgfqpoint{3.502561in}{2.073396in}}%
\pgfpathlineto{\pgfqpoint{3.763398in}{2.297457in}}%
\pgfpathlineto{\pgfqpoint{3.502561in}{2.073396in}}%
\pgfpathclose%
\pgfusepath{fill}%
\end{pgfscope}%
\begin{pgfscope}%
\pgfpathrectangle{\pgfqpoint{0.017500in}{0.017500in}}{\pgfqpoint{6.218282in}{3.580372in}} %
\pgfusepath{clip}%
\pgfsetbuttcap%
\pgfsetroundjoin%
\definecolor{currentfill}{rgb}{0.895885,0.433075,0.338681}%
\pgfsetfillcolor{currentfill}%
\pgfsetlinewidth{0.000000pt}%
\definecolor{currentstroke}{rgb}{0.000000,0.000000,0.000000}%
\pgfsetstrokecolor{currentstroke}%
\pgfsetdash{}{0pt}%
\pgfpathmoveto{\pgfqpoint{3.502561in}{2.073396in}}%
\pgfpathlineto{\pgfqpoint{3.763398in}{2.297457in}}%
\pgfpathlineto{\pgfqpoint{3.502561in}{2.073396in}}%
\pgfpathclose%
\pgfusepath{fill}%
\end{pgfscope}%
\begin{pgfscope}%
\pgfpathrectangle{\pgfqpoint{0.017500in}{0.017500in}}{\pgfqpoint{6.218282in}{3.580372in}} %
\pgfusepath{clip}%
\pgfsetbuttcap%
\pgfsetroundjoin%
\definecolor{currentfill}{rgb}{0.895885,0.433075,0.338681}%
\pgfsetfillcolor{currentfill}%
\pgfsetlinewidth{0.000000pt}%
\definecolor{currentstroke}{rgb}{0.000000,0.000000,0.000000}%
\pgfsetstrokecolor{currentstroke}%
\pgfsetdash{}{0pt}%
\pgfpathmoveto{\pgfqpoint{3.502561in}{2.073396in}}%
\pgfpathlineto{\pgfqpoint{3.763398in}{2.297457in}}%
\pgfpathlineto{\pgfqpoint{3.502561in}{2.073396in}}%
\pgfpathclose%
\pgfusepath{fill}%
\end{pgfscope}%
\begin{pgfscope}%
\pgfpathrectangle{\pgfqpoint{0.017500in}{0.017500in}}{\pgfqpoint{6.218282in}{3.580372in}} %
\pgfusepath{clip}%
\pgfsetbuttcap%
\pgfsetroundjoin%
\definecolor{currentfill}{rgb}{0.895885,0.433075,0.338681}%
\pgfsetfillcolor{currentfill}%
\pgfsetlinewidth{0.000000pt}%
\definecolor{currentstroke}{rgb}{0.000000,0.000000,0.000000}%
\pgfsetstrokecolor{currentstroke}%
\pgfsetdash{}{0pt}%
\pgfpathmoveto{\pgfqpoint{3.502561in}{2.073396in}}%
\pgfpathlineto{\pgfqpoint{3.763398in}{2.297457in}}%
\pgfpathlineto{\pgfqpoint{3.502561in}{2.073396in}}%
\pgfpathclose%
\pgfusepath{fill}%
\end{pgfscope}%
\begin{pgfscope}%
\pgfpathrectangle{\pgfqpoint{0.017500in}{0.017500in}}{\pgfqpoint{6.218282in}{3.580372in}} %
\pgfusepath{clip}%
\pgfsetbuttcap%
\pgfsetroundjoin%
\definecolor{currentfill}{rgb}{0.895885,0.433075,0.338681}%
\pgfsetfillcolor{currentfill}%
\pgfsetlinewidth{0.000000pt}%
\definecolor{currentstroke}{rgb}{0.000000,0.000000,0.000000}%
\pgfsetstrokecolor{currentstroke}%
\pgfsetdash{}{0pt}%
\pgfpathmoveto{\pgfqpoint{3.502561in}{2.073396in}}%
\pgfpathlineto{\pgfqpoint{3.763398in}{2.297457in}}%
\pgfpathlineto{\pgfqpoint{3.502561in}{2.073396in}}%
\pgfpathclose%
\pgfusepath{fill}%
\end{pgfscope}%
\begin{pgfscope}%
\pgfpathrectangle{\pgfqpoint{0.017500in}{0.017500in}}{\pgfqpoint{6.218282in}{3.580372in}} %
\pgfusepath{clip}%
\pgfsetbuttcap%
\pgfsetroundjoin%
\definecolor{currentfill}{rgb}{0.895885,0.433075,0.338681}%
\pgfsetfillcolor{currentfill}%
\pgfsetlinewidth{0.000000pt}%
\definecolor{currentstroke}{rgb}{0.000000,0.000000,0.000000}%
\pgfsetstrokecolor{currentstroke}%
\pgfsetdash{}{0pt}%
\pgfpathmoveto{\pgfqpoint{3.502561in}{2.073396in}}%
\pgfpathlineto{\pgfqpoint{3.763398in}{2.297457in}}%
\pgfpathlineto{\pgfqpoint{3.502561in}{2.073396in}}%
\pgfpathclose%
\pgfusepath{fill}%
\end{pgfscope}%
\begin{pgfscope}%
\pgfpathrectangle{\pgfqpoint{0.017500in}{0.017500in}}{\pgfqpoint{6.218282in}{3.580372in}} %
\pgfusepath{clip}%
\pgfsetbuttcap%
\pgfsetroundjoin%
\definecolor{currentfill}{rgb}{0.895885,0.433075,0.338681}%
\pgfsetfillcolor{currentfill}%
\pgfsetlinewidth{0.000000pt}%
\definecolor{currentstroke}{rgb}{0.000000,0.000000,0.000000}%
\pgfsetstrokecolor{currentstroke}%
\pgfsetdash{}{0pt}%
\pgfpathmoveto{\pgfqpoint{3.502561in}{2.073396in}}%
\pgfpathlineto{\pgfqpoint{3.763398in}{2.297457in}}%
\pgfpathlineto{\pgfqpoint{3.502561in}{2.073396in}}%
\pgfpathclose%
\pgfusepath{fill}%
\end{pgfscope}%
\begin{pgfscope}%
\pgfpathrectangle{\pgfqpoint{0.017500in}{0.017500in}}{\pgfqpoint{6.218282in}{3.580372in}} %
\pgfusepath{clip}%
\pgfsetbuttcap%
\pgfsetroundjoin%
\definecolor{currentfill}{rgb}{0.895885,0.433075,0.338681}%
\pgfsetfillcolor{currentfill}%
\pgfsetlinewidth{0.000000pt}%
\definecolor{currentstroke}{rgb}{0.000000,0.000000,0.000000}%
\pgfsetstrokecolor{currentstroke}%
\pgfsetdash{}{0pt}%
\pgfpathmoveto{\pgfqpoint{3.502561in}{2.073396in}}%
\pgfpathlineto{\pgfqpoint{3.763398in}{2.297457in}}%
\pgfpathlineto{\pgfqpoint{3.502561in}{2.073396in}}%
\pgfpathclose%
\pgfusepath{fill}%
\end{pgfscope}%
\begin{pgfscope}%
\pgfpathrectangle{\pgfqpoint{0.017500in}{0.017500in}}{\pgfqpoint{6.218282in}{3.580372in}} %
\pgfusepath{clip}%
\pgfsetbuttcap%
\pgfsetroundjoin%
\definecolor{currentfill}{rgb}{0.895885,0.433075,0.338681}%
\pgfsetfillcolor{currentfill}%
\pgfsetlinewidth{0.000000pt}%
\definecolor{currentstroke}{rgb}{0.000000,0.000000,0.000000}%
\pgfsetstrokecolor{currentstroke}%
\pgfsetdash{}{0pt}%
\pgfpathmoveto{\pgfqpoint{3.502561in}{2.073396in}}%
\pgfpathlineto{\pgfqpoint{3.763398in}{2.297457in}}%
\pgfpathlineto{\pgfqpoint{3.502561in}{2.073396in}}%
\pgfpathclose%
\pgfusepath{fill}%
\end{pgfscope}%
\begin{pgfscope}%
\pgfpathrectangle{\pgfqpoint{0.017500in}{0.017500in}}{\pgfqpoint{6.218282in}{3.580372in}} %
\pgfusepath{clip}%
\pgfsetbuttcap%
\pgfsetroundjoin%
\definecolor{currentfill}{rgb}{0.895885,0.433075,0.338681}%
\pgfsetfillcolor{currentfill}%
\pgfsetlinewidth{0.000000pt}%
\definecolor{currentstroke}{rgb}{0.000000,0.000000,0.000000}%
\pgfsetstrokecolor{currentstroke}%
\pgfsetdash{}{0pt}%
\pgfpathmoveto{\pgfqpoint{3.502561in}{2.073396in}}%
\pgfpathlineto{\pgfqpoint{3.763398in}{2.297457in}}%
\pgfpathlineto{\pgfqpoint{3.502561in}{2.073396in}}%
\pgfpathclose%
\pgfusepath{fill}%
\end{pgfscope}%
\begin{pgfscope}%
\pgfpathrectangle{\pgfqpoint{0.017500in}{0.017500in}}{\pgfqpoint{6.218282in}{3.580372in}} %
\pgfusepath{clip}%
\pgfsetbuttcap%
\pgfsetroundjoin%
\definecolor{currentfill}{rgb}{0.895885,0.433075,0.338681}%
\pgfsetfillcolor{currentfill}%
\pgfsetlinewidth{0.000000pt}%
\definecolor{currentstroke}{rgb}{0.000000,0.000000,0.000000}%
\pgfsetstrokecolor{currentstroke}%
\pgfsetdash{}{0pt}%
\pgfpathmoveto{\pgfqpoint{3.502561in}{2.073396in}}%
\pgfpathlineto{\pgfqpoint{3.763398in}{2.297457in}}%
\pgfpathlineto{\pgfqpoint{3.502561in}{2.073396in}}%
\pgfpathclose%
\pgfusepath{fill}%
\end{pgfscope}%
\begin{pgfscope}%
\pgfpathrectangle{\pgfqpoint{0.017500in}{0.017500in}}{\pgfqpoint{6.218282in}{3.580372in}} %
\pgfusepath{clip}%
\pgfsetbuttcap%
\pgfsetroundjoin%
\definecolor{currentfill}{rgb}{0.895885,0.433075,0.338681}%
\pgfsetfillcolor{currentfill}%
\pgfsetlinewidth{0.000000pt}%
\definecolor{currentstroke}{rgb}{0.000000,0.000000,0.000000}%
\pgfsetstrokecolor{currentstroke}%
\pgfsetdash{}{0pt}%
\pgfpathmoveto{\pgfqpoint{3.502561in}{2.073396in}}%
\pgfpathlineto{\pgfqpoint{3.763398in}{2.297457in}}%
\pgfpathlineto{\pgfqpoint{3.502561in}{2.073396in}}%
\pgfpathclose%
\pgfusepath{fill}%
\end{pgfscope}%
\begin{pgfscope}%
\pgfpathrectangle{\pgfqpoint{0.017500in}{0.017500in}}{\pgfqpoint{6.218282in}{3.580372in}} %
\pgfusepath{clip}%
\pgfsetbuttcap%
\pgfsetroundjoin%
\definecolor{currentfill}{rgb}{0.895885,0.433075,0.338681}%
\pgfsetfillcolor{currentfill}%
\pgfsetlinewidth{0.000000pt}%
\definecolor{currentstroke}{rgb}{0.000000,0.000000,0.000000}%
\pgfsetstrokecolor{currentstroke}%
\pgfsetdash{}{0pt}%
\pgfpathmoveto{\pgfqpoint{3.502561in}{2.073396in}}%
\pgfpathlineto{\pgfqpoint{3.763398in}{2.297457in}}%
\pgfpathlineto{\pgfqpoint{3.502561in}{2.073396in}}%
\pgfpathclose%
\pgfusepath{fill}%
\end{pgfscope}%
\begin{pgfscope}%
\pgfpathrectangle{\pgfqpoint{0.017500in}{0.017500in}}{\pgfqpoint{6.218282in}{3.580372in}} %
\pgfusepath{clip}%
\pgfsetbuttcap%
\pgfsetroundjoin%
\definecolor{currentfill}{rgb}{0.895885,0.433075,0.338681}%
\pgfsetfillcolor{currentfill}%
\pgfsetlinewidth{0.000000pt}%
\definecolor{currentstroke}{rgb}{0.000000,0.000000,0.000000}%
\pgfsetstrokecolor{currentstroke}%
\pgfsetdash{}{0pt}%
\pgfpathmoveto{\pgfqpoint{3.502561in}{2.073396in}}%
\pgfpathlineto{\pgfqpoint{3.763398in}{2.297457in}}%
\pgfpathlineto{\pgfqpoint{3.502561in}{2.073396in}}%
\pgfpathclose%
\pgfusepath{fill}%
\end{pgfscope}%
\begin{pgfscope}%
\pgfpathrectangle{\pgfqpoint{0.017500in}{0.017500in}}{\pgfqpoint{6.218282in}{3.580372in}} %
\pgfusepath{clip}%
\pgfsetbuttcap%
\pgfsetroundjoin%
\definecolor{currentfill}{rgb}{0.895885,0.433075,0.338681}%
\pgfsetfillcolor{currentfill}%
\pgfsetlinewidth{0.000000pt}%
\definecolor{currentstroke}{rgb}{0.000000,0.000000,0.000000}%
\pgfsetstrokecolor{currentstroke}%
\pgfsetdash{}{0pt}%
\pgfpathmoveto{\pgfqpoint{3.502561in}{2.073396in}}%
\pgfpathlineto{\pgfqpoint{3.763398in}{2.297457in}}%
\pgfpathlineto{\pgfqpoint{3.502561in}{2.073396in}}%
\pgfpathclose%
\pgfusepath{fill}%
\end{pgfscope}%
\begin{pgfscope}%
\pgfpathrectangle{\pgfqpoint{0.017500in}{0.017500in}}{\pgfqpoint{6.218282in}{3.580372in}} %
\pgfusepath{clip}%
\pgfsetbuttcap%
\pgfsetroundjoin%
\definecolor{currentfill}{rgb}{0.895885,0.433075,0.338681}%
\pgfsetfillcolor{currentfill}%
\pgfsetlinewidth{0.000000pt}%
\definecolor{currentstroke}{rgb}{0.000000,0.000000,0.000000}%
\pgfsetstrokecolor{currentstroke}%
\pgfsetdash{}{0pt}%
\pgfpathmoveto{\pgfqpoint{3.502561in}{2.073396in}}%
\pgfpathlineto{\pgfqpoint{3.763398in}{2.297457in}}%
\pgfpathlineto{\pgfqpoint{3.502561in}{2.073396in}}%
\pgfpathclose%
\pgfusepath{fill}%
\end{pgfscope}%
\begin{pgfscope}%
\pgfpathrectangle{\pgfqpoint{0.017500in}{0.017500in}}{\pgfqpoint{6.218282in}{3.580372in}} %
\pgfusepath{clip}%
\pgfsetbuttcap%
\pgfsetroundjoin%
\definecolor{currentfill}{rgb}{0.895885,0.433075,0.338681}%
\pgfsetfillcolor{currentfill}%
\pgfsetlinewidth{0.000000pt}%
\definecolor{currentstroke}{rgb}{0.000000,0.000000,0.000000}%
\pgfsetstrokecolor{currentstroke}%
\pgfsetdash{}{0pt}%
\pgfpathmoveto{\pgfqpoint{3.502561in}{2.073396in}}%
\pgfpathlineto{\pgfqpoint{3.763398in}{2.297457in}}%
\pgfpathlineto{\pgfqpoint{3.502561in}{2.073396in}}%
\pgfpathclose%
\pgfusepath{fill}%
\end{pgfscope}%
\begin{pgfscope}%
\pgfpathrectangle{\pgfqpoint{0.017500in}{0.017500in}}{\pgfqpoint{6.218282in}{3.580372in}} %
\pgfusepath{clip}%
\pgfsetbuttcap%
\pgfsetroundjoin%
\definecolor{currentfill}{rgb}{0.895885,0.433075,0.338681}%
\pgfsetfillcolor{currentfill}%
\pgfsetlinewidth{0.000000pt}%
\definecolor{currentstroke}{rgb}{0.000000,0.000000,0.000000}%
\pgfsetstrokecolor{currentstroke}%
\pgfsetdash{}{0pt}%
\pgfpathmoveto{\pgfqpoint{3.502561in}{2.073396in}}%
\pgfpathlineto{\pgfqpoint{3.763398in}{2.297457in}}%
\pgfpathlineto{\pgfqpoint{3.502561in}{2.073396in}}%
\pgfpathclose%
\pgfusepath{fill}%
\end{pgfscope}%
\begin{pgfscope}%
\pgfpathrectangle{\pgfqpoint{0.017500in}{0.017500in}}{\pgfqpoint{6.218282in}{3.580372in}} %
\pgfusepath{clip}%
\pgfsetbuttcap%
\pgfsetroundjoin%
\definecolor{currentfill}{rgb}{0.895885,0.433075,0.338681}%
\pgfsetfillcolor{currentfill}%
\pgfsetlinewidth{0.000000pt}%
\definecolor{currentstroke}{rgb}{0.000000,0.000000,0.000000}%
\pgfsetstrokecolor{currentstroke}%
\pgfsetdash{}{0pt}%
\pgfpathmoveto{\pgfqpoint{3.502561in}{2.073396in}}%
\pgfpathlineto{\pgfqpoint{3.763398in}{2.297457in}}%
\pgfpathlineto{\pgfqpoint{3.502561in}{2.073396in}}%
\pgfpathclose%
\pgfusepath{fill}%
\end{pgfscope}%
\begin{pgfscope}%
\pgfpathrectangle{\pgfqpoint{0.017500in}{0.017500in}}{\pgfqpoint{6.218282in}{3.580372in}} %
\pgfusepath{clip}%
\pgfsetbuttcap%
\pgfsetroundjoin%
\definecolor{currentfill}{rgb}{0.895885,0.433075,0.338681}%
\pgfsetfillcolor{currentfill}%
\pgfsetlinewidth{0.000000pt}%
\definecolor{currentstroke}{rgb}{0.000000,0.000000,0.000000}%
\pgfsetstrokecolor{currentstroke}%
\pgfsetdash{}{0pt}%
\pgfpathmoveto{\pgfqpoint{3.502561in}{2.073396in}}%
\pgfpathlineto{\pgfqpoint{3.763398in}{2.297457in}}%
\pgfpathlineto{\pgfqpoint{3.502561in}{2.073396in}}%
\pgfpathclose%
\pgfusepath{fill}%
\end{pgfscope}%
\begin{pgfscope}%
\pgfpathrectangle{\pgfqpoint{0.017500in}{0.017500in}}{\pgfqpoint{6.218282in}{3.580372in}} %
\pgfusepath{clip}%
\pgfsetbuttcap%
\pgfsetroundjoin%
\definecolor{currentfill}{rgb}{0.895885,0.433075,0.338681}%
\pgfsetfillcolor{currentfill}%
\pgfsetlinewidth{0.000000pt}%
\definecolor{currentstroke}{rgb}{0.000000,0.000000,0.000000}%
\pgfsetstrokecolor{currentstroke}%
\pgfsetdash{}{0pt}%
\pgfpathmoveto{\pgfqpoint{3.502561in}{2.073396in}}%
\pgfpathlineto{\pgfqpoint{3.763398in}{2.297457in}}%
\pgfpathlineto{\pgfqpoint{3.502561in}{2.073396in}}%
\pgfpathclose%
\pgfusepath{fill}%
\end{pgfscope}%
\begin{pgfscope}%
\pgfpathrectangle{\pgfqpoint{0.017500in}{0.017500in}}{\pgfqpoint{6.218282in}{3.580372in}} %
\pgfusepath{clip}%
\pgfsetbuttcap%
\pgfsetroundjoin%
\definecolor{currentfill}{rgb}{0.895885,0.433075,0.338681}%
\pgfsetfillcolor{currentfill}%
\pgfsetlinewidth{0.000000pt}%
\definecolor{currentstroke}{rgb}{0.000000,0.000000,0.000000}%
\pgfsetstrokecolor{currentstroke}%
\pgfsetdash{}{0pt}%
\pgfpathmoveto{\pgfqpoint{3.502561in}{2.073396in}}%
\pgfpathlineto{\pgfqpoint{3.763398in}{2.297457in}}%
\pgfpathlineto{\pgfqpoint{3.502561in}{2.073396in}}%
\pgfpathclose%
\pgfusepath{fill}%
\end{pgfscope}%
\begin{pgfscope}%
\pgfpathrectangle{\pgfqpoint{0.017500in}{0.017500in}}{\pgfqpoint{6.218282in}{3.580372in}} %
\pgfusepath{clip}%
\pgfsetbuttcap%
\pgfsetroundjoin%
\definecolor{currentfill}{rgb}{0.895885,0.433075,0.338681}%
\pgfsetfillcolor{currentfill}%
\pgfsetlinewidth{0.000000pt}%
\definecolor{currentstroke}{rgb}{0.000000,0.000000,0.000000}%
\pgfsetstrokecolor{currentstroke}%
\pgfsetdash{}{0pt}%
\pgfpathmoveto{\pgfqpoint{3.502561in}{2.073396in}}%
\pgfpathlineto{\pgfqpoint{3.763398in}{2.297457in}}%
\pgfpathlineto{\pgfqpoint{3.502561in}{2.073396in}}%
\pgfpathclose%
\pgfusepath{fill}%
\end{pgfscope}%
\begin{pgfscope}%
\pgfpathrectangle{\pgfqpoint{0.017500in}{0.017500in}}{\pgfqpoint{6.218282in}{3.580372in}} %
\pgfusepath{clip}%
\pgfsetbuttcap%
\pgfsetroundjoin%
\definecolor{currentfill}{rgb}{0.895885,0.433075,0.338681}%
\pgfsetfillcolor{currentfill}%
\pgfsetlinewidth{0.000000pt}%
\definecolor{currentstroke}{rgb}{0.000000,0.000000,0.000000}%
\pgfsetstrokecolor{currentstroke}%
\pgfsetdash{}{0pt}%
\pgfpathmoveto{\pgfqpoint{3.502561in}{2.073396in}}%
\pgfpathlineto{\pgfqpoint{3.763398in}{2.297457in}}%
\pgfpathlineto{\pgfqpoint{3.502561in}{2.073396in}}%
\pgfpathclose%
\pgfusepath{fill}%
\end{pgfscope}%
\begin{pgfscope}%
\pgfpathrectangle{\pgfqpoint{0.017500in}{0.017500in}}{\pgfqpoint{6.218282in}{3.580372in}} %
\pgfusepath{clip}%
\pgfsetbuttcap%
\pgfsetroundjoin%
\definecolor{currentfill}{rgb}{0.895885,0.433075,0.338681}%
\pgfsetfillcolor{currentfill}%
\pgfsetlinewidth{0.000000pt}%
\definecolor{currentstroke}{rgb}{0.000000,0.000000,0.000000}%
\pgfsetstrokecolor{currentstroke}%
\pgfsetdash{}{0pt}%
\pgfpathmoveto{\pgfqpoint{3.502561in}{2.073396in}}%
\pgfpathlineto{\pgfqpoint{3.763398in}{2.297457in}}%
\pgfpathlineto{\pgfqpoint{3.502561in}{2.073396in}}%
\pgfpathclose%
\pgfusepath{fill}%
\end{pgfscope}%
\begin{pgfscope}%
\pgfpathrectangle{\pgfqpoint{0.017500in}{0.017500in}}{\pgfqpoint{6.218282in}{3.580372in}} %
\pgfusepath{clip}%
\pgfsetbuttcap%
\pgfsetroundjoin%
\definecolor{currentfill}{rgb}{0.895885,0.433075,0.338681}%
\pgfsetfillcolor{currentfill}%
\pgfsetlinewidth{0.000000pt}%
\definecolor{currentstroke}{rgb}{0.000000,0.000000,0.000000}%
\pgfsetstrokecolor{currentstroke}%
\pgfsetdash{}{0pt}%
\pgfpathmoveto{\pgfqpoint{3.502561in}{2.073396in}}%
\pgfpathlineto{\pgfqpoint{3.763398in}{2.297457in}}%
\pgfpathlineto{\pgfqpoint{3.502561in}{2.073396in}}%
\pgfpathclose%
\pgfusepath{fill}%
\end{pgfscope}%
\begin{pgfscope}%
\pgfpathrectangle{\pgfqpoint{0.017500in}{0.017500in}}{\pgfqpoint{6.218282in}{3.580372in}} %
\pgfusepath{clip}%
\pgfsetbuttcap%
\pgfsetroundjoin%
\definecolor{currentfill}{rgb}{0.895885,0.433075,0.338681}%
\pgfsetfillcolor{currentfill}%
\pgfsetlinewidth{0.000000pt}%
\definecolor{currentstroke}{rgb}{0.000000,0.000000,0.000000}%
\pgfsetstrokecolor{currentstroke}%
\pgfsetdash{}{0pt}%
\pgfpathmoveto{\pgfqpoint{3.502561in}{2.073396in}}%
\pgfpathlineto{\pgfqpoint{3.763398in}{2.297457in}}%
\pgfpathlineto{\pgfqpoint{3.502561in}{2.073396in}}%
\pgfpathclose%
\pgfusepath{fill}%
\end{pgfscope}%
\begin{pgfscope}%
\pgfpathrectangle{\pgfqpoint{0.017500in}{0.017500in}}{\pgfqpoint{6.218282in}{3.580372in}} %
\pgfusepath{clip}%
\pgfsetbuttcap%
\pgfsetroundjoin%
\definecolor{currentfill}{rgb}{0.888390,0.417703,0.327898}%
\pgfsetfillcolor{currentfill}%
\pgfsetlinewidth{0.000000pt}%
\definecolor{currentstroke}{rgb}{0.000000,0.000000,0.000000}%
\pgfsetstrokecolor{currentstroke}%
\pgfsetdash{}{0pt}%
\pgfpathmoveto{\pgfqpoint{3.502561in}{2.073396in}}%
\pgfpathlineto{\pgfqpoint{3.968076in}{1.994042in}}%
\pgfpathlineto{\pgfqpoint{4.226975in}{2.236848in}}%
\pgfpathlineto{\pgfqpoint{3.763398in}{2.297457in}}%
\pgfpathlineto{\pgfqpoint{3.502561in}{2.073396in}}%
\pgfpathclose%
\pgfusepath{fill}%
\end{pgfscope}%
\begin{pgfscope}%
\pgfpathrectangle{\pgfqpoint{0.017500in}{0.017500in}}{\pgfqpoint{6.218282in}{3.580372in}} %
\pgfusepath{clip}%
\pgfsetbuttcap%
\pgfsetroundjoin%
\definecolor{currentfill}{rgb}{0.888390,0.417703,0.327898}%
\pgfsetfillcolor{currentfill}%
\pgfsetlinewidth{0.000000pt}%
\definecolor{currentstroke}{rgb}{0.000000,0.000000,0.000000}%
\pgfsetstrokecolor{currentstroke}%
\pgfsetdash{}{0pt}%
\pgfpathmoveto{\pgfqpoint{3.502561in}{2.073396in}}%
\pgfpathlineto{\pgfqpoint{3.968076in}{1.994042in}}%
\pgfpathlineto{\pgfqpoint{4.226975in}{2.236848in}}%
\pgfpathlineto{\pgfqpoint{3.763398in}{2.297457in}}%
\pgfpathlineto{\pgfqpoint{3.502561in}{2.073396in}}%
\pgfpathclose%
\pgfusepath{fill}%
\end{pgfscope}%
\begin{pgfscope}%
\pgfpathrectangle{\pgfqpoint{0.017500in}{0.017500in}}{\pgfqpoint{6.218282in}{3.580372in}} %
\pgfusepath{clip}%
\pgfsetbuttcap%
\pgfsetroundjoin%
\definecolor{currentfill}{rgb}{0.888390,0.417703,0.327898}%
\pgfsetfillcolor{currentfill}%
\pgfsetlinewidth{0.000000pt}%
\definecolor{currentstroke}{rgb}{0.000000,0.000000,0.000000}%
\pgfsetstrokecolor{currentstroke}%
\pgfsetdash{}{0pt}%
\pgfpathmoveto{\pgfqpoint{3.502561in}{2.073396in}}%
\pgfpathlineto{\pgfqpoint{3.968076in}{1.994042in}}%
\pgfpathlineto{\pgfqpoint{4.226975in}{2.236848in}}%
\pgfpathlineto{\pgfqpoint{3.763398in}{2.297457in}}%
\pgfpathlineto{\pgfqpoint{3.502561in}{2.073396in}}%
\pgfpathclose%
\pgfusepath{fill}%
\end{pgfscope}%
\begin{pgfscope}%
\pgfpathrectangle{\pgfqpoint{0.017500in}{0.017500in}}{\pgfqpoint{6.218282in}{3.580372in}} %
\pgfusepath{clip}%
\pgfsetbuttcap%
\pgfsetroundjoin%
\definecolor{currentfill}{rgb}{0.888390,0.417703,0.327898}%
\pgfsetfillcolor{currentfill}%
\pgfsetlinewidth{0.000000pt}%
\definecolor{currentstroke}{rgb}{0.000000,0.000000,0.000000}%
\pgfsetstrokecolor{currentstroke}%
\pgfsetdash{}{0pt}%
\pgfpathmoveto{\pgfqpoint{3.502561in}{2.073396in}}%
\pgfpathlineto{\pgfqpoint{3.968076in}{1.994042in}}%
\pgfpathlineto{\pgfqpoint{4.226975in}{2.236848in}}%
\pgfpathlineto{\pgfqpoint{3.763398in}{2.297457in}}%
\pgfpathlineto{\pgfqpoint{3.502561in}{2.073396in}}%
\pgfpathclose%
\pgfusepath{fill}%
\end{pgfscope}%
\begin{pgfscope}%
\pgfpathrectangle{\pgfqpoint{0.017500in}{0.017500in}}{\pgfqpoint{6.218282in}{3.580372in}} %
\pgfusepath{clip}%
\pgfsetbuttcap%
\pgfsetroundjoin%
\definecolor{currentfill}{rgb}{0.888390,0.417703,0.327898}%
\pgfsetfillcolor{currentfill}%
\pgfsetlinewidth{0.000000pt}%
\definecolor{currentstroke}{rgb}{0.000000,0.000000,0.000000}%
\pgfsetstrokecolor{currentstroke}%
\pgfsetdash{}{0pt}%
\pgfpathmoveto{\pgfqpoint{3.502561in}{2.073396in}}%
\pgfpathlineto{\pgfqpoint{3.968076in}{1.994042in}}%
\pgfpathlineto{\pgfqpoint{4.226975in}{2.236848in}}%
\pgfpathlineto{\pgfqpoint{3.763398in}{2.297457in}}%
\pgfpathlineto{\pgfqpoint{3.502561in}{2.073396in}}%
\pgfpathclose%
\pgfusepath{fill}%
\end{pgfscope}%
\begin{pgfscope}%
\pgfpathrectangle{\pgfqpoint{0.017500in}{0.017500in}}{\pgfqpoint{6.218282in}{3.580372in}} %
\pgfusepath{clip}%
\pgfsetbuttcap%
\pgfsetroundjoin%
\definecolor{currentfill}{rgb}{0.888390,0.417703,0.327898}%
\pgfsetfillcolor{currentfill}%
\pgfsetlinewidth{0.000000pt}%
\definecolor{currentstroke}{rgb}{0.000000,0.000000,0.000000}%
\pgfsetstrokecolor{currentstroke}%
\pgfsetdash{}{0pt}%
\pgfpathmoveto{\pgfqpoint{3.502561in}{2.073396in}}%
\pgfpathlineto{\pgfqpoint{3.968076in}{1.994042in}}%
\pgfpathlineto{\pgfqpoint{4.226975in}{2.236848in}}%
\pgfpathlineto{\pgfqpoint{3.763398in}{2.297457in}}%
\pgfpathlineto{\pgfqpoint{3.502561in}{2.073396in}}%
\pgfpathclose%
\pgfusepath{fill}%
\end{pgfscope}%
\begin{pgfscope}%
\pgfpathrectangle{\pgfqpoint{0.017500in}{0.017500in}}{\pgfqpoint{6.218282in}{3.580372in}} %
\pgfusepath{clip}%
\pgfsetbuttcap%
\pgfsetroundjoin%
\definecolor{currentfill}{rgb}{0.892138,0.425389,0.333289}%
\pgfsetfillcolor{currentfill}%
\pgfsetlinewidth{0.000000pt}%
\definecolor{currentstroke}{rgb}{0.000000,0.000000,0.000000}%
\pgfsetstrokecolor{currentstroke}%
\pgfsetdash{}{0pt}%
\pgfpathmoveto{\pgfqpoint{3.968076in}{1.994042in}}%
\pgfpathlineto{\pgfqpoint{4.226975in}{2.236848in}}%
\pgfpathlineto{\pgfqpoint{3.968076in}{1.994042in}}%
\pgfpathclose%
\pgfusepath{fill}%
\end{pgfscope}%
\begin{pgfscope}%
\pgfpathrectangle{\pgfqpoint{0.017500in}{0.017500in}}{\pgfqpoint{6.218282in}{3.580372in}} %
\pgfusepath{clip}%
\pgfsetbuttcap%
\pgfsetroundjoin%
\definecolor{currentfill}{rgb}{0.892138,0.425389,0.333289}%
\pgfsetfillcolor{currentfill}%
\pgfsetlinewidth{0.000000pt}%
\definecolor{currentstroke}{rgb}{0.000000,0.000000,0.000000}%
\pgfsetstrokecolor{currentstroke}%
\pgfsetdash{}{0pt}%
\pgfpathmoveto{\pgfqpoint{3.968076in}{1.994042in}}%
\pgfpathlineto{\pgfqpoint{4.226975in}{2.236848in}}%
\pgfpathlineto{\pgfqpoint{3.968076in}{1.994042in}}%
\pgfpathclose%
\pgfusepath{fill}%
\end{pgfscope}%
\begin{pgfscope}%
\pgfpathrectangle{\pgfqpoint{0.017500in}{0.017500in}}{\pgfqpoint{6.218282in}{3.580372in}} %
\pgfusepath{clip}%
\pgfsetbuttcap%
\pgfsetroundjoin%
\definecolor{currentfill}{rgb}{0.892138,0.425389,0.333289}%
\pgfsetfillcolor{currentfill}%
\pgfsetlinewidth{0.000000pt}%
\definecolor{currentstroke}{rgb}{0.000000,0.000000,0.000000}%
\pgfsetstrokecolor{currentstroke}%
\pgfsetdash{}{0pt}%
\pgfpathmoveto{\pgfqpoint{3.968076in}{1.994042in}}%
\pgfpathlineto{\pgfqpoint{4.226975in}{2.236848in}}%
\pgfpathlineto{\pgfqpoint{3.968076in}{1.994042in}}%
\pgfpathclose%
\pgfusepath{fill}%
\end{pgfscope}%
\begin{pgfscope}%
\pgfpathrectangle{\pgfqpoint{0.017500in}{0.017500in}}{\pgfqpoint{6.218282in}{3.580372in}} %
\pgfusepath{clip}%
\pgfsetbuttcap%
\pgfsetroundjoin%
\definecolor{currentfill}{rgb}{0.892138,0.425389,0.333289}%
\pgfsetfillcolor{currentfill}%
\pgfsetlinewidth{0.000000pt}%
\definecolor{currentstroke}{rgb}{0.000000,0.000000,0.000000}%
\pgfsetstrokecolor{currentstroke}%
\pgfsetdash{}{0pt}%
\pgfpathmoveto{\pgfqpoint{3.968076in}{1.994042in}}%
\pgfpathlineto{\pgfqpoint{4.226975in}{2.236848in}}%
\pgfpathlineto{\pgfqpoint{3.968076in}{1.994042in}}%
\pgfpathclose%
\pgfusepath{fill}%
\end{pgfscope}%
\begin{pgfscope}%
\pgfpathrectangle{\pgfqpoint{0.017500in}{0.017500in}}{\pgfqpoint{6.218282in}{3.580372in}} %
\pgfusepath{clip}%
\pgfsetbuttcap%
\pgfsetroundjoin%
\definecolor{currentfill}{rgb}{0.892138,0.425389,0.333289}%
\pgfsetfillcolor{currentfill}%
\pgfsetlinewidth{0.000000pt}%
\definecolor{currentstroke}{rgb}{0.000000,0.000000,0.000000}%
\pgfsetstrokecolor{currentstroke}%
\pgfsetdash{}{0pt}%
\pgfpathmoveto{\pgfqpoint{3.968076in}{1.994042in}}%
\pgfpathlineto{\pgfqpoint{4.226975in}{2.236848in}}%
\pgfpathlineto{\pgfqpoint{3.968076in}{1.994042in}}%
\pgfpathclose%
\pgfusepath{fill}%
\end{pgfscope}%
\begin{pgfscope}%
\pgfpathrectangle{\pgfqpoint{0.017500in}{0.017500in}}{\pgfqpoint{6.218282in}{3.580372in}} %
\pgfusepath{clip}%
\pgfsetbuttcap%
\pgfsetroundjoin%
\definecolor{currentfill}{rgb}{0.892138,0.425389,0.333289}%
\pgfsetfillcolor{currentfill}%
\pgfsetlinewidth{0.000000pt}%
\definecolor{currentstroke}{rgb}{0.000000,0.000000,0.000000}%
\pgfsetstrokecolor{currentstroke}%
\pgfsetdash{}{0pt}%
\pgfpathmoveto{\pgfqpoint{3.968076in}{1.994042in}}%
\pgfpathlineto{\pgfqpoint{4.226975in}{2.236848in}}%
\pgfpathlineto{\pgfqpoint{3.968076in}{1.994042in}}%
\pgfpathclose%
\pgfusepath{fill}%
\end{pgfscope}%
\begin{pgfscope}%
\pgfpathrectangle{\pgfqpoint{0.017500in}{0.017500in}}{\pgfqpoint{6.218282in}{3.580372in}} %
\pgfusepath{clip}%
\pgfsetbuttcap%
\pgfsetroundjoin%
\definecolor{currentfill}{rgb}{0.892138,0.425389,0.333289}%
\pgfsetfillcolor{currentfill}%
\pgfsetlinewidth{0.000000pt}%
\definecolor{currentstroke}{rgb}{0.000000,0.000000,0.000000}%
\pgfsetstrokecolor{currentstroke}%
\pgfsetdash{}{0pt}%
\pgfpathmoveto{\pgfqpoint{3.968076in}{1.994042in}}%
\pgfpathlineto{\pgfqpoint{4.226975in}{2.236848in}}%
\pgfpathlineto{\pgfqpoint{3.968076in}{1.994042in}}%
\pgfpathclose%
\pgfusepath{fill}%
\end{pgfscope}%
\begin{pgfscope}%
\pgfpathrectangle{\pgfqpoint{0.017500in}{0.017500in}}{\pgfqpoint{6.218282in}{3.580372in}} %
\pgfusepath{clip}%
\pgfsetbuttcap%
\pgfsetroundjoin%
\definecolor{currentfill}{rgb}{0.892138,0.425389,0.333289}%
\pgfsetfillcolor{currentfill}%
\pgfsetlinewidth{0.000000pt}%
\definecolor{currentstroke}{rgb}{0.000000,0.000000,0.000000}%
\pgfsetstrokecolor{currentstroke}%
\pgfsetdash{}{0pt}%
\pgfpathmoveto{\pgfqpoint{3.968076in}{1.994042in}}%
\pgfpathlineto{\pgfqpoint{4.226975in}{2.236848in}}%
\pgfpathlineto{\pgfqpoint{3.968076in}{1.994042in}}%
\pgfpathclose%
\pgfusepath{fill}%
\end{pgfscope}%
\begin{pgfscope}%
\pgfpathrectangle{\pgfqpoint{0.017500in}{0.017500in}}{\pgfqpoint{6.218282in}{3.580372in}} %
\pgfusepath{clip}%
\pgfsetbuttcap%
\pgfsetroundjoin%
\definecolor{currentfill}{rgb}{0.892138,0.425389,0.333289}%
\pgfsetfillcolor{currentfill}%
\pgfsetlinewidth{0.000000pt}%
\definecolor{currentstroke}{rgb}{0.000000,0.000000,0.000000}%
\pgfsetstrokecolor{currentstroke}%
\pgfsetdash{}{0pt}%
\pgfpathmoveto{\pgfqpoint{3.968076in}{1.994042in}}%
\pgfpathlineto{\pgfqpoint{4.226975in}{2.236848in}}%
\pgfpathlineto{\pgfqpoint{3.968076in}{1.994042in}}%
\pgfpathclose%
\pgfusepath{fill}%
\end{pgfscope}%
\begin{pgfscope}%
\pgfpathrectangle{\pgfqpoint{0.017500in}{0.017500in}}{\pgfqpoint{6.218282in}{3.580372in}} %
\pgfusepath{clip}%
\pgfsetbuttcap%
\pgfsetroundjoin%
\definecolor{currentfill}{rgb}{0.892138,0.425389,0.333289}%
\pgfsetfillcolor{currentfill}%
\pgfsetlinewidth{0.000000pt}%
\definecolor{currentstroke}{rgb}{0.000000,0.000000,0.000000}%
\pgfsetstrokecolor{currentstroke}%
\pgfsetdash{}{0pt}%
\pgfpathmoveto{\pgfqpoint{3.968076in}{1.994042in}}%
\pgfpathlineto{\pgfqpoint{4.226975in}{2.236848in}}%
\pgfpathlineto{\pgfqpoint{3.968076in}{1.994042in}}%
\pgfpathclose%
\pgfusepath{fill}%
\end{pgfscope}%
\begin{pgfscope}%
\pgfpathrectangle{\pgfqpoint{0.017500in}{0.017500in}}{\pgfqpoint{6.218282in}{3.580372in}} %
\pgfusepath{clip}%
\pgfsetbuttcap%
\pgfsetroundjoin%
\definecolor{currentfill}{rgb}{0.892138,0.425389,0.333289}%
\pgfsetfillcolor{currentfill}%
\pgfsetlinewidth{0.000000pt}%
\definecolor{currentstroke}{rgb}{0.000000,0.000000,0.000000}%
\pgfsetstrokecolor{currentstroke}%
\pgfsetdash{}{0pt}%
\pgfpathmoveto{\pgfqpoint{3.968076in}{1.994042in}}%
\pgfpathlineto{\pgfqpoint{4.226975in}{2.236848in}}%
\pgfpathlineto{\pgfqpoint{3.968076in}{1.994042in}}%
\pgfpathclose%
\pgfusepath{fill}%
\end{pgfscope}%
\begin{pgfscope}%
\pgfpathrectangle{\pgfqpoint{0.017500in}{0.017500in}}{\pgfqpoint{6.218282in}{3.580372in}} %
\pgfusepath{clip}%
\pgfsetbuttcap%
\pgfsetroundjoin%
\definecolor{currentfill}{rgb}{0.892138,0.425389,0.333289}%
\pgfsetfillcolor{currentfill}%
\pgfsetlinewidth{0.000000pt}%
\definecolor{currentstroke}{rgb}{0.000000,0.000000,0.000000}%
\pgfsetstrokecolor{currentstroke}%
\pgfsetdash{}{0pt}%
\pgfpathmoveto{\pgfqpoint{3.968076in}{1.994042in}}%
\pgfpathlineto{\pgfqpoint{4.226975in}{2.236848in}}%
\pgfpathlineto{\pgfqpoint{3.968076in}{1.994042in}}%
\pgfpathclose%
\pgfusepath{fill}%
\end{pgfscope}%
\begin{pgfscope}%
\pgfpathrectangle{\pgfqpoint{0.017500in}{0.017500in}}{\pgfqpoint{6.218282in}{3.580372in}} %
\pgfusepath{clip}%
\pgfsetbuttcap%
\pgfsetroundjoin%
\definecolor{currentfill}{rgb}{0.892138,0.425389,0.333289}%
\pgfsetfillcolor{currentfill}%
\pgfsetlinewidth{0.000000pt}%
\definecolor{currentstroke}{rgb}{0.000000,0.000000,0.000000}%
\pgfsetstrokecolor{currentstroke}%
\pgfsetdash{}{0pt}%
\pgfpathmoveto{\pgfqpoint{3.968076in}{1.994042in}}%
\pgfpathlineto{\pgfqpoint{4.226975in}{2.236848in}}%
\pgfpathlineto{\pgfqpoint{3.968076in}{1.994042in}}%
\pgfpathclose%
\pgfusepath{fill}%
\end{pgfscope}%
\begin{pgfscope}%
\pgfpathrectangle{\pgfqpoint{0.017500in}{0.017500in}}{\pgfqpoint{6.218282in}{3.580372in}} %
\pgfusepath{clip}%
\pgfsetbuttcap%
\pgfsetroundjoin%
\definecolor{currentfill}{rgb}{0.892138,0.425389,0.333289}%
\pgfsetfillcolor{currentfill}%
\pgfsetlinewidth{0.000000pt}%
\definecolor{currentstroke}{rgb}{0.000000,0.000000,0.000000}%
\pgfsetstrokecolor{currentstroke}%
\pgfsetdash{}{0pt}%
\pgfpathmoveto{\pgfqpoint{3.968076in}{1.994042in}}%
\pgfpathlineto{\pgfqpoint{4.226975in}{2.236848in}}%
\pgfpathlineto{\pgfqpoint{3.968076in}{1.994042in}}%
\pgfpathclose%
\pgfusepath{fill}%
\end{pgfscope}%
\begin{pgfscope}%
\pgfpathrectangle{\pgfqpoint{0.017500in}{0.017500in}}{\pgfqpoint{6.218282in}{3.580372in}} %
\pgfusepath{clip}%
\pgfsetbuttcap%
\pgfsetroundjoin%
\definecolor{currentfill}{rgb}{0.892138,0.425389,0.333289}%
\pgfsetfillcolor{currentfill}%
\pgfsetlinewidth{0.000000pt}%
\definecolor{currentstroke}{rgb}{0.000000,0.000000,0.000000}%
\pgfsetstrokecolor{currentstroke}%
\pgfsetdash{}{0pt}%
\pgfpathmoveto{\pgfqpoint{3.968076in}{1.994042in}}%
\pgfpathlineto{\pgfqpoint{4.226975in}{2.236848in}}%
\pgfpathlineto{\pgfqpoint{3.968076in}{1.994042in}}%
\pgfpathclose%
\pgfusepath{fill}%
\end{pgfscope}%
\begin{pgfscope}%
\pgfpathrectangle{\pgfqpoint{0.017500in}{0.017500in}}{\pgfqpoint{6.218282in}{3.580372in}} %
\pgfusepath{clip}%
\pgfsetbuttcap%
\pgfsetroundjoin%
\definecolor{currentfill}{rgb}{0.892138,0.425389,0.333289}%
\pgfsetfillcolor{currentfill}%
\pgfsetlinewidth{0.000000pt}%
\definecolor{currentstroke}{rgb}{0.000000,0.000000,0.000000}%
\pgfsetstrokecolor{currentstroke}%
\pgfsetdash{}{0pt}%
\pgfpathmoveto{\pgfqpoint{3.968076in}{1.994042in}}%
\pgfpathlineto{\pgfqpoint{4.226975in}{2.236848in}}%
\pgfpathlineto{\pgfqpoint{3.968076in}{1.994042in}}%
\pgfpathclose%
\pgfusepath{fill}%
\end{pgfscope}%
\begin{pgfscope}%
\pgfpathrectangle{\pgfqpoint{0.017500in}{0.017500in}}{\pgfqpoint{6.218282in}{3.580372in}} %
\pgfusepath{clip}%
\pgfsetbuttcap%
\pgfsetroundjoin%
\definecolor{currentfill}{rgb}{0.892138,0.425389,0.333289}%
\pgfsetfillcolor{currentfill}%
\pgfsetlinewidth{0.000000pt}%
\definecolor{currentstroke}{rgb}{0.000000,0.000000,0.000000}%
\pgfsetstrokecolor{currentstroke}%
\pgfsetdash{}{0pt}%
\pgfpathmoveto{\pgfqpoint{3.968076in}{1.994042in}}%
\pgfpathlineto{\pgfqpoint{4.226975in}{2.236848in}}%
\pgfpathlineto{\pgfqpoint{3.968076in}{1.994042in}}%
\pgfpathclose%
\pgfusepath{fill}%
\end{pgfscope}%
\begin{pgfscope}%
\pgfpathrectangle{\pgfqpoint{0.017500in}{0.017500in}}{\pgfqpoint{6.218282in}{3.580372in}} %
\pgfusepath{clip}%
\pgfsetbuttcap%
\pgfsetroundjoin%
\definecolor{currentfill}{rgb}{0.892138,0.425389,0.333289}%
\pgfsetfillcolor{currentfill}%
\pgfsetlinewidth{0.000000pt}%
\definecolor{currentstroke}{rgb}{0.000000,0.000000,0.000000}%
\pgfsetstrokecolor{currentstroke}%
\pgfsetdash{}{0pt}%
\pgfpathmoveto{\pgfqpoint{3.968076in}{1.994042in}}%
\pgfpathlineto{\pgfqpoint{4.226975in}{2.236848in}}%
\pgfpathlineto{\pgfqpoint{3.968076in}{1.994042in}}%
\pgfpathclose%
\pgfusepath{fill}%
\end{pgfscope}%
\begin{pgfscope}%
\pgfpathrectangle{\pgfqpoint{0.017500in}{0.017500in}}{\pgfqpoint{6.218282in}{3.580372in}} %
\pgfusepath{clip}%
\pgfsetbuttcap%
\pgfsetroundjoin%
\definecolor{currentfill}{rgb}{0.892138,0.425389,0.333289}%
\pgfsetfillcolor{currentfill}%
\pgfsetlinewidth{0.000000pt}%
\definecolor{currentstroke}{rgb}{0.000000,0.000000,0.000000}%
\pgfsetstrokecolor{currentstroke}%
\pgfsetdash{}{0pt}%
\pgfpathmoveto{\pgfqpoint{3.968076in}{1.994042in}}%
\pgfpathlineto{\pgfqpoint{4.226975in}{2.236848in}}%
\pgfpathlineto{\pgfqpoint{3.968076in}{1.994042in}}%
\pgfpathclose%
\pgfusepath{fill}%
\end{pgfscope}%
\begin{pgfscope}%
\pgfpathrectangle{\pgfqpoint{0.017500in}{0.017500in}}{\pgfqpoint{6.218282in}{3.580372in}} %
\pgfusepath{clip}%
\pgfsetbuttcap%
\pgfsetroundjoin%
\definecolor{currentfill}{rgb}{0.892138,0.425389,0.333289}%
\pgfsetfillcolor{currentfill}%
\pgfsetlinewidth{0.000000pt}%
\definecolor{currentstroke}{rgb}{0.000000,0.000000,0.000000}%
\pgfsetstrokecolor{currentstroke}%
\pgfsetdash{}{0pt}%
\pgfpathmoveto{\pgfqpoint{3.968076in}{1.994042in}}%
\pgfpathlineto{\pgfqpoint{4.226975in}{2.236848in}}%
\pgfpathlineto{\pgfqpoint{3.968076in}{1.994042in}}%
\pgfpathclose%
\pgfusepath{fill}%
\end{pgfscope}%
\begin{pgfscope}%
\pgfpathrectangle{\pgfqpoint{0.017500in}{0.017500in}}{\pgfqpoint{6.218282in}{3.580372in}} %
\pgfusepath{clip}%
\pgfsetbuttcap%
\pgfsetroundjoin%
\definecolor{currentfill}{rgb}{0.892138,0.425389,0.333289}%
\pgfsetfillcolor{currentfill}%
\pgfsetlinewidth{0.000000pt}%
\definecolor{currentstroke}{rgb}{0.000000,0.000000,0.000000}%
\pgfsetstrokecolor{currentstroke}%
\pgfsetdash{}{0pt}%
\pgfpathmoveto{\pgfqpoint{3.968076in}{1.994042in}}%
\pgfpathlineto{\pgfqpoint{4.226975in}{2.236848in}}%
\pgfpathlineto{\pgfqpoint{3.968076in}{1.994042in}}%
\pgfpathclose%
\pgfusepath{fill}%
\end{pgfscope}%
\begin{pgfscope}%
\pgfpathrectangle{\pgfqpoint{0.017500in}{0.017500in}}{\pgfqpoint{6.218282in}{3.580372in}} %
\pgfusepath{clip}%
\pgfsetbuttcap%
\pgfsetroundjoin%
\definecolor{currentfill}{rgb}{0.892138,0.425389,0.333289}%
\pgfsetfillcolor{currentfill}%
\pgfsetlinewidth{0.000000pt}%
\definecolor{currentstroke}{rgb}{0.000000,0.000000,0.000000}%
\pgfsetstrokecolor{currentstroke}%
\pgfsetdash{}{0pt}%
\pgfpathmoveto{\pgfqpoint{3.968076in}{1.994042in}}%
\pgfpathlineto{\pgfqpoint{4.226975in}{2.236848in}}%
\pgfpathlineto{\pgfqpoint{3.968076in}{1.994042in}}%
\pgfpathclose%
\pgfusepath{fill}%
\end{pgfscope}%
\begin{pgfscope}%
\pgfpathrectangle{\pgfqpoint{0.017500in}{0.017500in}}{\pgfqpoint{6.218282in}{3.580372in}} %
\pgfusepath{clip}%
\pgfsetbuttcap%
\pgfsetroundjoin%
\definecolor{currentfill}{rgb}{0.892138,0.425389,0.333289}%
\pgfsetfillcolor{currentfill}%
\pgfsetlinewidth{0.000000pt}%
\definecolor{currentstroke}{rgb}{0.000000,0.000000,0.000000}%
\pgfsetstrokecolor{currentstroke}%
\pgfsetdash{}{0pt}%
\pgfpathmoveto{\pgfqpoint{3.968076in}{1.994042in}}%
\pgfpathlineto{\pgfqpoint{4.226975in}{2.236848in}}%
\pgfpathlineto{\pgfqpoint{3.968076in}{1.994042in}}%
\pgfpathclose%
\pgfusepath{fill}%
\end{pgfscope}%
\begin{pgfscope}%
\pgfpathrectangle{\pgfqpoint{0.017500in}{0.017500in}}{\pgfqpoint{6.218282in}{3.580372in}} %
\pgfusepath{clip}%
\pgfsetbuttcap%
\pgfsetroundjoin%
\definecolor{currentfill}{rgb}{0.892138,0.425389,0.333289}%
\pgfsetfillcolor{currentfill}%
\pgfsetlinewidth{0.000000pt}%
\definecolor{currentstroke}{rgb}{0.000000,0.000000,0.000000}%
\pgfsetstrokecolor{currentstroke}%
\pgfsetdash{}{0pt}%
\pgfpathmoveto{\pgfqpoint{3.968076in}{1.994042in}}%
\pgfpathlineto{\pgfqpoint{4.226975in}{2.236848in}}%
\pgfpathlineto{\pgfqpoint{3.968076in}{1.994042in}}%
\pgfpathclose%
\pgfusepath{fill}%
\end{pgfscope}%
\begin{pgfscope}%
\pgfpathrectangle{\pgfqpoint{0.017500in}{0.017500in}}{\pgfqpoint{6.218282in}{3.580372in}} %
\pgfusepath{clip}%
\pgfsetbuttcap%
\pgfsetroundjoin%
\definecolor{currentfill}{rgb}{0.892138,0.425389,0.333289}%
\pgfsetfillcolor{currentfill}%
\pgfsetlinewidth{0.000000pt}%
\definecolor{currentstroke}{rgb}{0.000000,0.000000,0.000000}%
\pgfsetstrokecolor{currentstroke}%
\pgfsetdash{}{0pt}%
\pgfpathmoveto{\pgfqpoint{3.968076in}{1.994042in}}%
\pgfpathlineto{\pgfqpoint{4.226975in}{2.236848in}}%
\pgfpathlineto{\pgfqpoint{3.968076in}{1.994042in}}%
\pgfpathclose%
\pgfusepath{fill}%
\end{pgfscope}%
\begin{pgfscope}%
\pgfpathrectangle{\pgfqpoint{0.017500in}{0.017500in}}{\pgfqpoint{6.218282in}{3.580372in}} %
\pgfusepath{clip}%
\pgfsetbuttcap%
\pgfsetroundjoin%
\definecolor{currentfill}{rgb}{0.892138,0.425389,0.333289}%
\pgfsetfillcolor{currentfill}%
\pgfsetlinewidth{0.000000pt}%
\definecolor{currentstroke}{rgb}{0.000000,0.000000,0.000000}%
\pgfsetstrokecolor{currentstroke}%
\pgfsetdash{}{0pt}%
\pgfpathmoveto{\pgfqpoint{3.968076in}{1.994042in}}%
\pgfpathlineto{\pgfqpoint{4.226975in}{2.236848in}}%
\pgfpathlineto{\pgfqpoint{3.968076in}{1.994042in}}%
\pgfpathclose%
\pgfusepath{fill}%
\end{pgfscope}%
\begin{pgfscope}%
\pgfpathrectangle{\pgfqpoint{0.017500in}{0.017500in}}{\pgfqpoint{6.218282in}{3.580372in}} %
\pgfusepath{clip}%
\pgfsetbuttcap%
\pgfsetroundjoin%
\definecolor{currentfill}{rgb}{0.892138,0.425389,0.333289}%
\pgfsetfillcolor{currentfill}%
\pgfsetlinewidth{0.000000pt}%
\definecolor{currentstroke}{rgb}{0.000000,0.000000,0.000000}%
\pgfsetstrokecolor{currentstroke}%
\pgfsetdash{}{0pt}%
\pgfpathmoveto{\pgfqpoint{3.968076in}{1.994042in}}%
\pgfpathlineto{\pgfqpoint{4.226975in}{2.236848in}}%
\pgfpathlineto{\pgfqpoint{3.968076in}{1.994042in}}%
\pgfpathclose%
\pgfusepath{fill}%
\end{pgfscope}%
\begin{pgfscope}%
\pgfpathrectangle{\pgfqpoint{0.017500in}{0.017500in}}{\pgfqpoint{6.218282in}{3.580372in}} %
\pgfusepath{clip}%
\pgfsetbuttcap%
\pgfsetroundjoin%
\definecolor{currentfill}{rgb}{0.892138,0.425389,0.333289}%
\pgfsetfillcolor{currentfill}%
\pgfsetlinewidth{0.000000pt}%
\definecolor{currentstroke}{rgb}{0.000000,0.000000,0.000000}%
\pgfsetstrokecolor{currentstroke}%
\pgfsetdash{}{0pt}%
\pgfpathmoveto{\pgfqpoint{3.968076in}{1.994042in}}%
\pgfpathlineto{\pgfqpoint{4.226975in}{2.236848in}}%
\pgfpathlineto{\pgfqpoint{3.968076in}{1.994042in}}%
\pgfpathclose%
\pgfusepath{fill}%
\end{pgfscope}%
\begin{pgfscope}%
\pgfpathrectangle{\pgfqpoint{0.017500in}{0.017500in}}{\pgfqpoint{6.218282in}{3.580372in}} %
\pgfusepath{clip}%
\pgfsetbuttcap%
\pgfsetroundjoin%
\definecolor{currentfill}{rgb}{0.892138,0.425389,0.333289}%
\pgfsetfillcolor{currentfill}%
\pgfsetlinewidth{0.000000pt}%
\definecolor{currentstroke}{rgb}{0.000000,0.000000,0.000000}%
\pgfsetstrokecolor{currentstroke}%
\pgfsetdash{}{0pt}%
\pgfpathmoveto{\pgfqpoint{3.968076in}{1.994042in}}%
\pgfpathlineto{\pgfqpoint{4.226975in}{2.236848in}}%
\pgfpathlineto{\pgfqpoint{3.968076in}{1.994042in}}%
\pgfpathclose%
\pgfusepath{fill}%
\end{pgfscope}%
\begin{pgfscope}%
\pgfpathrectangle{\pgfqpoint{0.017500in}{0.017500in}}{\pgfqpoint{6.218282in}{3.580372in}} %
\pgfusepath{clip}%
\pgfsetbuttcap%
\pgfsetroundjoin%
\definecolor{currentfill}{rgb}{0.892138,0.425389,0.333289}%
\pgfsetfillcolor{currentfill}%
\pgfsetlinewidth{0.000000pt}%
\definecolor{currentstroke}{rgb}{0.000000,0.000000,0.000000}%
\pgfsetstrokecolor{currentstroke}%
\pgfsetdash{}{0pt}%
\pgfpathmoveto{\pgfqpoint{3.968076in}{1.994042in}}%
\pgfpathlineto{\pgfqpoint{4.226975in}{2.236848in}}%
\pgfpathlineto{\pgfqpoint{3.968076in}{1.994042in}}%
\pgfpathclose%
\pgfusepath{fill}%
\end{pgfscope}%
\end{pgfpicture}%
\makeatother%
\endgroup%

	\caption{Learning surface for SemEval and CONLL2000 Chunk.}
\end{figure}
\begin{figure}[h!]
	\hspace*{-2cm}%% Creator: Matplotlib, PGF backend
%%
%% To include the figure in your LaTeX document, write
%%   \input{<filename>.pgf}
%%
%% Make sure the required packages are loaded in your preamble
%%   \usepackage{pgf}
%%
%% Figures using additional raster images can only be included by \input if
%% they are in the same directory as the main LaTeX file. For loading figures
%% from other directories you can use the `import` package
%%   \usepackage{import}
%% and then include the figures with
%%   \import{<path to file>}{<filename>.pgf}
%%
%% Matplotlib used the following preamble
%%   \usepackage{fontspec}
%%   \setmainfont{Palatino}
%%   \setsansfont{Lucida Grande}
%%   \setmonofont{Andale Mono}
%%
\begingroup%
\makeatletter%
\begin{pgfpicture}%
\pgfpathrectangle{\pgfpointorigin}{\pgfqpoint{6.253282in}{3.615372in}}%
\pgfusepath{use as bounding box, clip}%
\begin{pgfscope}%
\pgfsetbuttcap%
\pgfsetmiterjoin%
\definecolor{currentfill}{rgb}{1.000000,1.000000,1.000000}%
\pgfsetfillcolor{currentfill}%
\pgfsetlinewidth{0.000000pt}%
\definecolor{currentstroke}{rgb}{1.000000,1.000000,1.000000}%
\pgfsetstrokecolor{currentstroke}%
\pgfsetdash{}{0pt}%
\pgfpathmoveto{\pgfqpoint{0.000000in}{0.000000in}}%
\pgfpathlineto{\pgfqpoint{6.253282in}{0.000000in}}%
\pgfpathlineto{\pgfqpoint{6.253282in}{3.615372in}}%
\pgfpathlineto{\pgfqpoint{0.000000in}{3.615372in}}%
\pgfpathclose%
\pgfusepath{fill}%
\end{pgfscope}%
\begin{pgfscope}%
\pgfsetbuttcap%
\pgfsetmiterjoin%
\definecolor{currentfill}{rgb}{1.000000,1.000000,1.000000}%
\pgfsetfillcolor{currentfill}%
\pgfsetlinewidth{0.000000pt}%
\definecolor{currentstroke}{rgb}{0.000000,0.000000,0.000000}%
\pgfsetstrokecolor{currentstroke}%
\pgfsetstrokeopacity{0.000000}%
\pgfsetdash{}{0pt}%
\pgfpathmoveto{\pgfqpoint{0.017500in}{0.017500in}}%
\pgfpathlineto{\pgfqpoint{6.235782in}{0.017500in}}%
\pgfpathlineto{\pgfqpoint{6.235782in}{3.597872in}}%
\pgfpathlineto{\pgfqpoint{0.017500in}{3.597872in}}%
\pgfpathclose%
\pgfusepath{fill}%
\end{pgfscope}%
\begin{pgfscope}%
\pgfsetbuttcap%
\pgfsetmiterjoin%
\definecolor{currentfill}{rgb}{0.950000,0.950000,0.950000}%
\pgfsetfillcolor{currentfill}%
\pgfsetfillopacity{0.500000}%
\pgfsetlinewidth{0.200750pt}%
\definecolor{currentstroke}{rgb}{0.950000,0.950000,0.950000}%
\pgfsetstrokecolor{currentstroke}%
\pgfsetstrokeopacity{0.500000}%
\pgfsetdash{}{0pt}%
\pgfpathmoveto{\pgfqpoint{1.233525in}{0.981833in}}%
\pgfpathlineto{\pgfqpoint{2.713039in}{1.712806in}}%
\pgfpathlineto{\pgfqpoint{2.692063in}{3.119783in}}%
\pgfpathlineto{\pgfqpoint{1.144557in}{2.450885in}}%
\pgfusepath{stroke,fill}%
\end{pgfscope}%
\begin{pgfscope}%
\pgfsetbuttcap%
\pgfsetmiterjoin%
\definecolor{currentfill}{rgb}{0.900000,0.900000,0.900000}%
\pgfsetfillcolor{currentfill}%
\pgfsetfillopacity{0.500000}%
\pgfsetlinewidth{0.200750pt}%
\definecolor{currentstroke}{rgb}{0.900000,0.900000,0.900000}%
\pgfsetstrokecolor{currentstroke}%
\pgfsetstrokeopacity{0.500000}%
\pgfsetdash{}{0pt}%
\pgfpathmoveto{\pgfqpoint{2.713039in}{1.712806in}}%
\pgfpathlineto{\pgfqpoint{5.135278in}{1.301993in}}%
\pgfpathlineto{\pgfqpoint{5.219480in}{2.744306in}}%
\pgfpathlineto{\pgfqpoint{2.692063in}{3.119783in}}%
\pgfusepath{stroke,fill}%
\end{pgfscope}%
\begin{pgfscope}%
\pgfsetbuttcap%
\pgfsetmiterjoin%
\definecolor{currentfill}{rgb}{0.925000,0.925000,0.925000}%
\pgfsetfillcolor{currentfill}%
\pgfsetfillopacity{0.500000}%
\pgfsetlinewidth{0.200750pt}%
\definecolor{currentstroke}{rgb}{0.925000,0.925000,0.925000}%
\pgfsetstrokecolor{currentstroke}%
\pgfsetstrokeopacity{0.500000}%
\pgfsetdash{}{0pt}%
\pgfpathmoveto{\pgfqpoint{1.233525in}{0.981833in}}%
\pgfpathlineto{\pgfqpoint{3.760968in}{0.515142in}}%
\pgfpathlineto{\pgfqpoint{5.135278in}{1.301993in}}%
\pgfpathlineto{\pgfqpoint{2.713039in}{1.712806in}}%
\pgfusepath{stroke,fill}%
\end{pgfscope}%
\begin{pgfscope}%
\pgfsetrectcap%
\pgfsetroundjoin%
\pgfsetlinewidth{0.501875pt}%
\definecolor{currentstroke}{rgb}{0.000000,0.000000,0.000000}%
\pgfsetstrokecolor{currentstroke}%
\pgfsetdash{}{0pt}%
\pgfpathmoveto{\pgfqpoint{1.233525in}{0.981833in}}%
\pgfpathlineto{\pgfqpoint{3.760968in}{0.515142in}}%
\pgfusepath{stroke}%
\end{pgfscope}%
\begin{pgfscope}%
\pgftext[x=2.292998in,y=0.574477in,left,base,rotate=349.538214]{\rmfamily\fontsize{10.000000}{12.000000}\selectfont }%
\end{pgfscope}%
\begin{pgfscope}%
\pgftext[x=1.953595in,y=0.474946in,left,base,rotate=349.538214]{\rmfamily\fontsize{10.000000}{12.000000}\selectfont fraction of}%
\end{pgfscope}%
\begin{pgfscope}%
\pgftext[x=1.482715in,y=0.399693in,left,base,rotate=349.538214]{\rmfamily\fontsize{10.000000}{12.000000}\selectfont CONLL2000 Chunk data}%
\end{pgfscope}%
\begin{pgfscope}%
\pgfsetbuttcap%
\pgfsetroundjoin%
\pgfsetlinewidth{0.803000pt}%
\definecolor{currentstroke}{rgb}{0.690196,0.690196,0.690196}%
\pgfsetstrokecolor{currentstroke}%
\pgfsetdash{}{0pt}%
\pgfpathmoveto{\pgfqpoint{1.388733in}{0.953174in}}%
\pgfpathlineto{\pgfqpoint{2.862115in}{1.687522in}}%
\pgfpathlineto{\pgfqpoint{2.847388in}{3.096708in}}%
\pgfusepath{stroke}%
\end{pgfscope}%
\begin{pgfscope}%
\pgfsetbuttcap%
\pgfsetroundjoin%
\pgfsetlinewidth{0.803000pt}%
\definecolor{currentstroke}{rgb}{0.690196,0.690196,0.690196}%
\pgfsetstrokecolor{currentstroke}%
\pgfsetdash{}{0pt}%
\pgfpathmoveto{\pgfqpoint{1.818260in}{0.873862in}}%
\pgfpathlineto{\pgfqpoint{3.274447in}{1.617591in}}%
\pgfpathlineto{\pgfqpoint{3.277159in}{3.032860in}}%
\pgfusepath{stroke}%
\end{pgfscope}%
\begin{pgfscope}%
\pgfsetbuttcap%
\pgfsetroundjoin%
\pgfsetlinewidth{0.803000pt}%
\definecolor{currentstroke}{rgb}{0.690196,0.690196,0.690196}%
\pgfsetstrokecolor{currentstroke}%
\pgfsetdash{}{0pt}%
\pgfpathmoveto{\pgfqpoint{2.253467in}{0.793501in}}%
\pgfpathlineto{\pgfqpoint{3.691895in}{1.546791in}}%
\pgfpathlineto{\pgfqpoint{3.712491in}{2.968187in}}%
\pgfusepath{stroke}%
\end{pgfscope}%
\begin{pgfscope}%
\pgfsetbuttcap%
\pgfsetroundjoin%
\pgfsetlinewidth{0.803000pt}%
\definecolor{currentstroke}{rgb}{0.690196,0.690196,0.690196}%
\pgfsetstrokecolor{currentstroke}%
\pgfsetdash{}{0pt}%
\pgfpathmoveto{\pgfqpoint{2.694467in}{0.712071in}}%
\pgfpathlineto{\pgfqpoint{4.114556in}{1.475108in}}%
\pgfpathlineto{\pgfqpoint{4.153492in}{2.902671in}}%
\pgfusepath{stroke}%
\end{pgfscope}%
\begin{pgfscope}%
\pgfsetbuttcap%
\pgfsetroundjoin%
\pgfsetlinewidth{0.803000pt}%
\definecolor{currentstroke}{rgb}{0.690196,0.690196,0.690196}%
\pgfsetstrokecolor{currentstroke}%
\pgfsetdash{}{0pt}%
\pgfpathmoveto{\pgfqpoint{3.141375in}{0.629550in}}%
\pgfpathlineto{\pgfqpoint{4.542526in}{1.402524in}}%
\pgfpathlineto{\pgfqpoint{4.600274in}{2.836296in}}%
\pgfusepath{stroke}%
\end{pgfscope}%
\begin{pgfscope}%
\pgfsetbuttcap%
\pgfsetroundjoin%
\pgfsetlinewidth{0.803000pt}%
\definecolor{currentstroke}{rgb}{0.690196,0.690196,0.690196}%
\pgfsetstrokecolor{currentstroke}%
\pgfsetdash{}{0pt}%
\pgfpathmoveto{\pgfqpoint{3.594311in}{0.545915in}}%
\pgfpathlineto{\pgfqpoint{4.975908in}{1.329022in}}%
\pgfpathlineto{\pgfqpoint{5.052952in}{2.769046in}}%
\pgfusepath{stroke}%
\end{pgfscope}%
\begin{pgfscope}%
\pgfsetrectcap%
\pgfsetroundjoin%
\pgfsetlinewidth{0.803000pt}%
\definecolor{currentstroke}{rgb}{0.000000,0.000000,0.000000}%
\pgfsetstrokecolor{currentstroke}%
\pgfsetdash{}{0pt}%
\pgfpathmoveto{\pgfqpoint{1.401276in}{0.959425in}}%
\pgfpathlineto{\pgfqpoint{1.363606in}{0.940650in}}%
\pgfusepath{stroke}%
\end{pgfscope}%
\begin{pgfscope}%
\pgftext[x=1.295334in,y=0.815921in,,top]{\rmfamily\fontsize{8.000000}{9.600000}\selectfont 0.0}%
\end{pgfscope}%
\begin{pgfscope}%
\pgfsetrectcap%
\pgfsetroundjoin%
\pgfsetlinewidth{0.803000pt}%
\definecolor{currentstroke}{rgb}{0.000000,0.000000,0.000000}%
\pgfsetstrokecolor{currentstroke}%
\pgfsetdash{}{0pt}%
\pgfpathmoveto{\pgfqpoint{1.830663in}{0.880196in}}%
\pgfpathlineto{\pgfqpoint{1.793417in}{0.861173in}}%
\pgfusepath{stroke}%
\end{pgfscope}%
\begin{pgfscope}%
\pgftext[x=1.724832in,y=0.735598in,,top]{\rmfamily\fontsize{8.000000}{9.600000}\selectfont 0.2}%
\end{pgfscope}%
\begin{pgfscope}%
\pgfsetrectcap%
\pgfsetroundjoin%
\pgfsetlinewidth{0.803000pt}%
\definecolor{currentstroke}{rgb}{0.000000,0.000000,0.000000}%
\pgfsetstrokecolor{currentstroke}%
\pgfsetdash{}{0pt}%
\pgfpathmoveto{\pgfqpoint{2.265723in}{0.799919in}}%
\pgfpathlineto{\pgfqpoint{2.228917in}{0.780644in}}%
\pgfusepath{stroke}%
\end{pgfscope}%
\begin{pgfscope}%
\pgftext[x=2.160017in,y=0.654212in,,top]{\rmfamily\fontsize{8.000000}{9.600000}\selectfont 0.4}%
\end{pgfscope}%
\begin{pgfscope}%
\pgfsetrectcap%
\pgfsetroundjoin%
\pgfsetlinewidth{0.803000pt}%
\definecolor{currentstroke}{rgb}{0.000000,0.000000,0.000000}%
\pgfsetstrokecolor{currentstroke}%
\pgfsetdash{}{0pt}%
\pgfpathmoveto{\pgfqpoint{2.706571in}{0.718575in}}%
\pgfpathlineto{\pgfqpoint{2.670219in}{0.699042in}}%
\pgfusepath{stroke}%
\end{pgfscope}%
\begin{pgfscope}%
\pgftext[x=2.601002in,y=0.571741in,,top]{\rmfamily\fontsize{8.000000}{9.600000}\selectfont 0.6}%
\end{pgfscope}%
\begin{pgfscope}%
\pgfsetrectcap%
\pgfsetroundjoin%
\pgfsetlinewidth{0.803000pt}%
\definecolor{currentstroke}{rgb}{0.000000,0.000000,0.000000}%
\pgfsetstrokecolor{currentstroke}%
\pgfsetdash{}{0pt}%
\pgfpathmoveto{\pgfqpoint{3.153323in}{0.636141in}}%
\pgfpathlineto{\pgfqpoint{3.117440in}{0.616346in}}%
\pgfusepath{stroke}%
\end{pgfscope}%
\begin{pgfscope}%
\pgftext[x=3.047904in,y=0.488164in,,top]{\rmfamily\fontsize{8.000000}{9.600000}\selectfont 0.8}%
\end{pgfscope}%
\begin{pgfscope}%
\pgfsetrectcap%
\pgfsetroundjoin%
\pgfsetlinewidth{0.803000pt}%
\definecolor{currentstroke}{rgb}{0.000000,0.000000,0.000000}%
\pgfsetstrokecolor{currentstroke}%
\pgfsetdash{}{0pt}%
\pgfpathmoveto{\pgfqpoint{3.606097in}{0.552596in}}%
\pgfpathlineto{\pgfqpoint{3.570701in}{0.532532in}}%
\pgfusepath{stroke}%
\end{pgfscope}%
\begin{pgfscope}%
\pgftext[x=3.500842in,y=0.403458in,,top]{\rmfamily\fontsize{8.000000}{9.600000}\selectfont 1.0}%
\end{pgfscope}%
\begin{pgfscope}%
\pgfsetrectcap%
\pgfsetroundjoin%
\pgfsetlinewidth{0.501875pt}%
\definecolor{currentstroke}{rgb}{0.000000,0.000000,0.000000}%
\pgfsetstrokecolor{currentstroke}%
\pgfsetdash{}{0pt}%
\pgfpathmoveto{\pgfqpoint{5.135278in}{1.301993in}}%
\pgfpathlineto{\pgfqpoint{3.760968in}{0.515142in}}%
\pgfusepath{stroke}%
\end{pgfscope}%
\begin{pgfscope}%
\pgftext[x=4.748349in,y=0.779064in,left,base,rotate=29.792981]{\rmfamily\fontsize{10.000000}{12.000000}\selectfont }%
\end{pgfscope}%
\begin{pgfscope}%
\pgftext[x=4.553640in,y=0.483787in,left,base,rotate=29.792981]{\rmfamily\fontsize{10.000000}{12.000000}\selectfont fraction of}%
\end{pgfscope}%
\begin{pgfscope}%
\pgftext[x=4.542818in,y=0.293793in,left,base,rotate=29.792981]{\rmfamily\fontsize{10.000000}{12.000000}\selectfont SemEval data}%
\end{pgfscope}%
\begin{pgfscope}%
\pgfsetbuttcap%
\pgfsetroundjoin%
\pgfsetlinewidth{0.803000pt}%
\definecolor{currentstroke}{rgb}{0.690196,0.690196,0.690196}%
\pgfsetstrokecolor{currentstroke}%
\pgfsetdash{}{0pt}%
\pgfpathmoveto{\pgfqpoint{1.249232in}{2.496130in}}%
\pgfpathlineto{\pgfqpoint{1.333347in}{1.031151in}}%
\pgfpathlineto{\pgfqpoint{3.853895in}{0.568347in}}%
\pgfusepath{stroke}%
\end{pgfscope}%
\begin{pgfscope}%
\pgfsetbuttcap%
\pgfsetroundjoin%
\pgfsetlinewidth{0.803000pt}%
\definecolor{currentstroke}{rgb}{0.690196,0.690196,0.690196}%
\pgfsetstrokecolor{currentstroke}%
\pgfsetdash{}{0pt}%
\pgfpathmoveto{\pgfqpoint{1.531798in}{2.618267in}}%
\pgfpathlineto{\pgfqpoint{1.602996in}{1.164374in}}%
\pgfpathlineto{\pgfqpoint{4.104771in}{0.711984in}}%
\pgfusepath{stroke}%
\end{pgfscope}%
\begin{pgfscope}%
\pgfsetbuttcap%
\pgfsetroundjoin%
\pgfsetlinewidth{0.803000pt}%
\definecolor{currentstroke}{rgb}{0.690196,0.690196,0.690196}%
\pgfsetstrokecolor{currentstroke}%
\pgfsetdash{}{0pt}%
\pgfpathmoveto{\pgfqpoint{1.807886in}{2.737604in}}%
\pgfpathlineto{\pgfqpoint{1.866722in}{1.294672in}}%
\pgfpathlineto{\pgfqpoint{4.349930in}{0.852348in}}%
\pgfusepath{stroke}%
\end{pgfscope}%
\begin{pgfscope}%
\pgfsetbuttcap%
\pgfsetroundjoin%
\pgfsetlinewidth{0.803000pt}%
\definecolor{currentstroke}{rgb}{0.690196,0.690196,0.690196}%
\pgfsetstrokecolor{currentstroke}%
\pgfsetdash{}{0pt}%
\pgfpathmoveto{\pgfqpoint{2.077717in}{2.854237in}}%
\pgfpathlineto{\pgfqpoint{2.124720in}{1.422139in}}%
\pgfpathlineto{\pgfqpoint{4.589565in}{0.989549in}}%
\pgfusepath{stroke}%
\end{pgfscope}%
\begin{pgfscope}%
\pgfsetbuttcap%
\pgfsetroundjoin%
\pgfsetlinewidth{0.803000pt}%
\definecolor{currentstroke}{rgb}{0.690196,0.690196,0.690196}%
\pgfsetstrokecolor{currentstroke}%
\pgfsetdash{}{0pt}%
\pgfpathmoveto{\pgfqpoint{2.341501in}{2.968255in}}%
\pgfpathlineto{\pgfqpoint{2.377173in}{1.546866in}}%
\pgfpathlineto{\pgfqpoint{4.823859in}{1.123693in}}%
\pgfusepath{stroke}%
\end{pgfscope}%
\begin{pgfscope}%
\pgfsetbuttcap%
\pgfsetroundjoin%
\pgfsetlinewidth{0.803000pt}%
\definecolor{currentstroke}{rgb}{0.690196,0.690196,0.690196}%
\pgfsetstrokecolor{currentstroke}%
\pgfsetdash{}{0pt}%
\pgfpathmoveto{\pgfqpoint{2.599439in}{3.079747in}}%
\pgfpathlineto{\pgfqpoint{2.624258in}{1.668942in}}%
\pgfpathlineto{\pgfqpoint{5.052991in}{1.254881in}}%
\pgfusepath{stroke}%
\end{pgfscope}%
\begin{pgfscope}%
\pgfsetrectcap%
\pgfsetroundjoin%
\pgfsetlinewidth{0.803000pt}%
\definecolor{currentstroke}{rgb}{0.000000,0.000000,0.000000}%
\pgfsetstrokecolor{currentstroke}%
\pgfsetdash{}{0pt}%
\pgfpathmoveto{\pgfqpoint{3.832959in}{0.572191in}}%
\pgfpathlineto{\pgfqpoint{3.895805in}{0.560652in}}%
\pgfusepath{stroke}%
\end{pgfscope}%
\begin{pgfscope}%
\pgftext[x=4.015897in,y=0.448445in,,top]{\rmfamily\fontsize{8.000000}{9.600000}\selectfont 0.0}%
\end{pgfscope}%
\begin{pgfscope}%
\pgfsetrectcap%
\pgfsetroundjoin%
\pgfsetlinewidth{0.803000pt}%
\definecolor{currentstroke}{rgb}{0.000000,0.000000,0.000000}%
\pgfsetstrokecolor{currentstroke}%
\pgfsetdash{}{0pt}%
\pgfpathmoveto{\pgfqpoint{4.084000in}{0.715740in}}%
\pgfpathlineto{\pgfqpoint{4.146351in}{0.704465in}}%
\pgfusepath{stroke}%
\end{pgfscope}%
\begin{pgfscope}%
\pgftext[x=4.264814in,y=0.593411in,,top]{\rmfamily\fontsize{8.000000}{9.600000}\selectfont 0.2}%
\end{pgfscope}%
\begin{pgfscope}%
\pgfsetrectcap%
\pgfsetroundjoin%
\pgfsetlinewidth{0.803000pt}%
\definecolor{currentstroke}{rgb}{0.000000,0.000000,0.000000}%
\pgfsetstrokecolor{currentstroke}%
\pgfsetdash{}{0pt}%
\pgfpathmoveto{\pgfqpoint{4.329322in}{0.856019in}}%
\pgfpathlineto{\pgfqpoint{4.391184in}{0.845000in}}%
\pgfusepath{stroke}%
\end{pgfscope}%
\begin{pgfscope}%
\pgftext[x=4.508060in,y=0.735074in,,top]{\rmfamily\fontsize{8.000000}{9.600000}\selectfont 0.4}%
\end{pgfscope}%
\begin{pgfscope}%
\pgfsetrectcap%
\pgfsetroundjoin%
\pgfsetlinewidth{0.803000pt}%
\definecolor{currentstroke}{rgb}{0.000000,0.000000,0.000000}%
\pgfsetstrokecolor{currentstroke}%
\pgfsetdash{}{0pt}%
\pgfpathmoveto{\pgfqpoint{4.569117in}{0.993138in}}%
\pgfpathlineto{\pgfqpoint{4.630496in}{0.982365in}}%
\pgfusepath{stroke}%
\end{pgfscope}%
\begin{pgfscope}%
\pgftext[x=4.745827in,y=0.873546in,,top]{\rmfamily\fontsize{8.000000}{9.600000}\selectfont 0.6}%
\end{pgfscope}%
\begin{pgfscope}%
\pgfsetrectcap%
\pgfsetroundjoin%
\pgfsetlinewidth{0.803000pt}%
\definecolor{currentstroke}{rgb}{0.000000,0.000000,0.000000}%
\pgfsetstrokecolor{currentstroke}%
\pgfsetdash{}{0pt}%
\pgfpathmoveto{\pgfqpoint{4.803571in}{1.127202in}}%
\pgfpathlineto{\pgfqpoint{4.864473in}{1.116668in}}%
\pgfusepath{stroke}%
\end{pgfscope}%
\begin{pgfscope}%
\pgftext[x=4.978298in,y=1.008934in,,top]{\rmfamily\fontsize{8.000000}{9.600000}\selectfont 0.8}%
\end{pgfscope}%
\begin{pgfscope}%
\pgfsetrectcap%
\pgfsetroundjoin%
\pgfsetlinewidth{0.803000pt}%
\definecolor{currentstroke}{rgb}{0.000000,0.000000,0.000000}%
\pgfsetstrokecolor{currentstroke}%
\pgfsetdash{}{0pt}%
\pgfpathmoveto{\pgfqpoint{5.032859in}{1.258313in}}%
\pgfpathlineto{\pgfqpoint{5.093291in}{1.248010in}}%
\pgfusepath{stroke}%
\end{pgfscope}%
\begin{pgfscope}%
\pgftext[x=5.205648in,y=1.141340in,,top]{\rmfamily\fontsize{8.000000}{9.600000}\selectfont 1.0}%
\end{pgfscope}%
\begin{pgfscope}%
\pgfsetrectcap%
\pgfsetroundjoin%
\pgfsetlinewidth{0.501875pt}%
\definecolor{currentstroke}{rgb}{0.000000,0.000000,0.000000}%
\pgfsetstrokecolor{currentstroke}%
\pgfsetdash{}{0pt}%
\pgfpathmoveto{\pgfqpoint{5.135278in}{1.301993in}}%
\pgfpathlineto{\pgfqpoint{5.219480in}{2.744306in}}%
\pgfusepath{stroke}%
\end{pgfscope}%
\begin{pgfscope}%
\pgftext[x=5.653991in,y=2.051089in,left,base,rotate=86.658871]{\rmfamily\fontsize{10.000000}{12.000000}\selectfont }%
\end{pgfscope}%
\begin{pgfscope}%
\pgftext[x=5.798047in,y=1.781819in,left,base,rotate=86.658871]{\rmfamily\fontsize{10.000000}{12.000000}\selectfont mean F1}%
\end{pgfscope}%
\begin{pgfscope}%
\pgfsetbuttcap%
\pgfsetroundjoin%
\pgfsetlinewidth{0.803000pt}%
\definecolor{currentstroke}{rgb}{0.690196,0.690196,0.690196}%
\pgfsetstrokecolor{currentstroke}%
\pgfsetdash{}{0pt}%
\pgfpathmoveto{\pgfqpoint{5.143397in}{1.441079in}}%
\pgfpathlineto{\pgfqpoint{2.711014in}{1.848672in}}%
\pgfpathlineto{\pgfqpoint{1.224955in}{1.123344in}}%
\pgfusepath{stroke}%
\end{pgfscope}%
\begin{pgfscope}%
\pgfsetbuttcap%
\pgfsetroundjoin%
\pgfsetlinewidth{0.803000pt}%
\definecolor{currentstroke}{rgb}{0.690196,0.690196,0.690196}%
\pgfsetstrokecolor{currentstroke}%
\pgfsetdash{}{0pt}%
\pgfpathmoveto{\pgfqpoint{5.155442in}{1.647401in}}%
\pgfpathlineto{\pgfqpoint{2.708010in}{2.050144in}}%
\pgfpathlineto{\pgfqpoint{1.212238in}{1.333324in}}%
\pgfusepath{stroke}%
\end{pgfscope}%
\begin{pgfscope}%
\pgfsetbuttcap%
\pgfsetroundjoin%
\pgfsetlinewidth{0.803000pt}%
\definecolor{currentstroke}{rgb}{0.690196,0.690196,0.690196}%
\pgfsetstrokecolor{currentstroke}%
\pgfsetdash{}{0pt}%
\pgfpathmoveto{\pgfqpoint{5.167639in}{1.856311in}}%
\pgfpathlineto{\pgfqpoint{2.704970in}{2.254053in}}%
\pgfpathlineto{\pgfqpoint{1.199357in}{1.546011in}}%
\pgfusepath{stroke}%
\end{pgfscope}%
\begin{pgfscope}%
\pgfsetbuttcap%
\pgfsetroundjoin%
\pgfsetlinewidth{0.803000pt}%
\definecolor{currentstroke}{rgb}{0.690196,0.690196,0.690196}%
\pgfsetstrokecolor{currentstroke}%
\pgfsetdash{}{0pt}%
\pgfpathmoveto{\pgfqpoint{5.179989in}{2.067857in}}%
\pgfpathlineto{\pgfqpoint{2.701893in}{2.460443in}}%
\pgfpathlineto{\pgfqpoint{1.186310in}{1.761458in}}%
\pgfusepath{stroke}%
\end{pgfscope}%
\begin{pgfscope}%
\pgfsetbuttcap%
\pgfsetroundjoin%
\pgfsetlinewidth{0.803000pt}%
\definecolor{currentstroke}{rgb}{0.690196,0.690196,0.690196}%
\pgfsetstrokecolor{currentstroke}%
\pgfsetdash{}{0pt}%
\pgfpathmoveto{\pgfqpoint{5.192496in}{2.282090in}}%
\pgfpathlineto{\pgfqpoint{2.698778in}{2.669360in}}%
\pgfpathlineto{\pgfqpoint{1.173091in}{1.979718in}}%
\pgfusepath{stroke}%
\end{pgfscope}%
\begin{pgfscope}%
\pgfsetbuttcap%
\pgfsetroundjoin%
\pgfsetlinewidth{0.803000pt}%
\definecolor{currentstroke}{rgb}{0.690196,0.690196,0.690196}%
\pgfsetstrokecolor{currentstroke}%
\pgfsetdash{}{0pt}%
\pgfpathmoveto{\pgfqpoint{5.205162in}{2.499062in}}%
\pgfpathlineto{\pgfqpoint{2.695625in}{2.880851in}}%
\pgfpathlineto{\pgfqpoint{1.159699in}{2.200846in}}%
\pgfusepath{stroke}%
\end{pgfscope}%
\begin{pgfscope}%
\pgfsetrectcap%
\pgfsetroundjoin%
\pgfsetlinewidth{0.803000pt}%
\definecolor{currentstroke}{rgb}{0.000000,0.000000,0.000000}%
\pgfsetstrokecolor{currentstroke}%
\pgfsetdash{}{0pt}%
\pgfpathmoveto{\pgfqpoint{5.123235in}{1.444458in}}%
\pgfpathlineto{\pgfqpoint{5.183758in}{1.434316in}}%
\pgfusepath{stroke}%
\end{pgfscope}%
\begin{pgfscope}%
\pgftext[x=5.381724in,y=1.460404in,,top]{\rmfamily\fontsize{8.000000}{9.600000}\selectfont 0.1}%
\end{pgfscope}%
\begin{pgfscope}%
\pgfsetrectcap%
\pgfsetroundjoin%
\pgfsetlinewidth{0.803000pt}%
\definecolor{currentstroke}{rgb}{0.000000,0.000000,0.000000}%
\pgfsetstrokecolor{currentstroke}%
\pgfsetdash{}{0pt}%
\pgfpathmoveto{\pgfqpoint{5.135151in}{1.650740in}}%
\pgfpathlineto{\pgfqpoint{5.196062in}{1.640717in}}%
\pgfusepath{stroke}%
\end{pgfscope}%
\begin{pgfscope}%
\pgftext[x=5.395231in,y=1.666500in,,top]{\rmfamily\fontsize{8.000000}{9.600000}\selectfont 0.2}%
\end{pgfscope}%
\begin{pgfscope}%
\pgfsetrectcap%
\pgfsetroundjoin%
\pgfsetlinewidth{0.803000pt}%
\definecolor{currentstroke}{rgb}{0.000000,0.000000,0.000000}%
\pgfsetstrokecolor{currentstroke}%
\pgfsetdash{}{0pt}%
\pgfpathmoveto{\pgfqpoint{5.147216in}{1.859609in}}%
\pgfpathlineto{\pgfqpoint{5.208520in}{1.849708in}}%
\pgfusepath{stroke}%
\end{pgfscope}%
\begin{pgfscope}%
\pgftext[x=5.408908in,y=1.875177in,,top]{\rmfamily\fontsize{8.000000}{9.600000}\selectfont 0.3}%
\end{pgfscope}%
\begin{pgfscope}%
\pgfsetrectcap%
\pgfsetroundjoin%
\pgfsetlinewidth{0.803000pt}%
\definecolor{currentstroke}{rgb}{0.000000,0.000000,0.000000}%
\pgfsetstrokecolor{currentstroke}%
\pgfsetdash{}{0pt}%
\pgfpathmoveto{\pgfqpoint{5.159434in}{2.071113in}}%
\pgfpathlineto{\pgfqpoint{5.221135in}{2.061338in}}%
\pgfusepath{stroke}%
\end{pgfscope}%
\begin{pgfscope}%
\pgftext[x=5.422757in,y=2.086482in,,top]{\rmfamily\fontsize{8.000000}{9.600000}\selectfont 0.4}%
\end{pgfscope}%
\begin{pgfscope}%
\pgfsetrectcap%
\pgfsetroundjoin%
\pgfsetlinewidth{0.803000pt}%
\definecolor{currentstroke}{rgb}{0.000000,0.000000,0.000000}%
\pgfsetstrokecolor{currentstroke}%
\pgfsetdash{}{0pt}%
\pgfpathmoveto{\pgfqpoint{5.171806in}{2.285303in}}%
\pgfpathlineto{\pgfqpoint{5.233911in}{2.275658in}}%
\pgfusepath{stroke}%
\end{pgfscope}%
\begin{pgfscope}%
\pgftext[x=5.436782in,y=2.300467in,,top]{\rmfamily\fontsize{8.000000}{9.600000}\selectfont 0.5}%
\end{pgfscope}%
\begin{pgfscope}%
\pgfsetrectcap%
\pgfsetroundjoin%
\pgfsetlinewidth{0.803000pt}%
\definecolor{currentstroke}{rgb}{0.000000,0.000000,0.000000}%
\pgfsetstrokecolor{currentstroke}%
\pgfsetdash{}{0pt}%
\pgfpathmoveto{\pgfqpoint{5.184337in}{2.502230in}}%
\pgfpathlineto{\pgfqpoint{5.246851in}{2.492720in}}%
\pgfusepath{stroke}%
\end{pgfscope}%
\begin{pgfscope}%
\pgftext[x=5.450985in,y=2.517183in,,top]{\rmfamily\fontsize{8.000000}{9.600000}\selectfont 0.6}%
\end{pgfscope}%
\begin{pgfscope}%
\pgfpathrectangle{\pgfqpoint{0.017500in}{0.017500in}}{\pgfqpoint{6.218282in}{3.580372in}} %
\pgfusepath{clip}%
\pgfsetbuttcap%
\pgfsetroundjoin%
\definecolor{currentfill}{rgb}{0.906154,0.842091,0.806151}%
\pgfsetfillcolor{currentfill}%
\pgfsetlinewidth{0.000000pt}%
\definecolor{currentstroke}{rgb}{0.000000,0.000000,0.000000}%
\pgfsetstrokecolor{currentstroke}%
\pgfsetdash{}{0pt}%
\pgfpathmoveto{\pgfqpoint{2.755563in}{3.027050in}}%
\pgfpathlineto{\pgfqpoint{1.486050in}{1.042638in}}%
\pgfpathlineto{\pgfqpoint{2.755563in}{3.027050in}}%
\pgfpathclose%
\pgfusepath{fill}%
\end{pgfscope}%
\begin{pgfscope}%
\pgfpathrectangle{\pgfqpoint{0.017500in}{0.017500in}}{\pgfqpoint{6.218282in}{3.580372in}} %
\pgfusepath{clip}%
\pgfsetbuttcap%
\pgfsetroundjoin%
\definecolor{currentfill}{rgb}{0.906154,0.842091,0.806151}%
\pgfsetfillcolor{currentfill}%
\pgfsetlinewidth{0.000000pt}%
\definecolor{currentstroke}{rgb}{0.000000,0.000000,0.000000}%
\pgfsetstrokecolor{currentstroke}%
\pgfsetdash{}{0pt}%
\pgfpathmoveto{\pgfqpoint{2.755563in}{3.027050in}}%
\pgfpathlineto{\pgfqpoint{1.486050in}{1.042638in}}%
\pgfpathlineto{\pgfqpoint{2.755563in}{3.027050in}}%
\pgfpathclose%
\pgfusepath{fill}%
\end{pgfscope}%
\begin{pgfscope}%
\pgfpathrectangle{\pgfqpoint{0.017500in}{0.017500in}}{\pgfqpoint{6.218282in}{3.580372in}} %
\pgfusepath{clip}%
\pgfsetbuttcap%
\pgfsetroundjoin%
\definecolor{currentfill}{rgb}{0.906154,0.842091,0.806151}%
\pgfsetfillcolor{currentfill}%
\pgfsetlinewidth{0.000000pt}%
\definecolor{currentstroke}{rgb}{0.000000,0.000000,0.000000}%
\pgfsetstrokecolor{currentstroke}%
\pgfsetdash{}{0pt}%
\pgfpathmoveto{\pgfqpoint{2.755563in}{3.027050in}}%
\pgfpathlineto{\pgfqpoint{1.486050in}{1.042638in}}%
\pgfpathlineto{\pgfqpoint{2.755563in}{3.027050in}}%
\pgfpathclose%
\pgfusepath{fill}%
\end{pgfscope}%
\begin{pgfscope}%
\pgfpathrectangle{\pgfqpoint{0.017500in}{0.017500in}}{\pgfqpoint{6.218282in}{3.580372in}} %
\pgfusepath{clip}%
\pgfsetbuttcap%
\pgfsetroundjoin%
\definecolor{currentfill}{rgb}{0.906154,0.842091,0.806151}%
\pgfsetfillcolor{currentfill}%
\pgfsetlinewidth{0.000000pt}%
\definecolor{currentstroke}{rgb}{0.000000,0.000000,0.000000}%
\pgfsetstrokecolor{currentstroke}%
\pgfsetdash{}{0pt}%
\pgfpathmoveto{\pgfqpoint{2.755563in}{3.027050in}}%
\pgfpathlineto{\pgfqpoint{1.486050in}{1.042638in}}%
\pgfpathlineto{\pgfqpoint{2.755563in}{3.027050in}}%
\pgfpathclose%
\pgfusepath{fill}%
\end{pgfscope}%
\begin{pgfscope}%
\pgfpathrectangle{\pgfqpoint{0.017500in}{0.017500in}}{\pgfqpoint{6.218282in}{3.580372in}} %
\pgfusepath{clip}%
\pgfsetbuttcap%
\pgfsetroundjoin%
\definecolor{currentfill}{rgb}{0.906154,0.842091,0.806151}%
\pgfsetfillcolor{currentfill}%
\pgfsetlinewidth{0.000000pt}%
\definecolor{currentstroke}{rgb}{0.000000,0.000000,0.000000}%
\pgfsetstrokecolor{currentstroke}%
\pgfsetdash{}{0pt}%
\pgfpathmoveto{\pgfqpoint{2.755563in}{3.027050in}}%
\pgfpathlineto{\pgfqpoint{1.486050in}{1.042638in}}%
\pgfpathlineto{\pgfqpoint{2.755563in}{3.027050in}}%
\pgfpathclose%
\pgfusepath{fill}%
\end{pgfscope}%
\begin{pgfscope}%
\pgfpathrectangle{\pgfqpoint{0.017500in}{0.017500in}}{\pgfqpoint{6.218282in}{3.580372in}} %
\pgfusepath{clip}%
\pgfsetbuttcap%
\pgfsetroundjoin%
\definecolor{currentfill}{rgb}{0.906154,0.842091,0.806151}%
\pgfsetfillcolor{currentfill}%
\pgfsetlinewidth{0.000000pt}%
\definecolor{currentstroke}{rgb}{0.000000,0.000000,0.000000}%
\pgfsetstrokecolor{currentstroke}%
\pgfsetdash{}{0pt}%
\pgfpathmoveto{\pgfqpoint{2.755563in}{3.027050in}}%
\pgfpathlineto{\pgfqpoint{1.486050in}{1.042638in}}%
\pgfpathlineto{\pgfqpoint{2.755563in}{3.027050in}}%
\pgfpathclose%
\pgfusepath{fill}%
\end{pgfscope}%
\begin{pgfscope}%
\pgfpathrectangle{\pgfqpoint{0.017500in}{0.017500in}}{\pgfqpoint{6.218282in}{3.580372in}} %
\pgfusepath{clip}%
\pgfsetbuttcap%
\pgfsetroundjoin%
\definecolor{currentfill}{rgb}{0.906154,0.842091,0.806151}%
\pgfsetfillcolor{currentfill}%
\pgfsetlinewidth{0.000000pt}%
\definecolor{currentstroke}{rgb}{0.000000,0.000000,0.000000}%
\pgfsetstrokecolor{currentstroke}%
\pgfsetdash{}{0pt}%
\pgfpathmoveto{\pgfqpoint{2.755563in}{3.027050in}}%
\pgfpathlineto{\pgfqpoint{1.486050in}{1.042638in}}%
\pgfpathlineto{\pgfqpoint{2.755563in}{3.027050in}}%
\pgfpathclose%
\pgfusepath{fill}%
\end{pgfscope}%
\begin{pgfscope}%
\pgfpathrectangle{\pgfqpoint{0.017500in}{0.017500in}}{\pgfqpoint{6.218282in}{3.580372in}} %
\pgfusepath{clip}%
\pgfsetbuttcap%
\pgfsetroundjoin%
\definecolor{currentfill}{rgb}{0.906154,0.842091,0.806151}%
\pgfsetfillcolor{currentfill}%
\pgfsetlinewidth{0.000000pt}%
\definecolor{currentstroke}{rgb}{0.000000,0.000000,0.000000}%
\pgfsetstrokecolor{currentstroke}%
\pgfsetdash{}{0pt}%
\pgfpathmoveto{\pgfqpoint{2.755563in}{3.027050in}}%
\pgfpathlineto{\pgfqpoint{1.486050in}{1.042638in}}%
\pgfpathlineto{\pgfqpoint{2.755563in}{3.027050in}}%
\pgfpathclose%
\pgfusepath{fill}%
\end{pgfscope}%
\begin{pgfscope}%
\pgfpathrectangle{\pgfqpoint{0.017500in}{0.017500in}}{\pgfqpoint{6.218282in}{3.580372in}} %
\pgfusepath{clip}%
\pgfsetbuttcap%
\pgfsetroundjoin%
\definecolor{currentfill}{rgb}{0.906154,0.842091,0.806151}%
\pgfsetfillcolor{currentfill}%
\pgfsetlinewidth{0.000000pt}%
\definecolor{currentstroke}{rgb}{0.000000,0.000000,0.000000}%
\pgfsetstrokecolor{currentstroke}%
\pgfsetdash{}{0pt}%
\pgfpathmoveto{\pgfqpoint{2.755563in}{3.027050in}}%
\pgfpathlineto{\pgfqpoint{1.486050in}{1.042638in}}%
\pgfpathlineto{\pgfqpoint{2.755563in}{3.027050in}}%
\pgfpathclose%
\pgfusepath{fill}%
\end{pgfscope}%
\begin{pgfscope}%
\pgfpathrectangle{\pgfqpoint{0.017500in}{0.017500in}}{\pgfqpoint{6.218282in}{3.580372in}} %
\pgfusepath{clip}%
\pgfsetbuttcap%
\pgfsetroundjoin%
\definecolor{currentfill}{rgb}{0.906154,0.842091,0.806151}%
\pgfsetfillcolor{currentfill}%
\pgfsetlinewidth{0.000000pt}%
\definecolor{currentstroke}{rgb}{0.000000,0.000000,0.000000}%
\pgfsetstrokecolor{currentstroke}%
\pgfsetdash{}{0pt}%
\pgfpathmoveto{\pgfqpoint{2.755563in}{3.027050in}}%
\pgfpathlineto{\pgfqpoint{1.486050in}{1.042638in}}%
\pgfpathlineto{\pgfqpoint{2.755563in}{3.027050in}}%
\pgfpathclose%
\pgfusepath{fill}%
\end{pgfscope}%
\begin{pgfscope}%
\pgfpathrectangle{\pgfqpoint{0.017500in}{0.017500in}}{\pgfqpoint{6.218282in}{3.580372in}} %
\pgfusepath{clip}%
\pgfsetbuttcap%
\pgfsetroundjoin%
\definecolor{currentfill}{rgb}{0.906154,0.842091,0.806151}%
\pgfsetfillcolor{currentfill}%
\pgfsetlinewidth{0.000000pt}%
\definecolor{currentstroke}{rgb}{0.000000,0.000000,0.000000}%
\pgfsetstrokecolor{currentstroke}%
\pgfsetdash{}{0pt}%
\pgfpathmoveto{\pgfqpoint{2.755563in}{3.027050in}}%
\pgfpathlineto{\pgfqpoint{1.486050in}{1.042638in}}%
\pgfpathlineto{\pgfqpoint{2.755563in}{3.027050in}}%
\pgfpathclose%
\pgfusepath{fill}%
\end{pgfscope}%
\begin{pgfscope}%
\pgfpathrectangle{\pgfqpoint{0.017500in}{0.017500in}}{\pgfqpoint{6.218282in}{3.580372in}} %
\pgfusepath{clip}%
\pgfsetbuttcap%
\pgfsetroundjoin%
\definecolor{currentfill}{rgb}{0.906154,0.842091,0.806151}%
\pgfsetfillcolor{currentfill}%
\pgfsetlinewidth{0.000000pt}%
\definecolor{currentstroke}{rgb}{0.000000,0.000000,0.000000}%
\pgfsetstrokecolor{currentstroke}%
\pgfsetdash{}{0pt}%
\pgfpathmoveto{\pgfqpoint{2.755563in}{3.027050in}}%
\pgfpathlineto{\pgfqpoint{1.486050in}{1.042638in}}%
\pgfpathlineto{\pgfqpoint{2.755563in}{3.027050in}}%
\pgfpathclose%
\pgfusepath{fill}%
\end{pgfscope}%
\begin{pgfscope}%
\pgfpathrectangle{\pgfqpoint{0.017500in}{0.017500in}}{\pgfqpoint{6.218282in}{3.580372in}} %
\pgfusepath{clip}%
\pgfsetbuttcap%
\pgfsetroundjoin%
\definecolor{currentfill}{rgb}{0.906154,0.842091,0.806151}%
\pgfsetfillcolor{currentfill}%
\pgfsetlinewidth{0.000000pt}%
\definecolor{currentstroke}{rgb}{0.000000,0.000000,0.000000}%
\pgfsetstrokecolor{currentstroke}%
\pgfsetdash{}{0pt}%
\pgfpathmoveto{\pgfqpoint{2.755563in}{3.027050in}}%
\pgfpathlineto{\pgfqpoint{1.486050in}{1.042638in}}%
\pgfpathlineto{\pgfqpoint{2.755563in}{3.027050in}}%
\pgfpathclose%
\pgfusepath{fill}%
\end{pgfscope}%
\begin{pgfscope}%
\pgfpathrectangle{\pgfqpoint{0.017500in}{0.017500in}}{\pgfqpoint{6.218282in}{3.580372in}} %
\pgfusepath{clip}%
\pgfsetbuttcap%
\pgfsetroundjoin%
\definecolor{currentfill}{rgb}{0.906154,0.842091,0.806151}%
\pgfsetfillcolor{currentfill}%
\pgfsetlinewidth{0.000000pt}%
\definecolor{currentstroke}{rgb}{0.000000,0.000000,0.000000}%
\pgfsetstrokecolor{currentstroke}%
\pgfsetdash{}{0pt}%
\pgfpathmoveto{\pgfqpoint{2.755563in}{3.027050in}}%
\pgfpathlineto{\pgfqpoint{1.486050in}{1.042638in}}%
\pgfpathlineto{\pgfqpoint{2.755563in}{3.027050in}}%
\pgfpathclose%
\pgfusepath{fill}%
\end{pgfscope}%
\begin{pgfscope}%
\pgfpathrectangle{\pgfqpoint{0.017500in}{0.017500in}}{\pgfqpoint{6.218282in}{3.580372in}} %
\pgfusepath{clip}%
\pgfsetbuttcap%
\pgfsetroundjoin%
\definecolor{currentfill}{rgb}{0.906154,0.842091,0.806151}%
\pgfsetfillcolor{currentfill}%
\pgfsetlinewidth{0.000000pt}%
\definecolor{currentstroke}{rgb}{0.000000,0.000000,0.000000}%
\pgfsetstrokecolor{currentstroke}%
\pgfsetdash{}{0pt}%
\pgfpathmoveto{\pgfqpoint{2.755563in}{3.027050in}}%
\pgfpathlineto{\pgfqpoint{1.486050in}{1.042638in}}%
\pgfpathlineto{\pgfqpoint{2.755563in}{3.027050in}}%
\pgfpathclose%
\pgfusepath{fill}%
\end{pgfscope}%
\begin{pgfscope}%
\pgfpathrectangle{\pgfqpoint{0.017500in}{0.017500in}}{\pgfqpoint{6.218282in}{3.580372in}} %
\pgfusepath{clip}%
\pgfsetbuttcap%
\pgfsetroundjoin%
\definecolor{currentfill}{rgb}{0.906154,0.842091,0.806151}%
\pgfsetfillcolor{currentfill}%
\pgfsetlinewidth{0.000000pt}%
\definecolor{currentstroke}{rgb}{0.000000,0.000000,0.000000}%
\pgfsetstrokecolor{currentstroke}%
\pgfsetdash{}{0pt}%
\pgfpathmoveto{\pgfqpoint{2.755563in}{3.027050in}}%
\pgfpathlineto{\pgfqpoint{1.486050in}{1.042638in}}%
\pgfpathlineto{\pgfqpoint{2.755563in}{3.027050in}}%
\pgfpathclose%
\pgfusepath{fill}%
\end{pgfscope}%
\begin{pgfscope}%
\pgfpathrectangle{\pgfqpoint{0.017500in}{0.017500in}}{\pgfqpoint{6.218282in}{3.580372in}} %
\pgfusepath{clip}%
\pgfsetbuttcap%
\pgfsetroundjoin%
\definecolor{currentfill}{rgb}{0.906154,0.842091,0.806151}%
\pgfsetfillcolor{currentfill}%
\pgfsetlinewidth{0.000000pt}%
\definecolor{currentstroke}{rgb}{0.000000,0.000000,0.000000}%
\pgfsetstrokecolor{currentstroke}%
\pgfsetdash{}{0pt}%
\pgfpathmoveto{\pgfqpoint{2.755563in}{3.027050in}}%
\pgfpathlineto{\pgfqpoint{1.486050in}{1.042638in}}%
\pgfpathlineto{\pgfqpoint{2.755563in}{3.027050in}}%
\pgfpathclose%
\pgfusepath{fill}%
\end{pgfscope}%
\begin{pgfscope}%
\pgfpathrectangle{\pgfqpoint{0.017500in}{0.017500in}}{\pgfqpoint{6.218282in}{3.580372in}} %
\pgfusepath{clip}%
\pgfsetbuttcap%
\pgfsetroundjoin%
\definecolor{currentfill}{rgb}{0.906154,0.842091,0.806151}%
\pgfsetfillcolor{currentfill}%
\pgfsetlinewidth{0.000000pt}%
\definecolor{currentstroke}{rgb}{0.000000,0.000000,0.000000}%
\pgfsetstrokecolor{currentstroke}%
\pgfsetdash{}{0pt}%
\pgfpathmoveto{\pgfqpoint{2.755563in}{3.027050in}}%
\pgfpathlineto{\pgfqpoint{1.486050in}{1.042638in}}%
\pgfpathlineto{\pgfqpoint{2.755563in}{3.027050in}}%
\pgfpathclose%
\pgfusepath{fill}%
\end{pgfscope}%
\begin{pgfscope}%
\pgfpathrectangle{\pgfqpoint{0.017500in}{0.017500in}}{\pgfqpoint{6.218282in}{3.580372in}} %
\pgfusepath{clip}%
\pgfsetbuttcap%
\pgfsetroundjoin%
\definecolor{currentfill}{rgb}{0.906154,0.842091,0.806151}%
\pgfsetfillcolor{currentfill}%
\pgfsetlinewidth{0.000000pt}%
\definecolor{currentstroke}{rgb}{0.000000,0.000000,0.000000}%
\pgfsetstrokecolor{currentstroke}%
\pgfsetdash{}{0pt}%
\pgfpathmoveto{\pgfqpoint{2.755563in}{3.027050in}}%
\pgfpathlineto{\pgfqpoint{1.486050in}{1.042638in}}%
\pgfpathlineto{\pgfqpoint{2.755563in}{3.027050in}}%
\pgfpathclose%
\pgfusepath{fill}%
\end{pgfscope}%
\begin{pgfscope}%
\pgfpathrectangle{\pgfqpoint{0.017500in}{0.017500in}}{\pgfqpoint{6.218282in}{3.580372in}} %
\pgfusepath{clip}%
\pgfsetbuttcap%
\pgfsetroundjoin%
\definecolor{currentfill}{rgb}{0.906154,0.842091,0.806151}%
\pgfsetfillcolor{currentfill}%
\pgfsetlinewidth{0.000000pt}%
\definecolor{currentstroke}{rgb}{0.000000,0.000000,0.000000}%
\pgfsetstrokecolor{currentstroke}%
\pgfsetdash{}{0pt}%
\pgfpathmoveto{\pgfqpoint{2.755563in}{3.027050in}}%
\pgfpathlineto{\pgfqpoint{1.486050in}{1.042638in}}%
\pgfpathlineto{\pgfqpoint{2.755563in}{3.027050in}}%
\pgfpathclose%
\pgfusepath{fill}%
\end{pgfscope}%
\begin{pgfscope}%
\pgfpathrectangle{\pgfqpoint{0.017500in}{0.017500in}}{\pgfqpoint{6.218282in}{3.580372in}} %
\pgfusepath{clip}%
\pgfsetbuttcap%
\pgfsetroundjoin%
\definecolor{currentfill}{rgb}{0.906154,0.842091,0.806151}%
\pgfsetfillcolor{currentfill}%
\pgfsetlinewidth{0.000000pt}%
\definecolor{currentstroke}{rgb}{0.000000,0.000000,0.000000}%
\pgfsetstrokecolor{currentstroke}%
\pgfsetdash{}{0pt}%
\pgfpathmoveto{\pgfqpoint{2.755563in}{3.027050in}}%
\pgfpathlineto{\pgfqpoint{1.486050in}{1.042638in}}%
\pgfpathlineto{\pgfqpoint{2.755563in}{3.027050in}}%
\pgfpathclose%
\pgfusepath{fill}%
\end{pgfscope}%
\begin{pgfscope}%
\pgfpathrectangle{\pgfqpoint{0.017500in}{0.017500in}}{\pgfqpoint{6.218282in}{3.580372in}} %
\pgfusepath{clip}%
\pgfsetbuttcap%
\pgfsetroundjoin%
\definecolor{currentfill}{rgb}{0.906154,0.842091,0.806151}%
\pgfsetfillcolor{currentfill}%
\pgfsetlinewidth{0.000000pt}%
\definecolor{currentstroke}{rgb}{0.000000,0.000000,0.000000}%
\pgfsetstrokecolor{currentstroke}%
\pgfsetdash{}{0pt}%
\pgfpathmoveto{\pgfqpoint{2.755563in}{3.027050in}}%
\pgfpathlineto{\pgfqpoint{1.486050in}{1.042638in}}%
\pgfpathlineto{\pgfqpoint{2.755563in}{3.027050in}}%
\pgfpathclose%
\pgfusepath{fill}%
\end{pgfscope}%
\begin{pgfscope}%
\pgfpathrectangle{\pgfqpoint{0.017500in}{0.017500in}}{\pgfqpoint{6.218282in}{3.580372in}} %
\pgfusepath{clip}%
\pgfsetbuttcap%
\pgfsetroundjoin%
\definecolor{currentfill}{rgb}{0.906154,0.842091,0.806151}%
\pgfsetfillcolor{currentfill}%
\pgfsetlinewidth{0.000000pt}%
\definecolor{currentstroke}{rgb}{0.000000,0.000000,0.000000}%
\pgfsetstrokecolor{currentstroke}%
\pgfsetdash{}{0pt}%
\pgfpathmoveto{\pgfqpoint{2.755563in}{3.027050in}}%
\pgfpathlineto{\pgfqpoint{1.486050in}{1.042638in}}%
\pgfpathlineto{\pgfqpoint{2.755563in}{3.027050in}}%
\pgfpathclose%
\pgfusepath{fill}%
\end{pgfscope}%
\begin{pgfscope}%
\pgfpathrectangle{\pgfqpoint{0.017500in}{0.017500in}}{\pgfqpoint{6.218282in}{3.580372in}} %
\pgfusepath{clip}%
\pgfsetbuttcap%
\pgfsetroundjoin%
\definecolor{currentfill}{rgb}{0.906154,0.842091,0.806151}%
\pgfsetfillcolor{currentfill}%
\pgfsetlinewidth{0.000000pt}%
\definecolor{currentstroke}{rgb}{0.000000,0.000000,0.000000}%
\pgfsetstrokecolor{currentstroke}%
\pgfsetdash{}{0pt}%
\pgfpathmoveto{\pgfqpoint{2.755563in}{3.027050in}}%
\pgfpathlineto{\pgfqpoint{1.486050in}{1.042638in}}%
\pgfpathlineto{\pgfqpoint{2.755563in}{3.027050in}}%
\pgfpathclose%
\pgfusepath{fill}%
\end{pgfscope}%
\begin{pgfscope}%
\pgfpathrectangle{\pgfqpoint{0.017500in}{0.017500in}}{\pgfqpoint{6.218282in}{3.580372in}} %
\pgfusepath{clip}%
\pgfsetbuttcap%
\pgfsetroundjoin%
\definecolor{currentfill}{rgb}{0.906154,0.842091,0.806151}%
\pgfsetfillcolor{currentfill}%
\pgfsetlinewidth{0.000000pt}%
\definecolor{currentstroke}{rgb}{0.000000,0.000000,0.000000}%
\pgfsetstrokecolor{currentstroke}%
\pgfsetdash{}{0pt}%
\pgfpathmoveto{\pgfqpoint{2.755563in}{3.027050in}}%
\pgfpathlineto{\pgfqpoint{1.486050in}{1.042638in}}%
\pgfpathlineto{\pgfqpoint{2.755563in}{3.027050in}}%
\pgfpathclose%
\pgfusepath{fill}%
\end{pgfscope}%
\begin{pgfscope}%
\pgfpathrectangle{\pgfqpoint{0.017500in}{0.017500in}}{\pgfqpoint{6.218282in}{3.580372in}} %
\pgfusepath{clip}%
\pgfsetbuttcap%
\pgfsetroundjoin%
\definecolor{currentfill}{rgb}{0.705673,0.015556,0.150233}%
\pgfsetfillcolor{currentfill}%
\pgfsetlinewidth{0.000000pt}%
\definecolor{currentstroke}{rgb}{0.000000,0.000000,0.000000}%
\pgfsetstrokecolor{currentstroke}%
\pgfsetdash{}{0pt}%
\pgfpathmoveto{\pgfqpoint{2.499217in}{2.903684in}}%
\pgfpathlineto{\pgfqpoint{2.755563in}{3.027050in}}%
\pgfpathlineto{\pgfqpoint{2.499217in}{2.903684in}}%
\pgfpathclose%
\pgfusepath{fill}%
\end{pgfscope}%
\begin{pgfscope}%
\pgfpathrectangle{\pgfqpoint{0.017500in}{0.017500in}}{\pgfqpoint{6.218282in}{3.580372in}} %
\pgfusepath{clip}%
\pgfsetbuttcap%
\pgfsetroundjoin%
\definecolor{currentfill}{rgb}{0.705673,0.015556,0.150233}%
\pgfsetfillcolor{currentfill}%
\pgfsetlinewidth{0.000000pt}%
\definecolor{currentstroke}{rgb}{0.000000,0.000000,0.000000}%
\pgfsetstrokecolor{currentstroke}%
\pgfsetdash{}{0pt}%
\pgfpathmoveto{\pgfqpoint{2.499217in}{2.903684in}}%
\pgfpathlineto{\pgfqpoint{2.755563in}{3.027050in}}%
\pgfpathlineto{\pgfqpoint{2.499217in}{2.903684in}}%
\pgfpathclose%
\pgfusepath{fill}%
\end{pgfscope}%
\begin{pgfscope}%
\pgfpathrectangle{\pgfqpoint{0.017500in}{0.017500in}}{\pgfqpoint{6.218282in}{3.580372in}} %
\pgfusepath{clip}%
\pgfsetbuttcap%
\pgfsetroundjoin%
\definecolor{currentfill}{rgb}{0.705673,0.015556,0.150233}%
\pgfsetfillcolor{currentfill}%
\pgfsetlinewidth{0.000000pt}%
\definecolor{currentstroke}{rgb}{0.000000,0.000000,0.000000}%
\pgfsetstrokecolor{currentstroke}%
\pgfsetdash{}{0pt}%
\pgfpathmoveto{\pgfqpoint{2.499217in}{2.903684in}}%
\pgfpathlineto{\pgfqpoint{2.755563in}{3.027050in}}%
\pgfpathlineto{\pgfqpoint{2.499217in}{2.903684in}}%
\pgfpathclose%
\pgfusepath{fill}%
\end{pgfscope}%
\begin{pgfscope}%
\pgfpathrectangle{\pgfqpoint{0.017500in}{0.017500in}}{\pgfqpoint{6.218282in}{3.580372in}} %
\pgfusepath{clip}%
\pgfsetbuttcap%
\pgfsetroundjoin%
\definecolor{currentfill}{rgb}{0.705673,0.015556,0.150233}%
\pgfsetfillcolor{currentfill}%
\pgfsetlinewidth{0.000000pt}%
\definecolor{currentstroke}{rgb}{0.000000,0.000000,0.000000}%
\pgfsetstrokecolor{currentstroke}%
\pgfsetdash{}{0pt}%
\pgfpathmoveto{\pgfqpoint{2.499217in}{2.903684in}}%
\pgfpathlineto{\pgfqpoint{2.755563in}{3.027050in}}%
\pgfpathlineto{\pgfqpoint{2.499217in}{2.903684in}}%
\pgfpathclose%
\pgfusepath{fill}%
\end{pgfscope}%
\begin{pgfscope}%
\pgfpathrectangle{\pgfqpoint{0.017500in}{0.017500in}}{\pgfqpoint{6.218282in}{3.580372in}} %
\pgfusepath{clip}%
\pgfsetbuttcap%
\pgfsetroundjoin%
\definecolor{currentfill}{rgb}{0.705673,0.015556,0.150233}%
\pgfsetfillcolor{currentfill}%
\pgfsetlinewidth{0.000000pt}%
\definecolor{currentstroke}{rgb}{0.000000,0.000000,0.000000}%
\pgfsetstrokecolor{currentstroke}%
\pgfsetdash{}{0pt}%
\pgfpathmoveto{\pgfqpoint{2.499217in}{2.903684in}}%
\pgfpathlineto{\pgfqpoint{2.755563in}{3.027050in}}%
\pgfpathlineto{\pgfqpoint{2.499217in}{2.903684in}}%
\pgfpathclose%
\pgfusepath{fill}%
\end{pgfscope}%
\begin{pgfscope}%
\pgfpathrectangle{\pgfqpoint{0.017500in}{0.017500in}}{\pgfqpoint{6.218282in}{3.580372in}} %
\pgfusepath{clip}%
\pgfsetbuttcap%
\pgfsetroundjoin%
\definecolor{currentfill}{rgb}{0.705673,0.015556,0.150233}%
\pgfsetfillcolor{currentfill}%
\pgfsetlinewidth{0.000000pt}%
\definecolor{currentstroke}{rgb}{0.000000,0.000000,0.000000}%
\pgfsetstrokecolor{currentstroke}%
\pgfsetdash{}{0pt}%
\pgfpathmoveto{\pgfqpoint{2.499217in}{2.903684in}}%
\pgfpathlineto{\pgfqpoint{2.755563in}{3.027050in}}%
\pgfpathlineto{\pgfqpoint{2.499217in}{2.903684in}}%
\pgfpathclose%
\pgfusepath{fill}%
\end{pgfscope}%
\begin{pgfscope}%
\pgfpathrectangle{\pgfqpoint{0.017500in}{0.017500in}}{\pgfqpoint{6.218282in}{3.580372in}} %
\pgfusepath{clip}%
\pgfsetbuttcap%
\pgfsetroundjoin%
\definecolor{currentfill}{rgb}{0.705673,0.015556,0.150233}%
\pgfsetfillcolor{currentfill}%
\pgfsetlinewidth{0.000000pt}%
\definecolor{currentstroke}{rgb}{0.000000,0.000000,0.000000}%
\pgfsetstrokecolor{currentstroke}%
\pgfsetdash{}{0pt}%
\pgfpathmoveto{\pgfqpoint{2.499217in}{2.903684in}}%
\pgfpathlineto{\pgfqpoint{2.755563in}{3.027050in}}%
\pgfpathlineto{\pgfqpoint{2.499217in}{2.903684in}}%
\pgfpathclose%
\pgfusepath{fill}%
\end{pgfscope}%
\begin{pgfscope}%
\pgfpathrectangle{\pgfqpoint{0.017500in}{0.017500in}}{\pgfqpoint{6.218282in}{3.580372in}} %
\pgfusepath{clip}%
\pgfsetbuttcap%
\pgfsetroundjoin%
\definecolor{currentfill}{rgb}{0.705673,0.015556,0.150233}%
\pgfsetfillcolor{currentfill}%
\pgfsetlinewidth{0.000000pt}%
\definecolor{currentstroke}{rgb}{0.000000,0.000000,0.000000}%
\pgfsetstrokecolor{currentstroke}%
\pgfsetdash{}{0pt}%
\pgfpathmoveto{\pgfqpoint{2.499217in}{2.903684in}}%
\pgfpathlineto{\pgfqpoint{2.755563in}{3.027050in}}%
\pgfpathlineto{\pgfqpoint{2.499217in}{2.903684in}}%
\pgfpathclose%
\pgfusepath{fill}%
\end{pgfscope}%
\begin{pgfscope}%
\pgfpathrectangle{\pgfqpoint{0.017500in}{0.017500in}}{\pgfqpoint{6.218282in}{3.580372in}} %
\pgfusepath{clip}%
\pgfsetbuttcap%
\pgfsetroundjoin%
\definecolor{currentfill}{rgb}{0.705673,0.015556,0.150233}%
\pgfsetfillcolor{currentfill}%
\pgfsetlinewidth{0.000000pt}%
\definecolor{currentstroke}{rgb}{0.000000,0.000000,0.000000}%
\pgfsetstrokecolor{currentstroke}%
\pgfsetdash{}{0pt}%
\pgfpathmoveto{\pgfqpoint{2.499217in}{2.903684in}}%
\pgfpathlineto{\pgfqpoint{2.755563in}{3.027050in}}%
\pgfpathlineto{\pgfqpoint{2.499217in}{2.903684in}}%
\pgfpathclose%
\pgfusepath{fill}%
\end{pgfscope}%
\begin{pgfscope}%
\pgfpathrectangle{\pgfqpoint{0.017500in}{0.017500in}}{\pgfqpoint{6.218282in}{3.580372in}} %
\pgfusepath{clip}%
\pgfsetbuttcap%
\pgfsetroundjoin%
\definecolor{currentfill}{rgb}{0.705673,0.015556,0.150233}%
\pgfsetfillcolor{currentfill}%
\pgfsetlinewidth{0.000000pt}%
\definecolor{currentstroke}{rgb}{0.000000,0.000000,0.000000}%
\pgfsetstrokecolor{currentstroke}%
\pgfsetdash{}{0pt}%
\pgfpathmoveto{\pgfqpoint{2.499217in}{2.903684in}}%
\pgfpathlineto{\pgfqpoint{2.755563in}{3.027050in}}%
\pgfpathlineto{\pgfqpoint{2.499217in}{2.903684in}}%
\pgfpathclose%
\pgfusepath{fill}%
\end{pgfscope}%
\begin{pgfscope}%
\pgfpathrectangle{\pgfqpoint{0.017500in}{0.017500in}}{\pgfqpoint{6.218282in}{3.580372in}} %
\pgfusepath{clip}%
\pgfsetbuttcap%
\pgfsetroundjoin%
\definecolor{currentfill}{rgb}{0.705673,0.015556,0.150233}%
\pgfsetfillcolor{currentfill}%
\pgfsetlinewidth{0.000000pt}%
\definecolor{currentstroke}{rgb}{0.000000,0.000000,0.000000}%
\pgfsetstrokecolor{currentstroke}%
\pgfsetdash{}{0pt}%
\pgfpathmoveto{\pgfqpoint{2.499217in}{2.903684in}}%
\pgfpathlineto{\pgfqpoint{2.755563in}{3.027050in}}%
\pgfpathlineto{\pgfqpoint{2.499217in}{2.903684in}}%
\pgfpathclose%
\pgfusepath{fill}%
\end{pgfscope}%
\begin{pgfscope}%
\pgfpathrectangle{\pgfqpoint{0.017500in}{0.017500in}}{\pgfqpoint{6.218282in}{3.580372in}} %
\pgfusepath{clip}%
\pgfsetbuttcap%
\pgfsetroundjoin%
\definecolor{currentfill}{rgb}{0.705673,0.015556,0.150233}%
\pgfsetfillcolor{currentfill}%
\pgfsetlinewidth{0.000000pt}%
\definecolor{currentstroke}{rgb}{0.000000,0.000000,0.000000}%
\pgfsetstrokecolor{currentstroke}%
\pgfsetdash{}{0pt}%
\pgfpathmoveto{\pgfqpoint{2.499217in}{2.903684in}}%
\pgfpathlineto{\pgfqpoint{2.755563in}{3.027050in}}%
\pgfpathlineto{\pgfqpoint{2.499217in}{2.903684in}}%
\pgfpathclose%
\pgfusepath{fill}%
\end{pgfscope}%
\begin{pgfscope}%
\pgfpathrectangle{\pgfqpoint{0.017500in}{0.017500in}}{\pgfqpoint{6.218282in}{3.580372in}} %
\pgfusepath{clip}%
\pgfsetbuttcap%
\pgfsetroundjoin%
\definecolor{currentfill}{rgb}{0.705673,0.015556,0.150233}%
\pgfsetfillcolor{currentfill}%
\pgfsetlinewidth{0.000000pt}%
\definecolor{currentstroke}{rgb}{0.000000,0.000000,0.000000}%
\pgfsetstrokecolor{currentstroke}%
\pgfsetdash{}{0pt}%
\pgfpathmoveto{\pgfqpoint{2.499217in}{2.903684in}}%
\pgfpathlineto{\pgfqpoint{2.755563in}{3.027050in}}%
\pgfpathlineto{\pgfqpoint{2.499217in}{2.903684in}}%
\pgfpathclose%
\pgfusepath{fill}%
\end{pgfscope}%
\begin{pgfscope}%
\pgfpathrectangle{\pgfqpoint{0.017500in}{0.017500in}}{\pgfqpoint{6.218282in}{3.580372in}} %
\pgfusepath{clip}%
\pgfsetbuttcap%
\pgfsetroundjoin%
\definecolor{currentfill}{rgb}{0.705673,0.015556,0.150233}%
\pgfsetfillcolor{currentfill}%
\pgfsetlinewidth{0.000000pt}%
\definecolor{currentstroke}{rgb}{0.000000,0.000000,0.000000}%
\pgfsetstrokecolor{currentstroke}%
\pgfsetdash{}{0pt}%
\pgfpathmoveto{\pgfqpoint{2.499217in}{2.903684in}}%
\pgfpathlineto{\pgfqpoint{2.755563in}{3.027050in}}%
\pgfpathlineto{\pgfqpoint{2.499217in}{2.903684in}}%
\pgfpathclose%
\pgfusepath{fill}%
\end{pgfscope}%
\begin{pgfscope}%
\pgfpathrectangle{\pgfqpoint{0.017500in}{0.017500in}}{\pgfqpoint{6.218282in}{3.580372in}} %
\pgfusepath{clip}%
\pgfsetbuttcap%
\pgfsetroundjoin%
\definecolor{currentfill}{rgb}{0.705673,0.015556,0.150233}%
\pgfsetfillcolor{currentfill}%
\pgfsetlinewidth{0.000000pt}%
\definecolor{currentstroke}{rgb}{0.000000,0.000000,0.000000}%
\pgfsetstrokecolor{currentstroke}%
\pgfsetdash{}{0pt}%
\pgfpathmoveto{\pgfqpoint{2.499217in}{2.903684in}}%
\pgfpathlineto{\pgfqpoint{2.755563in}{3.027050in}}%
\pgfpathlineto{\pgfqpoint{2.499217in}{2.903684in}}%
\pgfpathclose%
\pgfusepath{fill}%
\end{pgfscope}%
\begin{pgfscope}%
\pgfpathrectangle{\pgfqpoint{0.017500in}{0.017500in}}{\pgfqpoint{6.218282in}{3.580372in}} %
\pgfusepath{clip}%
\pgfsetbuttcap%
\pgfsetroundjoin%
\definecolor{currentfill}{rgb}{0.705673,0.015556,0.150233}%
\pgfsetfillcolor{currentfill}%
\pgfsetlinewidth{0.000000pt}%
\definecolor{currentstroke}{rgb}{0.000000,0.000000,0.000000}%
\pgfsetstrokecolor{currentstroke}%
\pgfsetdash{}{0pt}%
\pgfpathmoveto{\pgfqpoint{2.499217in}{2.903684in}}%
\pgfpathlineto{\pgfqpoint{2.755563in}{3.027050in}}%
\pgfpathlineto{\pgfqpoint{2.499217in}{2.903684in}}%
\pgfpathclose%
\pgfusepath{fill}%
\end{pgfscope}%
\begin{pgfscope}%
\pgfpathrectangle{\pgfqpoint{0.017500in}{0.017500in}}{\pgfqpoint{6.218282in}{3.580372in}} %
\pgfusepath{clip}%
\pgfsetbuttcap%
\pgfsetroundjoin%
\definecolor{currentfill}{rgb}{0.705673,0.015556,0.150233}%
\pgfsetfillcolor{currentfill}%
\pgfsetlinewidth{0.000000pt}%
\definecolor{currentstroke}{rgb}{0.000000,0.000000,0.000000}%
\pgfsetstrokecolor{currentstroke}%
\pgfsetdash{}{0pt}%
\pgfpathmoveto{\pgfqpoint{2.499217in}{2.903684in}}%
\pgfpathlineto{\pgfqpoint{2.755563in}{3.027050in}}%
\pgfpathlineto{\pgfqpoint{2.499217in}{2.903684in}}%
\pgfpathclose%
\pgfusepath{fill}%
\end{pgfscope}%
\begin{pgfscope}%
\pgfpathrectangle{\pgfqpoint{0.017500in}{0.017500in}}{\pgfqpoint{6.218282in}{3.580372in}} %
\pgfusepath{clip}%
\pgfsetbuttcap%
\pgfsetroundjoin%
\definecolor{currentfill}{rgb}{0.705673,0.015556,0.150233}%
\pgfsetfillcolor{currentfill}%
\pgfsetlinewidth{0.000000pt}%
\definecolor{currentstroke}{rgb}{0.000000,0.000000,0.000000}%
\pgfsetstrokecolor{currentstroke}%
\pgfsetdash{}{0pt}%
\pgfpathmoveto{\pgfqpoint{2.499217in}{2.903684in}}%
\pgfpathlineto{\pgfqpoint{2.755563in}{3.027050in}}%
\pgfpathlineto{\pgfqpoint{2.499217in}{2.903684in}}%
\pgfpathclose%
\pgfusepath{fill}%
\end{pgfscope}%
\begin{pgfscope}%
\pgfpathrectangle{\pgfqpoint{0.017500in}{0.017500in}}{\pgfqpoint{6.218282in}{3.580372in}} %
\pgfusepath{clip}%
\pgfsetbuttcap%
\pgfsetroundjoin%
\definecolor{currentfill}{rgb}{0.705673,0.015556,0.150233}%
\pgfsetfillcolor{currentfill}%
\pgfsetlinewidth{0.000000pt}%
\definecolor{currentstroke}{rgb}{0.000000,0.000000,0.000000}%
\pgfsetstrokecolor{currentstroke}%
\pgfsetdash{}{0pt}%
\pgfpathmoveto{\pgfqpoint{2.499217in}{2.903684in}}%
\pgfpathlineto{\pgfqpoint{2.755563in}{3.027050in}}%
\pgfpathlineto{\pgfqpoint{2.499217in}{2.903684in}}%
\pgfpathclose%
\pgfusepath{fill}%
\end{pgfscope}%
\begin{pgfscope}%
\pgfpathrectangle{\pgfqpoint{0.017500in}{0.017500in}}{\pgfqpoint{6.218282in}{3.580372in}} %
\pgfusepath{clip}%
\pgfsetbuttcap%
\pgfsetroundjoin%
\definecolor{currentfill}{rgb}{0.705673,0.015556,0.150233}%
\pgfsetfillcolor{currentfill}%
\pgfsetlinewidth{0.000000pt}%
\definecolor{currentstroke}{rgb}{0.000000,0.000000,0.000000}%
\pgfsetstrokecolor{currentstroke}%
\pgfsetdash{}{0pt}%
\pgfpathmoveto{\pgfqpoint{2.499217in}{2.903684in}}%
\pgfpathlineto{\pgfqpoint{2.755563in}{3.027050in}}%
\pgfpathlineto{\pgfqpoint{2.499217in}{2.903684in}}%
\pgfpathclose%
\pgfusepath{fill}%
\end{pgfscope}%
\begin{pgfscope}%
\pgfpathrectangle{\pgfqpoint{0.017500in}{0.017500in}}{\pgfqpoint{6.218282in}{3.580372in}} %
\pgfusepath{clip}%
\pgfsetbuttcap%
\pgfsetroundjoin%
\definecolor{currentfill}{rgb}{0.705673,0.015556,0.150233}%
\pgfsetfillcolor{currentfill}%
\pgfsetlinewidth{0.000000pt}%
\definecolor{currentstroke}{rgb}{0.000000,0.000000,0.000000}%
\pgfsetstrokecolor{currentstroke}%
\pgfsetdash{}{0pt}%
\pgfpathmoveto{\pgfqpoint{2.499217in}{2.903684in}}%
\pgfpathlineto{\pgfqpoint{2.755563in}{3.027050in}}%
\pgfpathlineto{\pgfqpoint{2.499217in}{2.903684in}}%
\pgfpathclose%
\pgfusepath{fill}%
\end{pgfscope}%
\begin{pgfscope}%
\pgfpathrectangle{\pgfqpoint{0.017500in}{0.017500in}}{\pgfqpoint{6.218282in}{3.580372in}} %
\pgfusepath{clip}%
\pgfsetbuttcap%
\pgfsetroundjoin%
\definecolor{currentfill}{rgb}{0.705673,0.015556,0.150233}%
\pgfsetfillcolor{currentfill}%
\pgfsetlinewidth{0.000000pt}%
\definecolor{currentstroke}{rgb}{0.000000,0.000000,0.000000}%
\pgfsetstrokecolor{currentstroke}%
\pgfsetdash{}{0pt}%
\pgfpathmoveto{\pgfqpoint{2.499217in}{2.903684in}}%
\pgfpathlineto{\pgfqpoint{2.755563in}{3.027050in}}%
\pgfpathlineto{\pgfqpoint{2.499217in}{2.903684in}}%
\pgfpathclose%
\pgfusepath{fill}%
\end{pgfscope}%
\begin{pgfscope}%
\pgfpathrectangle{\pgfqpoint{0.017500in}{0.017500in}}{\pgfqpoint{6.218282in}{3.580372in}} %
\pgfusepath{clip}%
\pgfsetbuttcap%
\pgfsetroundjoin%
\definecolor{currentfill}{rgb}{0.705673,0.015556,0.150233}%
\pgfsetfillcolor{currentfill}%
\pgfsetlinewidth{0.000000pt}%
\definecolor{currentstroke}{rgb}{0.000000,0.000000,0.000000}%
\pgfsetstrokecolor{currentstroke}%
\pgfsetdash{}{0pt}%
\pgfpathmoveto{\pgfqpoint{2.499217in}{2.903684in}}%
\pgfpathlineto{\pgfqpoint{2.755563in}{3.027050in}}%
\pgfpathlineto{\pgfqpoint{2.499217in}{2.903684in}}%
\pgfpathclose%
\pgfusepath{fill}%
\end{pgfscope}%
\begin{pgfscope}%
\pgfpathrectangle{\pgfqpoint{0.017500in}{0.017500in}}{\pgfqpoint{6.218282in}{3.580372in}} %
\pgfusepath{clip}%
\pgfsetbuttcap%
\pgfsetroundjoin%
\definecolor{currentfill}{rgb}{0.705673,0.015556,0.150233}%
\pgfsetfillcolor{currentfill}%
\pgfsetlinewidth{0.000000pt}%
\definecolor{currentstroke}{rgb}{0.000000,0.000000,0.000000}%
\pgfsetstrokecolor{currentstroke}%
\pgfsetdash{}{0pt}%
\pgfpathmoveto{\pgfqpoint{2.499217in}{2.903684in}}%
\pgfpathlineto{\pgfqpoint{2.755563in}{3.027050in}}%
\pgfpathlineto{\pgfqpoint{2.499217in}{2.903684in}}%
\pgfpathclose%
\pgfusepath{fill}%
\end{pgfscope}%
\begin{pgfscope}%
\pgfpathrectangle{\pgfqpoint{0.017500in}{0.017500in}}{\pgfqpoint{6.218282in}{3.580372in}} %
\pgfusepath{clip}%
\pgfsetbuttcap%
\pgfsetroundjoin%
\definecolor{currentfill}{rgb}{0.705673,0.015556,0.150233}%
\pgfsetfillcolor{currentfill}%
\pgfsetlinewidth{0.000000pt}%
\definecolor{currentstroke}{rgb}{0.000000,0.000000,0.000000}%
\pgfsetstrokecolor{currentstroke}%
\pgfsetdash{}{0pt}%
\pgfpathmoveto{\pgfqpoint{2.499217in}{2.903684in}}%
\pgfpathlineto{\pgfqpoint{2.755563in}{3.027050in}}%
\pgfpathlineto{\pgfqpoint{2.499217in}{2.903684in}}%
\pgfpathclose%
\pgfusepath{fill}%
\end{pgfscope}%
\begin{pgfscope}%
\pgfpathrectangle{\pgfqpoint{0.017500in}{0.017500in}}{\pgfqpoint{6.218282in}{3.580372in}} %
\pgfusepath{clip}%
\pgfsetbuttcap%
\pgfsetroundjoin%
\definecolor{currentfill}{rgb}{0.705673,0.015556,0.150233}%
\pgfsetfillcolor{currentfill}%
\pgfsetlinewidth{0.000000pt}%
\definecolor{currentstroke}{rgb}{0.000000,0.000000,0.000000}%
\pgfsetstrokecolor{currentstroke}%
\pgfsetdash{}{0pt}%
\pgfpathmoveto{\pgfqpoint{2.499217in}{2.903684in}}%
\pgfpathlineto{\pgfqpoint{2.755563in}{3.027050in}}%
\pgfpathlineto{\pgfqpoint{2.499217in}{2.903684in}}%
\pgfpathclose%
\pgfusepath{fill}%
\end{pgfscope}%
\begin{pgfscope}%
\pgfpathrectangle{\pgfqpoint{0.017500in}{0.017500in}}{\pgfqpoint{6.218282in}{3.580372in}} %
\pgfusepath{clip}%
\pgfsetbuttcap%
\pgfsetroundjoin%
\definecolor{currentfill}{rgb}{0.705673,0.015556,0.150233}%
\pgfsetfillcolor{currentfill}%
\pgfsetlinewidth{0.000000pt}%
\definecolor{currentstroke}{rgb}{0.000000,0.000000,0.000000}%
\pgfsetstrokecolor{currentstroke}%
\pgfsetdash{}{0pt}%
\pgfpathmoveto{\pgfqpoint{2.499217in}{2.903684in}}%
\pgfpathlineto{\pgfqpoint{2.755563in}{3.027050in}}%
\pgfpathlineto{\pgfqpoint{2.499217in}{2.903684in}}%
\pgfpathclose%
\pgfusepath{fill}%
\end{pgfscope}%
\begin{pgfscope}%
\pgfpathrectangle{\pgfqpoint{0.017500in}{0.017500in}}{\pgfqpoint{6.218282in}{3.580372in}} %
\pgfusepath{clip}%
\pgfsetbuttcap%
\pgfsetroundjoin%
\definecolor{currentfill}{rgb}{0.705673,0.015556,0.150233}%
\pgfsetfillcolor{currentfill}%
\pgfsetlinewidth{0.000000pt}%
\definecolor{currentstroke}{rgb}{0.000000,0.000000,0.000000}%
\pgfsetstrokecolor{currentstroke}%
\pgfsetdash{}{0pt}%
\pgfpathmoveto{\pgfqpoint{2.499217in}{2.903684in}}%
\pgfpathlineto{\pgfqpoint{2.755563in}{3.027050in}}%
\pgfpathlineto{\pgfqpoint{2.499217in}{2.903684in}}%
\pgfpathclose%
\pgfusepath{fill}%
\end{pgfscope}%
\begin{pgfscope}%
\pgfpathrectangle{\pgfqpoint{0.017500in}{0.017500in}}{\pgfqpoint{6.218282in}{3.580372in}} %
\pgfusepath{clip}%
\pgfsetbuttcap%
\pgfsetroundjoin%
\definecolor{currentfill}{rgb}{0.705673,0.015556,0.150233}%
\pgfsetfillcolor{currentfill}%
\pgfsetlinewidth{0.000000pt}%
\definecolor{currentstroke}{rgb}{0.000000,0.000000,0.000000}%
\pgfsetstrokecolor{currentstroke}%
\pgfsetdash{}{0pt}%
\pgfpathmoveto{\pgfqpoint{2.499217in}{2.903684in}}%
\pgfpathlineto{\pgfqpoint{2.755563in}{3.027050in}}%
\pgfpathlineto{\pgfqpoint{2.499217in}{2.903684in}}%
\pgfpathclose%
\pgfusepath{fill}%
\end{pgfscope}%
\begin{pgfscope}%
\pgfpathrectangle{\pgfqpoint{0.017500in}{0.017500in}}{\pgfqpoint{6.218282in}{3.580372in}} %
\pgfusepath{clip}%
\pgfsetbuttcap%
\pgfsetroundjoin%
\definecolor{currentfill}{rgb}{0.705673,0.015556,0.150233}%
\pgfsetfillcolor{currentfill}%
\pgfsetlinewidth{0.000000pt}%
\definecolor{currentstroke}{rgb}{0.000000,0.000000,0.000000}%
\pgfsetstrokecolor{currentstroke}%
\pgfsetdash{}{0pt}%
\pgfpathmoveto{\pgfqpoint{2.499217in}{2.903684in}}%
\pgfpathlineto{\pgfqpoint{2.755563in}{3.027050in}}%
\pgfpathlineto{\pgfqpoint{2.499217in}{2.903684in}}%
\pgfpathclose%
\pgfusepath{fill}%
\end{pgfscope}%
\begin{pgfscope}%
\pgfpathrectangle{\pgfqpoint{0.017500in}{0.017500in}}{\pgfqpoint{6.218282in}{3.580372in}} %
\pgfusepath{clip}%
\pgfsetbuttcap%
\pgfsetroundjoin%
\definecolor{currentfill}{rgb}{0.809329,0.852974,0.922323}%
\pgfsetfillcolor{currentfill}%
\pgfsetlinewidth{0.000000pt}%
\definecolor{currentstroke}{rgb}{0.000000,0.000000,0.000000}%
\pgfsetstrokecolor{currentstroke}%
\pgfsetdash{}{0pt}%
\pgfpathmoveto{\pgfqpoint{2.755563in}{3.027050in}}%
\pgfpathlineto{\pgfqpoint{3.186131in}{2.954270in}}%
\pgfpathlineto{\pgfqpoint{1.914830in}{0.967594in}}%
\pgfpathlineto{\pgfqpoint{1.486050in}{1.042638in}}%
\pgfpathlineto{\pgfqpoint{2.755563in}{3.027050in}}%
\pgfpathclose%
\pgfusepath{fill}%
\end{pgfscope}%
\begin{pgfscope}%
\pgfpathrectangle{\pgfqpoint{0.017500in}{0.017500in}}{\pgfqpoint{6.218282in}{3.580372in}} %
\pgfusepath{clip}%
\pgfsetbuttcap%
\pgfsetroundjoin%
\definecolor{currentfill}{rgb}{0.809329,0.852974,0.922323}%
\pgfsetfillcolor{currentfill}%
\pgfsetlinewidth{0.000000pt}%
\definecolor{currentstroke}{rgb}{0.000000,0.000000,0.000000}%
\pgfsetstrokecolor{currentstroke}%
\pgfsetdash{}{0pt}%
\pgfpathmoveto{\pgfqpoint{2.755563in}{3.027050in}}%
\pgfpathlineto{\pgfqpoint{3.186131in}{2.954270in}}%
\pgfpathlineto{\pgfqpoint{1.914830in}{0.967594in}}%
\pgfpathlineto{\pgfqpoint{1.486050in}{1.042638in}}%
\pgfpathlineto{\pgfqpoint{2.755563in}{3.027050in}}%
\pgfpathclose%
\pgfusepath{fill}%
\end{pgfscope}%
\begin{pgfscope}%
\pgfpathrectangle{\pgfqpoint{0.017500in}{0.017500in}}{\pgfqpoint{6.218282in}{3.580372in}} %
\pgfusepath{clip}%
\pgfsetbuttcap%
\pgfsetroundjoin%
\definecolor{currentfill}{rgb}{0.809329,0.852974,0.922323}%
\pgfsetfillcolor{currentfill}%
\pgfsetlinewidth{0.000000pt}%
\definecolor{currentstroke}{rgb}{0.000000,0.000000,0.000000}%
\pgfsetstrokecolor{currentstroke}%
\pgfsetdash{}{0pt}%
\pgfpathmoveto{\pgfqpoint{2.755563in}{3.027050in}}%
\pgfpathlineto{\pgfqpoint{3.186131in}{2.954270in}}%
\pgfpathlineto{\pgfqpoint{1.914830in}{0.967594in}}%
\pgfpathlineto{\pgfqpoint{1.486050in}{1.042638in}}%
\pgfpathlineto{\pgfqpoint{2.755563in}{3.027050in}}%
\pgfpathclose%
\pgfusepath{fill}%
\end{pgfscope}%
\begin{pgfscope}%
\pgfpathrectangle{\pgfqpoint{0.017500in}{0.017500in}}{\pgfqpoint{6.218282in}{3.580372in}} %
\pgfusepath{clip}%
\pgfsetbuttcap%
\pgfsetroundjoin%
\definecolor{currentfill}{rgb}{0.809329,0.852974,0.922323}%
\pgfsetfillcolor{currentfill}%
\pgfsetlinewidth{0.000000pt}%
\definecolor{currentstroke}{rgb}{0.000000,0.000000,0.000000}%
\pgfsetstrokecolor{currentstroke}%
\pgfsetdash{}{0pt}%
\pgfpathmoveto{\pgfqpoint{2.755563in}{3.027050in}}%
\pgfpathlineto{\pgfqpoint{3.186131in}{2.954270in}}%
\pgfpathlineto{\pgfqpoint{1.914830in}{0.967594in}}%
\pgfpathlineto{\pgfqpoint{1.486050in}{1.042638in}}%
\pgfpathlineto{\pgfqpoint{2.755563in}{3.027050in}}%
\pgfpathclose%
\pgfusepath{fill}%
\end{pgfscope}%
\begin{pgfscope}%
\pgfpathrectangle{\pgfqpoint{0.017500in}{0.017500in}}{\pgfqpoint{6.218282in}{3.580372in}} %
\pgfusepath{clip}%
\pgfsetbuttcap%
\pgfsetroundjoin%
\definecolor{currentfill}{rgb}{0.809329,0.852974,0.922323}%
\pgfsetfillcolor{currentfill}%
\pgfsetlinewidth{0.000000pt}%
\definecolor{currentstroke}{rgb}{0.000000,0.000000,0.000000}%
\pgfsetstrokecolor{currentstroke}%
\pgfsetdash{}{0pt}%
\pgfpathmoveto{\pgfqpoint{2.755563in}{3.027050in}}%
\pgfpathlineto{\pgfqpoint{3.186131in}{2.954270in}}%
\pgfpathlineto{\pgfqpoint{1.914830in}{0.967594in}}%
\pgfpathlineto{\pgfqpoint{1.486050in}{1.042638in}}%
\pgfpathlineto{\pgfqpoint{2.755563in}{3.027050in}}%
\pgfpathclose%
\pgfusepath{fill}%
\end{pgfscope}%
\begin{pgfscope}%
\pgfpathrectangle{\pgfqpoint{0.017500in}{0.017500in}}{\pgfqpoint{6.218282in}{3.580372in}} %
\pgfusepath{clip}%
\pgfsetbuttcap%
\pgfsetroundjoin%
\definecolor{currentfill}{rgb}{0.711554,0.033337,0.154485}%
\pgfsetfillcolor{currentfill}%
\pgfsetlinewidth{0.000000pt}%
\definecolor{currentstroke}{rgb}{0.000000,0.000000,0.000000}%
\pgfsetstrokecolor{currentstroke}%
\pgfsetdash{}{0pt}%
\pgfpathmoveto{\pgfqpoint{2.499217in}{2.903684in}}%
\pgfpathlineto{\pgfqpoint{2.932968in}{2.820668in}}%
\pgfpathlineto{\pgfqpoint{3.186131in}{2.954270in}}%
\pgfpathlineto{\pgfqpoint{2.755563in}{3.027050in}}%
\pgfpathlineto{\pgfqpoint{2.499217in}{2.903684in}}%
\pgfpathclose%
\pgfusepath{fill}%
\end{pgfscope}%
\begin{pgfscope}%
\pgfpathrectangle{\pgfqpoint{0.017500in}{0.017500in}}{\pgfqpoint{6.218282in}{3.580372in}} %
\pgfusepath{clip}%
\pgfsetbuttcap%
\pgfsetroundjoin%
\definecolor{currentfill}{rgb}{0.711554,0.033337,0.154485}%
\pgfsetfillcolor{currentfill}%
\pgfsetlinewidth{0.000000pt}%
\definecolor{currentstroke}{rgb}{0.000000,0.000000,0.000000}%
\pgfsetstrokecolor{currentstroke}%
\pgfsetdash{}{0pt}%
\pgfpathmoveto{\pgfqpoint{2.499217in}{2.903684in}}%
\pgfpathlineto{\pgfqpoint{2.932968in}{2.820668in}}%
\pgfpathlineto{\pgfqpoint{3.186131in}{2.954270in}}%
\pgfpathlineto{\pgfqpoint{2.755563in}{3.027050in}}%
\pgfpathlineto{\pgfqpoint{2.499217in}{2.903684in}}%
\pgfpathclose%
\pgfusepath{fill}%
\end{pgfscope}%
\begin{pgfscope}%
\pgfpathrectangle{\pgfqpoint{0.017500in}{0.017500in}}{\pgfqpoint{6.218282in}{3.580372in}} %
\pgfusepath{clip}%
\pgfsetbuttcap%
\pgfsetroundjoin%
\definecolor{currentfill}{rgb}{0.711554,0.033337,0.154485}%
\pgfsetfillcolor{currentfill}%
\pgfsetlinewidth{0.000000pt}%
\definecolor{currentstroke}{rgb}{0.000000,0.000000,0.000000}%
\pgfsetstrokecolor{currentstroke}%
\pgfsetdash{}{0pt}%
\pgfpathmoveto{\pgfqpoint{2.499217in}{2.903684in}}%
\pgfpathlineto{\pgfqpoint{2.932968in}{2.820668in}}%
\pgfpathlineto{\pgfqpoint{3.186131in}{2.954270in}}%
\pgfpathlineto{\pgfqpoint{2.755563in}{3.027050in}}%
\pgfpathlineto{\pgfqpoint{2.499217in}{2.903684in}}%
\pgfpathclose%
\pgfusepath{fill}%
\end{pgfscope}%
\begin{pgfscope}%
\pgfpathrectangle{\pgfqpoint{0.017500in}{0.017500in}}{\pgfqpoint{6.218282in}{3.580372in}} %
\pgfusepath{clip}%
\pgfsetbuttcap%
\pgfsetroundjoin%
\definecolor{currentfill}{rgb}{0.711554,0.033337,0.154485}%
\pgfsetfillcolor{currentfill}%
\pgfsetlinewidth{0.000000pt}%
\definecolor{currentstroke}{rgb}{0.000000,0.000000,0.000000}%
\pgfsetstrokecolor{currentstroke}%
\pgfsetdash{}{0pt}%
\pgfpathmoveto{\pgfqpoint{2.499217in}{2.903684in}}%
\pgfpathlineto{\pgfqpoint{2.932968in}{2.820668in}}%
\pgfpathlineto{\pgfqpoint{3.186131in}{2.954270in}}%
\pgfpathlineto{\pgfqpoint{2.755563in}{3.027050in}}%
\pgfpathlineto{\pgfqpoint{2.499217in}{2.903684in}}%
\pgfpathclose%
\pgfusepath{fill}%
\end{pgfscope}%
\begin{pgfscope}%
\pgfpathrectangle{\pgfqpoint{0.017500in}{0.017500in}}{\pgfqpoint{6.218282in}{3.580372in}} %
\pgfusepath{clip}%
\pgfsetbuttcap%
\pgfsetroundjoin%
\definecolor{currentfill}{rgb}{0.711554,0.033337,0.154485}%
\pgfsetfillcolor{currentfill}%
\pgfsetlinewidth{0.000000pt}%
\definecolor{currentstroke}{rgb}{0.000000,0.000000,0.000000}%
\pgfsetstrokecolor{currentstroke}%
\pgfsetdash{}{0pt}%
\pgfpathmoveto{\pgfqpoint{2.499217in}{2.903684in}}%
\pgfpathlineto{\pgfqpoint{2.932968in}{2.820668in}}%
\pgfpathlineto{\pgfqpoint{3.186131in}{2.954270in}}%
\pgfpathlineto{\pgfqpoint{2.755563in}{3.027050in}}%
\pgfpathlineto{\pgfqpoint{2.499217in}{2.903684in}}%
\pgfpathclose%
\pgfusepath{fill}%
\end{pgfscope}%
\begin{pgfscope}%
\pgfpathrectangle{\pgfqpoint{0.017500in}{0.017500in}}{\pgfqpoint{6.218282in}{3.580372in}} %
\pgfusepath{clip}%
\pgfsetbuttcap%
\pgfsetroundjoin%
\definecolor{currentfill}{rgb}{0.711554,0.033337,0.154485}%
\pgfsetfillcolor{currentfill}%
\pgfsetlinewidth{0.000000pt}%
\definecolor{currentstroke}{rgb}{0.000000,0.000000,0.000000}%
\pgfsetstrokecolor{currentstroke}%
\pgfsetdash{}{0pt}%
\pgfpathmoveto{\pgfqpoint{2.499217in}{2.903684in}}%
\pgfpathlineto{\pgfqpoint{2.932968in}{2.820668in}}%
\pgfpathlineto{\pgfqpoint{3.186131in}{2.954270in}}%
\pgfpathlineto{\pgfqpoint{2.755563in}{3.027050in}}%
\pgfpathlineto{\pgfqpoint{2.499217in}{2.903684in}}%
\pgfpathclose%
\pgfusepath{fill}%
\end{pgfscope}%
\begin{pgfscope}%
\pgfpathrectangle{\pgfqpoint{0.017500in}{0.017500in}}{\pgfqpoint{6.218282in}{3.580372in}} %
\pgfusepath{clip}%
\pgfsetbuttcap%
\pgfsetroundjoin%
\definecolor{currentfill}{rgb}{0.902849,0.844796,0.811970}%
\pgfsetfillcolor{currentfill}%
\pgfsetlinewidth{0.000000pt}%
\definecolor{currentstroke}{rgb}{0.000000,0.000000,0.000000}%
\pgfsetstrokecolor{currentstroke}%
\pgfsetdash{}{0pt}%
\pgfpathmoveto{\pgfqpoint{3.186131in}{2.954270in}}%
\pgfpathlineto{\pgfqpoint{1.914830in}{0.967594in}}%
\pgfpathlineto{\pgfqpoint{3.186131in}{2.954270in}}%
\pgfpathclose%
\pgfusepath{fill}%
\end{pgfscope}%
\begin{pgfscope}%
\pgfpathrectangle{\pgfqpoint{0.017500in}{0.017500in}}{\pgfqpoint{6.218282in}{3.580372in}} %
\pgfusepath{clip}%
\pgfsetbuttcap%
\pgfsetroundjoin%
\definecolor{currentfill}{rgb}{0.902849,0.844796,0.811970}%
\pgfsetfillcolor{currentfill}%
\pgfsetlinewidth{0.000000pt}%
\definecolor{currentstroke}{rgb}{0.000000,0.000000,0.000000}%
\pgfsetstrokecolor{currentstroke}%
\pgfsetdash{}{0pt}%
\pgfpathmoveto{\pgfqpoint{3.186131in}{2.954270in}}%
\pgfpathlineto{\pgfqpoint{1.914830in}{0.967594in}}%
\pgfpathlineto{\pgfqpoint{3.186131in}{2.954270in}}%
\pgfpathclose%
\pgfusepath{fill}%
\end{pgfscope}%
\begin{pgfscope}%
\pgfpathrectangle{\pgfqpoint{0.017500in}{0.017500in}}{\pgfqpoint{6.218282in}{3.580372in}} %
\pgfusepath{clip}%
\pgfsetbuttcap%
\pgfsetroundjoin%
\definecolor{currentfill}{rgb}{0.902849,0.844796,0.811970}%
\pgfsetfillcolor{currentfill}%
\pgfsetlinewidth{0.000000pt}%
\definecolor{currentstroke}{rgb}{0.000000,0.000000,0.000000}%
\pgfsetstrokecolor{currentstroke}%
\pgfsetdash{}{0pt}%
\pgfpathmoveto{\pgfqpoint{3.186131in}{2.954270in}}%
\pgfpathlineto{\pgfqpoint{1.914830in}{0.967594in}}%
\pgfpathlineto{\pgfqpoint{3.186131in}{2.954270in}}%
\pgfpathclose%
\pgfusepath{fill}%
\end{pgfscope}%
\begin{pgfscope}%
\pgfpathrectangle{\pgfqpoint{0.017500in}{0.017500in}}{\pgfqpoint{6.218282in}{3.580372in}} %
\pgfusepath{clip}%
\pgfsetbuttcap%
\pgfsetroundjoin%
\definecolor{currentfill}{rgb}{0.902849,0.844796,0.811970}%
\pgfsetfillcolor{currentfill}%
\pgfsetlinewidth{0.000000pt}%
\definecolor{currentstroke}{rgb}{0.000000,0.000000,0.000000}%
\pgfsetstrokecolor{currentstroke}%
\pgfsetdash{}{0pt}%
\pgfpathmoveto{\pgfqpoint{3.186131in}{2.954270in}}%
\pgfpathlineto{\pgfqpoint{1.914830in}{0.967594in}}%
\pgfpathlineto{\pgfqpoint{3.186131in}{2.954270in}}%
\pgfpathclose%
\pgfusepath{fill}%
\end{pgfscope}%
\begin{pgfscope}%
\pgfpathrectangle{\pgfqpoint{0.017500in}{0.017500in}}{\pgfqpoint{6.218282in}{3.580372in}} %
\pgfusepath{clip}%
\pgfsetbuttcap%
\pgfsetroundjoin%
\definecolor{currentfill}{rgb}{0.902849,0.844796,0.811970}%
\pgfsetfillcolor{currentfill}%
\pgfsetlinewidth{0.000000pt}%
\definecolor{currentstroke}{rgb}{0.000000,0.000000,0.000000}%
\pgfsetstrokecolor{currentstroke}%
\pgfsetdash{}{0pt}%
\pgfpathmoveto{\pgfqpoint{3.186131in}{2.954270in}}%
\pgfpathlineto{\pgfqpoint{1.914830in}{0.967594in}}%
\pgfpathlineto{\pgfqpoint{3.186131in}{2.954270in}}%
\pgfpathclose%
\pgfusepath{fill}%
\end{pgfscope}%
\begin{pgfscope}%
\pgfpathrectangle{\pgfqpoint{0.017500in}{0.017500in}}{\pgfqpoint{6.218282in}{3.580372in}} %
\pgfusepath{clip}%
\pgfsetbuttcap%
\pgfsetroundjoin%
\definecolor{currentfill}{rgb}{0.902849,0.844796,0.811970}%
\pgfsetfillcolor{currentfill}%
\pgfsetlinewidth{0.000000pt}%
\definecolor{currentstroke}{rgb}{0.000000,0.000000,0.000000}%
\pgfsetstrokecolor{currentstroke}%
\pgfsetdash{}{0pt}%
\pgfpathmoveto{\pgfqpoint{3.186131in}{2.954270in}}%
\pgfpathlineto{\pgfqpoint{1.914830in}{0.967594in}}%
\pgfpathlineto{\pgfqpoint{3.186131in}{2.954270in}}%
\pgfpathclose%
\pgfusepath{fill}%
\end{pgfscope}%
\begin{pgfscope}%
\pgfpathrectangle{\pgfqpoint{0.017500in}{0.017500in}}{\pgfqpoint{6.218282in}{3.580372in}} %
\pgfusepath{clip}%
\pgfsetbuttcap%
\pgfsetroundjoin%
\definecolor{currentfill}{rgb}{0.902849,0.844796,0.811970}%
\pgfsetfillcolor{currentfill}%
\pgfsetlinewidth{0.000000pt}%
\definecolor{currentstroke}{rgb}{0.000000,0.000000,0.000000}%
\pgfsetstrokecolor{currentstroke}%
\pgfsetdash{}{0pt}%
\pgfpathmoveto{\pgfqpoint{3.186131in}{2.954270in}}%
\pgfpathlineto{\pgfqpoint{1.914830in}{0.967594in}}%
\pgfpathlineto{\pgfqpoint{3.186131in}{2.954270in}}%
\pgfpathclose%
\pgfusepath{fill}%
\end{pgfscope}%
\begin{pgfscope}%
\pgfpathrectangle{\pgfqpoint{0.017500in}{0.017500in}}{\pgfqpoint{6.218282in}{3.580372in}} %
\pgfusepath{clip}%
\pgfsetbuttcap%
\pgfsetroundjoin%
\definecolor{currentfill}{rgb}{0.902849,0.844796,0.811970}%
\pgfsetfillcolor{currentfill}%
\pgfsetlinewidth{0.000000pt}%
\definecolor{currentstroke}{rgb}{0.000000,0.000000,0.000000}%
\pgfsetstrokecolor{currentstroke}%
\pgfsetdash{}{0pt}%
\pgfpathmoveto{\pgfqpoint{3.186131in}{2.954270in}}%
\pgfpathlineto{\pgfqpoint{1.914830in}{0.967594in}}%
\pgfpathlineto{\pgfqpoint{3.186131in}{2.954270in}}%
\pgfpathclose%
\pgfusepath{fill}%
\end{pgfscope}%
\begin{pgfscope}%
\pgfpathrectangle{\pgfqpoint{0.017500in}{0.017500in}}{\pgfqpoint{6.218282in}{3.580372in}} %
\pgfusepath{clip}%
\pgfsetbuttcap%
\pgfsetroundjoin%
\definecolor{currentfill}{rgb}{0.902849,0.844796,0.811970}%
\pgfsetfillcolor{currentfill}%
\pgfsetlinewidth{0.000000pt}%
\definecolor{currentstroke}{rgb}{0.000000,0.000000,0.000000}%
\pgfsetstrokecolor{currentstroke}%
\pgfsetdash{}{0pt}%
\pgfpathmoveto{\pgfqpoint{3.186131in}{2.954270in}}%
\pgfpathlineto{\pgfqpoint{1.914830in}{0.967594in}}%
\pgfpathlineto{\pgfqpoint{3.186131in}{2.954270in}}%
\pgfpathclose%
\pgfusepath{fill}%
\end{pgfscope}%
\begin{pgfscope}%
\pgfpathrectangle{\pgfqpoint{0.017500in}{0.017500in}}{\pgfqpoint{6.218282in}{3.580372in}} %
\pgfusepath{clip}%
\pgfsetbuttcap%
\pgfsetroundjoin%
\definecolor{currentfill}{rgb}{0.902849,0.844796,0.811970}%
\pgfsetfillcolor{currentfill}%
\pgfsetlinewidth{0.000000pt}%
\definecolor{currentstroke}{rgb}{0.000000,0.000000,0.000000}%
\pgfsetstrokecolor{currentstroke}%
\pgfsetdash{}{0pt}%
\pgfpathmoveto{\pgfqpoint{3.186131in}{2.954270in}}%
\pgfpathlineto{\pgfqpoint{1.914830in}{0.967594in}}%
\pgfpathlineto{\pgfqpoint{3.186131in}{2.954270in}}%
\pgfpathclose%
\pgfusepath{fill}%
\end{pgfscope}%
\begin{pgfscope}%
\pgfpathrectangle{\pgfqpoint{0.017500in}{0.017500in}}{\pgfqpoint{6.218282in}{3.580372in}} %
\pgfusepath{clip}%
\pgfsetbuttcap%
\pgfsetroundjoin%
\definecolor{currentfill}{rgb}{0.902849,0.844796,0.811970}%
\pgfsetfillcolor{currentfill}%
\pgfsetlinewidth{0.000000pt}%
\definecolor{currentstroke}{rgb}{0.000000,0.000000,0.000000}%
\pgfsetstrokecolor{currentstroke}%
\pgfsetdash{}{0pt}%
\pgfpathmoveto{\pgfqpoint{3.186131in}{2.954270in}}%
\pgfpathlineto{\pgfqpoint{1.914830in}{0.967594in}}%
\pgfpathlineto{\pgfqpoint{3.186131in}{2.954270in}}%
\pgfpathclose%
\pgfusepath{fill}%
\end{pgfscope}%
\begin{pgfscope}%
\pgfpathrectangle{\pgfqpoint{0.017500in}{0.017500in}}{\pgfqpoint{6.218282in}{3.580372in}} %
\pgfusepath{clip}%
\pgfsetbuttcap%
\pgfsetroundjoin%
\definecolor{currentfill}{rgb}{0.902849,0.844796,0.811970}%
\pgfsetfillcolor{currentfill}%
\pgfsetlinewidth{0.000000pt}%
\definecolor{currentstroke}{rgb}{0.000000,0.000000,0.000000}%
\pgfsetstrokecolor{currentstroke}%
\pgfsetdash{}{0pt}%
\pgfpathmoveto{\pgfqpoint{3.186131in}{2.954270in}}%
\pgfpathlineto{\pgfqpoint{1.914830in}{0.967594in}}%
\pgfpathlineto{\pgfqpoint{3.186131in}{2.954270in}}%
\pgfpathclose%
\pgfusepath{fill}%
\end{pgfscope}%
\begin{pgfscope}%
\pgfpathrectangle{\pgfqpoint{0.017500in}{0.017500in}}{\pgfqpoint{6.218282in}{3.580372in}} %
\pgfusepath{clip}%
\pgfsetbuttcap%
\pgfsetroundjoin%
\definecolor{currentfill}{rgb}{0.902849,0.844796,0.811970}%
\pgfsetfillcolor{currentfill}%
\pgfsetlinewidth{0.000000pt}%
\definecolor{currentstroke}{rgb}{0.000000,0.000000,0.000000}%
\pgfsetstrokecolor{currentstroke}%
\pgfsetdash{}{0pt}%
\pgfpathmoveto{\pgfqpoint{3.186131in}{2.954270in}}%
\pgfpathlineto{\pgfqpoint{1.914830in}{0.967594in}}%
\pgfpathlineto{\pgfqpoint{3.186131in}{2.954270in}}%
\pgfpathclose%
\pgfusepath{fill}%
\end{pgfscope}%
\begin{pgfscope}%
\pgfpathrectangle{\pgfqpoint{0.017500in}{0.017500in}}{\pgfqpoint{6.218282in}{3.580372in}} %
\pgfusepath{clip}%
\pgfsetbuttcap%
\pgfsetroundjoin%
\definecolor{currentfill}{rgb}{0.902849,0.844796,0.811970}%
\pgfsetfillcolor{currentfill}%
\pgfsetlinewidth{0.000000pt}%
\definecolor{currentstroke}{rgb}{0.000000,0.000000,0.000000}%
\pgfsetstrokecolor{currentstroke}%
\pgfsetdash{}{0pt}%
\pgfpathmoveto{\pgfqpoint{3.186131in}{2.954270in}}%
\pgfpathlineto{\pgfqpoint{1.914830in}{0.967594in}}%
\pgfpathlineto{\pgfqpoint{3.186131in}{2.954270in}}%
\pgfpathclose%
\pgfusepath{fill}%
\end{pgfscope}%
\begin{pgfscope}%
\pgfpathrectangle{\pgfqpoint{0.017500in}{0.017500in}}{\pgfqpoint{6.218282in}{3.580372in}} %
\pgfusepath{clip}%
\pgfsetbuttcap%
\pgfsetroundjoin%
\definecolor{currentfill}{rgb}{0.902849,0.844796,0.811970}%
\pgfsetfillcolor{currentfill}%
\pgfsetlinewidth{0.000000pt}%
\definecolor{currentstroke}{rgb}{0.000000,0.000000,0.000000}%
\pgfsetstrokecolor{currentstroke}%
\pgfsetdash{}{0pt}%
\pgfpathmoveto{\pgfqpoint{3.186131in}{2.954270in}}%
\pgfpathlineto{\pgfqpoint{1.914830in}{0.967594in}}%
\pgfpathlineto{\pgfqpoint{3.186131in}{2.954270in}}%
\pgfpathclose%
\pgfusepath{fill}%
\end{pgfscope}%
\begin{pgfscope}%
\pgfpathrectangle{\pgfqpoint{0.017500in}{0.017500in}}{\pgfqpoint{6.218282in}{3.580372in}} %
\pgfusepath{clip}%
\pgfsetbuttcap%
\pgfsetroundjoin%
\definecolor{currentfill}{rgb}{0.902849,0.844796,0.811970}%
\pgfsetfillcolor{currentfill}%
\pgfsetlinewidth{0.000000pt}%
\definecolor{currentstroke}{rgb}{0.000000,0.000000,0.000000}%
\pgfsetstrokecolor{currentstroke}%
\pgfsetdash{}{0pt}%
\pgfpathmoveto{\pgfqpoint{3.186131in}{2.954270in}}%
\pgfpathlineto{\pgfqpoint{1.914830in}{0.967594in}}%
\pgfpathlineto{\pgfqpoint{3.186131in}{2.954270in}}%
\pgfpathclose%
\pgfusepath{fill}%
\end{pgfscope}%
\begin{pgfscope}%
\pgfpathrectangle{\pgfqpoint{0.017500in}{0.017500in}}{\pgfqpoint{6.218282in}{3.580372in}} %
\pgfusepath{clip}%
\pgfsetbuttcap%
\pgfsetroundjoin%
\definecolor{currentfill}{rgb}{0.902849,0.844796,0.811970}%
\pgfsetfillcolor{currentfill}%
\pgfsetlinewidth{0.000000pt}%
\definecolor{currentstroke}{rgb}{0.000000,0.000000,0.000000}%
\pgfsetstrokecolor{currentstroke}%
\pgfsetdash{}{0pt}%
\pgfpathmoveto{\pgfqpoint{3.186131in}{2.954270in}}%
\pgfpathlineto{\pgfqpoint{1.914830in}{0.967594in}}%
\pgfpathlineto{\pgfqpoint{3.186131in}{2.954270in}}%
\pgfpathclose%
\pgfusepath{fill}%
\end{pgfscope}%
\begin{pgfscope}%
\pgfpathrectangle{\pgfqpoint{0.017500in}{0.017500in}}{\pgfqpoint{6.218282in}{3.580372in}} %
\pgfusepath{clip}%
\pgfsetbuttcap%
\pgfsetroundjoin%
\definecolor{currentfill}{rgb}{0.902849,0.844796,0.811970}%
\pgfsetfillcolor{currentfill}%
\pgfsetlinewidth{0.000000pt}%
\definecolor{currentstroke}{rgb}{0.000000,0.000000,0.000000}%
\pgfsetstrokecolor{currentstroke}%
\pgfsetdash{}{0pt}%
\pgfpathmoveto{\pgfqpoint{3.186131in}{2.954270in}}%
\pgfpathlineto{\pgfqpoint{1.914830in}{0.967594in}}%
\pgfpathlineto{\pgfqpoint{3.186131in}{2.954270in}}%
\pgfpathclose%
\pgfusepath{fill}%
\end{pgfscope}%
\begin{pgfscope}%
\pgfpathrectangle{\pgfqpoint{0.017500in}{0.017500in}}{\pgfqpoint{6.218282in}{3.580372in}} %
\pgfusepath{clip}%
\pgfsetbuttcap%
\pgfsetroundjoin%
\definecolor{currentfill}{rgb}{0.902849,0.844796,0.811970}%
\pgfsetfillcolor{currentfill}%
\pgfsetlinewidth{0.000000pt}%
\definecolor{currentstroke}{rgb}{0.000000,0.000000,0.000000}%
\pgfsetstrokecolor{currentstroke}%
\pgfsetdash{}{0pt}%
\pgfpathmoveto{\pgfqpoint{3.186131in}{2.954270in}}%
\pgfpathlineto{\pgfqpoint{1.914830in}{0.967594in}}%
\pgfpathlineto{\pgfqpoint{3.186131in}{2.954270in}}%
\pgfpathclose%
\pgfusepath{fill}%
\end{pgfscope}%
\begin{pgfscope}%
\pgfpathrectangle{\pgfqpoint{0.017500in}{0.017500in}}{\pgfqpoint{6.218282in}{3.580372in}} %
\pgfusepath{clip}%
\pgfsetbuttcap%
\pgfsetroundjoin%
\definecolor{currentfill}{rgb}{0.902849,0.844796,0.811970}%
\pgfsetfillcolor{currentfill}%
\pgfsetlinewidth{0.000000pt}%
\definecolor{currentstroke}{rgb}{0.000000,0.000000,0.000000}%
\pgfsetstrokecolor{currentstroke}%
\pgfsetdash{}{0pt}%
\pgfpathmoveto{\pgfqpoint{3.186131in}{2.954270in}}%
\pgfpathlineto{\pgfqpoint{1.914830in}{0.967594in}}%
\pgfpathlineto{\pgfqpoint{3.186131in}{2.954270in}}%
\pgfpathclose%
\pgfusepath{fill}%
\end{pgfscope}%
\begin{pgfscope}%
\pgfpathrectangle{\pgfqpoint{0.017500in}{0.017500in}}{\pgfqpoint{6.218282in}{3.580372in}} %
\pgfusepath{clip}%
\pgfsetbuttcap%
\pgfsetroundjoin%
\definecolor{currentfill}{rgb}{0.902849,0.844796,0.811970}%
\pgfsetfillcolor{currentfill}%
\pgfsetlinewidth{0.000000pt}%
\definecolor{currentstroke}{rgb}{0.000000,0.000000,0.000000}%
\pgfsetstrokecolor{currentstroke}%
\pgfsetdash{}{0pt}%
\pgfpathmoveto{\pgfqpoint{3.186131in}{2.954270in}}%
\pgfpathlineto{\pgfqpoint{1.914830in}{0.967594in}}%
\pgfpathlineto{\pgfqpoint{3.186131in}{2.954270in}}%
\pgfpathclose%
\pgfusepath{fill}%
\end{pgfscope}%
\begin{pgfscope}%
\pgfpathrectangle{\pgfqpoint{0.017500in}{0.017500in}}{\pgfqpoint{6.218282in}{3.580372in}} %
\pgfusepath{clip}%
\pgfsetbuttcap%
\pgfsetroundjoin%
\definecolor{currentfill}{rgb}{0.902849,0.844796,0.811970}%
\pgfsetfillcolor{currentfill}%
\pgfsetlinewidth{0.000000pt}%
\definecolor{currentstroke}{rgb}{0.000000,0.000000,0.000000}%
\pgfsetstrokecolor{currentstroke}%
\pgfsetdash{}{0pt}%
\pgfpathmoveto{\pgfqpoint{3.186131in}{2.954270in}}%
\pgfpathlineto{\pgfqpoint{1.914830in}{0.967594in}}%
\pgfpathlineto{\pgfqpoint{3.186131in}{2.954270in}}%
\pgfpathclose%
\pgfusepath{fill}%
\end{pgfscope}%
\begin{pgfscope}%
\pgfpathrectangle{\pgfqpoint{0.017500in}{0.017500in}}{\pgfqpoint{6.218282in}{3.580372in}} %
\pgfusepath{clip}%
\pgfsetbuttcap%
\pgfsetroundjoin%
\definecolor{currentfill}{rgb}{0.902849,0.844796,0.811970}%
\pgfsetfillcolor{currentfill}%
\pgfsetlinewidth{0.000000pt}%
\definecolor{currentstroke}{rgb}{0.000000,0.000000,0.000000}%
\pgfsetstrokecolor{currentstroke}%
\pgfsetdash{}{0pt}%
\pgfpathmoveto{\pgfqpoint{3.186131in}{2.954270in}}%
\pgfpathlineto{\pgfqpoint{1.914830in}{0.967594in}}%
\pgfpathlineto{\pgfqpoint{3.186131in}{2.954270in}}%
\pgfpathclose%
\pgfusepath{fill}%
\end{pgfscope}%
\begin{pgfscope}%
\pgfpathrectangle{\pgfqpoint{0.017500in}{0.017500in}}{\pgfqpoint{6.218282in}{3.580372in}} %
\pgfusepath{clip}%
\pgfsetbuttcap%
\pgfsetroundjoin%
\definecolor{currentfill}{rgb}{0.902849,0.844796,0.811970}%
\pgfsetfillcolor{currentfill}%
\pgfsetlinewidth{0.000000pt}%
\definecolor{currentstroke}{rgb}{0.000000,0.000000,0.000000}%
\pgfsetstrokecolor{currentstroke}%
\pgfsetdash{}{0pt}%
\pgfpathmoveto{\pgfqpoint{3.186131in}{2.954270in}}%
\pgfpathlineto{\pgfqpoint{1.914830in}{0.967594in}}%
\pgfpathlineto{\pgfqpoint{3.186131in}{2.954270in}}%
\pgfpathclose%
\pgfusepath{fill}%
\end{pgfscope}%
\begin{pgfscope}%
\pgfpathrectangle{\pgfqpoint{0.017500in}{0.017500in}}{\pgfqpoint{6.218282in}{3.580372in}} %
\pgfusepath{clip}%
\pgfsetbuttcap%
\pgfsetroundjoin%
\definecolor{currentfill}{rgb}{0.902849,0.844796,0.811970}%
\pgfsetfillcolor{currentfill}%
\pgfsetlinewidth{0.000000pt}%
\definecolor{currentstroke}{rgb}{0.000000,0.000000,0.000000}%
\pgfsetstrokecolor{currentstroke}%
\pgfsetdash{}{0pt}%
\pgfpathmoveto{\pgfqpoint{3.186131in}{2.954270in}}%
\pgfpathlineto{\pgfqpoint{1.914830in}{0.967594in}}%
\pgfpathlineto{\pgfqpoint{3.186131in}{2.954270in}}%
\pgfpathclose%
\pgfusepath{fill}%
\end{pgfscope}%
\begin{pgfscope}%
\pgfpathrectangle{\pgfqpoint{0.017500in}{0.017500in}}{\pgfqpoint{6.218282in}{3.580372in}} %
\pgfusepath{clip}%
\pgfsetbuttcap%
\pgfsetroundjoin%
\definecolor{currentfill}{rgb}{0.723315,0.068898,0.162989}%
\pgfsetfillcolor{currentfill}%
\pgfsetlinewidth{0.000000pt}%
\definecolor{currentstroke}{rgb}{0.000000,0.000000,0.000000}%
\pgfsetstrokecolor{currentstroke}%
\pgfsetdash{}{0pt}%
\pgfpathmoveto{\pgfqpoint{2.932968in}{2.820668in}}%
\pgfpathlineto{\pgfqpoint{3.186131in}{2.954270in}}%
\pgfpathlineto{\pgfqpoint{2.932968in}{2.820668in}}%
\pgfpathclose%
\pgfusepath{fill}%
\end{pgfscope}%
\begin{pgfscope}%
\pgfpathrectangle{\pgfqpoint{0.017500in}{0.017500in}}{\pgfqpoint{6.218282in}{3.580372in}} %
\pgfusepath{clip}%
\pgfsetbuttcap%
\pgfsetroundjoin%
\definecolor{currentfill}{rgb}{0.723315,0.068898,0.162989}%
\pgfsetfillcolor{currentfill}%
\pgfsetlinewidth{0.000000pt}%
\definecolor{currentstroke}{rgb}{0.000000,0.000000,0.000000}%
\pgfsetstrokecolor{currentstroke}%
\pgfsetdash{}{0pt}%
\pgfpathmoveto{\pgfqpoint{2.932968in}{2.820668in}}%
\pgfpathlineto{\pgfqpoint{3.186131in}{2.954270in}}%
\pgfpathlineto{\pgfqpoint{2.932968in}{2.820668in}}%
\pgfpathclose%
\pgfusepath{fill}%
\end{pgfscope}%
\begin{pgfscope}%
\pgfpathrectangle{\pgfqpoint{0.017500in}{0.017500in}}{\pgfqpoint{6.218282in}{3.580372in}} %
\pgfusepath{clip}%
\pgfsetbuttcap%
\pgfsetroundjoin%
\definecolor{currentfill}{rgb}{0.723315,0.068898,0.162989}%
\pgfsetfillcolor{currentfill}%
\pgfsetlinewidth{0.000000pt}%
\definecolor{currentstroke}{rgb}{0.000000,0.000000,0.000000}%
\pgfsetstrokecolor{currentstroke}%
\pgfsetdash{}{0pt}%
\pgfpathmoveto{\pgfqpoint{2.932968in}{2.820668in}}%
\pgfpathlineto{\pgfqpoint{3.186131in}{2.954270in}}%
\pgfpathlineto{\pgfqpoint{2.932968in}{2.820668in}}%
\pgfpathclose%
\pgfusepath{fill}%
\end{pgfscope}%
\begin{pgfscope}%
\pgfpathrectangle{\pgfqpoint{0.017500in}{0.017500in}}{\pgfqpoint{6.218282in}{3.580372in}} %
\pgfusepath{clip}%
\pgfsetbuttcap%
\pgfsetroundjoin%
\definecolor{currentfill}{rgb}{0.723315,0.068898,0.162989}%
\pgfsetfillcolor{currentfill}%
\pgfsetlinewidth{0.000000pt}%
\definecolor{currentstroke}{rgb}{0.000000,0.000000,0.000000}%
\pgfsetstrokecolor{currentstroke}%
\pgfsetdash{}{0pt}%
\pgfpathmoveto{\pgfqpoint{2.932968in}{2.820668in}}%
\pgfpathlineto{\pgfqpoint{3.186131in}{2.954270in}}%
\pgfpathlineto{\pgfqpoint{2.932968in}{2.820668in}}%
\pgfpathclose%
\pgfusepath{fill}%
\end{pgfscope}%
\begin{pgfscope}%
\pgfpathrectangle{\pgfqpoint{0.017500in}{0.017500in}}{\pgfqpoint{6.218282in}{3.580372in}} %
\pgfusepath{clip}%
\pgfsetbuttcap%
\pgfsetroundjoin%
\definecolor{currentfill}{rgb}{0.723315,0.068898,0.162989}%
\pgfsetfillcolor{currentfill}%
\pgfsetlinewidth{0.000000pt}%
\definecolor{currentstroke}{rgb}{0.000000,0.000000,0.000000}%
\pgfsetstrokecolor{currentstroke}%
\pgfsetdash{}{0pt}%
\pgfpathmoveto{\pgfqpoint{2.932968in}{2.820668in}}%
\pgfpathlineto{\pgfqpoint{3.186131in}{2.954270in}}%
\pgfpathlineto{\pgfqpoint{2.932968in}{2.820668in}}%
\pgfpathclose%
\pgfusepath{fill}%
\end{pgfscope}%
\begin{pgfscope}%
\pgfpathrectangle{\pgfqpoint{0.017500in}{0.017500in}}{\pgfqpoint{6.218282in}{3.580372in}} %
\pgfusepath{clip}%
\pgfsetbuttcap%
\pgfsetroundjoin%
\definecolor{currentfill}{rgb}{0.723315,0.068898,0.162989}%
\pgfsetfillcolor{currentfill}%
\pgfsetlinewidth{0.000000pt}%
\definecolor{currentstroke}{rgb}{0.000000,0.000000,0.000000}%
\pgfsetstrokecolor{currentstroke}%
\pgfsetdash{}{0pt}%
\pgfpathmoveto{\pgfqpoint{2.932968in}{2.820668in}}%
\pgfpathlineto{\pgfqpoint{3.186131in}{2.954270in}}%
\pgfpathlineto{\pgfqpoint{2.932968in}{2.820668in}}%
\pgfpathclose%
\pgfusepath{fill}%
\end{pgfscope}%
\begin{pgfscope}%
\pgfpathrectangle{\pgfqpoint{0.017500in}{0.017500in}}{\pgfqpoint{6.218282in}{3.580372in}} %
\pgfusepath{clip}%
\pgfsetbuttcap%
\pgfsetroundjoin%
\definecolor{currentfill}{rgb}{0.723315,0.068898,0.162989}%
\pgfsetfillcolor{currentfill}%
\pgfsetlinewidth{0.000000pt}%
\definecolor{currentstroke}{rgb}{0.000000,0.000000,0.000000}%
\pgfsetstrokecolor{currentstroke}%
\pgfsetdash{}{0pt}%
\pgfpathmoveto{\pgfqpoint{2.932968in}{2.820668in}}%
\pgfpathlineto{\pgfqpoint{3.186131in}{2.954270in}}%
\pgfpathlineto{\pgfqpoint{2.932968in}{2.820668in}}%
\pgfpathclose%
\pgfusepath{fill}%
\end{pgfscope}%
\begin{pgfscope}%
\pgfpathrectangle{\pgfqpoint{0.017500in}{0.017500in}}{\pgfqpoint{6.218282in}{3.580372in}} %
\pgfusepath{clip}%
\pgfsetbuttcap%
\pgfsetroundjoin%
\definecolor{currentfill}{rgb}{0.723315,0.068898,0.162989}%
\pgfsetfillcolor{currentfill}%
\pgfsetlinewidth{0.000000pt}%
\definecolor{currentstroke}{rgb}{0.000000,0.000000,0.000000}%
\pgfsetstrokecolor{currentstroke}%
\pgfsetdash{}{0pt}%
\pgfpathmoveto{\pgfqpoint{2.932968in}{2.820668in}}%
\pgfpathlineto{\pgfqpoint{3.186131in}{2.954270in}}%
\pgfpathlineto{\pgfqpoint{2.932968in}{2.820668in}}%
\pgfpathclose%
\pgfusepath{fill}%
\end{pgfscope}%
\begin{pgfscope}%
\pgfpathrectangle{\pgfqpoint{0.017500in}{0.017500in}}{\pgfqpoint{6.218282in}{3.580372in}} %
\pgfusepath{clip}%
\pgfsetbuttcap%
\pgfsetroundjoin%
\definecolor{currentfill}{rgb}{0.723315,0.068898,0.162989}%
\pgfsetfillcolor{currentfill}%
\pgfsetlinewidth{0.000000pt}%
\definecolor{currentstroke}{rgb}{0.000000,0.000000,0.000000}%
\pgfsetstrokecolor{currentstroke}%
\pgfsetdash{}{0pt}%
\pgfpathmoveto{\pgfqpoint{2.932968in}{2.820668in}}%
\pgfpathlineto{\pgfqpoint{3.186131in}{2.954270in}}%
\pgfpathlineto{\pgfqpoint{2.932968in}{2.820668in}}%
\pgfpathclose%
\pgfusepath{fill}%
\end{pgfscope}%
\begin{pgfscope}%
\pgfpathrectangle{\pgfqpoint{0.017500in}{0.017500in}}{\pgfqpoint{6.218282in}{3.580372in}} %
\pgfusepath{clip}%
\pgfsetbuttcap%
\pgfsetroundjoin%
\definecolor{currentfill}{rgb}{0.723315,0.068898,0.162989}%
\pgfsetfillcolor{currentfill}%
\pgfsetlinewidth{0.000000pt}%
\definecolor{currentstroke}{rgb}{0.000000,0.000000,0.000000}%
\pgfsetstrokecolor{currentstroke}%
\pgfsetdash{}{0pt}%
\pgfpathmoveto{\pgfqpoint{2.932968in}{2.820668in}}%
\pgfpathlineto{\pgfqpoint{3.186131in}{2.954270in}}%
\pgfpathlineto{\pgfqpoint{2.932968in}{2.820668in}}%
\pgfpathclose%
\pgfusepath{fill}%
\end{pgfscope}%
\begin{pgfscope}%
\pgfpathrectangle{\pgfqpoint{0.017500in}{0.017500in}}{\pgfqpoint{6.218282in}{3.580372in}} %
\pgfusepath{clip}%
\pgfsetbuttcap%
\pgfsetroundjoin%
\definecolor{currentfill}{rgb}{0.723315,0.068898,0.162989}%
\pgfsetfillcolor{currentfill}%
\pgfsetlinewidth{0.000000pt}%
\definecolor{currentstroke}{rgb}{0.000000,0.000000,0.000000}%
\pgfsetstrokecolor{currentstroke}%
\pgfsetdash{}{0pt}%
\pgfpathmoveto{\pgfqpoint{2.932968in}{2.820668in}}%
\pgfpathlineto{\pgfqpoint{3.186131in}{2.954270in}}%
\pgfpathlineto{\pgfqpoint{2.932968in}{2.820668in}}%
\pgfpathclose%
\pgfusepath{fill}%
\end{pgfscope}%
\begin{pgfscope}%
\pgfpathrectangle{\pgfqpoint{0.017500in}{0.017500in}}{\pgfqpoint{6.218282in}{3.580372in}} %
\pgfusepath{clip}%
\pgfsetbuttcap%
\pgfsetroundjoin%
\definecolor{currentfill}{rgb}{0.723315,0.068898,0.162989}%
\pgfsetfillcolor{currentfill}%
\pgfsetlinewidth{0.000000pt}%
\definecolor{currentstroke}{rgb}{0.000000,0.000000,0.000000}%
\pgfsetstrokecolor{currentstroke}%
\pgfsetdash{}{0pt}%
\pgfpathmoveto{\pgfqpoint{2.932968in}{2.820668in}}%
\pgfpathlineto{\pgfqpoint{3.186131in}{2.954270in}}%
\pgfpathlineto{\pgfqpoint{2.932968in}{2.820668in}}%
\pgfpathclose%
\pgfusepath{fill}%
\end{pgfscope}%
\begin{pgfscope}%
\pgfpathrectangle{\pgfqpoint{0.017500in}{0.017500in}}{\pgfqpoint{6.218282in}{3.580372in}} %
\pgfusepath{clip}%
\pgfsetbuttcap%
\pgfsetroundjoin%
\definecolor{currentfill}{rgb}{0.723315,0.068898,0.162989}%
\pgfsetfillcolor{currentfill}%
\pgfsetlinewidth{0.000000pt}%
\definecolor{currentstroke}{rgb}{0.000000,0.000000,0.000000}%
\pgfsetstrokecolor{currentstroke}%
\pgfsetdash{}{0pt}%
\pgfpathmoveto{\pgfqpoint{2.932968in}{2.820668in}}%
\pgfpathlineto{\pgfqpoint{3.186131in}{2.954270in}}%
\pgfpathlineto{\pgfqpoint{2.932968in}{2.820668in}}%
\pgfpathclose%
\pgfusepath{fill}%
\end{pgfscope}%
\begin{pgfscope}%
\pgfpathrectangle{\pgfqpoint{0.017500in}{0.017500in}}{\pgfqpoint{6.218282in}{3.580372in}} %
\pgfusepath{clip}%
\pgfsetbuttcap%
\pgfsetroundjoin%
\definecolor{currentfill}{rgb}{0.723315,0.068898,0.162989}%
\pgfsetfillcolor{currentfill}%
\pgfsetlinewidth{0.000000pt}%
\definecolor{currentstroke}{rgb}{0.000000,0.000000,0.000000}%
\pgfsetstrokecolor{currentstroke}%
\pgfsetdash{}{0pt}%
\pgfpathmoveto{\pgfqpoint{2.932968in}{2.820668in}}%
\pgfpathlineto{\pgfqpoint{3.186131in}{2.954270in}}%
\pgfpathlineto{\pgfqpoint{2.932968in}{2.820668in}}%
\pgfpathclose%
\pgfusepath{fill}%
\end{pgfscope}%
\begin{pgfscope}%
\pgfpathrectangle{\pgfqpoint{0.017500in}{0.017500in}}{\pgfqpoint{6.218282in}{3.580372in}} %
\pgfusepath{clip}%
\pgfsetbuttcap%
\pgfsetroundjoin%
\definecolor{currentfill}{rgb}{0.723315,0.068898,0.162989}%
\pgfsetfillcolor{currentfill}%
\pgfsetlinewidth{0.000000pt}%
\definecolor{currentstroke}{rgb}{0.000000,0.000000,0.000000}%
\pgfsetstrokecolor{currentstroke}%
\pgfsetdash{}{0pt}%
\pgfpathmoveto{\pgfqpoint{2.932968in}{2.820668in}}%
\pgfpathlineto{\pgfqpoint{3.186131in}{2.954270in}}%
\pgfpathlineto{\pgfqpoint{2.932968in}{2.820668in}}%
\pgfpathclose%
\pgfusepath{fill}%
\end{pgfscope}%
\begin{pgfscope}%
\pgfpathrectangle{\pgfqpoint{0.017500in}{0.017500in}}{\pgfqpoint{6.218282in}{3.580372in}} %
\pgfusepath{clip}%
\pgfsetbuttcap%
\pgfsetroundjoin%
\definecolor{currentfill}{rgb}{0.723315,0.068898,0.162989}%
\pgfsetfillcolor{currentfill}%
\pgfsetlinewidth{0.000000pt}%
\definecolor{currentstroke}{rgb}{0.000000,0.000000,0.000000}%
\pgfsetstrokecolor{currentstroke}%
\pgfsetdash{}{0pt}%
\pgfpathmoveto{\pgfqpoint{2.932968in}{2.820668in}}%
\pgfpathlineto{\pgfqpoint{3.186131in}{2.954270in}}%
\pgfpathlineto{\pgfqpoint{2.932968in}{2.820668in}}%
\pgfpathclose%
\pgfusepath{fill}%
\end{pgfscope}%
\begin{pgfscope}%
\pgfpathrectangle{\pgfqpoint{0.017500in}{0.017500in}}{\pgfqpoint{6.218282in}{3.580372in}} %
\pgfusepath{clip}%
\pgfsetbuttcap%
\pgfsetroundjoin%
\definecolor{currentfill}{rgb}{0.723315,0.068898,0.162989}%
\pgfsetfillcolor{currentfill}%
\pgfsetlinewidth{0.000000pt}%
\definecolor{currentstroke}{rgb}{0.000000,0.000000,0.000000}%
\pgfsetstrokecolor{currentstroke}%
\pgfsetdash{}{0pt}%
\pgfpathmoveto{\pgfqpoint{2.932968in}{2.820668in}}%
\pgfpathlineto{\pgfqpoint{3.186131in}{2.954270in}}%
\pgfpathlineto{\pgfqpoint{2.932968in}{2.820668in}}%
\pgfpathclose%
\pgfusepath{fill}%
\end{pgfscope}%
\begin{pgfscope}%
\pgfpathrectangle{\pgfqpoint{0.017500in}{0.017500in}}{\pgfqpoint{6.218282in}{3.580372in}} %
\pgfusepath{clip}%
\pgfsetbuttcap%
\pgfsetroundjoin%
\definecolor{currentfill}{rgb}{0.723315,0.068898,0.162989}%
\pgfsetfillcolor{currentfill}%
\pgfsetlinewidth{0.000000pt}%
\definecolor{currentstroke}{rgb}{0.000000,0.000000,0.000000}%
\pgfsetstrokecolor{currentstroke}%
\pgfsetdash{}{0pt}%
\pgfpathmoveto{\pgfqpoint{2.932968in}{2.820668in}}%
\pgfpathlineto{\pgfqpoint{3.186131in}{2.954270in}}%
\pgfpathlineto{\pgfqpoint{2.932968in}{2.820668in}}%
\pgfpathclose%
\pgfusepath{fill}%
\end{pgfscope}%
\begin{pgfscope}%
\pgfpathrectangle{\pgfqpoint{0.017500in}{0.017500in}}{\pgfqpoint{6.218282in}{3.580372in}} %
\pgfusepath{clip}%
\pgfsetbuttcap%
\pgfsetroundjoin%
\definecolor{currentfill}{rgb}{0.723315,0.068898,0.162989}%
\pgfsetfillcolor{currentfill}%
\pgfsetlinewidth{0.000000pt}%
\definecolor{currentstroke}{rgb}{0.000000,0.000000,0.000000}%
\pgfsetstrokecolor{currentstroke}%
\pgfsetdash{}{0pt}%
\pgfpathmoveto{\pgfqpoint{2.932968in}{2.820668in}}%
\pgfpathlineto{\pgfqpoint{3.186131in}{2.954270in}}%
\pgfpathlineto{\pgfqpoint{2.932968in}{2.820668in}}%
\pgfpathclose%
\pgfusepath{fill}%
\end{pgfscope}%
\begin{pgfscope}%
\pgfpathrectangle{\pgfqpoint{0.017500in}{0.017500in}}{\pgfqpoint{6.218282in}{3.580372in}} %
\pgfusepath{clip}%
\pgfsetbuttcap%
\pgfsetroundjoin%
\definecolor{currentfill}{rgb}{0.723315,0.068898,0.162989}%
\pgfsetfillcolor{currentfill}%
\pgfsetlinewidth{0.000000pt}%
\definecolor{currentstroke}{rgb}{0.000000,0.000000,0.000000}%
\pgfsetstrokecolor{currentstroke}%
\pgfsetdash{}{0pt}%
\pgfpathmoveto{\pgfqpoint{2.932968in}{2.820668in}}%
\pgfpathlineto{\pgfqpoint{3.186131in}{2.954270in}}%
\pgfpathlineto{\pgfqpoint{2.932968in}{2.820668in}}%
\pgfpathclose%
\pgfusepath{fill}%
\end{pgfscope}%
\begin{pgfscope}%
\pgfpathrectangle{\pgfqpoint{0.017500in}{0.017500in}}{\pgfqpoint{6.218282in}{3.580372in}} %
\pgfusepath{clip}%
\pgfsetbuttcap%
\pgfsetroundjoin%
\definecolor{currentfill}{rgb}{0.723315,0.068898,0.162989}%
\pgfsetfillcolor{currentfill}%
\pgfsetlinewidth{0.000000pt}%
\definecolor{currentstroke}{rgb}{0.000000,0.000000,0.000000}%
\pgfsetstrokecolor{currentstroke}%
\pgfsetdash{}{0pt}%
\pgfpathmoveto{\pgfqpoint{2.932968in}{2.820668in}}%
\pgfpathlineto{\pgfqpoint{3.186131in}{2.954270in}}%
\pgfpathlineto{\pgfqpoint{2.932968in}{2.820668in}}%
\pgfpathclose%
\pgfusepath{fill}%
\end{pgfscope}%
\begin{pgfscope}%
\pgfpathrectangle{\pgfqpoint{0.017500in}{0.017500in}}{\pgfqpoint{6.218282in}{3.580372in}} %
\pgfusepath{clip}%
\pgfsetbuttcap%
\pgfsetroundjoin%
\definecolor{currentfill}{rgb}{0.723315,0.068898,0.162989}%
\pgfsetfillcolor{currentfill}%
\pgfsetlinewidth{0.000000pt}%
\definecolor{currentstroke}{rgb}{0.000000,0.000000,0.000000}%
\pgfsetstrokecolor{currentstroke}%
\pgfsetdash{}{0pt}%
\pgfpathmoveto{\pgfqpoint{2.932968in}{2.820668in}}%
\pgfpathlineto{\pgfqpoint{3.186131in}{2.954270in}}%
\pgfpathlineto{\pgfqpoint{2.932968in}{2.820668in}}%
\pgfpathclose%
\pgfusepath{fill}%
\end{pgfscope}%
\begin{pgfscope}%
\pgfpathrectangle{\pgfqpoint{0.017500in}{0.017500in}}{\pgfqpoint{6.218282in}{3.580372in}} %
\pgfusepath{clip}%
\pgfsetbuttcap%
\pgfsetroundjoin%
\definecolor{currentfill}{rgb}{0.723315,0.068898,0.162989}%
\pgfsetfillcolor{currentfill}%
\pgfsetlinewidth{0.000000pt}%
\definecolor{currentstroke}{rgb}{0.000000,0.000000,0.000000}%
\pgfsetstrokecolor{currentstroke}%
\pgfsetdash{}{0pt}%
\pgfpathmoveto{\pgfqpoint{2.932968in}{2.820668in}}%
\pgfpathlineto{\pgfqpoint{3.186131in}{2.954270in}}%
\pgfpathlineto{\pgfqpoint{2.932968in}{2.820668in}}%
\pgfpathclose%
\pgfusepath{fill}%
\end{pgfscope}%
\begin{pgfscope}%
\pgfpathrectangle{\pgfqpoint{0.017500in}{0.017500in}}{\pgfqpoint{6.218282in}{3.580372in}} %
\pgfusepath{clip}%
\pgfsetbuttcap%
\pgfsetroundjoin%
\definecolor{currentfill}{rgb}{0.723315,0.068898,0.162989}%
\pgfsetfillcolor{currentfill}%
\pgfsetlinewidth{0.000000pt}%
\definecolor{currentstroke}{rgb}{0.000000,0.000000,0.000000}%
\pgfsetstrokecolor{currentstroke}%
\pgfsetdash{}{0pt}%
\pgfpathmoveto{\pgfqpoint{2.932968in}{2.820668in}}%
\pgfpathlineto{\pgfqpoint{3.186131in}{2.954270in}}%
\pgfpathlineto{\pgfqpoint{2.932968in}{2.820668in}}%
\pgfpathclose%
\pgfusepath{fill}%
\end{pgfscope}%
\begin{pgfscope}%
\pgfpathrectangle{\pgfqpoint{0.017500in}{0.017500in}}{\pgfqpoint{6.218282in}{3.580372in}} %
\pgfusepath{clip}%
\pgfsetbuttcap%
\pgfsetroundjoin%
\definecolor{currentfill}{rgb}{0.723315,0.068898,0.162989}%
\pgfsetfillcolor{currentfill}%
\pgfsetlinewidth{0.000000pt}%
\definecolor{currentstroke}{rgb}{0.000000,0.000000,0.000000}%
\pgfsetstrokecolor{currentstroke}%
\pgfsetdash{}{0pt}%
\pgfpathmoveto{\pgfqpoint{2.932968in}{2.820668in}}%
\pgfpathlineto{\pgfqpoint{3.186131in}{2.954270in}}%
\pgfpathlineto{\pgfqpoint{2.932968in}{2.820668in}}%
\pgfpathclose%
\pgfusepath{fill}%
\end{pgfscope}%
\begin{pgfscope}%
\pgfpathrectangle{\pgfqpoint{0.017500in}{0.017500in}}{\pgfqpoint{6.218282in}{3.580372in}} %
\pgfusepath{clip}%
\pgfsetbuttcap%
\pgfsetroundjoin%
\definecolor{currentfill}{rgb}{0.723315,0.068898,0.162989}%
\pgfsetfillcolor{currentfill}%
\pgfsetlinewidth{0.000000pt}%
\definecolor{currentstroke}{rgb}{0.000000,0.000000,0.000000}%
\pgfsetstrokecolor{currentstroke}%
\pgfsetdash{}{0pt}%
\pgfpathmoveto{\pgfqpoint{2.932968in}{2.820668in}}%
\pgfpathlineto{\pgfqpoint{3.186131in}{2.954270in}}%
\pgfpathlineto{\pgfqpoint{2.932968in}{2.820668in}}%
\pgfpathclose%
\pgfusepath{fill}%
\end{pgfscope}%
\begin{pgfscope}%
\pgfpathrectangle{\pgfqpoint{0.017500in}{0.017500in}}{\pgfqpoint{6.218282in}{3.580372in}} %
\pgfusepath{clip}%
\pgfsetbuttcap%
\pgfsetroundjoin%
\definecolor{currentfill}{rgb}{0.723315,0.068898,0.162989}%
\pgfsetfillcolor{currentfill}%
\pgfsetlinewidth{0.000000pt}%
\definecolor{currentstroke}{rgb}{0.000000,0.000000,0.000000}%
\pgfsetstrokecolor{currentstroke}%
\pgfsetdash{}{0pt}%
\pgfpathmoveto{\pgfqpoint{2.932968in}{2.820668in}}%
\pgfpathlineto{\pgfqpoint{3.186131in}{2.954270in}}%
\pgfpathlineto{\pgfqpoint{2.932968in}{2.820668in}}%
\pgfpathclose%
\pgfusepath{fill}%
\end{pgfscope}%
\begin{pgfscope}%
\pgfpathrectangle{\pgfqpoint{0.017500in}{0.017500in}}{\pgfqpoint{6.218282in}{3.580372in}} %
\pgfusepath{clip}%
\pgfsetbuttcap%
\pgfsetroundjoin%
\definecolor{currentfill}{rgb}{0.723315,0.068898,0.162989}%
\pgfsetfillcolor{currentfill}%
\pgfsetlinewidth{0.000000pt}%
\definecolor{currentstroke}{rgb}{0.000000,0.000000,0.000000}%
\pgfsetstrokecolor{currentstroke}%
\pgfsetdash{}{0pt}%
\pgfpathmoveto{\pgfqpoint{2.932968in}{2.820668in}}%
\pgfpathlineto{\pgfqpoint{3.186131in}{2.954270in}}%
\pgfpathlineto{\pgfqpoint{2.932968in}{2.820668in}}%
\pgfpathclose%
\pgfusepath{fill}%
\end{pgfscope}%
\begin{pgfscope}%
\pgfpathrectangle{\pgfqpoint{0.017500in}{0.017500in}}{\pgfqpoint{6.218282in}{3.580372in}} %
\pgfusepath{clip}%
\pgfsetbuttcap%
\pgfsetroundjoin%
\definecolor{currentfill}{rgb}{0.723315,0.068898,0.162989}%
\pgfsetfillcolor{currentfill}%
\pgfsetlinewidth{0.000000pt}%
\definecolor{currentstroke}{rgb}{0.000000,0.000000,0.000000}%
\pgfsetstrokecolor{currentstroke}%
\pgfsetdash{}{0pt}%
\pgfpathmoveto{\pgfqpoint{2.932968in}{2.820668in}}%
\pgfpathlineto{\pgfqpoint{3.186131in}{2.954270in}}%
\pgfpathlineto{\pgfqpoint{2.932968in}{2.820668in}}%
\pgfpathclose%
\pgfusepath{fill}%
\end{pgfscope}%
\begin{pgfscope}%
\pgfpathrectangle{\pgfqpoint{0.017500in}{0.017500in}}{\pgfqpoint{6.218282in}{3.580372in}} %
\pgfusepath{clip}%
\pgfsetbuttcap%
\pgfsetroundjoin%
\definecolor{currentfill}{rgb}{0.723315,0.068898,0.162989}%
\pgfsetfillcolor{currentfill}%
\pgfsetlinewidth{0.000000pt}%
\definecolor{currentstroke}{rgb}{0.000000,0.000000,0.000000}%
\pgfsetstrokecolor{currentstroke}%
\pgfsetdash{}{0pt}%
\pgfpathmoveto{\pgfqpoint{2.932968in}{2.820668in}}%
\pgfpathlineto{\pgfqpoint{3.186131in}{2.954270in}}%
\pgfpathlineto{\pgfqpoint{2.932968in}{2.820668in}}%
\pgfpathclose%
\pgfusepath{fill}%
\end{pgfscope}%
\begin{pgfscope}%
\pgfpathrectangle{\pgfqpoint{0.017500in}{0.017500in}}{\pgfqpoint{6.218282in}{3.580372in}} %
\pgfusepath{clip}%
\pgfsetbuttcap%
\pgfsetroundjoin%
\definecolor{currentfill}{rgb}{0.804965,0.851666,0.926165}%
\pgfsetfillcolor{currentfill}%
\pgfsetlinewidth{0.000000pt}%
\definecolor{currentstroke}{rgb}{0.000000,0.000000,0.000000}%
\pgfsetstrokecolor{currentstroke}%
\pgfsetdash{}{0pt}%
\pgfpathmoveto{\pgfqpoint{3.186131in}{2.954270in}}%
\pgfpathlineto{\pgfqpoint{3.622205in}{2.891176in}}%
\pgfpathlineto{\pgfqpoint{2.349853in}{0.872712in}}%
\pgfpathlineto{\pgfqpoint{1.914830in}{0.967594in}}%
\pgfpathlineto{\pgfqpoint{3.186131in}{2.954270in}}%
\pgfpathclose%
\pgfusepath{fill}%
\end{pgfscope}%
\begin{pgfscope}%
\pgfpathrectangle{\pgfqpoint{0.017500in}{0.017500in}}{\pgfqpoint{6.218282in}{3.580372in}} %
\pgfusepath{clip}%
\pgfsetbuttcap%
\pgfsetroundjoin%
\definecolor{currentfill}{rgb}{0.804965,0.851666,0.926165}%
\pgfsetfillcolor{currentfill}%
\pgfsetlinewidth{0.000000pt}%
\definecolor{currentstroke}{rgb}{0.000000,0.000000,0.000000}%
\pgfsetstrokecolor{currentstroke}%
\pgfsetdash{}{0pt}%
\pgfpathmoveto{\pgfqpoint{3.186131in}{2.954270in}}%
\pgfpathlineto{\pgfqpoint{3.622205in}{2.891176in}}%
\pgfpathlineto{\pgfqpoint{2.349853in}{0.872712in}}%
\pgfpathlineto{\pgfqpoint{1.914830in}{0.967594in}}%
\pgfpathlineto{\pgfqpoint{3.186131in}{2.954270in}}%
\pgfpathclose%
\pgfusepath{fill}%
\end{pgfscope}%
\begin{pgfscope}%
\pgfpathrectangle{\pgfqpoint{0.017500in}{0.017500in}}{\pgfqpoint{6.218282in}{3.580372in}} %
\pgfusepath{clip}%
\pgfsetbuttcap%
\pgfsetroundjoin%
\definecolor{currentfill}{rgb}{0.804965,0.851666,0.926165}%
\pgfsetfillcolor{currentfill}%
\pgfsetlinewidth{0.000000pt}%
\definecolor{currentstroke}{rgb}{0.000000,0.000000,0.000000}%
\pgfsetstrokecolor{currentstroke}%
\pgfsetdash{}{0pt}%
\pgfpathmoveto{\pgfqpoint{3.186131in}{2.954270in}}%
\pgfpathlineto{\pgfqpoint{3.622205in}{2.891176in}}%
\pgfpathlineto{\pgfqpoint{2.349853in}{0.872712in}}%
\pgfpathlineto{\pgfqpoint{1.914830in}{0.967594in}}%
\pgfpathlineto{\pgfqpoint{3.186131in}{2.954270in}}%
\pgfpathclose%
\pgfusepath{fill}%
\end{pgfscope}%
\begin{pgfscope}%
\pgfpathrectangle{\pgfqpoint{0.017500in}{0.017500in}}{\pgfqpoint{6.218282in}{3.580372in}} %
\pgfusepath{clip}%
\pgfsetbuttcap%
\pgfsetroundjoin%
\definecolor{currentfill}{rgb}{0.804965,0.851666,0.926165}%
\pgfsetfillcolor{currentfill}%
\pgfsetlinewidth{0.000000pt}%
\definecolor{currentstroke}{rgb}{0.000000,0.000000,0.000000}%
\pgfsetstrokecolor{currentstroke}%
\pgfsetdash{}{0pt}%
\pgfpathmoveto{\pgfqpoint{3.186131in}{2.954270in}}%
\pgfpathlineto{\pgfqpoint{3.622205in}{2.891176in}}%
\pgfpathlineto{\pgfqpoint{2.349853in}{0.872712in}}%
\pgfpathlineto{\pgfqpoint{1.914830in}{0.967594in}}%
\pgfpathlineto{\pgfqpoint{3.186131in}{2.954270in}}%
\pgfpathclose%
\pgfusepath{fill}%
\end{pgfscope}%
\begin{pgfscope}%
\pgfpathrectangle{\pgfqpoint{0.017500in}{0.017500in}}{\pgfqpoint{6.218282in}{3.580372in}} %
\pgfusepath{clip}%
\pgfsetbuttcap%
\pgfsetroundjoin%
\definecolor{currentfill}{rgb}{0.804965,0.851666,0.926165}%
\pgfsetfillcolor{currentfill}%
\pgfsetlinewidth{0.000000pt}%
\definecolor{currentstroke}{rgb}{0.000000,0.000000,0.000000}%
\pgfsetstrokecolor{currentstroke}%
\pgfsetdash{}{0pt}%
\pgfpathmoveto{\pgfqpoint{3.186131in}{2.954270in}}%
\pgfpathlineto{\pgfqpoint{3.622205in}{2.891176in}}%
\pgfpathlineto{\pgfqpoint{2.349853in}{0.872712in}}%
\pgfpathlineto{\pgfqpoint{1.914830in}{0.967594in}}%
\pgfpathlineto{\pgfqpoint{3.186131in}{2.954270in}}%
\pgfpathclose%
\pgfusepath{fill}%
\end{pgfscope}%
\begin{pgfscope}%
\pgfpathrectangle{\pgfqpoint{0.017500in}{0.017500in}}{\pgfqpoint{6.218282in}{3.580372in}} %
\pgfusepath{clip}%
\pgfsetbuttcap%
\pgfsetroundjoin%
\definecolor{currentfill}{rgb}{0.717435,0.051118,0.158737}%
\pgfsetfillcolor{currentfill}%
\pgfsetlinewidth{0.000000pt}%
\definecolor{currentstroke}{rgb}{0.000000,0.000000,0.000000}%
\pgfsetstrokecolor{currentstroke}%
\pgfsetdash{}{0pt}%
\pgfpathmoveto{\pgfqpoint{2.932968in}{2.820668in}}%
\pgfpathlineto{\pgfqpoint{3.372093in}{2.761754in}}%
\pgfpathlineto{\pgfqpoint{3.622205in}{2.891176in}}%
\pgfpathlineto{\pgfqpoint{3.186131in}{2.954270in}}%
\pgfpathlineto{\pgfqpoint{2.932968in}{2.820668in}}%
\pgfpathclose%
\pgfusepath{fill}%
\end{pgfscope}%
\begin{pgfscope}%
\pgfpathrectangle{\pgfqpoint{0.017500in}{0.017500in}}{\pgfqpoint{6.218282in}{3.580372in}} %
\pgfusepath{clip}%
\pgfsetbuttcap%
\pgfsetroundjoin%
\definecolor{currentfill}{rgb}{0.717435,0.051118,0.158737}%
\pgfsetfillcolor{currentfill}%
\pgfsetlinewidth{0.000000pt}%
\definecolor{currentstroke}{rgb}{0.000000,0.000000,0.000000}%
\pgfsetstrokecolor{currentstroke}%
\pgfsetdash{}{0pt}%
\pgfpathmoveto{\pgfqpoint{2.932968in}{2.820668in}}%
\pgfpathlineto{\pgfqpoint{3.372093in}{2.761754in}}%
\pgfpathlineto{\pgfqpoint{3.622205in}{2.891176in}}%
\pgfpathlineto{\pgfqpoint{3.186131in}{2.954270in}}%
\pgfpathlineto{\pgfqpoint{2.932968in}{2.820668in}}%
\pgfpathclose%
\pgfusepath{fill}%
\end{pgfscope}%
\begin{pgfscope}%
\pgfpathrectangle{\pgfqpoint{0.017500in}{0.017500in}}{\pgfqpoint{6.218282in}{3.580372in}} %
\pgfusepath{clip}%
\pgfsetbuttcap%
\pgfsetroundjoin%
\definecolor{currentfill}{rgb}{0.717435,0.051118,0.158737}%
\pgfsetfillcolor{currentfill}%
\pgfsetlinewidth{0.000000pt}%
\definecolor{currentstroke}{rgb}{0.000000,0.000000,0.000000}%
\pgfsetstrokecolor{currentstroke}%
\pgfsetdash{}{0pt}%
\pgfpathmoveto{\pgfqpoint{2.932968in}{2.820668in}}%
\pgfpathlineto{\pgfqpoint{3.372093in}{2.761754in}}%
\pgfpathlineto{\pgfqpoint{3.622205in}{2.891176in}}%
\pgfpathlineto{\pgfqpoint{3.186131in}{2.954270in}}%
\pgfpathlineto{\pgfqpoint{2.932968in}{2.820668in}}%
\pgfpathclose%
\pgfusepath{fill}%
\end{pgfscope}%
\begin{pgfscope}%
\pgfpathrectangle{\pgfqpoint{0.017500in}{0.017500in}}{\pgfqpoint{6.218282in}{3.580372in}} %
\pgfusepath{clip}%
\pgfsetbuttcap%
\pgfsetroundjoin%
\definecolor{currentfill}{rgb}{0.717435,0.051118,0.158737}%
\pgfsetfillcolor{currentfill}%
\pgfsetlinewidth{0.000000pt}%
\definecolor{currentstroke}{rgb}{0.000000,0.000000,0.000000}%
\pgfsetstrokecolor{currentstroke}%
\pgfsetdash{}{0pt}%
\pgfpathmoveto{\pgfqpoint{2.932968in}{2.820668in}}%
\pgfpathlineto{\pgfqpoint{3.372093in}{2.761754in}}%
\pgfpathlineto{\pgfqpoint{3.622205in}{2.891176in}}%
\pgfpathlineto{\pgfqpoint{3.186131in}{2.954270in}}%
\pgfpathlineto{\pgfqpoint{2.932968in}{2.820668in}}%
\pgfpathclose%
\pgfusepath{fill}%
\end{pgfscope}%
\begin{pgfscope}%
\pgfpathrectangle{\pgfqpoint{0.017500in}{0.017500in}}{\pgfqpoint{6.218282in}{3.580372in}} %
\pgfusepath{clip}%
\pgfsetbuttcap%
\pgfsetroundjoin%
\definecolor{currentfill}{rgb}{0.717435,0.051118,0.158737}%
\pgfsetfillcolor{currentfill}%
\pgfsetlinewidth{0.000000pt}%
\definecolor{currentstroke}{rgb}{0.000000,0.000000,0.000000}%
\pgfsetstrokecolor{currentstroke}%
\pgfsetdash{}{0pt}%
\pgfpathmoveto{\pgfqpoint{2.932968in}{2.820668in}}%
\pgfpathlineto{\pgfqpoint{3.372093in}{2.761754in}}%
\pgfpathlineto{\pgfqpoint{3.622205in}{2.891176in}}%
\pgfpathlineto{\pgfqpoint{3.186131in}{2.954270in}}%
\pgfpathlineto{\pgfqpoint{2.932968in}{2.820668in}}%
\pgfpathclose%
\pgfusepath{fill}%
\end{pgfscope}%
\begin{pgfscope}%
\pgfpathrectangle{\pgfqpoint{0.017500in}{0.017500in}}{\pgfqpoint{6.218282in}{3.580372in}} %
\pgfusepath{clip}%
\pgfsetbuttcap%
\pgfsetroundjoin%
\definecolor{currentfill}{rgb}{0.717435,0.051118,0.158737}%
\pgfsetfillcolor{currentfill}%
\pgfsetlinewidth{0.000000pt}%
\definecolor{currentstroke}{rgb}{0.000000,0.000000,0.000000}%
\pgfsetstrokecolor{currentstroke}%
\pgfsetdash{}{0pt}%
\pgfpathmoveto{\pgfqpoint{2.932968in}{2.820668in}}%
\pgfpathlineto{\pgfqpoint{3.372093in}{2.761754in}}%
\pgfpathlineto{\pgfqpoint{3.622205in}{2.891176in}}%
\pgfpathlineto{\pgfqpoint{3.186131in}{2.954270in}}%
\pgfpathlineto{\pgfqpoint{2.932968in}{2.820668in}}%
\pgfpathclose%
\pgfusepath{fill}%
\end{pgfscope}%
\begin{pgfscope}%
\pgfpathrectangle{\pgfqpoint{0.017500in}{0.017500in}}{\pgfqpoint{6.218282in}{3.580372in}} %
\pgfusepath{clip}%
\pgfsetbuttcap%
\pgfsetroundjoin%
\definecolor{currentfill}{rgb}{0.740957,0.122240,0.175744}%
\pgfsetfillcolor{currentfill}%
\pgfsetlinewidth{0.000000pt}%
\definecolor{currentstroke}{rgb}{0.000000,0.000000,0.000000}%
\pgfsetstrokecolor{currentstroke}%
\pgfsetdash{}{0pt}%
\pgfpathmoveto{\pgfqpoint{2.237924in}{2.752812in}}%
\pgfpathlineto{\pgfqpoint{2.499217in}{2.903684in}}%
\pgfpathlineto{\pgfqpoint{2.237924in}{2.752812in}}%
\pgfpathclose%
\pgfusepath{fill}%
\end{pgfscope}%
\begin{pgfscope}%
\pgfpathrectangle{\pgfqpoint{0.017500in}{0.017500in}}{\pgfqpoint{6.218282in}{3.580372in}} %
\pgfusepath{clip}%
\pgfsetbuttcap%
\pgfsetroundjoin%
\definecolor{currentfill}{rgb}{0.740957,0.122240,0.175744}%
\pgfsetfillcolor{currentfill}%
\pgfsetlinewidth{0.000000pt}%
\definecolor{currentstroke}{rgb}{0.000000,0.000000,0.000000}%
\pgfsetstrokecolor{currentstroke}%
\pgfsetdash{}{0pt}%
\pgfpathmoveto{\pgfqpoint{2.237924in}{2.752812in}}%
\pgfpathlineto{\pgfqpoint{2.499217in}{2.903684in}}%
\pgfpathlineto{\pgfqpoint{2.237924in}{2.752812in}}%
\pgfpathclose%
\pgfusepath{fill}%
\end{pgfscope}%
\begin{pgfscope}%
\pgfpathrectangle{\pgfqpoint{0.017500in}{0.017500in}}{\pgfqpoint{6.218282in}{3.580372in}} %
\pgfusepath{clip}%
\pgfsetbuttcap%
\pgfsetroundjoin%
\definecolor{currentfill}{rgb}{0.740957,0.122240,0.175744}%
\pgfsetfillcolor{currentfill}%
\pgfsetlinewidth{0.000000pt}%
\definecolor{currentstroke}{rgb}{0.000000,0.000000,0.000000}%
\pgfsetstrokecolor{currentstroke}%
\pgfsetdash{}{0pt}%
\pgfpathmoveto{\pgfqpoint{2.237924in}{2.752812in}}%
\pgfpathlineto{\pgfqpoint{2.499217in}{2.903684in}}%
\pgfpathlineto{\pgfqpoint{2.237924in}{2.752812in}}%
\pgfpathclose%
\pgfusepath{fill}%
\end{pgfscope}%
\begin{pgfscope}%
\pgfpathrectangle{\pgfqpoint{0.017500in}{0.017500in}}{\pgfqpoint{6.218282in}{3.580372in}} %
\pgfusepath{clip}%
\pgfsetbuttcap%
\pgfsetroundjoin%
\definecolor{currentfill}{rgb}{0.740957,0.122240,0.175744}%
\pgfsetfillcolor{currentfill}%
\pgfsetlinewidth{0.000000pt}%
\definecolor{currentstroke}{rgb}{0.000000,0.000000,0.000000}%
\pgfsetstrokecolor{currentstroke}%
\pgfsetdash{}{0pt}%
\pgfpathmoveto{\pgfqpoint{2.237924in}{2.752812in}}%
\pgfpathlineto{\pgfqpoint{2.499217in}{2.903684in}}%
\pgfpathlineto{\pgfqpoint{2.237924in}{2.752812in}}%
\pgfpathclose%
\pgfusepath{fill}%
\end{pgfscope}%
\begin{pgfscope}%
\pgfpathrectangle{\pgfqpoint{0.017500in}{0.017500in}}{\pgfqpoint{6.218282in}{3.580372in}} %
\pgfusepath{clip}%
\pgfsetbuttcap%
\pgfsetroundjoin%
\definecolor{currentfill}{rgb}{0.740957,0.122240,0.175744}%
\pgfsetfillcolor{currentfill}%
\pgfsetlinewidth{0.000000pt}%
\definecolor{currentstroke}{rgb}{0.000000,0.000000,0.000000}%
\pgfsetstrokecolor{currentstroke}%
\pgfsetdash{}{0pt}%
\pgfpathmoveto{\pgfqpoint{2.237924in}{2.752812in}}%
\pgfpathlineto{\pgfqpoint{2.499217in}{2.903684in}}%
\pgfpathlineto{\pgfqpoint{2.237924in}{2.752812in}}%
\pgfpathclose%
\pgfusepath{fill}%
\end{pgfscope}%
\begin{pgfscope}%
\pgfpathrectangle{\pgfqpoint{0.017500in}{0.017500in}}{\pgfqpoint{6.218282in}{3.580372in}} %
\pgfusepath{clip}%
\pgfsetbuttcap%
\pgfsetroundjoin%
\definecolor{currentfill}{rgb}{0.740957,0.122240,0.175744}%
\pgfsetfillcolor{currentfill}%
\pgfsetlinewidth{0.000000pt}%
\definecolor{currentstroke}{rgb}{0.000000,0.000000,0.000000}%
\pgfsetstrokecolor{currentstroke}%
\pgfsetdash{}{0pt}%
\pgfpathmoveto{\pgfqpoint{2.237924in}{2.752812in}}%
\pgfpathlineto{\pgfqpoint{2.499217in}{2.903684in}}%
\pgfpathlineto{\pgfqpoint{2.237924in}{2.752812in}}%
\pgfpathclose%
\pgfusepath{fill}%
\end{pgfscope}%
\begin{pgfscope}%
\pgfpathrectangle{\pgfqpoint{0.017500in}{0.017500in}}{\pgfqpoint{6.218282in}{3.580372in}} %
\pgfusepath{clip}%
\pgfsetbuttcap%
\pgfsetroundjoin%
\definecolor{currentfill}{rgb}{0.740957,0.122240,0.175744}%
\pgfsetfillcolor{currentfill}%
\pgfsetlinewidth{0.000000pt}%
\definecolor{currentstroke}{rgb}{0.000000,0.000000,0.000000}%
\pgfsetstrokecolor{currentstroke}%
\pgfsetdash{}{0pt}%
\pgfpathmoveto{\pgfqpoint{2.237924in}{2.752812in}}%
\pgfpathlineto{\pgfqpoint{2.499217in}{2.903684in}}%
\pgfpathlineto{\pgfqpoint{2.237924in}{2.752812in}}%
\pgfpathclose%
\pgfusepath{fill}%
\end{pgfscope}%
\begin{pgfscope}%
\pgfpathrectangle{\pgfqpoint{0.017500in}{0.017500in}}{\pgfqpoint{6.218282in}{3.580372in}} %
\pgfusepath{clip}%
\pgfsetbuttcap%
\pgfsetroundjoin%
\definecolor{currentfill}{rgb}{0.740957,0.122240,0.175744}%
\pgfsetfillcolor{currentfill}%
\pgfsetlinewidth{0.000000pt}%
\definecolor{currentstroke}{rgb}{0.000000,0.000000,0.000000}%
\pgfsetstrokecolor{currentstroke}%
\pgfsetdash{}{0pt}%
\pgfpathmoveto{\pgfqpoint{2.237924in}{2.752812in}}%
\pgfpathlineto{\pgfqpoint{2.499217in}{2.903684in}}%
\pgfpathlineto{\pgfqpoint{2.237924in}{2.752812in}}%
\pgfpathclose%
\pgfusepath{fill}%
\end{pgfscope}%
\begin{pgfscope}%
\pgfpathrectangle{\pgfqpoint{0.017500in}{0.017500in}}{\pgfqpoint{6.218282in}{3.580372in}} %
\pgfusepath{clip}%
\pgfsetbuttcap%
\pgfsetroundjoin%
\definecolor{currentfill}{rgb}{0.740957,0.122240,0.175744}%
\pgfsetfillcolor{currentfill}%
\pgfsetlinewidth{0.000000pt}%
\definecolor{currentstroke}{rgb}{0.000000,0.000000,0.000000}%
\pgfsetstrokecolor{currentstroke}%
\pgfsetdash{}{0pt}%
\pgfpathmoveto{\pgfqpoint{2.237924in}{2.752812in}}%
\pgfpathlineto{\pgfqpoint{2.499217in}{2.903684in}}%
\pgfpathlineto{\pgfqpoint{2.237924in}{2.752812in}}%
\pgfpathclose%
\pgfusepath{fill}%
\end{pgfscope}%
\begin{pgfscope}%
\pgfpathrectangle{\pgfqpoint{0.017500in}{0.017500in}}{\pgfqpoint{6.218282in}{3.580372in}} %
\pgfusepath{clip}%
\pgfsetbuttcap%
\pgfsetroundjoin%
\definecolor{currentfill}{rgb}{0.740957,0.122240,0.175744}%
\pgfsetfillcolor{currentfill}%
\pgfsetlinewidth{0.000000pt}%
\definecolor{currentstroke}{rgb}{0.000000,0.000000,0.000000}%
\pgfsetstrokecolor{currentstroke}%
\pgfsetdash{}{0pt}%
\pgfpathmoveto{\pgfqpoint{2.237924in}{2.752812in}}%
\pgfpathlineto{\pgfqpoint{2.499217in}{2.903684in}}%
\pgfpathlineto{\pgfqpoint{2.237924in}{2.752812in}}%
\pgfpathclose%
\pgfusepath{fill}%
\end{pgfscope}%
\begin{pgfscope}%
\pgfpathrectangle{\pgfqpoint{0.017500in}{0.017500in}}{\pgfqpoint{6.218282in}{3.580372in}} %
\pgfusepath{clip}%
\pgfsetbuttcap%
\pgfsetroundjoin%
\definecolor{currentfill}{rgb}{0.740957,0.122240,0.175744}%
\pgfsetfillcolor{currentfill}%
\pgfsetlinewidth{0.000000pt}%
\definecolor{currentstroke}{rgb}{0.000000,0.000000,0.000000}%
\pgfsetstrokecolor{currentstroke}%
\pgfsetdash{}{0pt}%
\pgfpathmoveto{\pgfqpoint{2.237924in}{2.752812in}}%
\pgfpathlineto{\pgfqpoint{2.499217in}{2.903684in}}%
\pgfpathlineto{\pgfqpoint{2.237924in}{2.752812in}}%
\pgfpathclose%
\pgfusepath{fill}%
\end{pgfscope}%
\begin{pgfscope}%
\pgfpathrectangle{\pgfqpoint{0.017500in}{0.017500in}}{\pgfqpoint{6.218282in}{3.580372in}} %
\pgfusepath{clip}%
\pgfsetbuttcap%
\pgfsetroundjoin%
\definecolor{currentfill}{rgb}{0.740957,0.122240,0.175744}%
\pgfsetfillcolor{currentfill}%
\pgfsetlinewidth{0.000000pt}%
\definecolor{currentstroke}{rgb}{0.000000,0.000000,0.000000}%
\pgfsetstrokecolor{currentstroke}%
\pgfsetdash{}{0pt}%
\pgfpathmoveto{\pgfqpoint{2.237924in}{2.752812in}}%
\pgfpathlineto{\pgfqpoint{2.499217in}{2.903684in}}%
\pgfpathlineto{\pgfqpoint{2.237924in}{2.752812in}}%
\pgfpathclose%
\pgfusepath{fill}%
\end{pgfscope}%
\begin{pgfscope}%
\pgfpathrectangle{\pgfqpoint{0.017500in}{0.017500in}}{\pgfqpoint{6.218282in}{3.580372in}} %
\pgfusepath{clip}%
\pgfsetbuttcap%
\pgfsetroundjoin%
\definecolor{currentfill}{rgb}{0.740957,0.122240,0.175744}%
\pgfsetfillcolor{currentfill}%
\pgfsetlinewidth{0.000000pt}%
\definecolor{currentstroke}{rgb}{0.000000,0.000000,0.000000}%
\pgfsetstrokecolor{currentstroke}%
\pgfsetdash{}{0pt}%
\pgfpathmoveto{\pgfqpoint{2.237924in}{2.752812in}}%
\pgfpathlineto{\pgfqpoint{2.499217in}{2.903684in}}%
\pgfpathlineto{\pgfqpoint{2.237924in}{2.752812in}}%
\pgfpathclose%
\pgfusepath{fill}%
\end{pgfscope}%
\begin{pgfscope}%
\pgfpathrectangle{\pgfqpoint{0.017500in}{0.017500in}}{\pgfqpoint{6.218282in}{3.580372in}} %
\pgfusepath{clip}%
\pgfsetbuttcap%
\pgfsetroundjoin%
\definecolor{currentfill}{rgb}{0.740957,0.122240,0.175744}%
\pgfsetfillcolor{currentfill}%
\pgfsetlinewidth{0.000000pt}%
\definecolor{currentstroke}{rgb}{0.000000,0.000000,0.000000}%
\pgfsetstrokecolor{currentstroke}%
\pgfsetdash{}{0pt}%
\pgfpathmoveto{\pgfqpoint{2.237924in}{2.752812in}}%
\pgfpathlineto{\pgfqpoint{2.499217in}{2.903684in}}%
\pgfpathlineto{\pgfqpoint{2.237924in}{2.752812in}}%
\pgfpathclose%
\pgfusepath{fill}%
\end{pgfscope}%
\begin{pgfscope}%
\pgfpathrectangle{\pgfqpoint{0.017500in}{0.017500in}}{\pgfqpoint{6.218282in}{3.580372in}} %
\pgfusepath{clip}%
\pgfsetbuttcap%
\pgfsetroundjoin%
\definecolor{currentfill}{rgb}{0.740957,0.122240,0.175744}%
\pgfsetfillcolor{currentfill}%
\pgfsetlinewidth{0.000000pt}%
\definecolor{currentstroke}{rgb}{0.000000,0.000000,0.000000}%
\pgfsetstrokecolor{currentstroke}%
\pgfsetdash{}{0pt}%
\pgfpathmoveto{\pgfqpoint{2.237924in}{2.752812in}}%
\pgfpathlineto{\pgfqpoint{2.499217in}{2.903684in}}%
\pgfpathlineto{\pgfqpoint{2.237924in}{2.752812in}}%
\pgfpathclose%
\pgfusepath{fill}%
\end{pgfscope}%
\begin{pgfscope}%
\pgfpathrectangle{\pgfqpoint{0.017500in}{0.017500in}}{\pgfqpoint{6.218282in}{3.580372in}} %
\pgfusepath{clip}%
\pgfsetbuttcap%
\pgfsetroundjoin%
\definecolor{currentfill}{rgb}{0.740957,0.122240,0.175744}%
\pgfsetfillcolor{currentfill}%
\pgfsetlinewidth{0.000000pt}%
\definecolor{currentstroke}{rgb}{0.000000,0.000000,0.000000}%
\pgfsetstrokecolor{currentstroke}%
\pgfsetdash{}{0pt}%
\pgfpathmoveto{\pgfqpoint{2.237924in}{2.752812in}}%
\pgfpathlineto{\pgfqpoint{2.499217in}{2.903684in}}%
\pgfpathlineto{\pgfqpoint{2.237924in}{2.752812in}}%
\pgfpathclose%
\pgfusepath{fill}%
\end{pgfscope}%
\begin{pgfscope}%
\pgfpathrectangle{\pgfqpoint{0.017500in}{0.017500in}}{\pgfqpoint{6.218282in}{3.580372in}} %
\pgfusepath{clip}%
\pgfsetbuttcap%
\pgfsetroundjoin%
\definecolor{currentfill}{rgb}{0.740957,0.122240,0.175744}%
\pgfsetfillcolor{currentfill}%
\pgfsetlinewidth{0.000000pt}%
\definecolor{currentstroke}{rgb}{0.000000,0.000000,0.000000}%
\pgfsetstrokecolor{currentstroke}%
\pgfsetdash{}{0pt}%
\pgfpathmoveto{\pgfqpoint{2.237924in}{2.752812in}}%
\pgfpathlineto{\pgfqpoint{2.499217in}{2.903684in}}%
\pgfpathlineto{\pgfqpoint{2.237924in}{2.752812in}}%
\pgfpathclose%
\pgfusepath{fill}%
\end{pgfscope}%
\begin{pgfscope}%
\pgfpathrectangle{\pgfqpoint{0.017500in}{0.017500in}}{\pgfqpoint{6.218282in}{3.580372in}} %
\pgfusepath{clip}%
\pgfsetbuttcap%
\pgfsetroundjoin%
\definecolor{currentfill}{rgb}{0.740957,0.122240,0.175744}%
\pgfsetfillcolor{currentfill}%
\pgfsetlinewidth{0.000000pt}%
\definecolor{currentstroke}{rgb}{0.000000,0.000000,0.000000}%
\pgfsetstrokecolor{currentstroke}%
\pgfsetdash{}{0pt}%
\pgfpathmoveto{\pgfqpoint{2.237924in}{2.752812in}}%
\pgfpathlineto{\pgfqpoint{2.499217in}{2.903684in}}%
\pgfpathlineto{\pgfqpoint{2.237924in}{2.752812in}}%
\pgfpathclose%
\pgfusepath{fill}%
\end{pgfscope}%
\begin{pgfscope}%
\pgfpathrectangle{\pgfqpoint{0.017500in}{0.017500in}}{\pgfqpoint{6.218282in}{3.580372in}} %
\pgfusepath{clip}%
\pgfsetbuttcap%
\pgfsetroundjoin%
\definecolor{currentfill}{rgb}{0.740957,0.122240,0.175744}%
\pgfsetfillcolor{currentfill}%
\pgfsetlinewidth{0.000000pt}%
\definecolor{currentstroke}{rgb}{0.000000,0.000000,0.000000}%
\pgfsetstrokecolor{currentstroke}%
\pgfsetdash{}{0pt}%
\pgfpathmoveto{\pgfqpoint{2.237924in}{2.752812in}}%
\pgfpathlineto{\pgfqpoint{2.499217in}{2.903684in}}%
\pgfpathlineto{\pgfqpoint{2.237924in}{2.752812in}}%
\pgfpathclose%
\pgfusepath{fill}%
\end{pgfscope}%
\begin{pgfscope}%
\pgfpathrectangle{\pgfqpoint{0.017500in}{0.017500in}}{\pgfqpoint{6.218282in}{3.580372in}} %
\pgfusepath{clip}%
\pgfsetbuttcap%
\pgfsetroundjoin%
\definecolor{currentfill}{rgb}{0.740957,0.122240,0.175744}%
\pgfsetfillcolor{currentfill}%
\pgfsetlinewidth{0.000000pt}%
\definecolor{currentstroke}{rgb}{0.000000,0.000000,0.000000}%
\pgfsetstrokecolor{currentstroke}%
\pgfsetdash{}{0pt}%
\pgfpathmoveto{\pgfqpoint{2.237924in}{2.752812in}}%
\pgfpathlineto{\pgfqpoint{2.499217in}{2.903684in}}%
\pgfpathlineto{\pgfqpoint{2.237924in}{2.752812in}}%
\pgfpathclose%
\pgfusepath{fill}%
\end{pgfscope}%
\begin{pgfscope}%
\pgfpathrectangle{\pgfqpoint{0.017500in}{0.017500in}}{\pgfqpoint{6.218282in}{3.580372in}} %
\pgfusepath{clip}%
\pgfsetbuttcap%
\pgfsetroundjoin%
\definecolor{currentfill}{rgb}{0.740957,0.122240,0.175744}%
\pgfsetfillcolor{currentfill}%
\pgfsetlinewidth{0.000000pt}%
\definecolor{currentstroke}{rgb}{0.000000,0.000000,0.000000}%
\pgfsetstrokecolor{currentstroke}%
\pgfsetdash{}{0pt}%
\pgfpathmoveto{\pgfqpoint{2.237924in}{2.752812in}}%
\pgfpathlineto{\pgfqpoint{2.499217in}{2.903684in}}%
\pgfpathlineto{\pgfqpoint{2.237924in}{2.752812in}}%
\pgfpathclose%
\pgfusepath{fill}%
\end{pgfscope}%
\begin{pgfscope}%
\pgfpathrectangle{\pgfqpoint{0.017500in}{0.017500in}}{\pgfqpoint{6.218282in}{3.580372in}} %
\pgfusepath{clip}%
\pgfsetbuttcap%
\pgfsetroundjoin%
\definecolor{currentfill}{rgb}{0.740957,0.122240,0.175744}%
\pgfsetfillcolor{currentfill}%
\pgfsetlinewidth{0.000000pt}%
\definecolor{currentstroke}{rgb}{0.000000,0.000000,0.000000}%
\pgfsetstrokecolor{currentstroke}%
\pgfsetdash{}{0pt}%
\pgfpathmoveto{\pgfqpoint{2.237924in}{2.752812in}}%
\pgfpathlineto{\pgfqpoint{2.499217in}{2.903684in}}%
\pgfpathlineto{\pgfqpoint{2.237924in}{2.752812in}}%
\pgfpathclose%
\pgfusepath{fill}%
\end{pgfscope}%
\begin{pgfscope}%
\pgfpathrectangle{\pgfqpoint{0.017500in}{0.017500in}}{\pgfqpoint{6.218282in}{3.580372in}} %
\pgfusepath{clip}%
\pgfsetbuttcap%
\pgfsetroundjoin%
\definecolor{currentfill}{rgb}{0.740957,0.122240,0.175744}%
\pgfsetfillcolor{currentfill}%
\pgfsetlinewidth{0.000000pt}%
\definecolor{currentstroke}{rgb}{0.000000,0.000000,0.000000}%
\pgfsetstrokecolor{currentstroke}%
\pgfsetdash{}{0pt}%
\pgfpathmoveto{\pgfqpoint{2.237924in}{2.752812in}}%
\pgfpathlineto{\pgfqpoint{2.499217in}{2.903684in}}%
\pgfpathlineto{\pgfqpoint{2.237924in}{2.752812in}}%
\pgfpathclose%
\pgfusepath{fill}%
\end{pgfscope}%
\begin{pgfscope}%
\pgfpathrectangle{\pgfqpoint{0.017500in}{0.017500in}}{\pgfqpoint{6.218282in}{3.580372in}} %
\pgfusepath{clip}%
\pgfsetbuttcap%
\pgfsetroundjoin%
\definecolor{currentfill}{rgb}{0.740957,0.122240,0.175744}%
\pgfsetfillcolor{currentfill}%
\pgfsetlinewidth{0.000000pt}%
\definecolor{currentstroke}{rgb}{0.000000,0.000000,0.000000}%
\pgfsetstrokecolor{currentstroke}%
\pgfsetdash{}{0pt}%
\pgfpathmoveto{\pgfqpoint{2.237924in}{2.752812in}}%
\pgfpathlineto{\pgfqpoint{2.499217in}{2.903684in}}%
\pgfpathlineto{\pgfqpoint{2.237924in}{2.752812in}}%
\pgfpathclose%
\pgfusepath{fill}%
\end{pgfscope}%
\begin{pgfscope}%
\pgfpathrectangle{\pgfqpoint{0.017500in}{0.017500in}}{\pgfqpoint{6.218282in}{3.580372in}} %
\pgfusepath{clip}%
\pgfsetbuttcap%
\pgfsetroundjoin%
\definecolor{currentfill}{rgb}{0.740957,0.122240,0.175744}%
\pgfsetfillcolor{currentfill}%
\pgfsetlinewidth{0.000000pt}%
\definecolor{currentstroke}{rgb}{0.000000,0.000000,0.000000}%
\pgfsetstrokecolor{currentstroke}%
\pgfsetdash{}{0pt}%
\pgfpathmoveto{\pgfqpoint{2.237924in}{2.752812in}}%
\pgfpathlineto{\pgfqpoint{2.499217in}{2.903684in}}%
\pgfpathlineto{\pgfqpoint{2.237924in}{2.752812in}}%
\pgfpathclose%
\pgfusepath{fill}%
\end{pgfscope}%
\begin{pgfscope}%
\pgfpathrectangle{\pgfqpoint{0.017500in}{0.017500in}}{\pgfqpoint{6.218282in}{3.580372in}} %
\pgfusepath{clip}%
\pgfsetbuttcap%
\pgfsetroundjoin%
\definecolor{currentfill}{rgb}{0.740957,0.122240,0.175744}%
\pgfsetfillcolor{currentfill}%
\pgfsetlinewidth{0.000000pt}%
\definecolor{currentstroke}{rgb}{0.000000,0.000000,0.000000}%
\pgfsetstrokecolor{currentstroke}%
\pgfsetdash{}{0pt}%
\pgfpathmoveto{\pgfqpoint{2.237924in}{2.752812in}}%
\pgfpathlineto{\pgfqpoint{2.499217in}{2.903684in}}%
\pgfpathlineto{\pgfqpoint{2.237924in}{2.752812in}}%
\pgfpathclose%
\pgfusepath{fill}%
\end{pgfscope}%
\begin{pgfscope}%
\pgfpathrectangle{\pgfqpoint{0.017500in}{0.017500in}}{\pgfqpoint{6.218282in}{3.580372in}} %
\pgfusepath{clip}%
\pgfsetbuttcap%
\pgfsetroundjoin%
\definecolor{currentfill}{rgb}{0.740957,0.122240,0.175744}%
\pgfsetfillcolor{currentfill}%
\pgfsetlinewidth{0.000000pt}%
\definecolor{currentstroke}{rgb}{0.000000,0.000000,0.000000}%
\pgfsetstrokecolor{currentstroke}%
\pgfsetdash{}{0pt}%
\pgfpathmoveto{\pgfqpoint{2.237924in}{2.752812in}}%
\pgfpathlineto{\pgfqpoint{2.499217in}{2.903684in}}%
\pgfpathlineto{\pgfqpoint{2.237924in}{2.752812in}}%
\pgfpathclose%
\pgfusepath{fill}%
\end{pgfscope}%
\begin{pgfscope}%
\pgfpathrectangle{\pgfqpoint{0.017500in}{0.017500in}}{\pgfqpoint{6.218282in}{3.580372in}} %
\pgfusepath{clip}%
\pgfsetbuttcap%
\pgfsetroundjoin%
\definecolor{currentfill}{rgb}{0.740957,0.122240,0.175744}%
\pgfsetfillcolor{currentfill}%
\pgfsetlinewidth{0.000000pt}%
\definecolor{currentstroke}{rgb}{0.000000,0.000000,0.000000}%
\pgfsetstrokecolor{currentstroke}%
\pgfsetdash{}{0pt}%
\pgfpathmoveto{\pgfqpoint{2.237924in}{2.752812in}}%
\pgfpathlineto{\pgfqpoint{2.499217in}{2.903684in}}%
\pgfpathlineto{\pgfqpoint{2.237924in}{2.752812in}}%
\pgfpathclose%
\pgfusepath{fill}%
\end{pgfscope}%
\begin{pgfscope}%
\pgfpathrectangle{\pgfqpoint{0.017500in}{0.017500in}}{\pgfqpoint{6.218282in}{3.580372in}} %
\pgfusepath{clip}%
\pgfsetbuttcap%
\pgfsetroundjoin%
\definecolor{currentfill}{rgb}{0.740957,0.122240,0.175744}%
\pgfsetfillcolor{currentfill}%
\pgfsetlinewidth{0.000000pt}%
\definecolor{currentstroke}{rgb}{0.000000,0.000000,0.000000}%
\pgfsetstrokecolor{currentstroke}%
\pgfsetdash{}{0pt}%
\pgfpathmoveto{\pgfqpoint{2.237924in}{2.752812in}}%
\pgfpathlineto{\pgfqpoint{2.499217in}{2.903684in}}%
\pgfpathlineto{\pgfqpoint{2.237924in}{2.752812in}}%
\pgfpathclose%
\pgfusepath{fill}%
\end{pgfscope}%
\begin{pgfscope}%
\pgfpathrectangle{\pgfqpoint{0.017500in}{0.017500in}}{\pgfqpoint{6.218282in}{3.580372in}} %
\pgfusepath{clip}%
\pgfsetbuttcap%
\pgfsetroundjoin%
\definecolor{currentfill}{rgb}{0.740957,0.122240,0.175744}%
\pgfsetfillcolor{currentfill}%
\pgfsetlinewidth{0.000000pt}%
\definecolor{currentstroke}{rgb}{0.000000,0.000000,0.000000}%
\pgfsetstrokecolor{currentstroke}%
\pgfsetdash{}{0pt}%
\pgfpathmoveto{\pgfqpoint{2.237924in}{2.752812in}}%
\pgfpathlineto{\pgfqpoint{2.499217in}{2.903684in}}%
\pgfpathlineto{\pgfqpoint{2.237924in}{2.752812in}}%
\pgfpathclose%
\pgfusepath{fill}%
\end{pgfscope}%
\begin{pgfscope}%
\pgfpathrectangle{\pgfqpoint{0.017500in}{0.017500in}}{\pgfqpoint{6.218282in}{3.580372in}} %
\pgfusepath{clip}%
\pgfsetbuttcap%
\pgfsetroundjoin%
\definecolor{currentfill}{rgb}{0.899543,0.847500,0.817789}%
\pgfsetfillcolor{currentfill}%
\pgfsetlinewidth{0.000000pt}%
\definecolor{currentstroke}{rgb}{0.000000,0.000000,0.000000}%
\pgfsetstrokecolor{currentstroke}%
\pgfsetdash{}{0pt}%
\pgfpathmoveto{\pgfqpoint{3.622205in}{2.891176in}}%
\pgfpathlineto{\pgfqpoint{2.349853in}{0.872712in}}%
\pgfpathlineto{\pgfqpoint{3.622205in}{2.891176in}}%
\pgfpathclose%
\pgfusepath{fill}%
\end{pgfscope}%
\begin{pgfscope}%
\pgfpathrectangle{\pgfqpoint{0.017500in}{0.017500in}}{\pgfqpoint{6.218282in}{3.580372in}} %
\pgfusepath{clip}%
\pgfsetbuttcap%
\pgfsetroundjoin%
\definecolor{currentfill}{rgb}{0.899543,0.847500,0.817789}%
\pgfsetfillcolor{currentfill}%
\pgfsetlinewidth{0.000000pt}%
\definecolor{currentstroke}{rgb}{0.000000,0.000000,0.000000}%
\pgfsetstrokecolor{currentstroke}%
\pgfsetdash{}{0pt}%
\pgfpathmoveto{\pgfqpoint{3.622205in}{2.891176in}}%
\pgfpathlineto{\pgfqpoint{2.349853in}{0.872712in}}%
\pgfpathlineto{\pgfqpoint{3.622205in}{2.891176in}}%
\pgfpathclose%
\pgfusepath{fill}%
\end{pgfscope}%
\begin{pgfscope}%
\pgfpathrectangle{\pgfqpoint{0.017500in}{0.017500in}}{\pgfqpoint{6.218282in}{3.580372in}} %
\pgfusepath{clip}%
\pgfsetbuttcap%
\pgfsetroundjoin%
\definecolor{currentfill}{rgb}{0.899543,0.847500,0.817789}%
\pgfsetfillcolor{currentfill}%
\pgfsetlinewidth{0.000000pt}%
\definecolor{currentstroke}{rgb}{0.000000,0.000000,0.000000}%
\pgfsetstrokecolor{currentstroke}%
\pgfsetdash{}{0pt}%
\pgfpathmoveto{\pgfqpoint{3.622205in}{2.891176in}}%
\pgfpathlineto{\pgfqpoint{2.349853in}{0.872712in}}%
\pgfpathlineto{\pgfqpoint{3.622205in}{2.891176in}}%
\pgfpathclose%
\pgfusepath{fill}%
\end{pgfscope}%
\begin{pgfscope}%
\pgfpathrectangle{\pgfqpoint{0.017500in}{0.017500in}}{\pgfqpoint{6.218282in}{3.580372in}} %
\pgfusepath{clip}%
\pgfsetbuttcap%
\pgfsetroundjoin%
\definecolor{currentfill}{rgb}{0.899543,0.847500,0.817789}%
\pgfsetfillcolor{currentfill}%
\pgfsetlinewidth{0.000000pt}%
\definecolor{currentstroke}{rgb}{0.000000,0.000000,0.000000}%
\pgfsetstrokecolor{currentstroke}%
\pgfsetdash{}{0pt}%
\pgfpathmoveto{\pgfqpoint{3.622205in}{2.891176in}}%
\pgfpathlineto{\pgfqpoint{2.349853in}{0.872712in}}%
\pgfpathlineto{\pgfqpoint{3.622205in}{2.891176in}}%
\pgfpathclose%
\pgfusepath{fill}%
\end{pgfscope}%
\begin{pgfscope}%
\pgfpathrectangle{\pgfqpoint{0.017500in}{0.017500in}}{\pgfqpoint{6.218282in}{3.580372in}} %
\pgfusepath{clip}%
\pgfsetbuttcap%
\pgfsetroundjoin%
\definecolor{currentfill}{rgb}{0.899543,0.847500,0.817789}%
\pgfsetfillcolor{currentfill}%
\pgfsetlinewidth{0.000000pt}%
\definecolor{currentstroke}{rgb}{0.000000,0.000000,0.000000}%
\pgfsetstrokecolor{currentstroke}%
\pgfsetdash{}{0pt}%
\pgfpathmoveto{\pgfqpoint{3.622205in}{2.891176in}}%
\pgfpathlineto{\pgfqpoint{2.349853in}{0.872712in}}%
\pgfpathlineto{\pgfqpoint{3.622205in}{2.891176in}}%
\pgfpathclose%
\pgfusepath{fill}%
\end{pgfscope}%
\begin{pgfscope}%
\pgfpathrectangle{\pgfqpoint{0.017500in}{0.017500in}}{\pgfqpoint{6.218282in}{3.580372in}} %
\pgfusepath{clip}%
\pgfsetbuttcap%
\pgfsetroundjoin%
\definecolor{currentfill}{rgb}{0.899543,0.847500,0.817789}%
\pgfsetfillcolor{currentfill}%
\pgfsetlinewidth{0.000000pt}%
\definecolor{currentstroke}{rgb}{0.000000,0.000000,0.000000}%
\pgfsetstrokecolor{currentstroke}%
\pgfsetdash{}{0pt}%
\pgfpathmoveto{\pgfqpoint{3.622205in}{2.891176in}}%
\pgfpathlineto{\pgfqpoint{2.349853in}{0.872712in}}%
\pgfpathlineto{\pgfqpoint{3.622205in}{2.891176in}}%
\pgfpathclose%
\pgfusepath{fill}%
\end{pgfscope}%
\begin{pgfscope}%
\pgfpathrectangle{\pgfqpoint{0.017500in}{0.017500in}}{\pgfqpoint{6.218282in}{3.580372in}} %
\pgfusepath{clip}%
\pgfsetbuttcap%
\pgfsetroundjoin%
\definecolor{currentfill}{rgb}{0.899543,0.847500,0.817789}%
\pgfsetfillcolor{currentfill}%
\pgfsetlinewidth{0.000000pt}%
\definecolor{currentstroke}{rgb}{0.000000,0.000000,0.000000}%
\pgfsetstrokecolor{currentstroke}%
\pgfsetdash{}{0pt}%
\pgfpathmoveto{\pgfqpoint{3.622205in}{2.891176in}}%
\pgfpathlineto{\pgfqpoint{2.349853in}{0.872712in}}%
\pgfpathlineto{\pgfqpoint{3.622205in}{2.891176in}}%
\pgfpathclose%
\pgfusepath{fill}%
\end{pgfscope}%
\begin{pgfscope}%
\pgfpathrectangle{\pgfqpoint{0.017500in}{0.017500in}}{\pgfqpoint{6.218282in}{3.580372in}} %
\pgfusepath{clip}%
\pgfsetbuttcap%
\pgfsetroundjoin%
\definecolor{currentfill}{rgb}{0.899543,0.847500,0.817789}%
\pgfsetfillcolor{currentfill}%
\pgfsetlinewidth{0.000000pt}%
\definecolor{currentstroke}{rgb}{0.000000,0.000000,0.000000}%
\pgfsetstrokecolor{currentstroke}%
\pgfsetdash{}{0pt}%
\pgfpathmoveto{\pgfqpoint{3.622205in}{2.891176in}}%
\pgfpathlineto{\pgfqpoint{2.349853in}{0.872712in}}%
\pgfpathlineto{\pgfqpoint{3.622205in}{2.891176in}}%
\pgfpathclose%
\pgfusepath{fill}%
\end{pgfscope}%
\begin{pgfscope}%
\pgfpathrectangle{\pgfqpoint{0.017500in}{0.017500in}}{\pgfqpoint{6.218282in}{3.580372in}} %
\pgfusepath{clip}%
\pgfsetbuttcap%
\pgfsetroundjoin%
\definecolor{currentfill}{rgb}{0.899543,0.847500,0.817789}%
\pgfsetfillcolor{currentfill}%
\pgfsetlinewidth{0.000000pt}%
\definecolor{currentstroke}{rgb}{0.000000,0.000000,0.000000}%
\pgfsetstrokecolor{currentstroke}%
\pgfsetdash{}{0pt}%
\pgfpathmoveto{\pgfqpoint{3.622205in}{2.891176in}}%
\pgfpathlineto{\pgfqpoint{2.349853in}{0.872712in}}%
\pgfpathlineto{\pgfqpoint{3.622205in}{2.891176in}}%
\pgfpathclose%
\pgfusepath{fill}%
\end{pgfscope}%
\begin{pgfscope}%
\pgfpathrectangle{\pgfqpoint{0.017500in}{0.017500in}}{\pgfqpoint{6.218282in}{3.580372in}} %
\pgfusepath{clip}%
\pgfsetbuttcap%
\pgfsetroundjoin%
\definecolor{currentfill}{rgb}{0.899543,0.847500,0.817789}%
\pgfsetfillcolor{currentfill}%
\pgfsetlinewidth{0.000000pt}%
\definecolor{currentstroke}{rgb}{0.000000,0.000000,0.000000}%
\pgfsetstrokecolor{currentstroke}%
\pgfsetdash{}{0pt}%
\pgfpathmoveto{\pgfqpoint{3.622205in}{2.891176in}}%
\pgfpathlineto{\pgfqpoint{2.349853in}{0.872712in}}%
\pgfpathlineto{\pgfqpoint{3.622205in}{2.891176in}}%
\pgfpathclose%
\pgfusepath{fill}%
\end{pgfscope}%
\begin{pgfscope}%
\pgfpathrectangle{\pgfqpoint{0.017500in}{0.017500in}}{\pgfqpoint{6.218282in}{3.580372in}} %
\pgfusepath{clip}%
\pgfsetbuttcap%
\pgfsetroundjoin%
\definecolor{currentfill}{rgb}{0.899543,0.847500,0.817789}%
\pgfsetfillcolor{currentfill}%
\pgfsetlinewidth{0.000000pt}%
\definecolor{currentstroke}{rgb}{0.000000,0.000000,0.000000}%
\pgfsetstrokecolor{currentstroke}%
\pgfsetdash{}{0pt}%
\pgfpathmoveto{\pgfqpoint{3.622205in}{2.891176in}}%
\pgfpathlineto{\pgfqpoint{2.349853in}{0.872712in}}%
\pgfpathlineto{\pgfqpoint{3.622205in}{2.891176in}}%
\pgfpathclose%
\pgfusepath{fill}%
\end{pgfscope}%
\begin{pgfscope}%
\pgfpathrectangle{\pgfqpoint{0.017500in}{0.017500in}}{\pgfqpoint{6.218282in}{3.580372in}} %
\pgfusepath{clip}%
\pgfsetbuttcap%
\pgfsetroundjoin%
\definecolor{currentfill}{rgb}{0.899543,0.847500,0.817789}%
\pgfsetfillcolor{currentfill}%
\pgfsetlinewidth{0.000000pt}%
\definecolor{currentstroke}{rgb}{0.000000,0.000000,0.000000}%
\pgfsetstrokecolor{currentstroke}%
\pgfsetdash{}{0pt}%
\pgfpathmoveto{\pgfqpoint{3.622205in}{2.891176in}}%
\pgfpathlineto{\pgfqpoint{2.349853in}{0.872712in}}%
\pgfpathlineto{\pgfqpoint{3.622205in}{2.891176in}}%
\pgfpathclose%
\pgfusepath{fill}%
\end{pgfscope}%
\begin{pgfscope}%
\pgfpathrectangle{\pgfqpoint{0.017500in}{0.017500in}}{\pgfqpoint{6.218282in}{3.580372in}} %
\pgfusepath{clip}%
\pgfsetbuttcap%
\pgfsetroundjoin%
\definecolor{currentfill}{rgb}{0.899543,0.847500,0.817789}%
\pgfsetfillcolor{currentfill}%
\pgfsetlinewidth{0.000000pt}%
\definecolor{currentstroke}{rgb}{0.000000,0.000000,0.000000}%
\pgfsetstrokecolor{currentstroke}%
\pgfsetdash{}{0pt}%
\pgfpathmoveto{\pgfqpoint{3.622205in}{2.891176in}}%
\pgfpathlineto{\pgfqpoint{2.349853in}{0.872712in}}%
\pgfpathlineto{\pgfqpoint{3.622205in}{2.891176in}}%
\pgfpathclose%
\pgfusepath{fill}%
\end{pgfscope}%
\begin{pgfscope}%
\pgfpathrectangle{\pgfqpoint{0.017500in}{0.017500in}}{\pgfqpoint{6.218282in}{3.580372in}} %
\pgfusepath{clip}%
\pgfsetbuttcap%
\pgfsetroundjoin%
\definecolor{currentfill}{rgb}{0.899543,0.847500,0.817789}%
\pgfsetfillcolor{currentfill}%
\pgfsetlinewidth{0.000000pt}%
\definecolor{currentstroke}{rgb}{0.000000,0.000000,0.000000}%
\pgfsetstrokecolor{currentstroke}%
\pgfsetdash{}{0pt}%
\pgfpathmoveto{\pgfqpoint{3.622205in}{2.891176in}}%
\pgfpathlineto{\pgfqpoint{2.349853in}{0.872712in}}%
\pgfpathlineto{\pgfqpoint{3.622205in}{2.891176in}}%
\pgfpathclose%
\pgfusepath{fill}%
\end{pgfscope}%
\begin{pgfscope}%
\pgfpathrectangle{\pgfqpoint{0.017500in}{0.017500in}}{\pgfqpoint{6.218282in}{3.580372in}} %
\pgfusepath{clip}%
\pgfsetbuttcap%
\pgfsetroundjoin%
\definecolor{currentfill}{rgb}{0.899543,0.847500,0.817789}%
\pgfsetfillcolor{currentfill}%
\pgfsetlinewidth{0.000000pt}%
\definecolor{currentstroke}{rgb}{0.000000,0.000000,0.000000}%
\pgfsetstrokecolor{currentstroke}%
\pgfsetdash{}{0pt}%
\pgfpathmoveto{\pgfqpoint{3.622205in}{2.891176in}}%
\pgfpathlineto{\pgfqpoint{2.349853in}{0.872712in}}%
\pgfpathlineto{\pgfqpoint{3.622205in}{2.891176in}}%
\pgfpathclose%
\pgfusepath{fill}%
\end{pgfscope}%
\begin{pgfscope}%
\pgfpathrectangle{\pgfqpoint{0.017500in}{0.017500in}}{\pgfqpoint{6.218282in}{3.580372in}} %
\pgfusepath{clip}%
\pgfsetbuttcap%
\pgfsetroundjoin%
\definecolor{currentfill}{rgb}{0.899543,0.847500,0.817789}%
\pgfsetfillcolor{currentfill}%
\pgfsetlinewidth{0.000000pt}%
\definecolor{currentstroke}{rgb}{0.000000,0.000000,0.000000}%
\pgfsetstrokecolor{currentstroke}%
\pgfsetdash{}{0pt}%
\pgfpathmoveto{\pgfqpoint{3.622205in}{2.891176in}}%
\pgfpathlineto{\pgfqpoint{2.349853in}{0.872712in}}%
\pgfpathlineto{\pgfqpoint{3.622205in}{2.891176in}}%
\pgfpathclose%
\pgfusepath{fill}%
\end{pgfscope}%
\begin{pgfscope}%
\pgfpathrectangle{\pgfqpoint{0.017500in}{0.017500in}}{\pgfqpoint{6.218282in}{3.580372in}} %
\pgfusepath{clip}%
\pgfsetbuttcap%
\pgfsetroundjoin%
\definecolor{currentfill}{rgb}{0.899543,0.847500,0.817789}%
\pgfsetfillcolor{currentfill}%
\pgfsetlinewidth{0.000000pt}%
\definecolor{currentstroke}{rgb}{0.000000,0.000000,0.000000}%
\pgfsetstrokecolor{currentstroke}%
\pgfsetdash{}{0pt}%
\pgfpathmoveto{\pgfqpoint{3.622205in}{2.891176in}}%
\pgfpathlineto{\pgfqpoint{2.349853in}{0.872712in}}%
\pgfpathlineto{\pgfqpoint{3.622205in}{2.891176in}}%
\pgfpathclose%
\pgfusepath{fill}%
\end{pgfscope}%
\begin{pgfscope}%
\pgfpathrectangle{\pgfqpoint{0.017500in}{0.017500in}}{\pgfqpoint{6.218282in}{3.580372in}} %
\pgfusepath{clip}%
\pgfsetbuttcap%
\pgfsetroundjoin%
\definecolor{currentfill}{rgb}{0.899543,0.847500,0.817789}%
\pgfsetfillcolor{currentfill}%
\pgfsetlinewidth{0.000000pt}%
\definecolor{currentstroke}{rgb}{0.000000,0.000000,0.000000}%
\pgfsetstrokecolor{currentstroke}%
\pgfsetdash{}{0pt}%
\pgfpathmoveto{\pgfqpoint{3.622205in}{2.891176in}}%
\pgfpathlineto{\pgfqpoint{2.349853in}{0.872712in}}%
\pgfpathlineto{\pgfqpoint{3.622205in}{2.891176in}}%
\pgfpathclose%
\pgfusepath{fill}%
\end{pgfscope}%
\begin{pgfscope}%
\pgfpathrectangle{\pgfqpoint{0.017500in}{0.017500in}}{\pgfqpoint{6.218282in}{3.580372in}} %
\pgfusepath{clip}%
\pgfsetbuttcap%
\pgfsetroundjoin%
\definecolor{currentfill}{rgb}{0.899543,0.847500,0.817789}%
\pgfsetfillcolor{currentfill}%
\pgfsetlinewidth{0.000000pt}%
\definecolor{currentstroke}{rgb}{0.000000,0.000000,0.000000}%
\pgfsetstrokecolor{currentstroke}%
\pgfsetdash{}{0pt}%
\pgfpathmoveto{\pgfqpoint{3.622205in}{2.891176in}}%
\pgfpathlineto{\pgfqpoint{2.349853in}{0.872712in}}%
\pgfpathlineto{\pgfqpoint{3.622205in}{2.891176in}}%
\pgfpathclose%
\pgfusepath{fill}%
\end{pgfscope}%
\begin{pgfscope}%
\pgfpathrectangle{\pgfqpoint{0.017500in}{0.017500in}}{\pgfqpoint{6.218282in}{3.580372in}} %
\pgfusepath{clip}%
\pgfsetbuttcap%
\pgfsetroundjoin%
\definecolor{currentfill}{rgb}{0.899543,0.847500,0.817789}%
\pgfsetfillcolor{currentfill}%
\pgfsetlinewidth{0.000000pt}%
\definecolor{currentstroke}{rgb}{0.000000,0.000000,0.000000}%
\pgfsetstrokecolor{currentstroke}%
\pgfsetdash{}{0pt}%
\pgfpathmoveto{\pgfqpoint{3.622205in}{2.891176in}}%
\pgfpathlineto{\pgfqpoint{2.349853in}{0.872712in}}%
\pgfpathlineto{\pgfqpoint{3.622205in}{2.891176in}}%
\pgfpathclose%
\pgfusepath{fill}%
\end{pgfscope}%
\begin{pgfscope}%
\pgfpathrectangle{\pgfqpoint{0.017500in}{0.017500in}}{\pgfqpoint{6.218282in}{3.580372in}} %
\pgfusepath{clip}%
\pgfsetbuttcap%
\pgfsetroundjoin%
\definecolor{currentfill}{rgb}{0.899543,0.847500,0.817789}%
\pgfsetfillcolor{currentfill}%
\pgfsetlinewidth{0.000000pt}%
\definecolor{currentstroke}{rgb}{0.000000,0.000000,0.000000}%
\pgfsetstrokecolor{currentstroke}%
\pgfsetdash{}{0pt}%
\pgfpathmoveto{\pgfqpoint{3.622205in}{2.891176in}}%
\pgfpathlineto{\pgfqpoint{2.349853in}{0.872712in}}%
\pgfpathlineto{\pgfqpoint{3.622205in}{2.891176in}}%
\pgfpathclose%
\pgfusepath{fill}%
\end{pgfscope}%
\begin{pgfscope}%
\pgfpathrectangle{\pgfqpoint{0.017500in}{0.017500in}}{\pgfqpoint{6.218282in}{3.580372in}} %
\pgfusepath{clip}%
\pgfsetbuttcap%
\pgfsetroundjoin%
\definecolor{currentfill}{rgb}{0.899543,0.847500,0.817789}%
\pgfsetfillcolor{currentfill}%
\pgfsetlinewidth{0.000000pt}%
\definecolor{currentstroke}{rgb}{0.000000,0.000000,0.000000}%
\pgfsetstrokecolor{currentstroke}%
\pgfsetdash{}{0pt}%
\pgfpathmoveto{\pgfqpoint{3.622205in}{2.891176in}}%
\pgfpathlineto{\pgfqpoint{2.349853in}{0.872712in}}%
\pgfpathlineto{\pgfqpoint{3.622205in}{2.891176in}}%
\pgfpathclose%
\pgfusepath{fill}%
\end{pgfscope}%
\begin{pgfscope}%
\pgfpathrectangle{\pgfqpoint{0.017500in}{0.017500in}}{\pgfqpoint{6.218282in}{3.580372in}} %
\pgfusepath{clip}%
\pgfsetbuttcap%
\pgfsetroundjoin%
\definecolor{currentfill}{rgb}{0.899543,0.847500,0.817789}%
\pgfsetfillcolor{currentfill}%
\pgfsetlinewidth{0.000000pt}%
\definecolor{currentstroke}{rgb}{0.000000,0.000000,0.000000}%
\pgfsetstrokecolor{currentstroke}%
\pgfsetdash{}{0pt}%
\pgfpathmoveto{\pgfqpoint{3.622205in}{2.891176in}}%
\pgfpathlineto{\pgfqpoint{2.349853in}{0.872712in}}%
\pgfpathlineto{\pgfqpoint{3.622205in}{2.891176in}}%
\pgfpathclose%
\pgfusepath{fill}%
\end{pgfscope}%
\begin{pgfscope}%
\pgfpathrectangle{\pgfqpoint{0.017500in}{0.017500in}}{\pgfqpoint{6.218282in}{3.580372in}} %
\pgfusepath{clip}%
\pgfsetbuttcap%
\pgfsetroundjoin%
\definecolor{currentfill}{rgb}{0.899543,0.847500,0.817789}%
\pgfsetfillcolor{currentfill}%
\pgfsetlinewidth{0.000000pt}%
\definecolor{currentstroke}{rgb}{0.000000,0.000000,0.000000}%
\pgfsetstrokecolor{currentstroke}%
\pgfsetdash{}{0pt}%
\pgfpathmoveto{\pgfqpoint{3.622205in}{2.891176in}}%
\pgfpathlineto{\pgfqpoint{2.349853in}{0.872712in}}%
\pgfpathlineto{\pgfqpoint{3.622205in}{2.891176in}}%
\pgfpathclose%
\pgfusepath{fill}%
\end{pgfscope}%
\begin{pgfscope}%
\pgfpathrectangle{\pgfqpoint{0.017500in}{0.017500in}}{\pgfqpoint{6.218282in}{3.580372in}} %
\pgfusepath{clip}%
\pgfsetbuttcap%
\pgfsetroundjoin%
\definecolor{currentfill}{rgb}{0.899543,0.847500,0.817789}%
\pgfsetfillcolor{currentfill}%
\pgfsetlinewidth{0.000000pt}%
\definecolor{currentstroke}{rgb}{0.000000,0.000000,0.000000}%
\pgfsetstrokecolor{currentstroke}%
\pgfsetdash{}{0pt}%
\pgfpathmoveto{\pgfqpoint{3.622205in}{2.891176in}}%
\pgfpathlineto{\pgfqpoint{2.349853in}{0.872712in}}%
\pgfpathlineto{\pgfqpoint{3.622205in}{2.891176in}}%
\pgfpathclose%
\pgfusepath{fill}%
\end{pgfscope}%
\begin{pgfscope}%
\pgfpathrectangle{\pgfqpoint{0.017500in}{0.017500in}}{\pgfqpoint{6.218282in}{3.580372in}} %
\pgfusepath{clip}%
\pgfsetbuttcap%
\pgfsetroundjoin%
\definecolor{currentfill}{rgb}{0.746838,0.140021,0.179996}%
\pgfsetfillcolor{currentfill}%
\pgfsetlinewidth{0.000000pt}%
\definecolor{currentstroke}{rgb}{0.000000,0.000000,0.000000}%
\pgfsetstrokecolor{currentstroke}%
\pgfsetdash{}{0pt}%
\pgfpathmoveto{\pgfqpoint{2.237924in}{2.752812in}}%
\pgfpathlineto{\pgfqpoint{2.674254in}{2.688576in}}%
\pgfpathlineto{\pgfqpoint{2.932968in}{2.820668in}}%
\pgfpathlineto{\pgfqpoint{2.499217in}{2.903684in}}%
\pgfpathlineto{\pgfqpoint{2.237924in}{2.752812in}}%
\pgfpathclose%
\pgfusepath{fill}%
\end{pgfscope}%
\begin{pgfscope}%
\pgfpathrectangle{\pgfqpoint{0.017500in}{0.017500in}}{\pgfqpoint{6.218282in}{3.580372in}} %
\pgfusepath{clip}%
\pgfsetbuttcap%
\pgfsetroundjoin%
\definecolor{currentfill}{rgb}{0.746838,0.140021,0.179996}%
\pgfsetfillcolor{currentfill}%
\pgfsetlinewidth{0.000000pt}%
\definecolor{currentstroke}{rgb}{0.000000,0.000000,0.000000}%
\pgfsetstrokecolor{currentstroke}%
\pgfsetdash{}{0pt}%
\pgfpathmoveto{\pgfqpoint{2.237924in}{2.752812in}}%
\pgfpathlineto{\pgfqpoint{2.674254in}{2.688576in}}%
\pgfpathlineto{\pgfqpoint{2.932968in}{2.820668in}}%
\pgfpathlineto{\pgfqpoint{2.499217in}{2.903684in}}%
\pgfpathlineto{\pgfqpoint{2.237924in}{2.752812in}}%
\pgfpathclose%
\pgfusepath{fill}%
\end{pgfscope}%
\begin{pgfscope}%
\pgfpathrectangle{\pgfqpoint{0.017500in}{0.017500in}}{\pgfqpoint{6.218282in}{3.580372in}} %
\pgfusepath{clip}%
\pgfsetbuttcap%
\pgfsetroundjoin%
\definecolor{currentfill}{rgb}{0.746838,0.140021,0.179996}%
\pgfsetfillcolor{currentfill}%
\pgfsetlinewidth{0.000000pt}%
\definecolor{currentstroke}{rgb}{0.000000,0.000000,0.000000}%
\pgfsetstrokecolor{currentstroke}%
\pgfsetdash{}{0pt}%
\pgfpathmoveto{\pgfqpoint{2.237924in}{2.752812in}}%
\pgfpathlineto{\pgfqpoint{2.674254in}{2.688576in}}%
\pgfpathlineto{\pgfqpoint{2.932968in}{2.820668in}}%
\pgfpathlineto{\pgfqpoint{2.499217in}{2.903684in}}%
\pgfpathlineto{\pgfqpoint{2.237924in}{2.752812in}}%
\pgfpathclose%
\pgfusepath{fill}%
\end{pgfscope}%
\begin{pgfscope}%
\pgfpathrectangle{\pgfqpoint{0.017500in}{0.017500in}}{\pgfqpoint{6.218282in}{3.580372in}} %
\pgfusepath{clip}%
\pgfsetbuttcap%
\pgfsetroundjoin%
\definecolor{currentfill}{rgb}{0.746838,0.140021,0.179996}%
\pgfsetfillcolor{currentfill}%
\pgfsetlinewidth{0.000000pt}%
\definecolor{currentstroke}{rgb}{0.000000,0.000000,0.000000}%
\pgfsetstrokecolor{currentstroke}%
\pgfsetdash{}{0pt}%
\pgfpathmoveto{\pgfqpoint{2.237924in}{2.752812in}}%
\pgfpathlineto{\pgfqpoint{2.674254in}{2.688576in}}%
\pgfpathlineto{\pgfqpoint{2.932968in}{2.820668in}}%
\pgfpathlineto{\pgfqpoint{2.499217in}{2.903684in}}%
\pgfpathlineto{\pgfqpoint{2.237924in}{2.752812in}}%
\pgfpathclose%
\pgfusepath{fill}%
\end{pgfscope}%
\begin{pgfscope}%
\pgfpathrectangle{\pgfqpoint{0.017500in}{0.017500in}}{\pgfqpoint{6.218282in}{3.580372in}} %
\pgfusepath{clip}%
\pgfsetbuttcap%
\pgfsetroundjoin%
\definecolor{currentfill}{rgb}{0.746838,0.140021,0.179996}%
\pgfsetfillcolor{currentfill}%
\pgfsetlinewidth{0.000000pt}%
\definecolor{currentstroke}{rgb}{0.000000,0.000000,0.000000}%
\pgfsetstrokecolor{currentstroke}%
\pgfsetdash{}{0pt}%
\pgfpathmoveto{\pgfqpoint{2.237924in}{2.752812in}}%
\pgfpathlineto{\pgfqpoint{2.674254in}{2.688576in}}%
\pgfpathlineto{\pgfqpoint{2.932968in}{2.820668in}}%
\pgfpathlineto{\pgfqpoint{2.499217in}{2.903684in}}%
\pgfpathlineto{\pgfqpoint{2.237924in}{2.752812in}}%
\pgfpathclose%
\pgfusepath{fill}%
\end{pgfscope}%
\begin{pgfscope}%
\pgfpathrectangle{\pgfqpoint{0.017500in}{0.017500in}}{\pgfqpoint{6.218282in}{3.580372in}} %
\pgfusepath{clip}%
\pgfsetbuttcap%
\pgfsetroundjoin%
\definecolor{currentfill}{rgb}{0.746838,0.140021,0.179996}%
\pgfsetfillcolor{currentfill}%
\pgfsetlinewidth{0.000000pt}%
\definecolor{currentstroke}{rgb}{0.000000,0.000000,0.000000}%
\pgfsetstrokecolor{currentstroke}%
\pgfsetdash{}{0pt}%
\pgfpathmoveto{\pgfqpoint{2.237924in}{2.752812in}}%
\pgfpathlineto{\pgfqpoint{2.674254in}{2.688576in}}%
\pgfpathlineto{\pgfqpoint{2.932968in}{2.820668in}}%
\pgfpathlineto{\pgfqpoint{2.499217in}{2.903684in}}%
\pgfpathlineto{\pgfqpoint{2.237924in}{2.752812in}}%
\pgfpathclose%
\pgfusepath{fill}%
\end{pgfscope}%
\begin{pgfscope}%
\pgfpathrectangle{\pgfqpoint{0.017500in}{0.017500in}}{\pgfqpoint{6.218282in}{3.580372in}} %
\pgfusepath{clip}%
\pgfsetbuttcap%
\pgfsetroundjoin%
\definecolor{currentfill}{rgb}{0.717435,0.051118,0.158737}%
\pgfsetfillcolor{currentfill}%
\pgfsetlinewidth{0.000000pt}%
\definecolor{currentstroke}{rgb}{0.000000,0.000000,0.000000}%
\pgfsetstrokecolor{currentstroke}%
\pgfsetdash{}{0pt}%
\pgfpathmoveto{\pgfqpoint{3.372093in}{2.761754in}}%
\pgfpathlineto{\pgfqpoint{3.622205in}{2.891176in}}%
\pgfpathlineto{\pgfqpoint{3.372093in}{2.761754in}}%
\pgfpathclose%
\pgfusepath{fill}%
\end{pgfscope}%
\begin{pgfscope}%
\pgfpathrectangle{\pgfqpoint{0.017500in}{0.017500in}}{\pgfqpoint{6.218282in}{3.580372in}} %
\pgfusepath{clip}%
\pgfsetbuttcap%
\pgfsetroundjoin%
\definecolor{currentfill}{rgb}{0.717435,0.051118,0.158737}%
\pgfsetfillcolor{currentfill}%
\pgfsetlinewidth{0.000000pt}%
\definecolor{currentstroke}{rgb}{0.000000,0.000000,0.000000}%
\pgfsetstrokecolor{currentstroke}%
\pgfsetdash{}{0pt}%
\pgfpathmoveto{\pgfqpoint{3.372093in}{2.761754in}}%
\pgfpathlineto{\pgfqpoint{3.622205in}{2.891176in}}%
\pgfpathlineto{\pgfqpoint{3.372093in}{2.761754in}}%
\pgfpathclose%
\pgfusepath{fill}%
\end{pgfscope}%
\begin{pgfscope}%
\pgfpathrectangle{\pgfqpoint{0.017500in}{0.017500in}}{\pgfqpoint{6.218282in}{3.580372in}} %
\pgfusepath{clip}%
\pgfsetbuttcap%
\pgfsetroundjoin%
\definecolor{currentfill}{rgb}{0.717435,0.051118,0.158737}%
\pgfsetfillcolor{currentfill}%
\pgfsetlinewidth{0.000000pt}%
\definecolor{currentstroke}{rgb}{0.000000,0.000000,0.000000}%
\pgfsetstrokecolor{currentstroke}%
\pgfsetdash{}{0pt}%
\pgfpathmoveto{\pgfqpoint{3.372093in}{2.761754in}}%
\pgfpathlineto{\pgfqpoint{3.622205in}{2.891176in}}%
\pgfpathlineto{\pgfqpoint{3.372093in}{2.761754in}}%
\pgfpathclose%
\pgfusepath{fill}%
\end{pgfscope}%
\begin{pgfscope}%
\pgfpathrectangle{\pgfqpoint{0.017500in}{0.017500in}}{\pgfqpoint{6.218282in}{3.580372in}} %
\pgfusepath{clip}%
\pgfsetbuttcap%
\pgfsetroundjoin%
\definecolor{currentfill}{rgb}{0.717435,0.051118,0.158737}%
\pgfsetfillcolor{currentfill}%
\pgfsetlinewidth{0.000000pt}%
\definecolor{currentstroke}{rgb}{0.000000,0.000000,0.000000}%
\pgfsetstrokecolor{currentstroke}%
\pgfsetdash{}{0pt}%
\pgfpathmoveto{\pgfqpoint{3.372093in}{2.761754in}}%
\pgfpathlineto{\pgfqpoint{3.622205in}{2.891176in}}%
\pgfpathlineto{\pgfqpoint{3.372093in}{2.761754in}}%
\pgfpathclose%
\pgfusepath{fill}%
\end{pgfscope}%
\begin{pgfscope}%
\pgfpathrectangle{\pgfqpoint{0.017500in}{0.017500in}}{\pgfqpoint{6.218282in}{3.580372in}} %
\pgfusepath{clip}%
\pgfsetbuttcap%
\pgfsetroundjoin%
\definecolor{currentfill}{rgb}{0.717435,0.051118,0.158737}%
\pgfsetfillcolor{currentfill}%
\pgfsetlinewidth{0.000000pt}%
\definecolor{currentstroke}{rgb}{0.000000,0.000000,0.000000}%
\pgfsetstrokecolor{currentstroke}%
\pgfsetdash{}{0pt}%
\pgfpathmoveto{\pgfqpoint{3.372093in}{2.761754in}}%
\pgfpathlineto{\pgfqpoint{3.622205in}{2.891176in}}%
\pgfpathlineto{\pgfqpoint{3.372093in}{2.761754in}}%
\pgfpathclose%
\pgfusepath{fill}%
\end{pgfscope}%
\begin{pgfscope}%
\pgfpathrectangle{\pgfqpoint{0.017500in}{0.017500in}}{\pgfqpoint{6.218282in}{3.580372in}} %
\pgfusepath{clip}%
\pgfsetbuttcap%
\pgfsetroundjoin%
\definecolor{currentfill}{rgb}{0.717435,0.051118,0.158737}%
\pgfsetfillcolor{currentfill}%
\pgfsetlinewidth{0.000000pt}%
\definecolor{currentstroke}{rgb}{0.000000,0.000000,0.000000}%
\pgfsetstrokecolor{currentstroke}%
\pgfsetdash{}{0pt}%
\pgfpathmoveto{\pgfqpoint{3.372093in}{2.761754in}}%
\pgfpathlineto{\pgfqpoint{3.622205in}{2.891176in}}%
\pgfpathlineto{\pgfqpoint{3.372093in}{2.761754in}}%
\pgfpathclose%
\pgfusepath{fill}%
\end{pgfscope}%
\begin{pgfscope}%
\pgfpathrectangle{\pgfqpoint{0.017500in}{0.017500in}}{\pgfqpoint{6.218282in}{3.580372in}} %
\pgfusepath{clip}%
\pgfsetbuttcap%
\pgfsetroundjoin%
\definecolor{currentfill}{rgb}{0.717435,0.051118,0.158737}%
\pgfsetfillcolor{currentfill}%
\pgfsetlinewidth{0.000000pt}%
\definecolor{currentstroke}{rgb}{0.000000,0.000000,0.000000}%
\pgfsetstrokecolor{currentstroke}%
\pgfsetdash{}{0pt}%
\pgfpathmoveto{\pgfqpoint{3.372093in}{2.761754in}}%
\pgfpathlineto{\pgfqpoint{3.622205in}{2.891176in}}%
\pgfpathlineto{\pgfqpoint{3.372093in}{2.761754in}}%
\pgfpathclose%
\pgfusepath{fill}%
\end{pgfscope}%
\begin{pgfscope}%
\pgfpathrectangle{\pgfqpoint{0.017500in}{0.017500in}}{\pgfqpoint{6.218282in}{3.580372in}} %
\pgfusepath{clip}%
\pgfsetbuttcap%
\pgfsetroundjoin%
\definecolor{currentfill}{rgb}{0.717435,0.051118,0.158737}%
\pgfsetfillcolor{currentfill}%
\pgfsetlinewidth{0.000000pt}%
\definecolor{currentstroke}{rgb}{0.000000,0.000000,0.000000}%
\pgfsetstrokecolor{currentstroke}%
\pgfsetdash{}{0pt}%
\pgfpathmoveto{\pgfqpoint{3.372093in}{2.761754in}}%
\pgfpathlineto{\pgfqpoint{3.622205in}{2.891176in}}%
\pgfpathlineto{\pgfqpoint{3.372093in}{2.761754in}}%
\pgfpathclose%
\pgfusepath{fill}%
\end{pgfscope}%
\begin{pgfscope}%
\pgfpathrectangle{\pgfqpoint{0.017500in}{0.017500in}}{\pgfqpoint{6.218282in}{3.580372in}} %
\pgfusepath{clip}%
\pgfsetbuttcap%
\pgfsetroundjoin%
\definecolor{currentfill}{rgb}{0.717435,0.051118,0.158737}%
\pgfsetfillcolor{currentfill}%
\pgfsetlinewidth{0.000000pt}%
\definecolor{currentstroke}{rgb}{0.000000,0.000000,0.000000}%
\pgfsetstrokecolor{currentstroke}%
\pgfsetdash{}{0pt}%
\pgfpathmoveto{\pgfqpoint{3.372093in}{2.761754in}}%
\pgfpathlineto{\pgfqpoint{3.622205in}{2.891176in}}%
\pgfpathlineto{\pgfqpoint{3.372093in}{2.761754in}}%
\pgfpathclose%
\pgfusepath{fill}%
\end{pgfscope}%
\begin{pgfscope}%
\pgfpathrectangle{\pgfqpoint{0.017500in}{0.017500in}}{\pgfqpoint{6.218282in}{3.580372in}} %
\pgfusepath{clip}%
\pgfsetbuttcap%
\pgfsetroundjoin%
\definecolor{currentfill}{rgb}{0.717435,0.051118,0.158737}%
\pgfsetfillcolor{currentfill}%
\pgfsetlinewidth{0.000000pt}%
\definecolor{currentstroke}{rgb}{0.000000,0.000000,0.000000}%
\pgfsetstrokecolor{currentstroke}%
\pgfsetdash{}{0pt}%
\pgfpathmoveto{\pgfqpoint{3.372093in}{2.761754in}}%
\pgfpathlineto{\pgfqpoint{3.622205in}{2.891176in}}%
\pgfpathlineto{\pgfqpoint{3.372093in}{2.761754in}}%
\pgfpathclose%
\pgfusepath{fill}%
\end{pgfscope}%
\begin{pgfscope}%
\pgfpathrectangle{\pgfqpoint{0.017500in}{0.017500in}}{\pgfqpoint{6.218282in}{3.580372in}} %
\pgfusepath{clip}%
\pgfsetbuttcap%
\pgfsetroundjoin%
\definecolor{currentfill}{rgb}{0.717435,0.051118,0.158737}%
\pgfsetfillcolor{currentfill}%
\pgfsetlinewidth{0.000000pt}%
\definecolor{currentstroke}{rgb}{0.000000,0.000000,0.000000}%
\pgfsetstrokecolor{currentstroke}%
\pgfsetdash{}{0pt}%
\pgfpathmoveto{\pgfqpoint{3.372093in}{2.761754in}}%
\pgfpathlineto{\pgfqpoint{3.622205in}{2.891176in}}%
\pgfpathlineto{\pgfqpoint{3.372093in}{2.761754in}}%
\pgfpathclose%
\pgfusepath{fill}%
\end{pgfscope}%
\begin{pgfscope}%
\pgfpathrectangle{\pgfqpoint{0.017500in}{0.017500in}}{\pgfqpoint{6.218282in}{3.580372in}} %
\pgfusepath{clip}%
\pgfsetbuttcap%
\pgfsetroundjoin%
\definecolor{currentfill}{rgb}{0.717435,0.051118,0.158737}%
\pgfsetfillcolor{currentfill}%
\pgfsetlinewidth{0.000000pt}%
\definecolor{currentstroke}{rgb}{0.000000,0.000000,0.000000}%
\pgfsetstrokecolor{currentstroke}%
\pgfsetdash{}{0pt}%
\pgfpathmoveto{\pgfqpoint{3.372093in}{2.761754in}}%
\pgfpathlineto{\pgfqpoint{3.622205in}{2.891176in}}%
\pgfpathlineto{\pgfqpoint{3.372093in}{2.761754in}}%
\pgfpathclose%
\pgfusepath{fill}%
\end{pgfscope}%
\begin{pgfscope}%
\pgfpathrectangle{\pgfqpoint{0.017500in}{0.017500in}}{\pgfqpoint{6.218282in}{3.580372in}} %
\pgfusepath{clip}%
\pgfsetbuttcap%
\pgfsetroundjoin%
\definecolor{currentfill}{rgb}{0.717435,0.051118,0.158737}%
\pgfsetfillcolor{currentfill}%
\pgfsetlinewidth{0.000000pt}%
\definecolor{currentstroke}{rgb}{0.000000,0.000000,0.000000}%
\pgfsetstrokecolor{currentstroke}%
\pgfsetdash{}{0pt}%
\pgfpathmoveto{\pgfqpoint{3.372093in}{2.761754in}}%
\pgfpathlineto{\pgfqpoint{3.622205in}{2.891176in}}%
\pgfpathlineto{\pgfqpoint{3.372093in}{2.761754in}}%
\pgfpathclose%
\pgfusepath{fill}%
\end{pgfscope}%
\begin{pgfscope}%
\pgfpathrectangle{\pgfqpoint{0.017500in}{0.017500in}}{\pgfqpoint{6.218282in}{3.580372in}} %
\pgfusepath{clip}%
\pgfsetbuttcap%
\pgfsetroundjoin%
\definecolor{currentfill}{rgb}{0.717435,0.051118,0.158737}%
\pgfsetfillcolor{currentfill}%
\pgfsetlinewidth{0.000000pt}%
\definecolor{currentstroke}{rgb}{0.000000,0.000000,0.000000}%
\pgfsetstrokecolor{currentstroke}%
\pgfsetdash{}{0pt}%
\pgfpathmoveto{\pgfqpoint{3.372093in}{2.761754in}}%
\pgfpathlineto{\pgfqpoint{3.622205in}{2.891176in}}%
\pgfpathlineto{\pgfqpoint{3.372093in}{2.761754in}}%
\pgfpathclose%
\pgfusepath{fill}%
\end{pgfscope}%
\begin{pgfscope}%
\pgfpathrectangle{\pgfqpoint{0.017500in}{0.017500in}}{\pgfqpoint{6.218282in}{3.580372in}} %
\pgfusepath{clip}%
\pgfsetbuttcap%
\pgfsetroundjoin%
\definecolor{currentfill}{rgb}{0.717435,0.051118,0.158737}%
\pgfsetfillcolor{currentfill}%
\pgfsetlinewidth{0.000000pt}%
\definecolor{currentstroke}{rgb}{0.000000,0.000000,0.000000}%
\pgfsetstrokecolor{currentstroke}%
\pgfsetdash{}{0pt}%
\pgfpathmoveto{\pgfqpoint{3.372093in}{2.761754in}}%
\pgfpathlineto{\pgfqpoint{3.622205in}{2.891176in}}%
\pgfpathlineto{\pgfqpoint{3.372093in}{2.761754in}}%
\pgfpathclose%
\pgfusepath{fill}%
\end{pgfscope}%
\begin{pgfscope}%
\pgfpathrectangle{\pgfqpoint{0.017500in}{0.017500in}}{\pgfqpoint{6.218282in}{3.580372in}} %
\pgfusepath{clip}%
\pgfsetbuttcap%
\pgfsetroundjoin%
\definecolor{currentfill}{rgb}{0.717435,0.051118,0.158737}%
\pgfsetfillcolor{currentfill}%
\pgfsetlinewidth{0.000000pt}%
\definecolor{currentstroke}{rgb}{0.000000,0.000000,0.000000}%
\pgfsetstrokecolor{currentstroke}%
\pgfsetdash{}{0pt}%
\pgfpathmoveto{\pgfqpoint{3.372093in}{2.761754in}}%
\pgfpathlineto{\pgfqpoint{3.622205in}{2.891176in}}%
\pgfpathlineto{\pgfqpoint{3.372093in}{2.761754in}}%
\pgfpathclose%
\pgfusepath{fill}%
\end{pgfscope}%
\begin{pgfscope}%
\pgfpathrectangle{\pgfqpoint{0.017500in}{0.017500in}}{\pgfqpoint{6.218282in}{3.580372in}} %
\pgfusepath{clip}%
\pgfsetbuttcap%
\pgfsetroundjoin%
\definecolor{currentfill}{rgb}{0.717435,0.051118,0.158737}%
\pgfsetfillcolor{currentfill}%
\pgfsetlinewidth{0.000000pt}%
\definecolor{currentstroke}{rgb}{0.000000,0.000000,0.000000}%
\pgfsetstrokecolor{currentstroke}%
\pgfsetdash{}{0pt}%
\pgfpathmoveto{\pgfqpoint{3.372093in}{2.761754in}}%
\pgfpathlineto{\pgfqpoint{3.622205in}{2.891176in}}%
\pgfpathlineto{\pgfqpoint{3.372093in}{2.761754in}}%
\pgfpathclose%
\pgfusepath{fill}%
\end{pgfscope}%
\begin{pgfscope}%
\pgfpathrectangle{\pgfqpoint{0.017500in}{0.017500in}}{\pgfqpoint{6.218282in}{3.580372in}} %
\pgfusepath{clip}%
\pgfsetbuttcap%
\pgfsetroundjoin%
\definecolor{currentfill}{rgb}{0.717435,0.051118,0.158737}%
\pgfsetfillcolor{currentfill}%
\pgfsetlinewidth{0.000000pt}%
\definecolor{currentstroke}{rgb}{0.000000,0.000000,0.000000}%
\pgfsetstrokecolor{currentstroke}%
\pgfsetdash{}{0pt}%
\pgfpathmoveto{\pgfqpoint{3.372093in}{2.761754in}}%
\pgfpathlineto{\pgfqpoint{3.622205in}{2.891176in}}%
\pgfpathlineto{\pgfqpoint{3.372093in}{2.761754in}}%
\pgfpathclose%
\pgfusepath{fill}%
\end{pgfscope}%
\begin{pgfscope}%
\pgfpathrectangle{\pgfqpoint{0.017500in}{0.017500in}}{\pgfqpoint{6.218282in}{3.580372in}} %
\pgfusepath{clip}%
\pgfsetbuttcap%
\pgfsetroundjoin%
\definecolor{currentfill}{rgb}{0.717435,0.051118,0.158737}%
\pgfsetfillcolor{currentfill}%
\pgfsetlinewidth{0.000000pt}%
\definecolor{currentstroke}{rgb}{0.000000,0.000000,0.000000}%
\pgfsetstrokecolor{currentstroke}%
\pgfsetdash{}{0pt}%
\pgfpathmoveto{\pgfqpoint{3.372093in}{2.761754in}}%
\pgfpathlineto{\pgfqpoint{3.622205in}{2.891176in}}%
\pgfpathlineto{\pgfqpoint{3.372093in}{2.761754in}}%
\pgfpathclose%
\pgfusepath{fill}%
\end{pgfscope}%
\begin{pgfscope}%
\pgfpathrectangle{\pgfqpoint{0.017500in}{0.017500in}}{\pgfqpoint{6.218282in}{3.580372in}} %
\pgfusepath{clip}%
\pgfsetbuttcap%
\pgfsetroundjoin%
\definecolor{currentfill}{rgb}{0.717435,0.051118,0.158737}%
\pgfsetfillcolor{currentfill}%
\pgfsetlinewidth{0.000000pt}%
\definecolor{currentstroke}{rgb}{0.000000,0.000000,0.000000}%
\pgfsetstrokecolor{currentstroke}%
\pgfsetdash{}{0pt}%
\pgfpathmoveto{\pgfqpoint{3.372093in}{2.761754in}}%
\pgfpathlineto{\pgfqpoint{3.622205in}{2.891176in}}%
\pgfpathlineto{\pgfqpoint{3.372093in}{2.761754in}}%
\pgfpathclose%
\pgfusepath{fill}%
\end{pgfscope}%
\begin{pgfscope}%
\pgfpathrectangle{\pgfqpoint{0.017500in}{0.017500in}}{\pgfqpoint{6.218282in}{3.580372in}} %
\pgfusepath{clip}%
\pgfsetbuttcap%
\pgfsetroundjoin%
\definecolor{currentfill}{rgb}{0.717435,0.051118,0.158737}%
\pgfsetfillcolor{currentfill}%
\pgfsetlinewidth{0.000000pt}%
\definecolor{currentstroke}{rgb}{0.000000,0.000000,0.000000}%
\pgfsetstrokecolor{currentstroke}%
\pgfsetdash{}{0pt}%
\pgfpathmoveto{\pgfqpoint{3.372093in}{2.761754in}}%
\pgfpathlineto{\pgfqpoint{3.622205in}{2.891176in}}%
\pgfpathlineto{\pgfqpoint{3.372093in}{2.761754in}}%
\pgfpathclose%
\pgfusepath{fill}%
\end{pgfscope}%
\begin{pgfscope}%
\pgfpathrectangle{\pgfqpoint{0.017500in}{0.017500in}}{\pgfqpoint{6.218282in}{3.580372in}} %
\pgfusepath{clip}%
\pgfsetbuttcap%
\pgfsetroundjoin%
\definecolor{currentfill}{rgb}{0.717435,0.051118,0.158737}%
\pgfsetfillcolor{currentfill}%
\pgfsetlinewidth{0.000000pt}%
\definecolor{currentstroke}{rgb}{0.000000,0.000000,0.000000}%
\pgfsetstrokecolor{currentstroke}%
\pgfsetdash{}{0pt}%
\pgfpathmoveto{\pgfqpoint{3.372093in}{2.761754in}}%
\pgfpathlineto{\pgfqpoint{3.622205in}{2.891176in}}%
\pgfpathlineto{\pgfqpoint{3.372093in}{2.761754in}}%
\pgfpathclose%
\pgfusepath{fill}%
\end{pgfscope}%
\begin{pgfscope}%
\pgfpathrectangle{\pgfqpoint{0.017500in}{0.017500in}}{\pgfqpoint{6.218282in}{3.580372in}} %
\pgfusepath{clip}%
\pgfsetbuttcap%
\pgfsetroundjoin%
\definecolor{currentfill}{rgb}{0.717435,0.051118,0.158737}%
\pgfsetfillcolor{currentfill}%
\pgfsetlinewidth{0.000000pt}%
\definecolor{currentstroke}{rgb}{0.000000,0.000000,0.000000}%
\pgfsetstrokecolor{currentstroke}%
\pgfsetdash{}{0pt}%
\pgfpathmoveto{\pgfqpoint{3.372093in}{2.761754in}}%
\pgfpathlineto{\pgfqpoint{3.622205in}{2.891176in}}%
\pgfpathlineto{\pgfqpoint{3.372093in}{2.761754in}}%
\pgfpathclose%
\pgfusepath{fill}%
\end{pgfscope}%
\begin{pgfscope}%
\pgfpathrectangle{\pgfqpoint{0.017500in}{0.017500in}}{\pgfqpoint{6.218282in}{3.580372in}} %
\pgfusepath{clip}%
\pgfsetbuttcap%
\pgfsetroundjoin%
\definecolor{currentfill}{rgb}{0.717435,0.051118,0.158737}%
\pgfsetfillcolor{currentfill}%
\pgfsetlinewidth{0.000000pt}%
\definecolor{currentstroke}{rgb}{0.000000,0.000000,0.000000}%
\pgfsetstrokecolor{currentstroke}%
\pgfsetdash{}{0pt}%
\pgfpathmoveto{\pgfqpoint{3.372093in}{2.761754in}}%
\pgfpathlineto{\pgfqpoint{3.622205in}{2.891176in}}%
\pgfpathlineto{\pgfqpoint{3.372093in}{2.761754in}}%
\pgfpathclose%
\pgfusepath{fill}%
\end{pgfscope}%
\begin{pgfscope}%
\pgfpathrectangle{\pgfqpoint{0.017500in}{0.017500in}}{\pgfqpoint{6.218282in}{3.580372in}} %
\pgfusepath{clip}%
\pgfsetbuttcap%
\pgfsetroundjoin%
\definecolor{currentfill}{rgb}{0.717435,0.051118,0.158737}%
\pgfsetfillcolor{currentfill}%
\pgfsetlinewidth{0.000000pt}%
\definecolor{currentstroke}{rgb}{0.000000,0.000000,0.000000}%
\pgfsetstrokecolor{currentstroke}%
\pgfsetdash{}{0pt}%
\pgfpathmoveto{\pgfqpoint{3.372093in}{2.761754in}}%
\pgfpathlineto{\pgfqpoint{3.622205in}{2.891176in}}%
\pgfpathlineto{\pgfqpoint{3.372093in}{2.761754in}}%
\pgfpathclose%
\pgfusepath{fill}%
\end{pgfscope}%
\begin{pgfscope}%
\pgfpathrectangle{\pgfqpoint{0.017500in}{0.017500in}}{\pgfqpoint{6.218282in}{3.580372in}} %
\pgfusepath{clip}%
\pgfsetbuttcap%
\pgfsetroundjoin%
\definecolor{currentfill}{rgb}{0.717435,0.051118,0.158737}%
\pgfsetfillcolor{currentfill}%
\pgfsetlinewidth{0.000000pt}%
\definecolor{currentstroke}{rgb}{0.000000,0.000000,0.000000}%
\pgfsetstrokecolor{currentstroke}%
\pgfsetdash{}{0pt}%
\pgfpathmoveto{\pgfqpoint{3.372093in}{2.761754in}}%
\pgfpathlineto{\pgfqpoint{3.622205in}{2.891176in}}%
\pgfpathlineto{\pgfqpoint{3.372093in}{2.761754in}}%
\pgfpathclose%
\pgfusepath{fill}%
\end{pgfscope}%
\begin{pgfscope}%
\pgfpathrectangle{\pgfqpoint{0.017500in}{0.017500in}}{\pgfqpoint{6.218282in}{3.580372in}} %
\pgfusepath{clip}%
\pgfsetbuttcap%
\pgfsetroundjoin%
\definecolor{currentfill}{rgb}{0.717435,0.051118,0.158737}%
\pgfsetfillcolor{currentfill}%
\pgfsetlinewidth{0.000000pt}%
\definecolor{currentstroke}{rgb}{0.000000,0.000000,0.000000}%
\pgfsetstrokecolor{currentstroke}%
\pgfsetdash{}{0pt}%
\pgfpathmoveto{\pgfqpoint{3.372093in}{2.761754in}}%
\pgfpathlineto{\pgfqpoint{3.622205in}{2.891176in}}%
\pgfpathlineto{\pgfqpoint{3.372093in}{2.761754in}}%
\pgfpathclose%
\pgfusepath{fill}%
\end{pgfscope}%
\begin{pgfscope}%
\pgfpathrectangle{\pgfqpoint{0.017500in}{0.017500in}}{\pgfqpoint{6.218282in}{3.580372in}} %
\pgfusepath{clip}%
\pgfsetbuttcap%
\pgfsetroundjoin%
\definecolor{currentfill}{rgb}{0.717435,0.051118,0.158737}%
\pgfsetfillcolor{currentfill}%
\pgfsetlinewidth{0.000000pt}%
\definecolor{currentstroke}{rgb}{0.000000,0.000000,0.000000}%
\pgfsetstrokecolor{currentstroke}%
\pgfsetdash{}{0pt}%
\pgfpathmoveto{\pgfqpoint{3.372093in}{2.761754in}}%
\pgfpathlineto{\pgfqpoint{3.622205in}{2.891176in}}%
\pgfpathlineto{\pgfqpoint{3.372093in}{2.761754in}}%
\pgfpathclose%
\pgfusepath{fill}%
\end{pgfscope}%
\begin{pgfscope}%
\pgfpathrectangle{\pgfqpoint{0.017500in}{0.017500in}}{\pgfqpoint{6.218282in}{3.580372in}} %
\pgfusepath{clip}%
\pgfsetbuttcap%
\pgfsetroundjoin%
\definecolor{currentfill}{rgb}{0.717435,0.051118,0.158737}%
\pgfsetfillcolor{currentfill}%
\pgfsetlinewidth{0.000000pt}%
\definecolor{currentstroke}{rgb}{0.000000,0.000000,0.000000}%
\pgfsetstrokecolor{currentstroke}%
\pgfsetdash{}{0pt}%
\pgfpathmoveto{\pgfqpoint{3.372093in}{2.761754in}}%
\pgfpathlineto{\pgfqpoint{3.622205in}{2.891176in}}%
\pgfpathlineto{\pgfqpoint{3.372093in}{2.761754in}}%
\pgfpathclose%
\pgfusepath{fill}%
\end{pgfscope}%
\begin{pgfscope}%
\pgfpathrectangle{\pgfqpoint{0.017500in}{0.017500in}}{\pgfqpoint{6.218282in}{3.580372in}} %
\pgfusepath{clip}%
\pgfsetbuttcap%
\pgfsetroundjoin%
\definecolor{currentfill}{rgb}{0.717435,0.051118,0.158737}%
\pgfsetfillcolor{currentfill}%
\pgfsetlinewidth{0.000000pt}%
\definecolor{currentstroke}{rgb}{0.000000,0.000000,0.000000}%
\pgfsetstrokecolor{currentstroke}%
\pgfsetdash{}{0pt}%
\pgfpathmoveto{\pgfqpoint{3.372093in}{2.761754in}}%
\pgfpathlineto{\pgfqpoint{3.622205in}{2.891176in}}%
\pgfpathlineto{\pgfqpoint{3.372093in}{2.761754in}}%
\pgfpathclose%
\pgfusepath{fill}%
\end{pgfscope}%
\begin{pgfscope}%
\pgfpathrectangle{\pgfqpoint{0.017500in}{0.017500in}}{\pgfqpoint{6.218282in}{3.580372in}} %
\pgfusepath{clip}%
\pgfsetbuttcap%
\pgfsetroundjoin%
\definecolor{currentfill}{rgb}{0.800601,0.850358,0.930008}%
\pgfsetfillcolor{currentfill}%
\pgfsetlinewidth{0.000000pt}%
\definecolor{currentstroke}{rgb}{0.000000,0.000000,0.000000}%
\pgfsetstrokecolor{currentstroke}%
\pgfsetdash{}{0pt}%
\pgfpathmoveto{\pgfqpoint{3.622205in}{2.891176in}}%
\pgfpathlineto{\pgfqpoint{4.063701in}{2.813820in}}%
\pgfpathlineto{\pgfqpoint{2.789950in}{0.798346in}}%
\pgfpathlineto{\pgfqpoint{2.349853in}{0.872712in}}%
\pgfpathlineto{\pgfqpoint{3.622205in}{2.891176in}}%
\pgfpathclose%
\pgfusepath{fill}%
\end{pgfscope}%
\begin{pgfscope}%
\pgfpathrectangle{\pgfqpoint{0.017500in}{0.017500in}}{\pgfqpoint{6.218282in}{3.580372in}} %
\pgfusepath{clip}%
\pgfsetbuttcap%
\pgfsetroundjoin%
\definecolor{currentfill}{rgb}{0.800601,0.850358,0.930008}%
\pgfsetfillcolor{currentfill}%
\pgfsetlinewidth{0.000000pt}%
\definecolor{currentstroke}{rgb}{0.000000,0.000000,0.000000}%
\pgfsetstrokecolor{currentstroke}%
\pgfsetdash{}{0pt}%
\pgfpathmoveto{\pgfqpoint{3.622205in}{2.891176in}}%
\pgfpathlineto{\pgfqpoint{4.063701in}{2.813820in}}%
\pgfpathlineto{\pgfqpoint{2.789950in}{0.798346in}}%
\pgfpathlineto{\pgfqpoint{2.349853in}{0.872712in}}%
\pgfpathlineto{\pgfqpoint{3.622205in}{2.891176in}}%
\pgfpathclose%
\pgfusepath{fill}%
\end{pgfscope}%
\begin{pgfscope}%
\pgfpathrectangle{\pgfqpoint{0.017500in}{0.017500in}}{\pgfqpoint{6.218282in}{3.580372in}} %
\pgfusepath{clip}%
\pgfsetbuttcap%
\pgfsetroundjoin%
\definecolor{currentfill}{rgb}{0.800601,0.850358,0.930008}%
\pgfsetfillcolor{currentfill}%
\pgfsetlinewidth{0.000000pt}%
\definecolor{currentstroke}{rgb}{0.000000,0.000000,0.000000}%
\pgfsetstrokecolor{currentstroke}%
\pgfsetdash{}{0pt}%
\pgfpathmoveto{\pgfqpoint{3.622205in}{2.891176in}}%
\pgfpathlineto{\pgfqpoint{4.063701in}{2.813820in}}%
\pgfpathlineto{\pgfqpoint{2.789950in}{0.798346in}}%
\pgfpathlineto{\pgfqpoint{2.349853in}{0.872712in}}%
\pgfpathlineto{\pgfqpoint{3.622205in}{2.891176in}}%
\pgfpathclose%
\pgfusepath{fill}%
\end{pgfscope}%
\begin{pgfscope}%
\pgfpathrectangle{\pgfqpoint{0.017500in}{0.017500in}}{\pgfqpoint{6.218282in}{3.580372in}} %
\pgfusepath{clip}%
\pgfsetbuttcap%
\pgfsetroundjoin%
\definecolor{currentfill}{rgb}{0.800601,0.850358,0.930008}%
\pgfsetfillcolor{currentfill}%
\pgfsetlinewidth{0.000000pt}%
\definecolor{currentstroke}{rgb}{0.000000,0.000000,0.000000}%
\pgfsetstrokecolor{currentstroke}%
\pgfsetdash{}{0pt}%
\pgfpathmoveto{\pgfqpoint{3.622205in}{2.891176in}}%
\pgfpathlineto{\pgfqpoint{4.063701in}{2.813820in}}%
\pgfpathlineto{\pgfqpoint{2.789950in}{0.798346in}}%
\pgfpathlineto{\pgfqpoint{2.349853in}{0.872712in}}%
\pgfpathlineto{\pgfqpoint{3.622205in}{2.891176in}}%
\pgfpathclose%
\pgfusepath{fill}%
\end{pgfscope}%
\begin{pgfscope}%
\pgfpathrectangle{\pgfqpoint{0.017500in}{0.017500in}}{\pgfqpoint{6.218282in}{3.580372in}} %
\pgfusepath{clip}%
\pgfsetbuttcap%
\pgfsetroundjoin%
\definecolor{currentfill}{rgb}{0.800601,0.850358,0.930008}%
\pgfsetfillcolor{currentfill}%
\pgfsetlinewidth{0.000000pt}%
\definecolor{currentstroke}{rgb}{0.000000,0.000000,0.000000}%
\pgfsetstrokecolor{currentstroke}%
\pgfsetdash{}{0pt}%
\pgfpathmoveto{\pgfqpoint{3.622205in}{2.891176in}}%
\pgfpathlineto{\pgfqpoint{4.063701in}{2.813820in}}%
\pgfpathlineto{\pgfqpoint{2.789950in}{0.798346in}}%
\pgfpathlineto{\pgfqpoint{2.349853in}{0.872712in}}%
\pgfpathlineto{\pgfqpoint{3.622205in}{2.891176in}}%
\pgfpathclose%
\pgfusepath{fill}%
\end{pgfscope}%
\begin{pgfscope}%
\pgfpathrectangle{\pgfqpoint{0.017500in}{0.017500in}}{\pgfqpoint{6.218282in}{3.580372in}} %
\pgfusepath{clip}%
\pgfsetbuttcap%
\pgfsetroundjoin%
\definecolor{currentfill}{rgb}{0.723315,0.068898,0.162989}%
\pgfsetfillcolor{currentfill}%
\pgfsetlinewidth{0.000000pt}%
\definecolor{currentstroke}{rgb}{0.000000,0.000000,0.000000}%
\pgfsetstrokecolor{currentstroke}%
\pgfsetdash{}{0pt}%
\pgfpathmoveto{\pgfqpoint{3.372093in}{2.761754in}}%
\pgfpathlineto{\pgfqpoint{3.816789in}{2.677291in}}%
\pgfpathlineto{\pgfqpoint{4.063701in}{2.813820in}}%
\pgfpathlineto{\pgfqpoint{3.622205in}{2.891176in}}%
\pgfpathlineto{\pgfqpoint{3.372093in}{2.761754in}}%
\pgfpathclose%
\pgfusepath{fill}%
\end{pgfscope}%
\begin{pgfscope}%
\pgfpathrectangle{\pgfqpoint{0.017500in}{0.017500in}}{\pgfqpoint{6.218282in}{3.580372in}} %
\pgfusepath{clip}%
\pgfsetbuttcap%
\pgfsetroundjoin%
\definecolor{currentfill}{rgb}{0.723315,0.068898,0.162989}%
\pgfsetfillcolor{currentfill}%
\pgfsetlinewidth{0.000000pt}%
\definecolor{currentstroke}{rgb}{0.000000,0.000000,0.000000}%
\pgfsetstrokecolor{currentstroke}%
\pgfsetdash{}{0pt}%
\pgfpathmoveto{\pgfqpoint{3.372093in}{2.761754in}}%
\pgfpathlineto{\pgfqpoint{3.816789in}{2.677291in}}%
\pgfpathlineto{\pgfqpoint{4.063701in}{2.813820in}}%
\pgfpathlineto{\pgfqpoint{3.622205in}{2.891176in}}%
\pgfpathlineto{\pgfqpoint{3.372093in}{2.761754in}}%
\pgfpathclose%
\pgfusepath{fill}%
\end{pgfscope}%
\begin{pgfscope}%
\pgfpathrectangle{\pgfqpoint{0.017500in}{0.017500in}}{\pgfqpoint{6.218282in}{3.580372in}} %
\pgfusepath{clip}%
\pgfsetbuttcap%
\pgfsetroundjoin%
\definecolor{currentfill}{rgb}{0.723315,0.068898,0.162989}%
\pgfsetfillcolor{currentfill}%
\pgfsetlinewidth{0.000000pt}%
\definecolor{currentstroke}{rgb}{0.000000,0.000000,0.000000}%
\pgfsetstrokecolor{currentstroke}%
\pgfsetdash{}{0pt}%
\pgfpathmoveto{\pgfqpoint{3.372093in}{2.761754in}}%
\pgfpathlineto{\pgfqpoint{3.816789in}{2.677291in}}%
\pgfpathlineto{\pgfqpoint{4.063701in}{2.813820in}}%
\pgfpathlineto{\pgfqpoint{3.622205in}{2.891176in}}%
\pgfpathlineto{\pgfqpoint{3.372093in}{2.761754in}}%
\pgfpathclose%
\pgfusepath{fill}%
\end{pgfscope}%
\begin{pgfscope}%
\pgfpathrectangle{\pgfqpoint{0.017500in}{0.017500in}}{\pgfqpoint{6.218282in}{3.580372in}} %
\pgfusepath{clip}%
\pgfsetbuttcap%
\pgfsetroundjoin%
\definecolor{currentfill}{rgb}{0.723315,0.068898,0.162989}%
\pgfsetfillcolor{currentfill}%
\pgfsetlinewidth{0.000000pt}%
\definecolor{currentstroke}{rgb}{0.000000,0.000000,0.000000}%
\pgfsetstrokecolor{currentstroke}%
\pgfsetdash{}{0pt}%
\pgfpathmoveto{\pgfqpoint{3.372093in}{2.761754in}}%
\pgfpathlineto{\pgfqpoint{3.816789in}{2.677291in}}%
\pgfpathlineto{\pgfqpoint{4.063701in}{2.813820in}}%
\pgfpathlineto{\pgfqpoint{3.622205in}{2.891176in}}%
\pgfpathlineto{\pgfqpoint{3.372093in}{2.761754in}}%
\pgfpathclose%
\pgfusepath{fill}%
\end{pgfscope}%
\begin{pgfscope}%
\pgfpathrectangle{\pgfqpoint{0.017500in}{0.017500in}}{\pgfqpoint{6.218282in}{3.580372in}} %
\pgfusepath{clip}%
\pgfsetbuttcap%
\pgfsetroundjoin%
\definecolor{currentfill}{rgb}{0.723315,0.068898,0.162989}%
\pgfsetfillcolor{currentfill}%
\pgfsetlinewidth{0.000000pt}%
\definecolor{currentstroke}{rgb}{0.000000,0.000000,0.000000}%
\pgfsetstrokecolor{currentstroke}%
\pgfsetdash{}{0pt}%
\pgfpathmoveto{\pgfqpoint{3.372093in}{2.761754in}}%
\pgfpathlineto{\pgfqpoint{3.816789in}{2.677291in}}%
\pgfpathlineto{\pgfqpoint{4.063701in}{2.813820in}}%
\pgfpathlineto{\pgfqpoint{3.622205in}{2.891176in}}%
\pgfpathlineto{\pgfqpoint{3.372093in}{2.761754in}}%
\pgfpathclose%
\pgfusepath{fill}%
\end{pgfscope}%
\begin{pgfscope}%
\pgfpathrectangle{\pgfqpoint{0.017500in}{0.017500in}}{\pgfqpoint{6.218282in}{3.580372in}} %
\pgfusepath{clip}%
\pgfsetbuttcap%
\pgfsetroundjoin%
\definecolor{currentfill}{rgb}{0.723315,0.068898,0.162989}%
\pgfsetfillcolor{currentfill}%
\pgfsetlinewidth{0.000000pt}%
\definecolor{currentstroke}{rgb}{0.000000,0.000000,0.000000}%
\pgfsetstrokecolor{currentstroke}%
\pgfsetdash{}{0pt}%
\pgfpathmoveto{\pgfqpoint{3.372093in}{2.761754in}}%
\pgfpathlineto{\pgfqpoint{3.816789in}{2.677291in}}%
\pgfpathlineto{\pgfqpoint{4.063701in}{2.813820in}}%
\pgfpathlineto{\pgfqpoint{3.622205in}{2.891176in}}%
\pgfpathlineto{\pgfqpoint{3.372093in}{2.761754in}}%
\pgfpathclose%
\pgfusepath{fill}%
\end{pgfscope}%
\begin{pgfscope}%
\pgfpathrectangle{\pgfqpoint{0.017500in}{0.017500in}}{\pgfqpoint{6.218282in}{3.580372in}} %
\pgfusepath{clip}%
\pgfsetbuttcap%
\pgfsetroundjoin%
\definecolor{currentfill}{rgb}{0.243520,0.319189,0.771672}%
\pgfsetfillcolor{currentfill}%
\pgfsetlinewidth{0.000000pt}%
\definecolor{currentstroke}{rgb}{0.000000,0.000000,0.000000}%
\pgfsetstrokecolor{currentstroke}%
\pgfsetdash{}{0pt}%
\pgfpathmoveto{\pgfqpoint{1.486050in}{1.042638in}}%
\pgfpathlineto{\pgfqpoint{1.701778in}{2.375902in}}%
\pgfpathlineto{\pgfqpoint{1.486050in}{1.042638in}}%
\pgfpathclose%
\pgfusepath{fill}%
\end{pgfscope}%
\begin{pgfscope}%
\pgfpathrectangle{\pgfqpoint{0.017500in}{0.017500in}}{\pgfqpoint{6.218282in}{3.580372in}} %
\pgfusepath{clip}%
\pgfsetbuttcap%
\pgfsetroundjoin%
\definecolor{currentfill}{rgb}{0.243520,0.319189,0.771672}%
\pgfsetfillcolor{currentfill}%
\pgfsetlinewidth{0.000000pt}%
\definecolor{currentstroke}{rgb}{0.000000,0.000000,0.000000}%
\pgfsetstrokecolor{currentstroke}%
\pgfsetdash{}{0pt}%
\pgfpathmoveto{\pgfqpoint{1.486050in}{1.042638in}}%
\pgfpathlineto{\pgfqpoint{1.701778in}{2.375902in}}%
\pgfpathlineto{\pgfqpoint{1.486050in}{1.042638in}}%
\pgfpathclose%
\pgfusepath{fill}%
\end{pgfscope}%
\begin{pgfscope}%
\pgfpathrectangle{\pgfqpoint{0.017500in}{0.017500in}}{\pgfqpoint{6.218282in}{3.580372in}} %
\pgfusepath{clip}%
\pgfsetbuttcap%
\pgfsetroundjoin%
\definecolor{currentfill}{rgb}{0.243520,0.319189,0.771672}%
\pgfsetfillcolor{currentfill}%
\pgfsetlinewidth{0.000000pt}%
\definecolor{currentstroke}{rgb}{0.000000,0.000000,0.000000}%
\pgfsetstrokecolor{currentstroke}%
\pgfsetdash{}{0pt}%
\pgfpathmoveto{\pgfqpoint{1.486050in}{1.042638in}}%
\pgfpathlineto{\pgfqpoint{1.701778in}{2.375902in}}%
\pgfpathlineto{\pgfqpoint{1.486050in}{1.042638in}}%
\pgfpathclose%
\pgfusepath{fill}%
\end{pgfscope}%
\begin{pgfscope}%
\pgfpathrectangle{\pgfqpoint{0.017500in}{0.017500in}}{\pgfqpoint{6.218282in}{3.580372in}} %
\pgfusepath{clip}%
\pgfsetbuttcap%
\pgfsetroundjoin%
\definecolor{currentfill}{rgb}{0.243520,0.319189,0.771672}%
\pgfsetfillcolor{currentfill}%
\pgfsetlinewidth{0.000000pt}%
\definecolor{currentstroke}{rgb}{0.000000,0.000000,0.000000}%
\pgfsetstrokecolor{currentstroke}%
\pgfsetdash{}{0pt}%
\pgfpathmoveto{\pgfqpoint{1.486050in}{1.042638in}}%
\pgfpathlineto{\pgfqpoint{1.701778in}{2.375902in}}%
\pgfpathlineto{\pgfqpoint{1.486050in}{1.042638in}}%
\pgfpathclose%
\pgfusepath{fill}%
\end{pgfscope}%
\begin{pgfscope}%
\pgfpathrectangle{\pgfqpoint{0.017500in}{0.017500in}}{\pgfqpoint{6.218282in}{3.580372in}} %
\pgfusepath{clip}%
\pgfsetbuttcap%
\pgfsetroundjoin%
\definecolor{currentfill}{rgb}{0.243520,0.319189,0.771672}%
\pgfsetfillcolor{currentfill}%
\pgfsetlinewidth{0.000000pt}%
\definecolor{currentstroke}{rgb}{0.000000,0.000000,0.000000}%
\pgfsetstrokecolor{currentstroke}%
\pgfsetdash{}{0pt}%
\pgfpathmoveto{\pgfqpoint{1.486050in}{1.042638in}}%
\pgfpathlineto{\pgfqpoint{1.701778in}{2.375902in}}%
\pgfpathlineto{\pgfqpoint{1.486050in}{1.042638in}}%
\pgfpathclose%
\pgfusepath{fill}%
\end{pgfscope}%
\begin{pgfscope}%
\pgfpathrectangle{\pgfqpoint{0.017500in}{0.017500in}}{\pgfqpoint{6.218282in}{3.580372in}} %
\pgfusepath{clip}%
\pgfsetbuttcap%
\pgfsetroundjoin%
\definecolor{currentfill}{rgb}{0.243520,0.319189,0.771672}%
\pgfsetfillcolor{currentfill}%
\pgfsetlinewidth{0.000000pt}%
\definecolor{currentstroke}{rgb}{0.000000,0.000000,0.000000}%
\pgfsetstrokecolor{currentstroke}%
\pgfsetdash{}{0pt}%
\pgfpathmoveto{\pgfqpoint{1.486050in}{1.042638in}}%
\pgfpathlineto{\pgfqpoint{1.701778in}{2.375902in}}%
\pgfpathlineto{\pgfqpoint{1.486050in}{1.042638in}}%
\pgfpathclose%
\pgfusepath{fill}%
\end{pgfscope}%
\begin{pgfscope}%
\pgfpathrectangle{\pgfqpoint{0.017500in}{0.017500in}}{\pgfqpoint{6.218282in}{3.580372in}} %
\pgfusepath{clip}%
\pgfsetbuttcap%
\pgfsetroundjoin%
\definecolor{currentfill}{rgb}{0.243520,0.319189,0.771672}%
\pgfsetfillcolor{currentfill}%
\pgfsetlinewidth{0.000000pt}%
\definecolor{currentstroke}{rgb}{0.000000,0.000000,0.000000}%
\pgfsetstrokecolor{currentstroke}%
\pgfsetdash{}{0pt}%
\pgfpathmoveto{\pgfqpoint{1.486050in}{1.042638in}}%
\pgfpathlineto{\pgfqpoint{1.701778in}{2.375902in}}%
\pgfpathlineto{\pgfqpoint{1.486050in}{1.042638in}}%
\pgfpathclose%
\pgfusepath{fill}%
\end{pgfscope}%
\begin{pgfscope}%
\pgfpathrectangle{\pgfqpoint{0.017500in}{0.017500in}}{\pgfqpoint{6.218282in}{3.580372in}} %
\pgfusepath{clip}%
\pgfsetbuttcap%
\pgfsetroundjoin%
\definecolor{currentfill}{rgb}{0.243520,0.319189,0.771672}%
\pgfsetfillcolor{currentfill}%
\pgfsetlinewidth{0.000000pt}%
\definecolor{currentstroke}{rgb}{0.000000,0.000000,0.000000}%
\pgfsetstrokecolor{currentstroke}%
\pgfsetdash{}{0pt}%
\pgfpathmoveto{\pgfqpoint{1.486050in}{1.042638in}}%
\pgfpathlineto{\pgfqpoint{1.701778in}{2.375902in}}%
\pgfpathlineto{\pgfqpoint{1.486050in}{1.042638in}}%
\pgfpathclose%
\pgfusepath{fill}%
\end{pgfscope}%
\begin{pgfscope}%
\pgfpathrectangle{\pgfqpoint{0.017500in}{0.017500in}}{\pgfqpoint{6.218282in}{3.580372in}} %
\pgfusepath{clip}%
\pgfsetbuttcap%
\pgfsetroundjoin%
\definecolor{currentfill}{rgb}{0.243520,0.319189,0.771672}%
\pgfsetfillcolor{currentfill}%
\pgfsetlinewidth{0.000000pt}%
\definecolor{currentstroke}{rgb}{0.000000,0.000000,0.000000}%
\pgfsetstrokecolor{currentstroke}%
\pgfsetdash{}{0pt}%
\pgfpathmoveto{\pgfqpoint{1.486050in}{1.042638in}}%
\pgfpathlineto{\pgfqpoint{1.701778in}{2.375902in}}%
\pgfpathlineto{\pgfqpoint{1.486050in}{1.042638in}}%
\pgfpathclose%
\pgfusepath{fill}%
\end{pgfscope}%
\begin{pgfscope}%
\pgfpathrectangle{\pgfqpoint{0.017500in}{0.017500in}}{\pgfqpoint{6.218282in}{3.580372in}} %
\pgfusepath{clip}%
\pgfsetbuttcap%
\pgfsetroundjoin%
\definecolor{currentfill}{rgb}{0.243520,0.319189,0.771672}%
\pgfsetfillcolor{currentfill}%
\pgfsetlinewidth{0.000000pt}%
\definecolor{currentstroke}{rgb}{0.000000,0.000000,0.000000}%
\pgfsetstrokecolor{currentstroke}%
\pgfsetdash{}{0pt}%
\pgfpathmoveto{\pgfqpoint{1.486050in}{1.042638in}}%
\pgfpathlineto{\pgfqpoint{1.701778in}{2.375902in}}%
\pgfpathlineto{\pgfqpoint{1.486050in}{1.042638in}}%
\pgfpathclose%
\pgfusepath{fill}%
\end{pgfscope}%
\begin{pgfscope}%
\pgfpathrectangle{\pgfqpoint{0.017500in}{0.017500in}}{\pgfqpoint{6.218282in}{3.580372in}} %
\pgfusepath{clip}%
\pgfsetbuttcap%
\pgfsetroundjoin%
\definecolor{currentfill}{rgb}{0.243520,0.319189,0.771672}%
\pgfsetfillcolor{currentfill}%
\pgfsetlinewidth{0.000000pt}%
\definecolor{currentstroke}{rgb}{0.000000,0.000000,0.000000}%
\pgfsetstrokecolor{currentstroke}%
\pgfsetdash{}{0pt}%
\pgfpathmoveto{\pgfqpoint{1.486050in}{1.042638in}}%
\pgfpathlineto{\pgfqpoint{1.701778in}{2.375902in}}%
\pgfpathlineto{\pgfqpoint{1.486050in}{1.042638in}}%
\pgfpathclose%
\pgfusepath{fill}%
\end{pgfscope}%
\begin{pgfscope}%
\pgfpathrectangle{\pgfqpoint{0.017500in}{0.017500in}}{\pgfqpoint{6.218282in}{3.580372in}} %
\pgfusepath{clip}%
\pgfsetbuttcap%
\pgfsetroundjoin%
\definecolor{currentfill}{rgb}{0.243520,0.319189,0.771672}%
\pgfsetfillcolor{currentfill}%
\pgfsetlinewidth{0.000000pt}%
\definecolor{currentstroke}{rgb}{0.000000,0.000000,0.000000}%
\pgfsetstrokecolor{currentstroke}%
\pgfsetdash{}{0pt}%
\pgfpathmoveto{\pgfqpoint{1.486050in}{1.042638in}}%
\pgfpathlineto{\pgfqpoint{1.701778in}{2.375902in}}%
\pgfpathlineto{\pgfqpoint{1.486050in}{1.042638in}}%
\pgfpathclose%
\pgfusepath{fill}%
\end{pgfscope}%
\begin{pgfscope}%
\pgfpathrectangle{\pgfqpoint{0.017500in}{0.017500in}}{\pgfqpoint{6.218282in}{3.580372in}} %
\pgfusepath{clip}%
\pgfsetbuttcap%
\pgfsetroundjoin%
\definecolor{currentfill}{rgb}{0.243520,0.319189,0.771672}%
\pgfsetfillcolor{currentfill}%
\pgfsetlinewidth{0.000000pt}%
\definecolor{currentstroke}{rgb}{0.000000,0.000000,0.000000}%
\pgfsetstrokecolor{currentstroke}%
\pgfsetdash{}{0pt}%
\pgfpathmoveto{\pgfqpoint{1.486050in}{1.042638in}}%
\pgfpathlineto{\pgfqpoint{1.701778in}{2.375902in}}%
\pgfpathlineto{\pgfqpoint{1.486050in}{1.042638in}}%
\pgfpathclose%
\pgfusepath{fill}%
\end{pgfscope}%
\begin{pgfscope}%
\pgfpathrectangle{\pgfqpoint{0.017500in}{0.017500in}}{\pgfqpoint{6.218282in}{3.580372in}} %
\pgfusepath{clip}%
\pgfsetbuttcap%
\pgfsetroundjoin%
\definecolor{currentfill}{rgb}{0.243520,0.319189,0.771672}%
\pgfsetfillcolor{currentfill}%
\pgfsetlinewidth{0.000000pt}%
\definecolor{currentstroke}{rgb}{0.000000,0.000000,0.000000}%
\pgfsetstrokecolor{currentstroke}%
\pgfsetdash{}{0pt}%
\pgfpathmoveto{\pgfqpoint{1.486050in}{1.042638in}}%
\pgfpathlineto{\pgfqpoint{1.701778in}{2.375902in}}%
\pgfpathlineto{\pgfqpoint{1.486050in}{1.042638in}}%
\pgfpathclose%
\pgfusepath{fill}%
\end{pgfscope}%
\begin{pgfscope}%
\pgfpathrectangle{\pgfqpoint{0.017500in}{0.017500in}}{\pgfqpoint{6.218282in}{3.580372in}} %
\pgfusepath{clip}%
\pgfsetbuttcap%
\pgfsetroundjoin%
\definecolor{currentfill}{rgb}{0.243520,0.319189,0.771672}%
\pgfsetfillcolor{currentfill}%
\pgfsetlinewidth{0.000000pt}%
\definecolor{currentstroke}{rgb}{0.000000,0.000000,0.000000}%
\pgfsetstrokecolor{currentstroke}%
\pgfsetdash{}{0pt}%
\pgfpathmoveto{\pgfqpoint{1.486050in}{1.042638in}}%
\pgfpathlineto{\pgfqpoint{1.701778in}{2.375902in}}%
\pgfpathlineto{\pgfqpoint{1.486050in}{1.042638in}}%
\pgfpathclose%
\pgfusepath{fill}%
\end{pgfscope}%
\begin{pgfscope}%
\pgfpathrectangle{\pgfqpoint{0.017500in}{0.017500in}}{\pgfqpoint{6.218282in}{3.580372in}} %
\pgfusepath{clip}%
\pgfsetbuttcap%
\pgfsetroundjoin%
\definecolor{currentfill}{rgb}{0.243520,0.319189,0.771672}%
\pgfsetfillcolor{currentfill}%
\pgfsetlinewidth{0.000000pt}%
\definecolor{currentstroke}{rgb}{0.000000,0.000000,0.000000}%
\pgfsetstrokecolor{currentstroke}%
\pgfsetdash{}{0pt}%
\pgfpathmoveto{\pgfqpoint{1.486050in}{1.042638in}}%
\pgfpathlineto{\pgfqpoint{1.701778in}{2.375902in}}%
\pgfpathlineto{\pgfqpoint{1.486050in}{1.042638in}}%
\pgfpathclose%
\pgfusepath{fill}%
\end{pgfscope}%
\begin{pgfscope}%
\pgfpathrectangle{\pgfqpoint{0.017500in}{0.017500in}}{\pgfqpoint{6.218282in}{3.580372in}} %
\pgfusepath{clip}%
\pgfsetbuttcap%
\pgfsetroundjoin%
\definecolor{currentfill}{rgb}{0.243520,0.319189,0.771672}%
\pgfsetfillcolor{currentfill}%
\pgfsetlinewidth{0.000000pt}%
\definecolor{currentstroke}{rgb}{0.000000,0.000000,0.000000}%
\pgfsetstrokecolor{currentstroke}%
\pgfsetdash{}{0pt}%
\pgfpathmoveto{\pgfqpoint{1.486050in}{1.042638in}}%
\pgfpathlineto{\pgfqpoint{1.701778in}{2.375902in}}%
\pgfpathlineto{\pgfqpoint{1.486050in}{1.042638in}}%
\pgfpathclose%
\pgfusepath{fill}%
\end{pgfscope}%
\begin{pgfscope}%
\pgfpathrectangle{\pgfqpoint{0.017500in}{0.017500in}}{\pgfqpoint{6.218282in}{3.580372in}} %
\pgfusepath{clip}%
\pgfsetbuttcap%
\pgfsetroundjoin%
\definecolor{currentfill}{rgb}{0.243520,0.319189,0.771672}%
\pgfsetfillcolor{currentfill}%
\pgfsetlinewidth{0.000000pt}%
\definecolor{currentstroke}{rgb}{0.000000,0.000000,0.000000}%
\pgfsetstrokecolor{currentstroke}%
\pgfsetdash{}{0pt}%
\pgfpathmoveto{\pgfqpoint{1.486050in}{1.042638in}}%
\pgfpathlineto{\pgfqpoint{1.701778in}{2.375902in}}%
\pgfpathlineto{\pgfqpoint{1.486050in}{1.042638in}}%
\pgfpathclose%
\pgfusepath{fill}%
\end{pgfscope}%
\begin{pgfscope}%
\pgfpathrectangle{\pgfqpoint{0.017500in}{0.017500in}}{\pgfqpoint{6.218282in}{3.580372in}} %
\pgfusepath{clip}%
\pgfsetbuttcap%
\pgfsetroundjoin%
\definecolor{currentfill}{rgb}{0.243520,0.319189,0.771672}%
\pgfsetfillcolor{currentfill}%
\pgfsetlinewidth{0.000000pt}%
\definecolor{currentstroke}{rgb}{0.000000,0.000000,0.000000}%
\pgfsetstrokecolor{currentstroke}%
\pgfsetdash{}{0pt}%
\pgfpathmoveto{\pgfqpoint{1.486050in}{1.042638in}}%
\pgfpathlineto{\pgfqpoint{1.701778in}{2.375902in}}%
\pgfpathlineto{\pgfqpoint{1.486050in}{1.042638in}}%
\pgfpathclose%
\pgfusepath{fill}%
\end{pgfscope}%
\begin{pgfscope}%
\pgfpathrectangle{\pgfqpoint{0.017500in}{0.017500in}}{\pgfqpoint{6.218282in}{3.580372in}} %
\pgfusepath{clip}%
\pgfsetbuttcap%
\pgfsetroundjoin%
\definecolor{currentfill}{rgb}{0.243520,0.319189,0.771672}%
\pgfsetfillcolor{currentfill}%
\pgfsetlinewidth{0.000000pt}%
\definecolor{currentstroke}{rgb}{0.000000,0.000000,0.000000}%
\pgfsetstrokecolor{currentstroke}%
\pgfsetdash{}{0pt}%
\pgfpathmoveto{\pgfqpoint{1.486050in}{1.042638in}}%
\pgfpathlineto{\pgfqpoint{1.701778in}{2.375902in}}%
\pgfpathlineto{\pgfqpoint{1.486050in}{1.042638in}}%
\pgfpathclose%
\pgfusepath{fill}%
\end{pgfscope}%
\begin{pgfscope}%
\pgfpathrectangle{\pgfqpoint{0.017500in}{0.017500in}}{\pgfqpoint{6.218282in}{3.580372in}} %
\pgfusepath{clip}%
\pgfsetbuttcap%
\pgfsetroundjoin%
\definecolor{currentfill}{rgb}{0.243520,0.319189,0.771672}%
\pgfsetfillcolor{currentfill}%
\pgfsetlinewidth{0.000000pt}%
\definecolor{currentstroke}{rgb}{0.000000,0.000000,0.000000}%
\pgfsetstrokecolor{currentstroke}%
\pgfsetdash{}{0pt}%
\pgfpathmoveto{\pgfqpoint{1.486050in}{1.042638in}}%
\pgfpathlineto{\pgfqpoint{1.701778in}{2.375902in}}%
\pgfpathlineto{\pgfqpoint{1.486050in}{1.042638in}}%
\pgfpathclose%
\pgfusepath{fill}%
\end{pgfscope}%
\begin{pgfscope}%
\pgfpathrectangle{\pgfqpoint{0.017500in}{0.017500in}}{\pgfqpoint{6.218282in}{3.580372in}} %
\pgfusepath{clip}%
\pgfsetbuttcap%
\pgfsetroundjoin%
\definecolor{currentfill}{rgb}{0.243520,0.319189,0.771672}%
\pgfsetfillcolor{currentfill}%
\pgfsetlinewidth{0.000000pt}%
\definecolor{currentstroke}{rgb}{0.000000,0.000000,0.000000}%
\pgfsetstrokecolor{currentstroke}%
\pgfsetdash{}{0pt}%
\pgfpathmoveto{\pgfqpoint{1.486050in}{1.042638in}}%
\pgfpathlineto{\pgfqpoint{1.701778in}{2.375902in}}%
\pgfpathlineto{\pgfqpoint{1.486050in}{1.042638in}}%
\pgfpathclose%
\pgfusepath{fill}%
\end{pgfscope}%
\begin{pgfscope}%
\pgfpathrectangle{\pgfqpoint{0.017500in}{0.017500in}}{\pgfqpoint{6.218282in}{3.580372in}} %
\pgfusepath{clip}%
\pgfsetbuttcap%
\pgfsetroundjoin%
\definecolor{currentfill}{rgb}{0.243520,0.319189,0.771672}%
\pgfsetfillcolor{currentfill}%
\pgfsetlinewidth{0.000000pt}%
\definecolor{currentstroke}{rgb}{0.000000,0.000000,0.000000}%
\pgfsetstrokecolor{currentstroke}%
\pgfsetdash{}{0pt}%
\pgfpathmoveto{\pgfqpoint{1.486050in}{1.042638in}}%
\pgfpathlineto{\pgfqpoint{1.701778in}{2.375902in}}%
\pgfpathlineto{\pgfqpoint{1.486050in}{1.042638in}}%
\pgfpathclose%
\pgfusepath{fill}%
\end{pgfscope}%
\begin{pgfscope}%
\pgfpathrectangle{\pgfqpoint{0.017500in}{0.017500in}}{\pgfqpoint{6.218282in}{3.580372in}} %
\pgfusepath{clip}%
\pgfsetbuttcap%
\pgfsetroundjoin%
\definecolor{currentfill}{rgb}{0.243520,0.319189,0.771672}%
\pgfsetfillcolor{currentfill}%
\pgfsetlinewidth{0.000000pt}%
\definecolor{currentstroke}{rgb}{0.000000,0.000000,0.000000}%
\pgfsetstrokecolor{currentstroke}%
\pgfsetdash{}{0pt}%
\pgfpathmoveto{\pgfqpoint{1.486050in}{1.042638in}}%
\pgfpathlineto{\pgfqpoint{1.701778in}{2.375902in}}%
\pgfpathlineto{\pgfqpoint{1.486050in}{1.042638in}}%
\pgfpathclose%
\pgfusepath{fill}%
\end{pgfscope}%
\begin{pgfscope}%
\pgfpathrectangle{\pgfqpoint{0.017500in}{0.017500in}}{\pgfqpoint{6.218282in}{3.580372in}} %
\pgfusepath{clip}%
\pgfsetbuttcap%
\pgfsetroundjoin%
\definecolor{currentfill}{rgb}{0.243520,0.319189,0.771672}%
\pgfsetfillcolor{currentfill}%
\pgfsetlinewidth{0.000000pt}%
\definecolor{currentstroke}{rgb}{0.000000,0.000000,0.000000}%
\pgfsetstrokecolor{currentstroke}%
\pgfsetdash{}{0pt}%
\pgfpathmoveto{\pgfqpoint{1.486050in}{1.042638in}}%
\pgfpathlineto{\pgfqpoint{1.701778in}{2.375902in}}%
\pgfpathlineto{\pgfqpoint{1.486050in}{1.042638in}}%
\pgfpathclose%
\pgfusepath{fill}%
\end{pgfscope}%
\begin{pgfscope}%
\pgfpathrectangle{\pgfqpoint{0.017500in}{0.017500in}}{\pgfqpoint{6.218282in}{3.580372in}} %
\pgfusepath{clip}%
\pgfsetbuttcap%
\pgfsetroundjoin%
\definecolor{currentfill}{rgb}{0.243520,0.319189,0.771672}%
\pgfsetfillcolor{currentfill}%
\pgfsetlinewidth{0.000000pt}%
\definecolor{currentstroke}{rgb}{0.000000,0.000000,0.000000}%
\pgfsetstrokecolor{currentstroke}%
\pgfsetdash{}{0pt}%
\pgfpathmoveto{\pgfqpoint{1.486050in}{1.042638in}}%
\pgfpathlineto{\pgfqpoint{1.701778in}{2.375902in}}%
\pgfpathlineto{\pgfqpoint{1.486050in}{1.042638in}}%
\pgfpathclose%
\pgfusepath{fill}%
\end{pgfscope}%
\begin{pgfscope}%
\pgfpathrectangle{\pgfqpoint{0.017500in}{0.017500in}}{\pgfqpoint{6.218282in}{3.580372in}} %
\pgfusepath{clip}%
\pgfsetbuttcap%
\pgfsetroundjoin%
\definecolor{currentfill}{rgb}{0.243520,0.319189,0.771672}%
\pgfsetfillcolor{currentfill}%
\pgfsetlinewidth{0.000000pt}%
\definecolor{currentstroke}{rgb}{0.000000,0.000000,0.000000}%
\pgfsetstrokecolor{currentstroke}%
\pgfsetdash{}{0pt}%
\pgfpathmoveto{\pgfqpoint{1.486050in}{1.042638in}}%
\pgfpathlineto{\pgfqpoint{1.701778in}{2.375902in}}%
\pgfpathlineto{\pgfqpoint{1.486050in}{1.042638in}}%
\pgfpathclose%
\pgfusepath{fill}%
\end{pgfscope}%
\begin{pgfscope}%
\pgfpathrectangle{\pgfqpoint{0.017500in}{0.017500in}}{\pgfqpoint{6.218282in}{3.580372in}} %
\pgfusepath{clip}%
\pgfsetbuttcap%
\pgfsetroundjoin%
\definecolor{currentfill}{rgb}{0.243520,0.319189,0.771672}%
\pgfsetfillcolor{currentfill}%
\pgfsetlinewidth{0.000000pt}%
\definecolor{currentstroke}{rgb}{0.000000,0.000000,0.000000}%
\pgfsetstrokecolor{currentstroke}%
\pgfsetdash{}{0pt}%
\pgfpathmoveto{\pgfqpoint{1.486050in}{1.042638in}}%
\pgfpathlineto{\pgfqpoint{1.701778in}{2.375902in}}%
\pgfpathlineto{\pgfqpoint{1.486050in}{1.042638in}}%
\pgfpathclose%
\pgfusepath{fill}%
\end{pgfscope}%
\begin{pgfscope}%
\pgfpathrectangle{\pgfqpoint{0.017500in}{0.017500in}}{\pgfqpoint{6.218282in}{3.580372in}} %
\pgfusepath{clip}%
\pgfsetbuttcap%
\pgfsetroundjoin%
\definecolor{currentfill}{rgb}{0.243520,0.319189,0.771672}%
\pgfsetfillcolor{currentfill}%
\pgfsetlinewidth{0.000000pt}%
\definecolor{currentstroke}{rgb}{0.000000,0.000000,0.000000}%
\pgfsetstrokecolor{currentstroke}%
\pgfsetdash{}{0pt}%
\pgfpathmoveto{\pgfqpoint{1.486050in}{1.042638in}}%
\pgfpathlineto{\pgfqpoint{1.701778in}{2.375902in}}%
\pgfpathlineto{\pgfqpoint{1.486050in}{1.042638in}}%
\pgfpathclose%
\pgfusepath{fill}%
\end{pgfscope}%
\begin{pgfscope}%
\pgfpathrectangle{\pgfqpoint{0.017500in}{0.017500in}}{\pgfqpoint{6.218282in}{3.580372in}} %
\pgfusepath{clip}%
\pgfsetbuttcap%
\pgfsetroundjoin%
\definecolor{currentfill}{rgb}{0.243520,0.319189,0.771672}%
\pgfsetfillcolor{currentfill}%
\pgfsetlinewidth{0.000000pt}%
\definecolor{currentstroke}{rgb}{0.000000,0.000000,0.000000}%
\pgfsetstrokecolor{currentstroke}%
\pgfsetdash{}{0pt}%
\pgfpathmoveto{\pgfqpoint{1.486050in}{1.042638in}}%
\pgfpathlineto{\pgfqpoint{1.701778in}{2.375902in}}%
\pgfpathlineto{\pgfqpoint{1.486050in}{1.042638in}}%
\pgfpathclose%
\pgfusepath{fill}%
\end{pgfscope}%
\begin{pgfscope}%
\pgfpathrectangle{\pgfqpoint{0.017500in}{0.017500in}}{\pgfqpoint{6.218282in}{3.580372in}} %
\pgfusepath{clip}%
\pgfsetbuttcap%
\pgfsetroundjoin%
\definecolor{currentfill}{rgb}{0.746838,0.140021,0.179996}%
\pgfsetfillcolor{currentfill}%
\pgfsetlinewidth{0.000000pt}%
\definecolor{currentstroke}{rgb}{0.000000,0.000000,0.000000}%
\pgfsetstrokecolor{currentstroke}%
\pgfsetdash{}{0pt}%
\pgfpathmoveto{\pgfqpoint{2.674254in}{2.688576in}}%
\pgfpathlineto{\pgfqpoint{2.932968in}{2.820668in}}%
\pgfpathlineto{\pgfqpoint{2.674254in}{2.688576in}}%
\pgfpathclose%
\pgfusepath{fill}%
\end{pgfscope}%
\begin{pgfscope}%
\pgfpathrectangle{\pgfqpoint{0.017500in}{0.017500in}}{\pgfqpoint{6.218282in}{3.580372in}} %
\pgfusepath{clip}%
\pgfsetbuttcap%
\pgfsetroundjoin%
\definecolor{currentfill}{rgb}{0.746838,0.140021,0.179996}%
\pgfsetfillcolor{currentfill}%
\pgfsetlinewidth{0.000000pt}%
\definecolor{currentstroke}{rgb}{0.000000,0.000000,0.000000}%
\pgfsetstrokecolor{currentstroke}%
\pgfsetdash{}{0pt}%
\pgfpathmoveto{\pgfqpoint{2.674254in}{2.688576in}}%
\pgfpathlineto{\pgfqpoint{2.932968in}{2.820668in}}%
\pgfpathlineto{\pgfqpoint{2.674254in}{2.688576in}}%
\pgfpathclose%
\pgfusepath{fill}%
\end{pgfscope}%
\begin{pgfscope}%
\pgfpathrectangle{\pgfqpoint{0.017500in}{0.017500in}}{\pgfqpoint{6.218282in}{3.580372in}} %
\pgfusepath{clip}%
\pgfsetbuttcap%
\pgfsetroundjoin%
\definecolor{currentfill}{rgb}{0.746838,0.140021,0.179996}%
\pgfsetfillcolor{currentfill}%
\pgfsetlinewidth{0.000000pt}%
\definecolor{currentstroke}{rgb}{0.000000,0.000000,0.000000}%
\pgfsetstrokecolor{currentstroke}%
\pgfsetdash{}{0pt}%
\pgfpathmoveto{\pgfqpoint{2.674254in}{2.688576in}}%
\pgfpathlineto{\pgfqpoint{2.932968in}{2.820668in}}%
\pgfpathlineto{\pgfqpoint{2.674254in}{2.688576in}}%
\pgfpathclose%
\pgfusepath{fill}%
\end{pgfscope}%
\begin{pgfscope}%
\pgfpathrectangle{\pgfqpoint{0.017500in}{0.017500in}}{\pgfqpoint{6.218282in}{3.580372in}} %
\pgfusepath{clip}%
\pgfsetbuttcap%
\pgfsetroundjoin%
\definecolor{currentfill}{rgb}{0.746838,0.140021,0.179996}%
\pgfsetfillcolor{currentfill}%
\pgfsetlinewidth{0.000000pt}%
\definecolor{currentstroke}{rgb}{0.000000,0.000000,0.000000}%
\pgfsetstrokecolor{currentstroke}%
\pgfsetdash{}{0pt}%
\pgfpathmoveto{\pgfqpoint{2.674254in}{2.688576in}}%
\pgfpathlineto{\pgfqpoint{2.932968in}{2.820668in}}%
\pgfpathlineto{\pgfqpoint{2.674254in}{2.688576in}}%
\pgfpathclose%
\pgfusepath{fill}%
\end{pgfscope}%
\begin{pgfscope}%
\pgfpathrectangle{\pgfqpoint{0.017500in}{0.017500in}}{\pgfqpoint{6.218282in}{3.580372in}} %
\pgfusepath{clip}%
\pgfsetbuttcap%
\pgfsetroundjoin%
\definecolor{currentfill}{rgb}{0.746838,0.140021,0.179996}%
\pgfsetfillcolor{currentfill}%
\pgfsetlinewidth{0.000000pt}%
\definecolor{currentstroke}{rgb}{0.000000,0.000000,0.000000}%
\pgfsetstrokecolor{currentstroke}%
\pgfsetdash{}{0pt}%
\pgfpathmoveto{\pgfqpoint{2.674254in}{2.688576in}}%
\pgfpathlineto{\pgfqpoint{2.932968in}{2.820668in}}%
\pgfpathlineto{\pgfqpoint{2.674254in}{2.688576in}}%
\pgfpathclose%
\pgfusepath{fill}%
\end{pgfscope}%
\begin{pgfscope}%
\pgfpathrectangle{\pgfqpoint{0.017500in}{0.017500in}}{\pgfqpoint{6.218282in}{3.580372in}} %
\pgfusepath{clip}%
\pgfsetbuttcap%
\pgfsetroundjoin%
\definecolor{currentfill}{rgb}{0.746838,0.140021,0.179996}%
\pgfsetfillcolor{currentfill}%
\pgfsetlinewidth{0.000000pt}%
\definecolor{currentstroke}{rgb}{0.000000,0.000000,0.000000}%
\pgfsetstrokecolor{currentstroke}%
\pgfsetdash{}{0pt}%
\pgfpathmoveto{\pgfqpoint{2.674254in}{2.688576in}}%
\pgfpathlineto{\pgfqpoint{2.932968in}{2.820668in}}%
\pgfpathlineto{\pgfqpoint{2.674254in}{2.688576in}}%
\pgfpathclose%
\pgfusepath{fill}%
\end{pgfscope}%
\begin{pgfscope}%
\pgfpathrectangle{\pgfqpoint{0.017500in}{0.017500in}}{\pgfqpoint{6.218282in}{3.580372in}} %
\pgfusepath{clip}%
\pgfsetbuttcap%
\pgfsetroundjoin%
\definecolor{currentfill}{rgb}{0.746838,0.140021,0.179996}%
\pgfsetfillcolor{currentfill}%
\pgfsetlinewidth{0.000000pt}%
\definecolor{currentstroke}{rgb}{0.000000,0.000000,0.000000}%
\pgfsetstrokecolor{currentstroke}%
\pgfsetdash{}{0pt}%
\pgfpathmoveto{\pgfqpoint{2.674254in}{2.688576in}}%
\pgfpathlineto{\pgfqpoint{2.932968in}{2.820668in}}%
\pgfpathlineto{\pgfqpoint{2.674254in}{2.688576in}}%
\pgfpathclose%
\pgfusepath{fill}%
\end{pgfscope}%
\begin{pgfscope}%
\pgfpathrectangle{\pgfqpoint{0.017500in}{0.017500in}}{\pgfqpoint{6.218282in}{3.580372in}} %
\pgfusepath{clip}%
\pgfsetbuttcap%
\pgfsetroundjoin%
\definecolor{currentfill}{rgb}{0.746838,0.140021,0.179996}%
\pgfsetfillcolor{currentfill}%
\pgfsetlinewidth{0.000000pt}%
\definecolor{currentstroke}{rgb}{0.000000,0.000000,0.000000}%
\pgfsetstrokecolor{currentstroke}%
\pgfsetdash{}{0pt}%
\pgfpathmoveto{\pgfqpoint{2.674254in}{2.688576in}}%
\pgfpathlineto{\pgfqpoint{2.932968in}{2.820668in}}%
\pgfpathlineto{\pgfqpoint{2.674254in}{2.688576in}}%
\pgfpathclose%
\pgfusepath{fill}%
\end{pgfscope}%
\begin{pgfscope}%
\pgfpathrectangle{\pgfqpoint{0.017500in}{0.017500in}}{\pgfqpoint{6.218282in}{3.580372in}} %
\pgfusepath{clip}%
\pgfsetbuttcap%
\pgfsetroundjoin%
\definecolor{currentfill}{rgb}{0.746838,0.140021,0.179996}%
\pgfsetfillcolor{currentfill}%
\pgfsetlinewidth{0.000000pt}%
\definecolor{currentstroke}{rgb}{0.000000,0.000000,0.000000}%
\pgfsetstrokecolor{currentstroke}%
\pgfsetdash{}{0pt}%
\pgfpathmoveto{\pgfqpoint{2.674254in}{2.688576in}}%
\pgfpathlineto{\pgfqpoint{2.932968in}{2.820668in}}%
\pgfpathlineto{\pgfqpoint{2.674254in}{2.688576in}}%
\pgfpathclose%
\pgfusepath{fill}%
\end{pgfscope}%
\begin{pgfscope}%
\pgfpathrectangle{\pgfqpoint{0.017500in}{0.017500in}}{\pgfqpoint{6.218282in}{3.580372in}} %
\pgfusepath{clip}%
\pgfsetbuttcap%
\pgfsetroundjoin%
\definecolor{currentfill}{rgb}{0.746838,0.140021,0.179996}%
\pgfsetfillcolor{currentfill}%
\pgfsetlinewidth{0.000000pt}%
\definecolor{currentstroke}{rgb}{0.000000,0.000000,0.000000}%
\pgfsetstrokecolor{currentstroke}%
\pgfsetdash{}{0pt}%
\pgfpathmoveto{\pgfqpoint{2.674254in}{2.688576in}}%
\pgfpathlineto{\pgfqpoint{2.932968in}{2.820668in}}%
\pgfpathlineto{\pgfqpoint{2.674254in}{2.688576in}}%
\pgfpathclose%
\pgfusepath{fill}%
\end{pgfscope}%
\begin{pgfscope}%
\pgfpathrectangle{\pgfqpoint{0.017500in}{0.017500in}}{\pgfqpoint{6.218282in}{3.580372in}} %
\pgfusepath{clip}%
\pgfsetbuttcap%
\pgfsetroundjoin%
\definecolor{currentfill}{rgb}{0.746838,0.140021,0.179996}%
\pgfsetfillcolor{currentfill}%
\pgfsetlinewidth{0.000000pt}%
\definecolor{currentstroke}{rgb}{0.000000,0.000000,0.000000}%
\pgfsetstrokecolor{currentstroke}%
\pgfsetdash{}{0pt}%
\pgfpathmoveto{\pgfqpoint{2.674254in}{2.688576in}}%
\pgfpathlineto{\pgfqpoint{2.932968in}{2.820668in}}%
\pgfpathlineto{\pgfqpoint{2.674254in}{2.688576in}}%
\pgfpathclose%
\pgfusepath{fill}%
\end{pgfscope}%
\begin{pgfscope}%
\pgfpathrectangle{\pgfqpoint{0.017500in}{0.017500in}}{\pgfqpoint{6.218282in}{3.580372in}} %
\pgfusepath{clip}%
\pgfsetbuttcap%
\pgfsetroundjoin%
\definecolor{currentfill}{rgb}{0.746838,0.140021,0.179996}%
\pgfsetfillcolor{currentfill}%
\pgfsetlinewidth{0.000000pt}%
\definecolor{currentstroke}{rgb}{0.000000,0.000000,0.000000}%
\pgfsetstrokecolor{currentstroke}%
\pgfsetdash{}{0pt}%
\pgfpathmoveto{\pgfqpoint{2.674254in}{2.688576in}}%
\pgfpathlineto{\pgfqpoint{2.932968in}{2.820668in}}%
\pgfpathlineto{\pgfqpoint{2.674254in}{2.688576in}}%
\pgfpathclose%
\pgfusepath{fill}%
\end{pgfscope}%
\begin{pgfscope}%
\pgfpathrectangle{\pgfqpoint{0.017500in}{0.017500in}}{\pgfqpoint{6.218282in}{3.580372in}} %
\pgfusepath{clip}%
\pgfsetbuttcap%
\pgfsetroundjoin%
\definecolor{currentfill}{rgb}{0.746838,0.140021,0.179996}%
\pgfsetfillcolor{currentfill}%
\pgfsetlinewidth{0.000000pt}%
\definecolor{currentstroke}{rgb}{0.000000,0.000000,0.000000}%
\pgfsetstrokecolor{currentstroke}%
\pgfsetdash{}{0pt}%
\pgfpathmoveto{\pgfqpoint{2.674254in}{2.688576in}}%
\pgfpathlineto{\pgfqpoint{2.932968in}{2.820668in}}%
\pgfpathlineto{\pgfqpoint{2.674254in}{2.688576in}}%
\pgfpathclose%
\pgfusepath{fill}%
\end{pgfscope}%
\begin{pgfscope}%
\pgfpathrectangle{\pgfqpoint{0.017500in}{0.017500in}}{\pgfqpoint{6.218282in}{3.580372in}} %
\pgfusepath{clip}%
\pgfsetbuttcap%
\pgfsetroundjoin%
\definecolor{currentfill}{rgb}{0.746838,0.140021,0.179996}%
\pgfsetfillcolor{currentfill}%
\pgfsetlinewidth{0.000000pt}%
\definecolor{currentstroke}{rgb}{0.000000,0.000000,0.000000}%
\pgfsetstrokecolor{currentstroke}%
\pgfsetdash{}{0pt}%
\pgfpathmoveto{\pgfqpoint{2.674254in}{2.688576in}}%
\pgfpathlineto{\pgfqpoint{2.932968in}{2.820668in}}%
\pgfpathlineto{\pgfqpoint{2.674254in}{2.688576in}}%
\pgfpathclose%
\pgfusepath{fill}%
\end{pgfscope}%
\begin{pgfscope}%
\pgfpathrectangle{\pgfqpoint{0.017500in}{0.017500in}}{\pgfqpoint{6.218282in}{3.580372in}} %
\pgfusepath{clip}%
\pgfsetbuttcap%
\pgfsetroundjoin%
\definecolor{currentfill}{rgb}{0.746838,0.140021,0.179996}%
\pgfsetfillcolor{currentfill}%
\pgfsetlinewidth{0.000000pt}%
\definecolor{currentstroke}{rgb}{0.000000,0.000000,0.000000}%
\pgfsetstrokecolor{currentstroke}%
\pgfsetdash{}{0pt}%
\pgfpathmoveto{\pgfqpoint{2.674254in}{2.688576in}}%
\pgfpathlineto{\pgfqpoint{2.932968in}{2.820668in}}%
\pgfpathlineto{\pgfqpoint{2.674254in}{2.688576in}}%
\pgfpathclose%
\pgfusepath{fill}%
\end{pgfscope}%
\begin{pgfscope}%
\pgfpathrectangle{\pgfqpoint{0.017500in}{0.017500in}}{\pgfqpoint{6.218282in}{3.580372in}} %
\pgfusepath{clip}%
\pgfsetbuttcap%
\pgfsetroundjoin%
\definecolor{currentfill}{rgb}{0.746838,0.140021,0.179996}%
\pgfsetfillcolor{currentfill}%
\pgfsetlinewidth{0.000000pt}%
\definecolor{currentstroke}{rgb}{0.000000,0.000000,0.000000}%
\pgfsetstrokecolor{currentstroke}%
\pgfsetdash{}{0pt}%
\pgfpathmoveto{\pgfqpoint{2.674254in}{2.688576in}}%
\pgfpathlineto{\pgfqpoint{2.932968in}{2.820668in}}%
\pgfpathlineto{\pgfqpoint{2.674254in}{2.688576in}}%
\pgfpathclose%
\pgfusepath{fill}%
\end{pgfscope}%
\begin{pgfscope}%
\pgfpathrectangle{\pgfqpoint{0.017500in}{0.017500in}}{\pgfqpoint{6.218282in}{3.580372in}} %
\pgfusepath{clip}%
\pgfsetbuttcap%
\pgfsetroundjoin%
\definecolor{currentfill}{rgb}{0.746838,0.140021,0.179996}%
\pgfsetfillcolor{currentfill}%
\pgfsetlinewidth{0.000000pt}%
\definecolor{currentstroke}{rgb}{0.000000,0.000000,0.000000}%
\pgfsetstrokecolor{currentstroke}%
\pgfsetdash{}{0pt}%
\pgfpathmoveto{\pgfqpoint{2.674254in}{2.688576in}}%
\pgfpathlineto{\pgfqpoint{2.932968in}{2.820668in}}%
\pgfpathlineto{\pgfqpoint{2.674254in}{2.688576in}}%
\pgfpathclose%
\pgfusepath{fill}%
\end{pgfscope}%
\begin{pgfscope}%
\pgfpathrectangle{\pgfqpoint{0.017500in}{0.017500in}}{\pgfqpoint{6.218282in}{3.580372in}} %
\pgfusepath{clip}%
\pgfsetbuttcap%
\pgfsetroundjoin%
\definecolor{currentfill}{rgb}{0.746838,0.140021,0.179996}%
\pgfsetfillcolor{currentfill}%
\pgfsetlinewidth{0.000000pt}%
\definecolor{currentstroke}{rgb}{0.000000,0.000000,0.000000}%
\pgfsetstrokecolor{currentstroke}%
\pgfsetdash{}{0pt}%
\pgfpathmoveto{\pgfqpoint{2.674254in}{2.688576in}}%
\pgfpathlineto{\pgfqpoint{2.932968in}{2.820668in}}%
\pgfpathlineto{\pgfqpoint{2.674254in}{2.688576in}}%
\pgfpathclose%
\pgfusepath{fill}%
\end{pgfscope}%
\begin{pgfscope}%
\pgfpathrectangle{\pgfqpoint{0.017500in}{0.017500in}}{\pgfqpoint{6.218282in}{3.580372in}} %
\pgfusepath{clip}%
\pgfsetbuttcap%
\pgfsetroundjoin%
\definecolor{currentfill}{rgb}{0.746838,0.140021,0.179996}%
\pgfsetfillcolor{currentfill}%
\pgfsetlinewidth{0.000000pt}%
\definecolor{currentstroke}{rgb}{0.000000,0.000000,0.000000}%
\pgfsetstrokecolor{currentstroke}%
\pgfsetdash{}{0pt}%
\pgfpathmoveto{\pgfqpoint{2.674254in}{2.688576in}}%
\pgfpathlineto{\pgfqpoint{2.932968in}{2.820668in}}%
\pgfpathlineto{\pgfqpoint{2.674254in}{2.688576in}}%
\pgfpathclose%
\pgfusepath{fill}%
\end{pgfscope}%
\begin{pgfscope}%
\pgfpathrectangle{\pgfqpoint{0.017500in}{0.017500in}}{\pgfqpoint{6.218282in}{3.580372in}} %
\pgfusepath{clip}%
\pgfsetbuttcap%
\pgfsetroundjoin%
\definecolor{currentfill}{rgb}{0.746838,0.140021,0.179996}%
\pgfsetfillcolor{currentfill}%
\pgfsetlinewidth{0.000000pt}%
\definecolor{currentstroke}{rgb}{0.000000,0.000000,0.000000}%
\pgfsetstrokecolor{currentstroke}%
\pgfsetdash{}{0pt}%
\pgfpathmoveto{\pgfqpoint{2.674254in}{2.688576in}}%
\pgfpathlineto{\pgfqpoint{2.932968in}{2.820668in}}%
\pgfpathlineto{\pgfqpoint{2.674254in}{2.688576in}}%
\pgfpathclose%
\pgfusepath{fill}%
\end{pgfscope}%
\begin{pgfscope}%
\pgfpathrectangle{\pgfqpoint{0.017500in}{0.017500in}}{\pgfqpoint{6.218282in}{3.580372in}} %
\pgfusepath{clip}%
\pgfsetbuttcap%
\pgfsetroundjoin%
\definecolor{currentfill}{rgb}{0.746838,0.140021,0.179996}%
\pgfsetfillcolor{currentfill}%
\pgfsetlinewidth{0.000000pt}%
\definecolor{currentstroke}{rgb}{0.000000,0.000000,0.000000}%
\pgfsetstrokecolor{currentstroke}%
\pgfsetdash{}{0pt}%
\pgfpathmoveto{\pgfqpoint{2.674254in}{2.688576in}}%
\pgfpathlineto{\pgfqpoint{2.932968in}{2.820668in}}%
\pgfpathlineto{\pgfqpoint{2.674254in}{2.688576in}}%
\pgfpathclose%
\pgfusepath{fill}%
\end{pgfscope}%
\begin{pgfscope}%
\pgfpathrectangle{\pgfqpoint{0.017500in}{0.017500in}}{\pgfqpoint{6.218282in}{3.580372in}} %
\pgfusepath{clip}%
\pgfsetbuttcap%
\pgfsetroundjoin%
\definecolor{currentfill}{rgb}{0.746838,0.140021,0.179996}%
\pgfsetfillcolor{currentfill}%
\pgfsetlinewidth{0.000000pt}%
\definecolor{currentstroke}{rgb}{0.000000,0.000000,0.000000}%
\pgfsetstrokecolor{currentstroke}%
\pgfsetdash{}{0pt}%
\pgfpathmoveto{\pgfqpoint{2.674254in}{2.688576in}}%
\pgfpathlineto{\pgfqpoint{2.932968in}{2.820668in}}%
\pgfpathlineto{\pgfqpoint{2.674254in}{2.688576in}}%
\pgfpathclose%
\pgfusepath{fill}%
\end{pgfscope}%
\begin{pgfscope}%
\pgfpathrectangle{\pgfqpoint{0.017500in}{0.017500in}}{\pgfqpoint{6.218282in}{3.580372in}} %
\pgfusepath{clip}%
\pgfsetbuttcap%
\pgfsetroundjoin%
\definecolor{currentfill}{rgb}{0.746838,0.140021,0.179996}%
\pgfsetfillcolor{currentfill}%
\pgfsetlinewidth{0.000000pt}%
\definecolor{currentstroke}{rgb}{0.000000,0.000000,0.000000}%
\pgfsetstrokecolor{currentstroke}%
\pgfsetdash{}{0pt}%
\pgfpathmoveto{\pgfqpoint{2.674254in}{2.688576in}}%
\pgfpathlineto{\pgfqpoint{2.932968in}{2.820668in}}%
\pgfpathlineto{\pgfqpoint{2.674254in}{2.688576in}}%
\pgfpathclose%
\pgfusepath{fill}%
\end{pgfscope}%
\begin{pgfscope}%
\pgfpathrectangle{\pgfqpoint{0.017500in}{0.017500in}}{\pgfqpoint{6.218282in}{3.580372in}} %
\pgfusepath{clip}%
\pgfsetbuttcap%
\pgfsetroundjoin%
\definecolor{currentfill}{rgb}{0.746838,0.140021,0.179996}%
\pgfsetfillcolor{currentfill}%
\pgfsetlinewidth{0.000000pt}%
\definecolor{currentstroke}{rgb}{0.000000,0.000000,0.000000}%
\pgfsetstrokecolor{currentstroke}%
\pgfsetdash{}{0pt}%
\pgfpathmoveto{\pgfqpoint{2.674254in}{2.688576in}}%
\pgfpathlineto{\pgfqpoint{2.932968in}{2.820668in}}%
\pgfpathlineto{\pgfqpoint{2.674254in}{2.688576in}}%
\pgfpathclose%
\pgfusepath{fill}%
\end{pgfscope}%
\begin{pgfscope}%
\pgfpathrectangle{\pgfqpoint{0.017500in}{0.017500in}}{\pgfqpoint{6.218282in}{3.580372in}} %
\pgfusepath{clip}%
\pgfsetbuttcap%
\pgfsetroundjoin%
\definecolor{currentfill}{rgb}{0.746838,0.140021,0.179996}%
\pgfsetfillcolor{currentfill}%
\pgfsetlinewidth{0.000000pt}%
\definecolor{currentstroke}{rgb}{0.000000,0.000000,0.000000}%
\pgfsetstrokecolor{currentstroke}%
\pgfsetdash{}{0pt}%
\pgfpathmoveto{\pgfqpoint{2.674254in}{2.688576in}}%
\pgfpathlineto{\pgfqpoint{2.932968in}{2.820668in}}%
\pgfpathlineto{\pgfqpoint{2.674254in}{2.688576in}}%
\pgfpathclose%
\pgfusepath{fill}%
\end{pgfscope}%
\begin{pgfscope}%
\pgfpathrectangle{\pgfqpoint{0.017500in}{0.017500in}}{\pgfqpoint{6.218282in}{3.580372in}} %
\pgfusepath{clip}%
\pgfsetbuttcap%
\pgfsetroundjoin%
\definecolor{currentfill}{rgb}{0.746838,0.140021,0.179996}%
\pgfsetfillcolor{currentfill}%
\pgfsetlinewidth{0.000000pt}%
\definecolor{currentstroke}{rgb}{0.000000,0.000000,0.000000}%
\pgfsetstrokecolor{currentstroke}%
\pgfsetdash{}{0pt}%
\pgfpathmoveto{\pgfqpoint{2.674254in}{2.688576in}}%
\pgfpathlineto{\pgfqpoint{2.932968in}{2.820668in}}%
\pgfpathlineto{\pgfqpoint{2.674254in}{2.688576in}}%
\pgfpathclose%
\pgfusepath{fill}%
\end{pgfscope}%
\begin{pgfscope}%
\pgfpathrectangle{\pgfqpoint{0.017500in}{0.017500in}}{\pgfqpoint{6.218282in}{3.580372in}} %
\pgfusepath{clip}%
\pgfsetbuttcap%
\pgfsetroundjoin%
\definecolor{currentfill}{rgb}{0.746838,0.140021,0.179996}%
\pgfsetfillcolor{currentfill}%
\pgfsetlinewidth{0.000000pt}%
\definecolor{currentstroke}{rgb}{0.000000,0.000000,0.000000}%
\pgfsetstrokecolor{currentstroke}%
\pgfsetdash{}{0pt}%
\pgfpathmoveto{\pgfqpoint{2.674254in}{2.688576in}}%
\pgfpathlineto{\pgfqpoint{2.932968in}{2.820668in}}%
\pgfpathlineto{\pgfqpoint{2.674254in}{2.688576in}}%
\pgfpathclose%
\pgfusepath{fill}%
\end{pgfscope}%
\begin{pgfscope}%
\pgfpathrectangle{\pgfqpoint{0.017500in}{0.017500in}}{\pgfqpoint{6.218282in}{3.580372in}} %
\pgfusepath{clip}%
\pgfsetbuttcap%
\pgfsetroundjoin%
\definecolor{currentfill}{rgb}{0.746838,0.140021,0.179996}%
\pgfsetfillcolor{currentfill}%
\pgfsetlinewidth{0.000000pt}%
\definecolor{currentstroke}{rgb}{0.000000,0.000000,0.000000}%
\pgfsetstrokecolor{currentstroke}%
\pgfsetdash{}{0pt}%
\pgfpathmoveto{\pgfqpoint{2.674254in}{2.688576in}}%
\pgfpathlineto{\pgfqpoint{2.932968in}{2.820668in}}%
\pgfpathlineto{\pgfqpoint{2.674254in}{2.688576in}}%
\pgfpathclose%
\pgfusepath{fill}%
\end{pgfscope}%
\begin{pgfscope}%
\pgfpathrectangle{\pgfqpoint{0.017500in}{0.017500in}}{\pgfqpoint{6.218282in}{3.580372in}} %
\pgfusepath{clip}%
\pgfsetbuttcap%
\pgfsetroundjoin%
\definecolor{currentfill}{rgb}{0.746838,0.140021,0.179996}%
\pgfsetfillcolor{currentfill}%
\pgfsetlinewidth{0.000000pt}%
\definecolor{currentstroke}{rgb}{0.000000,0.000000,0.000000}%
\pgfsetstrokecolor{currentstroke}%
\pgfsetdash{}{0pt}%
\pgfpathmoveto{\pgfqpoint{2.674254in}{2.688576in}}%
\pgfpathlineto{\pgfqpoint{2.932968in}{2.820668in}}%
\pgfpathlineto{\pgfqpoint{2.674254in}{2.688576in}}%
\pgfpathclose%
\pgfusepath{fill}%
\end{pgfscope}%
\begin{pgfscope}%
\pgfpathrectangle{\pgfqpoint{0.017500in}{0.017500in}}{\pgfqpoint{6.218282in}{3.580372in}} %
\pgfusepath{clip}%
\pgfsetbuttcap%
\pgfsetroundjoin%
\definecolor{currentfill}{rgb}{0.746838,0.140021,0.179996}%
\pgfsetfillcolor{currentfill}%
\pgfsetlinewidth{0.000000pt}%
\definecolor{currentstroke}{rgb}{0.000000,0.000000,0.000000}%
\pgfsetstrokecolor{currentstroke}%
\pgfsetdash{}{0pt}%
\pgfpathmoveto{\pgfqpoint{2.674254in}{2.688576in}}%
\pgfpathlineto{\pgfqpoint{2.932968in}{2.820668in}}%
\pgfpathlineto{\pgfqpoint{2.674254in}{2.688576in}}%
\pgfpathclose%
\pgfusepath{fill}%
\end{pgfscope}%
\begin{pgfscope}%
\pgfpathrectangle{\pgfqpoint{0.017500in}{0.017500in}}{\pgfqpoint{6.218282in}{3.580372in}} %
\pgfusepath{clip}%
\pgfsetbuttcap%
\pgfsetroundjoin%
\definecolor{currentfill}{rgb}{0.895882,0.849906,0.823499}%
\pgfsetfillcolor{currentfill}%
\pgfsetlinewidth{0.000000pt}%
\definecolor{currentstroke}{rgb}{0.000000,0.000000,0.000000}%
\pgfsetstrokecolor{currentstroke}%
\pgfsetdash{}{0pt}%
\pgfpathmoveto{\pgfqpoint{4.063701in}{2.813820in}}%
\pgfpathlineto{\pgfqpoint{2.789950in}{0.798346in}}%
\pgfpathlineto{\pgfqpoint{4.063701in}{2.813820in}}%
\pgfpathclose%
\pgfusepath{fill}%
\end{pgfscope}%
\begin{pgfscope}%
\pgfpathrectangle{\pgfqpoint{0.017500in}{0.017500in}}{\pgfqpoint{6.218282in}{3.580372in}} %
\pgfusepath{clip}%
\pgfsetbuttcap%
\pgfsetroundjoin%
\definecolor{currentfill}{rgb}{0.895882,0.849906,0.823499}%
\pgfsetfillcolor{currentfill}%
\pgfsetlinewidth{0.000000pt}%
\definecolor{currentstroke}{rgb}{0.000000,0.000000,0.000000}%
\pgfsetstrokecolor{currentstroke}%
\pgfsetdash{}{0pt}%
\pgfpathmoveto{\pgfqpoint{4.063701in}{2.813820in}}%
\pgfpathlineto{\pgfqpoint{2.789950in}{0.798346in}}%
\pgfpathlineto{\pgfqpoint{4.063701in}{2.813820in}}%
\pgfpathclose%
\pgfusepath{fill}%
\end{pgfscope}%
\begin{pgfscope}%
\pgfpathrectangle{\pgfqpoint{0.017500in}{0.017500in}}{\pgfqpoint{6.218282in}{3.580372in}} %
\pgfusepath{clip}%
\pgfsetbuttcap%
\pgfsetroundjoin%
\definecolor{currentfill}{rgb}{0.895882,0.849906,0.823499}%
\pgfsetfillcolor{currentfill}%
\pgfsetlinewidth{0.000000pt}%
\definecolor{currentstroke}{rgb}{0.000000,0.000000,0.000000}%
\pgfsetstrokecolor{currentstroke}%
\pgfsetdash{}{0pt}%
\pgfpathmoveto{\pgfqpoint{4.063701in}{2.813820in}}%
\pgfpathlineto{\pgfqpoint{2.789950in}{0.798346in}}%
\pgfpathlineto{\pgfqpoint{4.063701in}{2.813820in}}%
\pgfpathclose%
\pgfusepath{fill}%
\end{pgfscope}%
\begin{pgfscope}%
\pgfpathrectangle{\pgfqpoint{0.017500in}{0.017500in}}{\pgfqpoint{6.218282in}{3.580372in}} %
\pgfusepath{clip}%
\pgfsetbuttcap%
\pgfsetroundjoin%
\definecolor{currentfill}{rgb}{0.895882,0.849906,0.823499}%
\pgfsetfillcolor{currentfill}%
\pgfsetlinewidth{0.000000pt}%
\definecolor{currentstroke}{rgb}{0.000000,0.000000,0.000000}%
\pgfsetstrokecolor{currentstroke}%
\pgfsetdash{}{0pt}%
\pgfpathmoveto{\pgfqpoint{4.063701in}{2.813820in}}%
\pgfpathlineto{\pgfqpoint{2.789950in}{0.798346in}}%
\pgfpathlineto{\pgfqpoint{4.063701in}{2.813820in}}%
\pgfpathclose%
\pgfusepath{fill}%
\end{pgfscope}%
\begin{pgfscope}%
\pgfpathrectangle{\pgfqpoint{0.017500in}{0.017500in}}{\pgfqpoint{6.218282in}{3.580372in}} %
\pgfusepath{clip}%
\pgfsetbuttcap%
\pgfsetroundjoin%
\definecolor{currentfill}{rgb}{0.895882,0.849906,0.823499}%
\pgfsetfillcolor{currentfill}%
\pgfsetlinewidth{0.000000pt}%
\definecolor{currentstroke}{rgb}{0.000000,0.000000,0.000000}%
\pgfsetstrokecolor{currentstroke}%
\pgfsetdash{}{0pt}%
\pgfpathmoveto{\pgfqpoint{4.063701in}{2.813820in}}%
\pgfpathlineto{\pgfqpoint{2.789950in}{0.798346in}}%
\pgfpathlineto{\pgfqpoint{4.063701in}{2.813820in}}%
\pgfpathclose%
\pgfusepath{fill}%
\end{pgfscope}%
\begin{pgfscope}%
\pgfpathrectangle{\pgfqpoint{0.017500in}{0.017500in}}{\pgfqpoint{6.218282in}{3.580372in}} %
\pgfusepath{clip}%
\pgfsetbuttcap%
\pgfsetroundjoin%
\definecolor{currentfill}{rgb}{0.895882,0.849906,0.823499}%
\pgfsetfillcolor{currentfill}%
\pgfsetlinewidth{0.000000pt}%
\definecolor{currentstroke}{rgb}{0.000000,0.000000,0.000000}%
\pgfsetstrokecolor{currentstroke}%
\pgfsetdash{}{0pt}%
\pgfpathmoveto{\pgfqpoint{4.063701in}{2.813820in}}%
\pgfpathlineto{\pgfqpoint{2.789950in}{0.798346in}}%
\pgfpathlineto{\pgfqpoint{4.063701in}{2.813820in}}%
\pgfpathclose%
\pgfusepath{fill}%
\end{pgfscope}%
\begin{pgfscope}%
\pgfpathrectangle{\pgfqpoint{0.017500in}{0.017500in}}{\pgfqpoint{6.218282in}{3.580372in}} %
\pgfusepath{clip}%
\pgfsetbuttcap%
\pgfsetroundjoin%
\definecolor{currentfill}{rgb}{0.895882,0.849906,0.823499}%
\pgfsetfillcolor{currentfill}%
\pgfsetlinewidth{0.000000pt}%
\definecolor{currentstroke}{rgb}{0.000000,0.000000,0.000000}%
\pgfsetstrokecolor{currentstroke}%
\pgfsetdash{}{0pt}%
\pgfpathmoveto{\pgfqpoint{4.063701in}{2.813820in}}%
\pgfpathlineto{\pgfqpoint{2.789950in}{0.798346in}}%
\pgfpathlineto{\pgfqpoint{4.063701in}{2.813820in}}%
\pgfpathclose%
\pgfusepath{fill}%
\end{pgfscope}%
\begin{pgfscope}%
\pgfpathrectangle{\pgfqpoint{0.017500in}{0.017500in}}{\pgfqpoint{6.218282in}{3.580372in}} %
\pgfusepath{clip}%
\pgfsetbuttcap%
\pgfsetroundjoin%
\definecolor{currentfill}{rgb}{0.895882,0.849906,0.823499}%
\pgfsetfillcolor{currentfill}%
\pgfsetlinewidth{0.000000pt}%
\definecolor{currentstroke}{rgb}{0.000000,0.000000,0.000000}%
\pgfsetstrokecolor{currentstroke}%
\pgfsetdash{}{0pt}%
\pgfpathmoveto{\pgfqpoint{4.063701in}{2.813820in}}%
\pgfpathlineto{\pgfqpoint{2.789950in}{0.798346in}}%
\pgfpathlineto{\pgfqpoint{4.063701in}{2.813820in}}%
\pgfpathclose%
\pgfusepath{fill}%
\end{pgfscope}%
\begin{pgfscope}%
\pgfpathrectangle{\pgfqpoint{0.017500in}{0.017500in}}{\pgfqpoint{6.218282in}{3.580372in}} %
\pgfusepath{clip}%
\pgfsetbuttcap%
\pgfsetroundjoin%
\definecolor{currentfill}{rgb}{0.895882,0.849906,0.823499}%
\pgfsetfillcolor{currentfill}%
\pgfsetlinewidth{0.000000pt}%
\definecolor{currentstroke}{rgb}{0.000000,0.000000,0.000000}%
\pgfsetstrokecolor{currentstroke}%
\pgfsetdash{}{0pt}%
\pgfpathmoveto{\pgfqpoint{4.063701in}{2.813820in}}%
\pgfpathlineto{\pgfqpoint{2.789950in}{0.798346in}}%
\pgfpathlineto{\pgfqpoint{4.063701in}{2.813820in}}%
\pgfpathclose%
\pgfusepath{fill}%
\end{pgfscope}%
\begin{pgfscope}%
\pgfpathrectangle{\pgfqpoint{0.017500in}{0.017500in}}{\pgfqpoint{6.218282in}{3.580372in}} %
\pgfusepath{clip}%
\pgfsetbuttcap%
\pgfsetroundjoin%
\definecolor{currentfill}{rgb}{0.895882,0.849906,0.823499}%
\pgfsetfillcolor{currentfill}%
\pgfsetlinewidth{0.000000pt}%
\definecolor{currentstroke}{rgb}{0.000000,0.000000,0.000000}%
\pgfsetstrokecolor{currentstroke}%
\pgfsetdash{}{0pt}%
\pgfpathmoveto{\pgfqpoint{4.063701in}{2.813820in}}%
\pgfpathlineto{\pgfqpoint{2.789950in}{0.798346in}}%
\pgfpathlineto{\pgfqpoint{4.063701in}{2.813820in}}%
\pgfpathclose%
\pgfusepath{fill}%
\end{pgfscope}%
\begin{pgfscope}%
\pgfpathrectangle{\pgfqpoint{0.017500in}{0.017500in}}{\pgfqpoint{6.218282in}{3.580372in}} %
\pgfusepath{clip}%
\pgfsetbuttcap%
\pgfsetroundjoin%
\definecolor{currentfill}{rgb}{0.895882,0.849906,0.823499}%
\pgfsetfillcolor{currentfill}%
\pgfsetlinewidth{0.000000pt}%
\definecolor{currentstroke}{rgb}{0.000000,0.000000,0.000000}%
\pgfsetstrokecolor{currentstroke}%
\pgfsetdash{}{0pt}%
\pgfpathmoveto{\pgfqpoint{4.063701in}{2.813820in}}%
\pgfpathlineto{\pgfqpoint{2.789950in}{0.798346in}}%
\pgfpathlineto{\pgfqpoint{4.063701in}{2.813820in}}%
\pgfpathclose%
\pgfusepath{fill}%
\end{pgfscope}%
\begin{pgfscope}%
\pgfpathrectangle{\pgfqpoint{0.017500in}{0.017500in}}{\pgfqpoint{6.218282in}{3.580372in}} %
\pgfusepath{clip}%
\pgfsetbuttcap%
\pgfsetroundjoin%
\definecolor{currentfill}{rgb}{0.895882,0.849906,0.823499}%
\pgfsetfillcolor{currentfill}%
\pgfsetlinewidth{0.000000pt}%
\definecolor{currentstroke}{rgb}{0.000000,0.000000,0.000000}%
\pgfsetstrokecolor{currentstroke}%
\pgfsetdash{}{0pt}%
\pgfpathmoveto{\pgfqpoint{4.063701in}{2.813820in}}%
\pgfpathlineto{\pgfqpoint{2.789950in}{0.798346in}}%
\pgfpathlineto{\pgfqpoint{4.063701in}{2.813820in}}%
\pgfpathclose%
\pgfusepath{fill}%
\end{pgfscope}%
\begin{pgfscope}%
\pgfpathrectangle{\pgfqpoint{0.017500in}{0.017500in}}{\pgfqpoint{6.218282in}{3.580372in}} %
\pgfusepath{clip}%
\pgfsetbuttcap%
\pgfsetroundjoin%
\definecolor{currentfill}{rgb}{0.895882,0.849906,0.823499}%
\pgfsetfillcolor{currentfill}%
\pgfsetlinewidth{0.000000pt}%
\definecolor{currentstroke}{rgb}{0.000000,0.000000,0.000000}%
\pgfsetstrokecolor{currentstroke}%
\pgfsetdash{}{0pt}%
\pgfpathmoveto{\pgfqpoint{4.063701in}{2.813820in}}%
\pgfpathlineto{\pgfqpoint{2.789950in}{0.798346in}}%
\pgfpathlineto{\pgfqpoint{4.063701in}{2.813820in}}%
\pgfpathclose%
\pgfusepath{fill}%
\end{pgfscope}%
\begin{pgfscope}%
\pgfpathrectangle{\pgfqpoint{0.017500in}{0.017500in}}{\pgfqpoint{6.218282in}{3.580372in}} %
\pgfusepath{clip}%
\pgfsetbuttcap%
\pgfsetroundjoin%
\definecolor{currentfill}{rgb}{0.895882,0.849906,0.823499}%
\pgfsetfillcolor{currentfill}%
\pgfsetlinewidth{0.000000pt}%
\definecolor{currentstroke}{rgb}{0.000000,0.000000,0.000000}%
\pgfsetstrokecolor{currentstroke}%
\pgfsetdash{}{0pt}%
\pgfpathmoveto{\pgfqpoint{4.063701in}{2.813820in}}%
\pgfpathlineto{\pgfqpoint{2.789950in}{0.798346in}}%
\pgfpathlineto{\pgfqpoint{4.063701in}{2.813820in}}%
\pgfpathclose%
\pgfusepath{fill}%
\end{pgfscope}%
\begin{pgfscope}%
\pgfpathrectangle{\pgfqpoint{0.017500in}{0.017500in}}{\pgfqpoint{6.218282in}{3.580372in}} %
\pgfusepath{clip}%
\pgfsetbuttcap%
\pgfsetroundjoin%
\definecolor{currentfill}{rgb}{0.895882,0.849906,0.823499}%
\pgfsetfillcolor{currentfill}%
\pgfsetlinewidth{0.000000pt}%
\definecolor{currentstroke}{rgb}{0.000000,0.000000,0.000000}%
\pgfsetstrokecolor{currentstroke}%
\pgfsetdash{}{0pt}%
\pgfpathmoveto{\pgfqpoint{4.063701in}{2.813820in}}%
\pgfpathlineto{\pgfqpoint{2.789950in}{0.798346in}}%
\pgfpathlineto{\pgfqpoint{4.063701in}{2.813820in}}%
\pgfpathclose%
\pgfusepath{fill}%
\end{pgfscope}%
\begin{pgfscope}%
\pgfpathrectangle{\pgfqpoint{0.017500in}{0.017500in}}{\pgfqpoint{6.218282in}{3.580372in}} %
\pgfusepath{clip}%
\pgfsetbuttcap%
\pgfsetroundjoin%
\definecolor{currentfill}{rgb}{0.895882,0.849906,0.823499}%
\pgfsetfillcolor{currentfill}%
\pgfsetlinewidth{0.000000pt}%
\definecolor{currentstroke}{rgb}{0.000000,0.000000,0.000000}%
\pgfsetstrokecolor{currentstroke}%
\pgfsetdash{}{0pt}%
\pgfpathmoveto{\pgfqpoint{4.063701in}{2.813820in}}%
\pgfpathlineto{\pgfqpoint{2.789950in}{0.798346in}}%
\pgfpathlineto{\pgfqpoint{4.063701in}{2.813820in}}%
\pgfpathclose%
\pgfusepath{fill}%
\end{pgfscope}%
\begin{pgfscope}%
\pgfpathrectangle{\pgfqpoint{0.017500in}{0.017500in}}{\pgfqpoint{6.218282in}{3.580372in}} %
\pgfusepath{clip}%
\pgfsetbuttcap%
\pgfsetroundjoin%
\definecolor{currentfill}{rgb}{0.895882,0.849906,0.823499}%
\pgfsetfillcolor{currentfill}%
\pgfsetlinewidth{0.000000pt}%
\definecolor{currentstroke}{rgb}{0.000000,0.000000,0.000000}%
\pgfsetstrokecolor{currentstroke}%
\pgfsetdash{}{0pt}%
\pgfpathmoveto{\pgfqpoint{4.063701in}{2.813820in}}%
\pgfpathlineto{\pgfqpoint{2.789950in}{0.798346in}}%
\pgfpathlineto{\pgfqpoint{4.063701in}{2.813820in}}%
\pgfpathclose%
\pgfusepath{fill}%
\end{pgfscope}%
\begin{pgfscope}%
\pgfpathrectangle{\pgfqpoint{0.017500in}{0.017500in}}{\pgfqpoint{6.218282in}{3.580372in}} %
\pgfusepath{clip}%
\pgfsetbuttcap%
\pgfsetroundjoin%
\definecolor{currentfill}{rgb}{0.895882,0.849906,0.823499}%
\pgfsetfillcolor{currentfill}%
\pgfsetlinewidth{0.000000pt}%
\definecolor{currentstroke}{rgb}{0.000000,0.000000,0.000000}%
\pgfsetstrokecolor{currentstroke}%
\pgfsetdash{}{0pt}%
\pgfpathmoveto{\pgfqpoint{4.063701in}{2.813820in}}%
\pgfpathlineto{\pgfqpoint{2.789950in}{0.798346in}}%
\pgfpathlineto{\pgfqpoint{4.063701in}{2.813820in}}%
\pgfpathclose%
\pgfusepath{fill}%
\end{pgfscope}%
\begin{pgfscope}%
\pgfpathrectangle{\pgfqpoint{0.017500in}{0.017500in}}{\pgfqpoint{6.218282in}{3.580372in}} %
\pgfusepath{clip}%
\pgfsetbuttcap%
\pgfsetroundjoin%
\definecolor{currentfill}{rgb}{0.895882,0.849906,0.823499}%
\pgfsetfillcolor{currentfill}%
\pgfsetlinewidth{0.000000pt}%
\definecolor{currentstroke}{rgb}{0.000000,0.000000,0.000000}%
\pgfsetstrokecolor{currentstroke}%
\pgfsetdash{}{0pt}%
\pgfpathmoveto{\pgfqpoint{4.063701in}{2.813820in}}%
\pgfpathlineto{\pgfqpoint{2.789950in}{0.798346in}}%
\pgfpathlineto{\pgfqpoint{4.063701in}{2.813820in}}%
\pgfpathclose%
\pgfusepath{fill}%
\end{pgfscope}%
\begin{pgfscope}%
\pgfpathrectangle{\pgfqpoint{0.017500in}{0.017500in}}{\pgfqpoint{6.218282in}{3.580372in}} %
\pgfusepath{clip}%
\pgfsetbuttcap%
\pgfsetroundjoin%
\definecolor{currentfill}{rgb}{0.895882,0.849906,0.823499}%
\pgfsetfillcolor{currentfill}%
\pgfsetlinewidth{0.000000pt}%
\definecolor{currentstroke}{rgb}{0.000000,0.000000,0.000000}%
\pgfsetstrokecolor{currentstroke}%
\pgfsetdash{}{0pt}%
\pgfpathmoveto{\pgfqpoint{4.063701in}{2.813820in}}%
\pgfpathlineto{\pgfqpoint{2.789950in}{0.798346in}}%
\pgfpathlineto{\pgfqpoint{4.063701in}{2.813820in}}%
\pgfpathclose%
\pgfusepath{fill}%
\end{pgfscope}%
\begin{pgfscope}%
\pgfpathrectangle{\pgfqpoint{0.017500in}{0.017500in}}{\pgfqpoint{6.218282in}{3.580372in}} %
\pgfusepath{clip}%
\pgfsetbuttcap%
\pgfsetroundjoin%
\definecolor{currentfill}{rgb}{0.895882,0.849906,0.823499}%
\pgfsetfillcolor{currentfill}%
\pgfsetlinewidth{0.000000pt}%
\definecolor{currentstroke}{rgb}{0.000000,0.000000,0.000000}%
\pgfsetstrokecolor{currentstroke}%
\pgfsetdash{}{0pt}%
\pgfpathmoveto{\pgfqpoint{4.063701in}{2.813820in}}%
\pgfpathlineto{\pgfqpoint{2.789950in}{0.798346in}}%
\pgfpathlineto{\pgfqpoint{4.063701in}{2.813820in}}%
\pgfpathclose%
\pgfusepath{fill}%
\end{pgfscope}%
\begin{pgfscope}%
\pgfpathrectangle{\pgfqpoint{0.017500in}{0.017500in}}{\pgfqpoint{6.218282in}{3.580372in}} %
\pgfusepath{clip}%
\pgfsetbuttcap%
\pgfsetroundjoin%
\definecolor{currentfill}{rgb}{0.895882,0.849906,0.823499}%
\pgfsetfillcolor{currentfill}%
\pgfsetlinewidth{0.000000pt}%
\definecolor{currentstroke}{rgb}{0.000000,0.000000,0.000000}%
\pgfsetstrokecolor{currentstroke}%
\pgfsetdash{}{0pt}%
\pgfpathmoveto{\pgfqpoint{4.063701in}{2.813820in}}%
\pgfpathlineto{\pgfqpoint{2.789950in}{0.798346in}}%
\pgfpathlineto{\pgfqpoint{4.063701in}{2.813820in}}%
\pgfpathclose%
\pgfusepath{fill}%
\end{pgfscope}%
\begin{pgfscope}%
\pgfpathrectangle{\pgfqpoint{0.017500in}{0.017500in}}{\pgfqpoint{6.218282in}{3.580372in}} %
\pgfusepath{clip}%
\pgfsetbuttcap%
\pgfsetroundjoin%
\definecolor{currentfill}{rgb}{0.895882,0.849906,0.823499}%
\pgfsetfillcolor{currentfill}%
\pgfsetlinewidth{0.000000pt}%
\definecolor{currentstroke}{rgb}{0.000000,0.000000,0.000000}%
\pgfsetstrokecolor{currentstroke}%
\pgfsetdash{}{0pt}%
\pgfpathmoveto{\pgfqpoint{4.063701in}{2.813820in}}%
\pgfpathlineto{\pgfqpoint{2.789950in}{0.798346in}}%
\pgfpathlineto{\pgfqpoint{4.063701in}{2.813820in}}%
\pgfpathclose%
\pgfusepath{fill}%
\end{pgfscope}%
\begin{pgfscope}%
\pgfpathrectangle{\pgfqpoint{0.017500in}{0.017500in}}{\pgfqpoint{6.218282in}{3.580372in}} %
\pgfusepath{clip}%
\pgfsetbuttcap%
\pgfsetroundjoin%
\definecolor{currentfill}{rgb}{0.895882,0.849906,0.823499}%
\pgfsetfillcolor{currentfill}%
\pgfsetlinewidth{0.000000pt}%
\definecolor{currentstroke}{rgb}{0.000000,0.000000,0.000000}%
\pgfsetstrokecolor{currentstroke}%
\pgfsetdash{}{0pt}%
\pgfpathmoveto{\pgfqpoint{4.063701in}{2.813820in}}%
\pgfpathlineto{\pgfqpoint{2.789950in}{0.798346in}}%
\pgfpathlineto{\pgfqpoint{4.063701in}{2.813820in}}%
\pgfpathclose%
\pgfusepath{fill}%
\end{pgfscope}%
\begin{pgfscope}%
\pgfpathrectangle{\pgfqpoint{0.017500in}{0.017500in}}{\pgfqpoint{6.218282in}{3.580372in}} %
\pgfusepath{clip}%
\pgfsetbuttcap%
\pgfsetroundjoin%
\definecolor{currentfill}{rgb}{0.895882,0.849906,0.823499}%
\pgfsetfillcolor{currentfill}%
\pgfsetlinewidth{0.000000pt}%
\definecolor{currentstroke}{rgb}{0.000000,0.000000,0.000000}%
\pgfsetstrokecolor{currentstroke}%
\pgfsetdash{}{0pt}%
\pgfpathmoveto{\pgfqpoint{4.063701in}{2.813820in}}%
\pgfpathlineto{\pgfqpoint{2.789950in}{0.798346in}}%
\pgfpathlineto{\pgfqpoint{4.063701in}{2.813820in}}%
\pgfpathclose%
\pgfusepath{fill}%
\end{pgfscope}%
\begin{pgfscope}%
\pgfpathrectangle{\pgfqpoint{0.017500in}{0.017500in}}{\pgfqpoint{6.218282in}{3.580372in}} %
\pgfusepath{clip}%
\pgfsetbuttcap%
\pgfsetroundjoin%
\definecolor{currentfill}{rgb}{0.746838,0.140021,0.179996}%
\pgfsetfillcolor{currentfill}%
\pgfsetlinewidth{0.000000pt}%
\definecolor{currentstroke}{rgb}{0.000000,0.000000,0.000000}%
\pgfsetstrokecolor{currentstroke}%
\pgfsetdash{}{0pt}%
\pgfpathmoveto{\pgfqpoint{2.674254in}{2.688576in}}%
\pgfpathlineto{\pgfqpoint{3.116513in}{2.603812in}}%
\pgfpathlineto{\pgfqpoint{3.372093in}{2.761754in}}%
\pgfpathlineto{\pgfqpoint{2.932968in}{2.820668in}}%
\pgfpathlineto{\pgfqpoint{2.674254in}{2.688576in}}%
\pgfpathclose%
\pgfusepath{fill}%
\end{pgfscope}%
\begin{pgfscope}%
\pgfpathrectangle{\pgfqpoint{0.017500in}{0.017500in}}{\pgfqpoint{6.218282in}{3.580372in}} %
\pgfusepath{clip}%
\pgfsetbuttcap%
\pgfsetroundjoin%
\definecolor{currentfill}{rgb}{0.746838,0.140021,0.179996}%
\pgfsetfillcolor{currentfill}%
\pgfsetlinewidth{0.000000pt}%
\definecolor{currentstroke}{rgb}{0.000000,0.000000,0.000000}%
\pgfsetstrokecolor{currentstroke}%
\pgfsetdash{}{0pt}%
\pgfpathmoveto{\pgfqpoint{2.674254in}{2.688576in}}%
\pgfpathlineto{\pgfqpoint{3.116513in}{2.603812in}}%
\pgfpathlineto{\pgfqpoint{3.372093in}{2.761754in}}%
\pgfpathlineto{\pgfqpoint{2.932968in}{2.820668in}}%
\pgfpathlineto{\pgfqpoint{2.674254in}{2.688576in}}%
\pgfpathclose%
\pgfusepath{fill}%
\end{pgfscope}%
\begin{pgfscope}%
\pgfpathrectangle{\pgfqpoint{0.017500in}{0.017500in}}{\pgfqpoint{6.218282in}{3.580372in}} %
\pgfusepath{clip}%
\pgfsetbuttcap%
\pgfsetroundjoin%
\definecolor{currentfill}{rgb}{0.746838,0.140021,0.179996}%
\pgfsetfillcolor{currentfill}%
\pgfsetlinewidth{0.000000pt}%
\definecolor{currentstroke}{rgb}{0.000000,0.000000,0.000000}%
\pgfsetstrokecolor{currentstroke}%
\pgfsetdash{}{0pt}%
\pgfpathmoveto{\pgfqpoint{2.674254in}{2.688576in}}%
\pgfpathlineto{\pgfqpoint{3.116513in}{2.603812in}}%
\pgfpathlineto{\pgfqpoint{3.372093in}{2.761754in}}%
\pgfpathlineto{\pgfqpoint{2.932968in}{2.820668in}}%
\pgfpathlineto{\pgfqpoint{2.674254in}{2.688576in}}%
\pgfpathclose%
\pgfusepath{fill}%
\end{pgfscope}%
\begin{pgfscope}%
\pgfpathrectangle{\pgfqpoint{0.017500in}{0.017500in}}{\pgfqpoint{6.218282in}{3.580372in}} %
\pgfusepath{clip}%
\pgfsetbuttcap%
\pgfsetroundjoin%
\definecolor{currentfill}{rgb}{0.746838,0.140021,0.179996}%
\pgfsetfillcolor{currentfill}%
\pgfsetlinewidth{0.000000pt}%
\definecolor{currentstroke}{rgb}{0.000000,0.000000,0.000000}%
\pgfsetstrokecolor{currentstroke}%
\pgfsetdash{}{0pt}%
\pgfpathmoveto{\pgfqpoint{2.674254in}{2.688576in}}%
\pgfpathlineto{\pgfqpoint{3.116513in}{2.603812in}}%
\pgfpathlineto{\pgfqpoint{3.372093in}{2.761754in}}%
\pgfpathlineto{\pgfqpoint{2.932968in}{2.820668in}}%
\pgfpathlineto{\pgfqpoint{2.674254in}{2.688576in}}%
\pgfpathclose%
\pgfusepath{fill}%
\end{pgfscope}%
\begin{pgfscope}%
\pgfpathrectangle{\pgfqpoint{0.017500in}{0.017500in}}{\pgfqpoint{6.218282in}{3.580372in}} %
\pgfusepath{clip}%
\pgfsetbuttcap%
\pgfsetroundjoin%
\definecolor{currentfill}{rgb}{0.746838,0.140021,0.179996}%
\pgfsetfillcolor{currentfill}%
\pgfsetlinewidth{0.000000pt}%
\definecolor{currentstroke}{rgb}{0.000000,0.000000,0.000000}%
\pgfsetstrokecolor{currentstroke}%
\pgfsetdash{}{0pt}%
\pgfpathmoveto{\pgfqpoint{2.674254in}{2.688576in}}%
\pgfpathlineto{\pgfqpoint{3.116513in}{2.603812in}}%
\pgfpathlineto{\pgfqpoint{3.372093in}{2.761754in}}%
\pgfpathlineto{\pgfqpoint{2.932968in}{2.820668in}}%
\pgfpathlineto{\pgfqpoint{2.674254in}{2.688576in}}%
\pgfpathclose%
\pgfusepath{fill}%
\end{pgfscope}%
\begin{pgfscope}%
\pgfpathrectangle{\pgfqpoint{0.017500in}{0.017500in}}{\pgfqpoint{6.218282in}{3.580372in}} %
\pgfusepath{clip}%
\pgfsetbuttcap%
\pgfsetroundjoin%
\definecolor{currentfill}{rgb}{0.746838,0.140021,0.179996}%
\pgfsetfillcolor{currentfill}%
\pgfsetlinewidth{0.000000pt}%
\definecolor{currentstroke}{rgb}{0.000000,0.000000,0.000000}%
\pgfsetstrokecolor{currentstroke}%
\pgfsetdash{}{0pt}%
\pgfpathmoveto{\pgfqpoint{2.674254in}{2.688576in}}%
\pgfpathlineto{\pgfqpoint{3.116513in}{2.603812in}}%
\pgfpathlineto{\pgfqpoint{3.372093in}{2.761754in}}%
\pgfpathlineto{\pgfqpoint{2.932968in}{2.820668in}}%
\pgfpathlineto{\pgfqpoint{2.674254in}{2.688576in}}%
\pgfpathclose%
\pgfusepath{fill}%
\end{pgfscope}%
\begin{pgfscope}%
\pgfpathrectangle{\pgfqpoint{0.017500in}{0.017500in}}{\pgfqpoint{6.218282in}{3.580372in}} %
\pgfusepath{clip}%
\pgfsetbuttcap%
\pgfsetroundjoin%
\definecolor{currentfill}{rgb}{0.735077,0.104460,0.171492}%
\pgfsetfillcolor{currentfill}%
\pgfsetlinewidth{0.000000pt}%
\definecolor{currentstroke}{rgb}{0.000000,0.000000,0.000000}%
\pgfsetstrokecolor{currentstroke}%
\pgfsetdash{}{0pt}%
\pgfpathmoveto{\pgfqpoint{3.816789in}{2.677291in}}%
\pgfpathlineto{\pgfqpoint{4.063701in}{2.813820in}}%
\pgfpathlineto{\pgfqpoint{3.816789in}{2.677291in}}%
\pgfpathclose%
\pgfusepath{fill}%
\end{pgfscope}%
\begin{pgfscope}%
\pgfpathrectangle{\pgfqpoint{0.017500in}{0.017500in}}{\pgfqpoint{6.218282in}{3.580372in}} %
\pgfusepath{clip}%
\pgfsetbuttcap%
\pgfsetroundjoin%
\definecolor{currentfill}{rgb}{0.735077,0.104460,0.171492}%
\pgfsetfillcolor{currentfill}%
\pgfsetlinewidth{0.000000pt}%
\definecolor{currentstroke}{rgb}{0.000000,0.000000,0.000000}%
\pgfsetstrokecolor{currentstroke}%
\pgfsetdash{}{0pt}%
\pgfpathmoveto{\pgfqpoint{3.816789in}{2.677291in}}%
\pgfpathlineto{\pgfqpoint{4.063701in}{2.813820in}}%
\pgfpathlineto{\pgfqpoint{3.816789in}{2.677291in}}%
\pgfpathclose%
\pgfusepath{fill}%
\end{pgfscope}%
\begin{pgfscope}%
\pgfpathrectangle{\pgfqpoint{0.017500in}{0.017500in}}{\pgfqpoint{6.218282in}{3.580372in}} %
\pgfusepath{clip}%
\pgfsetbuttcap%
\pgfsetroundjoin%
\definecolor{currentfill}{rgb}{0.735077,0.104460,0.171492}%
\pgfsetfillcolor{currentfill}%
\pgfsetlinewidth{0.000000pt}%
\definecolor{currentstroke}{rgb}{0.000000,0.000000,0.000000}%
\pgfsetstrokecolor{currentstroke}%
\pgfsetdash{}{0pt}%
\pgfpathmoveto{\pgfqpoint{3.816789in}{2.677291in}}%
\pgfpathlineto{\pgfqpoint{4.063701in}{2.813820in}}%
\pgfpathlineto{\pgfqpoint{3.816789in}{2.677291in}}%
\pgfpathclose%
\pgfusepath{fill}%
\end{pgfscope}%
\begin{pgfscope}%
\pgfpathrectangle{\pgfqpoint{0.017500in}{0.017500in}}{\pgfqpoint{6.218282in}{3.580372in}} %
\pgfusepath{clip}%
\pgfsetbuttcap%
\pgfsetroundjoin%
\definecolor{currentfill}{rgb}{0.735077,0.104460,0.171492}%
\pgfsetfillcolor{currentfill}%
\pgfsetlinewidth{0.000000pt}%
\definecolor{currentstroke}{rgb}{0.000000,0.000000,0.000000}%
\pgfsetstrokecolor{currentstroke}%
\pgfsetdash{}{0pt}%
\pgfpathmoveto{\pgfqpoint{3.816789in}{2.677291in}}%
\pgfpathlineto{\pgfqpoint{4.063701in}{2.813820in}}%
\pgfpathlineto{\pgfqpoint{3.816789in}{2.677291in}}%
\pgfpathclose%
\pgfusepath{fill}%
\end{pgfscope}%
\begin{pgfscope}%
\pgfpathrectangle{\pgfqpoint{0.017500in}{0.017500in}}{\pgfqpoint{6.218282in}{3.580372in}} %
\pgfusepath{clip}%
\pgfsetbuttcap%
\pgfsetroundjoin%
\definecolor{currentfill}{rgb}{0.735077,0.104460,0.171492}%
\pgfsetfillcolor{currentfill}%
\pgfsetlinewidth{0.000000pt}%
\definecolor{currentstroke}{rgb}{0.000000,0.000000,0.000000}%
\pgfsetstrokecolor{currentstroke}%
\pgfsetdash{}{0pt}%
\pgfpathmoveto{\pgfqpoint{3.816789in}{2.677291in}}%
\pgfpathlineto{\pgfqpoint{4.063701in}{2.813820in}}%
\pgfpathlineto{\pgfqpoint{3.816789in}{2.677291in}}%
\pgfpathclose%
\pgfusepath{fill}%
\end{pgfscope}%
\begin{pgfscope}%
\pgfpathrectangle{\pgfqpoint{0.017500in}{0.017500in}}{\pgfqpoint{6.218282in}{3.580372in}} %
\pgfusepath{clip}%
\pgfsetbuttcap%
\pgfsetroundjoin%
\definecolor{currentfill}{rgb}{0.735077,0.104460,0.171492}%
\pgfsetfillcolor{currentfill}%
\pgfsetlinewidth{0.000000pt}%
\definecolor{currentstroke}{rgb}{0.000000,0.000000,0.000000}%
\pgfsetstrokecolor{currentstroke}%
\pgfsetdash{}{0pt}%
\pgfpathmoveto{\pgfqpoint{3.816789in}{2.677291in}}%
\pgfpathlineto{\pgfqpoint{4.063701in}{2.813820in}}%
\pgfpathlineto{\pgfqpoint{3.816789in}{2.677291in}}%
\pgfpathclose%
\pgfusepath{fill}%
\end{pgfscope}%
\begin{pgfscope}%
\pgfpathrectangle{\pgfqpoint{0.017500in}{0.017500in}}{\pgfqpoint{6.218282in}{3.580372in}} %
\pgfusepath{clip}%
\pgfsetbuttcap%
\pgfsetroundjoin%
\definecolor{currentfill}{rgb}{0.735077,0.104460,0.171492}%
\pgfsetfillcolor{currentfill}%
\pgfsetlinewidth{0.000000pt}%
\definecolor{currentstroke}{rgb}{0.000000,0.000000,0.000000}%
\pgfsetstrokecolor{currentstroke}%
\pgfsetdash{}{0pt}%
\pgfpathmoveto{\pgfqpoint{3.816789in}{2.677291in}}%
\pgfpathlineto{\pgfqpoint{4.063701in}{2.813820in}}%
\pgfpathlineto{\pgfqpoint{3.816789in}{2.677291in}}%
\pgfpathclose%
\pgfusepath{fill}%
\end{pgfscope}%
\begin{pgfscope}%
\pgfpathrectangle{\pgfqpoint{0.017500in}{0.017500in}}{\pgfqpoint{6.218282in}{3.580372in}} %
\pgfusepath{clip}%
\pgfsetbuttcap%
\pgfsetroundjoin%
\definecolor{currentfill}{rgb}{0.735077,0.104460,0.171492}%
\pgfsetfillcolor{currentfill}%
\pgfsetlinewidth{0.000000pt}%
\definecolor{currentstroke}{rgb}{0.000000,0.000000,0.000000}%
\pgfsetstrokecolor{currentstroke}%
\pgfsetdash{}{0pt}%
\pgfpathmoveto{\pgfqpoint{3.816789in}{2.677291in}}%
\pgfpathlineto{\pgfqpoint{4.063701in}{2.813820in}}%
\pgfpathlineto{\pgfqpoint{3.816789in}{2.677291in}}%
\pgfpathclose%
\pgfusepath{fill}%
\end{pgfscope}%
\begin{pgfscope}%
\pgfpathrectangle{\pgfqpoint{0.017500in}{0.017500in}}{\pgfqpoint{6.218282in}{3.580372in}} %
\pgfusepath{clip}%
\pgfsetbuttcap%
\pgfsetroundjoin%
\definecolor{currentfill}{rgb}{0.735077,0.104460,0.171492}%
\pgfsetfillcolor{currentfill}%
\pgfsetlinewidth{0.000000pt}%
\definecolor{currentstroke}{rgb}{0.000000,0.000000,0.000000}%
\pgfsetstrokecolor{currentstroke}%
\pgfsetdash{}{0pt}%
\pgfpathmoveto{\pgfqpoint{3.816789in}{2.677291in}}%
\pgfpathlineto{\pgfqpoint{4.063701in}{2.813820in}}%
\pgfpathlineto{\pgfqpoint{3.816789in}{2.677291in}}%
\pgfpathclose%
\pgfusepath{fill}%
\end{pgfscope}%
\begin{pgfscope}%
\pgfpathrectangle{\pgfqpoint{0.017500in}{0.017500in}}{\pgfqpoint{6.218282in}{3.580372in}} %
\pgfusepath{clip}%
\pgfsetbuttcap%
\pgfsetroundjoin%
\definecolor{currentfill}{rgb}{0.735077,0.104460,0.171492}%
\pgfsetfillcolor{currentfill}%
\pgfsetlinewidth{0.000000pt}%
\definecolor{currentstroke}{rgb}{0.000000,0.000000,0.000000}%
\pgfsetstrokecolor{currentstroke}%
\pgfsetdash{}{0pt}%
\pgfpathmoveto{\pgfqpoint{3.816789in}{2.677291in}}%
\pgfpathlineto{\pgfqpoint{4.063701in}{2.813820in}}%
\pgfpathlineto{\pgfqpoint{3.816789in}{2.677291in}}%
\pgfpathclose%
\pgfusepath{fill}%
\end{pgfscope}%
\begin{pgfscope}%
\pgfpathrectangle{\pgfqpoint{0.017500in}{0.017500in}}{\pgfqpoint{6.218282in}{3.580372in}} %
\pgfusepath{clip}%
\pgfsetbuttcap%
\pgfsetroundjoin%
\definecolor{currentfill}{rgb}{0.735077,0.104460,0.171492}%
\pgfsetfillcolor{currentfill}%
\pgfsetlinewidth{0.000000pt}%
\definecolor{currentstroke}{rgb}{0.000000,0.000000,0.000000}%
\pgfsetstrokecolor{currentstroke}%
\pgfsetdash{}{0pt}%
\pgfpathmoveto{\pgfqpoint{3.816789in}{2.677291in}}%
\pgfpathlineto{\pgfqpoint{4.063701in}{2.813820in}}%
\pgfpathlineto{\pgfqpoint{3.816789in}{2.677291in}}%
\pgfpathclose%
\pgfusepath{fill}%
\end{pgfscope}%
\begin{pgfscope}%
\pgfpathrectangle{\pgfqpoint{0.017500in}{0.017500in}}{\pgfqpoint{6.218282in}{3.580372in}} %
\pgfusepath{clip}%
\pgfsetbuttcap%
\pgfsetroundjoin%
\definecolor{currentfill}{rgb}{0.735077,0.104460,0.171492}%
\pgfsetfillcolor{currentfill}%
\pgfsetlinewidth{0.000000pt}%
\definecolor{currentstroke}{rgb}{0.000000,0.000000,0.000000}%
\pgfsetstrokecolor{currentstroke}%
\pgfsetdash{}{0pt}%
\pgfpathmoveto{\pgfqpoint{3.816789in}{2.677291in}}%
\pgfpathlineto{\pgfqpoint{4.063701in}{2.813820in}}%
\pgfpathlineto{\pgfqpoint{3.816789in}{2.677291in}}%
\pgfpathclose%
\pgfusepath{fill}%
\end{pgfscope}%
\begin{pgfscope}%
\pgfpathrectangle{\pgfqpoint{0.017500in}{0.017500in}}{\pgfqpoint{6.218282in}{3.580372in}} %
\pgfusepath{clip}%
\pgfsetbuttcap%
\pgfsetroundjoin%
\definecolor{currentfill}{rgb}{0.735077,0.104460,0.171492}%
\pgfsetfillcolor{currentfill}%
\pgfsetlinewidth{0.000000pt}%
\definecolor{currentstroke}{rgb}{0.000000,0.000000,0.000000}%
\pgfsetstrokecolor{currentstroke}%
\pgfsetdash{}{0pt}%
\pgfpathmoveto{\pgfqpoint{3.816789in}{2.677291in}}%
\pgfpathlineto{\pgfqpoint{4.063701in}{2.813820in}}%
\pgfpathlineto{\pgfqpoint{3.816789in}{2.677291in}}%
\pgfpathclose%
\pgfusepath{fill}%
\end{pgfscope}%
\begin{pgfscope}%
\pgfpathrectangle{\pgfqpoint{0.017500in}{0.017500in}}{\pgfqpoint{6.218282in}{3.580372in}} %
\pgfusepath{clip}%
\pgfsetbuttcap%
\pgfsetroundjoin%
\definecolor{currentfill}{rgb}{0.735077,0.104460,0.171492}%
\pgfsetfillcolor{currentfill}%
\pgfsetlinewidth{0.000000pt}%
\definecolor{currentstroke}{rgb}{0.000000,0.000000,0.000000}%
\pgfsetstrokecolor{currentstroke}%
\pgfsetdash{}{0pt}%
\pgfpathmoveto{\pgfqpoint{3.816789in}{2.677291in}}%
\pgfpathlineto{\pgfqpoint{4.063701in}{2.813820in}}%
\pgfpathlineto{\pgfqpoint{3.816789in}{2.677291in}}%
\pgfpathclose%
\pgfusepath{fill}%
\end{pgfscope}%
\begin{pgfscope}%
\pgfpathrectangle{\pgfqpoint{0.017500in}{0.017500in}}{\pgfqpoint{6.218282in}{3.580372in}} %
\pgfusepath{clip}%
\pgfsetbuttcap%
\pgfsetroundjoin%
\definecolor{currentfill}{rgb}{0.735077,0.104460,0.171492}%
\pgfsetfillcolor{currentfill}%
\pgfsetlinewidth{0.000000pt}%
\definecolor{currentstroke}{rgb}{0.000000,0.000000,0.000000}%
\pgfsetstrokecolor{currentstroke}%
\pgfsetdash{}{0pt}%
\pgfpathmoveto{\pgfqpoint{3.816789in}{2.677291in}}%
\pgfpathlineto{\pgfqpoint{4.063701in}{2.813820in}}%
\pgfpathlineto{\pgfqpoint{3.816789in}{2.677291in}}%
\pgfpathclose%
\pgfusepath{fill}%
\end{pgfscope}%
\begin{pgfscope}%
\pgfpathrectangle{\pgfqpoint{0.017500in}{0.017500in}}{\pgfqpoint{6.218282in}{3.580372in}} %
\pgfusepath{clip}%
\pgfsetbuttcap%
\pgfsetroundjoin%
\definecolor{currentfill}{rgb}{0.735077,0.104460,0.171492}%
\pgfsetfillcolor{currentfill}%
\pgfsetlinewidth{0.000000pt}%
\definecolor{currentstroke}{rgb}{0.000000,0.000000,0.000000}%
\pgfsetstrokecolor{currentstroke}%
\pgfsetdash{}{0pt}%
\pgfpathmoveto{\pgfqpoint{3.816789in}{2.677291in}}%
\pgfpathlineto{\pgfqpoint{4.063701in}{2.813820in}}%
\pgfpathlineto{\pgfqpoint{3.816789in}{2.677291in}}%
\pgfpathclose%
\pgfusepath{fill}%
\end{pgfscope}%
\begin{pgfscope}%
\pgfpathrectangle{\pgfqpoint{0.017500in}{0.017500in}}{\pgfqpoint{6.218282in}{3.580372in}} %
\pgfusepath{clip}%
\pgfsetbuttcap%
\pgfsetroundjoin%
\definecolor{currentfill}{rgb}{0.735077,0.104460,0.171492}%
\pgfsetfillcolor{currentfill}%
\pgfsetlinewidth{0.000000pt}%
\definecolor{currentstroke}{rgb}{0.000000,0.000000,0.000000}%
\pgfsetstrokecolor{currentstroke}%
\pgfsetdash{}{0pt}%
\pgfpathmoveto{\pgfqpoint{3.816789in}{2.677291in}}%
\pgfpathlineto{\pgfqpoint{4.063701in}{2.813820in}}%
\pgfpathlineto{\pgfqpoint{3.816789in}{2.677291in}}%
\pgfpathclose%
\pgfusepath{fill}%
\end{pgfscope}%
\begin{pgfscope}%
\pgfpathrectangle{\pgfqpoint{0.017500in}{0.017500in}}{\pgfqpoint{6.218282in}{3.580372in}} %
\pgfusepath{clip}%
\pgfsetbuttcap%
\pgfsetroundjoin%
\definecolor{currentfill}{rgb}{0.735077,0.104460,0.171492}%
\pgfsetfillcolor{currentfill}%
\pgfsetlinewidth{0.000000pt}%
\definecolor{currentstroke}{rgb}{0.000000,0.000000,0.000000}%
\pgfsetstrokecolor{currentstroke}%
\pgfsetdash{}{0pt}%
\pgfpathmoveto{\pgfqpoint{3.816789in}{2.677291in}}%
\pgfpathlineto{\pgfqpoint{4.063701in}{2.813820in}}%
\pgfpathlineto{\pgfqpoint{3.816789in}{2.677291in}}%
\pgfpathclose%
\pgfusepath{fill}%
\end{pgfscope}%
\begin{pgfscope}%
\pgfpathrectangle{\pgfqpoint{0.017500in}{0.017500in}}{\pgfqpoint{6.218282in}{3.580372in}} %
\pgfusepath{clip}%
\pgfsetbuttcap%
\pgfsetroundjoin%
\definecolor{currentfill}{rgb}{0.735077,0.104460,0.171492}%
\pgfsetfillcolor{currentfill}%
\pgfsetlinewidth{0.000000pt}%
\definecolor{currentstroke}{rgb}{0.000000,0.000000,0.000000}%
\pgfsetstrokecolor{currentstroke}%
\pgfsetdash{}{0pt}%
\pgfpathmoveto{\pgfqpoint{3.816789in}{2.677291in}}%
\pgfpathlineto{\pgfqpoint{4.063701in}{2.813820in}}%
\pgfpathlineto{\pgfqpoint{3.816789in}{2.677291in}}%
\pgfpathclose%
\pgfusepath{fill}%
\end{pgfscope}%
\begin{pgfscope}%
\pgfpathrectangle{\pgfqpoint{0.017500in}{0.017500in}}{\pgfqpoint{6.218282in}{3.580372in}} %
\pgfusepath{clip}%
\pgfsetbuttcap%
\pgfsetroundjoin%
\definecolor{currentfill}{rgb}{0.735077,0.104460,0.171492}%
\pgfsetfillcolor{currentfill}%
\pgfsetlinewidth{0.000000pt}%
\definecolor{currentstroke}{rgb}{0.000000,0.000000,0.000000}%
\pgfsetstrokecolor{currentstroke}%
\pgfsetdash{}{0pt}%
\pgfpathmoveto{\pgfqpoint{3.816789in}{2.677291in}}%
\pgfpathlineto{\pgfqpoint{4.063701in}{2.813820in}}%
\pgfpathlineto{\pgfqpoint{3.816789in}{2.677291in}}%
\pgfpathclose%
\pgfusepath{fill}%
\end{pgfscope}%
\begin{pgfscope}%
\pgfpathrectangle{\pgfqpoint{0.017500in}{0.017500in}}{\pgfqpoint{6.218282in}{3.580372in}} %
\pgfusepath{clip}%
\pgfsetbuttcap%
\pgfsetroundjoin%
\definecolor{currentfill}{rgb}{0.735077,0.104460,0.171492}%
\pgfsetfillcolor{currentfill}%
\pgfsetlinewidth{0.000000pt}%
\definecolor{currentstroke}{rgb}{0.000000,0.000000,0.000000}%
\pgfsetstrokecolor{currentstroke}%
\pgfsetdash{}{0pt}%
\pgfpathmoveto{\pgfqpoint{3.816789in}{2.677291in}}%
\pgfpathlineto{\pgfqpoint{4.063701in}{2.813820in}}%
\pgfpathlineto{\pgfqpoint{3.816789in}{2.677291in}}%
\pgfpathclose%
\pgfusepath{fill}%
\end{pgfscope}%
\begin{pgfscope}%
\pgfpathrectangle{\pgfqpoint{0.017500in}{0.017500in}}{\pgfqpoint{6.218282in}{3.580372in}} %
\pgfusepath{clip}%
\pgfsetbuttcap%
\pgfsetroundjoin%
\definecolor{currentfill}{rgb}{0.735077,0.104460,0.171492}%
\pgfsetfillcolor{currentfill}%
\pgfsetlinewidth{0.000000pt}%
\definecolor{currentstroke}{rgb}{0.000000,0.000000,0.000000}%
\pgfsetstrokecolor{currentstroke}%
\pgfsetdash{}{0pt}%
\pgfpathmoveto{\pgfqpoint{3.816789in}{2.677291in}}%
\pgfpathlineto{\pgfqpoint{4.063701in}{2.813820in}}%
\pgfpathlineto{\pgfqpoint{3.816789in}{2.677291in}}%
\pgfpathclose%
\pgfusepath{fill}%
\end{pgfscope}%
\begin{pgfscope}%
\pgfpathrectangle{\pgfqpoint{0.017500in}{0.017500in}}{\pgfqpoint{6.218282in}{3.580372in}} %
\pgfusepath{clip}%
\pgfsetbuttcap%
\pgfsetroundjoin%
\definecolor{currentfill}{rgb}{0.735077,0.104460,0.171492}%
\pgfsetfillcolor{currentfill}%
\pgfsetlinewidth{0.000000pt}%
\definecolor{currentstroke}{rgb}{0.000000,0.000000,0.000000}%
\pgfsetstrokecolor{currentstroke}%
\pgfsetdash{}{0pt}%
\pgfpathmoveto{\pgfqpoint{3.816789in}{2.677291in}}%
\pgfpathlineto{\pgfqpoint{4.063701in}{2.813820in}}%
\pgfpathlineto{\pgfqpoint{3.816789in}{2.677291in}}%
\pgfpathclose%
\pgfusepath{fill}%
\end{pgfscope}%
\begin{pgfscope}%
\pgfpathrectangle{\pgfqpoint{0.017500in}{0.017500in}}{\pgfqpoint{6.218282in}{3.580372in}} %
\pgfusepath{clip}%
\pgfsetbuttcap%
\pgfsetroundjoin%
\definecolor{currentfill}{rgb}{0.735077,0.104460,0.171492}%
\pgfsetfillcolor{currentfill}%
\pgfsetlinewidth{0.000000pt}%
\definecolor{currentstroke}{rgb}{0.000000,0.000000,0.000000}%
\pgfsetstrokecolor{currentstroke}%
\pgfsetdash{}{0pt}%
\pgfpathmoveto{\pgfqpoint{3.816789in}{2.677291in}}%
\pgfpathlineto{\pgfqpoint{4.063701in}{2.813820in}}%
\pgfpathlineto{\pgfqpoint{3.816789in}{2.677291in}}%
\pgfpathclose%
\pgfusepath{fill}%
\end{pgfscope}%
\begin{pgfscope}%
\pgfpathrectangle{\pgfqpoint{0.017500in}{0.017500in}}{\pgfqpoint{6.218282in}{3.580372in}} %
\pgfusepath{clip}%
\pgfsetbuttcap%
\pgfsetroundjoin%
\definecolor{currentfill}{rgb}{0.735077,0.104460,0.171492}%
\pgfsetfillcolor{currentfill}%
\pgfsetlinewidth{0.000000pt}%
\definecolor{currentstroke}{rgb}{0.000000,0.000000,0.000000}%
\pgfsetstrokecolor{currentstroke}%
\pgfsetdash{}{0pt}%
\pgfpathmoveto{\pgfqpoint{3.816789in}{2.677291in}}%
\pgfpathlineto{\pgfqpoint{4.063701in}{2.813820in}}%
\pgfpathlineto{\pgfqpoint{3.816789in}{2.677291in}}%
\pgfpathclose%
\pgfusepath{fill}%
\end{pgfscope}%
\begin{pgfscope}%
\pgfpathrectangle{\pgfqpoint{0.017500in}{0.017500in}}{\pgfqpoint{6.218282in}{3.580372in}} %
\pgfusepath{clip}%
\pgfsetbuttcap%
\pgfsetroundjoin%
\definecolor{currentfill}{rgb}{0.735077,0.104460,0.171492}%
\pgfsetfillcolor{currentfill}%
\pgfsetlinewidth{0.000000pt}%
\definecolor{currentstroke}{rgb}{0.000000,0.000000,0.000000}%
\pgfsetstrokecolor{currentstroke}%
\pgfsetdash{}{0pt}%
\pgfpathmoveto{\pgfqpoint{3.816789in}{2.677291in}}%
\pgfpathlineto{\pgfqpoint{4.063701in}{2.813820in}}%
\pgfpathlineto{\pgfqpoint{3.816789in}{2.677291in}}%
\pgfpathclose%
\pgfusepath{fill}%
\end{pgfscope}%
\begin{pgfscope}%
\pgfpathrectangle{\pgfqpoint{0.017500in}{0.017500in}}{\pgfqpoint{6.218282in}{3.580372in}} %
\pgfusepath{clip}%
\pgfsetbuttcap%
\pgfsetroundjoin%
\definecolor{currentfill}{rgb}{0.735077,0.104460,0.171492}%
\pgfsetfillcolor{currentfill}%
\pgfsetlinewidth{0.000000pt}%
\definecolor{currentstroke}{rgb}{0.000000,0.000000,0.000000}%
\pgfsetstrokecolor{currentstroke}%
\pgfsetdash{}{0pt}%
\pgfpathmoveto{\pgfqpoint{3.816789in}{2.677291in}}%
\pgfpathlineto{\pgfqpoint{4.063701in}{2.813820in}}%
\pgfpathlineto{\pgfqpoint{3.816789in}{2.677291in}}%
\pgfpathclose%
\pgfusepath{fill}%
\end{pgfscope}%
\begin{pgfscope}%
\pgfpathrectangle{\pgfqpoint{0.017500in}{0.017500in}}{\pgfqpoint{6.218282in}{3.580372in}} %
\pgfusepath{clip}%
\pgfsetbuttcap%
\pgfsetroundjoin%
\definecolor{currentfill}{rgb}{0.735077,0.104460,0.171492}%
\pgfsetfillcolor{currentfill}%
\pgfsetlinewidth{0.000000pt}%
\definecolor{currentstroke}{rgb}{0.000000,0.000000,0.000000}%
\pgfsetstrokecolor{currentstroke}%
\pgfsetdash{}{0pt}%
\pgfpathmoveto{\pgfqpoint{3.816789in}{2.677291in}}%
\pgfpathlineto{\pgfqpoint{4.063701in}{2.813820in}}%
\pgfpathlineto{\pgfqpoint{3.816789in}{2.677291in}}%
\pgfpathclose%
\pgfusepath{fill}%
\end{pgfscope}%
\begin{pgfscope}%
\pgfpathrectangle{\pgfqpoint{0.017500in}{0.017500in}}{\pgfqpoint{6.218282in}{3.580372in}} %
\pgfusepath{clip}%
\pgfsetbuttcap%
\pgfsetroundjoin%
\definecolor{currentfill}{rgb}{0.735077,0.104460,0.171492}%
\pgfsetfillcolor{currentfill}%
\pgfsetlinewidth{0.000000pt}%
\definecolor{currentstroke}{rgb}{0.000000,0.000000,0.000000}%
\pgfsetstrokecolor{currentstroke}%
\pgfsetdash{}{0pt}%
\pgfpathmoveto{\pgfqpoint{3.816789in}{2.677291in}}%
\pgfpathlineto{\pgfqpoint{4.063701in}{2.813820in}}%
\pgfpathlineto{\pgfqpoint{3.816789in}{2.677291in}}%
\pgfpathclose%
\pgfusepath{fill}%
\end{pgfscope}%
\begin{pgfscope}%
\pgfpathrectangle{\pgfqpoint{0.017500in}{0.017500in}}{\pgfqpoint{6.218282in}{3.580372in}} %
\pgfusepath{clip}%
\pgfsetbuttcap%
\pgfsetroundjoin%
\definecolor{currentfill}{rgb}{0.735077,0.104460,0.171492}%
\pgfsetfillcolor{currentfill}%
\pgfsetlinewidth{0.000000pt}%
\definecolor{currentstroke}{rgb}{0.000000,0.000000,0.000000}%
\pgfsetstrokecolor{currentstroke}%
\pgfsetdash{}{0pt}%
\pgfpathmoveto{\pgfqpoint{3.816789in}{2.677291in}}%
\pgfpathlineto{\pgfqpoint{4.063701in}{2.813820in}}%
\pgfpathlineto{\pgfqpoint{3.816789in}{2.677291in}}%
\pgfpathclose%
\pgfusepath{fill}%
\end{pgfscope}%
\begin{pgfscope}%
\pgfpathrectangle{\pgfqpoint{0.017500in}{0.017500in}}{\pgfqpoint{6.218282in}{3.580372in}} %
\pgfusepath{clip}%
\pgfsetbuttcap%
\pgfsetroundjoin%
\definecolor{currentfill}{rgb}{0.805723,0.259813,0.230562}%
\pgfsetfillcolor{currentfill}%
\pgfsetlinewidth{0.000000pt}%
\definecolor{currentstroke}{rgb}{0.000000,0.000000,0.000000}%
\pgfsetstrokecolor{currentstroke}%
\pgfsetdash{}{0pt}%
\pgfpathmoveto{\pgfqpoint{1.971633in}{2.586781in}}%
\pgfpathlineto{\pgfqpoint{2.237924in}{2.752812in}}%
\pgfpathlineto{\pgfqpoint{1.971633in}{2.586781in}}%
\pgfpathclose%
\pgfusepath{fill}%
\end{pgfscope}%
\begin{pgfscope}%
\pgfpathrectangle{\pgfqpoint{0.017500in}{0.017500in}}{\pgfqpoint{6.218282in}{3.580372in}} %
\pgfusepath{clip}%
\pgfsetbuttcap%
\pgfsetroundjoin%
\definecolor{currentfill}{rgb}{0.805723,0.259813,0.230562}%
\pgfsetfillcolor{currentfill}%
\pgfsetlinewidth{0.000000pt}%
\definecolor{currentstroke}{rgb}{0.000000,0.000000,0.000000}%
\pgfsetstrokecolor{currentstroke}%
\pgfsetdash{}{0pt}%
\pgfpathmoveto{\pgfqpoint{1.971633in}{2.586781in}}%
\pgfpathlineto{\pgfqpoint{2.237924in}{2.752812in}}%
\pgfpathlineto{\pgfqpoint{1.971633in}{2.586781in}}%
\pgfpathclose%
\pgfusepath{fill}%
\end{pgfscope}%
\begin{pgfscope}%
\pgfpathrectangle{\pgfqpoint{0.017500in}{0.017500in}}{\pgfqpoint{6.218282in}{3.580372in}} %
\pgfusepath{clip}%
\pgfsetbuttcap%
\pgfsetroundjoin%
\definecolor{currentfill}{rgb}{0.805723,0.259813,0.230562}%
\pgfsetfillcolor{currentfill}%
\pgfsetlinewidth{0.000000pt}%
\definecolor{currentstroke}{rgb}{0.000000,0.000000,0.000000}%
\pgfsetstrokecolor{currentstroke}%
\pgfsetdash{}{0pt}%
\pgfpathmoveto{\pgfqpoint{1.971633in}{2.586781in}}%
\pgfpathlineto{\pgfqpoint{2.237924in}{2.752812in}}%
\pgfpathlineto{\pgfqpoint{1.971633in}{2.586781in}}%
\pgfpathclose%
\pgfusepath{fill}%
\end{pgfscope}%
\begin{pgfscope}%
\pgfpathrectangle{\pgfqpoint{0.017500in}{0.017500in}}{\pgfqpoint{6.218282in}{3.580372in}} %
\pgfusepath{clip}%
\pgfsetbuttcap%
\pgfsetroundjoin%
\definecolor{currentfill}{rgb}{0.805723,0.259813,0.230562}%
\pgfsetfillcolor{currentfill}%
\pgfsetlinewidth{0.000000pt}%
\definecolor{currentstroke}{rgb}{0.000000,0.000000,0.000000}%
\pgfsetstrokecolor{currentstroke}%
\pgfsetdash{}{0pt}%
\pgfpathmoveto{\pgfqpoint{1.971633in}{2.586781in}}%
\pgfpathlineto{\pgfqpoint{2.237924in}{2.752812in}}%
\pgfpathlineto{\pgfqpoint{1.971633in}{2.586781in}}%
\pgfpathclose%
\pgfusepath{fill}%
\end{pgfscope}%
\begin{pgfscope}%
\pgfpathrectangle{\pgfqpoint{0.017500in}{0.017500in}}{\pgfqpoint{6.218282in}{3.580372in}} %
\pgfusepath{clip}%
\pgfsetbuttcap%
\pgfsetroundjoin%
\definecolor{currentfill}{rgb}{0.805723,0.259813,0.230562}%
\pgfsetfillcolor{currentfill}%
\pgfsetlinewidth{0.000000pt}%
\definecolor{currentstroke}{rgb}{0.000000,0.000000,0.000000}%
\pgfsetstrokecolor{currentstroke}%
\pgfsetdash{}{0pt}%
\pgfpathmoveto{\pgfqpoint{1.971633in}{2.586781in}}%
\pgfpathlineto{\pgfqpoint{2.237924in}{2.752812in}}%
\pgfpathlineto{\pgfqpoint{1.971633in}{2.586781in}}%
\pgfpathclose%
\pgfusepath{fill}%
\end{pgfscope}%
\begin{pgfscope}%
\pgfpathrectangle{\pgfqpoint{0.017500in}{0.017500in}}{\pgfqpoint{6.218282in}{3.580372in}} %
\pgfusepath{clip}%
\pgfsetbuttcap%
\pgfsetroundjoin%
\definecolor{currentfill}{rgb}{0.805723,0.259813,0.230562}%
\pgfsetfillcolor{currentfill}%
\pgfsetlinewidth{0.000000pt}%
\definecolor{currentstroke}{rgb}{0.000000,0.000000,0.000000}%
\pgfsetstrokecolor{currentstroke}%
\pgfsetdash{}{0pt}%
\pgfpathmoveto{\pgfqpoint{1.971633in}{2.586781in}}%
\pgfpathlineto{\pgfqpoint{2.237924in}{2.752812in}}%
\pgfpathlineto{\pgfqpoint{1.971633in}{2.586781in}}%
\pgfpathclose%
\pgfusepath{fill}%
\end{pgfscope}%
\begin{pgfscope}%
\pgfpathrectangle{\pgfqpoint{0.017500in}{0.017500in}}{\pgfqpoint{6.218282in}{3.580372in}} %
\pgfusepath{clip}%
\pgfsetbuttcap%
\pgfsetroundjoin%
\definecolor{currentfill}{rgb}{0.805723,0.259813,0.230562}%
\pgfsetfillcolor{currentfill}%
\pgfsetlinewidth{0.000000pt}%
\definecolor{currentstroke}{rgb}{0.000000,0.000000,0.000000}%
\pgfsetstrokecolor{currentstroke}%
\pgfsetdash{}{0pt}%
\pgfpathmoveto{\pgfqpoint{1.971633in}{2.586781in}}%
\pgfpathlineto{\pgfqpoint{2.237924in}{2.752812in}}%
\pgfpathlineto{\pgfqpoint{1.971633in}{2.586781in}}%
\pgfpathclose%
\pgfusepath{fill}%
\end{pgfscope}%
\begin{pgfscope}%
\pgfpathrectangle{\pgfqpoint{0.017500in}{0.017500in}}{\pgfqpoint{6.218282in}{3.580372in}} %
\pgfusepath{clip}%
\pgfsetbuttcap%
\pgfsetroundjoin%
\definecolor{currentfill}{rgb}{0.805723,0.259813,0.230562}%
\pgfsetfillcolor{currentfill}%
\pgfsetlinewidth{0.000000pt}%
\definecolor{currentstroke}{rgb}{0.000000,0.000000,0.000000}%
\pgfsetstrokecolor{currentstroke}%
\pgfsetdash{}{0pt}%
\pgfpathmoveto{\pgfqpoint{1.971633in}{2.586781in}}%
\pgfpathlineto{\pgfqpoint{2.237924in}{2.752812in}}%
\pgfpathlineto{\pgfqpoint{1.971633in}{2.586781in}}%
\pgfpathclose%
\pgfusepath{fill}%
\end{pgfscope}%
\begin{pgfscope}%
\pgfpathrectangle{\pgfqpoint{0.017500in}{0.017500in}}{\pgfqpoint{6.218282in}{3.580372in}} %
\pgfusepath{clip}%
\pgfsetbuttcap%
\pgfsetroundjoin%
\definecolor{currentfill}{rgb}{0.805723,0.259813,0.230562}%
\pgfsetfillcolor{currentfill}%
\pgfsetlinewidth{0.000000pt}%
\definecolor{currentstroke}{rgb}{0.000000,0.000000,0.000000}%
\pgfsetstrokecolor{currentstroke}%
\pgfsetdash{}{0pt}%
\pgfpathmoveto{\pgfqpoint{1.971633in}{2.586781in}}%
\pgfpathlineto{\pgfqpoint{2.237924in}{2.752812in}}%
\pgfpathlineto{\pgfqpoint{1.971633in}{2.586781in}}%
\pgfpathclose%
\pgfusepath{fill}%
\end{pgfscope}%
\begin{pgfscope}%
\pgfpathrectangle{\pgfqpoint{0.017500in}{0.017500in}}{\pgfqpoint{6.218282in}{3.580372in}} %
\pgfusepath{clip}%
\pgfsetbuttcap%
\pgfsetroundjoin%
\definecolor{currentfill}{rgb}{0.805723,0.259813,0.230562}%
\pgfsetfillcolor{currentfill}%
\pgfsetlinewidth{0.000000pt}%
\definecolor{currentstroke}{rgb}{0.000000,0.000000,0.000000}%
\pgfsetstrokecolor{currentstroke}%
\pgfsetdash{}{0pt}%
\pgfpathmoveto{\pgfqpoint{1.971633in}{2.586781in}}%
\pgfpathlineto{\pgfqpoint{2.237924in}{2.752812in}}%
\pgfpathlineto{\pgfqpoint{1.971633in}{2.586781in}}%
\pgfpathclose%
\pgfusepath{fill}%
\end{pgfscope}%
\begin{pgfscope}%
\pgfpathrectangle{\pgfqpoint{0.017500in}{0.017500in}}{\pgfqpoint{6.218282in}{3.580372in}} %
\pgfusepath{clip}%
\pgfsetbuttcap%
\pgfsetroundjoin%
\definecolor{currentfill}{rgb}{0.805723,0.259813,0.230562}%
\pgfsetfillcolor{currentfill}%
\pgfsetlinewidth{0.000000pt}%
\definecolor{currentstroke}{rgb}{0.000000,0.000000,0.000000}%
\pgfsetstrokecolor{currentstroke}%
\pgfsetdash{}{0pt}%
\pgfpathmoveto{\pgfqpoint{1.971633in}{2.586781in}}%
\pgfpathlineto{\pgfqpoint{2.237924in}{2.752812in}}%
\pgfpathlineto{\pgfqpoint{1.971633in}{2.586781in}}%
\pgfpathclose%
\pgfusepath{fill}%
\end{pgfscope}%
\begin{pgfscope}%
\pgfpathrectangle{\pgfqpoint{0.017500in}{0.017500in}}{\pgfqpoint{6.218282in}{3.580372in}} %
\pgfusepath{clip}%
\pgfsetbuttcap%
\pgfsetroundjoin%
\definecolor{currentfill}{rgb}{0.805723,0.259813,0.230562}%
\pgfsetfillcolor{currentfill}%
\pgfsetlinewidth{0.000000pt}%
\definecolor{currentstroke}{rgb}{0.000000,0.000000,0.000000}%
\pgfsetstrokecolor{currentstroke}%
\pgfsetdash{}{0pt}%
\pgfpathmoveto{\pgfqpoint{1.971633in}{2.586781in}}%
\pgfpathlineto{\pgfqpoint{2.237924in}{2.752812in}}%
\pgfpathlineto{\pgfqpoint{1.971633in}{2.586781in}}%
\pgfpathclose%
\pgfusepath{fill}%
\end{pgfscope}%
\begin{pgfscope}%
\pgfpathrectangle{\pgfqpoint{0.017500in}{0.017500in}}{\pgfqpoint{6.218282in}{3.580372in}} %
\pgfusepath{clip}%
\pgfsetbuttcap%
\pgfsetroundjoin%
\definecolor{currentfill}{rgb}{0.805723,0.259813,0.230562}%
\pgfsetfillcolor{currentfill}%
\pgfsetlinewidth{0.000000pt}%
\definecolor{currentstroke}{rgb}{0.000000,0.000000,0.000000}%
\pgfsetstrokecolor{currentstroke}%
\pgfsetdash{}{0pt}%
\pgfpathmoveto{\pgfqpoint{1.971633in}{2.586781in}}%
\pgfpathlineto{\pgfqpoint{2.237924in}{2.752812in}}%
\pgfpathlineto{\pgfqpoint{1.971633in}{2.586781in}}%
\pgfpathclose%
\pgfusepath{fill}%
\end{pgfscope}%
\begin{pgfscope}%
\pgfpathrectangle{\pgfqpoint{0.017500in}{0.017500in}}{\pgfqpoint{6.218282in}{3.580372in}} %
\pgfusepath{clip}%
\pgfsetbuttcap%
\pgfsetroundjoin%
\definecolor{currentfill}{rgb}{0.805723,0.259813,0.230562}%
\pgfsetfillcolor{currentfill}%
\pgfsetlinewidth{0.000000pt}%
\definecolor{currentstroke}{rgb}{0.000000,0.000000,0.000000}%
\pgfsetstrokecolor{currentstroke}%
\pgfsetdash{}{0pt}%
\pgfpathmoveto{\pgfqpoint{1.971633in}{2.586781in}}%
\pgfpathlineto{\pgfqpoint{2.237924in}{2.752812in}}%
\pgfpathlineto{\pgfqpoint{1.971633in}{2.586781in}}%
\pgfpathclose%
\pgfusepath{fill}%
\end{pgfscope}%
\begin{pgfscope}%
\pgfpathrectangle{\pgfqpoint{0.017500in}{0.017500in}}{\pgfqpoint{6.218282in}{3.580372in}} %
\pgfusepath{clip}%
\pgfsetbuttcap%
\pgfsetroundjoin%
\definecolor{currentfill}{rgb}{0.805723,0.259813,0.230562}%
\pgfsetfillcolor{currentfill}%
\pgfsetlinewidth{0.000000pt}%
\definecolor{currentstroke}{rgb}{0.000000,0.000000,0.000000}%
\pgfsetstrokecolor{currentstroke}%
\pgfsetdash{}{0pt}%
\pgfpathmoveto{\pgfqpoint{1.971633in}{2.586781in}}%
\pgfpathlineto{\pgfqpoint{2.237924in}{2.752812in}}%
\pgfpathlineto{\pgfqpoint{1.971633in}{2.586781in}}%
\pgfpathclose%
\pgfusepath{fill}%
\end{pgfscope}%
\begin{pgfscope}%
\pgfpathrectangle{\pgfqpoint{0.017500in}{0.017500in}}{\pgfqpoint{6.218282in}{3.580372in}} %
\pgfusepath{clip}%
\pgfsetbuttcap%
\pgfsetroundjoin%
\definecolor{currentfill}{rgb}{0.805723,0.259813,0.230562}%
\pgfsetfillcolor{currentfill}%
\pgfsetlinewidth{0.000000pt}%
\definecolor{currentstroke}{rgb}{0.000000,0.000000,0.000000}%
\pgfsetstrokecolor{currentstroke}%
\pgfsetdash{}{0pt}%
\pgfpathmoveto{\pgfqpoint{1.971633in}{2.586781in}}%
\pgfpathlineto{\pgfqpoint{2.237924in}{2.752812in}}%
\pgfpathlineto{\pgfqpoint{1.971633in}{2.586781in}}%
\pgfpathclose%
\pgfusepath{fill}%
\end{pgfscope}%
\begin{pgfscope}%
\pgfpathrectangle{\pgfqpoint{0.017500in}{0.017500in}}{\pgfqpoint{6.218282in}{3.580372in}} %
\pgfusepath{clip}%
\pgfsetbuttcap%
\pgfsetroundjoin%
\definecolor{currentfill}{rgb}{0.805723,0.259813,0.230562}%
\pgfsetfillcolor{currentfill}%
\pgfsetlinewidth{0.000000pt}%
\definecolor{currentstroke}{rgb}{0.000000,0.000000,0.000000}%
\pgfsetstrokecolor{currentstroke}%
\pgfsetdash{}{0pt}%
\pgfpathmoveto{\pgfqpoint{1.971633in}{2.586781in}}%
\pgfpathlineto{\pgfqpoint{2.237924in}{2.752812in}}%
\pgfpathlineto{\pgfqpoint{1.971633in}{2.586781in}}%
\pgfpathclose%
\pgfusepath{fill}%
\end{pgfscope}%
\begin{pgfscope}%
\pgfpathrectangle{\pgfqpoint{0.017500in}{0.017500in}}{\pgfqpoint{6.218282in}{3.580372in}} %
\pgfusepath{clip}%
\pgfsetbuttcap%
\pgfsetroundjoin%
\definecolor{currentfill}{rgb}{0.805723,0.259813,0.230562}%
\pgfsetfillcolor{currentfill}%
\pgfsetlinewidth{0.000000pt}%
\definecolor{currentstroke}{rgb}{0.000000,0.000000,0.000000}%
\pgfsetstrokecolor{currentstroke}%
\pgfsetdash{}{0pt}%
\pgfpathmoveto{\pgfqpoint{1.971633in}{2.586781in}}%
\pgfpathlineto{\pgfqpoint{2.237924in}{2.752812in}}%
\pgfpathlineto{\pgfqpoint{1.971633in}{2.586781in}}%
\pgfpathclose%
\pgfusepath{fill}%
\end{pgfscope}%
\begin{pgfscope}%
\pgfpathrectangle{\pgfqpoint{0.017500in}{0.017500in}}{\pgfqpoint{6.218282in}{3.580372in}} %
\pgfusepath{clip}%
\pgfsetbuttcap%
\pgfsetroundjoin%
\definecolor{currentfill}{rgb}{0.805723,0.259813,0.230562}%
\pgfsetfillcolor{currentfill}%
\pgfsetlinewidth{0.000000pt}%
\definecolor{currentstroke}{rgb}{0.000000,0.000000,0.000000}%
\pgfsetstrokecolor{currentstroke}%
\pgfsetdash{}{0pt}%
\pgfpathmoveto{\pgfqpoint{1.971633in}{2.586781in}}%
\pgfpathlineto{\pgfqpoint{2.237924in}{2.752812in}}%
\pgfpathlineto{\pgfqpoint{1.971633in}{2.586781in}}%
\pgfpathclose%
\pgfusepath{fill}%
\end{pgfscope}%
\begin{pgfscope}%
\pgfpathrectangle{\pgfqpoint{0.017500in}{0.017500in}}{\pgfqpoint{6.218282in}{3.580372in}} %
\pgfusepath{clip}%
\pgfsetbuttcap%
\pgfsetroundjoin%
\definecolor{currentfill}{rgb}{0.805723,0.259813,0.230562}%
\pgfsetfillcolor{currentfill}%
\pgfsetlinewidth{0.000000pt}%
\definecolor{currentstroke}{rgb}{0.000000,0.000000,0.000000}%
\pgfsetstrokecolor{currentstroke}%
\pgfsetdash{}{0pt}%
\pgfpathmoveto{\pgfqpoint{1.971633in}{2.586781in}}%
\pgfpathlineto{\pgfqpoint{2.237924in}{2.752812in}}%
\pgfpathlineto{\pgfqpoint{1.971633in}{2.586781in}}%
\pgfpathclose%
\pgfusepath{fill}%
\end{pgfscope}%
\begin{pgfscope}%
\pgfpathrectangle{\pgfqpoint{0.017500in}{0.017500in}}{\pgfqpoint{6.218282in}{3.580372in}} %
\pgfusepath{clip}%
\pgfsetbuttcap%
\pgfsetroundjoin%
\definecolor{currentfill}{rgb}{0.805723,0.259813,0.230562}%
\pgfsetfillcolor{currentfill}%
\pgfsetlinewidth{0.000000pt}%
\definecolor{currentstroke}{rgb}{0.000000,0.000000,0.000000}%
\pgfsetstrokecolor{currentstroke}%
\pgfsetdash{}{0pt}%
\pgfpathmoveto{\pgfqpoint{1.971633in}{2.586781in}}%
\pgfpathlineto{\pgfqpoint{2.237924in}{2.752812in}}%
\pgfpathlineto{\pgfqpoint{1.971633in}{2.586781in}}%
\pgfpathclose%
\pgfusepath{fill}%
\end{pgfscope}%
\begin{pgfscope}%
\pgfpathrectangle{\pgfqpoint{0.017500in}{0.017500in}}{\pgfqpoint{6.218282in}{3.580372in}} %
\pgfusepath{clip}%
\pgfsetbuttcap%
\pgfsetroundjoin%
\definecolor{currentfill}{rgb}{0.805723,0.259813,0.230562}%
\pgfsetfillcolor{currentfill}%
\pgfsetlinewidth{0.000000pt}%
\definecolor{currentstroke}{rgb}{0.000000,0.000000,0.000000}%
\pgfsetstrokecolor{currentstroke}%
\pgfsetdash{}{0pt}%
\pgfpathmoveto{\pgfqpoint{1.971633in}{2.586781in}}%
\pgfpathlineto{\pgfqpoint{2.237924in}{2.752812in}}%
\pgfpathlineto{\pgfqpoint{1.971633in}{2.586781in}}%
\pgfpathclose%
\pgfusepath{fill}%
\end{pgfscope}%
\begin{pgfscope}%
\pgfpathrectangle{\pgfqpoint{0.017500in}{0.017500in}}{\pgfqpoint{6.218282in}{3.580372in}} %
\pgfusepath{clip}%
\pgfsetbuttcap%
\pgfsetroundjoin%
\definecolor{currentfill}{rgb}{0.805723,0.259813,0.230562}%
\pgfsetfillcolor{currentfill}%
\pgfsetlinewidth{0.000000pt}%
\definecolor{currentstroke}{rgb}{0.000000,0.000000,0.000000}%
\pgfsetstrokecolor{currentstroke}%
\pgfsetdash{}{0pt}%
\pgfpathmoveto{\pgfqpoint{1.971633in}{2.586781in}}%
\pgfpathlineto{\pgfqpoint{2.237924in}{2.752812in}}%
\pgfpathlineto{\pgfqpoint{1.971633in}{2.586781in}}%
\pgfpathclose%
\pgfusepath{fill}%
\end{pgfscope}%
\begin{pgfscope}%
\pgfpathrectangle{\pgfqpoint{0.017500in}{0.017500in}}{\pgfqpoint{6.218282in}{3.580372in}} %
\pgfusepath{clip}%
\pgfsetbuttcap%
\pgfsetroundjoin%
\definecolor{currentfill}{rgb}{0.805723,0.259813,0.230562}%
\pgfsetfillcolor{currentfill}%
\pgfsetlinewidth{0.000000pt}%
\definecolor{currentstroke}{rgb}{0.000000,0.000000,0.000000}%
\pgfsetstrokecolor{currentstroke}%
\pgfsetdash{}{0pt}%
\pgfpathmoveto{\pgfqpoint{1.971633in}{2.586781in}}%
\pgfpathlineto{\pgfqpoint{2.237924in}{2.752812in}}%
\pgfpathlineto{\pgfqpoint{1.971633in}{2.586781in}}%
\pgfpathclose%
\pgfusepath{fill}%
\end{pgfscope}%
\begin{pgfscope}%
\pgfpathrectangle{\pgfqpoint{0.017500in}{0.017500in}}{\pgfqpoint{6.218282in}{3.580372in}} %
\pgfusepath{clip}%
\pgfsetbuttcap%
\pgfsetroundjoin%
\definecolor{currentfill}{rgb}{0.805723,0.259813,0.230562}%
\pgfsetfillcolor{currentfill}%
\pgfsetlinewidth{0.000000pt}%
\definecolor{currentstroke}{rgb}{0.000000,0.000000,0.000000}%
\pgfsetstrokecolor{currentstroke}%
\pgfsetdash{}{0pt}%
\pgfpathmoveto{\pgfqpoint{1.971633in}{2.586781in}}%
\pgfpathlineto{\pgfqpoint{2.237924in}{2.752812in}}%
\pgfpathlineto{\pgfqpoint{1.971633in}{2.586781in}}%
\pgfpathclose%
\pgfusepath{fill}%
\end{pgfscope}%
\begin{pgfscope}%
\pgfpathrectangle{\pgfqpoint{0.017500in}{0.017500in}}{\pgfqpoint{6.218282in}{3.580372in}} %
\pgfusepath{clip}%
\pgfsetbuttcap%
\pgfsetroundjoin%
\definecolor{currentfill}{rgb}{0.805723,0.259813,0.230562}%
\pgfsetfillcolor{currentfill}%
\pgfsetlinewidth{0.000000pt}%
\definecolor{currentstroke}{rgb}{0.000000,0.000000,0.000000}%
\pgfsetstrokecolor{currentstroke}%
\pgfsetdash{}{0pt}%
\pgfpathmoveto{\pgfqpoint{1.971633in}{2.586781in}}%
\pgfpathlineto{\pgfqpoint{2.237924in}{2.752812in}}%
\pgfpathlineto{\pgfqpoint{1.971633in}{2.586781in}}%
\pgfpathclose%
\pgfusepath{fill}%
\end{pgfscope}%
\begin{pgfscope}%
\pgfpathrectangle{\pgfqpoint{0.017500in}{0.017500in}}{\pgfqpoint{6.218282in}{3.580372in}} %
\pgfusepath{clip}%
\pgfsetbuttcap%
\pgfsetroundjoin%
\definecolor{currentfill}{rgb}{0.805723,0.259813,0.230562}%
\pgfsetfillcolor{currentfill}%
\pgfsetlinewidth{0.000000pt}%
\definecolor{currentstroke}{rgb}{0.000000,0.000000,0.000000}%
\pgfsetstrokecolor{currentstroke}%
\pgfsetdash{}{0pt}%
\pgfpathmoveto{\pgfqpoint{1.971633in}{2.586781in}}%
\pgfpathlineto{\pgfqpoint{2.237924in}{2.752812in}}%
\pgfpathlineto{\pgfqpoint{1.971633in}{2.586781in}}%
\pgfpathclose%
\pgfusepath{fill}%
\end{pgfscope}%
\begin{pgfscope}%
\pgfpathrectangle{\pgfqpoint{0.017500in}{0.017500in}}{\pgfqpoint{6.218282in}{3.580372in}} %
\pgfusepath{clip}%
\pgfsetbuttcap%
\pgfsetroundjoin%
\definecolor{currentfill}{rgb}{0.805723,0.259813,0.230562}%
\pgfsetfillcolor{currentfill}%
\pgfsetlinewidth{0.000000pt}%
\definecolor{currentstroke}{rgb}{0.000000,0.000000,0.000000}%
\pgfsetstrokecolor{currentstroke}%
\pgfsetdash{}{0pt}%
\pgfpathmoveto{\pgfqpoint{1.971633in}{2.586781in}}%
\pgfpathlineto{\pgfqpoint{2.237924in}{2.752812in}}%
\pgfpathlineto{\pgfqpoint{1.971633in}{2.586781in}}%
\pgfpathclose%
\pgfusepath{fill}%
\end{pgfscope}%
\begin{pgfscope}%
\pgfpathrectangle{\pgfqpoint{0.017500in}{0.017500in}}{\pgfqpoint{6.218282in}{3.580372in}} %
\pgfusepath{clip}%
\pgfsetbuttcap%
\pgfsetroundjoin%
\definecolor{currentfill}{rgb}{0.805723,0.259813,0.230562}%
\pgfsetfillcolor{currentfill}%
\pgfsetlinewidth{0.000000pt}%
\definecolor{currentstroke}{rgb}{0.000000,0.000000,0.000000}%
\pgfsetstrokecolor{currentstroke}%
\pgfsetdash{}{0pt}%
\pgfpathmoveto{\pgfqpoint{1.971633in}{2.586781in}}%
\pgfpathlineto{\pgfqpoint{2.237924in}{2.752812in}}%
\pgfpathlineto{\pgfqpoint{1.971633in}{2.586781in}}%
\pgfpathclose%
\pgfusepath{fill}%
\end{pgfscope}%
\begin{pgfscope}%
\pgfpathrectangle{\pgfqpoint{0.017500in}{0.017500in}}{\pgfqpoint{6.218282in}{3.580372in}} %
\pgfusepath{clip}%
\pgfsetbuttcap%
\pgfsetroundjoin%
\definecolor{currentfill}{rgb}{0.805723,0.259813,0.230562}%
\pgfsetfillcolor{currentfill}%
\pgfsetlinewidth{0.000000pt}%
\definecolor{currentstroke}{rgb}{0.000000,0.000000,0.000000}%
\pgfsetstrokecolor{currentstroke}%
\pgfsetdash{}{0pt}%
\pgfpathmoveto{\pgfqpoint{1.971633in}{2.586781in}}%
\pgfpathlineto{\pgfqpoint{2.237924in}{2.752812in}}%
\pgfpathlineto{\pgfqpoint{1.971633in}{2.586781in}}%
\pgfpathclose%
\pgfusepath{fill}%
\end{pgfscope}%
\begin{pgfscope}%
\pgfpathrectangle{\pgfqpoint{0.017500in}{0.017500in}}{\pgfqpoint{6.218282in}{3.580372in}} %
\pgfusepath{clip}%
\pgfsetbuttcap%
\pgfsetroundjoin%
\definecolor{currentfill}{rgb}{0.343278,0.459354,0.884122}%
\pgfsetfillcolor{currentfill}%
\pgfsetlinewidth{0.000000pt}%
\definecolor{currentstroke}{rgb}{0.000000,0.000000,0.000000}%
\pgfsetstrokecolor{currentstroke}%
\pgfsetdash{}{0pt}%
\pgfpathmoveto{\pgfqpoint{1.486050in}{1.042638in}}%
\pgfpathlineto{\pgfqpoint{1.914830in}{0.967594in}}%
\pgfpathlineto{\pgfqpoint{2.143341in}{2.291288in}}%
\pgfpathlineto{\pgfqpoint{1.701778in}{2.375902in}}%
\pgfpathlineto{\pgfqpoint{1.486050in}{1.042638in}}%
\pgfpathclose%
\pgfusepath{fill}%
\end{pgfscope}%
\begin{pgfscope}%
\pgfpathrectangle{\pgfqpoint{0.017500in}{0.017500in}}{\pgfqpoint{6.218282in}{3.580372in}} %
\pgfusepath{clip}%
\pgfsetbuttcap%
\pgfsetroundjoin%
\definecolor{currentfill}{rgb}{0.343278,0.459354,0.884122}%
\pgfsetfillcolor{currentfill}%
\pgfsetlinewidth{0.000000pt}%
\definecolor{currentstroke}{rgb}{0.000000,0.000000,0.000000}%
\pgfsetstrokecolor{currentstroke}%
\pgfsetdash{}{0pt}%
\pgfpathmoveto{\pgfqpoint{1.486050in}{1.042638in}}%
\pgfpathlineto{\pgfqpoint{1.914830in}{0.967594in}}%
\pgfpathlineto{\pgfqpoint{2.143341in}{2.291288in}}%
\pgfpathlineto{\pgfqpoint{1.701778in}{2.375902in}}%
\pgfpathlineto{\pgfqpoint{1.486050in}{1.042638in}}%
\pgfpathclose%
\pgfusepath{fill}%
\end{pgfscope}%
\begin{pgfscope}%
\pgfpathrectangle{\pgfqpoint{0.017500in}{0.017500in}}{\pgfqpoint{6.218282in}{3.580372in}} %
\pgfusepath{clip}%
\pgfsetbuttcap%
\pgfsetroundjoin%
\definecolor{currentfill}{rgb}{0.343278,0.459354,0.884122}%
\pgfsetfillcolor{currentfill}%
\pgfsetlinewidth{0.000000pt}%
\definecolor{currentstroke}{rgb}{0.000000,0.000000,0.000000}%
\pgfsetstrokecolor{currentstroke}%
\pgfsetdash{}{0pt}%
\pgfpathmoveto{\pgfqpoint{1.486050in}{1.042638in}}%
\pgfpathlineto{\pgfqpoint{1.914830in}{0.967594in}}%
\pgfpathlineto{\pgfqpoint{2.143341in}{2.291288in}}%
\pgfpathlineto{\pgfqpoint{1.701778in}{2.375902in}}%
\pgfpathlineto{\pgfqpoint{1.486050in}{1.042638in}}%
\pgfpathclose%
\pgfusepath{fill}%
\end{pgfscope}%
\begin{pgfscope}%
\pgfpathrectangle{\pgfqpoint{0.017500in}{0.017500in}}{\pgfqpoint{6.218282in}{3.580372in}} %
\pgfusepath{clip}%
\pgfsetbuttcap%
\pgfsetroundjoin%
\definecolor{currentfill}{rgb}{0.343278,0.459354,0.884122}%
\pgfsetfillcolor{currentfill}%
\pgfsetlinewidth{0.000000pt}%
\definecolor{currentstroke}{rgb}{0.000000,0.000000,0.000000}%
\pgfsetstrokecolor{currentstroke}%
\pgfsetdash{}{0pt}%
\pgfpathmoveto{\pgfqpoint{1.486050in}{1.042638in}}%
\pgfpathlineto{\pgfqpoint{1.914830in}{0.967594in}}%
\pgfpathlineto{\pgfqpoint{2.143341in}{2.291288in}}%
\pgfpathlineto{\pgfqpoint{1.701778in}{2.375902in}}%
\pgfpathlineto{\pgfqpoint{1.486050in}{1.042638in}}%
\pgfpathclose%
\pgfusepath{fill}%
\end{pgfscope}%
\begin{pgfscope}%
\pgfpathrectangle{\pgfqpoint{0.017500in}{0.017500in}}{\pgfqpoint{6.218282in}{3.580372in}} %
\pgfusepath{clip}%
\pgfsetbuttcap%
\pgfsetroundjoin%
\definecolor{currentfill}{rgb}{0.343278,0.459354,0.884122}%
\pgfsetfillcolor{currentfill}%
\pgfsetlinewidth{0.000000pt}%
\definecolor{currentstroke}{rgb}{0.000000,0.000000,0.000000}%
\pgfsetstrokecolor{currentstroke}%
\pgfsetdash{}{0pt}%
\pgfpathmoveto{\pgfqpoint{1.486050in}{1.042638in}}%
\pgfpathlineto{\pgfqpoint{1.914830in}{0.967594in}}%
\pgfpathlineto{\pgfqpoint{2.143341in}{2.291288in}}%
\pgfpathlineto{\pgfqpoint{1.701778in}{2.375902in}}%
\pgfpathlineto{\pgfqpoint{1.486050in}{1.042638in}}%
\pgfpathclose%
\pgfusepath{fill}%
\end{pgfscope}%
\begin{pgfscope}%
\pgfpathrectangle{\pgfqpoint{0.017500in}{0.017500in}}{\pgfqpoint{6.218282in}{3.580372in}} %
\pgfusepath{clip}%
\pgfsetbuttcap%
\pgfsetroundjoin%
\definecolor{currentfill}{rgb}{0.343278,0.459354,0.884122}%
\pgfsetfillcolor{currentfill}%
\pgfsetlinewidth{0.000000pt}%
\definecolor{currentstroke}{rgb}{0.000000,0.000000,0.000000}%
\pgfsetstrokecolor{currentstroke}%
\pgfsetdash{}{0pt}%
\pgfpathmoveto{\pgfqpoint{1.486050in}{1.042638in}}%
\pgfpathlineto{\pgfqpoint{1.914830in}{0.967594in}}%
\pgfpathlineto{\pgfqpoint{2.143341in}{2.291288in}}%
\pgfpathlineto{\pgfqpoint{1.701778in}{2.375902in}}%
\pgfpathlineto{\pgfqpoint{1.486050in}{1.042638in}}%
\pgfpathclose%
\pgfusepath{fill}%
\end{pgfscope}%
\begin{pgfscope}%
\pgfpathrectangle{\pgfqpoint{0.017500in}{0.017500in}}{\pgfqpoint{6.218282in}{3.580372in}} %
\pgfusepath{clip}%
\pgfsetbuttcap%
\pgfsetroundjoin%
\definecolor{currentfill}{rgb}{0.800601,0.850358,0.930008}%
\pgfsetfillcolor{currentfill}%
\pgfsetlinewidth{0.000000pt}%
\definecolor{currentstroke}{rgb}{0.000000,0.000000,0.000000}%
\pgfsetstrokecolor{currentstroke}%
\pgfsetdash{}{0pt}%
\pgfpathmoveto{\pgfqpoint{4.063701in}{2.813820in}}%
\pgfpathlineto{\pgfqpoint{4.511410in}{2.753953in}}%
\pgfpathlineto{\pgfqpoint{3.236099in}{0.723497in}}%
\pgfpathlineto{\pgfqpoint{2.789950in}{0.798346in}}%
\pgfpathlineto{\pgfqpoint{4.063701in}{2.813820in}}%
\pgfpathclose%
\pgfusepath{fill}%
\end{pgfscope}%
\begin{pgfscope}%
\pgfpathrectangle{\pgfqpoint{0.017500in}{0.017500in}}{\pgfqpoint{6.218282in}{3.580372in}} %
\pgfusepath{clip}%
\pgfsetbuttcap%
\pgfsetroundjoin%
\definecolor{currentfill}{rgb}{0.800601,0.850358,0.930008}%
\pgfsetfillcolor{currentfill}%
\pgfsetlinewidth{0.000000pt}%
\definecolor{currentstroke}{rgb}{0.000000,0.000000,0.000000}%
\pgfsetstrokecolor{currentstroke}%
\pgfsetdash{}{0pt}%
\pgfpathmoveto{\pgfqpoint{4.063701in}{2.813820in}}%
\pgfpathlineto{\pgfqpoint{4.511410in}{2.753953in}}%
\pgfpathlineto{\pgfqpoint{3.236099in}{0.723497in}}%
\pgfpathlineto{\pgfqpoint{2.789950in}{0.798346in}}%
\pgfpathlineto{\pgfqpoint{4.063701in}{2.813820in}}%
\pgfpathclose%
\pgfusepath{fill}%
\end{pgfscope}%
\begin{pgfscope}%
\pgfpathrectangle{\pgfqpoint{0.017500in}{0.017500in}}{\pgfqpoint{6.218282in}{3.580372in}} %
\pgfusepath{clip}%
\pgfsetbuttcap%
\pgfsetroundjoin%
\definecolor{currentfill}{rgb}{0.800601,0.850358,0.930008}%
\pgfsetfillcolor{currentfill}%
\pgfsetlinewidth{0.000000pt}%
\definecolor{currentstroke}{rgb}{0.000000,0.000000,0.000000}%
\pgfsetstrokecolor{currentstroke}%
\pgfsetdash{}{0pt}%
\pgfpathmoveto{\pgfqpoint{4.063701in}{2.813820in}}%
\pgfpathlineto{\pgfqpoint{4.511410in}{2.753953in}}%
\pgfpathlineto{\pgfqpoint{3.236099in}{0.723497in}}%
\pgfpathlineto{\pgfqpoint{2.789950in}{0.798346in}}%
\pgfpathlineto{\pgfqpoint{4.063701in}{2.813820in}}%
\pgfpathclose%
\pgfusepath{fill}%
\end{pgfscope}%
\begin{pgfscope}%
\pgfpathrectangle{\pgfqpoint{0.017500in}{0.017500in}}{\pgfqpoint{6.218282in}{3.580372in}} %
\pgfusepath{clip}%
\pgfsetbuttcap%
\pgfsetroundjoin%
\definecolor{currentfill}{rgb}{0.800601,0.850358,0.930008}%
\pgfsetfillcolor{currentfill}%
\pgfsetlinewidth{0.000000pt}%
\definecolor{currentstroke}{rgb}{0.000000,0.000000,0.000000}%
\pgfsetstrokecolor{currentstroke}%
\pgfsetdash{}{0pt}%
\pgfpathmoveto{\pgfqpoint{4.063701in}{2.813820in}}%
\pgfpathlineto{\pgfqpoint{4.511410in}{2.753953in}}%
\pgfpathlineto{\pgfqpoint{3.236099in}{0.723497in}}%
\pgfpathlineto{\pgfqpoint{2.789950in}{0.798346in}}%
\pgfpathlineto{\pgfqpoint{4.063701in}{2.813820in}}%
\pgfpathclose%
\pgfusepath{fill}%
\end{pgfscope}%
\begin{pgfscope}%
\pgfpathrectangle{\pgfqpoint{0.017500in}{0.017500in}}{\pgfqpoint{6.218282in}{3.580372in}} %
\pgfusepath{clip}%
\pgfsetbuttcap%
\pgfsetroundjoin%
\definecolor{currentfill}{rgb}{0.800601,0.850358,0.930008}%
\pgfsetfillcolor{currentfill}%
\pgfsetlinewidth{0.000000pt}%
\definecolor{currentstroke}{rgb}{0.000000,0.000000,0.000000}%
\pgfsetstrokecolor{currentstroke}%
\pgfsetdash{}{0pt}%
\pgfpathmoveto{\pgfqpoint{4.063701in}{2.813820in}}%
\pgfpathlineto{\pgfqpoint{4.511410in}{2.753953in}}%
\pgfpathlineto{\pgfqpoint{3.236099in}{0.723497in}}%
\pgfpathlineto{\pgfqpoint{2.789950in}{0.798346in}}%
\pgfpathlineto{\pgfqpoint{4.063701in}{2.813820in}}%
\pgfpathclose%
\pgfusepath{fill}%
\end{pgfscope}%
\begin{pgfscope}%
\pgfpathrectangle{\pgfqpoint{0.017500in}{0.017500in}}{\pgfqpoint{6.218282in}{3.580372in}} %
\pgfusepath{clip}%
\pgfsetbuttcap%
\pgfsetroundjoin%
\definecolor{currentfill}{rgb}{0.729196,0.086679,0.167240}%
\pgfsetfillcolor{currentfill}%
\pgfsetlinewidth{0.000000pt}%
\definecolor{currentstroke}{rgb}{0.000000,0.000000,0.000000}%
\pgfsetstrokecolor{currentstroke}%
\pgfsetdash{}{0pt}%
\pgfpathmoveto{\pgfqpoint{3.816789in}{2.677291in}}%
\pgfpathlineto{\pgfqpoint{4.267628in}{2.613520in}}%
\pgfpathlineto{\pgfqpoint{4.511410in}{2.753953in}}%
\pgfpathlineto{\pgfqpoint{4.063701in}{2.813820in}}%
\pgfpathlineto{\pgfqpoint{3.816789in}{2.677291in}}%
\pgfpathclose%
\pgfusepath{fill}%
\end{pgfscope}%
\begin{pgfscope}%
\pgfpathrectangle{\pgfqpoint{0.017500in}{0.017500in}}{\pgfqpoint{6.218282in}{3.580372in}} %
\pgfusepath{clip}%
\pgfsetbuttcap%
\pgfsetroundjoin%
\definecolor{currentfill}{rgb}{0.729196,0.086679,0.167240}%
\pgfsetfillcolor{currentfill}%
\pgfsetlinewidth{0.000000pt}%
\definecolor{currentstroke}{rgb}{0.000000,0.000000,0.000000}%
\pgfsetstrokecolor{currentstroke}%
\pgfsetdash{}{0pt}%
\pgfpathmoveto{\pgfqpoint{3.816789in}{2.677291in}}%
\pgfpathlineto{\pgfqpoint{4.267628in}{2.613520in}}%
\pgfpathlineto{\pgfqpoint{4.511410in}{2.753953in}}%
\pgfpathlineto{\pgfqpoint{4.063701in}{2.813820in}}%
\pgfpathlineto{\pgfqpoint{3.816789in}{2.677291in}}%
\pgfpathclose%
\pgfusepath{fill}%
\end{pgfscope}%
\begin{pgfscope}%
\pgfpathrectangle{\pgfqpoint{0.017500in}{0.017500in}}{\pgfqpoint{6.218282in}{3.580372in}} %
\pgfusepath{clip}%
\pgfsetbuttcap%
\pgfsetroundjoin%
\definecolor{currentfill}{rgb}{0.729196,0.086679,0.167240}%
\pgfsetfillcolor{currentfill}%
\pgfsetlinewidth{0.000000pt}%
\definecolor{currentstroke}{rgb}{0.000000,0.000000,0.000000}%
\pgfsetstrokecolor{currentstroke}%
\pgfsetdash{}{0pt}%
\pgfpathmoveto{\pgfqpoint{3.816789in}{2.677291in}}%
\pgfpathlineto{\pgfqpoint{4.267628in}{2.613520in}}%
\pgfpathlineto{\pgfqpoint{4.511410in}{2.753953in}}%
\pgfpathlineto{\pgfqpoint{4.063701in}{2.813820in}}%
\pgfpathlineto{\pgfqpoint{3.816789in}{2.677291in}}%
\pgfpathclose%
\pgfusepath{fill}%
\end{pgfscope}%
\begin{pgfscope}%
\pgfpathrectangle{\pgfqpoint{0.017500in}{0.017500in}}{\pgfqpoint{6.218282in}{3.580372in}} %
\pgfusepath{clip}%
\pgfsetbuttcap%
\pgfsetroundjoin%
\definecolor{currentfill}{rgb}{0.729196,0.086679,0.167240}%
\pgfsetfillcolor{currentfill}%
\pgfsetlinewidth{0.000000pt}%
\definecolor{currentstroke}{rgb}{0.000000,0.000000,0.000000}%
\pgfsetstrokecolor{currentstroke}%
\pgfsetdash{}{0pt}%
\pgfpathmoveto{\pgfqpoint{3.816789in}{2.677291in}}%
\pgfpathlineto{\pgfqpoint{4.267628in}{2.613520in}}%
\pgfpathlineto{\pgfqpoint{4.511410in}{2.753953in}}%
\pgfpathlineto{\pgfqpoint{4.063701in}{2.813820in}}%
\pgfpathlineto{\pgfqpoint{3.816789in}{2.677291in}}%
\pgfpathclose%
\pgfusepath{fill}%
\end{pgfscope}%
\begin{pgfscope}%
\pgfpathrectangle{\pgfqpoint{0.017500in}{0.017500in}}{\pgfqpoint{6.218282in}{3.580372in}} %
\pgfusepath{clip}%
\pgfsetbuttcap%
\pgfsetroundjoin%
\definecolor{currentfill}{rgb}{0.729196,0.086679,0.167240}%
\pgfsetfillcolor{currentfill}%
\pgfsetlinewidth{0.000000pt}%
\definecolor{currentstroke}{rgb}{0.000000,0.000000,0.000000}%
\pgfsetstrokecolor{currentstroke}%
\pgfsetdash{}{0pt}%
\pgfpathmoveto{\pgfqpoint{3.816789in}{2.677291in}}%
\pgfpathlineto{\pgfqpoint{4.267628in}{2.613520in}}%
\pgfpathlineto{\pgfqpoint{4.511410in}{2.753953in}}%
\pgfpathlineto{\pgfqpoint{4.063701in}{2.813820in}}%
\pgfpathlineto{\pgfqpoint{3.816789in}{2.677291in}}%
\pgfpathclose%
\pgfusepath{fill}%
\end{pgfscope}%
\begin{pgfscope}%
\pgfpathrectangle{\pgfqpoint{0.017500in}{0.017500in}}{\pgfqpoint{6.218282in}{3.580372in}} %
\pgfusepath{clip}%
\pgfsetbuttcap%
\pgfsetroundjoin%
\definecolor{currentfill}{rgb}{0.729196,0.086679,0.167240}%
\pgfsetfillcolor{currentfill}%
\pgfsetlinewidth{0.000000pt}%
\definecolor{currentstroke}{rgb}{0.000000,0.000000,0.000000}%
\pgfsetstrokecolor{currentstroke}%
\pgfsetdash{}{0pt}%
\pgfpathmoveto{\pgfqpoint{3.816789in}{2.677291in}}%
\pgfpathlineto{\pgfqpoint{4.267628in}{2.613520in}}%
\pgfpathlineto{\pgfqpoint{4.511410in}{2.753953in}}%
\pgfpathlineto{\pgfqpoint{4.063701in}{2.813820in}}%
\pgfpathlineto{\pgfqpoint{3.816789in}{2.677291in}}%
\pgfpathclose%
\pgfusepath{fill}%
\end{pgfscope}%
\begin{pgfscope}%
\pgfpathrectangle{\pgfqpoint{0.017500in}{0.017500in}}{\pgfqpoint{6.218282in}{3.580372in}} %
\pgfusepath{clip}%
\pgfsetbuttcap%
\pgfsetroundjoin%
\definecolor{currentfill}{rgb}{0.800830,0.250829,0.225696}%
\pgfsetfillcolor{currentfill}%
\pgfsetlinewidth{0.000000pt}%
\definecolor{currentstroke}{rgb}{0.000000,0.000000,0.000000}%
\pgfsetstrokecolor{currentstroke}%
\pgfsetdash{}{0pt}%
\pgfpathmoveto{\pgfqpoint{1.971633in}{2.586781in}}%
\pgfpathlineto{\pgfqpoint{2.410872in}{2.508083in}}%
\pgfpathlineto{\pgfqpoint{2.674254in}{2.688576in}}%
\pgfpathlineto{\pgfqpoint{2.237924in}{2.752812in}}%
\pgfpathlineto{\pgfqpoint{1.971633in}{2.586781in}}%
\pgfpathclose%
\pgfusepath{fill}%
\end{pgfscope}%
\begin{pgfscope}%
\pgfpathrectangle{\pgfqpoint{0.017500in}{0.017500in}}{\pgfqpoint{6.218282in}{3.580372in}} %
\pgfusepath{clip}%
\pgfsetbuttcap%
\pgfsetroundjoin%
\definecolor{currentfill}{rgb}{0.800830,0.250829,0.225696}%
\pgfsetfillcolor{currentfill}%
\pgfsetlinewidth{0.000000pt}%
\definecolor{currentstroke}{rgb}{0.000000,0.000000,0.000000}%
\pgfsetstrokecolor{currentstroke}%
\pgfsetdash{}{0pt}%
\pgfpathmoveto{\pgfqpoint{1.971633in}{2.586781in}}%
\pgfpathlineto{\pgfqpoint{2.410872in}{2.508083in}}%
\pgfpathlineto{\pgfqpoint{2.674254in}{2.688576in}}%
\pgfpathlineto{\pgfqpoint{2.237924in}{2.752812in}}%
\pgfpathlineto{\pgfqpoint{1.971633in}{2.586781in}}%
\pgfpathclose%
\pgfusepath{fill}%
\end{pgfscope}%
\begin{pgfscope}%
\pgfpathrectangle{\pgfqpoint{0.017500in}{0.017500in}}{\pgfqpoint{6.218282in}{3.580372in}} %
\pgfusepath{clip}%
\pgfsetbuttcap%
\pgfsetroundjoin%
\definecolor{currentfill}{rgb}{0.800830,0.250829,0.225696}%
\pgfsetfillcolor{currentfill}%
\pgfsetlinewidth{0.000000pt}%
\definecolor{currentstroke}{rgb}{0.000000,0.000000,0.000000}%
\pgfsetstrokecolor{currentstroke}%
\pgfsetdash{}{0pt}%
\pgfpathmoveto{\pgfqpoint{1.971633in}{2.586781in}}%
\pgfpathlineto{\pgfqpoint{2.410872in}{2.508083in}}%
\pgfpathlineto{\pgfqpoint{2.674254in}{2.688576in}}%
\pgfpathlineto{\pgfqpoint{2.237924in}{2.752812in}}%
\pgfpathlineto{\pgfqpoint{1.971633in}{2.586781in}}%
\pgfpathclose%
\pgfusepath{fill}%
\end{pgfscope}%
\begin{pgfscope}%
\pgfpathrectangle{\pgfqpoint{0.017500in}{0.017500in}}{\pgfqpoint{6.218282in}{3.580372in}} %
\pgfusepath{clip}%
\pgfsetbuttcap%
\pgfsetroundjoin%
\definecolor{currentfill}{rgb}{0.800830,0.250829,0.225696}%
\pgfsetfillcolor{currentfill}%
\pgfsetlinewidth{0.000000pt}%
\definecolor{currentstroke}{rgb}{0.000000,0.000000,0.000000}%
\pgfsetstrokecolor{currentstroke}%
\pgfsetdash{}{0pt}%
\pgfpathmoveto{\pgfqpoint{1.971633in}{2.586781in}}%
\pgfpathlineto{\pgfqpoint{2.410872in}{2.508083in}}%
\pgfpathlineto{\pgfqpoint{2.674254in}{2.688576in}}%
\pgfpathlineto{\pgfqpoint{2.237924in}{2.752812in}}%
\pgfpathlineto{\pgfqpoint{1.971633in}{2.586781in}}%
\pgfpathclose%
\pgfusepath{fill}%
\end{pgfscope}%
\begin{pgfscope}%
\pgfpathrectangle{\pgfqpoint{0.017500in}{0.017500in}}{\pgfqpoint{6.218282in}{3.580372in}} %
\pgfusepath{clip}%
\pgfsetbuttcap%
\pgfsetroundjoin%
\definecolor{currentfill}{rgb}{0.800830,0.250829,0.225696}%
\pgfsetfillcolor{currentfill}%
\pgfsetlinewidth{0.000000pt}%
\definecolor{currentstroke}{rgb}{0.000000,0.000000,0.000000}%
\pgfsetstrokecolor{currentstroke}%
\pgfsetdash{}{0pt}%
\pgfpathmoveto{\pgfqpoint{1.971633in}{2.586781in}}%
\pgfpathlineto{\pgfqpoint{2.410872in}{2.508083in}}%
\pgfpathlineto{\pgfqpoint{2.674254in}{2.688576in}}%
\pgfpathlineto{\pgfqpoint{2.237924in}{2.752812in}}%
\pgfpathlineto{\pgfqpoint{1.971633in}{2.586781in}}%
\pgfpathclose%
\pgfusepath{fill}%
\end{pgfscope}%
\begin{pgfscope}%
\pgfpathrectangle{\pgfqpoint{0.017500in}{0.017500in}}{\pgfqpoint{6.218282in}{3.580372in}} %
\pgfusepath{clip}%
\pgfsetbuttcap%
\pgfsetroundjoin%
\definecolor{currentfill}{rgb}{0.800830,0.250829,0.225696}%
\pgfsetfillcolor{currentfill}%
\pgfsetlinewidth{0.000000pt}%
\definecolor{currentstroke}{rgb}{0.000000,0.000000,0.000000}%
\pgfsetstrokecolor{currentstroke}%
\pgfsetdash{}{0pt}%
\pgfpathmoveto{\pgfqpoint{1.971633in}{2.586781in}}%
\pgfpathlineto{\pgfqpoint{2.410872in}{2.508083in}}%
\pgfpathlineto{\pgfqpoint{2.674254in}{2.688576in}}%
\pgfpathlineto{\pgfqpoint{2.237924in}{2.752812in}}%
\pgfpathlineto{\pgfqpoint{1.971633in}{2.586781in}}%
\pgfpathclose%
\pgfusepath{fill}%
\end{pgfscope}%
\begin{pgfscope}%
\pgfpathrectangle{\pgfqpoint{0.017500in}{0.017500in}}{\pgfqpoint{6.218282in}{3.580372in}} %
\pgfusepath{clip}%
\pgfsetbuttcap%
\pgfsetroundjoin%
\definecolor{currentfill}{rgb}{0.243520,0.319189,0.771672}%
\pgfsetfillcolor{currentfill}%
\pgfsetlinewidth{0.000000pt}%
\definecolor{currentstroke}{rgb}{0.000000,0.000000,0.000000}%
\pgfsetstrokecolor{currentstroke}%
\pgfsetdash{}{0pt}%
\pgfpathmoveto{\pgfqpoint{1.914830in}{0.967594in}}%
\pgfpathlineto{\pgfqpoint{2.143341in}{2.291288in}}%
\pgfpathlineto{\pgfqpoint{1.914830in}{0.967594in}}%
\pgfpathclose%
\pgfusepath{fill}%
\end{pgfscope}%
\begin{pgfscope}%
\pgfpathrectangle{\pgfqpoint{0.017500in}{0.017500in}}{\pgfqpoint{6.218282in}{3.580372in}} %
\pgfusepath{clip}%
\pgfsetbuttcap%
\pgfsetroundjoin%
\definecolor{currentfill}{rgb}{0.243520,0.319189,0.771672}%
\pgfsetfillcolor{currentfill}%
\pgfsetlinewidth{0.000000pt}%
\definecolor{currentstroke}{rgb}{0.000000,0.000000,0.000000}%
\pgfsetstrokecolor{currentstroke}%
\pgfsetdash{}{0pt}%
\pgfpathmoveto{\pgfqpoint{1.914830in}{0.967594in}}%
\pgfpathlineto{\pgfqpoint{2.143341in}{2.291288in}}%
\pgfpathlineto{\pgfqpoint{1.914830in}{0.967594in}}%
\pgfpathclose%
\pgfusepath{fill}%
\end{pgfscope}%
\begin{pgfscope}%
\pgfpathrectangle{\pgfqpoint{0.017500in}{0.017500in}}{\pgfqpoint{6.218282in}{3.580372in}} %
\pgfusepath{clip}%
\pgfsetbuttcap%
\pgfsetroundjoin%
\definecolor{currentfill}{rgb}{0.243520,0.319189,0.771672}%
\pgfsetfillcolor{currentfill}%
\pgfsetlinewidth{0.000000pt}%
\definecolor{currentstroke}{rgb}{0.000000,0.000000,0.000000}%
\pgfsetstrokecolor{currentstroke}%
\pgfsetdash{}{0pt}%
\pgfpathmoveto{\pgfqpoint{1.914830in}{0.967594in}}%
\pgfpathlineto{\pgfqpoint{2.143341in}{2.291288in}}%
\pgfpathlineto{\pgfqpoint{1.914830in}{0.967594in}}%
\pgfpathclose%
\pgfusepath{fill}%
\end{pgfscope}%
\begin{pgfscope}%
\pgfpathrectangle{\pgfqpoint{0.017500in}{0.017500in}}{\pgfqpoint{6.218282in}{3.580372in}} %
\pgfusepath{clip}%
\pgfsetbuttcap%
\pgfsetroundjoin%
\definecolor{currentfill}{rgb}{0.243520,0.319189,0.771672}%
\pgfsetfillcolor{currentfill}%
\pgfsetlinewidth{0.000000pt}%
\definecolor{currentstroke}{rgb}{0.000000,0.000000,0.000000}%
\pgfsetstrokecolor{currentstroke}%
\pgfsetdash{}{0pt}%
\pgfpathmoveto{\pgfqpoint{1.914830in}{0.967594in}}%
\pgfpathlineto{\pgfqpoint{2.143341in}{2.291288in}}%
\pgfpathlineto{\pgfqpoint{1.914830in}{0.967594in}}%
\pgfpathclose%
\pgfusepath{fill}%
\end{pgfscope}%
\begin{pgfscope}%
\pgfpathrectangle{\pgfqpoint{0.017500in}{0.017500in}}{\pgfqpoint{6.218282in}{3.580372in}} %
\pgfusepath{clip}%
\pgfsetbuttcap%
\pgfsetroundjoin%
\definecolor{currentfill}{rgb}{0.243520,0.319189,0.771672}%
\pgfsetfillcolor{currentfill}%
\pgfsetlinewidth{0.000000pt}%
\definecolor{currentstroke}{rgb}{0.000000,0.000000,0.000000}%
\pgfsetstrokecolor{currentstroke}%
\pgfsetdash{}{0pt}%
\pgfpathmoveto{\pgfqpoint{1.914830in}{0.967594in}}%
\pgfpathlineto{\pgfqpoint{2.143341in}{2.291288in}}%
\pgfpathlineto{\pgfqpoint{1.914830in}{0.967594in}}%
\pgfpathclose%
\pgfusepath{fill}%
\end{pgfscope}%
\begin{pgfscope}%
\pgfpathrectangle{\pgfqpoint{0.017500in}{0.017500in}}{\pgfqpoint{6.218282in}{3.580372in}} %
\pgfusepath{clip}%
\pgfsetbuttcap%
\pgfsetroundjoin%
\definecolor{currentfill}{rgb}{0.243520,0.319189,0.771672}%
\pgfsetfillcolor{currentfill}%
\pgfsetlinewidth{0.000000pt}%
\definecolor{currentstroke}{rgb}{0.000000,0.000000,0.000000}%
\pgfsetstrokecolor{currentstroke}%
\pgfsetdash{}{0pt}%
\pgfpathmoveto{\pgfqpoint{1.914830in}{0.967594in}}%
\pgfpathlineto{\pgfqpoint{2.143341in}{2.291288in}}%
\pgfpathlineto{\pgfqpoint{1.914830in}{0.967594in}}%
\pgfpathclose%
\pgfusepath{fill}%
\end{pgfscope}%
\begin{pgfscope}%
\pgfpathrectangle{\pgfqpoint{0.017500in}{0.017500in}}{\pgfqpoint{6.218282in}{3.580372in}} %
\pgfusepath{clip}%
\pgfsetbuttcap%
\pgfsetroundjoin%
\definecolor{currentfill}{rgb}{0.243520,0.319189,0.771672}%
\pgfsetfillcolor{currentfill}%
\pgfsetlinewidth{0.000000pt}%
\definecolor{currentstroke}{rgb}{0.000000,0.000000,0.000000}%
\pgfsetstrokecolor{currentstroke}%
\pgfsetdash{}{0pt}%
\pgfpathmoveto{\pgfqpoint{1.914830in}{0.967594in}}%
\pgfpathlineto{\pgfqpoint{2.143341in}{2.291288in}}%
\pgfpathlineto{\pgfqpoint{1.914830in}{0.967594in}}%
\pgfpathclose%
\pgfusepath{fill}%
\end{pgfscope}%
\begin{pgfscope}%
\pgfpathrectangle{\pgfqpoint{0.017500in}{0.017500in}}{\pgfqpoint{6.218282in}{3.580372in}} %
\pgfusepath{clip}%
\pgfsetbuttcap%
\pgfsetroundjoin%
\definecolor{currentfill}{rgb}{0.243520,0.319189,0.771672}%
\pgfsetfillcolor{currentfill}%
\pgfsetlinewidth{0.000000pt}%
\definecolor{currentstroke}{rgb}{0.000000,0.000000,0.000000}%
\pgfsetstrokecolor{currentstroke}%
\pgfsetdash{}{0pt}%
\pgfpathmoveto{\pgfqpoint{1.914830in}{0.967594in}}%
\pgfpathlineto{\pgfqpoint{2.143341in}{2.291288in}}%
\pgfpathlineto{\pgfqpoint{1.914830in}{0.967594in}}%
\pgfpathclose%
\pgfusepath{fill}%
\end{pgfscope}%
\begin{pgfscope}%
\pgfpathrectangle{\pgfqpoint{0.017500in}{0.017500in}}{\pgfqpoint{6.218282in}{3.580372in}} %
\pgfusepath{clip}%
\pgfsetbuttcap%
\pgfsetroundjoin%
\definecolor{currentfill}{rgb}{0.243520,0.319189,0.771672}%
\pgfsetfillcolor{currentfill}%
\pgfsetlinewidth{0.000000pt}%
\definecolor{currentstroke}{rgb}{0.000000,0.000000,0.000000}%
\pgfsetstrokecolor{currentstroke}%
\pgfsetdash{}{0pt}%
\pgfpathmoveto{\pgfqpoint{1.914830in}{0.967594in}}%
\pgfpathlineto{\pgfqpoint{2.143341in}{2.291288in}}%
\pgfpathlineto{\pgfqpoint{1.914830in}{0.967594in}}%
\pgfpathclose%
\pgfusepath{fill}%
\end{pgfscope}%
\begin{pgfscope}%
\pgfpathrectangle{\pgfqpoint{0.017500in}{0.017500in}}{\pgfqpoint{6.218282in}{3.580372in}} %
\pgfusepath{clip}%
\pgfsetbuttcap%
\pgfsetroundjoin%
\definecolor{currentfill}{rgb}{0.243520,0.319189,0.771672}%
\pgfsetfillcolor{currentfill}%
\pgfsetlinewidth{0.000000pt}%
\definecolor{currentstroke}{rgb}{0.000000,0.000000,0.000000}%
\pgfsetstrokecolor{currentstroke}%
\pgfsetdash{}{0pt}%
\pgfpathmoveto{\pgfqpoint{1.914830in}{0.967594in}}%
\pgfpathlineto{\pgfqpoint{2.143341in}{2.291288in}}%
\pgfpathlineto{\pgfqpoint{1.914830in}{0.967594in}}%
\pgfpathclose%
\pgfusepath{fill}%
\end{pgfscope}%
\begin{pgfscope}%
\pgfpathrectangle{\pgfqpoint{0.017500in}{0.017500in}}{\pgfqpoint{6.218282in}{3.580372in}} %
\pgfusepath{clip}%
\pgfsetbuttcap%
\pgfsetroundjoin%
\definecolor{currentfill}{rgb}{0.243520,0.319189,0.771672}%
\pgfsetfillcolor{currentfill}%
\pgfsetlinewidth{0.000000pt}%
\definecolor{currentstroke}{rgb}{0.000000,0.000000,0.000000}%
\pgfsetstrokecolor{currentstroke}%
\pgfsetdash{}{0pt}%
\pgfpathmoveto{\pgfqpoint{1.914830in}{0.967594in}}%
\pgfpathlineto{\pgfqpoint{2.143341in}{2.291288in}}%
\pgfpathlineto{\pgfqpoint{1.914830in}{0.967594in}}%
\pgfpathclose%
\pgfusepath{fill}%
\end{pgfscope}%
\begin{pgfscope}%
\pgfpathrectangle{\pgfqpoint{0.017500in}{0.017500in}}{\pgfqpoint{6.218282in}{3.580372in}} %
\pgfusepath{clip}%
\pgfsetbuttcap%
\pgfsetroundjoin%
\definecolor{currentfill}{rgb}{0.243520,0.319189,0.771672}%
\pgfsetfillcolor{currentfill}%
\pgfsetlinewidth{0.000000pt}%
\definecolor{currentstroke}{rgb}{0.000000,0.000000,0.000000}%
\pgfsetstrokecolor{currentstroke}%
\pgfsetdash{}{0pt}%
\pgfpathmoveto{\pgfqpoint{1.914830in}{0.967594in}}%
\pgfpathlineto{\pgfqpoint{2.143341in}{2.291288in}}%
\pgfpathlineto{\pgfqpoint{1.914830in}{0.967594in}}%
\pgfpathclose%
\pgfusepath{fill}%
\end{pgfscope}%
\begin{pgfscope}%
\pgfpathrectangle{\pgfqpoint{0.017500in}{0.017500in}}{\pgfqpoint{6.218282in}{3.580372in}} %
\pgfusepath{clip}%
\pgfsetbuttcap%
\pgfsetroundjoin%
\definecolor{currentfill}{rgb}{0.243520,0.319189,0.771672}%
\pgfsetfillcolor{currentfill}%
\pgfsetlinewidth{0.000000pt}%
\definecolor{currentstroke}{rgb}{0.000000,0.000000,0.000000}%
\pgfsetstrokecolor{currentstroke}%
\pgfsetdash{}{0pt}%
\pgfpathmoveto{\pgfqpoint{1.914830in}{0.967594in}}%
\pgfpathlineto{\pgfqpoint{2.143341in}{2.291288in}}%
\pgfpathlineto{\pgfqpoint{1.914830in}{0.967594in}}%
\pgfpathclose%
\pgfusepath{fill}%
\end{pgfscope}%
\begin{pgfscope}%
\pgfpathrectangle{\pgfqpoint{0.017500in}{0.017500in}}{\pgfqpoint{6.218282in}{3.580372in}} %
\pgfusepath{clip}%
\pgfsetbuttcap%
\pgfsetroundjoin%
\definecolor{currentfill}{rgb}{0.243520,0.319189,0.771672}%
\pgfsetfillcolor{currentfill}%
\pgfsetlinewidth{0.000000pt}%
\definecolor{currentstroke}{rgb}{0.000000,0.000000,0.000000}%
\pgfsetstrokecolor{currentstroke}%
\pgfsetdash{}{0pt}%
\pgfpathmoveto{\pgfqpoint{1.914830in}{0.967594in}}%
\pgfpathlineto{\pgfqpoint{2.143341in}{2.291288in}}%
\pgfpathlineto{\pgfqpoint{1.914830in}{0.967594in}}%
\pgfpathclose%
\pgfusepath{fill}%
\end{pgfscope}%
\begin{pgfscope}%
\pgfpathrectangle{\pgfqpoint{0.017500in}{0.017500in}}{\pgfqpoint{6.218282in}{3.580372in}} %
\pgfusepath{clip}%
\pgfsetbuttcap%
\pgfsetroundjoin%
\definecolor{currentfill}{rgb}{0.243520,0.319189,0.771672}%
\pgfsetfillcolor{currentfill}%
\pgfsetlinewidth{0.000000pt}%
\definecolor{currentstroke}{rgb}{0.000000,0.000000,0.000000}%
\pgfsetstrokecolor{currentstroke}%
\pgfsetdash{}{0pt}%
\pgfpathmoveto{\pgfqpoint{1.914830in}{0.967594in}}%
\pgfpathlineto{\pgfqpoint{2.143341in}{2.291288in}}%
\pgfpathlineto{\pgfqpoint{1.914830in}{0.967594in}}%
\pgfpathclose%
\pgfusepath{fill}%
\end{pgfscope}%
\begin{pgfscope}%
\pgfpathrectangle{\pgfqpoint{0.017500in}{0.017500in}}{\pgfqpoint{6.218282in}{3.580372in}} %
\pgfusepath{clip}%
\pgfsetbuttcap%
\pgfsetroundjoin%
\definecolor{currentfill}{rgb}{0.243520,0.319189,0.771672}%
\pgfsetfillcolor{currentfill}%
\pgfsetlinewidth{0.000000pt}%
\definecolor{currentstroke}{rgb}{0.000000,0.000000,0.000000}%
\pgfsetstrokecolor{currentstroke}%
\pgfsetdash{}{0pt}%
\pgfpathmoveto{\pgfqpoint{1.914830in}{0.967594in}}%
\pgfpathlineto{\pgfqpoint{2.143341in}{2.291288in}}%
\pgfpathlineto{\pgfqpoint{1.914830in}{0.967594in}}%
\pgfpathclose%
\pgfusepath{fill}%
\end{pgfscope}%
\begin{pgfscope}%
\pgfpathrectangle{\pgfqpoint{0.017500in}{0.017500in}}{\pgfqpoint{6.218282in}{3.580372in}} %
\pgfusepath{clip}%
\pgfsetbuttcap%
\pgfsetroundjoin%
\definecolor{currentfill}{rgb}{0.243520,0.319189,0.771672}%
\pgfsetfillcolor{currentfill}%
\pgfsetlinewidth{0.000000pt}%
\definecolor{currentstroke}{rgb}{0.000000,0.000000,0.000000}%
\pgfsetstrokecolor{currentstroke}%
\pgfsetdash{}{0pt}%
\pgfpathmoveto{\pgfqpoint{1.914830in}{0.967594in}}%
\pgfpathlineto{\pgfqpoint{2.143341in}{2.291288in}}%
\pgfpathlineto{\pgfqpoint{1.914830in}{0.967594in}}%
\pgfpathclose%
\pgfusepath{fill}%
\end{pgfscope}%
\begin{pgfscope}%
\pgfpathrectangle{\pgfqpoint{0.017500in}{0.017500in}}{\pgfqpoint{6.218282in}{3.580372in}} %
\pgfusepath{clip}%
\pgfsetbuttcap%
\pgfsetroundjoin%
\definecolor{currentfill}{rgb}{0.243520,0.319189,0.771672}%
\pgfsetfillcolor{currentfill}%
\pgfsetlinewidth{0.000000pt}%
\definecolor{currentstroke}{rgb}{0.000000,0.000000,0.000000}%
\pgfsetstrokecolor{currentstroke}%
\pgfsetdash{}{0pt}%
\pgfpathmoveto{\pgfqpoint{1.914830in}{0.967594in}}%
\pgfpathlineto{\pgfqpoint{2.143341in}{2.291288in}}%
\pgfpathlineto{\pgfqpoint{1.914830in}{0.967594in}}%
\pgfpathclose%
\pgfusepath{fill}%
\end{pgfscope}%
\begin{pgfscope}%
\pgfpathrectangle{\pgfqpoint{0.017500in}{0.017500in}}{\pgfqpoint{6.218282in}{3.580372in}} %
\pgfusepath{clip}%
\pgfsetbuttcap%
\pgfsetroundjoin%
\definecolor{currentfill}{rgb}{0.243520,0.319189,0.771672}%
\pgfsetfillcolor{currentfill}%
\pgfsetlinewidth{0.000000pt}%
\definecolor{currentstroke}{rgb}{0.000000,0.000000,0.000000}%
\pgfsetstrokecolor{currentstroke}%
\pgfsetdash{}{0pt}%
\pgfpathmoveto{\pgfqpoint{1.914830in}{0.967594in}}%
\pgfpathlineto{\pgfqpoint{2.143341in}{2.291288in}}%
\pgfpathlineto{\pgfqpoint{1.914830in}{0.967594in}}%
\pgfpathclose%
\pgfusepath{fill}%
\end{pgfscope}%
\begin{pgfscope}%
\pgfpathrectangle{\pgfqpoint{0.017500in}{0.017500in}}{\pgfqpoint{6.218282in}{3.580372in}} %
\pgfusepath{clip}%
\pgfsetbuttcap%
\pgfsetroundjoin%
\definecolor{currentfill}{rgb}{0.243520,0.319189,0.771672}%
\pgfsetfillcolor{currentfill}%
\pgfsetlinewidth{0.000000pt}%
\definecolor{currentstroke}{rgb}{0.000000,0.000000,0.000000}%
\pgfsetstrokecolor{currentstroke}%
\pgfsetdash{}{0pt}%
\pgfpathmoveto{\pgfqpoint{1.914830in}{0.967594in}}%
\pgfpathlineto{\pgfqpoint{2.143341in}{2.291288in}}%
\pgfpathlineto{\pgfqpoint{1.914830in}{0.967594in}}%
\pgfpathclose%
\pgfusepath{fill}%
\end{pgfscope}%
\begin{pgfscope}%
\pgfpathrectangle{\pgfqpoint{0.017500in}{0.017500in}}{\pgfqpoint{6.218282in}{3.580372in}} %
\pgfusepath{clip}%
\pgfsetbuttcap%
\pgfsetroundjoin%
\definecolor{currentfill}{rgb}{0.243520,0.319189,0.771672}%
\pgfsetfillcolor{currentfill}%
\pgfsetlinewidth{0.000000pt}%
\definecolor{currentstroke}{rgb}{0.000000,0.000000,0.000000}%
\pgfsetstrokecolor{currentstroke}%
\pgfsetdash{}{0pt}%
\pgfpathmoveto{\pgfqpoint{1.914830in}{0.967594in}}%
\pgfpathlineto{\pgfqpoint{2.143341in}{2.291288in}}%
\pgfpathlineto{\pgfqpoint{1.914830in}{0.967594in}}%
\pgfpathclose%
\pgfusepath{fill}%
\end{pgfscope}%
\begin{pgfscope}%
\pgfpathrectangle{\pgfqpoint{0.017500in}{0.017500in}}{\pgfqpoint{6.218282in}{3.580372in}} %
\pgfusepath{clip}%
\pgfsetbuttcap%
\pgfsetroundjoin%
\definecolor{currentfill}{rgb}{0.243520,0.319189,0.771672}%
\pgfsetfillcolor{currentfill}%
\pgfsetlinewidth{0.000000pt}%
\definecolor{currentstroke}{rgb}{0.000000,0.000000,0.000000}%
\pgfsetstrokecolor{currentstroke}%
\pgfsetdash{}{0pt}%
\pgfpathmoveto{\pgfqpoint{1.914830in}{0.967594in}}%
\pgfpathlineto{\pgfqpoint{2.143341in}{2.291288in}}%
\pgfpathlineto{\pgfqpoint{1.914830in}{0.967594in}}%
\pgfpathclose%
\pgfusepath{fill}%
\end{pgfscope}%
\begin{pgfscope}%
\pgfpathrectangle{\pgfqpoint{0.017500in}{0.017500in}}{\pgfqpoint{6.218282in}{3.580372in}} %
\pgfusepath{clip}%
\pgfsetbuttcap%
\pgfsetroundjoin%
\definecolor{currentfill}{rgb}{0.243520,0.319189,0.771672}%
\pgfsetfillcolor{currentfill}%
\pgfsetlinewidth{0.000000pt}%
\definecolor{currentstroke}{rgb}{0.000000,0.000000,0.000000}%
\pgfsetstrokecolor{currentstroke}%
\pgfsetdash{}{0pt}%
\pgfpathmoveto{\pgfqpoint{1.914830in}{0.967594in}}%
\pgfpathlineto{\pgfqpoint{2.143341in}{2.291288in}}%
\pgfpathlineto{\pgfqpoint{1.914830in}{0.967594in}}%
\pgfpathclose%
\pgfusepath{fill}%
\end{pgfscope}%
\begin{pgfscope}%
\pgfpathrectangle{\pgfqpoint{0.017500in}{0.017500in}}{\pgfqpoint{6.218282in}{3.580372in}} %
\pgfusepath{clip}%
\pgfsetbuttcap%
\pgfsetroundjoin%
\definecolor{currentfill}{rgb}{0.243520,0.319189,0.771672}%
\pgfsetfillcolor{currentfill}%
\pgfsetlinewidth{0.000000pt}%
\definecolor{currentstroke}{rgb}{0.000000,0.000000,0.000000}%
\pgfsetstrokecolor{currentstroke}%
\pgfsetdash{}{0pt}%
\pgfpathmoveto{\pgfqpoint{1.914830in}{0.967594in}}%
\pgfpathlineto{\pgfqpoint{2.143341in}{2.291288in}}%
\pgfpathlineto{\pgfqpoint{1.914830in}{0.967594in}}%
\pgfpathclose%
\pgfusepath{fill}%
\end{pgfscope}%
\begin{pgfscope}%
\pgfpathrectangle{\pgfqpoint{0.017500in}{0.017500in}}{\pgfqpoint{6.218282in}{3.580372in}} %
\pgfusepath{clip}%
\pgfsetbuttcap%
\pgfsetroundjoin%
\definecolor{currentfill}{rgb}{0.243520,0.319189,0.771672}%
\pgfsetfillcolor{currentfill}%
\pgfsetlinewidth{0.000000pt}%
\definecolor{currentstroke}{rgb}{0.000000,0.000000,0.000000}%
\pgfsetstrokecolor{currentstroke}%
\pgfsetdash{}{0pt}%
\pgfpathmoveto{\pgfqpoint{1.914830in}{0.967594in}}%
\pgfpathlineto{\pgfqpoint{2.143341in}{2.291288in}}%
\pgfpathlineto{\pgfqpoint{1.914830in}{0.967594in}}%
\pgfpathclose%
\pgfusepath{fill}%
\end{pgfscope}%
\begin{pgfscope}%
\pgfpathrectangle{\pgfqpoint{0.017500in}{0.017500in}}{\pgfqpoint{6.218282in}{3.580372in}} %
\pgfusepath{clip}%
\pgfsetbuttcap%
\pgfsetroundjoin%
\definecolor{currentfill}{rgb}{0.243520,0.319189,0.771672}%
\pgfsetfillcolor{currentfill}%
\pgfsetlinewidth{0.000000pt}%
\definecolor{currentstroke}{rgb}{0.000000,0.000000,0.000000}%
\pgfsetstrokecolor{currentstroke}%
\pgfsetdash{}{0pt}%
\pgfpathmoveto{\pgfqpoint{1.914830in}{0.967594in}}%
\pgfpathlineto{\pgfqpoint{2.143341in}{2.291288in}}%
\pgfpathlineto{\pgfqpoint{1.914830in}{0.967594in}}%
\pgfpathclose%
\pgfusepath{fill}%
\end{pgfscope}%
\begin{pgfscope}%
\pgfpathrectangle{\pgfqpoint{0.017500in}{0.017500in}}{\pgfqpoint{6.218282in}{3.580372in}} %
\pgfusepath{clip}%
\pgfsetbuttcap%
\pgfsetroundjoin%
\definecolor{currentfill}{rgb}{0.243520,0.319189,0.771672}%
\pgfsetfillcolor{currentfill}%
\pgfsetlinewidth{0.000000pt}%
\definecolor{currentstroke}{rgb}{0.000000,0.000000,0.000000}%
\pgfsetstrokecolor{currentstroke}%
\pgfsetdash{}{0pt}%
\pgfpathmoveto{\pgfqpoint{1.914830in}{0.967594in}}%
\pgfpathlineto{\pgfqpoint{2.143341in}{2.291288in}}%
\pgfpathlineto{\pgfqpoint{1.914830in}{0.967594in}}%
\pgfpathclose%
\pgfusepath{fill}%
\end{pgfscope}%
\begin{pgfscope}%
\pgfpathrectangle{\pgfqpoint{0.017500in}{0.017500in}}{\pgfqpoint{6.218282in}{3.580372in}} %
\pgfusepath{clip}%
\pgfsetbuttcap%
\pgfsetroundjoin%
\definecolor{currentfill}{rgb}{0.243520,0.319189,0.771672}%
\pgfsetfillcolor{currentfill}%
\pgfsetlinewidth{0.000000pt}%
\definecolor{currentstroke}{rgb}{0.000000,0.000000,0.000000}%
\pgfsetstrokecolor{currentstroke}%
\pgfsetdash{}{0pt}%
\pgfpathmoveto{\pgfqpoint{1.914830in}{0.967594in}}%
\pgfpathlineto{\pgfqpoint{2.143341in}{2.291288in}}%
\pgfpathlineto{\pgfqpoint{1.914830in}{0.967594in}}%
\pgfpathclose%
\pgfusepath{fill}%
\end{pgfscope}%
\begin{pgfscope}%
\pgfpathrectangle{\pgfqpoint{0.017500in}{0.017500in}}{\pgfqpoint{6.218282in}{3.580372in}} %
\pgfusepath{clip}%
\pgfsetbuttcap%
\pgfsetroundjoin%
\definecolor{currentfill}{rgb}{0.243520,0.319189,0.771672}%
\pgfsetfillcolor{currentfill}%
\pgfsetlinewidth{0.000000pt}%
\definecolor{currentstroke}{rgb}{0.000000,0.000000,0.000000}%
\pgfsetstrokecolor{currentstroke}%
\pgfsetdash{}{0pt}%
\pgfpathmoveto{\pgfqpoint{1.914830in}{0.967594in}}%
\pgfpathlineto{\pgfqpoint{2.143341in}{2.291288in}}%
\pgfpathlineto{\pgfqpoint{1.914830in}{0.967594in}}%
\pgfpathclose%
\pgfusepath{fill}%
\end{pgfscope}%
\begin{pgfscope}%
\pgfpathrectangle{\pgfqpoint{0.017500in}{0.017500in}}{\pgfqpoint{6.218282in}{3.580372in}} %
\pgfusepath{clip}%
\pgfsetbuttcap%
\pgfsetroundjoin%
\definecolor{currentfill}{rgb}{0.243520,0.319189,0.771672}%
\pgfsetfillcolor{currentfill}%
\pgfsetlinewidth{0.000000pt}%
\definecolor{currentstroke}{rgb}{0.000000,0.000000,0.000000}%
\pgfsetstrokecolor{currentstroke}%
\pgfsetdash{}{0pt}%
\pgfpathmoveto{\pgfqpoint{1.914830in}{0.967594in}}%
\pgfpathlineto{\pgfqpoint{2.143341in}{2.291288in}}%
\pgfpathlineto{\pgfqpoint{1.914830in}{0.967594in}}%
\pgfpathclose%
\pgfusepath{fill}%
\end{pgfscope}%
\begin{pgfscope}%
\pgfpathrectangle{\pgfqpoint{0.017500in}{0.017500in}}{\pgfqpoint{6.218282in}{3.580372in}} %
\pgfusepath{clip}%
\pgfsetbuttcap%
\pgfsetroundjoin%
\definecolor{currentfill}{rgb}{0.758112,0.168122,0.188827}%
\pgfsetfillcolor{currentfill}%
\pgfsetlinewidth{0.000000pt}%
\definecolor{currentstroke}{rgb}{0.000000,0.000000,0.000000}%
\pgfsetstrokecolor{currentstroke}%
\pgfsetdash{}{0pt}%
\pgfpathmoveto{\pgfqpoint{3.116513in}{2.603812in}}%
\pgfpathlineto{\pgfqpoint{3.372093in}{2.761754in}}%
\pgfpathlineto{\pgfqpoint{3.116513in}{2.603812in}}%
\pgfpathclose%
\pgfusepath{fill}%
\end{pgfscope}%
\begin{pgfscope}%
\pgfpathrectangle{\pgfqpoint{0.017500in}{0.017500in}}{\pgfqpoint{6.218282in}{3.580372in}} %
\pgfusepath{clip}%
\pgfsetbuttcap%
\pgfsetroundjoin%
\definecolor{currentfill}{rgb}{0.758112,0.168122,0.188827}%
\pgfsetfillcolor{currentfill}%
\pgfsetlinewidth{0.000000pt}%
\definecolor{currentstroke}{rgb}{0.000000,0.000000,0.000000}%
\pgfsetstrokecolor{currentstroke}%
\pgfsetdash{}{0pt}%
\pgfpathmoveto{\pgfqpoint{3.116513in}{2.603812in}}%
\pgfpathlineto{\pgfqpoint{3.372093in}{2.761754in}}%
\pgfpathlineto{\pgfqpoint{3.116513in}{2.603812in}}%
\pgfpathclose%
\pgfusepath{fill}%
\end{pgfscope}%
\begin{pgfscope}%
\pgfpathrectangle{\pgfqpoint{0.017500in}{0.017500in}}{\pgfqpoint{6.218282in}{3.580372in}} %
\pgfusepath{clip}%
\pgfsetbuttcap%
\pgfsetroundjoin%
\definecolor{currentfill}{rgb}{0.758112,0.168122,0.188827}%
\pgfsetfillcolor{currentfill}%
\pgfsetlinewidth{0.000000pt}%
\definecolor{currentstroke}{rgb}{0.000000,0.000000,0.000000}%
\pgfsetstrokecolor{currentstroke}%
\pgfsetdash{}{0pt}%
\pgfpathmoveto{\pgfqpoint{3.116513in}{2.603812in}}%
\pgfpathlineto{\pgfqpoint{3.372093in}{2.761754in}}%
\pgfpathlineto{\pgfqpoint{3.116513in}{2.603812in}}%
\pgfpathclose%
\pgfusepath{fill}%
\end{pgfscope}%
\begin{pgfscope}%
\pgfpathrectangle{\pgfqpoint{0.017500in}{0.017500in}}{\pgfqpoint{6.218282in}{3.580372in}} %
\pgfusepath{clip}%
\pgfsetbuttcap%
\pgfsetroundjoin%
\definecolor{currentfill}{rgb}{0.758112,0.168122,0.188827}%
\pgfsetfillcolor{currentfill}%
\pgfsetlinewidth{0.000000pt}%
\definecolor{currentstroke}{rgb}{0.000000,0.000000,0.000000}%
\pgfsetstrokecolor{currentstroke}%
\pgfsetdash{}{0pt}%
\pgfpathmoveto{\pgfqpoint{3.116513in}{2.603812in}}%
\pgfpathlineto{\pgfqpoint{3.372093in}{2.761754in}}%
\pgfpathlineto{\pgfqpoint{3.116513in}{2.603812in}}%
\pgfpathclose%
\pgfusepath{fill}%
\end{pgfscope}%
\begin{pgfscope}%
\pgfpathrectangle{\pgfqpoint{0.017500in}{0.017500in}}{\pgfqpoint{6.218282in}{3.580372in}} %
\pgfusepath{clip}%
\pgfsetbuttcap%
\pgfsetroundjoin%
\definecolor{currentfill}{rgb}{0.758112,0.168122,0.188827}%
\pgfsetfillcolor{currentfill}%
\pgfsetlinewidth{0.000000pt}%
\definecolor{currentstroke}{rgb}{0.000000,0.000000,0.000000}%
\pgfsetstrokecolor{currentstroke}%
\pgfsetdash{}{0pt}%
\pgfpathmoveto{\pgfqpoint{3.116513in}{2.603812in}}%
\pgfpathlineto{\pgfqpoint{3.372093in}{2.761754in}}%
\pgfpathlineto{\pgfqpoint{3.116513in}{2.603812in}}%
\pgfpathclose%
\pgfusepath{fill}%
\end{pgfscope}%
\begin{pgfscope}%
\pgfpathrectangle{\pgfqpoint{0.017500in}{0.017500in}}{\pgfqpoint{6.218282in}{3.580372in}} %
\pgfusepath{clip}%
\pgfsetbuttcap%
\pgfsetroundjoin%
\definecolor{currentfill}{rgb}{0.758112,0.168122,0.188827}%
\pgfsetfillcolor{currentfill}%
\pgfsetlinewidth{0.000000pt}%
\definecolor{currentstroke}{rgb}{0.000000,0.000000,0.000000}%
\pgfsetstrokecolor{currentstroke}%
\pgfsetdash{}{0pt}%
\pgfpathmoveto{\pgfqpoint{3.116513in}{2.603812in}}%
\pgfpathlineto{\pgfqpoint{3.372093in}{2.761754in}}%
\pgfpathlineto{\pgfqpoint{3.116513in}{2.603812in}}%
\pgfpathclose%
\pgfusepath{fill}%
\end{pgfscope}%
\begin{pgfscope}%
\pgfpathrectangle{\pgfqpoint{0.017500in}{0.017500in}}{\pgfqpoint{6.218282in}{3.580372in}} %
\pgfusepath{clip}%
\pgfsetbuttcap%
\pgfsetroundjoin%
\definecolor{currentfill}{rgb}{0.758112,0.168122,0.188827}%
\pgfsetfillcolor{currentfill}%
\pgfsetlinewidth{0.000000pt}%
\definecolor{currentstroke}{rgb}{0.000000,0.000000,0.000000}%
\pgfsetstrokecolor{currentstroke}%
\pgfsetdash{}{0pt}%
\pgfpathmoveto{\pgfqpoint{3.116513in}{2.603812in}}%
\pgfpathlineto{\pgfqpoint{3.372093in}{2.761754in}}%
\pgfpathlineto{\pgfqpoint{3.116513in}{2.603812in}}%
\pgfpathclose%
\pgfusepath{fill}%
\end{pgfscope}%
\begin{pgfscope}%
\pgfpathrectangle{\pgfqpoint{0.017500in}{0.017500in}}{\pgfqpoint{6.218282in}{3.580372in}} %
\pgfusepath{clip}%
\pgfsetbuttcap%
\pgfsetroundjoin%
\definecolor{currentfill}{rgb}{0.758112,0.168122,0.188827}%
\pgfsetfillcolor{currentfill}%
\pgfsetlinewidth{0.000000pt}%
\definecolor{currentstroke}{rgb}{0.000000,0.000000,0.000000}%
\pgfsetstrokecolor{currentstroke}%
\pgfsetdash{}{0pt}%
\pgfpathmoveto{\pgfqpoint{3.116513in}{2.603812in}}%
\pgfpathlineto{\pgfqpoint{3.372093in}{2.761754in}}%
\pgfpathlineto{\pgfqpoint{3.116513in}{2.603812in}}%
\pgfpathclose%
\pgfusepath{fill}%
\end{pgfscope}%
\begin{pgfscope}%
\pgfpathrectangle{\pgfqpoint{0.017500in}{0.017500in}}{\pgfqpoint{6.218282in}{3.580372in}} %
\pgfusepath{clip}%
\pgfsetbuttcap%
\pgfsetroundjoin%
\definecolor{currentfill}{rgb}{0.758112,0.168122,0.188827}%
\pgfsetfillcolor{currentfill}%
\pgfsetlinewidth{0.000000pt}%
\definecolor{currentstroke}{rgb}{0.000000,0.000000,0.000000}%
\pgfsetstrokecolor{currentstroke}%
\pgfsetdash{}{0pt}%
\pgfpathmoveto{\pgfqpoint{3.116513in}{2.603812in}}%
\pgfpathlineto{\pgfqpoint{3.372093in}{2.761754in}}%
\pgfpathlineto{\pgfqpoint{3.116513in}{2.603812in}}%
\pgfpathclose%
\pgfusepath{fill}%
\end{pgfscope}%
\begin{pgfscope}%
\pgfpathrectangle{\pgfqpoint{0.017500in}{0.017500in}}{\pgfqpoint{6.218282in}{3.580372in}} %
\pgfusepath{clip}%
\pgfsetbuttcap%
\pgfsetroundjoin%
\definecolor{currentfill}{rgb}{0.758112,0.168122,0.188827}%
\pgfsetfillcolor{currentfill}%
\pgfsetlinewidth{0.000000pt}%
\definecolor{currentstroke}{rgb}{0.000000,0.000000,0.000000}%
\pgfsetstrokecolor{currentstroke}%
\pgfsetdash{}{0pt}%
\pgfpathmoveto{\pgfqpoint{3.116513in}{2.603812in}}%
\pgfpathlineto{\pgfqpoint{3.372093in}{2.761754in}}%
\pgfpathlineto{\pgfqpoint{3.116513in}{2.603812in}}%
\pgfpathclose%
\pgfusepath{fill}%
\end{pgfscope}%
\begin{pgfscope}%
\pgfpathrectangle{\pgfqpoint{0.017500in}{0.017500in}}{\pgfqpoint{6.218282in}{3.580372in}} %
\pgfusepath{clip}%
\pgfsetbuttcap%
\pgfsetroundjoin%
\definecolor{currentfill}{rgb}{0.758112,0.168122,0.188827}%
\pgfsetfillcolor{currentfill}%
\pgfsetlinewidth{0.000000pt}%
\definecolor{currentstroke}{rgb}{0.000000,0.000000,0.000000}%
\pgfsetstrokecolor{currentstroke}%
\pgfsetdash{}{0pt}%
\pgfpathmoveto{\pgfqpoint{3.116513in}{2.603812in}}%
\pgfpathlineto{\pgfqpoint{3.372093in}{2.761754in}}%
\pgfpathlineto{\pgfqpoint{3.116513in}{2.603812in}}%
\pgfpathclose%
\pgfusepath{fill}%
\end{pgfscope}%
\begin{pgfscope}%
\pgfpathrectangle{\pgfqpoint{0.017500in}{0.017500in}}{\pgfqpoint{6.218282in}{3.580372in}} %
\pgfusepath{clip}%
\pgfsetbuttcap%
\pgfsetroundjoin%
\definecolor{currentfill}{rgb}{0.758112,0.168122,0.188827}%
\pgfsetfillcolor{currentfill}%
\pgfsetlinewidth{0.000000pt}%
\definecolor{currentstroke}{rgb}{0.000000,0.000000,0.000000}%
\pgfsetstrokecolor{currentstroke}%
\pgfsetdash{}{0pt}%
\pgfpathmoveto{\pgfqpoint{3.116513in}{2.603812in}}%
\pgfpathlineto{\pgfqpoint{3.372093in}{2.761754in}}%
\pgfpathlineto{\pgfqpoint{3.116513in}{2.603812in}}%
\pgfpathclose%
\pgfusepath{fill}%
\end{pgfscope}%
\begin{pgfscope}%
\pgfpathrectangle{\pgfqpoint{0.017500in}{0.017500in}}{\pgfqpoint{6.218282in}{3.580372in}} %
\pgfusepath{clip}%
\pgfsetbuttcap%
\pgfsetroundjoin%
\definecolor{currentfill}{rgb}{0.758112,0.168122,0.188827}%
\pgfsetfillcolor{currentfill}%
\pgfsetlinewidth{0.000000pt}%
\definecolor{currentstroke}{rgb}{0.000000,0.000000,0.000000}%
\pgfsetstrokecolor{currentstroke}%
\pgfsetdash{}{0pt}%
\pgfpathmoveto{\pgfqpoint{3.116513in}{2.603812in}}%
\pgfpathlineto{\pgfqpoint{3.372093in}{2.761754in}}%
\pgfpathlineto{\pgfqpoint{3.116513in}{2.603812in}}%
\pgfpathclose%
\pgfusepath{fill}%
\end{pgfscope}%
\begin{pgfscope}%
\pgfpathrectangle{\pgfqpoint{0.017500in}{0.017500in}}{\pgfqpoint{6.218282in}{3.580372in}} %
\pgfusepath{clip}%
\pgfsetbuttcap%
\pgfsetroundjoin%
\definecolor{currentfill}{rgb}{0.758112,0.168122,0.188827}%
\pgfsetfillcolor{currentfill}%
\pgfsetlinewidth{0.000000pt}%
\definecolor{currentstroke}{rgb}{0.000000,0.000000,0.000000}%
\pgfsetstrokecolor{currentstroke}%
\pgfsetdash{}{0pt}%
\pgfpathmoveto{\pgfqpoint{3.116513in}{2.603812in}}%
\pgfpathlineto{\pgfqpoint{3.372093in}{2.761754in}}%
\pgfpathlineto{\pgfqpoint{3.116513in}{2.603812in}}%
\pgfpathclose%
\pgfusepath{fill}%
\end{pgfscope}%
\begin{pgfscope}%
\pgfpathrectangle{\pgfqpoint{0.017500in}{0.017500in}}{\pgfqpoint{6.218282in}{3.580372in}} %
\pgfusepath{clip}%
\pgfsetbuttcap%
\pgfsetroundjoin%
\definecolor{currentfill}{rgb}{0.758112,0.168122,0.188827}%
\pgfsetfillcolor{currentfill}%
\pgfsetlinewidth{0.000000pt}%
\definecolor{currentstroke}{rgb}{0.000000,0.000000,0.000000}%
\pgfsetstrokecolor{currentstroke}%
\pgfsetdash{}{0pt}%
\pgfpathmoveto{\pgfqpoint{3.116513in}{2.603812in}}%
\pgfpathlineto{\pgfqpoint{3.372093in}{2.761754in}}%
\pgfpathlineto{\pgfqpoint{3.116513in}{2.603812in}}%
\pgfpathclose%
\pgfusepath{fill}%
\end{pgfscope}%
\begin{pgfscope}%
\pgfpathrectangle{\pgfqpoint{0.017500in}{0.017500in}}{\pgfqpoint{6.218282in}{3.580372in}} %
\pgfusepath{clip}%
\pgfsetbuttcap%
\pgfsetroundjoin%
\definecolor{currentfill}{rgb}{0.758112,0.168122,0.188827}%
\pgfsetfillcolor{currentfill}%
\pgfsetlinewidth{0.000000pt}%
\definecolor{currentstroke}{rgb}{0.000000,0.000000,0.000000}%
\pgfsetstrokecolor{currentstroke}%
\pgfsetdash{}{0pt}%
\pgfpathmoveto{\pgfqpoint{3.116513in}{2.603812in}}%
\pgfpathlineto{\pgfqpoint{3.372093in}{2.761754in}}%
\pgfpathlineto{\pgfqpoint{3.116513in}{2.603812in}}%
\pgfpathclose%
\pgfusepath{fill}%
\end{pgfscope}%
\begin{pgfscope}%
\pgfpathrectangle{\pgfqpoint{0.017500in}{0.017500in}}{\pgfqpoint{6.218282in}{3.580372in}} %
\pgfusepath{clip}%
\pgfsetbuttcap%
\pgfsetroundjoin%
\definecolor{currentfill}{rgb}{0.758112,0.168122,0.188827}%
\pgfsetfillcolor{currentfill}%
\pgfsetlinewidth{0.000000pt}%
\definecolor{currentstroke}{rgb}{0.000000,0.000000,0.000000}%
\pgfsetstrokecolor{currentstroke}%
\pgfsetdash{}{0pt}%
\pgfpathmoveto{\pgfqpoint{3.116513in}{2.603812in}}%
\pgfpathlineto{\pgfqpoint{3.372093in}{2.761754in}}%
\pgfpathlineto{\pgfqpoint{3.116513in}{2.603812in}}%
\pgfpathclose%
\pgfusepath{fill}%
\end{pgfscope}%
\begin{pgfscope}%
\pgfpathrectangle{\pgfqpoint{0.017500in}{0.017500in}}{\pgfqpoint{6.218282in}{3.580372in}} %
\pgfusepath{clip}%
\pgfsetbuttcap%
\pgfsetroundjoin%
\definecolor{currentfill}{rgb}{0.758112,0.168122,0.188827}%
\pgfsetfillcolor{currentfill}%
\pgfsetlinewidth{0.000000pt}%
\definecolor{currentstroke}{rgb}{0.000000,0.000000,0.000000}%
\pgfsetstrokecolor{currentstroke}%
\pgfsetdash{}{0pt}%
\pgfpathmoveto{\pgfqpoint{3.116513in}{2.603812in}}%
\pgfpathlineto{\pgfqpoint{3.372093in}{2.761754in}}%
\pgfpathlineto{\pgfqpoint{3.116513in}{2.603812in}}%
\pgfpathclose%
\pgfusepath{fill}%
\end{pgfscope}%
\begin{pgfscope}%
\pgfpathrectangle{\pgfqpoint{0.017500in}{0.017500in}}{\pgfqpoint{6.218282in}{3.580372in}} %
\pgfusepath{clip}%
\pgfsetbuttcap%
\pgfsetroundjoin%
\definecolor{currentfill}{rgb}{0.758112,0.168122,0.188827}%
\pgfsetfillcolor{currentfill}%
\pgfsetlinewidth{0.000000pt}%
\definecolor{currentstroke}{rgb}{0.000000,0.000000,0.000000}%
\pgfsetstrokecolor{currentstroke}%
\pgfsetdash{}{0pt}%
\pgfpathmoveto{\pgfqpoint{3.116513in}{2.603812in}}%
\pgfpathlineto{\pgfqpoint{3.372093in}{2.761754in}}%
\pgfpathlineto{\pgfqpoint{3.116513in}{2.603812in}}%
\pgfpathclose%
\pgfusepath{fill}%
\end{pgfscope}%
\begin{pgfscope}%
\pgfpathrectangle{\pgfqpoint{0.017500in}{0.017500in}}{\pgfqpoint{6.218282in}{3.580372in}} %
\pgfusepath{clip}%
\pgfsetbuttcap%
\pgfsetroundjoin%
\definecolor{currentfill}{rgb}{0.758112,0.168122,0.188827}%
\pgfsetfillcolor{currentfill}%
\pgfsetlinewidth{0.000000pt}%
\definecolor{currentstroke}{rgb}{0.000000,0.000000,0.000000}%
\pgfsetstrokecolor{currentstroke}%
\pgfsetdash{}{0pt}%
\pgfpathmoveto{\pgfqpoint{3.116513in}{2.603812in}}%
\pgfpathlineto{\pgfqpoint{3.372093in}{2.761754in}}%
\pgfpathlineto{\pgfqpoint{3.116513in}{2.603812in}}%
\pgfpathclose%
\pgfusepath{fill}%
\end{pgfscope}%
\begin{pgfscope}%
\pgfpathrectangle{\pgfqpoint{0.017500in}{0.017500in}}{\pgfqpoint{6.218282in}{3.580372in}} %
\pgfusepath{clip}%
\pgfsetbuttcap%
\pgfsetroundjoin%
\definecolor{currentfill}{rgb}{0.758112,0.168122,0.188827}%
\pgfsetfillcolor{currentfill}%
\pgfsetlinewidth{0.000000pt}%
\definecolor{currentstroke}{rgb}{0.000000,0.000000,0.000000}%
\pgfsetstrokecolor{currentstroke}%
\pgfsetdash{}{0pt}%
\pgfpathmoveto{\pgfqpoint{3.116513in}{2.603812in}}%
\pgfpathlineto{\pgfqpoint{3.372093in}{2.761754in}}%
\pgfpathlineto{\pgfqpoint{3.116513in}{2.603812in}}%
\pgfpathclose%
\pgfusepath{fill}%
\end{pgfscope}%
\begin{pgfscope}%
\pgfpathrectangle{\pgfqpoint{0.017500in}{0.017500in}}{\pgfqpoint{6.218282in}{3.580372in}} %
\pgfusepath{clip}%
\pgfsetbuttcap%
\pgfsetroundjoin%
\definecolor{currentfill}{rgb}{0.758112,0.168122,0.188827}%
\pgfsetfillcolor{currentfill}%
\pgfsetlinewidth{0.000000pt}%
\definecolor{currentstroke}{rgb}{0.000000,0.000000,0.000000}%
\pgfsetstrokecolor{currentstroke}%
\pgfsetdash{}{0pt}%
\pgfpathmoveto{\pgfqpoint{3.116513in}{2.603812in}}%
\pgfpathlineto{\pgfqpoint{3.372093in}{2.761754in}}%
\pgfpathlineto{\pgfqpoint{3.116513in}{2.603812in}}%
\pgfpathclose%
\pgfusepath{fill}%
\end{pgfscope}%
\begin{pgfscope}%
\pgfpathrectangle{\pgfqpoint{0.017500in}{0.017500in}}{\pgfqpoint{6.218282in}{3.580372in}} %
\pgfusepath{clip}%
\pgfsetbuttcap%
\pgfsetroundjoin%
\definecolor{currentfill}{rgb}{0.758112,0.168122,0.188827}%
\pgfsetfillcolor{currentfill}%
\pgfsetlinewidth{0.000000pt}%
\definecolor{currentstroke}{rgb}{0.000000,0.000000,0.000000}%
\pgfsetstrokecolor{currentstroke}%
\pgfsetdash{}{0pt}%
\pgfpathmoveto{\pgfqpoint{3.116513in}{2.603812in}}%
\pgfpathlineto{\pgfqpoint{3.372093in}{2.761754in}}%
\pgfpathlineto{\pgfqpoint{3.116513in}{2.603812in}}%
\pgfpathclose%
\pgfusepath{fill}%
\end{pgfscope}%
\begin{pgfscope}%
\pgfpathrectangle{\pgfqpoint{0.017500in}{0.017500in}}{\pgfqpoint{6.218282in}{3.580372in}} %
\pgfusepath{clip}%
\pgfsetbuttcap%
\pgfsetroundjoin%
\definecolor{currentfill}{rgb}{0.758112,0.168122,0.188827}%
\pgfsetfillcolor{currentfill}%
\pgfsetlinewidth{0.000000pt}%
\definecolor{currentstroke}{rgb}{0.000000,0.000000,0.000000}%
\pgfsetstrokecolor{currentstroke}%
\pgfsetdash{}{0pt}%
\pgfpathmoveto{\pgfqpoint{3.116513in}{2.603812in}}%
\pgfpathlineto{\pgfqpoint{3.372093in}{2.761754in}}%
\pgfpathlineto{\pgfqpoint{3.116513in}{2.603812in}}%
\pgfpathclose%
\pgfusepath{fill}%
\end{pgfscope}%
\begin{pgfscope}%
\pgfpathrectangle{\pgfqpoint{0.017500in}{0.017500in}}{\pgfqpoint{6.218282in}{3.580372in}} %
\pgfusepath{clip}%
\pgfsetbuttcap%
\pgfsetroundjoin%
\definecolor{currentfill}{rgb}{0.758112,0.168122,0.188827}%
\pgfsetfillcolor{currentfill}%
\pgfsetlinewidth{0.000000pt}%
\definecolor{currentstroke}{rgb}{0.000000,0.000000,0.000000}%
\pgfsetstrokecolor{currentstroke}%
\pgfsetdash{}{0pt}%
\pgfpathmoveto{\pgfqpoint{3.116513in}{2.603812in}}%
\pgfpathlineto{\pgfqpoint{3.372093in}{2.761754in}}%
\pgfpathlineto{\pgfqpoint{3.116513in}{2.603812in}}%
\pgfpathclose%
\pgfusepath{fill}%
\end{pgfscope}%
\begin{pgfscope}%
\pgfpathrectangle{\pgfqpoint{0.017500in}{0.017500in}}{\pgfqpoint{6.218282in}{3.580372in}} %
\pgfusepath{clip}%
\pgfsetbuttcap%
\pgfsetroundjoin%
\definecolor{currentfill}{rgb}{0.758112,0.168122,0.188827}%
\pgfsetfillcolor{currentfill}%
\pgfsetlinewidth{0.000000pt}%
\definecolor{currentstroke}{rgb}{0.000000,0.000000,0.000000}%
\pgfsetstrokecolor{currentstroke}%
\pgfsetdash{}{0pt}%
\pgfpathmoveto{\pgfqpoint{3.116513in}{2.603812in}}%
\pgfpathlineto{\pgfqpoint{3.372093in}{2.761754in}}%
\pgfpathlineto{\pgfqpoint{3.116513in}{2.603812in}}%
\pgfpathclose%
\pgfusepath{fill}%
\end{pgfscope}%
\begin{pgfscope}%
\pgfpathrectangle{\pgfqpoint{0.017500in}{0.017500in}}{\pgfqpoint{6.218282in}{3.580372in}} %
\pgfusepath{clip}%
\pgfsetbuttcap%
\pgfsetroundjoin%
\definecolor{currentfill}{rgb}{0.758112,0.168122,0.188827}%
\pgfsetfillcolor{currentfill}%
\pgfsetlinewidth{0.000000pt}%
\definecolor{currentstroke}{rgb}{0.000000,0.000000,0.000000}%
\pgfsetstrokecolor{currentstroke}%
\pgfsetdash{}{0pt}%
\pgfpathmoveto{\pgfqpoint{3.116513in}{2.603812in}}%
\pgfpathlineto{\pgfqpoint{3.372093in}{2.761754in}}%
\pgfpathlineto{\pgfqpoint{3.116513in}{2.603812in}}%
\pgfpathclose%
\pgfusepath{fill}%
\end{pgfscope}%
\begin{pgfscope}%
\pgfpathrectangle{\pgfqpoint{0.017500in}{0.017500in}}{\pgfqpoint{6.218282in}{3.580372in}} %
\pgfusepath{clip}%
\pgfsetbuttcap%
\pgfsetroundjoin%
\definecolor{currentfill}{rgb}{0.758112,0.168122,0.188827}%
\pgfsetfillcolor{currentfill}%
\pgfsetlinewidth{0.000000pt}%
\definecolor{currentstroke}{rgb}{0.000000,0.000000,0.000000}%
\pgfsetstrokecolor{currentstroke}%
\pgfsetdash{}{0pt}%
\pgfpathmoveto{\pgfqpoint{3.116513in}{2.603812in}}%
\pgfpathlineto{\pgfqpoint{3.372093in}{2.761754in}}%
\pgfpathlineto{\pgfqpoint{3.116513in}{2.603812in}}%
\pgfpathclose%
\pgfusepath{fill}%
\end{pgfscope}%
\begin{pgfscope}%
\pgfpathrectangle{\pgfqpoint{0.017500in}{0.017500in}}{\pgfqpoint{6.218282in}{3.580372in}} %
\pgfusepath{clip}%
\pgfsetbuttcap%
\pgfsetroundjoin%
\definecolor{currentfill}{rgb}{0.758112,0.168122,0.188827}%
\pgfsetfillcolor{currentfill}%
\pgfsetlinewidth{0.000000pt}%
\definecolor{currentstroke}{rgb}{0.000000,0.000000,0.000000}%
\pgfsetstrokecolor{currentstroke}%
\pgfsetdash{}{0pt}%
\pgfpathmoveto{\pgfqpoint{3.116513in}{2.603812in}}%
\pgfpathlineto{\pgfqpoint{3.372093in}{2.761754in}}%
\pgfpathlineto{\pgfqpoint{3.116513in}{2.603812in}}%
\pgfpathclose%
\pgfusepath{fill}%
\end{pgfscope}%
\begin{pgfscope}%
\pgfpathrectangle{\pgfqpoint{0.017500in}{0.017500in}}{\pgfqpoint{6.218282in}{3.580372in}} %
\pgfusepath{clip}%
\pgfsetbuttcap%
\pgfsetroundjoin%
\definecolor{currentfill}{rgb}{0.758112,0.168122,0.188827}%
\pgfsetfillcolor{currentfill}%
\pgfsetlinewidth{0.000000pt}%
\definecolor{currentstroke}{rgb}{0.000000,0.000000,0.000000}%
\pgfsetstrokecolor{currentstroke}%
\pgfsetdash{}{0pt}%
\pgfpathmoveto{\pgfqpoint{3.116513in}{2.603812in}}%
\pgfpathlineto{\pgfqpoint{3.372093in}{2.761754in}}%
\pgfpathlineto{\pgfqpoint{3.116513in}{2.603812in}}%
\pgfpathclose%
\pgfusepath{fill}%
\end{pgfscope}%
\begin{pgfscope}%
\pgfpathrectangle{\pgfqpoint{0.017500in}{0.017500in}}{\pgfqpoint{6.218282in}{3.580372in}} %
\pgfusepath{clip}%
\pgfsetbuttcap%
\pgfsetroundjoin%
\definecolor{currentfill}{rgb}{0.902849,0.844796,0.811970}%
\pgfsetfillcolor{currentfill}%
\pgfsetlinewidth{0.000000pt}%
\definecolor{currentstroke}{rgb}{0.000000,0.000000,0.000000}%
\pgfsetstrokecolor{currentstroke}%
\pgfsetdash{}{0pt}%
\pgfpathmoveto{\pgfqpoint{4.511410in}{2.753953in}}%
\pgfpathlineto{\pgfqpoint{3.236099in}{0.723497in}}%
\pgfpathlineto{\pgfqpoint{4.511410in}{2.753953in}}%
\pgfpathclose%
\pgfusepath{fill}%
\end{pgfscope}%
\begin{pgfscope}%
\pgfpathrectangle{\pgfqpoint{0.017500in}{0.017500in}}{\pgfqpoint{6.218282in}{3.580372in}} %
\pgfusepath{clip}%
\pgfsetbuttcap%
\pgfsetroundjoin%
\definecolor{currentfill}{rgb}{0.902849,0.844796,0.811970}%
\pgfsetfillcolor{currentfill}%
\pgfsetlinewidth{0.000000pt}%
\definecolor{currentstroke}{rgb}{0.000000,0.000000,0.000000}%
\pgfsetstrokecolor{currentstroke}%
\pgfsetdash{}{0pt}%
\pgfpathmoveto{\pgfqpoint{4.511410in}{2.753953in}}%
\pgfpathlineto{\pgfqpoint{3.236099in}{0.723497in}}%
\pgfpathlineto{\pgfqpoint{4.511410in}{2.753953in}}%
\pgfpathclose%
\pgfusepath{fill}%
\end{pgfscope}%
\begin{pgfscope}%
\pgfpathrectangle{\pgfqpoint{0.017500in}{0.017500in}}{\pgfqpoint{6.218282in}{3.580372in}} %
\pgfusepath{clip}%
\pgfsetbuttcap%
\pgfsetroundjoin%
\definecolor{currentfill}{rgb}{0.902849,0.844796,0.811970}%
\pgfsetfillcolor{currentfill}%
\pgfsetlinewidth{0.000000pt}%
\definecolor{currentstroke}{rgb}{0.000000,0.000000,0.000000}%
\pgfsetstrokecolor{currentstroke}%
\pgfsetdash{}{0pt}%
\pgfpathmoveto{\pgfqpoint{4.511410in}{2.753953in}}%
\pgfpathlineto{\pgfqpoint{3.236099in}{0.723497in}}%
\pgfpathlineto{\pgfqpoint{4.511410in}{2.753953in}}%
\pgfpathclose%
\pgfusepath{fill}%
\end{pgfscope}%
\begin{pgfscope}%
\pgfpathrectangle{\pgfqpoint{0.017500in}{0.017500in}}{\pgfqpoint{6.218282in}{3.580372in}} %
\pgfusepath{clip}%
\pgfsetbuttcap%
\pgfsetroundjoin%
\definecolor{currentfill}{rgb}{0.902849,0.844796,0.811970}%
\pgfsetfillcolor{currentfill}%
\pgfsetlinewidth{0.000000pt}%
\definecolor{currentstroke}{rgb}{0.000000,0.000000,0.000000}%
\pgfsetstrokecolor{currentstroke}%
\pgfsetdash{}{0pt}%
\pgfpathmoveto{\pgfqpoint{4.511410in}{2.753953in}}%
\pgfpathlineto{\pgfqpoint{3.236099in}{0.723497in}}%
\pgfpathlineto{\pgfqpoint{4.511410in}{2.753953in}}%
\pgfpathclose%
\pgfusepath{fill}%
\end{pgfscope}%
\begin{pgfscope}%
\pgfpathrectangle{\pgfqpoint{0.017500in}{0.017500in}}{\pgfqpoint{6.218282in}{3.580372in}} %
\pgfusepath{clip}%
\pgfsetbuttcap%
\pgfsetroundjoin%
\definecolor{currentfill}{rgb}{0.902849,0.844796,0.811970}%
\pgfsetfillcolor{currentfill}%
\pgfsetlinewidth{0.000000pt}%
\definecolor{currentstroke}{rgb}{0.000000,0.000000,0.000000}%
\pgfsetstrokecolor{currentstroke}%
\pgfsetdash{}{0pt}%
\pgfpathmoveto{\pgfqpoint{4.511410in}{2.753953in}}%
\pgfpathlineto{\pgfqpoint{3.236099in}{0.723497in}}%
\pgfpathlineto{\pgfqpoint{4.511410in}{2.753953in}}%
\pgfpathclose%
\pgfusepath{fill}%
\end{pgfscope}%
\begin{pgfscope}%
\pgfpathrectangle{\pgfqpoint{0.017500in}{0.017500in}}{\pgfqpoint{6.218282in}{3.580372in}} %
\pgfusepath{clip}%
\pgfsetbuttcap%
\pgfsetroundjoin%
\definecolor{currentfill}{rgb}{0.902849,0.844796,0.811970}%
\pgfsetfillcolor{currentfill}%
\pgfsetlinewidth{0.000000pt}%
\definecolor{currentstroke}{rgb}{0.000000,0.000000,0.000000}%
\pgfsetstrokecolor{currentstroke}%
\pgfsetdash{}{0pt}%
\pgfpathmoveto{\pgfqpoint{4.511410in}{2.753953in}}%
\pgfpathlineto{\pgfqpoint{3.236099in}{0.723497in}}%
\pgfpathlineto{\pgfqpoint{4.511410in}{2.753953in}}%
\pgfpathclose%
\pgfusepath{fill}%
\end{pgfscope}%
\begin{pgfscope}%
\pgfpathrectangle{\pgfqpoint{0.017500in}{0.017500in}}{\pgfqpoint{6.218282in}{3.580372in}} %
\pgfusepath{clip}%
\pgfsetbuttcap%
\pgfsetroundjoin%
\definecolor{currentfill}{rgb}{0.902849,0.844796,0.811970}%
\pgfsetfillcolor{currentfill}%
\pgfsetlinewidth{0.000000pt}%
\definecolor{currentstroke}{rgb}{0.000000,0.000000,0.000000}%
\pgfsetstrokecolor{currentstroke}%
\pgfsetdash{}{0pt}%
\pgfpathmoveto{\pgfqpoint{4.511410in}{2.753953in}}%
\pgfpathlineto{\pgfqpoint{3.236099in}{0.723497in}}%
\pgfpathlineto{\pgfqpoint{4.511410in}{2.753953in}}%
\pgfpathclose%
\pgfusepath{fill}%
\end{pgfscope}%
\begin{pgfscope}%
\pgfpathrectangle{\pgfqpoint{0.017500in}{0.017500in}}{\pgfqpoint{6.218282in}{3.580372in}} %
\pgfusepath{clip}%
\pgfsetbuttcap%
\pgfsetroundjoin%
\definecolor{currentfill}{rgb}{0.902849,0.844796,0.811970}%
\pgfsetfillcolor{currentfill}%
\pgfsetlinewidth{0.000000pt}%
\definecolor{currentstroke}{rgb}{0.000000,0.000000,0.000000}%
\pgfsetstrokecolor{currentstroke}%
\pgfsetdash{}{0pt}%
\pgfpathmoveto{\pgfqpoint{4.511410in}{2.753953in}}%
\pgfpathlineto{\pgfqpoint{3.236099in}{0.723497in}}%
\pgfpathlineto{\pgfqpoint{4.511410in}{2.753953in}}%
\pgfpathclose%
\pgfusepath{fill}%
\end{pgfscope}%
\begin{pgfscope}%
\pgfpathrectangle{\pgfqpoint{0.017500in}{0.017500in}}{\pgfqpoint{6.218282in}{3.580372in}} %
\pgfusepath{clip}%
\pgfsetbuttcap%
\pgfsetroundjoin%
\definecolor{currentfill}{rgb}{0.902849,0.844796,0.811970}%
\pgfsetfillcolor{currentfill}%
\pgfsetlinewidth{0.000000pt}%
\definecolor{currentstroke}{rgb}{0.000000,0.000000,0.000000}%
\pgfsetstrokecolor{currentstroke}%
\pgfsetdash{}{0pt}%
\pgfpathmoveto{\pgfqpoint{4.511410in}{2.753953in}}%
\pgfpathlineto{\pgfqpoint{3.236099in}{0.723497in}}%
\pgfpathlineto{\pgfqpoint{4.511410in}{2.753953in}}%
\pgfpathclose%
\pgfusepath{fill}%
\end{pgfscope}%
\begin{pgfscope}%
\pgfpathrectangle{\pgfqpoint{0.017500in}{0.017500in}}{\pgfqpoint{6.218282in}{3.580372in}} %
\pgfusepath{clip}%
\pgfsetbuttcap%
\pgfsetroundjoin%
\definecolor{currentfill}{rgb}{0.902849,0.844796,0.811970}%
\pgfsetfillcolor{currentfill}%
\pgfsetlinewidth{0.000000pt}%
\definecolor{currentstroke}{rgb}{0.000000,0.000000,0.000000}%
\pgfsetstrokecolor{currentstroke}%
\pgfsetdash{}{0pt}%
\pgfpathmoveto{\pgfqpoint{4.511410in}{2.753953in}}%
\pgfpathlineto{\pgfqpoint{3.236099in}{0.723497in}}%
\pgfpathlineto{\pgfqpoint{4.511410in}{2.753953in}}%
\pgfpathclose%
\pgfusepath{fill}%
\end{pgfscope}%
\begin{pgfscope}%
\pgfpathrectangle{\pgfqpoint{0.017500in}{0.017500in}}{\pgfqpoint{6.218282in}{3.580372in}} %
\pgfusepath{clip}%
\pgfsetbuttcap%
\pgfsetroundjoin%
\definecolor{currentfill}{rgb}{0.902849,0.844796,0.811970}%
\pgfsetfillcolor{currentfill}%
\pgfsetlinewidth{0.000000pt}%
\definecolor{currentstroke}{rgb}{0.000000,0.000000,0.000000}%
\pgfsetstrokecolor{currentstroke}%
\pgfsetdash{}{0pt}%
\pgfpathmoveto{\pgfqpoint{4.511410in}{2.753953in}}%
\pgfpathlineto{\pgfqpoint{3.236099in}{0.723497in}}%
\pgfpathlineto{\pgfqpoint{4.511410in}{2.753953in}}%
\pgfpathclose%
\pgfusepath{fill}%
\end{pgfscope}%
\begin{pgfscope}%
\pgfpathrectangle{\pgfqpoint{0.017500in}{0.017500in}}{\pgfqpoint{6.218282in}{3.580372in}} %
\pgfusepath{clip}%
\pgfsetbuttcap%
\pgfsetroundjoin%
\definecolor{currentfill}{rgb}{0.902849,0.844796,0.811970}%
\pgfsetfillcolor{currentfill}%
\pgfsetlinewidth{0.000000pt}%
\definecolor{currentstroke}{rgb}{0.000000,0.000000,0.000000}%
\pgfsetstrokecolor{currentstroke}%
\pgfsetdash{}{0pt}%
\pgfpathmoveto{\pgfqpoint{4.511410in}{2.753953in}}%
\pgfpathlineto{\pgfqpoint{3.236099in}{0.723497in}}%
\pgfpathlineto{\pgfqpoint{4.511410in}{2.753953in}}%
\pgfpathclose%
\pgfusepath{fill}%
\end{pgfscope}%
\begin{pgfscope}%
\pgfpathrectangle{\pgfqpoint{0.017500in}{0.017500in}}{\pgfqpoint{6.218282in}{3.580372in}} %
\pgfusepath{clip}%
\pgfsetbuttcap%
\pgfsetroundjoin%
\definecolor{currentfill}{rgb}{0.902849,0.844796,0.811970}%
\pgfsetfillcolor{currentfill}%
\pgfsetlinewidth{0.000000pt}%
\definecolor{currentstroke}{rgb}{0.000000,0.000000,0.000000}%
\pgfsetstrokecolor{currentstroke}%
\pgfsetdash{}{0pt}%
\pgfpathmoveto{\pgfqpoint{4.511410in}{2.753953in}}%
\pgfpathlineto{\pgfqpoint{3.236099in}{0.723497in}}%
\pgfpathlineto{\pgfqpoint{4.511410in}{2.753953in}}%
\pgfpathclose%
\pgfusepath{fill}%
\end{pgfscope}%
\begin{pgfscope}%
\pgfpathrectangle{\pgfqpoint{0.017500in}{0.017500in}}{\pgfqpoint{6.218282in}{3.580372in}} %
\pgfusepath{clip}%
\pgfsetbuttcap%
\pgfsetroundjoin%
\definecolor{currentfill}{rgb}{0.902849,0.844796,0.811970}%
\pgfsetfillcolor{currentfill}%
\pgfsetlinewidth{0.000000pt}%
\definecolor{currentstroke}{rgb}{0.000000,0.000000,0.000000}%
\pgfsetstrokecolor{currentstroke}%
\pgfsetdash{}{0pt}%
\pgfpathmoveto{\pgfqpoint{4.511410in}{2.753953in}}%
\pgfpathlineto{\pgfqpoint{3.236099in}{0.723497in}}%
\pgfpathlineto{\pgfqpoint{4.511410in}{2.753953in}}%
\pgfpathclose%
\pgfusepath{fill}%
\end{pgfscope}%
\begin{pgfscope}%
\pgfpathrectangle{\pgfqpoint{0.017500in}{0.017500in}}{\pgfqpoint{6.218282in}{3.580372in}} %
\pgfusepath{clip}%
\pgfsetbuttcap%
\pgfsetroundjoin%
\definecolor{currentfill}{rgb}{0.902849,0.844796,0.811970}%
\pgfsetfillcolor{currentfill}%
\pgfsetlinewidth{0.000000pt}%
\definecolor{currentstroke}{rgb}{0.000000,0.000000,0.000000}%
\pgfsetstrokecolor{currentstroke}%
\pgfsetdash{}{0pt}%
\pgfpathmoveto{\pgfqpoint{4.511410in}{2.753953in}}%
\pgfpathlineto{\pgfqpoint{3.236099in}{0.723497in}}%
\pgfpathlineto{\pgfqpoint{4.511410in}{2.753953in}}%
\pgfpathclose%
\pgfusepath{fill}%
\end{pgfscope}%
\begin{pgfscope}%
\pgfpathrectangle{\pgfqpoint{0.017500in}{0.017500in}}{\pgfqpoint{6.218282in}{3.580372in}} %
\pgfusepath{clip}%
\pgfsetbuttcap%
\pgfsetroundjoin%
\definecolor{currentfill}{rgb}{0.902849,0.844796,0.811970}%
\pgfsetfillcolor{currentfill}%
\pgfsetlinewidth{0.000000pt}%
\definecolor{currentstroke}{rgb}{0.000000,0.000000,0.000000}%
\pgfsetstrokecolor{currentstroke}%
\pgfsetdash{}{0pt}%
\pgfpathmoveto{\pgfqpoint{4.511410in}{2.753953in}}%
\pgfpathlineto{\pgfqpoint{3.236099in}{0.723497in}}%
\pgfpathlineto{\pgfqpoint{4.511410in}{2.753953in}}%
\pgfpathclose%
\pgfusepath{fill}%
\end{pgfscope}%
\begin{pgfscope}%
\pgfpathrectangle{\pgfqpoint{0.017500in}{0.017500in}}{\pgfqpoint{6.218282in}{3.580372in}} %
\pgfusepath{clip}%
\pgfsetbuttcap%
\pgfsetroundjoin%
\definecolor{currentfill}{rgb}{0.902849,0.844796,0.811970}%
\pgfsetfillcolor{currentfill}%
\pgfsetlinewidth{0.000000pt}%
\definecolor{currentstroke}{rgb}{0.000000,0.000000,0.000000}%
\pgfsetstrokecolor{currentstroke}%
\pgfsetdash{}{0pt}%
\pgfpathmoveto{\pgfqpoint{4.511410in}{2.753953in}}%
\pgfpathlineto{\pgfqpoint{3.236099in}{0.723497in}}%
\pgfpathlineto{\pgfqpoint{4.511410in}{2.753953in}}%
\pgfpathclose%
\pgfusepath{fill}%
\end{pgfscope}%
\begin{pgfscope}%
\pgfpathrectangle{\pgfqpoint{0.017500in}{0.017500in}}{\pgfqpoint{6.218282in}{3.580372in}} %
\pgfusepath{clip}%
\pgfsetbuttcap%
\pgfsetroundjoin%
\definecolor{currentfill}{rgb}{0.902849,0.844796,0.811970}%
\pgfsetfillcolor{currentfill}%
\pgfsetlinewidth{0.000000pt}%
\definecolor{currentstroke}{rgb}{0.000000,0.000000,0.000000}%
\pgfsetstrokecolor{currentstroke}%
\pgfsetdash{}{0pt}%
\pgfpathmoveto{\pgfqpoint{4.511410in}{2.753953in}}%
\pgfpathlineto{\pgfqpoint{3.236099in}{0.723497in}}%
\pgfpathlineto{\pgfqpoint{4.511410in}{2.753953in}}%
\pgfpathclose%
\pgfusepath{fill}%
\end{pgfscope}%
\begin{pgfscope}%
\pgfpathrectangle{\pgfqpoint{0.017500in}{0.017500in}}{\pgfqpoint{6.218282in}{3.580372in}} %
\pgfusepath{clip}%
\pgfsetbuttcap%
\pgfsetroundjoin%
\definecolor{currentfill}{rgb}{0.902849,0.844796,0.811970}%
\pgfsetfillcolor{currentfill}%
\pgfsetlinewidth{0.000000pt}%
\definecolor{currentstroke}{rgb}{0.000000,0.000000,0.000000}%
\pgfsetstrokecolor{currentstroke}%
\pgfsetdash{}{0pt}%
\pgfpathmoveto{\pgfqpoint{4.511410in}{2.753953in}}%
\pgfpathlineto{\pgfqpoint{3.236099in}{0.723497in}}%
\pgfpathlineto{\pgfqpoint{4.511410in}{2.753953in}}%
\pgfpathclose%
\pgfusepath{fill}%
\end{pgfscope}%
\begin{pgfscope}%
\pgfpathrectangle{\pgfqpoint{0.017500in}{0.017500in}}{\pgfqpoint{6.218282in}{3.580372in}} %
\pgfusepath{clip}%
\pgfsetbuttcap%
\pgfsetroundjoin%
\definecolor{currentfill}{rgb}{0.902849,0.844796,0.811970}%
\pgfsetfillcolor{currentfill}%
\pgfsetlinewidth{0.000000pt}%
\definecolor{currentstroke}{rgb}{0.000000,0.000000,0.000000}%
\pgfsetstrokecolor{currentstroke}%
\pgfsetdash{}{0pt}%
\pgfpathmoveto{\pgfqpoint{4.511410in}{2.753953in}}%
\pgfpathlineto{\pgfqpoint{3.236099in}{0.723497in}}%
\pgfpathlineto{\pgfqpoint{4.511410in}{2.753953in}}%
\pgfpathclose%
\pgfusepath{fill}%
\end{pgfscope}%
\begin{pgfscope}%
\pgfpathrectangle{\pgfqpoint{0.017500in}{0.017500in}}{\pgfqpoint{6.218282in}{3.580372in}} %
\pgfusepath{clip}%
\pgfsetbuttcap%
\pgfsetroundjoin%
\definecolor{currentfill}{rgb}{0.902849,0.844796,0.811970}%
\pgfsetfillcolor{currentfill}%
\pgfsetlinewidth{0.000000pt}%
\definecolor{currentstroke}{rgb}{0.000000,0.000000,0.000000}%
\pgfsetstrokecolor{currentstroke}%
\pgfsetdash{}{0pt}%
\pgfpathmoveto{\pgfqpoint{4.511410in}{2.753953in}}%
\pgfpathlineto{\pgfqpoint{3.236099in}{0.723497in}}%
\pgfpathlineto{\pgfqpoint{4.511410in}{2.753953in}}%
\pgfpathclose%
\pgfusepath{fill}%
\end{pgfscope}%
\begin{pgfscope}%
\pgfpathrectangle{\pgfqpoint{0.017500in}{0.017500in}}{\pgfqpoint{6.218282in}{3.580372in}} %
\pgfusepath{clip}%
\pgfsetbuttcap%
\pgfsetroundjoin%
\definecolor{currentfill}{rgb}{0.902849,0.844796,0.811970}%
\pgfsetfillcolor{currentfill}%
\pgfsetlinewidth{0.000000pt}%
\definecolor{currentstroke}{rgb}{0.000000,0.000000,0.000000}%
\pgfsetstrokecolor{currentstroke}%
\pgfsetdash{}{0pt}%
\pgfpathmoveto{\pgfqpoint{4.511410in}{2.753953in}}%
\pgfpathlineto{\pgfqpoint{3.236099in}{0.723497in}}%
\pgfpathlineto{\pgfqpoint{4.511410in}{2.753953in}}%
\pgfpathclose%
\pgfusepath{fill}%
\end{pgfscope}%
\begin{pgfscope}%
\pgfpathrectangle{\pgfqpoint{0.017500in}{0.017500in}}{\pgfqpoint{6.218282in}{3.580372in}} %
\pgfusepath{clip}%
\pgfsetbuttcap%
\pgfsetroundjoin%
\definecolor{currentfill}{rgb}{0.902849,0.844796,0.811970}%
\pgfsetfillcolor{currentfill}%
\pgfsetlinewidth{0.000000pt}%
\definecolor{currentstroke}{rgb}{0.000000,0.000000,0.000000}%
\pgfsetstrokecolor{currentstroke}%
\pgfsetdash{}{0pt}%
\pgfpathmoveto{\pgfqpoint{4.511410in}{2.753953in}}%
\pgfpathlineto{\pgfqpoint{3.236099in}{0.723497in}}%
\pgfpathlineto{\pgfqpoint{4.511410in}{2.753953in}}%
\pgfpathclose%
\pgfusepath{fill}%
\end{pgfscope}%
\begin{pgfscope}%
\pgfpathrectangle{\pgfqpoint{0.017500in}{0.017500in}}{\pgfqpoint{6.218282in}{3.580372in}} %
\pgfusepath{clip}%
\pgfsetbuttcap%
\pgfsetroundjoin%
\definecolor{currentfill}{rgb}{0.902849,0.844796,0.811970}%
\pgfsetfillcolor{currentfill}%
\pgfsetlinewidth{0.000000pt}%
\definecolor{currentstroke}{rgb}{0.000000,0.000000,0.000000}%
\pgfsetstrokecolor{currentstroke}%
\pgfsetdash{}{0pt}%
\pgfpathmoveto{\pgfqpoint{4.511410in}{2.753953in}}%
\pgfpathlineto{\pgfqpoint{3.236099in}{0.723497in}}%
\pgfpathlineto{\pgfqpoint{4.511410in}{2.753953in}}%
\pgfpathclose%
\pgfusepath{fill}%
\end{pgfscope}%
\begin{pgfscope}%
\pgfpathrectangle{\pgfqpoint{0.017500in}{0.017500in}}{\pgfqpoint{6.218282in}{3.580372in}} %
\pgfusepath{clip}%
\pgfsetbuttcap%
\pgfsetroundjoin%
\definecolor{currentfill}{rgb}{0.902849,0.844796,0.811970}%
\pgfsetfillcolor{currentfill}%
\pgfsetlinewidth{0.000000pt}%
\definecolor{currentstroke}{rgb}{0.000000,0.000000,0.000000}%
\pgfsetstrokecolor{currentstroke}%
\pgfsetdash{}{0pt}%
\pgfpathmoveto{\pgfqpoint{4.511410in}{2.753953in}}%
\pgfpathlineto{\pgfqpoint{3.236099in}{0.723497in}}%
\pgfpathlineto{\pgfqpoint{4.511410in}{2.753953in}}%
\pgfpathclose%
\pgfusepath{fill}%
\end{pgfscope}%
\begin{pgfscope}%
\pgfpathrectangle{\pgfqpoint{0.017500in}{0.017500in}}{\pgfqpoint{6.218282in}{3.580372in}} %
\pgfusepath{clip}%
\pgfsetbuttcap%
\pgfsetroundjoin%
\definecolor{currentfill}{rgb}{0.758112,0.168122,0.188827}%
\pgfsetfillcolor{currentfill}%
\pgfsetlinewidth{0.000000pt}%
\definecolor{currentstroke}{rgb}{0.000000,0.000000,0.000000}%
\pgfsetstrokecolor{currentstroke}%
\pgfsetdash{}{0pt}%
\pgfpathmoveto{\pgfqpoint{3.116513in}{2.603812in}}%
\pgfpathlineto{\pgfqpoint{3.564492in}{2.541257in}}%
\pgfpathlineto{\pgfqpoint{3.816789in}{2.677291in}}%
\pgfpathlineto{\pgfqpoint{3.372093in}{2.761754in}}%
\pgfpathlineto{\pgfqpoint{3.116513in}{2.603812in}}%
\pgfpathclose%
\pgfusepath{fill}%
\end{pgfscope}%
\begin{pgfscope}%
\pgfpathrectangle{\pgfqpoint{0.017500in}{0.017500in}}{\pgfqpoint{6.218282in}{3.580372in}} %
\pgfusepath{clip}%
\pgfsetbuttcap%
\pgfsetroundjoin%
\definecolor{currentfill}{rgb}{0.758112,0.168122,0.188827}%
\pgfsetfillcolor{currentfill}%
\pgfsetlinewidth{0.000000pt}%
\definecolor{currentstroke}{rgb}{0.000000,0.000000,0.000000}%
\pgfsetstrokecolor{currentstroke}%
\pgfsetdash{}{0pt}%
\pgfpathmoveto{\pgfqpoint{3.116513in}{2.603812in}}%
\pgfpathlineto{\pgfqpoint{3.564492in}{2.541257in}}%
\pgfpathlineto{\pgfqpoint{3.816789in}{2.677291in}}%
\pgfpathlineto{\pgfqpoint{3.372093in}{2.761754in}}%
\pgfpathlineto{\pgfqpoint{3.116513in}{2.603812in}}%
\pgfpathclose%
\pgfusepath{fill}%
\end{pgfscope}%
\begin{pgfscope}%
\pgfpathrectangle{\pgfqpoint{0.017500in}{0.017500in}}{\pgfqpoint{6.218282in}{3.580372in}} %
\pgfusepath{clip}%
\pgfsetbuttcap%
\pgfsetroundjoin%
\definecolor{currentfill}{rgb}{0.758112,0.168122,0.188827}%
\pgfsetfillcolor{currentfill}%
\pgfsetlinewidth{0.000000pt}%
\definecolor{currentstroke}{rgb}{0.000000,0.000000,0.000000}%
\pgfsetstrokecolor{currentstroke}%
\pgfsetdash{}{0pt}%
\pgfpathmoveto{\pgfqpoint{3.116513in}{2.603812in}}%
\pgfpathlineto{\pgfqpoint{3.564492in}{2.541257in}}%
\pgfpathlineto{\pgfqpoint{3.816789in}{2.677291in}}%
\pgfpathlineto{\pgfqpoint{3.372093in}{2.761754in}}%
\pgfpathlineto{\pgfqpoint{3.116513in}{2.603812in}}%
\pgfpathclose%
\pgfusepath{fill}%
\end{pgfscope}%
\begin{pgfscope}%
\pgfpathrectangle{\pgfqpoint{0.017500in}{0.017500in}}{\pgfqpoint{6.218282in}{3.580372in}} %
\pgfusepath{clip}%
\pgfsetbuttcap%
\pgfsetroundjoin%
\definecolor{currentfill}{rgb}{0.758112,0.168122,0.188827}%
\pgfsetfillcolor{currentfill}%
\pgfsetlinewidth{0.000000pt}%
\definecolor{currentstroke}{rgb}{0.000000,0.000000,0.000000}%
\pgfsetstrokecolor{currentstroke}%
\pgfsetdash{}{0pt}%
\pgfpathmoveto{\pgfqpoint{3.116513in}{2.603812in}}%
\pgfpathlineto{\pgfqpoint{3.564492in}{2.541257in}}%
\pgfpathlineto{\pgfqpoint{3.816789in}{2.677291in}}%
\pgfpathlineto{\pgfqpoint{3.372093in}{2.761754in}}%
\pgfpathlineto{\pgfqpoint{3.116513in}{2.603812in}}%
\pgfpathclose%
\pgfusepath{fill}%
\end{pgfscope}%
\begin{pgfscope}%
\pgfpathrectangle{\pgfqpoint{0.017500in}{0.017500in}}{\pgfqpoint{6.218282in}{3.580372in}} %
\pgfusepath{clip}%
\pgfsetbuttcap%
\pgfsetroundjoin%
\definecolor{currentfill}{rgb}{0.758112,0.168122,0.188827}%
\pgfsetfillcolor{currentfill}%
\pgfsetlinewidth{0.000000pt}%
\definecolor{currentstroke}{rgb}{0.000000,0.000000,0.000000}%
\pgfsetstrokecolor{currentstroke}%
\pgfsetdash{}{0pt}%
\pgfpathmoveto{\pgfqpoint{3.116513in}{2.603812in}}%
\pgfpathlineto{\pgfqpoint{3.564492in}{2.541257in}}%
\pgfpathlineto{\pgfqpoint{3.816789in}{2.677291in}}%
\pgfpathlineto{\pgfqpoint{3.372093in}{2.761754in}}%
\pgfpathlineto{\pgfqpoint{3.116513in}{2.603812in}}%
\pgfpathclose%
\pgfusepath{fill}%
\end{pgfscope}%
\begin{pgfscope}%
\pgfpathrectangle{\pgfqpoint{0.017500in}{0.017500in}}{\pgfqpoint{6.218282in}{3.580372in}} %
\pgfusepath{clip}%
\pgfsetbuttcap%
\pgfsetroundjoin%
\definecolor{currentfill}{rgb}{0.758112,0.168122,0.188827}%
\pgfsetfillcolor{currentfill}%
\pgfsetlinewidth{0.000000pt}%
\definecolor{currentstroke}{rgb}{0.000000,0.000000,0.000000}%
\pgfsetstrokecolor{currentstroke}%
\pgfsetdash{}{0pt}%
\pgfpathmoveto{\pgfqpoint{3.116513in}{2.603812in}}%
\pgfpathlineto{\pgfqpoint{3.564492in}{2.541257in}}%
\pgfpathlineto{\pgfqpoint{3.816789in}{2.677291in}}%
\pgfpathlineto{\pgfqpoint{3.372093in}{2.761754in}}%
\pgfpathlineto{\pgfqpoint{3.116513in}{2.603812in}}%
\pgfpathclose%
\pgfusepath{fill}%
\end{pgfscope}%
\begin{pgfscope}%
\pgfpathrectangle{\pgfqpoint{0.017500in}{0.017500in}}{\pgfqpoint{6.218282in}{3.580372in}} %
\pgfusepath{clip}%
\pgfsetbuttcap%
\pgfsetroundjoin%
\definecolor{currentfill}{rgb}{0.729196,0.086679,0.167240}%
\pgfsetfillcolor{currentfill}%
\pgfsetlinewidth{0.000000pt}%
\definecolor{currentstroke}{rgb}{0.000000,0.000000,0.000000}%
\pgfsetstrokecolor{currentstroke}%
\pgfsetdash{}{0pt}%
\pgfpathmoveto{\pgfqpoint{4.267628in}{2.613520in}}%
\pgfpathlineto{\pgfqpoint{4.511410in}{2.753953in}}%
\pgfpathlineto{\pgfqpoint{4.267628in}{2.613520in}}%
\pgfpathclose%
\pgfusepath{fill}%
\end{pgfscope}%
\begin{pgfscope}%
\pgfpathrectangle{\pgfqpoint{0.017500in}{0.017500in}}{\pgfqpoint{6.218282in}{3.580372in}} %
\pgfusepath{clip}%
\pgfsetbuttcap%
\pgfsetroundjoin%
\definecolor{currentfill}{rgb}{0.729196,0.086679,0.167240}%
\pgfsetfillcolor{currentfill}%
\pgfsetlinewidth{0.000000pt}%
\definecolor{currentstroke}{rgb}{0.000000,0.000000,0.000000}%
\pgfsetstrokecolor{currentstroke}%
\pgfsetdash{}{0pt}%
\pgfpathmoveto{\pgfqpoint{4.267628in}{2.613520in}}%
\pgfpathlineto{\pgfqpoint{4.511410in}{2.753953in}}%
\pgfpathlineto{\pgfqpoint{4.267628in}{2.613520in}}%
\pgfpathclose%
\pgfusepath{fill}%
\end{pgfscope}%
\begin{pgfscope}%
\pgfpathrectangle{\pgfqpoint{0.017500in}{0.017500in}}{\pgfqpoint{6.218282in}{3.580372in}} %
\pgfusepath{clip}%
\pgfsetbuttcap%
\pgfsetroundjoin%
\definecolor{currentfill}{rgb}{0.729196,0.086679,0.167240}%
\pgfsetfillcolor{currentfill}%
\pgfsetlinewidth{0.000000pt}%
\definecolor{currentstroke}{rgb}{0.000000,0.000000,0.000000}%
\pgfsetstrokecolor{currentstroke}%
\pgfsetdash{}{0pt}%
\pgfpathmoveto{\pgfqpoint{4.267628in}{2.613520in}}%
\pgfpathlineto{\pgfqpoint{4.511410in}{2.753953in}}%
\pgfpathlineto{\pgfqpoint{4.267628in}{2.613520in}}%
\pgfpathclose%
\pgfusepath{fill}%
\end{pgfscope}%
\begin{pgfscope}%
\pgfpathrectangle{\pgfqpoint{0.017500in}{0.017500in}}{\pgfqpoint{6.218282in}{3.580372in}} %
\pgfusepath{clip}%
\pgfsetbuttcap%
\pgfsetroundjoin%
\definecolor{currentfill}{rgb}{0.729196,0.086679,0.167240}%
\pgfsetfillcolor{currentfill}%
\pgfsetlinewidth{0.000000pt}%
\definecolor{currentstroke}{rgb}{0.000000,0.000000,0.000000}%
\pgfsetstrokecolor{currentstroke}%
\pgfsetdash{}{0pt}%
\pgfpathmoveto{\pgfqpoint{4.267628in}{2.613520in}}%
\pgfpathlineto{\pgfqpoint{4.511410in}{2.753953in}}%
\pgfpathlineto{\pgfqpoint{4.267628in}{2.613520in}}%
\pgfpathclose%
\pgfusepath{fill}%
\end{pgfscope}%
\begin{pgfscope}%
\pgfpathrectangle{\pgfqpoint{0.017500in}{0.017500in}}{\pgfqpoint{6.218282in}{3.580372in}} %
\pgfusepath{clip}%
\pgfsetbuttcap%
\pgfsetroundjoin%
\definecolor{currentfill}{rgb}{0.729196,0.086679,0.167240}%
\pgfsetfillcolor{currentfill}%
\pgfsetlinewidth{0.000000pt}%
\definecolor{currentstroke}{rgb}{0.000000,0.000000,0.000000}%
\pgfsetstrokecolor{currentstroke}%
\pgfsetdash{}{0pt}%
\pgfpathmoveto{\pgfqpoint{4.267628in}{2.613520in}}%
\pgfpathlineto{\pgfqpoint{4.511410in}{2.753953in}}%
\pgfpathlineto{\pgfqpoint{4.267628in}{2.613520in}}%
\pgfpathclose%
\pgfusepath{fill}%
\end{pgfscope}%
\begin{pgfscope}%
\pgfpathrectangle{\pgfqpoint{0.017500in}{0.017500in}}{\pgfqpoint{6.218282in}{3.580372in}} %
\pgfusepath{clip}%
\pgfsetbuttcap%
\pgfsetroundjoin%
\definecolor{currentfill}{rgb}{0.729196,0.086679,0.167240}%
\pgfsetfillcolor{currentfill}%
\pgfsetlinewidth{0.000000pt}%
\definecolor{currentstroke}{rgb}{0.000000,0.000000,0.000000}%
\pgfsetstrokecolor{currentstroke}%
\pgfsetdash{}{0pt}%
\pgfpathmoveto{\pgfqpoint{4.267628in}{2.613520in}}%
\pgfpathlineto{\pgfqpoint{4.511410in}{2.753953in}}%
\pgfpathlineto{\pgfqpoint{4.267628in}{2.613520in}}%
\pgfpathclose%
\pgfusepath{fill}%
\end{pgfscope}%
\begin{pgfscope}%
\pgfpathrectangle{\pgfqpoint{0.017500in}{0.017500in}}{\pgfqpoint{6.218282in}{3.580372in}} %
\pgfusepath{clip}%
\pgfsetbuttcap%
\pgfsetroundjoin%
\definecolor{currentfill}{rgb}{0.729196,0.086679,0.167240}%
\pgfsetfillcolor{currentfill}%
\pgfsetlinewidth{0.000000pt}%
\definecolor{currentstroke}{rgb}{0.000000,0.000000,0.000000}%
\pgfsetstrokecolor{currentstroke}%
\pgfsetdash{}{0pt}%
\pgfpathmoveto{\pgfqpoint{4.267628in}{2.613520in}}%
\pgfpathlineto{\pgfqpoint{4.511410in}{2.753953in}}%
\pgfpathlineto{\pgfqpoint{4.267628in}{2.613520in}}%
\pgfpathclose%
\pgfusepath{fill}%
\end{pgfscope}%
\begin{pgfscope}%
\pgfpathrectangle{\pgfqpoint{0.017500in}{0.017500in}}{\pgfqpoint{6.218282in}{3.580372in}} %
\pgfusepath{clip}%
\pgfsetbuttcap%
\pgfsetroundjoin%
\definecolor{currentfill}{rgb}{0.729196,0.086679,0.167240}%
\pgfsetfillcolor{currentfill}%
\pgfsetlinewidth{0.000000pt}%
\definecolor{currentstroke}{rgb}{0.000000,0.000000,0.000000}%
\pgfsetstrokecolor{currentstroke}%
\pgfsetdash{}{0pt}%
\pgfpathmoveto{\pgfqpoint{4.267628in}{2.613520in}}%
\pgfpathlineto{\pgfqpoint{4.511410in}{2.753953in}}%
\pgfpathlineto{\pgfqpoint{4.267628in}{2.613520in}}%
\pgfpathclose%
\pgfusepath{fill}%
\end{pgfscope}%
\begin{pgfscope}%
\pgfpathrectangle{\pgfqpoint{0.017500in}{0.017500in}}{\pgfqpoint{6.218282in}{3.580372in}} %
\pgfusepath{clip}%
\pgfsetbuttcap%
\pgfsetroundjoin%
\definecolor{currentfill}{rgb}{0.729196,0.086679,0.167240}%
\pgfsetfillcolor{currentfill}%
\pgfsetlinewidth{0.000000pt}%
\definecolor{currentstroke}{rgb}{0.000000,0.000000,0.000000}%
\pgfsetstrokecolor{currentstroke}%
\pgfsetdash{}{0pt}%
\pgfpathmoveto{\pgfqpoint{4.267628in}{2.613520in}}%
\pgfpathlineto{\pgfqpoint{4.511410in}{2.753953in}}%
\pgfpathlineto{\pgfqpoint{4.267628in}{2.613520in}}%
\pgfpathclose%
\pgfusepath{fill}%
\end{pgfscope}%
\begin{pgfscope}%
\pgfpathrectangle{\pgfqpoint{0.017500in}{0.017500in}}{\pgfqpoint{6.218282in}{3.580372in}} %
\pgfusepath{clip}%
\pgfsetbuttcap%
\pgfsetroundjoin%
\definecolor{currentfill}{rgb}{0.729196,0.086679,0.167240}%
\pgfsetfillcolor{currentfill}%
\pgfsetlinewidth{0.000000pt}%
\definecolor{currentstroke}{rgb}{0.000000,0.000000,0.000000}%
\pgfsetstrokecolor{currentstroke}%
\pgfsetdash{}{0pt}%
\pgfpathmoveto{\pgfqpoint{4.267628in}{2.613520in}}%
\pgfpathlineto{\pgfqpoint{4.511410in}{2.753953in}}%
\pgfpathlineto{\pgfqpoint{4.267628in}{2.613520in}}%
\pgfpathclose%
\pgfusepath{fill}%
\end{pgfscope}%
\begin{pgfscope}%
\pgfpathrectangle{\pgfqpoint{0.017500in}{0.017500in}}{\pgfqpoint{6.218282in}{3.580372in}} %
\pgfusepath{clip}%
\pgfsetbuttcap%
\pgfsetroundjoin%
\definecolor{currentfill}{rgb}{0.729196,0.086679,0.167240}%
\pgfsetfillcolor{currentfill}%
\pgfsetlinewidth{0.000000pt}%
\definecolor{currentstroke}{rgb}{0.000000,0.000000,0.000000}%
\pgfsetstrokecolor{currentstroke}%
\pgfsetdash{}{0pt}%
\pgfpathmoveto{\pgfqpoint{4.267628in}{2.613520in}}%
\pgfpathlineto{\pgfqpoint{4.511410in}{2.753953in}}%
\pgfpathlineto{\pgfqpoint{4.267628in}{2.613520in}}%
\pgfpathclose%
\pgfusepath{fill}%
\end{pgfscope}%
\begin{pgfscope}%
\pgfpathrectangle{\pgfqpoint{0.017500in}{0.017500in}}{\pgfqpoint{6.218282in}{3.580372in}} %
\pgfusepath{clip}%
\pgfsetbuttcap%
\pgfsetroundjoin%
\definecolor{currentfill}{rgb}{0.729196,0.086679,0.167240}%
\pgfsetfillcolor{currentfill}%
\pgfsetlinewidth{0.000000pt}%
\definecolor{currentstroke}{rgb}{0.000000,0.000000,0.000000}%
\pgfsetstrokecolor{currentstroke}%
\pgfsetdash{}{0pt}%
\pgfpathmoveto{\pgfqpoint{4.267628in}{2.613520in}}%
\pgfpathlineto{\pgfqpoint{4.511410in}{2.753953in}}%
\pgfpathlineto{\pgfqpoint{4.267628in}{2.613520in}}%
\pgfpathclose%
\pgfusepath{fill}%
\end{pgfscope}%
\begin{pgfscope}%
\pgfpathrectangle{\pgfqpoint{0.017500in}{0.017500in}}{\pgfqpoint{6.218282in}{3.580372in}} %
\pgfusepath{clip}%
\pgfsetbuttcap%
\pgfsetroundjoin%
\definecolor{currentfill}{rgb}{0.729196,0.086679,0.167240}%
\pgfsetfillcolor{currentfill}%
\pgfsetlinewidth{0.000000pt}%
\definecolor{currentstroke}{rgb}{0.000000,0.000000,0.000000}%
\pgfsetstrokecolor{currentstroke}%
\pgfsetdash{}{0pt}%
\pgfpathmoveto{\pgfqpoint{4.267628in}{2.613520in}}%
\pgfpathlineto{\pgfqpoint{4.511410in}{2.753953in}}%
\pgfpathlineto{\pgfqpoint{4.267628in}{2.613520in}}%
\pgfpathclose%
\pgfusepath{fill}%
\end{pgfscope}%
\begin{pgfscope}%
\pgfpathrectangle{\pgfqpoint{0.017500in}{0.017500in}}{\pgfqpoint{6.218282in}{3.580372in}} %
\pgfusepath{clip}%
\pgfsetbuttcap%
\pgfsetroundjoin%
\definecolor{currentfill}{rgb}{0.729196,0.086679,0.167240}%
\pgfsetfillcolor{currentfill}%
\pgfsetlinewidth{0.000000pt}%
\definecolor{currentstroke}{rgb}{0.000000,0.000000,0.000000}%
\pgfsetstrokecolor{currentstroke}%
\pgfsetdash{}{0pt}%
\pgfpathmoveto{\pgfqpoint{4.267628in}{2.613520in}}%
\pgfpathlineto{\pgfqpoint{4.511410in}{2.753953in}}%
\pgfpathlineto{\pgfqpoint{4.267628in}{2.613520in}}%
\pgfpathclose%
\pgfusepath{fill}%
\end{pgfscope}%
\begin{pgfscope}%
\pgfpathrectangle{\pgfqpoint{0.017500in}{0.017500in}}{\pgfqpoint{6.218282in}{3.580372in}} %
\pgfusepath{clip}%
\pgfsetbuttcap%
\pgfsetroundjoin%
\definecolor{currentfill}{rgb}{0.729196,0.086679,0.167240}%
\pgfsetfillcolor{currentfill}%
\pgfsetlinewidth{0.000000pt}%
\definecolor{currentstroke}{rgb}{0.000000,0.000000,0.000000}%
\pgfsetstrokecolor{currentstroke}%
\pgfsetdash{}{0pt}%
\pgfpathmoveto{\pgfqpoint{4.267628in}{2.613520in}}%
\pgfpathlineto{\pgfqpoint{4.511410in}{2.753953in}}%
\pgfpathlineto{\pgfqpoint{4.267628in}{2.613520in}}%
\pgfpathclose%
\pgfusepath{fill}%
\end{pgfscope}%
\begin{pgfscope}%
\pgfpathrectangle{\pgfqpoint{0.017500in}{0.017500in}}{\pgfqpoint{6.218282in}{3.580372in}} %
\pgfusepath{clip}%
\pgfsetbuttcap%
\pgfsetroundjoin%
\definecolor{currentfill}{rgb}{0.729196,0.086679,0.167240}%
\pgfsetfillcolor{currentfill}%
\pgfsetlinewidth{0.000000pt}%
\definecolor{currentstroke}{rgb}{0.000000,0.000000,0.000000}%
\pgfsetstrokecolor{currentstroke}%
\pgfsetdash{}{0pt}%
\pgfpathmoveto{\pgfqpoint{4.267628in}{2.613520in}}%
\pgfpathlineto{\pgfqpoint{4.511410in}{2.753953in}}%
\pgfpathlineto{\pgfqpoint{4.267628in}{2.613520in}}%
\pgfpathclose%
\pgfusepath{fill}%
\end{pgfscope}%
\begin{pgfscope}%
\pgfpathrectangle{\pgfqpoint{0.017500in}{0.017500in}}{\pgfqpoint{6.218282in}{3.580372in}} %
\pgfusepath{clip}%
\pgfsetbuttcap%
\pgfsetroundjoin%
\definecolor{currentfill}{rgb}{0.729196,0.086679,0.167240}%
\pgfsetfillcolor{currentfill}%
\pgfsetlinewidth{0.000000pt}%
\definecolor{currentstroke}{rgb}{0.000000,0.000000,0.000000}%
\pgfsetstrokecolor{currentstroke}%
\pgfsetdash{}{0pt}%
\pgfpathmoveto{\pgfqpoint{4.267628in}{2.613520in}}%
\pgfpathlineto{\pgfqpoint{4.511410in}{2.753953in}}%
\pgfpathlineto{\pgfqpoint{4.267628in}{2.613520in}}%
\pgfpathclose%
\pgfusepath{fill}%
\end{pgfscope}%
\begin{pgfscope}%
\pgfpathrectangle{\pgfqpoint{0.017500in}{0.017500in}}{\pgfqpoint{6.218282in}{3.580372in}} %
\pgfusepath{clip}%
\pgfsetbuttcap%
\pgfsetroundjoin%
\definecolor{currentfill}{rgb}{0.729196,0.086679,0.167240}%
\pgfsetfillcolor{currentfill}%
\pgfsetlinewidth{0.000000pt}%
\definecolor{currentstroke}{rgb}{0.000000,0.000000,0.000000}%
\pgfsetstrokecolor{currentstroke}%
\pgfsetdash{}{0pt}%
\pgfpathmoveto{\pgfqpoint{4.267628in}{2.613520in}}%
\pgfpathlineto{\pgfqpoint{4.511410in}{2.753953in}}%
\pgfpathlineto{\pgfqpoint{4.267628in}{2.613520in}}%
\pgfpathclose%
\pgfusepath{fill}%
\end{pgfscope}%
\begin{pgfscope}%
\pgfpathrectangle{\pgfqpoint{0.017500in}{0.017500in}}{\pgfqpoint{6.218282in}{3.580372in}} %
\pgfusepath{clip}%
\pgfsetbuttcap%
\pgfsetroundjoin%
\definecolor{currentfill}{rgb}{0.729196,0.086679,0.167240}%
\pgfsetfillcolor{currentfill}%
\pgfsetlinewidth{0.000000pt}%
\definecolor{currentstroke}{rgb}{0.000000,0.000000,0.000000}%
\pgfsetstrokecolor{currentstroke}%
\pgfsetdash{}{0pt}%
\pgfpathmoveto{\pgfqpoint{4.267628in}{2.613520in}}%
\pgfpathlineto{\pgfqpoint{4.511410in}{2.753953in}}%
\pgfpathlineto{\pgfqpoint{4.267628in}{2.613520in}}%
\pgfpathclose%
\pgfusepath{fill}%
\end{pgfscope}%
\begin{pgfscope}%
\pgfpathrectangle{\pgfqpoint{0.017500in}{0.017500in}}{\pgfqpoint{6.218282in}{3.580372in}} %
\pgfusepath{clip}%
\pgfsetbuttcap%
\pgfsetroundjoin%
\definecolor{currentfill}{rgb}{0.729196,0.086679,0.167240}%
\pgfsetfillcolor{currentfill}%
\pgfsetlinewidth{0.000000pt}%
\definecolor{currentstroke}{rgb}{0.000000,0.000000,0.000000}%
\pgfsetstrokecolor{currentstroke}%
\pgfsetdash{}{0pt}%
\pgfpathmoveto{\pgfqpoint{4.267628in}{2.613520in}}%
\pgfpathlineto{\pgfqpoint{4.511410in}{2.753953in}}%
\pgfpathlineto{\pgfqpoint{4.267628in}{2.613520in}}%
\pgfpathclose%
\pgfusepath{fill}%
\end{pgfscope}%
\begin{pgfscope}%
\pgfpathrectangle{\pgfqpoint{0.017500in}{0.017500in}}{\pgfqpoint{6.218282in}{3.580372in}} %
\pgfusepath{clip}%
\pgfsetbuttcap%
\pgfsetroundjoin%
\definecolor{currentfill}{rgb}{0.729196,0.086679,0.167240}%
\pgfsetfillcolor{currentfill}%
\pgfsetlinewidth{0.000000pt}%
\definecolor{currentstroke}{rgb}{0.000000,0.000000,0.000000}%
\pgfsetstrokecolor{currentstroke}%
\pgfsetdash{}{0pt}%
\pgfpathmoveto{\pgfqpoint{4.267628in}{2.613520in}}%
\pgfpathlineto{\pgfqpoint{4.511410in}{2.753953in}}%
\pgfpathlineto{\pgfqpoint{4.267628in}{2.613520in}}%
\pgfpathclose%
\pgfusepath{fill}%
\end{pgfscope}%
\begin{pgfscope}%
\pgfpathrectangle{\pgfqpoint{0.017500in}{0.017500in}}{\pgfqpoint{6.218282in}{3.580372in}} %
\pgfusepath{clip}%
\pgfsetbuttcap%
\pgfsetroundjoin%
\definecolor{currentfill}{rgb}{0.729196,0.086679,0.167240}%
\pgfsetfillcolor{currentfill}%
\pgfsetlinewidth{0.000000pt}%
\definecolor{currentstroke}{rgb}{0.000000,0.000000,0.000000}%
\pgfsetstrokecolor{currentstroke}%
\pgfsetdash{}{0pt}%
\pgfpathmoveto{\pgfqpoint{4.267628in}{2.613520in}}%
\pgfpathlineto{\pgfqpoint{4.511410in}{2.753953in}}%
\pgfpathlineto{\pgfqpoint{4.267628in}{2.613520in}}%
\pgfpathclose%
\pgfusepath{fill}%
\end{pgfscope}%
\begin{pgfscope}%
\pgfpathrectangle{\pgfqpoint{0.017500in}{0.017500in}}{\pgfqpoint{6.218282in}{3.580372in}} %
\pgfusepath{clip}%
\pgfsetbuttcap%
\pgfsetroundjoin%
\definecolor{currentfill}{rgb}{0.729196,0.086679,0.167240}%
\pgfsetfillcolor{currentfill}%
\pgfsetlinewidth{0.000000pt}%
\definecolor{currentstroke}{rgb}{0.000000,0.000000,0.000000}%
\pgfsetstrokecolor{currentstroke}%
\pgfsetdash{}{0pt}%
\pgfpathmoveto{\pgfqpoint{4.267628in}{2.613520in}}%
\pgfpathlineto{\pgfqpoint{4.511410in}{2.753953in}}%
\pgfpathlineto{\pgfqpoint{4.267628in}{2.613520in}}%
\pgfpathclose%
\pgfusepath{fill}%
\end{pgfscope}%
\begin{pgfscope}%
\pgfpathrectangle{\pgfqpoint{0.017500in}{0.017500in}}{\pgfqpoint{6.218282in}{3.580372in}} %
\pgfusepath{clip}%
\pgfsetbuttcap%
\pgfsetroundjoin%
\definecolor{currentfill}{rgb}{0.729196,0.086679,0.167240}%
\pgfsetfillcolor{currentfill}%
\pgfsetlinewidth{0.000000pt}%
\definecolor{currentstroke}{rgb}{0.000000,0.000000,0.000000}%
\pgfsetstrokecolor{currentstroke}%
\pgfsetdash{}{0pt}%
\pgfpathmoveto{\pgfqpoint{4.267628in}{2.613520in}}%
\pgfpathlineto{\pgfqpoint{4.511410in}{2.753953in}}%
\pgfpathlineto{\pgfqpoint{4.267628in}{2.613520in}}%
\pgfpathclose%
\pgfusepath{fill}%
\end{pgfscope}%
\begin{pgfscope}%
\pgfpathrectangle{\pgfqpoint{0.017500in}{0.017500in}}{\pgfqpoint{6.218282in}{3.580372in}} %
\pgfusepath{clip}%
\pgfsetbuttcap%
\pgfsetroundjoin%
\definecolor{currentfill}{rgb}{0.729196,0.086679,0.167240}%
\pgfsetfillcolor{currentfill}%
\pgfsetlinewidth{0.000000pt}%
\definecolor{currentstroke}{rgb}{0.000000,0.000000,0.000000}%
\pgfsetstrokecolor{currentstroke}%
\pgfsetdash{}{0pt}%
\pgfpathmoveto{\pgfqpoint{4.267628in}{2.613520in}}%
\pgfpathlineto{\pgfqpoint{4.511410in}{2.753953in}}%
\pgfpathlineto{\pgfqpoint{4.267628in}{2.613520in}}%
\pgfpathclose%
\pgfusepath{fill}%
\end{pgfscope}%
\begin{pgfscope}%
\pgfpathrectangle{\pgfqpoint{0.017500in}{0.017500in}}{\pgfqpoint{6.218282in}{3.580372in}} %
\pgfusepath{clip}%
\pgfsetbuttcap%
\pgfsetroundjoin%
\definecolor{currentfill}{rgb}{0.729196,0.086679,0.167240}%
\pgfsetfillcolor{currentfill}%
\pgfsetlinewidth{0.000000pt}%
\definecolor{currentstroke}{rgb}{0.000000,0.000000,0.000000}%
\pgfsetstrokecolor{currentstroke}%
\pgfsetdash{}{0pt}%
\pgfpathmoveto{\pgfqpoint{4.267628in}{2.613520in}}%
\pgfpathlineto{\pgfqpoint{4.511410in}{2.753953in}}%
\pgfpathlineto{\pgfqpoint{4.267628in}{2.613520in}}%
\pgfpathclose%
\pgfusepath{fill}%
\end{pgfscope}%
\begin{pgfscope}%
\pgfpathrectangle{\pgfqpoint{0.017500in}{0.017500in}}{\pgfqpoint{6.218282in}{3.580372in}} %
\pgfusepath{clip}%
\pgfsetbuttcap%
\pgfsetroundjoin%
\definecolor{currentfill}{rgb}{0.729196,0.086679,0.167240}%
\pgfsetfillcolor{currentfill}%
\pgfsetlinewidth{0.000000pt}%
\definecolor{currentstroke}{rgb}{0.000000,0.000000,0.000000}%
\pgfsetstrokecolor{currentstroke}%
\pgfsetdash{}{0pt}%
\pgfpathmoveto{\pgfqpoint{4.267628in}{2.613520in}}%
\pgfpathlineto{\pgfqpoint{4.511410in}{2.753953in}}%
\pgfpathlineto{\pgfqpoint{4.267628in}{2.613520in}}%
\pgfpathclose%
\pgfusepath{fill}%
\end{pgfscope}%
\begin{pgfscope}%
\pgfpathrectangle{\pgfqpoint{0.017500in}{0.017500in}}{\pgfqpoint{6.218282in}{3.580372in}} %
\pgfusepath{clip}%
\pgfsetbuttcap%
\pgfsetroundjoin%
\definecolor{currentfill}{rgb}{0.729196,0.086679,0.167240}%
\pgfsetfillcolor{currentfill}%
\pgfsetlinewidth{0.000000pt}%
\definecolor{currentstroke}{rgb}{0.000000,0.000000,0.000000}%
\pgfsetstrokecolor{currentstroke}%
\pgfsetdash{}{0pt}%
\pgfpathmoveto{\pgfqpoint{4.267628in}{2.613520in}}%
\pgfpathlineto{\pgfqpoint{4.511410in}{2.753953in}}%
\pgfpathlineto{\pgfqpoint{4.267628in}{2.613520in}}%
\pgfpathclose%
\pgfusepath{fill}%
\end{pgfscope}%
\begin{pgfscope}%
\pgfpathrectangle{\pgfqpoint{0.017500in}{0.017500in}}{\pgfqpoint{6.218282in}{3.580372in}} %
\pgfusepath{clip}%
\pgfsetbuttcap%
\pgfsetroundjoin%
\definecolor{currentfill}{rgb}{0.729196,0.086679,0.167240}%
\pgfsetfillcolor{currentfill}%
\pgfsetlinewidth{0.000000pt}%
\definecolor{currentstroke}{rgb}{0.000000,0.000000,0.000000}%
\pgfsetstrokecolor{currentstroke}%
\pgfsetdash{}{0pt}%
\pgfpathmoveto{\pgfqpoint{4.267628in}{2.613520in}}%
\pgfpathlineto{\pgfqpoint{4.511410in}{2.753953in}}%
\pgfpathlineto{\pgfqpoint{4.267628in}{2.613520in}}%
\pgfpathclose%
\pgfusepath{fill}%
\end{pgfscope}%
\begin{pgfscope}%
\pgfpathrectangle{\pgfqpoint{0.017500in}{0.017500in}}{\pgfqpoint{6.218282in}{3.580372in}} %
\pgfusepath{clip}%
\pgfsetbuttcap%
\pgfsetroundjoin%
\definecolor{currentfill}{rgb}{0.729196,0.086679,0.167240}%
\pgfsetfillcolor{currentfill}%
\pgfsetlinewidth{0.000000pt}%
\definecolor{currentstroke}{rgb}{0.000000,0.000000,0.000000}%
\pgfsetstrokecolor{currentstroke}%
\pgfsetdash{}{0pt}%
\pgfpathmoveto{\pgfqpoint{4.267628in}{2.613520in}}%
\pgfpathlineto{\pgfqpoint{4.511410in}{2.753953in}}%
\pgfpathlineto{\pgfqpoint{4.267628in}{2.613520in}}%
\pgfpathclose%
\pgfusepath{fill}%
\end{pgfscope}%
\begin{pgfscope}%
\pgfpathrectangle{\pgfqpoint{0.017500in}{0.017500in}}{\pgfqpoint{6.218282in}{3.580372in}} %
\pgfusepath{clip}%
\pgfsetbuttcap%
\pgfsetroundjoin%
\definecolor{currentfill}{rgb}{0.810616,0.268797,0.235428}%
\pgfsetfillcolor{currentfill}%
\pgfsetlinewidth{0.000000pt}%
\definecolor{currentstroke}{rgb}{0.000000,0.000000,0.000000}%
\pgfsetstrokecolor{currentstroke}%
\pgfsetdash{}{0pt}%
\pgfpathmoveto{\pgfqpoint{2.410872in}{2.508083in}}%
\pgfpathlineto{\pgfqpoint{2.674254in}{2.688576in}}%
\pgfpathlineto{\pgfqpoint{2.410872in}{2.508083in}}%
\pgfpathclose%
\pgfusepath{fill}%
\end{pgfscope}%
\begin{pgfscope}%
\pgfpathrectangle{\pgfqpoint{0.017500in}{0.017500in}}{\pgfqpoint{6.218282in}{3.580372in}} %
\pgfusepath{clip}%
\pgfsetbuttcap%
\pgfsetroundjoin%
\definecolor{currentfill}{rgb}{0.810616,0.268797,0.235428}%
\pgfsetfillcolor{currentfill}%
\pgfsetlinewidth{0.000000pt}%
\definecolor{currentstroke}{rgb}{0.000000,0.000000,0.000000}%
\pgfsetstrokecolor{currentstroke}%
\pgfsetdash{}{0pt}%
\pgfpathmoveto{\pgfqpoint{2.410872in}{2.508083in}}%
\pgfpathlineto{\pgfqpoint{2.674254in}{2.688576in}}%
\pgfpathlineto{\pgfqpoint{2.410872in}{2.508083in}}%
\pgfpathclose%
\pgfusepath{fill}%
\end{pgfscope}%
\begin{pgfscope}%
\pgfpathrectangle{\pgfqpoint{0.017500in}{0.017500in}}{\pgfqpoint{6.218282in}{3.580372in}} %
\pgfusepath{clip}%
\pgfsetbuttcap%
\pgfsetroundjoin%
\definecolor{currentfill}{rgb}{0.810616,0.268797,0.235428}%
\pgfsetfillcolor{currentfill}%
\pgfsetlinewidth{0.000000pt}%
\definecolor{currentstroke}{rgb}{0.000000,0.000000,0.000000}%
\pgfsetstrokecolor{currentstroke}%
\pgfsetdash{}{0pt}%
\pgfpathmoveto{\pgfqpoint{2.410872in}{2.508083in}}%
\pgfpathlineto{\pgfqpoint{2.674254in}{2.688576in}}%
\pgfpathlineto{\pgfqpoint{2.410872in}{2.508083in}}%
\pgfpathclose%
\pgfusepath{fill}%
\end{pgfscope}%
\begin{pgfscope}%
\pgfpathrectangle{\pgfqpoint{0.017500in}{0.017500in}}{\pgfqpoint{6.218282in}{3.580372in}} %
\pgfusepath{clip}%
\pgfsetbuttcap%
\pgfsetroundjoin%
\definecolor{currentfill}{rgb}{0.810616,0.268797,0.235428}%
\pgfsetfillcolor{currentfill}%
\pgfsetlinewidth{0.000000pt}%
\definecolor{currentstroke}{rgb}{0.000000,0.000000,0.000000}%
\pgfsetstrokecolor{currentstroke}%
\pgfsetdash{}{0pt}%
\pgfpathmoveto{\pgfqpoint{2.410872in}{2.508083in}}%
\pgfpathlineto{\pgfqpoint{2.674254in}{2.688576in}}%
\pgfpathlineto{\pgfqpoint{2.410872in}{2.508083in}}%
\pgfpathclose%
\pgfusepath{fill}%
\end{pgfscope}%
\begin{pgfscope}%
\pgfpathrectangle{\pgfqpoint{0.017500in}{0.017500in}}{\pgfqpoint{6.218282in}{3.580372in}} %
\pgfusepath{clip}%
\pgfsetbuttcap%
\pgfsetroundjoin%
\definecolor{currentfill}{rgb}{0.810616,0.268797,0.235428}%
\pgfsetfillcolor{currentfill}%
\pgfsetlinewidth{0.000000pt}%
\definecolor{currentstroke}{rgb}{0.000000,0.000000,0.000000}%
\pgfsetstrokecolor{currentstroke}%
\pgfsetdash{}{0pt}%
\pgfpathmoveto{\pgfqpoint{2.410872in}{2.508083in}}%
\pgfpathlineto{\pgfqpoint{2.674254in}{2.688576in}}%
\pgfpathlineto{\pgfqpoint{2.410872in}{2.508083in}}%
\pgfpathclose%
\pgfusepath{fill}%
\end{pgfscope}%
\begin{pgfscope}%
\pgfpathrectangle{\pgfqpoint{0.017500in}{0.017500in}}{\pgfqpoint{6.218282in}{3.580372in}} %
\pgfusepath{clip}%
\pgfsetbuttcap%
\pgfsetroundjoin%
\definecolor{currentfill}{rgb}{0.810616,0.268797,0.235428}%
\pgfsetfillcolor{currentfill}%
\pgfsetlinewidth{0.000000pt}%
\definecolor{currentstroke}{rgb}{0.000000,0.000000,0.000000}%
\pgfsetstrokecolor{currentstroke}%
\pgfsetdash{}{0pt}%
\pgfpathmoveto{\pgfqpoint{2.410872in}{2.508083in}}%
\pgfpathlineto{\pgfqpoint{2.674254in}{2.688576in}}%
\pgfpathlineto{\pgfqpoint{2.410872in}{2.508083in}}%
\pgfpathclose%
\pgfusepath{fill}%
\end{pgfscope}%
\begin{pgfscope}%
\pgfpathrectangle{\pgfqpoint{0.017500in}{0.017500in}}{\pgfqpoint{6.218282in}{3.580372in}} %
\pgfusepath{clip}%
\pgfsetbuttcap%
\pgfsetroundjoin%
\definecolor{currentfill}{rgb}{0.810616,0.268797,0.235428}%
\pgfsetfillcolor{currentfill}%
\pgfsetlinewidth{0.000000pt}%
\definecolor{currentstroke}{rgb}{0.000000,0.000000,0.000000}%
\pgfsetstrokecolor{currentstroke}%
\pgfsetdash{}{0pt}%
\pgfpathmoveto{\pgfqpoint{2.410872in}{2.508083in}}%
\pgfpathlineto{\pgfqpoint{2.674254in}{2.688576in}}%
\pgfpathlineto{\pgfqpoint{2.410872in}{2.508083in}}%
\pgfpathclose%
\pgfusepath{fill}%
\end{pgfscope}%
\begin{pgfscope}%
\pgfpathrectangle{\pgfqpoint{0.017500in}{0.017500in}}{\pgfqpoint{6.218282in}{3.580372in}} %
\pgfusepath{clip}%
\pgfsetbuttcap%
\pgfsetroundjoin%
\definecolor{currentfill}{rgb}{0.810616,0.268797,0.235428}%
\pgfsetfillcolor{currentfill}%
\pgfsetlinewidth{0.000000pt}%
\definecolor{currentstroke}{rgb}{0.000000,0.000000,0.000000}%
\pgfsetstrokecolor{currentstroke}%
\pgfsetdash{}{0pt}%
\pgfpathmoveto{\pgfqpoint{2.410872in}{2.508083in}}%
\pgfpathlineto{\pgfqpoint{2.674254in}{2.688576in}}%
\pgfpathlineto{\pgfqpoint{2.410872in}{2.508083in}}%
\pgfpathclose%
\pgfusepath{fill}%
\end{pgfscope}%
\begin{pgfscope}%
\pgfpathrectangle{\pgfqpoint{0.017500in}{0.017500in}}{\pgfqpoint{6.218282in}{3.580372in}} %
\pgfusepath{clip}%
\pgfsetbuttcap%
\pgfsetroundjoin%
\definecolor{currentfill}{rgb}{0.810616,0.268797,0.235428}%
\pgfsetfillcolor{currentfill}%
\pgfsetlinewidth{0.000000pt}%
\definecolor{currentstroke}{rgb}{0.000000,0.000000,0.000000}%
\pgfsetstrokecolor{currentstroke}%
\pgfsetdash{}{0pt}%
\pgfpathmoveto{\pgfqpoint{2.410872in}{2.508083in}}%
\pgfpathlineto{\pgfqpoint{2.674254in}{2.688576in}}%
\pgfpathlineto{\pgfqpoint{2.410872in}{2.508083in}}%
\pgfpathclose%
\pgfusepath{fill}%
\end{pgfscope}%
\begin{pgfscope}%
\pgfpathrectangle{\pgfqpoint{0.017500in}{0.017500in}}{\pgfqpoint{6.218282in}{3.580372in}} %
\pgfusepath{clip}%
\pgfsetbuttcap%
\pgfsetroundjoin%
\definecolor{currentfill}{rgb}{0.810616,0.268797,0.235428}%
\pgfsetfillcolor{currentfill}%
\pgfsetlinewidth{0.000000pt}%
\definecolor{currentstroke}{rgb}{0.000000,0.000000,0.000000}%
\pgfsetstrokecolor{currentstroke}%
\pgfsetdash{}{0pt}%
\pgfpathmoveto{\pgfqpoint{2.410872in}{2.508083in}}%
\pgfpathlineto{\pgfqpoint{2.674254in}{2.688576in}}%
\pgfpathlineto{\pgfqpoint{2.410872in}{2.508083in}}%
\pgfpathclose%
\pgfusepath{fill}%
\end{pgfscope}%
\begin{pgfscope}%
\pgfpathrectangle{\pgfqpoint{0.017500in}{0.017500in}}{\pgfqpoint{6.218282in}{3.580372in}} %
\pgfusepath{clip}%
\pgfsetbuttcap%
\pgfsetroundjoin%
\definecolor{currentfill}{rgb}{0.810616,0.268797,0.235428}%
\pgfsetfillcolor{currentfill}%
\pgfsetlinewidth{0.000000pt}%
\definecolor{currentstroke}{rgb}{0.000000,0.000000,0.000000}%
\pgfsetstrokecolor{currentstroke}%
\pgfsetdash{}{0pt}%
\pgfpathmoveto{\pgfqpoint{2.410872in}{2.508083in}}%
\pgfpathlineto{\pgfqpoint{2.674254in}{2.688576in}}%
\pgfpathlineto{\pgfqpoint{2.410872in}{2.508083in}}%
\pgfpathclose%
\pgfusepath{fill}%
\end{pgfscope}%
\begin{pgfscope}%
\pgfpathrectangle{\pgfqpoint{0.017500in}{0.017500in}}{\pgfqpoint{6.218282in}{3.580372in}} %
\pgfusepath{clip}%
\pgfsetbuttcap%
\pgfsetroundjoin%
\definecolor{currentfill}{rgb}{0.810616,0.268797,0.235428}%
\pgfsetfillcolor{currentfill}%
\pgfsetlinewidth{0.000000pt}%
\definecolor{currentstroke}{rgb}{0.000000,0.000000,0.000000}%
\pgfsetstrokecolor{currentstroke}%
\pgfsetdash{}{0pt}%
\pgfpathmoveto{\pgfqpoint{2.410872in}{2.508083in}}%
\pgfpathlineto{\pgfqpoint{2.674254in}{2.688576in}}%
\pgfpathlineto{\pgfqpoint{2.410872in}{2.508083in}}%
\pgfpathclose%
\pgfusepath{fill}%
\end{pgfscope}%
\begin{pgfscope}%
\pgfpathrectangle{\pgfqpoint{0.017500in}{0.017500in}}{\pgfqpoint{6.218282in}{3.580372in}} %
\pgfusepath{clip}%
\pgfsetbuttcap%
\pgfsetroundjoin%
\definecolor{currentfill}{rgb}{0.810616,0.268797,0.235428}%
\pgfsetfillcolor{currentfill}%
\pgfsetlinewidth{0.000000pt}%
\definecolor{currentstroke}{rgb}{0.000000,0.000000,0.000000}%
\pgfsetstrokecolor{currentstroke}%
\pgfsetdash{}{0pt}%
\pgfpathmoveto{\pgfqpoint{2.410872in}{2.508083in}}%
\pgfpathlineto{\pgfqpoint{2.674254in}{2.688576in}}%
\pgfpathlineto{\pgfqpoint{2.410872in}{2.508083in}}%
\pgfpathclose%
\pgfusepath{fill}%
\end{pgfscope}%
\begin{pgfscope}%
\pgfpathrectangle{\pgfqpoint{0.017500in}{0.017500in}}{\pgfqpoint{6.218282in}{3.580372in}} %
\pgfusepath{clip}%
\pgfsetbuttcap%
\pgfsetroundjoin%
\definecolor{currentfill}{rgb}{0.810616,0.268797,0.235428}%
\pgfsetfillcolor{currentfill}%
\pgfsetlinewidth{0.000000pt}%
\definecolor{currentstroke}{rgb}{0.000000,0.000000,0.000000}%
\pgfsetstrokecolor{currentstroke}%
\pgfsetdash{}{0pt}%
\pgfpathmoveto{\pgfqpoint{2.410872in}{2.508083in}}%
\pgfpathlineto{\pgfqpoint{2.674254in}{2.688576in}}%
\pgfpathlineto{\pgfqpoint{2.410872in}{2.508083in}}%
\pgfpathclose%
\pgfusepath{fill}%
\end{pgfscope}%
\begin{pgfscope}%
\pgfpathrectangle{\pgfqpoint{0.017500in}{0.017500in}}{\pgfqpoint{6.218282in}{3.580372in}} %
\pgfusepath{clip}%
\pgfsetbuttcap%
\pgfsetroundjoin%
\definecolor{currentfill}{rgb}{0.810616,0.268797,0.235428}%
\pgfsetfillcolor{currentfill}%
\pgfsetlinewidth{0.000000pt}%
\definecolor{currentstroke}{rgb}{0.000000,0.000000,0.000000}%
\pgfsetstrokecolor{currentstroke}%
\pgfsetdash{}{0pt}%
\pgfpathmoveto{\pgfqpoint{2.410872in}{2.508083in}}%
\pgfpathlineto{\pgfqpoint{2.674254in}{2.688576in}}%
\pgfpathlineto{\pgfqpoint{2.410872in}{2.508083in}}%
\pgfpathclose%
\pgfusepath{fill}%
\end{pgfscope}%
\begin{pgfscope}%
\pgfpathrectangle{\pgfqpoint{0.017500in}{0.017500in}}{\pgfqpoint{6.218282in}{3.580372in}} %
\pgfusepath{clip}%
\pgfsetbuttcap%
\pgfsetroundjoin%
\definecolor{currentfill}{rgb}{0.810616,0.268797,0.235428}%
\pgfsetfillcolor{currentfill}%
\pgfsetlinewidth{0.000000pt}%
\definecolor{currentstroke}{rgb}{0.000000,0.000000,0.000000}%
\pgfsetstrokecolor{currentstroke}%
\pgfsetdash{}{0pt}%
\pgfpathmoveto{\pgfqpoint{2.410872in}{2.508083in}}%
\pgfpathlineto{\pgfqpoint{2.674254in}{2.688576in}}%
\pgfpathlineto{\pgfqpoint{2.410872in}{2.508083in}}%
\pgfpathclose%
\pgfusepath{fill}%
\end{pgfscope}%
\begin{pgfscope}%
\pgfpathrectangle{\pgfqpoint{0.017500in}{0.017500in}}{\pgfqpoint{6.218282in}{3.580372in}} %
\pgfusepath{clip}%
\pgfsetbuttcap%
\pgfsetroundjoin%
\definecolor{currentfill}{rgb}{0.810616,0.268797,0.235428}%
\pgfsetfillcolor{currentfill}%
\pgfsetlinewidth{0.000000pt}%
\definecolor{currentstroke}{rgb}{0.000000,0.000000,0.000000}%
\pgfsetstrokecolor{currentstroke}%
\pgfsetdash{}{0pt}%
\pgfpathmoveto{\pgfqpoint{2.410872in}{2.508083in}}%
\pgfpathlineto{\pgfqpoint{2.674254in}{2.688576in}}%
\pgfpathlineto{\pgfqpoint{2.410872in}{2.508083in}}%
\pgfpathclose%
\pgfusepath{fill}%
\end{pgfscope}%
\begin{pgfscope}%
\pgfpathrectangle{\pgfqpoint{0.017500in}{0.017500in}}{\pgfqpoint{6.218282in}{3.580372in}} %
\pgfusepath{clip}%
\pgfsetbuttcap%
\pgfsetroundjoin%
\definecolor{currentfill}{rgb}{0.810616,0.268797,0.235428}%
\pgfsetfillcolor{currentfill}%
\pgfsetlinewidth{0.000000pt}%
\definecolor{currentstroke}{rgb}{0.000000,0.000000,0.000000}%
\pgfsetstrokecolor{currentstroke}%
\pgfsetdash{}{0pt}%
\pgfpathmoveto{\pgfqpoint{2.410872in}{2.508083in}}%
\pgfpathlineto{\pgfqpoint{2.674254in}{2.688576in}}%
\pgfpathlineto{\pgfqpoint{2.410872in}{2.508083in}}%
\pgfpathclose%
\pgfusepath{fill}%
\end{pgfscope}%
\begin{pgfscope}%
\pgfpathrectangle{\pgfqpoint{0.017500in}{0.017500in}}{\pgfqpoint{6.218282in}{3.580372in}} %
\pgfusepath{clip}%
\pgfsetbuttcap%
\pgfsetroundjoin%
\definecolor{currentfill}{rgb}{0.810616,0.268797,0.235428}%
\pgfsetfillcolor{currentfill}%
\pgfsetlinewidth{0.000000pt}%
\definecolor{currentstroke}{rgb}{0.000000,0.000000,0.000000}%
\pgfsetstrokecolor{currentstroke}%
\pgfsetdash{}{0pt}%
\pgfpathmoveto{\pgfqpoint{2.410872in}{2.508083in}}%
\pgfpathlineto{\pgfqpoint{2.674254in}{2.688576in}}%
\pgfpathlineto{\pgfqpoint{2.410872in}{2.508083in}}%
\pgfpathclose%
\pgfusepath{fill}%
\end{pgfscope}%
\begin{pgfscope}%
\pgfpathrectangle{\pgfqpoint{0.017500in}{0.017500in}}{\pgfqpoint{6.218282in}{3.580372in}} %
\pgfusepath{clip}%
\pgfsetbuttcap%
\pgfsetroundjoin%
\definecolor{currentfill}{rgb}{0.810616,0.268797,0.235428}%
\pgfsetfillcolor{currentfill}%
\pgfsetlinewidth{0.000000pt}%
\definecolor{currentstroke}{rgb}{0.000000,0.000000,0.000000}%
\pgfsetstrokecolor{currentstroke}%
\pgfsetdash{}{0pt}%
\pgfpathmoveto{\pgfqpoint{2.410872in}{2.508083in}}%
\pgfpathlineto{\pgfqpoint{2.674254in}{2.688576in}}%
\pgfpathlineto{\pgfqpoint{2.410872in}{2.508083in}}%
\pgfpathclose%
\pgfusepath{fill}%
\end{pgfscope}%
\begin{pgfscope}%
\pgfpathrectangle{\pgfqpoint{0.017500in}{0.017500in}}{\pgfqpoint{6.218282in}{3.580372in}} %
\pgfusepath{clip}%
\pgfsetbuttcap%
\pgfsetroundjoin%
\definecolor{currentfill}{rgb}{0.810616,0.268797,0.235428}%
\pgfsetfillcolor{currentfill}%
\pgfsetlinewidth{0.000000pt}%
\definecolor{currentstroke}{rgb}{0.000000,0.000000,0.000000}%
\pgfsetstrokecolor{currentstroke}%
\pgfsetdash{}{0pt}%
\pgfpathmoveto{\pgfqpoint{2.410872in}{2.508083in}}%
\pgfpathlineto{\pgfqpoint{2.674254in}{2.688576in}}%
\pgfpathlineto{\pgfqpoint{2.410872in}{2.508083in}}%
\pgfpathclose%
\pgfusepath{fill}%
\end{pgfscope}%
\begin{pgfscope}%
\pgfpathrectangle{\pgfqpoint{0.017500in}{0.017500in}}{\pgfqpoint{6.218282in}{3.580372in}} %
\pgfusepath{clip}%
\pgfsetbuttcap%
\pgfsetroundjoin%
\definecolor{currentfill}{rgb}{0.810616,0.268797,0.235428}%
\pgfsetfillcolor{currentfill}%
\pgfsetlinewidth{0.000000pt}%
\definecolor{currentstroke}{rgb}{0.000000,0.000000,0.000000}%
\pgfsetstrokecolor{currentstroke}%
\pgfsetdash{}{0pt}%
\pgfpathmoveto{\pgfqpoint{2.410872in}{2.508083in}}%
\pgfpathlineto{\pgfqpoint{2.674254in}{2.688576in}}%
\pgfpathlineto{\pgfqpoint{2.410872in}{2.508083in}}%
\pgfpathclose%
\pgfusepath{fill}%
\end{pgfscope}%
\begin{pgfscope}%
\pgfpathrectangle{\pgfqpoint{0.017500in}{0.017500in}}{\pgfqpoint{6.218282in}{3.580372in}} %
\pgfusepath{clip}%
\pgfsetbuttcap%
\pgfsetroundjoin%
\definecolor{currentfill}{rgb}{0.810616,0.268797,0.235428}%
\pgfsetfillcolor{currentfill}%
\pgfsetlinewidth{0.000000pt}%
\definecolor{currentstroke}{rgb}{0.000000,0.000000,0.000000}%
\pgfsetstrokecolor{currentstroke}%
\pgfsetdash{}{0pt}%
\pgfpathmoveto{\pgfqpoint{2.410872in}{2.508083in}}%
\pgfpathlineto{\pgfqpoint{2.674254in}{2.688576in}}%
\pgfpathlineto{\pgfqpoint{2.410872in}{2.508083in}}%
\pgfpathclose%
\pgfusepath{fill}%
\end{pgfscope}%
\begin{pgfscope}%
\pgfpathrectangle{\pgfqpoint{0.017500in}{0.017500in}}{\pgfqpoint{6.218282in}{3.580372in}} %
\pgfusepath{clip}%
\pgfsetbuttcap%
\pgfsetroundjoin%
\definecolor{currentfill}{rgb}{0.810616,0.268797,0.235428}%
\pgfsetfillcolor{currentfill}%
\pgfsetlinewidth{0.000000pt}%
\definecolor{currentstroke}{rgb}{0.000000,0.000000,0.000000}%
\pgfsetstrokecolor{currentstroke}%
\pgfsetdash{}{0pt}%
\pgfpathmoveto{\pgfqpoint{2.410872in}{2.508083in}}%
\pgfpathlineto{\pgfqpoint{2.674254in}{2.688576in}}%
\pgfpathlineto{\pgfqpoint{2.410872in}{2.508083in}}%
\pgfpathclose%
\pgfusepath{fill}%
\end{pgfscope}%
\begin{pgfscope}%
\pgfpathrectangle{\pgfqpoint{0.017500in}{0.017500in}}{\pgfqpoint{6.218282in}{3.580372in}} %
\pgfusepath{clip}%
\pgfsetbuttcap%
\pgfsetroundjoin%
\definecolor{currentfill}{rgb}{0.810616,0.268797,0.235428}%
\pgfsetfillcolor{currentfill}%
\pgfsetlinewidth{0.000000pt}%
\definecolor{currentstroke}{rgb}{0.000000,0.000000,0.000000}%
\pgfsetstrokecolor{currentstroke}%
\pgfsetdash{}{0pt}%
\pgfpathmoveto{\pgfqpoint{2.410872in}{2.508083in}}%
\pgfpathlineto{\pgfqpoint{2.674254in}{2.688576in}}%
\pgfpathlineto{\pgfqpoint{2.410872in}{2.508083in}}%
\pgfpathclose%
\pgfusepath{fill}%
\end{pgfscope}%
\begin{pgfscope}%
\pgfpathrectangle{\pgfqpoint{0.017500in}{0.017500in}}{\pgfqpoint{6.218282in}{3.580372in}} %
\pgfusepath{clip}%
\pgfsetbuttcap%
\pgfsetroundjoin%
\definecolor{currentfill}{rgb}{0.810616,0.268797,0.235428}%
\pgfsetfillcolor{currentfill}%
\pgfsetlinewidth{0.000000pt}%
\definecolor{currentstroke}{rgb}{0.000000,0.000000,0.000000}%
\pgfsetstrokecolor{currentstroke}%
\pgfsetdash{}{0pt}%
\pgfpathmoveto{\pgfqpoint{2.410872in}{2.508083in}}%
\pgfpathlineto{\pgfqpoint{2.674254in}{2.688576in}}%
\pgfpathlineto{\pgfqpoint{2.410872in}{2.508083in}}%
\pgfpathclose%
\pgfusepath{fill}%
\end{pgfscope}%
\begin{pgfscope}%
\pgfpathrectangle{\pgfqpoint{0.017500in}{0.017500in}}{\pgfqpoint{6.218282in}{3.580372in}} %
\pgfusepath{clip}%
\pgfsetbuttcap%
\pgfsetroundjoin%
\definecolor{currentfill}{rgb}{0.810616,0.268797,0.235428}%
\pgfsetfillcolor{currentfill}%
\pgfsetlinewidth{0.000000pt}%
\definecolor{currentstroke}{rgb}{0.000000,0.000000,0.000000}%
\pgfsetstrokecolor{currentstroke}%
\pgfsetdash{}{0pt}%
\pgfpathmoveto{\pgfqpoint{2.410872in}{2.508083in}}%
\pgfpathlineto{\pgfqpoint{2.674254in}{2.688576in}}%
\pgfpathlineto{\pgfqpoint{2.410872in}{2.508083in}}%
\pgfpathclose%
\pgfusepath{fill}%
\end{pgfscope}%
\begin{pgfscope}%
\pgfpathrectangle{\pgfqpoint{0.017500in}{0.017500in}}{\pgfqpoint{6.218282in}{3.580372in}} %
\pgfusepath{clip}%
\pgfsetbuttcap%
\pgfsetroundjoin%
\definecolor{currentfill}{rgb}{0.810616,0.268797,0.235428}%
\pgfsetfillcolor{currentfill}%
\pgfsetlinewidth{0.000000pt}%
\definecolor{currentstroke}{rgb}{0.000000,0.000000,0.000000}%
\pgfsetstrokecolor{currentstroke}%
\pgfsetdash{}{0pt}%
\pgfpathmoveto{\pgfqpoint{2.410872in}{2.508083in}}%
\pgfpathlineto{\pgfqpoint{2.674254in}{2.688576in}}%
\pgfpathlineto{\pgfqpoint{2.410872in}{2.508083in}}%
\pgfpathclose%
\pgfusepath{fill}%
\end{pgfscope}%
\begin{pgfscope}%
\pgfpathrectangle{\pgfqpoint{0.017500in}{0.017500in}}{\pgfqpoint{6.218282in}{3.580372in}} %
\pgfusepath{clip}%
\pgfsetbuttcap%
\pgfsetroundjoin%
\definecolor{currentfill}{rgb}{0.810616,0.268797,0.235428}%
\pgfsetfillcolor{currentfill}%
\pgfsetlinewidth{0.000000pt}%
\definecolor{currentstroke}{rgb}{0.000000,0.000000,0.000000}%
\pgfsetstrokecolor{currentstroke}%
\pgfsetdash{}{0pt}%
\pgfpathmoveto{\pgfqpoint{2.410872in}{2.508083in}}%
\pgfpathlineto{\pgfqpoint{2.674254in}{2.688576in}}%
\pgfpathlineto{\pgfqpoint{2.410872in}{2.508083in}}%
\pgfpathclose%
\pgfusepath{fill}%
\end{pgfscope}%
\begin{pgfscope}%
\pgfpathrectangle{\pgfqpoint{0.017500in}{0.017500in}}{\pgfqpoint{6.218282in}{3.580372in}} %
\pgfusepath{clip}%
\pgfsetbuttcap%
\pgfsetroundjoin%
\definecolor{currentfill}{rgb}{0.810616,0.268797,0.235428}%
\pgfsetfillcolor{currentfill}%
\pgfsetlinewidth{0.000000pt}%
\definecolor{currentstroke}{rgb}{0.000000,0.000000,0.000000}%
\pgfsetstrokecolor{currentstroke}%
\pgfsetdash{}{0pt}%
\pgfpathmoveto{\pgfqpoint{2.410872in}{2.508083in}}%
\pgfpathlineto{\pgfqpoint{2.674254in}{2.688576in}}%
\pgfpathlineto{\pgfqpoint{2.410872in}{2.508083in}}%
\pgfpathclose%
\pgfusepath{fill}%
\end{pgfscope}%
\begin{pgfscope}%
\pgfpathrectangle{\pgfqpoint{0.017500in}{0.017500in}}{\pgfqpoint{6.218282in}{3.580372in}} %
\pgfusepath{clip}%
\pgfsetbuttcap%
\pgfsetroundjoin%
\definecolor{currentfill}{rgb}{0.333490,0.446265,0.874452}%
\pgfsetfillcolor{currentfill}%
\pgfsetlinewidth{0.000000pt}%
\definecolor{currentstroke}{rgb}{0.000000,0.000000,0.000000}%
\pgfsetstrokecolor{currentstroke}%
\pgfsetdash{}{0pt}%
\pgfpathmoveto{\pgfqpoint{1.914830in}{0.967594in}}%
\pgfpathlineto{\pgfqpoint{2.349853in}{0.872712in}}%
\pgfpathlineto{\pgfqpoint{2.590432in}{2.210709in}}%
\pgfpathlineto{\pgfqpoint{2.143341in}{2.291288in}}%
\pgfpathlineto{\pgfqpoint{1.914830in}{0.967594in}}%
\pgfpathclose%
\pgfusepath{fill}%
\end{pgfscope}%
\begin{pgfscope}%
\pgfpathrectangle{\pgfqpoint{0.017500in}{0.017500in}}{\pgfqpoint{6.218282in}{3.580372in}} %
\pgfusepath{clip}%
\pgfsetbuttcap%
\pgfsetroundjoin%
\definecolor{currentfill}{rgb}{0.333490,0.446265,0.874452}%
\pgfsetfillcolor{currentfill}%
\pgfsetlinewidth{0.000000pt}%
\definecolor{currentstroke}{rgb}{0.000000,0.000000,0.000000}%
\pgfsetstrokecolor{currentstroke}%
\pgfsetdash{}{0pt}%
\pgfpathmoveto{\pgfqpoint{1.914830in}{0.967594in}}%
\pgfpathlineto{\pgfqpoint{2.349853in}{0.872712in}}%
\pgfpathlineto{\pgfqpoint{2.590432in}{2.210709in}}%
\pgfpathlineto{\pgfqpoint{2.143341in}{2.291288in}}%
\pgfpathlineto{\pgfqpoint{1.914830in}{0.967594in}}%
\pgfpathclose%
\pgfusepath{fill}%
\end{pgfscope}%
\begin{pgfscope}%
\pgfpathrectangle{\pgfqpoint{0.017500in}{0.017500in}}{\pgfqpoint{6.218282in}{3.580372in}} %
\pgfusepath{clip}%
\pgfsetbuttcap%
\pgfsetroundjoin%
\definecolor{currentfill}{rgb}{0.333490,0.446265,0.874452}%
\pgfsetfillcolor{currentfill}%
\pgfsetlinewidth{0.000000pt}%
\definecolor{currentstroke}{rgb}{0.000000,0.000000,0.000000}%
\pgfsetstrokecolor{currentstroke}%
\pgfsetdash{}{0pt}%
\pgfpathmoveto{\pgfqpoint{1.914830in}{0.967594in}}%
\pgfpathlineto{\pgfqpoint{2.349853in}{0.872712in}}%
\pgfpathlineto{\pgfqpoint{2.590432in}{2.210709in}}%
\pgfpathlineto{\pgfqpoint{2.143341in}{2.291288in}}%
\pgfpathlineto{\pgfqpoint{1.914830in}{0.967594in}}%
\pgfpathclose%
\pgfusepath{fill}%
\end{pgfscope}%
\begin{pgfscope}%
\pgfpathrectangle{\pgfqpoint{0.017500in}{0.017500in}}{\pgfqpoint{6.218282in}{3.580372in}} %
\pgfusepath{clip}%
\pgfsetbuttcap%
\pgfsetroundjoin%
\definecolor{currentfill}{rgb}{0.333490,0.446265,0.874452}%
\pgfsetfillcolor{currentfill}%
\pgfsetlinewidth{0.000000pt}%
\definecolor{currentstroke}{rgb}{0.000000,0.000000,0.000000}%
\pgfsetstrokecolor{currentstroke}%
\pgfsetdash{}{0pt}%
\pgfpathmoveto{\pgfqpoint{1.914830in}{0.967594in}}%
\pgfpathlineto{\pgfqpoint{2.349853in}{0.872712in}}%
\pgfpathlineto{\pgfqpoint{2.590432in}{2.210709in}}%
\pgfpathlineto{\pgfqpoint{2.143341in}{2.291288in}}%
\pgfpathlineto{\pgfqpoint{1.914830in}{0.967594in}}%
\pgfpathclose%
\pgfusepath{fill}%
\end{pgfscope}%
\begin{pgfscope}%
\pgfpathrectangle{\pgfqpoint{0.017500in}{0.017500in}}{\pgfqpoint{6.218282in}{3.580372in}} %
\pgfusepath{clip}%
\pgfsetbuttcap%
\pgfsetroundjoin%
\definecolor{currentfill}{rgb}{0.333490,0.446265,0.874452}%
\pgfsetfillcolor{currentfill}%
\pgfsetlinewidth{0.000000pt}%
\definecolor{currentstroke}{rgb}{0.000000,0.000000,0.000000}%
\pgfsetstrokecolor{currentstroke}%
\pgfsetdash{}{0pt}%
\pgfpathmoveto{\pgfqpoint{1.914830in}{0.967594in}}%
\pgfpathlineto{\pgfqpoint{2.349853in}{0.872712in}}%
\pgfpathlineto{\pgfqpoint{2.590432in}{2.210709in}}%
\pgfpathlineto{\pgfqpoint{2.143341in}{2.291288in}}%
\pgfpathlineto{\pgfqpoint{1.914830in}{0.967594in}}%
\pgfpathclose%
\pgfusepath{fill}%
\end{pgfscope}%
\begin{pgfscope}%
\pgfpathrectangle{\pgfqpoint{0.017500in}{0.017500in}}{\pgfqpoint{6.218282in}{3.580372in}} %
\pgfusepath{clip}%
\pgfsetbuttcap%
\pgfsetroundjoin%
\definecolor{currentfill}{rgb}{0.333490,0.446265,0.874452}%
\pgfsetfillcolor{currentfill}%
\pgfsetlinewidth{0.000000pt}%
\definecolor{currentstroke}{rgb}{0.000000,0.000000,0.000000}%
\pgfsetstrokecolor{currentstroke}%
\pgfsetdash{}{0pt}%
\pgfpathmoveto{\pgfqpoint{1.914830in}{0.967594in}}%
\pgfpathlineto{\pgfqpoint{2.349853in}{0.872712in}}%
\pgfpathlineto{\pgfqpoint{2.590432in}{2.210709in}}%
\pgfpathlineto{\pgfqpoint{2.143341in}{2.291288in}}%
\pgfpathlineto{\pgfqpoint{1.914830in}{0.967594in}}%
\pgfpathclose%
\pgfusepath{fill}%
\end{pgfscope}%
\begin{pgfscope}%
\pgfpathrectangle{\pgfqpoint{0.017500in}{0.017500in}}{\pgfqpoint{6.218282in}{3.580372in}} %
\pgfusepath{clip}%
\pgfsetbuttcap%
\pgfsetroundjoin%
\definecolor{currentfill}{rgb}{0.804965,0.851666,0.926165}%
\pgfsetfillcolor{currentfill}%
\pgfsetlinewidth{0.000000pt}%
\definecolor{currentstroke}{rgb}{0.000000,0.000000,0.000000}%
\pgfsetstrokecolor{currentstroke}%
\pgfsetdash{}{0pt}%
\pgfpathmoveto{\pgfqpoint{4.511410in}{2.753953in}}%
\pgfpathlineto{\pgfqpoint{4.964811in}{2.684961in}}%
\pgfpathlineto{\pgfqpoint{3.688351in}{0.641693in}}%
\pgfpathlineto{\pgfqpoint{3.236099in}{0.723497in}}%
\pgfpathlineto{\pgfqpoint{4.511410in}{2.753953in}}%
\pgfpathclose%
\pgfusepath{fill}%
\end{pgfscope}%
\begin{pgfscope}%
\pgfpathrectangle{\pgfqpoint{0.017500in}{0.017500in}}{\pgfqpoint{6.218282in}{3.580372in}} %
\pgfusepath{clip}%
\pgfsetbuttcap%
\pgfsetroundjoin%
\definecolor{currentfill}{rgb}{0.804965,0.851666,0.926165}%
\pgfsetfillcolor{currentfill}%
\pgfsetlinewidth{0.000000pt}%
\definecolor{currentstroke}{rgb}{0.000000,0.000000,0.000000}%
\pgfsetstrokecolor{currentstroke}%
\pgfsetdash{}{0pt}%
\pgfpathmoveto{\pgfqpoint{4.511410in}{2.753953in}}%
\pgfpathlineto{\pgfqpoint{4.964811in}{2.684961in}}%
\pgfpathlineto{\pgfqpoint{3.688351in}{0.641693in}}%
\pgfpathlineto{\pgfqpoint{3.236099in}{0.723497in}}%
\pgfpathlineto{\pgfqpoint{4.511410in}{2.753953in}}%
\pgfpathclose%
\pgfusepath{fill}%
\end{pgfscope}%
\begin{pgfscope}%
\pgfpathrectangle{\pgfqpoint{0.017500in}{0.017500in}}{\pgfqpoint{6.218282in}{3.580372in}} %
\pgfusepath{clip}%
\pgfsetbuttcap%
\pgfsetroundjoin%
\definecolor{currentfill}{rgb}{0.804965,0.851666,0.926165}%
\pgfsetfillcolor{currentfill}%
\pgfsetlinewidth{0.000000pt}%
\definecolor{currentstroke}{rgb}{0.000000,0.000000,0.000000}%
\pgfsetstrokecolor{currentstroke}%
\pgfsetdash{}{0pt}%
\pgfpathmoveto{\pgfqpoint{4.511410in}{2.753953in}}%
\pgfpathlineto{\pgfqpoint{4.964811in}{2.684961in}}%
\pgfpathlineto{\pgfqpoint{3.688351in}{0.641693in}}%
\pgfpathlineto{\pgfqpoint{3.236099in}{0.723497in}}%
\pgfpathlineto{\pgfqpoint{4.511410in}{2.753953in}}%
\pgfpathclose%
\pgfusepath{fill}%
\end{pgfscope}%
\begin{pgfscope}%
\pgfpathrectangle{\pgfqpoint{0.017500in}{0.017500in}}{\pgfqpoint{6.218282in}{3.580372in}} %
\pgfusepath{clip}%
\pgfsetbuttcap%
\pgfsetroundjoin%
\definecolor{currentfill}{rgb}{0.804965,0.851666,0.926165}%
\pgfsetfillcolor{currentfill}%
\pgfsetlinewidth{0.000000pt}%
\definecolor{currentstroke}{rgb}{0.000000,0.000000,0.000000}%
\pgfsetstrokecolor{currentstroke}%
\pgfsetdash{}{0pt}%
\pgfpathmoveto{\pgfqpoint{4.511410in}{2.753953in}}%
\pgfpathlineto{\pgfqpoint{4.964811in}{2.684961in}}%
\pgfpathlineto{\pgfqpoint{3.688351in}{0.641693in}}%
\pgfpathlineto{\pgfqpoint{3.236099in}{0.723497in}}%
\pgfpathlineto{\pgfqpoint{4.511410in}{2.753953in}}%
\pgfpathclose%
\pgfusepath{fill}%
\end{pgfscope}%
\begin{pgfscope}%
\pgfpathrectangle{\pgfqpoint{0.017500in}{0.017500in}}{\pgfqpoint{6.218282in}{3.580372in}} %
\pgfusepath{clip}%
\pgfsetbuttcap%
\pgfsetroundjoin%
\definecolor{currentfill}{rgb}{0.804965,0.851666,0.926165}%
\pgfsetfillcolor{currentfill}%
\pgfsetlinewidth{0.000000pt}%
\definecolor{currentstroke}{rgb}{0.000000,0.000000,0.000000}%
\pgfsetstrokecolor{currentstroke}%
\pgfsetdash{}{0pt}%
\pgfpathmoveto{\pgfqpoint{4.511410in}{2.753953in}}%
\pgfpathlineto{\pgfqpoint{4.964811in}{2.684961in}}%
\pgfpathlineto{\pgfqpoint{3.688351in}{0.641693in}}%
\pgfpathlineto{\pgfqpoint{3.236099in}{0.723497in}}%
\pgfpathlineto{\pgfqpoint{4.511410in}{2.753953in}}%
\pgfpathclose%
\pgfusepath{fill}%
\end{pgfscope}%
\begin{pgfscope}%
\pgfpathrectangle{\pgfqpoint{0.017500in}{0.017500in}}{\pgfqpoint{6.218282in}{3.580372in}} %
\pgfusepath{clip}%
\pgfsetbuttcap%
\pgfsetroundjoin%
\definecolor{currentfill}{rgb}{0.723315,0.068898,0.162989}%
\pgfsetfillcolor{currentfill}%
\pgfsetlinewidth{0.000000pt}%
\definecolor{currentstroke}{rgb}{0.000000,0.000000,0.000000}%
\pgfsetstrokecolor{currentstroke}%
\pgfsetdash{}{0pt}%
\pgfpathmoveto{\pgfqpoint{4.267628in}{2.613520in}}%
\pgfpathlineto{\pgfqpoint{4.724651in}{2.549289in}}%
\pgfpathlineto{\pgfqpoint{4.964811in}{2.684961in}}%
\pgfpathlineto{\pgfqpoint{4.511410in}{2.753953in}}%
\pgfpathlineto{\pgfqpoint{4.267628in}{2.613520in}}%
\pgfpathclose%
\pgfusepath{fill}%
\end{pgfscope}%
\begin{pgfscope}%
\pgfpathrectangle{\pgfqpoint{0.017500in}{0.017500in}}{\pgfqpoint{6.218282in}{3.580372in}} %
\pgfusepath{clip}%
\pgfsetbuttcap%
\pgfsetroundjoin%
\definecolor{currentfill}{rgb}{0.723315,0.068898,0.162989}%
\pgfsetfillcolor{currentfill}%
\pgfsetlinewidth{0.000000pt}%
\definecolor{currentstroke}{rgb}{0.000000,0.000000,0.000000}%
\pgfsetstrokecolor{currentstroke}%
\pgfsetdash{}{0pt}%
\pgfpathmoveto{\pgfqpoint{4.267628in}{2.613520in}}%
\pgfpathlineto{\pgfqpoint{4.724651in}{2.549289in}}%
\pgfpathlineto{\pgfqpoint{4.964811in}{2.684961in}}%
\pgfpathlineto{\pgfqpoint{4.511410in}{2.753953in}}%
\pgfpathlineto{\pgfqpoint{4.267628in}{2.613520in}}%
\pgfpathclose%
\pgfusepath{fill}%
\end{pgfscope}%
\begin{pgfscope}%
\pgfpathrectangle{\pgfqpoint{0.017500in}{0.017500in}}{\pgfqpoint{6.218282in}{3.580372in}} %
\pgfusepath{clip}%
\pgfsetbuttcap%
\pgfsetroundjoin%
\definecolor{currentfill}{rgb}{0.723315,0.068898,0.162989}%
\pgfsetfillcolor{currentfill}%
\pgfsetlinewidth{0.000000pt}%
\definecolor{currentstroke}{rgb}{0.000000,0.000000,0.000000}%
\pgfsetstrokecolor{currentstroke}%
\pgfsetdash{}{0pt}%
\pgfpathmoveto{\pgfqpoint{4.267628in}{2.613520in}}%
\pgfpathlineto{\pgfqpoint{4.724651in}{2.549289in}}%
\pgfpathlineto{\pgfqpoint{4.964811in}{2.684961in}}%
\pgfpathlineto{\pgfqpoint{4.511410in}{2.753953in}}%
\pgfpathlineto{\pgfqpoint{4.267628in}{2.613520in}}%
\pgfpathclose%
\pgfusepath{fill}%
\end{pgfscope}%
\begin{pgfscope}%
\pgfpathrectangle{\pgfqpoint{0.017500in}{0.017500in}}{\pgfqpoint{6.218282in}{3.580372in}} %
\pgfusepath{clip}%
\pgfsetbuttcap%
\pgfsetroundjoin%
\definecolor{currentfill}{rgb}{0.723315,0.068898,0.162989}%
\pgfsetfillcolor{currentfill}%
\pgfsetlinewidth{0.000000pt}%
\definecolor{currentstroke}{rgb}{0.000000,0.000000,0.000000}%
\pgfsetstrokecolor{currentstroke}%
\pgfsetdash{}{0pt}%
\pgfpathmoveto{\pgfqpoint{4.267628in}{2.613520in}}%
\pgfpathlineto{\pgfqpoint{4.724651in}{2.549289in}}%
\pgfpathlineto{\pgfqpoint{4.964811in}{2.684961in}}%
\pgfpathlineto{\pgfqpoint{4.511410in}{2.753953in}}%
\pgfpathlineto{\pgfqpoint{4.267628in}{2.613520in}}%
\pgfpathclose%
\pgfusepath{fill}%
\end{pgfscope}%
\begin{pgfscope}%
\pgfpathrectangle{\pgfqpoint{0.017500in}{0.017500in}}{\pgfqpoint{6.218282in}{3.580372in}} %
\pgfusepath{clip}%
\pgfsetbuttcap%
\pgfsetroundjoin%
\definecolor{currentfill}{rgb}{0.723315,0.068898,0.162989}%
\pgfsetfillcolor{currentfill}%
\pgfsetlinewidth{0.000000pt}%
\definecolor{currentstroke}{rgb}{0.000000,0.000000,0.000000}%
\pgfsetstrokecolor{currentstroke}%
\pgfsetdash{}{0pt}%
\pgfpathmoveto{\pgfqpoint{4.267628in}{2.613520in}}%
\pgfpathlineto{\pgfqpoint{4.724651in}{2.549289in}}%
\pgfpathlineto{\pgfqpoint{4.964811in}{2.684961in}}%
\pgfpathlineto{\pgfqpoint{4.511410in}{2.753953in}}%
\pgfpathlineto{\pgfqpoint{4.267628in}{2.613520in}}%
\pgfpathclose%
\pgfusepath{fill}%
\end{pgfscope}%
\begin{pgfscope}%
\pgfpathrectangle{\pgfqpoint{0.017500in}{0.017500in}}{\pgfqpoint{6.218282in}{3.580372in}} %
\pgfusepath{clip}%
\pgfsetbuttcap%
\pgfsetroundjoin%
\definecolor{currentfill}{rgb}{0.723315,0.068898,0.162989}%
\pgfsetfillcolor{currentfill}%
\pgfsetlinewidth{0.000000pt}%
\definecolor{currentstroke}{rgb}{0.000000,0.000000,0.000000}%
\pgfsetstrokecolor{currentstroke}%
\pgfsetdash{}{0pt}%
\pgfpathmoveto{\pgfqpoint{4.267628in}{2.613520in}}%
\pgfpathlineto{\pgfqpoint{4.724651in}{2.549289in}}%
\pgfpathlineto{\pgfqpoint{4.964811in}{2.684961in}}%
\pgfpathlineto{\pgfqpoint{4.511410in}{2.753953in}}%
\pgfpathlineto{\pgfqpoint{4.267628in}{2.613520in}}%
\pgfpathclose%
\pgfusepath{fill}%
\end{pgfscope}%
\begin{pgfscope}%
\pgfpathrectangle{\pgfqpoint{0.017500in}{0.017500in}}{\pgfqpoint{6.218282in}{3.580372in}} %
\pgfusepath{clip}%
\pgfsetbuttcap%
\pgfsetroundjoin%
\definecolor{currentfill}{rgb}{0.805723,0.259813,0.230562}%
\pgfsetfillcolor{currentfill}%
\pgfsetlinewidth{0.000000pt}%
\definecolor{currentstroke}{rgb}{0.000000,0.000000,0.000000}%
\pgfsetstrokecolor{currentstroke}%
\pgfsetdash{}{0pt}%
\pgfpathmoveto{\pgfqpoint{2.410872in}{2.508083in}}%
\pgfpathlineto{\pgfqpoint{2.855513in}{2.451418in}}%
\pgfpathlineto{\pgfqpoint{3.116513in}{2.603812in}}%
\pgfpathlineto{\pgfqpoint{2.674254in}{2.688576in}}%
\pgfpathlineto{\pgfqpoint{2.410872in}{2.508083in}}%
\pgfpathclose%
\pgfusepath{fill}%
\end{pgfscope}%
\begin{pgfscope}%
\pgfpathrectangle{\pgfqpoint{0.017500in}{0.017500in}}{\pgfqpoint{6.218282in}{3.580372in}} %
\pgfusepath{clip}%
\pgfsetbuttcap%
\pgfsetroundjoin%
\definecolor{currentfill}{rgb}{0.805723,0.259813,0.230562}%
\pgfsetfillcolor{currentfill}%
\pgfsetlinewidth{0.000000pt}%
\definecolor{currentstroke}{rgb}{0.000000,0.000000,0.000000}%
\pgfsetstrokecolor{currentstroke}%
\pgfsetdash{}{0pt}%
\pgfpathmoveto{\pgfqpoint{2.410872in}{2.508083in}}%
\pgfpathlineto{\pgfqpoint{2.855513in}{2.451418in}}%
\pgfpathlineto{\pgfqpoint{3.116513in}{2.603812in}}%
\pgfpathlineto{\pgfqpoint{2.674254in}{2.688576in}}%
\pgfpathlineto{\pgfqpoint{2.410872in}{2.508083in}}%
\pgfpathclose%
\pgfusepath{fill}%
\end{pgfscope}%
\begin{pgfscope}%
\pgfpathrectangle{\pgfqpoint{0.017500in}{0.017500in}}{\pgfqpoint{6.218282in}{3.580372in}} %
\pgfusepath{clip}%
\pgfsetbuttcap%
\pgfsetroundjoin%
\definecolor{currentfill}{rgb}{0.805723,0.259813,0.230562}%
\pgfsetfillcolor{currentfill}%
\pgfsetlinewidth{0.000000pt}%
\definecolor{currentstroke}{rgb}{0.000000,0.000000,0.000000}%
\pgfsetstrokecolor{currentstroke}%
\pgfsetdash{}{0pt}%
\pgfpathmoveto{\pgfqpoint{2.410872in}{2.508083in}}%
\pgfpathlineto{\pgfqpoint{2.855513in}{2.451418in}}%
\pgfpathlineto{\pgfqpoint{3.116513in}{2.603812in}}%
\pgfpathlineto{\pgfqpoint{2.674254in}{2.688576in}}%
\pgfpathlineto{\pgfqpoint{2.410872in}{2.508083in}}%
\pgfpathclose%
\pgfusepath{fill}%
\end{pgfscope}%
\begin{pgfscope}%
\pgfpathrectangle{\pgfqpoint{0.017500in}{0.017500in}}{\pgfqpoint{6.218282in}{3.580372in}} %
\pgfusepath{clip}%
\pgfsetbuttcap%
\pgfsetroundjoin%
\definecolor{currentfill}{rgb}{0.805723,0.259813,0.230562}%
\pgfsetfillcolor{currentfill}%
\pgfsetlinewidth{0.000000pt}%
\definecolor{currentstroke}{rgb}{0.000000,0.000000,0.000000}%
\pgfsetstrokecolor{currentstroke}%
\pgfsetdash{}{0pt}%
\pgfpathmoveto{\pgfqpoint{2.410872in}{2.508083in}}%
\pgfpathlineto{\pgfqpoint{2.855513in}{2.451418in}}%
\pgfpathlineto{\pgfqpoint{3.116513in}{2.603812in}}%
\pgfpathlineto{\pgfqpoint{2.674254in}{2.688576in}}%
\pgfpathlineto{\pgfqpoint{2.410872in}{2.508083in}}%
\pgfpathclose%
\pgfusepath{fill}%
\end{pgfscope}%
\begin{pgfscope}%
\pgfpathrectangle{\pgfqpoint{0.017500in}{0.017500in}}{\pgfqpoint{6.218282in}{3.580372in}} %
\pgfusepath{clip}%
\pgfsetbuttcap%
\pgfsetroundjoin%
\definecolor{currentfill}{rgb}{0.805723,0.259813,0.230562}%
\pgfsetfillcolor{currentfill}%
\pgfsetlinewidth{0.000000pt}%
\definecolor{currentstroke}{rgb}{0.000000,0.000000,0.000000}%
\pgfsetstrokecolor{currentstroke}%
\pgfsetdash{}{0pt}%
\pgfpathmoveto{\pgfqpoint{2.410872in}{2.508083in}}%
\pgfpathlineto{\pgfqpoint{2.855513in}{2.451418in}}%
\pgfpathlineto{\pgfqpoint{3.116513in}{2.603812in}}%
\pgfpathlineto{\pgfqpoint{2.674254in}{2.688576in}}%
\pgfpathlineto{\pgfqpoint{2.410872in}{2.508083in}}%
\pgfpathclose%
\pgfusepath{fill}%
\end{pgfscope}%
\begin{pgfscope}%
\pgfpathrectangle{\pgfqpoint{0.017500in}{0.017500in}}{\pgfqpoint{6.218282in}{3.580372in}} %
\pgfusepath{clip}%
\pgfsetbuttcap%
\pgfsetroundjoin%
\definecolor{currentfill}{rgb}{0.805723,0.259813,0.230562}%
\pgfsetfillcolor{currentfill}%
\pgfsetlinewidth{0.000000pt}%
\definecolor{currentstroke}{rgb}{0.000000,0.000000,0.000000}%
\pgfsetstrokecolor{currentstroke}%
\pgfsetdash{}{0pt}%
\pgfpathmoveto{\pgfqpoint{2.410872in}{2.508083in}}%
\pgfpathlineto{\pgfqpoint{2.855513in}{2.451418in}}%
\pgfpathlineto{\pgfqpoint{3.116513in}{2.603812in}}%
\pgfpathlineto{\pgfqpoint{2.674254in}{2.688576in}}%
\pgfpathlineto{\pgfqpoint{2.410872in}{2.508083in}}%
\pgfpathclose%
\pgfusepath{fill}%
\end{pgfscope}%
\begin{pgfscope}%
\pgfpathrectangle{\pgfqpoint{0.017500in}{0.017500in}}{\pgfqpoint{6.218282in}{3.580372in}} %
\pgfusepath{clip}%
\pgfsetbuttcap%
\pgfsetroundjoin%
\definecolor{currentfill}{rgb}{0.229806,0.298718,0.753683}%
\pgfsetfillcolor{currentfill}%
\pgfsetlinewidth{0.000000pt}%
\definecolor{currentstroke}{rgb}{0.000000,0.000000,0.000000}%
\pgfsetstrokecolor{currentstroke}%
\pgfsetdash{}{0pt}%
\pgfpathmoveto{\pgfqpoint{2.349853in}{0.872712in}}%
\pgfpathlineto{\pgfqpoint{2.590432in}{2.210709in}}%
\pgfpathlineto{\pgfqpoint{2.349853in}{0.872712in}}%
\pgfpathclose%
\pgfusepath{fill}%
\end{pgfscope}%
\begin{pgfscope}%
\pgfpathrectangle{\pgfqpoint{0.017500in}{0.017500in}}{\pgfqpoint{6.218282in}{3.580372in}} %
\pgfusepath{clip}%
\pgfsetbuttcap%
\pgfsetroundjoin%
\definecolor{currentfill}{rgb}{0.229806,0.298718,0.753683}%
\pgfsetfillcolor{currentfill}%
\pgfsetlinewidth{0.000000pt}%
\definecolor{currentstroke}{rgb}{0.000000,0.000000,0.000000}%
\pgfsetstrokecolor{currentstroke}%
\pgfsetdash{}{0pt}%
\pgfpathmoveto{\pgfqpoint{2.349853in}{0.872712in}}%
\pgfpathlineto{\pgfqpoint{2.590432in}{2.210709in}}%
\pgfpathlineto{\pgfqpoint{2.349853in}{0.872712in}}%
\pgfpathclose%
\pgfusepath{fill}%
\end{pgfscope}%
\begin{pgfscope}%
\pgfpathrectangle{\pgfqpoint{0.017500in}{0.017500in}}{\pgfqpoint{6.218282in}{3.580372in}} %
\pgfusepath{clip}%
\pgfsetbuttcap%
\pgfsetroundjoin%
\definecolor{currentfill}{rgb}{0.229806,0.298718,0.753683}%
\pgfsetfillcolor{currentfill}%
\pgfsetlinewidth{0.000000pt}%
\definecolor{currentstroke}{rgb}{0.000000,0.000000,0.000000}%
\pgfsetstrokecolor{currentstroke}%
\pgfsetdash{}{0pt}%
\pgfpathmoveto{\pgfqpoint{2.349853in}{0.872712in}}%
\pgfpathlineto{\pgfqpoint{2.590432in}{2.210709in}}%
\pgfpathlineto{\pgfqpoint{2.349853in}{0.872712in}}%
\pgfpathclose%
\pgfusepath{fill}%
\end{pgfscope}%
\begin{pgfscope}%
\pgfpathrectangle{\pgfqpoint{0.017500in}{0.017500in}}{\pgfqpoint{6.218282in}{3.580372in}} %
\pgfusepath{clip}%
\pgfsetbuttcap%
\pgfsetroundjoin%
\definecolor{currentfill}{rgb}{0.229806,0.298718,0.753683}%
\pgfsetfillcolor{currentfill}%
\pgfsetlinewidth{0.000000pt}%
\definecolor{currentstroke}{rgb}{0.000000,0.000000,0.000000}%
\pgfsetstrokecolor{currentstroke}%
\pgfsetdash{}{0pt}%
\pgfpathmoveto{\pgfqpoint{2.349853in}{0.872712in}}%
\pgfpathlineto{\pgfqpoint{2.590432in}{2.210709in}}%
\pgfpathlineto{\pgfqpoint{2.349853in}{0.872712in}}%
\pgfpathclose%
\pgfusepath{fill}%
\end{pgfscope}%
\begin{pgfscope}%
\pgfpathrectangle{\pgfqpoint{0.017500in}{0.017500in}}{\pgfqpoint{6.218282in}{3.580372in}} %
\pgfusepath{clip}%
\pgfsetbuttcap%
\pgfsetroundjoin%
\definecolor{currentfill}{rgb}{0.229806,0.298718,0.753683}%
\pgfsetfillcolor{currentfill}%
\pgfsetlinewidth{0.000000pt}%
\definecolor{currentstroke}{rgb}{0.000000,0.000000,0.000000}%
\pgfsetstrokecolor{currentstroke}%
\pgfsetdash{}{0pt}%
\pgfpathmoveto{\pgfqpoint{2.349853in}{0.872712in}}%
\pgfpathlineto{\pgfqpoint{2.590432in}{2.210709in}}%
\pgfpathlineto{\pgfqpoint{2.349853in}{0.872712in}}%
\pgfpathclose%
\pgfusepath{fill}%
\end{pgfscope}%
\begin{pgfscope}%
\pgfpathrectangle{\pgfqpoint{0.017500in}{0.017500in}}{\pgfqpoint{6.218282in}{3.580372in}} %
\pgfusepath{clip}%
\pgfsetbuttcap%
\pgfsetroundjoin%
\definecolor{currentfill}{rgb}{0.229806,0.298718,0.753683}%
\pgfsetfillcolor{currentfill}%
\pgfsetlinewidth{0.000000pt}%
\definecolor{currentstroke}{rgb}{0.000000,0.000000,0.000000}%
\pgfsetstrokecolor{currentstroke}%
\pgfsetdash{}{0pt}%
\pgfpathmoveto{\pgfqpoint{2.349853in}{0.872712in}}%
\pgfpathlineto{\pgfqpoint{2.590432in}{2.210709in}}%
\pgfpathlineto{\pgfqpoint{2.349853in}{0.872712in}}%
\pgfpathclose%
\pgfusepath{fill}%
\end{pgfscope}%
\begin{pgfscope}%
\pgfpathrectangle{\pgfqpoint{0.017500in}{0.017500in}}{\pgfqpoint{6.218282in}{3.580372in}} %
\pgfusepath{clip}%
\pgfsetbuttcap%
\pgfsetroundjoin%
\definecolor{currentfill}{rgb}{0.229806,0.298718,0.753683}%
\pgfsetfillcolor{currentfill}%
\pgfsetlinewidth{0.000000pt}%
\definecolor{currentstroke}{rgb}{0.000000,0.000000,0.000000}%
\pgfsetstrokecolor{currentstroke}%
\pgfsetdash{}{0pt}%
\pgfpathmoveto{\pgfqpoint{2.349853in}{0.872712in}}%
\pgfpathlineto{\pgfqpoint{2.590432in}{2.210709in}}%
\pgfpathlineto{\pgfqpoint{2.349853in}{0.872712in}}%
\pgfpathclose%
\pgfusepath{fill}%
\end{pgfscope}%
\begin{pgfscope}%
\pgfpathrectangle{\pgfqpoint{0.017500in}{0.017500in}}{\pgfqpoint{6.218282in}{3.580372in}} %
\pgfusepath{clip}%
\pgfsetbuttcap%
\pgfsetroundjoin%
\definecolor{currentfill}{rgb}{0.229806,0.298718,0.753683}%
\pgfsetfillcolor{currentfill}%
\pgfsetlinewidth{0.000000pt}%
\definecolor{currentstroke}{rgb}{0.000000,0.000000,0.000000}%
\pgfsetstrokecolor{currentstroke}%
\pgfsetdash{}{0pt}%
\pgfpathmoveto{\pgfqpoint{2.349853in}{0.872712in}}%
\pgfpathlineto{\pgfqpoint{2.590432in}{2.210709in}}%
\pgfpathlineto{\pgfqpoint{2.349853in}{0.872712in}}%
\pgfpathclose%
\pgfusepath{fill}%
\end{pgfscope}%
\begin{pgfscope}%
\pgfpathrectangle{\pgfqpoint{0.017500in}{0.017500in}}{\pgfqpoint{6.218282in}{3.580372in}} %
\pgfusepath{clip}%
\pgfsetbuttcap%
\pgfsetroundjoin%
\definecolor{currentfill}{rgb}{0.229806,0.298718,0.753683}%
\pgfsetfillcolor{currentfill}%
\pgfsetlinewidth{0.000000pt}%
\definecolor{currentstroke}{rgb}{0.000000,0.000000,0.000000}%
\pgfsetstrokecolor{currentstroke}%
\pgfsetdash{}{0pt}%
\pgfpathmoveto{\pgfqpoint{2.349853in}{0.872712in}}%
\pgfpathlineto{\pgfqpoint{2.590432in}{2.210709in}}%
\pgfpathlineto{\pgfqpoint{2.349853in}{0.872712in}}%
\pgfpathclose%
\pgfusepath{fill}%
\end{pgfscope}%
\begin{pgfscope}%
\pgfpathrectangle{\pgfqpoint{0.017500in}{0.017500in}}{\pgfqpoint{6.218282in}{3.580372in}} %
\pgfusepath{clip}%
\pgfsetbuttcap%
\pgfsetroundjoin%
\definecolor{currentfill}{rgb}{0.229806,0.298718,0.753683}%
\pgfsetfillcolor{currentfill}%
\pgfsetlinewidth{0.000000pt}%
\definecolor{currentstroke}{rgb}{0.000000,0.000000,0.000000}%
\pgfsetstrokecolor{currentstroke}%
\pgfsetdash{}{0pt}%
\pgfpathmoveto{\pgfqpoint{2.349853in}{0.872712in}}%
\pgfpathlineto{\pgfqpoint{2.590432in}{2.210709in}}%
\pgfpathlineto{\pgfqpoint{2.349853in}{0.872712in}}%
\pgfpathclose%
\pgfusepath{fill}%
\end{pgfscope}%
\begin{pgfscope}%
\pgfpathrectangle{\pgfqpoint{0.017500in}{0.017500in}}{\pgfqpoint{6.218282in}{3.580372in}} %
\pgfusepath{clip}%
\pgfsetbuttcap%
\pgfsetroundjoin%
\definecolor{currentfill}{rgb}{0.229806,0.298718,0.753683}%
\pgfsetfillcolor{currentfill}%
\pgfsetlinewidth{0.000000pt}%
\definecolor{currentstroke}{rgb}{0.000000,0.000000,0.000000}%
\pgfsetstrokecolor{currentstroke}%
\pgfsetdash{}{0pt}%
\pgfpathmoveto{\pgfqpoint{2.349853in}{0.872712in}}%
\pgfpathlineto{\pgfqpoint{2.590432in}{2.210709in}}%
\pgfpathlineto{\pgfqpoint{2.349853in}{0.872712in}}%
\pgfpathclose%
\pgfusepath{fill}%
\end{pgfscope}%
\begin{pgfscope}%
\pgfpathrectangle{\pgfqpoint{0.017500in}{0.017500in}}{\pgfqpoint{6.218282in}{3.580372in}} %
\pgfusepath{clip}%
\pgfsetbuttcap%
\pgfsetroundjoin%
\definecolor{currentfill}{rgb}{0.229806,0.298718,0.753683}%
\pgfsetfillcolor{currentfill}%
\pgfsetlinewidth{0.000000pt}%
\definecolor{currentstroke}{rgb}{0.000000,0.000000,0.000000}%
\pgfsetstrokecolor{currentstroke}%
\pgfsetdash{}{0pt}%
\pgfpathmoveto{\pgfqpoint{2.349853in}{0.872712in}}%
\pgfpathlineto{\pgfqpoint{2.590432in}{2.210709in}}%
\pgfpathlineto{\pgfqpoint{2.349853in}{0.872712in}}%
\pgfpathclose%
\pgfusepath{fill}%
\end{pgfscope}%
\begin{pgfscope}%
\pgfpathrectangle{\pgfqpoint{0.017500in}{0.017500in}}{\pgfqpoint{6.218282in}{3.580372in}} %
\pgfusepath{clip}%
\pgfsetbuttcap%
\pgfsetroundjoin%
\definecolor{currentfill}{rgb}{0.229806,0.298718,0.753683}%
\pgfsetfillcolor{currentfill}%
\pgfsetlinewidth{0.000000pt}%
\definecolor{currentstroke}{rgb}{0.000000,0.000000,0.000000}%
\pgfsetstrokecolor{currentstroke}%
\pgfsetdash{}{0pt}%
\pgfpathmoveto{\pgfqpoint{2.349853in}{0.872712in}}%
\pgfpathlineto{\pgfqpoint{2.590432in}{2.210709in}}%
\pgfpathlineto{\pgfqpoint{2.349853in}{0.872712in}}%
\pgfpathclose%
\pgfusepath{fill}%
\end{pgfscope}%
\begin{pgfscope}%
\pgfpathrectangle{\pgfqpoint{0.017500in}{0.017500in}}{\pgfqpoint{6.218282in}{3.580372in}} %
\pgfusepath{clip}%
\pgfsetbuttcap%
\pgfsetroundjoin%
\definecolor{currentfill}{rgb}{0.229806,0.298718,0.753683}%
\pgfsetfillcolor{currentfill}%
\pgfsetlinewidth{0.000000pt}%
\definecolor{currentstroke}{rgb}{0.000000,0.000000,0.000000}%
\pgfsetstrokecolor{currentstroke}%
\pgfsetdash{}{0pt}%
\pgfpathmoveto{\pgfqpoint{2.349853in}{0.872712in}}%
\pgfpathlineto{\pgfqpoint{2.590432in}{2.210709in}}%
\pgfpathlineto{\pgfqpoint{2.349853in}{0.872712in}}%
\pgfpathclose%
\pgfusepath{fill}%
\end{pgfscope}%
\begin{pgfscope}%
\pgfpathrectangle{\pgfqpoint{0.017500in}{0.017500in}}{\pgfqpoint{6.218282in}{3.580372in}} %
\pgfusepath{clip}%
\pgfsetbuttcap%
\pgfsetroundjoin%
\definecolor{currentfill}{rgb}{0.229806,0.298718,0.753683}%
\pgfsetfillcolor{currentfill}%
\pgfsetlinewidth{0.000000pt}%
\definecolor{currentstroke}{rgb}{0.000000,0.000000,0.000000}%
\pgfsetstrokecolor{currentstroke}%
\pgfsetdash{}{0pt}%
\pgfpathmoveto{\pgfqpoint{2.349853in}{0.872712in}}%
\pgfpathlineto{\pgfqpoint{2.590432in}{2.210709in}}%
\pgfpathlineto{\pgfqpoint{2.349853in}{0.872712in}}%
\pgfpathclose%
\pgfusepath{fill}%
\end{pgfscope}%
\begin{pgfscope}%
\pgfpathrectangle{\pgfqpoint{0.017500in}{0.017500in}}{\pgfqpoint{6.218282in}{3.580372in}} %
\pgfusepath{clip}%
\pgfsetbuttcap%
\pgfsetroundjoin%
\definecolor{currentfill}{rgb}{0.229806,0.298718,0.753683}%
\pgfsetfillcolor{currentfill}%
\pgfsetlinewidth{0.000000pt}%
\definecolor{currentstroke}{rgb}{0.000000,0.000000,0.000000}%
\pgfsetstrokecolor{currentstroke}%
\pgfsetdash{}{0pt}%
\pgfpathmoveto{\pgfqpoint{2.349853in}{0.872712in}}%
\pgfpathlineto{\pgfqpoint{2.590432in}{2.210709in}}%
\pgfpathlineto{\pgfqpoint{2.349853in}{0.872712in}}%
\pgfpathclose%
\pgfusepath{fill}%
\end{pgfscope}%
\begin{pgfscope}%
\pgfpathrectangle{\pgfqpoint{0.017500in}{0.017500in}}{\pgfqpoint{6.218282in}{3.580372in}} %
\pgfusepath{clip}%
\pgfsetbuttcap%
\pgfsetroundjoin%
\definecolor{currentfill}{rgb}{0.229806,0.298718,0.753683}%
\pgfsetfillcolor{currentfill}%
\pgfsetlinewidth{0.000000pt}%
\definecolor{currentstroke}{rgb}{0.000000,0.000000,0.000000}%
\pgfsetstrokecolor{currentstroke}%
\pgfsetdash{}{0pt}%
\pgfpathmoveto{\pgfqpoint{2.349853in}{0.872712in}}%
\pgfpathlineto{\pgfqpoint{2.590432in}{2.210709in}}%
\pgfpathlineto{\pgfqpoint{2.349853in}{0.872712in}}%
\pgfpathclose%
\pgfusepath{fill}%
\end{pgfscope}%
\begin{pgfscope}%
\pgfpathrectangle{\pgfqpoint{0.017500in}{0.017500in}}{\pgfqpoint{6.218282in}{3.580372in}} %
\pgfusepath{clip}%
\pgfsetbuttcap%
\pgfsetroundjoin%
\definecolor{currentfill}{rgb}{0.229806,0.298718,0.753683}%
\pgfsetfillcolor{currentfill}%
\pgfsetlinewidth{0.000000pt}%
\definecolor{currentstroke}{rgb}{0.000000,0.000000,0.000000}%
\pgfsetstrokecolor{currentstroke}%
\pgfsetdash{}{0pt}%
\pgfpathmoveto{\pgfqpoint{2.349853in}{0.872712in}}%
\pgfpathlineto{\pgfqpoint{2.590432in}{2.210709in}}%
\pgfpathlineto{\pgfqpoint{2.349853in}{0.872712in}}%
\pgfpathclose%
\pgfusepath{fill}%
\end{pgfscope}%
\begin{pgfscope}%
\pgfpathrectangle{\pgfqpoint{0.017500in}{0.017500in}}{\pgfqpoint{6.218282in}{3.580372in}} %
\pgfusepath{clip}%
\pgfsetbuttcap%
\pgfsetroundjoin%
\definecolor{currentfill}{rgb}{0.229806,0.298718,0.753683}%
\pgfsetfillcolor{currentfill}%
\pgfsetlinewidth{0.000000pt}%
\definecolor{currentstroke}{rgb}{0.000000,0.000000,0.000000}%
\pgfsetstrokecolor{currentstroke}%
\pgfsetdash{}{0pt}%
\pgfpathmoveto{\pgfqpoint{2.349853in}{0.872712in}}%
\pgfpathlineto{\pgfqpoint{2.590432in}{2.210709in}}%
\pgfpathlineto{\pgfqpoint{2.349853in}{0.872712in}}%
\pgfpathclose%
\pgfusepath{fill}%
\end{pgfscope}%
\begin{pgfscope}%
\pgfpathrectangle{\pgfqpoint{0.017500in}{0.017500in}}{\pgfqpoint{6.218282in}{3.580372in}} %
\pgfusepath{clip}%
\pgfsetbuttcap%
\pgfsetroundjoin%
\definecolor{currentfill}{rgb}{0.229806,0.298718,0.753683}%
\pgfsetfillcolor{currentfill}%
\pgfsetlinewidth{0.000000pt}%
\definecolor{currentstroke}{rgb}{0.000000,0.000000,0.000000}%
\pgfsetstrokecolor{currentstroke}%
\pgfsetdash{}{0pt}%
\pgfpathmoveto{\pgfqpoint{2.349853in}{0.872712in}}%
\pgfpathlineto{\pgfqpoint{2.590432in}{2.210709in}}%
\pgfpathlineto{\pgfqpoint{2.349853in}{0.872712in}}%
\pgfpathclose%
\pgfusepath{fill}%
\end{pgfscope}%
\begin{pgfscope}%
\pgfpathrectangle{\pgfqpoint{0.017500in}{0.017500in}}{\pgfqpoint{6.218282in}{3.580372in}} %
\pgfusepath{clip}%
\pgfsetbuttcap%
\pgfsetroundjoin%
\definecolor{currentfill}{rgb}{0.229806,0.298718,0.753683}%
\pgfsetfillcolor{currentfill}%
\pgfsetlinewidth{0.000000pt}%
\definecolor{currentstroke}{rgb}{0.000000,0.000000,0.000000}%
\pgfsetstrokecolor{currentstroke}%
\pgfsetdash{}{0pt}%
\pgfpathmoveto{\pgfqpoint{2.349853in}{0.872712in}}%
\pgfpathlineto{\pgfqpoint{2.590432in}{2.210709in}}%
\pgfpathlineto{\pgfqpoint{2.349853in}{0.872712in}}%
\pgfpathclose%
\pgfusepath{fill}%
\end{pgfscope}%
\begin{pgfscope}%
\pgfpathrectangle{\pgfqpoint{0.017500in}{0.017500in}}{\pgfqpoint{6.218282in}{3.580372in}} %
\pgfusepath{clip}%
\pgfsetbuttcap%
\pgfsetroundjoin%
\definecolor{currentfill}{rgb}{0.229806,0.298718,0.753683}%
\pgfsetfillcolor{currentfill}%
\pgfsetlinewidth{0.000000pt}%
\definecolor{currentstroke}{rgb}{0.000000,0.000000,0.000000}%
\pgfsetstrokecolor{currentstroke}%
\pgfsetdash{}{0pt}%
\pgfpathmoveto{\pgfqpoint{2.349853in}{0.872712in}}%
\pgfpathlineto{\pgfqpoint{2.590432in}{2.210709in}}%
\pgfpathlineto{\pgfqpoint{2.349853in}{0.872712in}}%
\pgfpathclose%
\pgfusepath{fill}%
\end{pgfscope}%
\begin{pgfscope}%
\pgfpathrectangle{\pgfqpoint{0.017500in}{0.017500in}}{\pgfqpoint{6.218282in}{3.580372in}} %
\pgfusepath{clip}%
\pgfsetbuttcap%
\pgfsetroundjoin%
\definecolor{currentfill}{rgb}{0.229806,0.298718,0.753683}%
\pgfsetfillcolor{currentfill}%
\pgfsetlinewidth{0.000000pt}%
\definecolor{currentstroke}{rgb}{0.000000,0.000000,0.000000}%
\pgfsetstrokecolor{currentstroke}%
\pgfsetdash{}{0pt}%
\pgfpathmoveto{\pgfqpoint{2.349853in}{0.872712in}}%
\pgfpathlineto{\pgfqpoint{2.590432in}{2.210709in}}%
\pgfpathlineto{\pgfqpoint{2.349853in}{0.872712in}}%
\pgfpathclose%
\pgfusepath{fill}%
\end{pgfscope}%
\begin{pgfscope}%
\pgfpathrectangle{\pgfqpoint{0.017500in}{0.017500in}}{\pgfqpoint{6.218282in}{3.580372in}} %
\pgfusepath{clip}%
\pgfsetbuttcap%
\pgfsetroundjoin%
\definecolor{currentfill}{rgb}{0.229806,0.298718,0.753683}%
\pgfsetfillcolor{currentfill}%
\pgfsetlinewidth{0.000000pt}%
\definecolor{currentstroke}{rgb}{0.000000,0.000000,0.000000}%
\pgfsetstrokecolor{currentstroke}%
\pgfsetdash{}{0pt}%
\pgfpathmoveto{\pgfqpoint{2.349853in}{0.872712in}}%
\pgfpathlineto{\pgfqpoint{2.590432in}{2.210709in}}%
\pgfpathlineto{\pgfqpoint{2.349853in}{0.872712in}}%
\pgfpathclose%
\pgfusepath{fill}%
\end{pgfscope}%
\begin{pgfscope}%
\pgfpathrectangle{\pgfqpoint{0.017500in}{0.017500in}}{\pgfqpoint{6.218282in}{3.580372in}} %
\pgfusepath{clip}%
\pgfsetbuttcap%
\pgfsetroundjoin%
\definecolor{currentfill}{rgb}{0.229806,0.298718,0.753683}%
\pgfsetfillcolor{currentfill}%
\pgfsetlinewidth{0.000000pt}%
\definecolor{currentstroke}{rgb}{0.000000,0.000000,0.000000}%
\pgfsetstrokecolor{currentstroke}%
\pgfsetdash{}{0pt}%
\pgfpathmoveto{\pgfqpoint{2.349853in}{0.872712in}}%
\pgfpathlineto{\pgfqpoint{2.590432in}{2.210709in}}%
\pgfpathlineto{\pgfqpoint{2.349853in}{0.872712in}}%
\pgfpathclose%
\pgfusepath{fill}%
\end{pgfscope}%
\begin{pgfscope}%
\pgfpathrectangle{\pgfqpoint{0.017500in}{0.017500in}}{\pgfqpoint{6.218282in}{3.580372in}} %
\pgfusepath{clip}%
\pgfsetbuttcap%
\pgfsetroundjoin%
\definecolor{currentfill}{rgb}{0.229806,0.298718,0.753683}%
\pgfsetfillcolor{currentfill}%
\pgfsetlinewidth{0.000000pt}%
\definecolor{currentstroke}{rgb}{0.000000,0.000000,0.000000}%
\pgfsetstrokecolor{currentstroke}%
\pgfsetdash{}{0pt}%
\pgfpathmoveto{\pgfqpoint{2.349853in}{0.872712in}}%
\pgfpathlineto{\pgfqpoint{2.590432in}{2.210709in}}%
\pgfpathlineto{\pgfqpoint{2.349853in}{0.872712in}}%
\pgfpathclose%
\pgfusepath{fill}%
\end{pgfscope}%
\begin{pgfscope}%
\pgfpathrectangle{\pgfqpoint{0.017500in}{0.017500in}}{\pgfqpoint{6.218282in}{3.580372in}} %
\pgfusepath{clip}%
\pgfsetbuttcap%
\pgfsetroundjoin%
\definecolor{currentfill}{rgb}{0.229806,0.298718,0.753683}%
\pgfsetfillcolor{currentfill}%
\pgfsetlinewidth{0.000000pt}%
\definecolor{currentstroke}{rgb}{0.000000,0.000000,0.000000}%
\pgfsetstrokecolor{currentstroke}%
\pgfsetdash{}{0pt}%
\pgfpathmoveto{\pgfqpoint{2.349853in}{0.872712in}}%
\pgfpathlineto{\pgfqpoint{2.590432in}{2.210709in}}%
\pgfpathlineto{\pgfqpoint{2.349853in}{0.872712in}}%
\pgfpathclose%
\pgfusepath{fill}%
\end{pgfscope}%
\begin{pgfscope}%
\pgfpathrectangle{\pgfqpoint{0.017500in}{0.017500in}}{\pgfqpoint{6.218282in}{3.580372in}} %
\pgfusepath{clip}%
\pgfsetbuttcap%
\pgfsetroundjoin%
\definecolor{currentfill}{rgb}{0.229806,0.298718,0.753683}%
\pgfsetfillcolor{currentfill}%
\pgfsetlinewidth{0.000000pt}%
\definecolor{currentstroke}{rgb}{0.000000,0.000000,0.000000}%
\pgfsetstrokecolor{currentstroke}%
\pgfsetdash{}{0pt}%
\pgfpathmoveto{\pgfqpoint{2.349853in}{0.872712in}}%
\pgfpathlineto{\pgfqpoint{2.590432in}{2.210709in}}%
\pgfpathlineto{\pgfqpoint{2.349853in}{0.872712in}}%
\pgfpathclose%
\pgfusepath{fill}%
\end{pgfscope}%
\begin{pgfscope}%
\pgfpathrectangle{\pgfqpoint{0.017500in}{0.017500in}}{\pgfqpoint{6.218282in}{3.580372in}} %
\pgfusepath{clip}%
\pgfsetbuttcap%
\pgfsetroundjoin%
\definecolor{currentfill}{rgb}{0.229806,0.298718,0.753683}%
\pgfsetfillcolor{currentfill}%
\pgfsetlinewidth{0.000000pt}%
\definecolor{currentstroke}{rgb}{0.000000,0.000000,0.000000}%
\pgfsetstrokecolor{currentstroke}%
\pgfsetdash{}{0pt}%
\pgfpathmoveto{\pgfqpoint{2.349853in}{0.872712in}}%
\pgfpathlineto{\pgfqpoint{2.590432in}{2.210709in}}%
\pgfpathlineto{\pgfqpoint{2.349853in}{0.872712in}}%
\pgfpathclose%
\pgfusepath{fill}%
\end{pgfscope}%
\begin{pgfscope}%
\pgfpathrectangle{\pgfqpoint{0.017500in}{0.017500in}}{\pgfqpoint{6.218282in}{3.580372in}} %
\pgfusepath{clip}%
\pgfsetbuttcap%
\pgfsetroundjoin%
\definecolor{currentfill}{rgb}{0.229806,0.298718,0.753683}%
\pgfsetfillcolor{currentfill}%
\pgfsetlinewidth{0.000000pt}%
\definecolor{currentstroke}{rgb}{0.000000,0.000000,0.000000}%
\pgfsetstrokecolor{currentstroke}%
\pgfsetdash{}{0pt}%
\pgfpathmoveto{\pgfqpoint{2.349853in}{0.872712in}}%
\pgfpathlineto{\pgfqpoint{2.590432in}{2.210709in}}%
\pgfpathlineto{\pgfqpoint{2.349853in}{0.872712in}}%
\pgfpathclose%
\pgfusepath{fill}%
\end{pgfscope}%
\begin{pgfscope}%
\pgfpathrectangle{\pgfqpoint{0.017500in}{0.017500in}}{\pgfqpoint{6.218282in}{3.580372in}} %
\pgfusepath{clip}%
\pgfsetbuttcap%
\pgfsetroundjoin%
\definecolor{currentfill}{rgb}{0.888390,0.417703,0.327898}%
\pgfsetfillcolor{currentfill}%
\pgfsetlinewidth{0.000000pt}%
\definecolor{currentstroke}{rgb}{0.000000,0.000000,0.000000}%
\pgfsetstrokecolor{currentstroke}%
\pgfsetdash{}{0pt}%
\pgfpathmoveto{\pgfqpoint{1.701778in}{2.375902in}}%
\pgfpathlineto{\pgfqpoint{1.971633in}{2.586781in}}%
\pgfpathlineto{\pgfqpoint{1.701778in}{2.375902in}}%
\pgfpathclose%
\pgfusepath{fill}%
\end{pgfscope}%
\begin{pgfscope}%
\pgfpathrectangle{\pgfqpoint{0.017500in}{0.017500in}}{\pgfqpoint{6.218282in}{3.580372in}} %
\pgfusepath{clip}%
\pgfsetbuttcap%
\pgfsetroundjoin%
\definecolor{currentfill}{rgb}{0.888390,0.417703,0.327898}%
\pgfsetfillcolor{currentfill}%
\pgfsetlinewidth{0.000000pt}%
\definecolor{currentstroke}{rgb}{0.000000,0.000000,0.000000}%
\pgfsetstrokecolor{currentstroke}%
\pgfsetdash{}{0pt}%
\pgfpathmoveto{\pgfqpoint{1.701778in}{2.375902in}}%
\pgfpathlineto{\pgfqpoint{1.971633in}{2.586781in}}%
\pgfpathlineto{\pgfqpoint{1.701778in}{2.375902in}}%
\pgfpathclose%
\pgfusepath{fill}%
\end{pgfscope}%
\begin{pgfscope}%
\pgfpathrectangle{\pgfqpoint{0.017500in}{0.017500in}}{\pgfqpoint{6.218282in}{3.580372in}} %
\pgfusepath{clip}%
\pgfsetbuttcap%
\pgfsetroundjoin%
\definecolor{currentfill}{rgb}{0.888390,0.417703,0.327898}%
\pgfsetfillcolor{currentfill}%
\pgfsetlinewidth{0.000000pt}%
\definecolor{currentstroke}{rgb}{0.000000,0.000000,0.000000}%
\pgfsetstrokecolor{currentstroke}%
\pgfsetdash{}{0pt}%
\pgfpathmoveto{\pgfqpoint{1.701778in}{2.375902in}}%
\pgfpathlineto{\pgfqpoint{1.971633in}{2.586781in}}%
\pgfpathlineto{\pgfqpoint{1.701778in}{2.375902in}}%
\pgfpathclose%
\pgfusepath{fill}%
\end{pgfscope}%
\begin{pgfscope}%
\pgfpathrectangle{\pgfqpoint{0.017500in}{0.017500in}}{\pgfqpoint{6.218282in}{3.580372in}} %
\pgfusepath{clip}%
\pgfsetbuttcap%
\pgfsetroundjoin%
\definecolor{currentfill}{rgb}{0.888390,0.417703,0.327898}%
\pgfsetfillcolor{currentfill}%
\pgfsetlinewidth{0.000000pt}%
\definecolor{currentstroke}{rgb}{0.000000,0.000000,0.000000}%
\pgfsetstrokecolor{currentstroke}%
\pgfsetdash{}{0pt}%
\pgfpathmoveto{\pgfqpoint{1.701778in}{2.375902in}}%
\pgfpathlineto{\pgfqpoint{1.971633in}{2.586781in}}%
\pgfpathlineto{\pgfqpoint{1.701778in}{2.375902in}}%
\pgfpathclose%
\pgfusepath{fill}%
\end{pgfscope}%
\begin{pgfscope}%
\pgfpathrectangle{\pgfqpoint{0.017500in}{0.017500in}}{\pgfqpoint{6.218282in}{3.580372in}} %
\pgfusepath{clip}%
\pgfsetbuttcap%
\pgfsetroundjoin%
\definecolor{currentfill}{rgb}{0.888390,0.417703,0.327898}%
\pgfsetfillcolor{currentfill}%
\pgfsetlinewidth{0.000000pt}%
\definecolor{currentstroke}{rgb}{0.000000,0.000000,0.000000}%
\pgfsetstrokecolor{currentstroke}%
\pgfsetdash{}{0pt}%
\pgfpathmoveto{\pgfqpoint{1.701778in}{2.375902in}}%
\pgfpathlineto{\pgfqpoint{1.971633in}{2.586781in}}%
\pgfpathlineto{\pgfqpoint{1.701778in}{2.375902in}}%
\pgfpathclose%
\pgfusepath{fill}%
\end{pgfscope}%
\begin{pgfscope}%
\pgfpathrectangle{\pgfqpoint{0.017500in}{0.017500in}}{\pgfqpoint{6.218282in}{3.580372in}} %
\pgfusepath{clip}%
\pgfsetbuttcap%
\pgfsetroundjoin%
\definecolor{currentfill}{rgb}{0.888390,0.417703,0.327898}%
\pgfsetfillcolor{currentfill}%
\pgfsetlinewidth{0.000000pt}%
\definecolor{currentstroke}{rgb}{0.000000,0.000000,0.000000}%
\pgfsetstrokecolor{currentstroke}%
\pgfsetdash{}{0pt}%
\pgfpathmoveto{\pgfqpoint{1.701778in}{2.375902in}}%
\pgfpathlineto{\pgfqpoint{1.971633in}{2.586781in}}%
\pgfpathlineto{\pgfqpoint{1.701778in}{2.375902in}}%
\pgfpathclose%
\pgfusepath{fill}%
\end{pgfscope}%
\begin{pgfscope}%
\pgfpathrectangle{\pgfqpoint{0.017500in}{0.017500in}}{\pgfqpoint{6.218282in}{3.580372in}} %
\pgfusepath{clip}%
\pgfsetbuttcap%
\pgfsetroundjoin%
\definecolor{currentfill}{rgb}{0.888390,0.417703,0.327898}%
\pgfsetfillcolor{currentfill}%
\pgfsetlinewidth{0.000000pt}%
\definecolor{currentstroke}{rgb}{0.000000,0.000000,0.000000}%
\pgfsetstrokecolor{currentstroke}%
\pgfsetdash{}{0pt}%
\pgfpathmoveto{\pgfqpoint{1.701778in}{2.375902in}}%
\pgfpathlineto{\pgfqpoint{1.971633in}{2.586781in}}%
\pgfpathlineto{\pgfqpoint{1.701778in}{2.375902in}}%
\pgfpathclose%
\pgfusepath{fill}%
\end{pgfscope}%
\begin{pgfscope}%
\pgfpathrectangle{\pgfqpoint{0.017500in}{0.017500in}}{\pgfqpoint{6.218282in}{3.580372in}} %
\pgfusepath{clip}%
\pgfsetbuttcap%
\pgfsetroundjoin%
\definecolor{currentfill}{rgb}{0.888390,0.417703,0.327898}%
\pgfsetfillcolor{currentfill}%
\pgfsetlinewidth{0.000000pt}%
\definecolor{currentstroke}{rgb}{0.000000,0.000000,0.000000}%
\pgfsetstrokecolor{currentstroke}%
\pgfsetdash{}{0pt}%
\pgfpathmoveto{\pgfqpoint{1.701778in}{2.375902in}}%
\pgfpathlineto{\pgfqpoint{1.971633in}{2.586781in}}%
\pgfpathlineto{\pgfqpoint{1.701778in}{2.375902in}}%
\pgfpathclose%
\pgfusepath{fill}%
\end{pgfscope}%
\begin{pgfscope}%
\pgfpathrectangle{\pgfqpoint{0.017500in}{0.017500in}}{\pgfqpoint{6.218282in}{3.580372in}} %
\pgfusepath{clip}%
\pgfsetbuttcap%
\pgfsetroundjoin%
\definecolor{currentfill}{rgb}{0.888390,0.417703,0.327898}%
\pgfsetfillcolor{currentfill}%
\pgfsetlinewidth{0.000000pt}%
\definecolor{currentstroke}{rgb}{0.000000,0.000000,0.000000}%
\pgfsetstrokecolor{currentstroke}%
\pgfsetdash{}{0pt}%
\pgfpathmoveto{\pgfqpoint{1.701778in}{2.375902in}}%
\pgfpathlineto{\pgfqpoint{1.971633in}{2.586781in}}%
\pgfpathlineto{\pgfqpoint{1.701778in}{2.375902in}}%
\pgfpathclose%
\pgfusepath{fill}%
\end{pgfscope}%
\begin{pgfscope}%
\pgfpathrectangle{\pgfqpoint{0.017500in}{0.017500in}}{\pgfqpoint{6.218282in}{3.580372in}} %
\pgfusepath{clip}%
\pgfsetbuttcap%
\pgfsetroundjoin%
\definecolor{currentfill}{rgb}{0.888390,0.417703,0.327898}%
\pgfsetfillcolor{currentfill}%
\pgfsetlinewidth{0.000000pt}%
\definecolor{currentstroke}{rgb}{0.000000,0.000000,0.000000}%
\pgfsetstrokecolor{currentstroke}%
\pgfsetdash{}{0pt}%
\pgfpathmoveto{\pgfqpoint{1.701778in}{2.375902in}}%
\pgfpathlineto{\pgfqpoint{1.971633in}{2.586781in}}%
\pgfpathlineto{\pgfqpoint{1.701778in}{2.375902in}}%
\pgfpathclose%
\pgfusepath{fill}%
\end{pgfscope}%
\begin{pgfscope}%
\pgfpathrectangle{\pgfqpoint{0.017500in}{0.017500in}}{\pgfqpoint{6.218282in}{3.580372in}} %
\pgfusepath{clip}%
\pgfsetbuttcap%
\pgfsetroundjoin%
\definecolor{currentfill}{rgb}{0.888390,0.417703,0.327898}%
\pgfsetfillcolor{currentfill}%
\pgfsetlinewidth{0.000000pt}%
\definecolor{currentstroke}{rgb}{0.000000,0.000000,0.000000}%
\pgfsetstrokecolor{currentstroke}%
\pgfsetdash{}{0pt}%
\pgfpathmoveto{\pgfqpoint{1.701778in}{2.375902in}}%
\pgfpathlineto{\pgfqpoint{1.971633in}{2.586781in}}%
\pgfpathlineto{\pgfqpoint{1.701778in}{2.375902in}}%
\pgfpathclose%
\pgfusepath{fill}%
\end{pgfscope}%
\begin{pgfscope}%
\pgfpathrectangle{\pgfqpoint{0.017500in}{0.017500in}}{\pgfqpoint{6.218282in}{3.580372in}} %
\pgfusepath{clip}%
\pgfsetbuttcap%
\pgfsetroundjoin%
\definecolor{currentfill}{rgb}{0.888390,0.417703,0.327898}%
\pgfsetfillcolor{currentfill}%
\pgfsetlinewidth{0.000000pt}%
\definecolor{currentstroke}{rgb}{0.000000,0.000000,0.000000}%
\pgfsetstrokecolor{currentstroke}%
\pgfsetdash{}{0pt}%
\pgfpathmoveto{\pgfqpoint{1.701778in}{2.375902in}}%
\pgfpathlineto{\pgfqpoint{1.971633in}{2.586781in}}%
\pgfpathlineto{\pgfqpoint{1.701778in}{2.375902in}}%
\pgfpathclose%
\pgfusepath{fill}%
\end{pgfscope}%
\begin{pgfscope}%
\pgfpathrectangle{\pgfqpoint{0.017500in}{0.017500in}}{\pgfqpoint{6.218282in}{3.580372in}} %
\pgfusepath{clip}%
\pgfsetbuttcap%
\pgfsetroundjoin%
\definecolor{currentfill}{rgb}{0.888390,0.417703,0.327898}%
\pgfsetfillcolor{currentfill}%
\pgfsetlinewidth{0.000000pt}%
\definecolor{currentstroke}{rgb}{0.000000,0.000000,0.000000}%
\pgfsetstrokecolor{currentstroke}%
\pgfsetdash{}{0pt}%
\pgfpathmoveto{\pgfqpoint{1.701778in}{2.375902in}}%
\pgfpathlineto{\pgfqpoint{1.971633in}{2.586781in}}%
\pgfpathlineto{\pgfqpoint{1.701778in}{2.375902in}}%
\pgfpathclose%
\pgfusepath{fill}%
\end{pgfscope}%
\begin{pgfscope}%
\pgfpathrectangle{\pgfqpoint{0.017500in}{0.017500in}}{\pgfqpoint{6.218282in}{3.580372in}} %
\pgfusepath{clip}%
\pgfsetbuttcap%
\pgfsetroundjoin%
\definecolor{currentfill}{rgb}{0.888390,0.417703,0.327898}%
\pgfsetfillcolor{currentfill}%
\pgfsetlinewidth{0.000000pt}%
\definecolor{currentstroke}{rgb}{0.000000,0.000000,0.000000}%
\pgfsetstrokecolor{currentstroke}%
\pgfsetdash{}{0pt}%
\pgfpathmoveto{\pgfqpoint{1.701778in}{2.375902in}}%
\pgfpathlineto{\pgfqpoint{1.971633in}{2.586781in}}%
\pgfpathlineto{\pgfqpoint{1.701778in}{2.375902in}}%
\pgfpathclose%
\pgfusepath{fill}%
\end{pgfscope}%
\begin{pgfscope}%
\pgfpathrectangle{\pgfqpoint{0.017500in}{0.017500in}}{\pgfqpoint{6.218282in}{3.580372in}} %
\pgfusepath{clip}%
\pgfsetbuttcap%
\pgfsetroundjoin%
\definecolor{currentfill}{rgb}{0.888390,0.417703,0.327898}%
\pgfsetfillcolor{currentfill}%
\pgfsetlinewidth{0.000000pt}%
\definecolor{currentstroke}{rgb}{0.000000,0.000000,0.000000}%
\pgfsetstrokecolor{currentstroke}%
\pgfsetdash{}{0pt}%
\pgfpathmoveto{\pgfqpoint{1.701778in}{2.375902in}}%
\pgfpathlineto{\pgfqpoint{1.971633in}{2.586781in}}%
\pgfpathlineto{\pgfqpoint{1.701778in}{2.375902in}}%
\pgfpathclose%
\pgfusepath{fill}%
\end{pgfscope}%
\begin{pgfscope}%
\pgfpathrectangle{\pgfqpoint{0.017500in}{0.017500in}}{\pgfqpoint{6.218282in}{3.580372in}} %
\pgfusepath{clip}%
\pgfsetbuttcap%
\pgfsetroundjoin%
\definecolor{currentfill}{rgb}{0.888390,0.417703,0.327898}%
\pgfsetfillcolor{currentfill}%
\pgfsetlinewidth{0.000000pt}%
\definecolor{currentstroke}{rgb}{0.000000,0.000000,0.000000}%
\pgfsetstrokecolor{currentstroke}%
\pgfsetdash{}{0pt}%
\pgfpathmoveto{\pgfqpoint{1.701778in}{2.375902in}}%
\pgfpathlineto{\pgfqpoint{1.971633in}{2.586781in}}%
\pgfpathlineto{\pgfqpoint{1.701778in}{2.375902in}}%
\pgfpathclose%
\pgfusepath{fill}%
\end{pgfscope}%
\begin{pgfscope}%
\pgfpathrectangle{\pgfqpoint{0.017500in}{0.017500in}}{\pgfqpoint{6.218282in}{3.580372in}} %
\pgfusepath{clip}%
\pgfsetbuttcap%
\pgfsetroundjoin%
\definecolor{currentfill}{rgb}{0.888390,0.417703,0.327898}%
\pgfsetfillcolor{currentfill}%
\pgfsetlinewidth{0.000000pt}%
\definecolor{currentstroke}{rgb}{0.000000,0.000000,0.000000}%
\pgfsetstrokecolor{currentstroke}%
\pgfsetdash{}{0pt}%
\pgfpathmoveto{\pgfqpoint{1.701778in}{2.375902in}}%
\pgfpathlineto{\pgfqpoint{1.971633in}{2.586781in}}%
\pgfpathlineto{\pgfqpoint{1.701778in}{2.375902in}}%
\pgfpathclose%
\pgfusepath{fill}%
\end{pgfscope}%
\begin{pgfscope}%
\pgfpathrectangle{\pgfqpoint{0.017500in}{0.017500in}}{\pgfqpoint{6.218282in}{3.580372in}} %
\pgfusepath{clip}%
\pgfsetbuttcap%
\pgfsetroundjoin%
\definecolor{currentfill}{rgb}{0.888390,0.417703,0.327898}%
\pgfsetfillcolor{currentfill}%
\pgfsetlinewidth{0.000000pt}%
\definecolor{currentstroke}{rgb}{0.000000,0.000000,0.000000}%
\pgfsetstrokecolor{currentstroke}%
\pgfsetdash{}{0pt}%
\pgfpathmoveto{\pgfqpoint{1.701778in}{2.375902in}}%
\pgfpathlineto{\pgfqpoint{1.971633in}{2.586781in}}%
\pgfpathlineto{\pgfqpoint{1.701778in}{2.375902in}}%
\pgfpathclose%
\pgfusepath{fill}%
\end{pgfscope}%
\begin{pgfscope}%
\pgfpathrectangle{\pgfqpoint{0.017500in}{0.017500in}}{\pgfqpoint{6.218282in}{3.580372in}} %
\pgfusepath{clip}%
\pgfsetbuttcap%
\pgfsetroundjoin%
\definecolor{currentfill}{rgb}{0.888390,0.417703,0.327898}%
\pgfsetfillcolor{currentfill}%
\pgfsetlinewidth{0.000000pt}%
\definecolor{currentstroke}{rgb}{0.000000,0.000000,0.000000}%
\pgfsetstrokecolor{currentstroke}%
\pgfsetdash{}{0pt}%
\pgfpathmoveto{\pgfqpoint{1.701778in}{2.375902in}}%
\pgfpathlineto{\pgfqpoint{1.971633in}{2.586781in}}%
\pgfpathlineto{\pgfqpoint{1.701778in}{2.375902in}}%
\pgfpathclose%
\pgfusepath{fill}%
\end{pgfscope}%
\begin{pgfscope}%
\pgfpathrectangle{\pgfqpoint{0.017500in}{0.017500in}}{\pgfqpoint{6.218282in}{3.580372in}} %
\pgfusepath{clip}%
\pgfsetbuttcap%
\pgfsetroundjoin%
\definecolor{currentfill}{rgb}{0.888390,0.417703,0.327898}%
\pgfsetfillcolor{currentfill}%
\pgfsetlinewidth{0.000000pt}%
\definecolor{currentstroke}{rgb}{0.000000,0.000000,0.000000}%
\pgfsetstrokecolor{currentstroke}%
\pgfsetdash{}{0pt}%
\pgfpathmoveto{\pgfqpoint{1.701778in}{2.375902in}}%
\pgfpathlineto{\pgfqpoint{1.971633in}{2.586781in}}%
\pgfpathlineto{\pgfqpoint{1.701778in}{2.375902in}}%
\pgfpathclose%
\pgfusepath{fill}%
\end{pgfscope}%
\begin{pgfscope}%
\pgfpathrectangle{\pgfqpoint{0.017500in}{0.017500in}}{\pgfqpoint{6.218282in}{3.580372in}} %
\pgfusepath{clip}%
\pgfsetbuttcap%
\pgfsetroundjoin%
\definecolor{currentfill}{rgb}{0.888390,0.417703,0.327898}%
\pgfsetfillcolor{currentfill}%
\pgfsetlinewidth{0.000000pt}%
\definecolor{currentstroke}{rgb}{0.000000,0.000000,0.000000}%
\pgfsetstrokecolor{currentstroke}%
\pgfsetdash{}{0pt}%
\pgfpathmoveto{\pgfqpoint{1.701778in}{2.375902in}}%
\pgfpathlineto{\pgfqpoint{1.971633in}{2.586781in}}%
\pgfpathlineto{\pgfqpoint{1.701778in}{2.375902in}}%
\pgfpathclose%
\pgfusepath{fill}%
\end{pgfscope}%
\begin{pgfscope}%
\pgfpathrectangle{\pgfqpoint{0.017500in}{0.017500in}}{\pgfqpoint{6.218282in}{3.580372in}} %
\pgfusepath{clip}%
\pgfsetbuttcap%
\pgfsetroundjoin%
\definecolor{currentfill}{rgb}{0.888390,0.417703,0.327898}%
\pgfsetfillcolor{currentfill}%
\pgfsetlinewidth{0.000000pt}%
\definecolor{currentstroke}{rgb}{0.000000,0.000000,0.000000}%
\pgfsetstrokecolor{currentstroke}%
\pgfsetdash{}{0pt}%
\pgfpathmoveto{\pgfqpoint{1.701778in}{2.375902in}}%
\pgfpathlineto{\pgfqpoint{1.971633in}{2.586781in}}%
\pgfpathlineto{\pgfqpoint{1.701778in}{2.375902in}}%
\pgfpathclose%
\pgfusepath{fill}%
\end{pgfscope}%
\begin{pgfscope}%
\pgfpathrectangle{\pgfqpoint{0.017500in}{0.017500in}}{\pgfqpoint{6.218282in}{3.580372in}} %
\pgfusepath{clip}%
\pgfsetbuttcap%
\pgfsetroundjoin%
\definecolor{currentfill}{rgb}{0.888390,0.417703,0.327898}%
\pgfsetfillcolor{currentfill}%
\pgfsetlinewidth{0.000000pt}%
\definecolor{currentstroke}{rgb}{0.000000,0.000000,0.000000}%
\pgfsetstrokecolor{currentstroke}%
\pgfsetdash{}{0pt}%
\pgfpathmoveto{\pgfqpoint{1.701778in}{2.375902in}}%
\pgfpathlineto{\pgfqpoint{1.971633in}{2.586781in}}%
\pgfpathlineto{\pgfqpoint{1.701778in}{2.375902in}}%
\pgfpathclose%
\pgfusepath{fill}%
\end{pgfscope}%
\begin{pgfscope}%
\pgfpathrectangle{\pgfqpoint{0.017500in}{0.017500in}}{\pgfqpoint{6.218282in}{3.580372in}} %
\pgfusepath{clip}%
\pgfsetbuttcap%
\pgfsetroundjoin%
\definecolor{currentfill}{rgb}{0.888390,0.417703,0.327898}%
\pgfsetfillcolor{currentfill}%
\pgfsetlinewidth{0.000000pt}%
\definecolor{currentstroke}{rgb}{0.000000,0.000000,0.000000}%
\pgfsetstrokecolor{currentstroke}%
\pgfsetdash{}{0pt}%
\pgfpathmoveto{\pgfqpoint{1.701778in}{2.375902in}}%
\pgfpathlineto{\pgfqpoint{1.971633in}{2.586781in}}%
\pgfpathlineto{\pgfqpoint{1.701778in}{2.375902in}}%
\pgfpathclose%
\pgfusepath{fill}%
\end{pgfscope}%
\begin{pgfscope}%
\pgfpathrectangle{\pgfqpoint{0.017500in}{0.017500in}}{\pgfqpoint{6.218282in}{3.580372in}} %
\pgfusepath{clip}%
\pgfsetbuttcap%
\pgfsetroundjoin%
\definecolor{currentfill}{rgb}{0.888390,0.417703,0.327898}%
\pgfsetfillcolor{currentfill}%
\pgfsetlinewidth{0.000000pt}%
\definecolor{currentstroke}{rgb}{0.000000,0.000000,0.000000}%
\pgfsetstrokecolor{currentstroke}%
\pgfsetdash{}{0pt}%
\pgfpathmoveto{\pgfqpoint{1.701778in}{2.375902in}}%
\pgfpathlineto{\pgfqpoint{1.971633in}{2.586781in}}%
\pgfpathlineto{\pgfqpoint{1.701778in}{2.375902in}}%
\pgfpathclose%
\pgfusepath{fill}%
\end{pgfscope}%
\begin{pgfscope}%
\pgfpathrectangle{\pgfqpoint{0.017500in}{0.017500in}}{\pgfqpoint{6.218282in}{3.580372in}} %
\pgfusepath{clip}%
\pgfsetbuttcap%
\pgfsetroundjoin%
\definecolor{currentfill}{rgb}{0.888390,0.417703,0.327898}%
\pgfsetfillcolor{currentfill}%
\pgfsetlinewidth{0.000000pt}%
\definecolor{currentstroke}{rgb}{0.000000,0.000000,0.000000}%
\pgfsetstrokecolor{currentstroke}%
\pgfsetdash{}{0pt}%
\pgfpathmoveto{\pgfqpoint{1.701778in}{2.375902in}}%
\pgfpathlineto{\pgfqpoint{1.971633in}{2.586781in}}%
\pgfpathlineto{\pgfqpoint{1.701778in}{2.375902in}}%
\pgfpathclose%
\pgfusepath{fill}%
\end{pgfscope}%
\begin{pgfscope}%
\pgfpathrectangle{\pgfqpoint{0.017500in}{0.017500in}}{\pgfqpoint{6.218282in}{3.580372in}} %
\pgfusepath{clip}%
\pgfsetbuttcap%
\pgfsetroundjoin%
\definecolor{currentfill}{rgb}{0.888390,0.417703,0.327898}%
\pgfsetfillcolor{currentfill}%
\pgfsetlinewidth{0.000000pt}%
\definecolor{currentstroke}{rgb}{0.000000,0.000000,0.000000}%
\pgfsetstrokecolor{currentstroke}%
\pgfsetdash{}{0pt}%
\pgfpathmoveto{\pgfqpoint{1.701778in}{2.375902in}}%
\pgfpathlineto{\pgfqpoint{1.971633in}{2.586781in}}%
\pgfpathlineto{\pgfqpoint{1.701778in}{2.375902in}}%
\pgfpathclose%
\pgfusepath{fill}%
\end{pgfscope}%
\begin{pgfscope}%
\pgfpathrectangle{\pgfqpoint{0.017500in}{0.017500in}}{\pgfqpoint{6.218282in}{3.580372in}} %
\pgfusepath{clip}%
\pgfsetbuttcap%
\pgfsetroundjoin%
\definecolor{currentfill}{rgb}{0.888390,0.417703,0.327898}%
\pgfsetfillcolor{currentfill}%
\pgfsetlinewidth{0.000000pt}%
\definecolor{currentstroke}{rgb}{0.000000,0.000000,0.000000}%
\pgfsetstrokecolor{currentstroke}%
\pgfsetdash{}{0pt}%
\pgfpathmoveto{\pgfqpoint{1.701778in}{2.375902in}}%
\pgfpathlineto{\pgfqpoint{1.971633in}{2.586781in}}%
\pgfpathlineto{\pgfqpoint{1.701778in}{2.375902in}}%
\pgfpathclose%
\pgfusepath{fill}%
\end{pgfscope}%
\begin{pgfscope}%
\pgfpathrectangle{\pgfqpoint{0.017500in}{0.017500in}}{\pgfqpoint{6.218282in}{3.580372in}} %
\pgfusepath{clip}%
\pgfsetbuttcap%
\pgfsetroundjoin%
\definecolor{currentfill}{rgb}{0.888390,0.417703,0.327898}%
\pgfsetfillcolor{currentfill}%
\pgfsetlinewidth{0.000000pt}%
\definecolor{currentstroke}{rgb}{0.000000,0.000000,0.000000}%
\pgfsetstrokecolor{currentstroke}%
\pgfsetdash{}{0pt}%
\pgfpathmoveto{\pgfqpoint{1.701778in}{2.375902in}}%
\pgfpathlineto{\pgfqpoint{1.971633in}{2.586781in}}%
\pgfpathlineto{\pgfqpoint{1.701778in}{2.375902in}}%
\pgfpathclose%
\pgfusepath{fill}%
\end{pgfscope}%
\begin{pgfscope}%
\pgfpathrectangle{\pgfqpoint{0.017500in}{0.017500in}}{\pgfqpoint{6.218282in}{3.580372in}} %
\pgfusepath{clip}%
\pgfsetbuttcap%
\pgfsetroundjoin%
\definecolor{currentfill}{rgb}{0.888390,0.417703,0.327898}%
\pgfsetfillcolor{currentfill}%
\pgfsetlinewidth{0.000000pt}%
\definecolor{currentstroke}{rgb}{0.000000,0.000000,0.000000}%
\pgfsetstrokecolor{currentstroke}%
\pgfsetdash{}{0pt}%
\pgfpathmoveto{\pgfqpoint{1.701778in}{2.375902in}}%
\pgfpathlineto{\pgfqpoint{1.971633in}{2.586781in}}%
\pgfpathlineto{\pgfqpoint{1.701778in}{2.375902in}}%
\pgfpathclose%
\pgfusepath{fill}%
\end{pgfscope}%
\begin{pgfscope}%
\pgfpathrectangle{\pgfqpoint{0.017500in}{0.017500in}}{\pgfqpoint{6.218282in}{3.580372in}} %
\pgfusepath{clip}%
\pgfsetbuttcap%
\pgfsetroundjoin%
\definecolor{currentfill}{rgb}{0.763520,0.178667,0.193396}%
\pgfsetfillcolor{currentfill}%
\pgfsetlinewidth{0.000000pt}%
\definecolor{currentstroke}{rgb}{0.000000,0.000000,0.000000}%
\pgfsetstrokecolor{currentstroke}%
\pgfsetdash{}{0pt}%
\pgfpathmoveto{\pgfqpoint{3.564492in}{2.541257in}}%
\pgfpathlineto{\pgfqpoint{3.816789in}{2.677291in}}%
\pgfpathlineto{\pgfqpoint{3.564492in}{2.541257in}}%
\pgfpathclose%
\pgfusepath{fill}%
\end{pgfscope}%
\begin{pgfscope}%
\pgfpathrectangle{\pgfqpoint{0.017500in}{0.017500in}}{\pgfqpoint{6.218282in}{3.580372in}} %
\pgfusepath{clip}%
\pgfsetbuttcap%
\pgfsetroundjoin%
\definecolor{currentfill}{rgb}{0.763520,0.178667,0.193396}%
\pgfsetfillcolor{currentfill}%
\pgfsetlinewidth{0.000000pt}%
\definecolor{currentstroke}{rgb}{0.000000,0.000000,0.000000}%
\pgfsetstrokecolor{currentstroke}%
\pgfsetdash{}{0pt}%
\pgfpathmoveto{\pgfqpoint{3.564492in}{2.541257in}}%
\pgfpathlineto{\pgfqpoint{3.816789in}{2.677291in}}%
\pgfpathlineto{\pgfqpoint{3.564492in}{2.541257in}}%
\pgfpathclose%
\pgfusepath{fill}%
\end{pgfscope}%
\begin{pgfscope}%
\pgfpathrectangle{\pgfqpoint{0.017500in}{0.017500in}}{\pgfqpoint{6.218282in}{3.580372in}} %
\pgfusepath{clip}%
\pgfsetbuttcap%
\pgfsetroundjoin%
\definecolor{currentfill}{rgb}{0.763520,0.178667,0.193396}%
\pgfsetfillcolor{currentfill}%
\pgfsetlinewidth{0.000000pt}%
\definecolor{currentstroke}{rgb}{0.000000,0.000000,0.000000}%
\pgfsetstrokecolor{currentstroke}%
\pgfsetdash{}{0pt}%
\pgfpathmoveto{\pgfqpoint{3.564492in}{2.541257in}}%
\pgfpathlineto{\pgfqpoint{3.816789in}{2.677291in}}%
\pgfpathlineto{\pgfqpoint{3.564492in}{2.541257in}}%
\pgfpathclose%
\pgfusepath{fill}%
\end{pgfscope}%
\begin{pgfscope}%
\pgfpathrectangle{\pgfqpoint{0.017500in}{0.017500in}}{\pgfqpoint{6.218282in}{3.580372in}} %
\pgfusepath{clip}%
\pgfsetbuttcap%
\pgfsetroundjoin%
\definecolor{currentfill}{rgb}{0.763520,0.178667,0.193396}%
\pgfsetfillcolor{currentfill}%
\pgfsetlinewidth{0.000000pt}%
\definecolor{currentstroke}{rgb}{0.000000,0.000000,0.000000}%
\pgfsetstrokecolor{currentstroke}%
\pgfsetdash{}{0pt}%
\pgfpathmoveto{\pgfqpoint{3.564492in}{2.541257in}}%
\pgfpathlineto{\pgfqpoint{3.816789in}{2.677291in}}%
\pgfpathlineto{\pgfqpoint{3.564492in}{2.541257in}}%
\pgfpathclose%
\pgfusepath{fill}%
\end{pgfscope}%
\begin{pgfscope}%
\pgfpathrectangle{\pgfqpoint{0.017500in}{0.017500in}}{\pgfqpoint{6.218282in}{3.580372in}} %
\pgfusepath{clip}%
\pgfsetbuttcap%
\pgfsetroundjoin%
\definecolor{currentfill}{rgb}{0.763520,0.178667,0.193396}%
\pgfsetfillcolor{currentfill}%
\pgfsetlinewidth{0.000000pt}%
\definecolor{currentstroke}{rgb}{0.000000,0.000000,0.000000}%
\pgfsetstrokecolor{currentstroke}%
\pgfsetdash{}{0pt}%
\pgfpathmoveto{\pgfqpoint{3.564492in}{2.541257in}}%
\pgfpathlineto{\pgfqpoint{3.816789in}{2.677291in}}%
\pgfpathlineto{\pgfqpoint{3.564492in}{2.541257in}}%
\pgfpathclose%
\pgfusepath{fill}%
\end{pgfscope}%
\begin{pgfscope}%
\pgfpathrectangle{\pgfqpoint{0.017500in}{0.017500in}}{\pgfqpoint{6.218282in}{3.580372in}} %
\pgfusepath{clip}%
\pgfsetbuttcap%
\pgfsetroundjoin%
\definecolor{currentfill}{rgb}{0.763520,0.178667,0.193396}%
\pgfsetfillcolor{currentfill}%
\pgfsetlinewidth{0.000000pt}%
\definecolor{currentstroke}{rgb}{0.000000,0.000000,0.000000}%
\pgfsetstrokecolor{currentstroke}%
\pgfsetdash{}{0pt}%
\pgfpathmoveto{\pgfqpoint{3.564492in}{2.541257in}}%
\pgfpathlineto{\pgfqpoint{3.816789in}{2.677291in}}%
\pgfpathlineto{\pgfqpoint{3.564492in}{2.541257in}}%
\pgfpathclose%
\pgfusepath{fill}%
\end{pgfscope}%
\begin{pgfscope}%
\pgfpathrectangle{\pgfqpoint{0.017500in}{0.017500in}}{\pgfqpoint{6.218282in}{3.580372in}} %
\pgfusepath{clip}%
\pgfsetbuttcap%
\pgfsetroundjoin%
\definecolor{currentfill}{rgb}{0.763520,0.178667,0.193396}%
\pgfsetfillcolor{currentfill}%
\pgfsetlinewidth{0.000000pt}%
\definecolor{currentstroke}{rgb}{0.000000,0.000000,0.000000}%
\pgfsetstrokecolor{currentstroke}%
\pgfsetdash{}{0pt}%
\pgfpathmoveto{\pgfqpoint{3.564492in}{2.541257in}}%
\pgfpathlineto{\pgfqpoint{3.816789in}{2.677291in}}%
\pgfpathlineto{\pgfqpoint{3.564492in}{2.541257in}}%
\pgfpathclose%
\pgfusepath{fill}%
\end{pgfscope}%
\begin{pgfscope}%
\pgfpathrectangle{\pgfqpoint{0.017500in}{0.017500in}}{\pgfqpoint{6.218282in}{3.580372in}} %
\pgfusepath{clip}%
\pgfsetbuttcap%
\pgfsetroundjoin%
\definecolor{currentfill}{rgb}{0.763520,0.178667,0.193396}%
\pgfsetfillcolor{currentfill}%
\pgfsetlinewidth{0.000000pt}%
\definecolor{currentstroke}{rgb}{0.000000,0.000000,0.000000}%
\pgfsetstrokecolor{currentstroke}%
\pgfsetdash{}{0pt}%
\pgfpathmoveto{\pgfqpoint{3.564492in}{2.541257in}}%
\pgfpathlineto{\pgfqpoint{3.816789in}{2.677291in}}%
\pgfpathlineto{\pgfqpoint{3.564492in}{2.541257in}}%
\pgfpathclose%
\pgfusepath{fill}%
\end{pgfscope}%
\begin{pgfscope}%
\pgfpathrectangle{\pgfqpoint{0.017500in}{0.017500in}}{\pgfqpoint{6.218282in}{3.580372in}} %
\pgfusepath{clip}%
\pgfsetbuttcap%
\pgfsetroundjoin%
\definecolor{currentfill}{rgb}{0.763520,0.178667,0.193396}%
\pgfsetfillcolor{currentfill}%
\pgfsetlinewidth{0.000000pt}%
\definecolor{currentstroke}{rgb}{0.000000,0.000000,0.000000}%
\pgfsetstrokecolor{currentstroke}%
\pgfsetdash{}{0pt}%
\pgfpathmoveto{\pgfqpoint{3.564492in}{2.541257in}}%
\pgfpathlineto{\pgfqpoint{3.816789in}{2.677291in}}%
\pgfpathlineto{\pgfqpoint{3.564492in}{2.541257in}}%
\pgfpathclose%
\pgfusepath{fill}%
\end{pgfscope}%
\begin{pgfscope}%
\pgfpathrectangle{\pgfqpoint{0.017500in}{0.017500in}}{\pgfqpoint{6.218282in}{3.580372in}} %
\pgfusepath{clip}%
\pgfsetbuttcap%
\pgfsetroundjoin%
\definecolor{currentfill}{rgb}{0.763520,0.178667,0.193396}%
\pgfsetfillcolor{currentfill}%
\pgfsetlinewidth{0.000000pt}%
\definecolor{currentstroke}{rgb}{0.000000,0.000000,0.000000}%
\pgfsetstrokecolor{currentstroke}%
\pgfsetdash{}{0pt}%
\pgfpathmoveto{\pgfqpoint{3.564492in}{2.541257in}}%
\pgfpathlineto{\pgfqpoint{3.816789in}{2.677291in}}%
\pgfpathlineto{\pgfqpoint{3.564492in}{2.541257in}}%
\pgfpathclose%
\pgfusepath{fill}%
\end{pgfscope}%
\begin{pgfscope}%
\pgfpathrectangle{\pgfqpoint{0.017500in}{0.017500in}}{\pgfqpoint{6.218282in}{3.580372in}} %
\pgfusepath{clip}%
\pgfsetbuttcap%
\pgfsetroundjoin%
\definecolor{currentfill}{rgb}{0.763520,0.178667,0.193396}%
\pgfsetfillcolor{currentfill}%
\pgfsetlinewidth{0.000000pt}%
\definecolor{currentstroke}{rgb}{0.000000,0.000000,0.000000}%
\pgfsetstrokecolor{currentstroke}%
\pgfsetdash{}{0pt}%
\pgfpathmoveto{\pgfqpoint{3.564492in}{2.541257in}}%
\pgfpathlineto{\pgfqpoint{3.816789in}{2.677291in}}%
\pgfpathlineto{\pgfqpoint{3.564492in}{2.541257in}}%
\pgfpathclose%
\pgfusepath{fill}%
\end{pgfscope}%
\begin{pgfscope}%
\pgfpathrectangle{\pgfqpoint{0.017500in}{0.017500in}}{\pgfqpoint{6.218282in}{3.580372in}} %
\pgfusepath{clip}%
\pgfsetbuttcap%
\pgfsetroundjoin%
\definecolor{currentfill}{rgb}{0.763520,0.178667,0.193396}%
\pgfsetfillcolor{currentfill}%
\pgfsetlinewidth{0.000000pt}%
\definecolor{currentstroke}{rgb}{0.000000,0.000000,0.000000}%
\pgfsetstrokecolor{currentstroke}%
\pgfsetdash{}{0pt}%
\pgfpathmoveto{\pgfqpoint{3.564492in}{2.541257in}}%
\pgfpathlineto{\pgfqpoint{3.816789in}{2.677291in}}%
\pgfpathlineto{\pgfqpoint{3.564492in}{2.541257in}}%
\pgfpathclose%
\pgfusepath{fill}%
\end{pgfscope}%
\begin{pgfscope}%
\pgfpathrectangle{\pgfqpoint{0.017500in}{0.017500in}}{\pgfqpoint{6.218282in}{3.580372in}} %
\pgfusepath{clip}%
\pgfsetbuttcap%
\pgfsetroundjoin%
\definecolor{currentfill}{rgb}{0.763520,0.178667,0.193396}%
\pgfsetfillcolor{currentfill}%
\pgfsetlinewidth{0.000000pt}%
\definecolor{currentstroke}{rgb}{0.000000,0.000000,0.000000}%
\pgfsetstrokecolor{currentstroke}%
\pgfsetdash{}{0pt}%
\pgfpathmoveto{\pgfqpoint{3.564492in}{2.541257in}}%
\pgfpathlineto{\pgfqpoint{3.816789in}{2.677291in}}%
\pgfpathlineto{\pgfqpoint{3.564492in}{2.541257in}}%
\pgfpathclose%
\pgfusepath{fill}%
\end{pgfscope}%
\begin{pgfscope}%
\pgfpathrectangle{\pgfqpoint{0.017500in}{0.017500in}}{\pgfqpoint{6.218282in}{3.580372in}} %
\pgfusepath{clip}%
\pgfsetbuttcap%
\pgfsetroundjoin%
\definecolor{currentfill}{rgb}{0.763520,0.178667,0.193396}%
\pgfsetfillcolor{currentfill}%
\pgfsetlinewidth{0.000000pt}%
\definecolor{currentstroke}{rgb}{0.000000,0.000000,0.000000}%
\pgfsetstrokecolor{currentstroke}%
\pgfsetdash{}{0pt}%
\pgfpathmoveto{\pgfqpoint{3.564492in}{2.541257in}}%
\pgfpathlineto{\pgfqpoint{3.816789in}{2.677291in}}%
\pgfpathlineto{\pgfqpoint{3.564492in}{2.541257in}}%
\pgfpathclose%
\pgfusepath{fill}%
\end{pgfscope}%
\begin{pgfscope}%
\pgfpathrectangle{\pgfqpoint{0.017500in}{0.017500in}}{\pgfqpoint{6.218282in}{3.580372in}} %
\pgfusepath{clip}%
\pgfsetbuttcap%
\pgfsetroundjoin%
\definecolor{currentfill}{rgb}{0.763520,0.178667,0.193396}%
\pgfsetfillcolor{currentfill}%
\pgfsetlinewidth{0.000000pt}%
\definecolor{currentstroke}{rgb}{0.000000,0.000000,0.000000}%
\pgfsetstrokecolor{currentstroke}%
\pgfsetdash{}{0pt}%
\pgfpathmoveto{\pgfqpoint{3.564492in}{2.541257in}}%
\pgfpathlineto{\pgfqpoint{3.816789in}{2.677291in}}%
\pgfpathlineto{\pgfqpoint{3.564492in}{2.541257in}}%
\pgfpathclose%
\pgfusepath{fill}%
\end{pgfscope}%
\begin{pgfscope}%
\pgfpathrectangle{\pgfqpoint{0.017500in}{0.017500in}}{\pgfqpoint{6.218282in}{3.580372in}} %
\pgfusepath{clip}%
\pgfsetbuttcap%
\pgfsetroundjoin%
\definecolor{currentfill}{rgb}{0.763520,0.178667,0.193396}%
\pgfsetfillcolor{currentfill}%
\pgfsetlinewidth{0.000000pt}%
\definecolor{currentstroke}{rgb}{0.000000,0.000000,0.000000}%
\pgfsetstrokecolor{currentstroke}%
\pgfsetdash{}{0pt}%
\pgfpathmoveto{\pgfqpoint{3.564492in}{2.541257in}}%
\pgfpathlineto{\pgfqpoint{3.816789in}{2.677291in}}%
\pgfpathlineto{\pgfqpoint{3.564492in}{2.541257in}}%
\pgfpathclose%
\pgfusepath{fill}%
\end{pgfscope}%
\begin{pgfscope}%
\pgfpathrectangle{\pgfqpoint{0.017500in}{0.017500in}}{\pgfqpoint{6.218282in}{3.580372in}} %
\pgfusepath{clip}%
\pgfsetbuttcap%
\pgfsetroundjoin%
\definecolor{currentfill}{rgb}{0.763520,0.178667,0.193396}%
\pgfsetfillcolor{currentfill}%
\pgfsetlinewidth{0.000000pt}%
\definecolor{currentstroke}{rgb}{0.000000,0.000000,0.000000}%
\pgfsetstrokecolor{currentstroke}%
\pgfsetdash{}{0pt}%
\pgfpathmoveto{\pgfqpoint{3.564492in}{2.541257in}}%
\pgfpathlineto{\pgfqpoint{3.816789in}{2.677291in}}%
\pgfpathlineto{\pgfqpoint{3.564492in}{2.541257in}}%
\pgfpathclose%
\pgfusepath{fill}%
\end{pgfscope}%
\begin{pgfscope}%
\pgfpathrectangle{\pgfqpoint{0.017500in}{0.017500in}}{\pgfqpoint{6.218282in}{3.580372in}} %
\pgfusepath{clip}%
\pgfsetbuttcap%
\pgfsetroundjoin%
\definecolor{currentfill}{rgb}{0.763520,0.178667,0.193396}%
\pgfsetfillcolor{currentfill}%
\pgfsetlinewidth{0.000000pt}%
\definecolor{currentstroke}{rgb}{0.000000,0.000000,0.000000}%
\pgfsetstrokecolor{currentstroke}%
\pgfsetdash{}{0pt}%
\pgfpathmoveto{\pgfqpoint{3.564492in}{2.541257in}}%
\pgfpathlineto{\pgfqpoint{3.816789in}{2.677291in}}%
\pgfpathlineto{\pgfqpoint{3.564492in}{2.541257in}}%
\pgfpathclose%
\pgfusepath{fill}%
\end{pgfscope}%
\begin{pgfscope}%
\pgfpathrectangle{\pgfqpoint{0.017500in}{0.017500in}}{\pgfqpoint{6.218282in}{3.580372in}} %
\pgfusepath{clip}%
\pgfsetbuttcap%
\pgfsetroundjoin%
\definecolor{currentfill}{rgb}{0.763520,0.178667,0.193396}%
\pgfsetfillcolor{currentfill}%
\pgfsetlinewidth{0.000000pt}%
\definecolor{currentstroke}{rgb}{0.000000,0.000000,0.000000}%
\pgfsetstrokecolor{currentstroke}%
\pgfsetdash{}{0pt}%
\pgfpathmoveto{\pgfqpoint{3.564492in}{2.541257in}}%
\pgfpathlineto{\pgfqpoint{3.816789in}{2.677291in}}%
\pgfpathlineto{\pgfqpoint{3.564492in}{2.541257in}}%
\pgfpathclose%
\pgfusepath{fill}%
\end{pgfscope}%
\begin{pgfscope}%
\pgfpathrectangle{\pgfqpoint{0.017500in}{0.017500in}}{\pgfqpoint{6.218282in}{3.580372in}} %
\pgfusepath{clip}%
\pgfsetbuttcap%
\pgfsetroundjoin%
\definecolor{currentfill}{rgb}{0.763520,0.178667,0.193396}%
\pgfsetfillcolor{currentfill}%
\pgfsetlinewidth{0.000000pt}%
\definecolor{currentstroke}{rgb}{0.000000,0.000000,0.000000}%
\pgfsetstrokecolor{currentstroke}%
\pgfsetdash{}{0pt}%
\pgfpathmoveto{\pgfqpoint{3.564492in}{2.541257in}}%
\pgfpathlineto{\pgfqpoint{3.816789in}{2.677291in}}%
\pgfpathlineto{\pgfqpoint{3.564492in}{2.541257in}}%
\pgfpathclose%
\pgfusepath{fill}%
\end{pgfscope}%
\begin{pgfscope}%
\pgfpathrectangle{\pgfqpoint{0.017500in}{0.017500in}}{\pgfqpoint{6.218282in}{3.580372in}} %
\pgfusepath{clip}%
\pgfsetbuttcap%
\pgfsetroundjoin%
\definecolor{currentfill}{rgb}{0.763520,0.178667,0.193396}%
\pgfsetfillcolor{currentfill}%
\pgfsetlinewidth{0.000000pt}%
\definecolor{currentstroke}{rgb}{0.000000,0.000000,0.000000}%
\pgfsetstrokecolor{currentstroke}%
\pgfsetdash{}{0pt}%
\pgfpathmoveto{\pgfqpoint{3.564492in}{2.541257in}}%
\pgfpathlineto{\pgfqpoint{3.816789in}{2.677291in}}%
\pgfpathlineto{\pgfqpoint{3.564492in}{2.541257in}}%
\pgfpathclose%
\pgfusepath{fill}%
\end{pgfscope}%
\begin{pgfscope}%
\pgfpathrectangle{\pgfqpoint{0.017500in}{0.017500in}}{\pgfqpoint{6.218282in}{3.580372in}} %
\pgfusepath{clip}%
\pgfsetbuttcap%
\pgfsetroundjoin%
\definecolor{currentfill}{rgb}{0.763520,0.178667,0.193396}%
\pgfsetfillcolor{currentfill}%
\pgfsetlinewidth{0.000000pt}%
\definecolor{currentstroke}{rgb}{0.000000,0.000000,0.000000}%
\pgfsetstrokecolor{currentstroke}%
\pgfsetdash{}{0pt}%
\pgfpathmoveto{\pgfqpoint{3.564492in}{2.541257in}}%
\pgfpathlineto{\pgfqpoint{3.816789in}{2.677291in}}%
\pgfpathlineto{\pgfqpoint{3.564492in}{2.541257in}}%
\pgfpathclose%
\pgfusepath{fill}%
\end{pgfscope}%
\begin{pgfscope}%
\pgfpathrectangle{\pgfqpoint{0.017500in}{0.017500in}}{\pgfqpoint{6.218282in}{3.580372in}} %
\pgfusepath{clip}%
\pgfsetbuttcap%
\pgfsetroundjoin%
\definecolor{currentfill}{rgb}{0.763520,0.178667,0.193396}%
\pgfsetfillcolor{currentfill}%
\pgfsetlinewidth{0.000000pt}%
\definecolor{currentstroke}{rgb}{0.000000,0.000000,0.000000}%
\pgfsetstrokecolor{currentstroke}%
\pgfsetdash{}{0pt}%
\pgfpathmoveto{\pgfqpoint{3.564492in}{2.541257in}}%
\pgfpathlineto{\pgfqpoint{3.816789in}{2.677291in}}%
\pgfpathlineto{\pgfqpoint{3.564492in}{2.541257in}}%
\pgfpathclose%
\pgfusepath{fill}%
\end{pgfscope}%
\begin{pgfscope}%
\pgfpathrectangle{\pgfqpoint{0.017500in}{0.017500in}}{\pgfqpoint{6.218282in}{3.580372in}} %
\pgfusepath{clip}%
\pgfsetbuttcap%
\pgfsetroundjoin%
\definecolor{currentfill}{rgb}{0.763520,0.178667,0.193396}%
\pgfsetfillcolor{currentfill}%
\pgfsetlinewidth{0.000000pt}%
\definecolor{currentstroke}{rgb}{0.000000,0.000000,0.000000}%
\pgfsetstrokecolor{currentstroke}%
\pgfsetdash{}{0pt}%
\pgfpathmoveto{\pgfqpoint{3.564492in}{2.541257in}}%
\pgfpathlineto{\pgfqpoint{3.816789in}{2.677291in}}%
\pgfpathlineto{\pgfqpoint{3.564492in}{2.541257in}}%
\pgfpathclose%
\pgfusepath{fill}%
\end{pgfscope}%
\begin{pgfscope}%
\pgfpathrectangle{\pgfqpoint{0.017500in}{0.017500in}}{\pgfqpoint{6.218282in}{3.580372in}} %
\pgfusepath{clip}%
\pgfsetbuttcap%
\pgfsetroundjoin%
\definecolor{currentfill}{rgb}{0.763520,0.178667,0.193396}%
\pgfsetfillcolor{currentfill}%
\pgfsetlinewidth{0.000000pt}%
\definecolor{currentstroke}{rgb}{0.000000,0.000000,0.000000}%
\pgfsetstrokecolor{currentstroke}%
\pgfsetdash{}{0pt}%
\pgfpathmoveto{\pgfqpoint{3.564492in}{2.541257in}}%
\pgfpathlineto{\pgfqpoint{3.816789in}{2.677291in}}%
\pgfpathlineto{\pgfqpoint{3.564492in}{2.541257in}}%
\pgfpathclose%
\pgfusepath{fill}%
\end{pgfscope}%
\begin{pgfscope}%
\pgfpathrectangle{\pgfqpoint{0.017500in}{0.017500in}}{\pgfqpoint{6.218282in}{3.580372in}} %
\pgfusepath{clip}%
\pgfsetbuttcap%
\pgfsetroundjoin%
\definecolor{currentfill}{rgb}{0.763520,0.178667,0.193396}%
\pgfsetfillcolor{currentfill}%
\pgfsetlinewidth{0.000000pt}%
\definecolor{currentstroke}{rgb}{0.000000,0.000000,0.000000}%
\pgfsetstrokecolor{currentstroke}%
\pgfsetdash{}{0pt}%
\pgfpathmoveto{\pgfqpoint{3.564492in}{2.541257in}}%
\pgfpathlineto{\pgfqpoint{3.816789in}{2.677291in}}%
\pgfpathlineto{\pgfqpoint{3.564492in}{2.541257in}}%
\pgfpathclose%
\pgfusepath{fill}%
\end{pgfscope}%
\begin{pgfscope}%
\pgfpathrectangle{\pgfqpoint{0.017500in}{0.017500in}}{\pgfqpoint{6.218282in}{3.580372in}} %
\pgfusepath{clip}%
\pgfsetbuttcap%
\pgfsetroundjoin%
\definecolor{currentfill}{rgb}{0.763520,0.178667,0.193396}%
\pgfsetfillcolor{currentfill}%
\pgfsetlinewidth{0.000000pt}%
\definecolor{currentstroke}{rgb}{0.000000,0.000000,0.000000}%
\pgfsetstrokecolor{currentstroke}%
\pgfsetdash{}{0pt}%
\pgfpathmoveto{\pgfqpoint{3.564492in}{2.541257in}}%
\pgfpathlineto{\pgfqpoint{3.816789in}{2.677291in}}%
\pgfpathlineto{\pgfqpoint{3.564492in}{2.541257in}}%
\pgfpathclose%
\pgfusepath{fill}%
\end{pgfscope}%
\begin{pgfscope}%
\pgfpathrectangle{\pgfqpoint{0.017500in}{0.017500in}}{\pgfqpoint{6.218282in}{3.580372in}} %
\pgfusepath{clip}%
\pgfsetbuttcap%
\pgfsetroundjoin%
\definecolor{currentfill}{rgb}{0.763520,0.178667,0.193396}%
\pgfsetfillcolor{currentfill}%
\pgfsetlinewidth{0.000000pt}%
\definecolor{currentstroke}{rgb}{0.000000,0.000000,0.000000}%
\pgfsetstrokecolor{currentstroke}%
\pgfsetdash{}{0pt}%
\pgfpathmoveto{\pgfqpoint{3.564492in}{2.541257in}}%
\pgfpathlineto{\pgfqpoint{3.816789in}{2.677291in}}%
\pgfpathlineto{\pgfqpoint{3.564492in}{2.541257in}}%
\pgfpathclose%
\pgfusepath{fill}%
\end{pgfscope}%
\begin{pgfscope}%
\pgfpathrectangle{\pgfqpoint{0.017500in}{0.017500in}}{\pgfqpoint{6.218282in}{3.580372in}} %
\pgfusepath{clip}%
\pgfsetbuttcap%
\pgfsetroundjoin%
\definecolor{currentfill}{rgb}{0.763520,0.178667,0.193396}%
\pgfsetfillcolor{currentfill}%
\pgfsetlinewidth{0.000000pt}%
\definecolor{currentstroke}{rgb}{0.000000,0.000000,0.000000}%
\pgfsetstrokecolor{currentstroke}%
\pgfsetdash{}{0pt}%
\pgfpathmoveto{\pgfqpoint{3.564492in}{2.541257in}}%
\pgfpathlineto{\pgfqpoint{3.816789in}{2.677291in}}%
\pgfpathlineto{\pgfqpoint{3.564492in}{2.541257in}}%
\pgfpathclose%
\pgfusepath{fill}%
\end{pgfscope}%
\begin{pgfscope}%
\pgfpathrectangle{\pgfqpoint{0.017500in}{0.017500in}}{\pgfqpoint{6.218282in}{3.580372in}} %
\pgfusepath{clip}%
\pgfsetbuttcap%
\pgfsetroundjoin%
\definecolor{currentfill}{rgb}{0.763520,0.178667,0.193396}%
\pgfsetfillcolor{currentfill}%
\pgfsetlinewidth{0.000000pt}%
\definecolor{currentstroke}{rgb}{0.000000,0.000000,0.000000}%
\pgfsetstrokecolor{currentstroke}%
\pgfsetdash{}{0pt}%
\pgfpathmoveto{\pgfqpoint{3.564492in}{2.541257in}}%
\pgfpathlineto{\pgfqpoint{3.816789in}{2.677291in}}%
\pgfpathlineto{\pgfqpoint{3.564492in}{2.541257in}}%
\pgfpathclose%
\pgfusepath{fill}%
\end{pgfscope}%
\begin{pgfscope}%
\pgfpathrectangle{\pgfqpoint{0.017500in}{0.017500in}}{\pgfqpoint{6.218282in}{3.580372in}} %
\pgfusepath{clip}%
\pgfsetbuttcap%
\pgfsetroundjoin%
\definecolor{currentfill}{rgb}{0.899543,0.847500,0.817789}%
\pgfsetfillcolor{currentfill}%
\pgfsetlinewidth{0.000000pt}%
\definecolor{currentstroke}{rgb}{0.000000,0.000000,0.000000}%
\pgfsetstrokecolor{currentstroke}%
\pgfsetdash{}{0pt}%
\pgfpathmoveto{\pgfqpoint{4.964811in}{2.684961in}}%
\pgfpathlineto{\pgfqpoint{3.688351in}{0.641693in}}%
\pgfpathlineto{\pgfqpoint{4.964811in}{2.684961in}}%
\pgfpathclose%
\pgfusepath{fill}%
\end{pgfscope}%
\begin{pgfscope}%
\pgfpathrectangle{\pgfqpoint{0.017500in}{0.017500in}}{\pgfqpoint{6.218282in}{3.580372in}} %
\pgfusepath{clip}%
\pgfsetbuttcap%
\pgfsetroundjoin%
\definecolor{currentfill}{rgb}{0.899543,0.847500,0.817789}%
\pgfsetfillcolor{currentfill}%
\pgfsetlinewidth{0.000000pt}%
\definecolor{currentstroke}{rgb}{0.000000,0.000000,0.000000}%
\pgfsetstrokecolor{currentstroke}%
\pgfsetdash{}{0pt}%
\pgfpathmoveto{\pgfqpoint{4.964811in}{2.684961in}}%
\pgfpathlineto{\pgfqpoint{3.688351in}{0.641693in}}%
\pgfpathlineto{\pgfqpoint{4.964811in}{2.684961in}}%
\pgfpathclose%
\pgfusepath{fill}%
\end{pgfscope}%
\begin{pgfscope}%
\pgfpathrectangle{\pgfqpoint{0.017500in}{0.017500in}}{\pgfqpoint{6.218282in}{3.580372in}} %
\pgfusepath{clip}%
\pgfsetbuttcap%
\pgfsetroundjoin%
\definecolor{currentfill}{rgb}{0.899543,0.847500,0.817789}%
\pgfsetfillcolor{currentfill}%
\pgfsetlinewidth{0.000000pt}%
\definecolor{currentstroke}{rgb}{0.000000,0.000000,0.000000}%
\pgfsetstrokecolor{currentstroke}%
\pgfsetdash{}{0pt}%
\pgfpathmoveto{\pgfqpoint{4.964811in}{2.684961in}}%
\pgfpathlineto{\pgfqpoint{3.688351in}{0.641693in}}%
\pgfpathlineto{\pgfqpoint{4.964811in}{2.684961in}}%
\pgfpathclose%
\pgfusepath{fill}%
\end{pgfscope}%
\begin{pgfscope}%
\pgfpathrectangle{\pgfqpoint{0.017500in}{0.017500in}}{\pgfqpoint{6.218282in}{3.580372in}} %
\pgfusepath{clip}%
\pgfsetbuttcap%
\pgfsetroundjoin%
\definecolor{currentfill}{rgb}{0.899543,0.847500,0.817789}%
\pgfsetfillcolor{currentfill}%
\pgfsetlinewidth{0.000000pt}%
\definecolor{currentstroke}{rgb}{0.000000,0.000000,0.000000}%
\pgfsetstrokecolor{currentstroke}%
\pgfsetdash{}{0pt}%
\pgfpathmoveto{\pgfqpoint{4.964811in}{2.684961in}}%
\pgfpathlineto{\pgfqpoint{3.688351in}{0.641693in}}%
\pgfpathlineto{\pgfqpoint{4.964811in}{2.684961in}}%
\pgfpathclose%
\pgfusepath{fill}%
\end{pgfscope}%
\begin{pgfscope}%
\pgfpathrectangle{\pgfqpoint{0.017500in}{0.017500in}}{\pgfqpoint{6.218282in}{3.580372in}} %
\pgfusepath{clip}%
\pgfsetbuttcap%
\pgfsetroundjoin%
\definecolor{currentfill}{rgb}{0.899543,0.847500,0.817789}%
\pgfsetfillcolor{currentfill}%
\pgfsetlinewidth{0.000000pt}%
\definecolor{currentstroke}{rgb}{0.000000,0.000000,0.000000}%
\pgfsetstrokecolor{currentstroke}%
\pgfsetdash{}{0pt}%
\pgfpathmoveto{\pgfqpoint{4.964811in}{2.684961in}}%
\pgfpathlineto{\pgfqpoint{3.688351in}{0.641693in}}%
\pgfpathlineto{\pgfqpoint{4.964811in}{2.684961in}}%
\pgfpathclose%
\pgfusepath{fill}%
\end{pgfscope}%
\begin{pgfscope}%
\pgfpathrectangle{\pgfqpoint{0.017500in}{0.017500in}}{\pgfqpoint{6.218282in}{3.580372in}} %
\pgfusepath{clip}%
\pgfsetbuttcap%
\pgfsetroundjoin%
\definecolor{currentfill}{rgb}{0.899543,0.847500,0.817789}%
\pgfsetfillcolor{currentfill}%
\pgfsetlinewidth{0.000000pt}%
\definecolor{currentstroke}{rgb}{0.000000,0.000000,0.000000}%
\pgfsetstrokecolor{currentstroke}%
\pgfsetdash{}{0pt}%
\pgfpathmoveto{\pgfqpoint{4.964811in}{2.684961in}}%
\pgfpathlineto{\pgfqpoint{3.688351in}{0.641693in}}%
\pgfpathlineto{\pgfqpoint{4.964811in}{2.684961in}}%
\pgfpathclose%
\pgfusepath{fill}%
\end{pgfscope}%
\begin{pgfscope}%
\pgfpathrectangle{\pgfqpoint{0.017500in}{0.017500in}}{\pgfqpoint{6.218282in}{3.580372in}} %
\pgfusepath{clip}%
\pgfsetbuttcap%
\pgfsetroundjoin%
\definecolor{currentfill}{rgb}{0.899543,0.847500,0.817789}%
\pgfsetfillcolor{currentfill}%
\pgfsetlinewidth{0.000000pt}%
\definecolor{currentstroke}{rgb}{0.000000,0.000000,0.000000}%
\pgfsetstrokecolor{currentstroke}%
\pgfsetdash{}{0pt}%
\pgfpathmoveto{\pgfqpoint{4.964811in}{2.684961in}}%
\pgfpathlineto{\pgfqpoint{3.688351in}{0.641693in}}%
\pgfpathlineto{\pgfqpoint{4.964811in}{2.684961in}}%
\pgfpathclose%
\pgfusepath{fill}%
\end{pgfscope}%
\begin{pgfscope}%
\pgfpathrectangle{\pgfqpoint{0.017500in}{0.017500in}}{\pgfqpoint{6.218282in}{3.580372in}} %
\pgfusepath{clip}%
\pgfsetbuttcap%
\pgfsetroundjoin%
\definecolor{currentfill}{rgb}{0.899543,0.847500,0.817789}%
\pgfsetfillcolor{currentfill}%
\pgfsetlinewidth{0.000000pt}%
\definecolor{currentstroke}{rgb}{0.000000,0.000000,0.000000}%
\pgfsetstrokecolor{currentstroke}%
\pgfsetdash{}{0pt}%
\pgfpathmoveto{\pgfqpoint{4.964811in}{2.684961in}}%
\pgfpathlineto{\pgfqpoint{3.688351in}{0.641693in}}%
\pgfpathlineto{\pgfqpoint{4.964811in}{2.684961in}}%
\pgfpathclose%
\pgfusepath{fill}%
\end{pgfscope}%
\begin{pgfscope}%
\pgfpathrectangle{\pgfqpoint{0.017500in}{0.017500in}}{\pgfqpoint{6.218282in}{3.580372in}} %
\pgfusepath{clip}%
\pgfsetbuttcap%
\pgfsetroundjoin%
\definecolor{currentfill}{rgb}{0.899543,0.847500,0.817789}%
\pgfsetfillcolor{currentfill}%
\pgfsetlinewidth{0.000000pt}%
\definecolor{currentstroke}{rgb}{0.000000,0.000000,0.000000}%
\pgfsetstrokecolor{currentstroke}%
\pgfsetdash{}{0pt}%
\pgfpathmoveto{\pgfqpoint{4.964811in}{2.684961in}}%
\pgfpathlineto{\pgfqpoint{3.688351in}{0.641693in}}%
\pgfpathlineto{\pgfqpoint{4.964811in}{2.684961in}}%
\pgfpathclose%
\pgfusepath{fill}%
\end{pgfscope}%
\begin{pgfscope}%
\pgfpathrectangle{\pgfqpoint{0.017500in}{0.017500in}}{\pgfqpoint{6.218282in}{3.580372in}} %
\pgfusepath{clip}%
\pgfsetbuttcap%
\pgfsetroundjoin%
\definecolor{currentfill}{rgb}{0.899543,0.847500,0.817789}%
\pgfsetfillcolor{currentfill}%
\pgfsetlinewidth{0.000000pt}%
\definecolor{currentstroke}{rgb}{0.000000,0.000000,0.000000}%
\pgfsetstrokecolor{currentstroke}%
\pgfsetdash{}{0pt}%
\pgfpathmoveto{\pgfqpoint{4.964811in}{2.684961in}}%
\pgfpathlineto{\pgfqpoint{3.688351in}{0.641693in}}%
\pgfpathlineto{\pgfqpoint{4.964811in}{2.684961in}}%
\pgfpathclose%
\pgfusepath{fill}%
\end{pgfscope}%
\begin{pgfscope}%
\pgfpathrectangle{\pgfqpoint{0.017500in}{0.017500in}}{\pgfqpoint{6.218282in}{3.580372in}} %
\pgfusepath{clip}%
\pgfsetbuttcap%
\pgfsetroundjoin%
\definecolor{currentfill}{rgb}{0.899543,0.847500,0.817789}%
\pgfsetfillcolor{currentfill}%
\pgfsetlinewidth{0.000000pt}%
\definecolor{currentstroke}{rgb}{0.000000,0.000000,0.000000}%
\pgfsetstrokecolor{currentstroke}%
\pgfsetdash{}{0pt}%
\pgfpathmoveto{\pgfqpoint{4.964811in}{2.684961in}}%
\pgfpathlineto{\pgfqpoint{3.688351in}{0.641693in}}%
\pgfpathlineto{\pgfqpoint{4.964811in}{2.684961in}}%
\pgfpathclose%
\pgfusepath{fill}%
\end{pgfscope}%
\begin{pgfscope}%
\pgfpathrectangle{\pgfqpoint{0.017500in}{0.017500in}}{\pgfqpoint{6.218282in}{3.580372in}} %
\pgfusepath{clip}%
\pgfsetbuttcap%
\pgfsetroundjoin%
\definecolor{currentfill}{rgb}{0.899543,0.847500,0.817789}%
\pgfsetfillcolor{currentfill}%
\pgfsetlinewidth{0.000000pt}%
\definecolor{currentstroke}{rgb}{0.000000,0.000000,0.000000}%
\pgfsetstrokecolor{currentstroke}%
\pgfsetdash{}{0pt}%
\pgfpathmoveto{\pgfqpoint{4.964811in}{2.684961in}}%
\pgfpathlineto{\pgfqpoint{3.688351in}{0.641693in}}%
\pgfpathlineto{\pgfqpoint{4.964811in}{2.684961in}}%
\pgfpathclose%
\pgfusepath{fill}%
\end{pgfscope}%
\begin{pgfscope}%
\pgfpathrectangle{\pgfqpoint{0.017500in}{0.017500in}}{\pgfqpoint{6.218282in}{3.580372in}} %
\pgfusepath{clip}%
\pgfsetbuttcap%
\pgfsetroundjoin%
\definecolor{currentfill}{rgb}{0.899543,0.847500,0.817789}%
\pgfsetfillcolor{currentfill}%
\pgfsetlinewidth{0.000000pt}%
\definecolor{currentstroke}{rgb}{0.000000,0.000000,0.000000}%
\pgfsetstrokecolor{currentstroke}%
\pgfsetdash{}{0pt}%
\pgfpathmoveto{\pgfqpoint{4.964811in}{2.684961in}}%
\pgfpathlineto{\pgfqpoint{3.688351in}{0.641693in}}%
\pgfpathlineto{\pgfqpoint{4.964811in}{2.684961in}}%
\pgfpathclose%
\pgfusepath{fill}%
\end{pgfscope}%
\begin{pgfscope}%
\pgfpathrectangle{\pgfqpoint{0.017500in}{0.017500in}}{\pgfqpoint{6.218282in}{3.580372in}} %
\pgfusepath{clip}%
\pgfsetbuttcap%
\pgfsetroundjoin%
\definecolor{currentfill}{rgb}{0.899543,0.847500,0.817789}%
\pgfsetfillcolor{currentfill}%
\pgfsetlinewidth{0.000000pt}%
\definecolor{currentstroke}{rgb}{0.000000,0.000000,0.000000}%
\pgfsetstrokecolor{currentstroke}%
\pgfsetdash{}{0pt}%
\pgfpathmoveto{\pgfqpoint{4.964811in}{2.684961in}}%
\pgfpathlineto{\pgfqpoint{3.688351in}{0.641693in}}%
\pgfpathlineto{\pgfqpoint{4.964811in}{2.684961in}}%
\pgfpathclose%
\pgfusepath{fill}%
\end{pgfscope}%
\begin{pgfscope}%
\pgfpathrectangle{\pgfqpoint{0.017500in}{0.017500in}}{\pgfqpoint{6.218282in}{3.580372in}} %
\pgfusepath{clip}%
\pgfsetbuttcap%
\pgfsetroundjoin%
\definecolor{currentfill}{rgb}{0.899543,0.847500,0.817789}%
\pgfsetfillcolor{currentfill}%
\pgfsetlinewidth{0.000000pt}%
\definecolor{currentstroke}{rgb}{0.000000,0.000000,0.000000}%
\pgfsetstrokecolor{currentstroke}%
\pgfsetdash{}{0pt}%
\pgfpathmoveto{\pgfqpoint{4.964811in}{2.684961in}}%
\pgfpathlineto{\pgfqpoint{3.688351in}{0.641693in}}%
\pgfpathlineto{\pgfqpoint{4.964811in}{2.684961in}}%
\pgfpathclose%
\pgfusepath{fill}%
\end{pgfscope}%
\begin{pgfscope}%
\pgfpathrectangle{\pgfqpoint{0.017500in}{0.017500in}}{\pgfqpoint{6.218282in}{3.580372in}} %
\pgfusepath{clip}%
\pgfsetbuttcap%
\pgfsetroundjoin%
\definecolor{currentfill}{rgb}{0.899543,0.847500,0.817789}%
\pgfsetfillcolor{currentfill}%
\pgfsetlinewidth{0.000000pt}%
\definecolor{currentstroke}{rgb}{0.000000,0.000000,0.000000}%
\pgfsetstrokecolor{currentstroke}%
\pgfsetdash{}{0pt}%
\pgfpathmoveto{\pgfqpoint{4.964811in}{2.684961in}}%
\pgfpathlineto{\pgfqpoint{3.688351in}{0.641693in}}%
\pgfpathlineto{\pgfqpoint{4.964811in}{2.684961in}}%
\pgfpathclose%
\pgfusepath{fill}%
\end{pgfscope}%
\begin{pgfscope}%
\pgfpathrectangle{\pgfqpoint{0.017500in}{0.017500in}}{\pgfqpoint{6.218282in}{3.580372in}} %
\pgfusepath{clip}%
\pgfsetbuttcap%
\pgfsetroundjoin%
\definecolor{currentfill}{rgb}{0.899543,0.847500,0.817789}%
\pgfsetfillcolor{currentfill}%
\pgfsetlinewidth{0.000000pt}%
\definecolor{currentstroke}{rgb}{0.000000,0.000000,0.000000}%
\pgfsetstrokecolor{currentstroke}%
\pgfsetdash{}{0pt}%
\pgfpathmoveto{\pgfqpoint{4.964811in}{2.684961in}}%
\pgfpathlineto{\pgfqpoint{3.688351in}{0.641693in}}%
\pgfpathlineto{\pgfqpoint{4.964811in}{2.684961in}}%
\pgfpathclose%
\pgfusepath{fill}%
\end{pgfscope}%
\begin{pgfscope}%
\pgfpathrectangle{\pgfqpoint{0.017500in}{0.017500in}}{\pgfqpoint{6.218282in}{3.580372in}} %
\pgfusepath{clip}%
\pgfsetbuttcap%
\pgfsetroundjoin%
\definecolor{currentfill}{rgb}{0.899543,0.847500,0.817789}%
\pgfsetfillcolor{currentfill}%
\pgfsetlinewidth{0.000000pt}%
\definecolor{currentstroke}{rgb}{0.000000,0.000000,0.000000}%
\pgfsetstrokecolor{currentstroke}%
\pgfsetdash{}{0pt}%
\pgfpathmoveto{\pgfqpoint{4.964811in}{2.684961in}}%
\pgfpathlineto{\pgfqpoint{3.688351in}{0.641693in}}%
\pgfpathlineto{\pgfqpoint{4.964811in}{2.684961in}}%
\pgfpathclose%
\pgfusepath{fill}%
\end{pgfscope}%
\begin{pgfscope}%
\pgfpathrectangle{\pgfqpoint{0.017500in}{0.017500in}}{\pgfqpoint{6.218282in}{3.580372in}} %
\pgfusepath{clip}%
\pgfsetbuttcap%
\pgfsetroundjoin%
\definecolor{currentfill}{rgb}{0.899543,0.847500,0.817789}%
\pgfsetfillcolor{currentfill}%
\pgfsetlinewidth{0.000000pt}%
\definecolor{currentstroke}{rgb}{0.000000,0.000000,0.000000}%
\pgfsetstrokecolor{currentstroke}%
\pgfsetdash{}{0pt}%
\pgfpathmoveto{\pgfqpoint{4.964811in}{2.684961in}}%
\pgfpathlineto{\pgfqpoint{3.688351in}{0.641693in}}%
\pgfpathlineto{\pgfqpoint{4.964811in}{2.684961in}}%
\pgfpathclose%
\pgfusepath{fill}%
\end{pgfscope}%
\begin{pgfscope}%
\pgfpathrectangle{\pgfqpoint{0.017500in}{0.017500in}}{\pgfqpoint{6.218282in}{3.580372in}} %
\pgfusepath{clip}%
\pgfsetbuttcap%
\pgfsetroundjoin%
\definecolor{currentfill}{rgb}{0.899543,0.847500,0.817789}%
\pgfsetfillcolor{currentfill}%
\pgfsetlinewidth{0.000000pt}%
\definecolor{currentstroke}{rgb}{0.000000,0.000000,0.000000}%
\pgfsetstrokecolor{currentstroke}%
\pgfsetdash{}{0pt}%
\pgfpathmoveto{\pgfqpoint{4.964811in}{2.684961in}}%
\pgfpathlineto{\pgfqpoint{3.688351in}{0.641693in}}%
\pgfpathlineto{\pgfqpoint{4.964811in}{2.684961in}}%
\pgfpathclose%
\pgfusepath{fill}%
\end{pgfscope}%
\begin{pgfscope}%
\pgfpathrectangle{\pgfqpoint{0.017500in}{0.017500in}}{\pgfqpoint{6.218282in}{3.580372in}} %
\pgfusepath{clip}%
\pgfsetbuttcap%
\pgfsetroundjoin%
\definecolor{currentfill}{rgb}{0.899543,0.847500,0.817789}%
\pgfsetfillcolor{currentfill}%
\pgfsetlinewidth{0.000000pt}%
\definecolor{currentstroke}{rgb}{0.000000,0.000000,0.000000}%
\pgfsetstrokecolor{currentstroke}%
\pgfsetdash{}{0pt}%
\pgfpathmoveto{\pgfqpoint{4.964811in}{2.684961in}}%
\pgfpathlineto{\pgfqpoint{3.688351in}{0.641693in}}%
\pgfpathlineto{\pgfqpoint{4.964811in}{2.684961in}}%
\pgfpathclose%
\pgfusepath{fill}%
\end{pgfscope}%
\begin{pgfscope}%
\pgfpathrectangle{\pgfqpoint{0.017500in}{0.017500in}}{\pgfqpoint{6.218282in}{3.580372in}} %
\pgfusepath{clip}%
\pgfsetbuttcap%
\pgfsetroundjoin%
\definecolor{currentfill}{rgb}{0.899543,0.847500,0.817789}%
\pgfsetfillcolor{currentfill}%
\pgfsetlinewidth{0.000000pt}%
\definecolor{currentstroke}{rgb}{0.000000,0.000000,0.000000}%
\pgfsetstrokecolor{currentstroke}%
\pgfsetdash{}{0pt}%
\pgfpathmoveto{\pgfqpoint{4.964811in}{2.684961in}}%
\pgfpathlineto{\pgfqpoint{3.688351in}{0.641693in}}%
\pgfpathlineto{\pgfqpoint{4.964811in}{2.684961in}}%
\pgfpathclose%
\pgfusepath{fill}%
\end{pgfscope}%
\begin{pgfscope}%
\pgfpathrectangle{\pgfqpoint{0.017500in}{0.017500in}}{\pgfqpoint{6.218282in}{3.580372in}} %
\pgfusepath{clip}%
\pgfsetbuttcap%
\pgfsetroundjoin%
\definecolor{currentfill}{rgb}{0.899543,0.847500,0.817789}%
\pgfsetfillcolor{currentfill}%
\pgfsetlinewidth{0.000000pt}%
\definecolor{currentstroke}{rgb}{0.000000,0.000000,0.000000}%
\pgfsetstrokecolor{currentstroke}%
\pgfsetdash{}{0pt}%
\pgfpathmoveto{\pgfqpoint{4.964811in}{2.684961in}}%
\pgfpathlineto{\pgfqpoint{3.688351in}{0.641693in}}%
\pgfpathlineto{\pgfqpoint{4.964811in}{2.684961in}}%
\pgfpathclose%
\pgfusepath{fill}%
\end{pgfscope}%
\begin{pgfscope}%
\pgfpathrectangle{\pgfqpoint{0.017500in}{0.017500in}}{\pgfqpoint{6.218282in}{3.580372in}} %
\pgfusepath{clip}%
\pgfsetbuttcap%
\pgfsetroundjoin%
\definecolor{currentfill}{rgb}{0.899543,0.847500,0.817789}%
\pgfsetfillcolor{currentfill}%
\pgfsetlinewidth{0.000000pt}%
\definecolor{currentstroke}{rgb}{0.000000,0.000000,0.000000}%
\pgfsetstrokecolor{currentstroke}%
\pgfsetdash{}{0pt}%
\pgfpathmoveto{\pgfqpoint{4.964811in}{2.684961in}}%
\pgfpathlineto{\pgfqpoint{3.688351in}{0.641693in}}%
\pgfpathlineto{\pgfqpoint{4.964811in}{2.684961in}}%
\pgfpathclose%
\pgfusepath{fill}%
\end{pgfscope}%
\begin{pgfscope}%
\pgfpathrectangle{\pgfqpoint{0.017500in}{0.017500in}}{\pgfqpoint{6.218282in}{3.580372in}} %
\pgfusepath{clip}%
\pgfsetbuttcap%
\pgfsetroundjoin%
\definecolor{currentfill}{rgb}{0.899543,0.847500,0.817789}%
\pgfsetfillcolor{currentfill}%
\pgfsetlinewidth{0.000000pt}%
\definecolor{currentstroke}{rgb}{0.000000,0.000000,0.000000}%
\pgfsetstrokecolor{currentstroke}%
\pgfsetdash{}{0pt}%
\pgfpathmoveto{\pgfqpoint{4.964811in}{2.684961in}}%
\pgfpathlineto{\pgfqpoint{3.688351in}{0.641693in}}%
\pgfpathlineto{\pgfqpoint{4.964811in}{2.684961in}}%
\pgfpathclose%
\pgfusepath{fill}%
\end{pgfscope}%
\begin{pgfscope}%
\pgfpathrectangle{\pgfqpoint{0.017500in}{0.017500in}}{\pgfqpoint{6.218282in}{3.580372in}} %
\pgfusepath{clip}%
\pgfsetbuttcap%
\pgfsetroundjoin%
\definecolor{currentfill}{rgb}{0.884643,0.410017,0.322507}%
\pgfsetfillcolor{currentfill}%
\pgfsetlinewidth{0.000000pt}%
\definecolor{currentstroke}{rgb}{0.000000,0.000000,0.000000}%
\pgfsetstrokecolor{currentstroke}%
\pgfsetdash{}{0pt}%
\pgfpathmoveto{\pgfqpoint{1.701778in}{2.375902in}}%
\pgfpathlineto{\pgfqpoint{2.143341in}{2.291288in}}%
\pgfpathlineto{\pgfqpoint{2.410872in}{2.508083in}}%
\pgfpathlineto{\pgfqpoint{1.971633in}{2.586781in}}%
\pgfpathlineto{\pgfqpoint{1.701778in}{2.375902in}}%
\pgfpathclose%
\pgfusepath{fill}%
\end{pgfscope}%
\begin{pgfscope}%
\pgfpathrectangle{\pgfqpoint{0.017500in}{0.017500in}}{\pgfqpoint{6.218282in}{3.580372in}} %
\pgfusepath{clip}%
\pgfsetbuttcap%
\pgfsetroundjoin%
\definecolor{currentfill}{rgb}{0.884643,0.410017,0.322507}%
\pgfsetfillcolor{currentfill}%
\pgfsetlinewidth{0.000000pt}%
\definecolor{currentstroke}{rgb}{0.000000,0.000000,0.000000}%
\pgfsetstrokecolor{currentstroke}%
\pgfsetdash{}{0pt}%
\pgfpathmoveto{\pgfqpoint{1.701778in}{2.375902in}}%
\pgfpathlineto{\pgfqpoint{2.143341in}{2.291288in}}%
\pgfpathlineto{\pgfqpoint{2.410872in}{2.508083in}}%
\pgfpathlineto{\pgfqpoint{1.971633in}{2.586781in}}%
\pgfpathlineto{\pgfqpoint{1.701778in}{2.375902in}}%
\pgfpathclose%
\pgfusepath{fill}%
\end{pgfscope}%
\begin{pgfscope}%
\pgfpathrectangle{\pgfqpoint{0.017500in}{0.017500in}}{\pgfqpoint{6.218282in}{3.580372in}} %
\pgfusepath{clip}%
\pgfsetbuttcap%
\pgfsetroundjoin%
\definecolor{currentfill}{rgb}{0.884643,0.410017,0.322507}%
\pgfsetfillcolor{currentfill}%
\pgfsetlinewidth{0.000000pt}%
\definecolor{currentstroke}{rgb}{0.000000,0.000000,0.000000}%
\pgfsetstrokecolor{currentstroke}%
\pgfsetdash{}{0pt}%
\pgfpathmoveto{\pgfqpoint{1.701778in}{2.375902in}}%
\pgfpathlineto{\pgfqpoint{2.143341in}{2.291288in}}%
\pgfpathlineto{\pgfqpoint{2.410872in}{2.508083in}}%
\pgfpathlineto{\pgfqpoint{1.971633in}{2.586781in}}%
\pgfpathlineto{\pgfqpoint{1.701778in}{2.375902in}}%
\pgfpathclose%
\pgfusepath{fill}%
\end{pgfscope}%
\begin{pgfscope}%
\pgfpathrectangle{\pgfqpoint{0.017500in}{0.017500in}}{\pgfqpoint{6.218282in}{3.580372in}} %
\pgfusepath{clip}%
\pgfsetbuttcap%
\pgfsetroundjoin%
\definecolor{currentfill}{rgb}{0.884643,0.410017,0.322507}%
\pgfsetfillcolor{currentfill}%
\pgfsetlinewidth{0.000000pt}%
\definecolor{currentstroke}{rgb}{0.000000,0.000000,0.000000}%
\pgfsetstrokecolor{currentstroke}%
\pgfsetdash{}{0pt}%
\pgfpathmoveto{\pgfqpoint{1.701778in}{2.375902in}}%
\pgfpathlineto{\pgfqpoint{2.143341in}{2.291288in}}%
\pgfpathlineto{\pgfqpoint{2.410872in}{2.508083in}}%
\pgfpathlineto{\pgfqpoint{1.971633in}{2.586781in}}%
\pgfpathlineto{\pgfqpoint{1.701778in}{2.375902in}}%
\pgfpathclose%
\pgfusepath{fill}%
\end{pgfscope}%
\begin{pgfscope}%
\pgfpathrectangle{\pgfqpoint{0.017500in}{0.017500in}}{\pgfqpoint{6.218282in}{3.580372in}} %
\pgfusepath{clip}%
\pgfsetbuttcap%
\pgfsetroundjoin%
\definecolor{currentfill}{rgb}{0.884643,0.410017,0.322507}%
\pgfsetfillcolor{currentfill}%
\pgfsetlinewidth{0.000000pt}%
\definecolor{currentstroke}{rgb}{0.000000,0.000000,0.000000}%
\pgfsetstrokecolor{currentstroke}%
\pgfsetdash{}{0pt}%
\pgfpathmoveto{\pgfqpoint{1.701778in}{2.375902in}}%
\pgfpathlineto{\pgfqpoint{2.143341in}{2.291288in}}%
\pgfpathlineto{\pgfqpoint{2.410872in}{2.508083in}}%
\pgfpathlineto{\pgfqpoint{1.971633in}{2.586781in}}%
\pgfpathlineto{\pgfqpoint{1.701778in}{2.375902in}}%
\pgfpathclose%
\pgfusepath{fill}%
\end{pgfscope}%
\begin{pgfscope}%
\pgfpathrectangle{\pgfqpoint{0.017500in}{0.017500in}}{\pgfqpoint{6.218282in}{3.580372in}} %
\pgfusepath{clip}%
\pgfsetbuttcap%
\pgfsetroundjoin%
\definecolor{currentfill}{rgb}{0.884643,0.410017,0.322507}%
\pgfsetfillcolor{currentfill}%
\pgfsetlinewidth{0.000000pt}%
\definecolor{currentstroke}{rgb}{0.000000,0.000000,0.000000}%
\pgfsetstrokecolor{currentstroke}%
\pgfsetdash{}{0pt}%
\pgfpathmoveto{\pgfqpoint{1.701778in}{2.375902in}}%
\pgfpathlineto{\pgfqpoint{2.143341in}{2.291288in}}%
\pgfpathlineto{\pgfqpoint{2.410872in}{2.508083in}}%
\pgfpathlineto{\pgfqpoint{1.971633in}{2.586781in}}%
\pgfpathlineto{\pgfqpoint{1.701778in}{2.375902in}}%
\pgfpathclose%
\pgfusepath{fill}%
\end{pgfscope}%
\begin{pgfscope}%
\pgfpathrectangle{\pgfqpoint{0.017500in}{0.017500in}}{\pgfqpoint{6.218282in}{3.580372in}} %
\pgfusepath{clip}%
\pgfsetbuttcap%
\pgfsetroundjoin%
\definecolor{currentfill}{rgb}{0.758112,0.168122,0.188827}%
\pgfsetfillcolor{currentfill}%
\pgfsetlinewidth{0.000000pt}%
\definecolor{currentstroke}{rgb}{0.000000,0.000000,0.000000}%
\pgfsetstrokecolor{currentstroke}%
\pgfsetdash{}{0pt}%
\pgfpathmoveto{\pgfqpoint{3.564492in}{2.541257in}}%
\pgfpathlineto{\pgfqpoint{4.018447in}{2.468167in}}%
\pgfpathlineto{\pgfqpoint{4.267628in}{2.613520in}}%
\pgfpathlineto{\pgfqpoint{3.816789in}{2.677291in}}%
\pgfpathlineto{\pgfqpoint{3.564492in}{2.541257in}}%
\pgfpathclose%
\pgfusepath{fill}%
\end{pgfscope}%
\begin{pgfscope}%
\pgfpathrectangle{\pgfqpoint{0.017500in}{0.017500in}}{\pgfqpoint{6.218282in}{3.580372in}} %
\pgfusepath{clip}%
\pgfsetbuttcap%
\pgfsetroundjoin%
\definecolor{currentfill}{rgb}{0.758112,0.168122,0.188827}%
\pgfsetfillcolor{currentfill}%
\pgfsetlinewidth{0.000000pt}%
\definecolor{currentstroke}{rgb}{0.000000,0.000000,0.000000}%
\pgfsetstrokecolor{currentstroke}%
\pgfsetdash{}{0pt}%
\pgfpathmoveto{\pgfqpoint{3.564492in}{2.541257in}}%
\pgfpathlineto{\pgfqpoint{4.018447in}{2.468167in}}%
\pgfpathlineto{\pgfqpoint{4.267628in}{2.613520in}}%
\pgfpathlineto{\pgfqpoint{3.816789in}{2.677291in}}%
\pgfpathlineto{\pgfqpoint{3.564492in}{2.541257in}}%
\pgfpathclose%
\pgfusepath{fill}%
\end{pgfscope}%
\begin{pgfscope}%
\pgfpathrectangle{\pgfqpoint{0.017500in}{0.017500in}}{\pgfqpoint{6.218282in}{3.580372in}} %
\pgfusepath{clip}%
\pgfsetbuttcap%
\pgfsetroundjoin%
\definecolor{currentfill}{rgb}{0.758112,0.168122,0.188827}%
\pgfsetfillcolor{currentfill}%
\pgfsetlinewidth{0.000000pt}%
\definecolor{currentstroke}{rgb}{0.000000,0.000000,0.000000}%
\pgfsetstrokecolor{currentstroke}%
\pgfsetdash{}{0pt}%
\pgfpathmoveto{\pgfqpoint{3.564492in}{2.541257in}}%
\pgfpathlineto{\pgfqpoint{4.018447in}{2.468167in}}%
\pgfpathlineto{\pgfqpoint{4.267628in}{2.613520in}}%
\pgfpathlineto{\pgfqpoint{3.816789in}{2.677291in}}%
\pgfpathlineto{\pgfqpoint{3.564492in}{2.541257in}}%
\pgfpathclose%
\pgfusepath{fill}%
\end{pgfscope}%
\begin{pgfscope}%
\pgfpathrectangle{\pgfqpoint{0.017500in}{0.017500in}}{\pgfqpoint{6.218282in}{3.580372in}} %
\pgfusepath{clip}%
\pgfsetbuttcap%
\pgfsetroundjoin%
\definecolor{currentfill}{rgb}{0.758112,0.168122,0.188827}%
\pgfsetfillcolor{currentfill}%
\pgfsetlinewidth{0.000000pt}%
\definecolor{currentstroke}{rgb}{0.000000,0.000000,0.000000}%
\pgfsetstrokecolor{currentstroke}%
\pgfsetdash{}{0pt}%
\pgfpathmoveto{\pgfqpoint{3.564492in}{2.541257in}}%
\pgfpathlineto{\pgfqpoint{4.018447in}{2.468167in}}%
\pgfpathlineto{\pgfqpoint{4.267628in}{2.613520in}}%
\pgfpathlineto{\pgfqpoint{3.816789in}{2.677291in}}%
\pgfpathlineto{\pgfqpoint{3.564492in}{2.541257in}}%
\pgfpathclose%
\pgfusepath{fill}%
\end{pgfscope}%
\begin{pgfscope}%
\pgfpathrectangle{\pgfqpoint{0.017500in}{0.017500in}}{\pgfqpoint{6.218282in}{3.580372in}} %
\pgfusepath{clip}%
\pgfsetbuttcap%
\pgfsetroundjoin%
\definecolor{currentfill}{rgb}{0.758112,0.168122,0.188827}%
\pgfsetfillcolor{currentfill}%
\pgfsetlinewidth{0.000000pt}%
\definecolor{currentstroke}{rgb}{0.000000,0.000000,0.000000}%
\pgfsetstrokecolor{currentstroke}%
\pgfsetdash{}{0pt}%
\pgfpathmoveto{\pgfqpoint{3.564492in}{2.541257in}}%
\pgfpathlineto{\pgfqpoint{4.018447in}{2.468167in}}%
\pgfpathlineto{\pgfqpoint{4.267628in}{2.613520in}}%
\pgfpathlineto{\pgfqpoint{3.816789in}{2.677291in}}%
\pgfpathlineto{\pgfqpoint{3.564492in}{2.541257in}}%
\pgfpathclose%
\pgfusepath{fill}%
\end{pgfscope}%
\begin{pgfscope}%
\pgfpathrectangle{\pgfqpoint{0.017500in}{0.017500in}}{\pgfqpoint{6.218282in}{3.580372in}} %
\pgfusepath{clip}%
\pgfsetbuttcap%
\pgfsetroundjoin%
\definecolor{currentfill}{rgb}{0.758112,0.168122,0.188827}%
\pgfsetfillcolor{currentfill}%
\pgfsetlinewidth{0.000000pt}%
\definecolor{currentstroke}{rgb}{0.000000,0.000000,0.000000}%
\pgfsetstrokecolor{currentstroke}%
\pgfsetdash{}{0pt}%
\pgfpathmoveto{\pgfqpoint{3.564492in}{2.541257in}}%
\pgfpathlineto{\pgfqpoint{4.018447in}{2.468167in}}%
\pgfpathlineto{\pgfqpoint{4.267628in}{2.613520in}}%
\pgfpathlineto{\pgfqpoint{3.816789in}{2.677291in}}%
\pgfpathlineto{\pgfqpoint{3.564492in}{2.541257in}}%
\pgfpathclose%
\pgfusepath{fill}%
\end{pgfscope}%
\begin{pgfscope}%
\pgfpathrectangle{\pgfqpoint{0.017500in}{0.017500in}}{\pgfqpoint{6.218282in}{3.580372in}} %
\pgfusepath{clip}%
\pgfsetbuttcap%
\pgfsetroundjoin%
\definecolor{currentfill}{rgb}{0.723315,0.068898,0.162989}%
\pgfsetfillcolor{currentfill}%
\pgfsetlinewidth{0.000000pt}%
\definecolor{currentstroke}{rgb}{0.000000,0.000000,0.000000}%
\pgfsetstrokecolor{currentstroke}%
\pgfsetdash{}{0pt}%
\pgfpathmoveto{\pgfqpoint{4.724651in}{2.549289in}}%
\pgfpathlineto{\pgfqpoint{4.964811in}{2.684961in}}%
\pgfpathlineto{\pgfqpoint{4.724651in}{2.549289in}}%
\pgfpathclose%
\pgfusepath{fill}%
\end{pgfscope}%
\begin{pgfscope}%
\pgfpathrectangle{\pgfqpoint{0.017500in}{0.017500in}}{\pgfqpoint{6.218282in}{3.580372in}} %
\pgfusepath{clip}%
\pgfsetbuttcap%
\pgfsetroundjoin%
\definecolor{currentfill}{rgb}{0.723315,0.068898,0.162989}%
\pgfsetfillcolor{currentfill}%
\pgfsetlinewidth{0.000000pt}%
\definecolor{currentstroke}{rgb}{0.000000,0.000000,0.000000}%
\pgfsetstrokecolor{currentstroke}%
\pgfsetdash{}{0pt}%
\pgfpathmoveto{\pgfqpoint{4.724651in}{2.549289in}}%
\pgfpathlineto{\pgfqpoint{4.964811in}{2.684961in}}%
\pgfpathlineto{\pgfqpoint{4.724651in}{2.549289in}}%
\pgfpathclose%
\pgfusepath{fill}%
\end{pgfscope}%
\begin{pgfscope}%
\pgfpathrectangle{\pgfqpoint{0.017500in}{0.017500in}}{\pgfqpoint{6.218282in}{3.580372in}} %
\pgfusepath{clip}%
\pgfsetbuttcap%
\pgfsetroundjoin%
\definecolor{currentfill}{rgb}{0.723315,0.068898,0.162989}%
\pgfsetfillcolor{currentfill}%
\pgfsetlinewidth{0.000000pt}%
\definecolor{currentstroke}{rgb}{0.000000,0.000000,0.000000}%
\pgfsetstrokecolor{currentstroke}%
\pgfsetdash{}{0pt}%
\pgfpathmoveto{\pgfqpoint{4.724651in}{2.549289in}}%
\pgfpathlineto{\pgfqpoint{4.964811in}{2.684961in}}%
\pgfpathlineto{\pgfqpoint{4.724651in}{2.549289in}}%
\pgfpathclose%
\pgfusepath{fill}%
\end{pgfscope}%
\begin{pgfscope}%
\pgfpathrectangle{\pgfqpoint{0.017500in}{0.017500in}}{\pgfqpoint{6.218282in}{3.580372in}} %
\pgfusepath{clip}%
\pgfsetbuttcap%
\pgfsetroundjoin%
\definecolor{currentfill}{rgb}{0.723315,0.068898,0.162989}%
\pgfsetfillcolor{currentfill}%
\pgfsetlinewidth{0.000000pt}%
\definecolor{currentstroke}{rgb}{0.000000,0.000000,0.000000}%
\pgfsetstrokecolor{currentstroke}%
\pgfsetdash{}{0pt}%
\pgfpathmoveto{\pgfqpoint{4.724651in}{2.549289in}}%
\pgfpathlineto{\pgfqpoint{4.964811in}{2.684961in}}%
\pgfpathlineto{\pgfqpoint{4.724651in}{2.549289in}}%
\pgfpathclose%
\pgfusepath{fill}%
\end{pgfscope}%
\begin{pgfscope}%
\pgfpathrectangle{\pgfqpoint{0.017500in}{0.017500in}}{\pgfqpoint{6.218282in}{3.580372in}} %
\pgfusepath{clip}%
\pgfsetbuttcap%
\pgfsetroundjoin%
\definecolor{currentfill}{rgb}{0.723315,0.068898,0.162989}%
\pgfsetfillcolor{currentfill}%
\pgfsetlinewidth{0.000000pt}%
\definecolor{currentstroke}{rgb}{0.000000,0.000000,0.000000}%
\pgfsetstrokecolor{currentstroke}%
\pgfsetdash{}{0pt}%
\pgfpathmoveto{\pgfqpoint{4.724651in}{2.549289in}}%
\pgfpathlineto{\pgfqpoint{4.964811in}{2.684961in}}%
\pgfpathlineto{\pgfqpoint{4.724651in}{2.549289in}}%
\pgfpathclose%
\pgfusepath{fill}%
\end{pgfscope}%
\begin{pgfscope}%
\pgfpathrectangle{\pgfqpoint{0.017500in}{0.017500in}}{\pgfqpoint{6.218282in}{3.580372in}} %
\pgfusepath{clip}%
\pgfsetbuttcap%
\pgfsetroundjoin%
\definecolor{currentfill}{rgb}{0.723315,0.068898,0.162989}%
\pgfsetfillcolor{currentfill}%
\pgfsetlinewidth{0.000000pt}%
\definecolor{currentstroke}{rgb}{0.000000,0.000000,0.000000}%
\pgfsetstrokecolor{currentstroke}%
\pgfsetdash{}{0pt}%
\pgfpathmoveto{\pgfqpoint{4.724651in}{2.549289in}}%
\pgfpathlineto{\pgfqpoint{4.964811in}{2.684961in}}%
\pgfpathlineto{\pgfqpoint{4.724651in}{2.549289in}}%
\pgfpathclose%
\pgfusepath{fill}%
\end{pgfscope}%
\begin{pgfscope}%
\pgfpathrectangle{\pgfqpoint{0.017500in}{0.017500in}}{\pgfqpoint{6.218282in}{3.580372in}} %
\pgfusepath{clip}%
\pgfsetbuttcap%
\pgfsetroundjoin%
\definecolor{currentfill}{rgb}{0.723315,0.068898,0.162989}%
\pgfsetfillcolor{currentfill}%
\pgfsetlinewidth{0.000000pt}%
\definecolor{currentstroke}{rgb}{0.000000,0.000000,0.000000}%
\pgfsetstrokecolor{currentstroke}%
\pgfsetdash{}{0pt}%
\pgfpathmoveto{\pgfqpoint{4.724651in}{2.549289in}}%
\pgfpathlineto{\pgfqpoint{4.964811in}{2.684961in}}%
\pgfpathlineto{\pgfqpoint{4.724651in}{2.549289in}}%
\pgfpathclose%
\pgfusepath{fill}%
\end{pgfscope}%
\begin{pgfscope}%
\pgfpathrectangle{\pgfqpoint{0.017500in}{0.017500in}}{\pgfqpoint{6.218282in}{3.580372in}} %
\pgfusepath{clip}%
\pgfsetbuttcap%
\pgfsetroundjoin%
\definecolor{currentfill}{rgb}{0.723315,0.068898,0.162989}%
\pgfsetfillcolor{currentfill}%
\pgfsetlinewidth{0.000000pt}%
\definecolor{currentstroke}{rgb}{0.000000,0.000000,0.000000}%
\pgfsetstrokecolor{currentstroke}%
\pgfsetdash{}{0pt}%
\pgfpathmoveto{\pgfqpoint{4.724651in}{2.549289in}}%
\pgfpathlineto{\pgfqpoint{4.964811in}{2.684961in}}%
\pgfpathlineto{\pgfqpoint{4.724651in}{2.549289in}}%
\pgfpathclose%
\pgfusepath{fill}%
\end{pgfscope}%
\begin{pgfscope}%
\pgfpathrectangle{\pgfqpoint{0.017500in}{0.017500in}}{\pgfqpoint{6.218282in}{3.580372in}} %
\pgfusepath{clip}%
\pgfsetbuttcap%
\pgfsetroundjoin%
\definecolor{currentfill}{rgb}{0.723315,0.068898,0.162989}%
\pgfsetfillcolor{currentfill}%
\pgfsetlinewidth{0.000000pt}%
\definecolor{currentstroke}{rgb}{0.000000,0.000000,0.000000}%
\pgfsetstrokecolor{currentstroke}%
\pgfsetdash{}{0pt}%
\pgfpathmoveto{\pgfqpoint{4.724651in}{2.549289in}}%
\pgfpathlineto{\pgfqpoint{4.964811in}{2.684961in}}%
\pgfpathlineto{\pgfqpoint{4.724651in}{2.549289in}}%
\pgfpathclose%
\pgfusepath{fill}%
\end{pgfscope}%
\begin{pgfscope}%
\pgfpathrectangle{\pgfqpoint{0.017500in}{0.017500in}}{\pgfqpoint{6.218282in}{3.580372in}} %
\pgfusepath{clip}%
\pgfsetbuttcap%
\pgfsetroundjoin%
\definecolor{currentfill}{rgb}{0.723315,0.068898,0.162989}%
\pgfsetfillcolor{currentfill}%
\pgfsetlinewidth{0.000000pt}%
\definecolor{currentstroke}{rgb}{0.000000,0.000000,0.000000}%
\pgfsetstrokecolor{currentstroke}%
\pgfsetdash{}{0pt}%
\pgfpathmoveto{\pgfqpoint{4.724651in}{2.549289in}}%
\pgfpathlineto{\pgfqpoint{4.964811in}{2.684961in}}%
\pgfpathlineto{\pgfqpoint{4.724651in}{2.549289in}}%
\pgfpathclose%
\pgfusepath{fill}%
\end{pgfscope}%
\begin{pgfscope}%
\pgfpathrectangle{\pgfqpoint{0.017500in}{0.017500in}}{\pgfqpoint{6.218282in}{3.580372in}} %
\pgfusepath{clip}%
\pgfsetbuttcap%
\pgfsetroundjoin%
\definecolor{currentfill}{rgb}{0.723315,0.068898,0.162989}%
\pgfsetfillcolor{currentfill}%
\pgfsetlinewidth{0.000000pt}%
\definecolor{currentstroke}{rgb}{0.000000,0.000000,0.000000}%
\pgfsetstrokecolor{currentstroke}%
\pgfsetdash{}{0pt}%
\pgfpathmoveto{\pgfqpoint{4.724651in}{2.549289in}}%
\pgfpathlineto{\pgfqpoint{4.964811in}{2.684961in}}%
\pgfpathlineto{\pgfqpoint{4.724651in}{2.549289in}}%
\pgfpathclose%
\pgfusepath{fill}%
\end{pgfscope}%
\begin{pgfscope}%
\pgfpathrectangle{\pgfqpoint{0.017500in}{0.017500in}}{\pgfqpoint{6.218282in}{3.580372in}} %
\pgfusepath{clip}%
\pgfsetbuttcap%
\pgfsetroundjoin%
\definecolor{currentfill}{rgb}{0.723315,0.068898,0.162989}%
\pgfsetfillcolor{currentfill}%
\pgfsetlinewidth{0.000000pt}%
\definecolor{currentstroke}{rgb}{0.000000,0.000000,0.000000}%
\pgfsetstrokecolor{currentstroke}%
\pgfsetdash{}{0pt}%
\pgfpathmoveto{\pgfqpoint{4.724651in}{2.549289in}}%
\pgfpathlineto{\pgfqpoint{4.964811in}{2.684961in}}%
\pgfpathlineto{\pgfqpoint{4.724651in}{2.549289in}}%
\pgfpathclose%
\pgfusepath{fill}%
\end{pgfscope}%
\begin{pgfscope}%
\pgfpathrectangle{\pgfqpoint{0.017500in}{0.017500in}}{\pgfqpoint{6.218282in}{3.580372in}} %
\pgfusepath{clip}%
\pgfsetbuttcap%
\pgfsetroundjoin%
\definecolor{currentfill}{rgb}{0.723315,0.068898,0.162989}%
\pgfsetfillcolor{currentfill}%
\pgfsetlinewidth{0.000000pt}%
\definecolor{currentstroke}{rgb}{0.000000,0.000000,0.000000}%
\pgfsetstrokecolor{currentstroke}%
\pgfsetdash{}{0pt}%
\pgfpathmoveto{\pgfqpoint{4.724651in}{2.549289in}}%
\pgfpathlineto{\pgfqpoint{4.964811in}{2.684961in}}%
\pgfpathlineto{\pgfqpoint{4.724651in}{2.549289in}}%
\pgfpathclose%
\pgfusepath{fill}%
\end{pgfscope}%
\begin{pgfscope}%
\pgfpathrectangle{\pgfqpoint{0.017500in}{0.017500in}}{\pgfqpoint{6.218282in}{3.580372in}} %
\pgfusepath{clip}%
\pgfsetbuttcap%
\pgfsetroundjoin%
\definecolor{currentfill}{rgb}{0.723315,0.068898,0.162989}%
\pgfsetfillcolor{currentfill}%
\pgfsetlinewidth{0.000000pt}%
\definecolor{currentstroke}{rgb}{0.000000,0.000000,0.000000}%
\pgfsetstrokecolor{currentstroke}%
\pgfsetdash{}{0pt}%
\pgfpathmoveto{\pgfqpoint{4.724651in}{2.549289in}}%
\pgfpathlineto{\pgfqpoint{4.964811in}{2.684961in}}%
\pgfpathlineto{\pgfqpoint{4.724651in}{2.549289in}}%
\pgfpathclose%
\pgfusepath{fill}%
\end{pgfscope}%
\begin{pgfscope}%
\pgfpathrectangle{\pgfqpoint{0.017500in}{0.017500in}}{\pgfqpoint{6.218282in}{3.580372in}} %
\pgfusepath{clip}%
\pgfsetbuttcap%
\pgfsetroundjoin%
\definecolor{currentfill}{rgb}{0.723315,0.068898,0.162989}%
\pgfsetfillcolor{currentfill}%
\pgfsetlinewidth{0.000000pt}%
\definecolor{currentstroke}{rgb}{0.000000,0.000000,0.000000}%
\pgfsetstrokecolor{currentstroke}%
\pgfsetdash{}{0pt}%
\pgfpathmoveto{\pgfqpoint{4.724651in}{2.549289in}}%
\pgfpathlineto{\pgfqpoint{4.964811in}{2.684961in}}%
\pgfpathlineto{\pgfqpoint{4.724651in}{2.549289in}}%
\pgfpathclose%
\pgfusepath{fill}%
\end{pgfscope}%
\begin{pgfscope}%
\pgfpathrectangle{\pgfqpoint{0.017500in}{0.017500in}}{\pgfqpoint{6.218282in}{3.580372in}} %
\pgfusepath{clip}%
\pgfsetbuttcap%
\pgfsetroundjoin%
\definecolor{currentfill}{rgb}{0.723315,0.068898,0.162989}%
\pgfsetfillcolor{currentfill}%
\pgfsetlinewidth{0.000000pt}%
\definecolor{currentstroke}{rgb}{0.000000,0.000000,0.000000}%
\pgfsetstrokecolor{currentstroke}%
\pgfsetdash{}{0pt}%
\pgfpathmoveto{\pgfqpoint{4.724651in}{2.549289in}}%
\pgfpathlineto{\pgfqpoint{4.964811in}{2.684961in}}%
\pgfpathlineto{\pgfqpoint{4.724651in}{2.549289in}}%
\pgfpathclose%
\pgfusepath{fill}%
\end{pgfscope}%
\begin{pgfscope}%
\pgfpathrectangle{\pgfqpoint{0.017500in}{0.017500in}}{\pgfqpoint{6.218282in}{3.580372in}} %
\pgfusepath{clip}%
\pgfsetbuttcap%
\pgfsetroundjoin%
\definecolor{currentfill}{rgb}{0.723315,0.068898,0.162989}%
\pgfsetfillcolor{currentfill}%
\pgfsetlinewidth{0.000000pt}%
\definecolor{currentstroke}{rgb}{0.000000,0.000000,0.000000}%
\pgfsetstrokecolor{currentstroke}%
\pgfsetdash{}{0pt}%
\pgfpathmoveto{\pgfqpoint{4.724651in}{2.549289in}}%
\pgfpathlineto{\pgfqpoint{4.964811in}{2.684961in}}%
\pgfpathlineto{\pgfqpoint{4.724651in}{2.549289in}}%
\pgfpathclose%
\pgfusepath{fill}%
\end{pgfscope}%
\begin{pgfscope}%
\pgfpathrectangle{\pgfqpoint{0.017500in}{0.017500in}}{\pgfqpoint{6.218282in}{3.580372in}} %
\pgfusepath{clip}%
\pgfsetbuttcap%
\pgfsetroundjoin%
\definecolor{currentfill}{rgb}{0.723315,0.068898,0.162989}%
\pgfsetfillcolor{currentfill}%
\pgfsetlinewidth{0.000000pt}%
\definecolor{currentstroke}{rgb}{0.000000,0.000000,0.000000}%
\pgfsetstrokecolor{currentstroke}%
\pgfsetdash{}{0pt}%
\pgfpathmoveto{\pgfqpoint{4.724651in}{2.549289in}}%
\pgfpathlineto{\pgfqpoint{4.964811in}{2.684961in}}%
\pgfpathlineto{\pgfqpoint{4.724651in}{2.549289in}}%
\pgfpathclose%
\pgfusepath{fill}%
\end{pgfscope}%
\begin{pgfscope}%
\pgfpathrectangle{\pgfqpoint{0.017500in}{0.017500in}}{\pgfqpoint{6.218282in}{3.580372in}} %
\pgfusepath{clip}%
\pgfsetbuttcap%
\pgfsetroundjoin%
\definecolor{currentfill}{rgb}{0.723315,0.068898,0.162989}%
\pgfsetfillcolor{currentfill}%
\pgfsetlinewidth{0.000000pt}%
\definecolor{currentstroke}{rgb}{0.000000,0.000000,0.000000}%
\pgfsetstrokecolor{currentstroke}%
\pgfsetdash{}{0pt}%
\pgfpathmoveto{\pgfqpoint{4.724651in}{2.549289in}}%
\pgfpathlineto{\pgfqpoint{4.964811in}{2.684961in}}%
\pgfpathlineto{\pgfqpoint{4.724651in}{2.549289in}}%
\pgfpathclose%
\pgfusepath{fill}%
\end{pgfscope}%
\begin{pgfscope}%
\pgfpathrectangle{\pgfqpoint{0.017500in}{0.017500in}}{\pgfqpoint{6.218282in}{3.580372in}} %
\pgfusepath{clip}%
\pgfsetbuttcap%
\pgfsetroundjoin%
\definecolor{currentfill}{rgb}{0.723315,0.068898,0.162989}%
\pgfsetfillcolor{currentfill}%
\pgfsetlinewidth{0.000000pt}%
\definecolor{currentstroke}{rgb}{0.000000,0.000000,0.000000}%
\pgfsetstrokecolor{currentstroke}%
\pgfsetdash{}{0pt}%
\pgfpathmoveto{\pgfqpoint{4.724651in}{2.549289in}}%
\pgfpathlineto{\pgfqpoint{4.964811in}{2.684961in}}%
\pgfpathlineto{\pgfqpoint{4.724651in}{2.549289in}}%
\pgfpathclose%
\pgfusepath{fill}%
\end{pgfscope}%
\begin{pgfscope}%
\pgfpathrectangle{\pgfqpoint{0.017500in}{0.017500in}}{\pgfqpoint{6.218282in}{3.580372in}} %
\pgfusepath{clip}%
\pgfsetbuttcap%
\pgfsetroundjoin%
\definecolor{currentfill}{rgb}{0.723315,0.068898,0.162989}%
\pgfsetfillcolor{currentfill}%
\pgfsetlinewidth{0.000000pt}%
\definecolor{currentstroke}{rgb}{0.000000,0.000000,0.000000}%
\pgfsetstrokecolor{currentstroke}%
\pgfsetdash{}{0pt}%
\pgfpathmoveto{\pgfqpoint{4.724651in}{2.549289in}}%
\pgfpathlineto{\pgfqpoint{4.964811in}{2.684961in}}%
\pgfpathlineto{\pgfqpoint{4.724651in}{2.549289in}}%
\pgfpathclose%
\pgfusepath{fill}%
\end{pgfscope}%
\begin{pgfscope}%
\pgfpathrectangle{\pgfqpoint{0.017500in}{0.017500in}}{\pgfqpoint{6.218282in}{3.580372in}} %
\pgfusepath{clip}%
\pgfsetbuttcap%
\pgfsetroundjoin%
\definecolor{currentfill}{rgb}{0.723315,0.068898,0.162989}%
\pgfsetfillcolor{currentfill}%
\pgfsetlinewidth{0.000000pt}%
\definecolor{currentstroke}{rgb}{0.000000,0.000000,0.000000}%
\pgfsetstrokecolor{currentstroke}%
\pgfsetdash{}{0pt}%
\pgfpathmoveto{\pgfqpoint{4.724651in}{2.549289in}}%
\pgfpathlineto{\pgfqpoint{4.964811in}{2.684961in}}%
\pgfpathlineto{\pgfqpoint{4.724651in}{2.549289in}}%
\pgfpathclose%
\pgfusepath{fill}%
\end{pgfscope}%
\begin{pgfscope}%
\pgfpathrectangle{\pgfqpoint{0.017500in}{0.017500in}}{\pgfqpoint{6.218282in}{3.580372in}} %
\pgfusepath{clip}%
\pgfsetbuttcap%
\pgfsetroundjoin%
\definecolor{currentfill}{rgb}{0.723315,0.068898,0.162989}%
\pgfsetfillcolor{currentfill}%
\pgfsetlinewidth{0.000000pt}%
\definecolor{currentstroke}{rgb}{0.000000,0.000000,0.000000}%
\pgfsetstrokecolor{currentstroke}%
\pgfsetdash{}{0pt}%
\pgfpathmoveto{\pgfqpoint{4.724651in}{2.549289in}}%
\pgfpathlineto{\pgfqpoint{4.964811in}{2.684961in}}%
\pgfpathlineto{\pgfqpoint{4.724651in}{2.549289in}}%
\pgfpathclose%
\pgfusepath{fill}%
\end{pgfscope}%
\begin{pgfscope}%
\pgfpathrectangle{\pgfqpoint{0.017500in}{0.017500in}}{\pgfqpoint{6.218282in}{3.580372in}} %
\pgfusepath{clip}%
\pgfsetbuttcap%
\pgfsetroundjoin%
\definecolor{currentfill}{rgb}{0.723315,0.068898,0.162989}%
\pgfsetfillcolor{currentfill}%
\pgfsetlinewidth{0.000000pt}%
\definecolor{currentstroke}{rgb}{0.000000,0.000000,0.000000}%
\pgfsetstrokecolor{currentstroke}%
\pgfsetdash{}{0pt}%
\pgfpathmoveto{\pgfqpoint{4.724651in}{2.549289in}}%
\pgfpathlineto{\pgfqpoint{4.964811in}{2.684961in}}%
\pgfpathlineto{\pgfqpoint{4.724651in}{2.549289in}}%
\pgfpathclose%
\pgfusepath{fill}%
\end{pgfscope}%
\begin{pgfscope}%
\pgfpathrectangle{\pgfqpoint{0.017500in}{0.017500in}}{\pgfqpoint{6.218282in}{3.580372in}} %
\pgfusepath{clip}%
\pgfsetbuttcap%
\pgfsetroundjoin%
\definecolor{currentfill}{rgb}{0.723315,0.068898,0.162989}%
\pgfsetfillcolor{currentfill}%
\pgfsetlinewidth{0.000000pt}%
\definecolor{currentstroke}{rgb}{0.000000,0.000000,0.000000}%
\pgfsetstrokecolor{currentstroke}%
\pgfsetdash{}{0pt}%
\pgfpathmoveto{\pgfqpoint{4.724651in}{2.549289in}}%
\pgfpathlineto{\pgfqpoint{4.964811in}{2.684961in}}%
\pgfpathlineto{\pgfqpoint{4.724651in}{2.549289in}}%
\pgfpathclose%
\pgfusepath{fill}%
\end{pgfscope}%
\begin{pgfscope}%
\pgfpathrectangle{\pgfqpoint{0.017500in}{0.017500in}}{\pgfqpoint{6.218282in}{3.580372in}} %
\pgfusepath{clip}%
\pgfsetbuttcap%
\pgfsetroundjoin%
\definecolor{currentfill}{rgb}{0.723315,0.068898,0.162989}%
\pgfsetfillcolor{currentfill}%
\pgfsetlinewidth{0.000000pt}%
\definecolor{currentstroke}{rgb}{0.000000,0.000000,0.000000}%
\pgfsetstrokecolor{currentstroke}%
\pgfsetdash{}{0pt}%
\pgfpathmoveto{\pgfqpoint{4.724651in}{2.549289in}}%
\pgfpathlineto{\pgfqpoint{4.964811in}{2.684961in}}%
\pgfpathlineto{\pgfqpoint{4.724651in}{2.549289in}}%
\pgfpathclose%
\pgfusepath{fill}%
\end{pgfscope}%
\begin{pgfscope}%
\pgfpathrectangle{\pgfqpoint{0.017500in}{0.017500in}}{\pgfqpoint{6.218282in}{3.580372in}} %
\pgfusepath{clip}%
\pgfsetbuttcap%
\pgfsetroundjoin%
\definecolor{currentfill}{rgb}{0.723315,0.068898,0.162989}%
\pgfsetfillcolor{currentfill}%
\pgfsetlinewidth{0.000000pt}%
\definecolor{currentstroke}{rgb}{0.000000,0.000000,0.000000}%
\pgfsetstrokecolor{currentstroke}%
\pgfsetdash{}{0pt}%
\pgfpathmoveto{\pgfqpoint{4.724651in}{2.549289in}}%
\pgfpathlineto{\pgfqpoint{4.964811in}{2.684961in}}%
\pgfpathlineto{\pgfqpoint{4.724651in}{2.549289in}}%
\pgfpathclose%
\pgfusepath{fill}%
\end{pgfscope}%
\begin{pgfscope}%
\pgfpathrectangle{\pgfqpoint{0.017500in}{0.017500in}}{\pgfqpoint{6.218282in}{3.580372in}} %
\pgfusepath{clip}%
\pgfsetbuttcap%
\pgfsetroundjoin%
\definecolor{currentfill}{rgb}{0.723315,0.068898,0.162989}%
\pgfsetfillcolor{currentfill}%
\pgfsetlinewidth{0.000000pt}%
\definecolor{currentstroke}{rgb}{0.000000,0.000000,0.000000}%
\pgfsetstrokecolor{currentstroke}%
\pgfsetdash{}{0pt}%
\pgfpathmoveto{\pgfqpoint{4.724651in}{2.549289in}}%
\pgfpathlineto{\pgfqpoint{4.964811in}{2.684961in}}%
\pgfpathlineto{\pgfqpoint{4.724651in}{2.549289in}}%
\pgfpathclose%
\pgfusepath{fill}%
\end{pgfscope}%
\begin{pgfscope}%
\pgfpathrectangle{\pgfqpoint{0.017500in}{0.017500in}}{\pgfqpoint{6.218282in}{3.580372in}} %
\pgfusepath{clip}%
\pgfsetbuttcap%
\pgfsetroundjoin%
\definecolor{currentfill}{rgb}{0.723315,0.068898,0.162989}%
\pgfsetfillcolor{currentfill}%
\pgfsetlinewidth{0.000000pt}%
\definecolor{currentstroke}{rgb}{0.000000,0.000000,0.000000}%
\pgfsetstrokecolor{currentstroke}%
\pgfsetdash{}{0pt}%
\pgfpathmoveto{\pgfqpoint{4.724651in}{2.549289in}}%
\pgfpathlineto{\pgfqpoint{4.964811in}{2.684961in}}%
\pgfpathlineto{\pgfqpoint{4.724651in}{2.549289in}}%
\pgfpathclose%
\pgfusepath{fill}%
\end{pgfscope}%
\begin{pgfscope}%
\pgfpathrectangle{\pgfqpoint{0.017500in}{0.017500in}}{\pgfqpoint{6.218282in}{3.580372in}} %
\pgfusepath{clip}%
\pgfsetbuttcap%
\pgfsetroundjoin%
\definecolor{currentfill}{rgb}{0.723315,0.068898,0.162989}%
\pgfsetfillcolor{currentfill}%
\pgfsetlinewidth{0.000000pt}%
\definecolor{currentstroke}{rgb}{0.000000,0.000000,0.000000}%
\pgfsetstrokecolor{currentstroke}%
\pgfsetdash{}{0pt}%
\pgfpathmoveto{\pgfqpoint{4.724651in}{2.549289in}}%
\pgfpathlineto{\pgfqpoint{4.964811in}{2.684961in}}%
\pgfpathlineto{\pgfqpoint{4.724651in}{2.549289in}}%
\pgfpathclose%
\pgfusepath{fill}%
\end{pgfscope}%
\begin{pgfscope}%
\pgfpathrectangle{\pgfqpoint{0.017500in}{0.017500in}}{\pgfqpoint{6.218282in}{3.580372in}} %
\pgfusepath{clip}%
\pgfsetbuttcap%
\pgfsetroundjoin%
\definecolor{currentfill}{rgb}{0.328604,0.439712,0.869587}%
\pgfsetfillcolor{currentfill}%
\pgfsetlinewidth{0.000000pt}%
\definecolor{currentstroke}{rgb}{0.000000,0.000000,0.000000}%
\pgfsetstrokecolor{currentstroke}%
\pgfsetdash{}{0pt}%
\pgfpathmoveto{\pgfqpoint{2.349853in}{0.872712in}}%
\pgfpathlineto{\pgfqpoint{2.789950in}{0.798346in}}%
\pgfpathlineto{\pgfqpoint{3.043326in}{2.141782in}}%
\pgfpathlineto{\pgfqpoint{2.590432in}{2.210709in}}%
\pgfpathlineto{\pgfqpoint{2.349853in}{0.872712in}}%
\pgfpathclose%
\pgfusepath{fill}%
\end{pgfscope}%
\begin{pgfscope}%
\pgfpathrectangle{\pgfqpoint{0.017500in}{0.017500in}}{\pgfqpoint{6.218282in}{3.580372in}} %
\pgfusepath{clip}%
\pgfsetbuttcap%
\pgfsetroundjoin%
\definecolor{currentfill}{rgb}{0.328604,0.439712,0.869587}%
\pgfsetfillcolor{currentfill}%
\pgfsetlinewidth{0.000000pt}%
\definecolor{currentstroke}{rgb}{0.000000,0.000000,0.000000}%
\pgfsetstrokecolor{currentstroke}%
\pgfsetdash{}{0pt}%
\pgfpathmoveto{\pgfqpoint{2.349853in}{0.872712in}}%
\pgfpathlineto{\pgfqpoint{2.789950in}{0.798346in}}%
\pgfpathlineto{\pgfqpoint{3.043326in}{2.141782in}}%
\pgfpathlineto{\pgfqpoint{2.590432in}{2.210709in}}%
\pgfpathlineto{\pgfqpoint{2.349853in}{0.872712in}}%
\pgfpathclose%
\pgfusepath{fill}%
\end{pgfscope}%
\begin{pgfscope}%
\pgfpathrectangle{\pgfqpoint{0.017500in}{0.017500in}}{\pgfqpoint{6.218282in}{3.580372in}} %
\pgfusepath{clip}%
\pgfsetbuttcap%
\pgfsetroundjoin%
\definecolor{currentfill}{rgb}{0.328604,0.439712,0.869587}%
\pgfsetfillcolor{currentfill}%
\pgfsetlinewidth{0.000000pt}%
\definecolor{currentstroke}{rgb}{0.000000,0.000000,0.000000}%
\pgfsetstrokecolor{currentstroke}%
\pgfsetdash{}{0pt}%
\pgfpathmoveto{\pgfqpoint{2.349853in}{0.872712in}}%
\pgfpathlineto{\pgfqpoint{2.789950in}{0.798346in}}%
\pgfpathlineto{\pgfqpoint{3.043326in}{2.141782in}}%
\pgfpathlineto{\pgfqpoint{2.590432in}{2.210709in}}%
\pgfpathlineto{\pgfqpoint{2.349853in}{0.872712in}}%
\pgfpathclose%
\pgfusepath{fill}%
\end{pgfscope}%
\begin{pgfscope}%
\pgfpathrectangle{\pgfqpoint{0.017500in}{0.017500in}}{\pgfqpoint{6.218282in}{3.580372in}} %
\pgfusepath{clip}%
\pgfsetbuttcap%
\pgfsetroundjoin%
\definecolor{currentfill}{rgb}{0.328604,0.439712,0.869587}%
\pgfsetfillcolor{currentfill}%
\pgfsetlinewidth{0.000000pt}%
\definecolor{currentstroke}{rgb}{0.000000,0.000000,0.000000}%
\pgfsetstrokecolor{currentstroke}%
\pgfsetdash{}{0pt}%
\pgfpathmoveto{\pgfqpoint{2.349853in}{0.872712in}}%
\pgfpathlineto{\pgfqpoint{2.789950in}{0.798346in}}%
\pgfpathlineto{\pgfqpoint{3.043326in}{2.141782in}}%
\pgfpathlineto{\pgfqpoint{2.590432in}{2.210709in}}%
\pgfpathlineto{\pgfqpoint{2.349853in}{0.872712in}}%
\pgfpathclose%
\pgfusepath{fill}%
\end{pgfscope}%
\begin{pgfscope}%
\pgfpathrectangle{\pgfqpoint{0.017500in}{0.017500in}}{\pgfqpoint{6.218282in}{3.580372in}} %
\pgfusepath{clip}%
\pgfsetbuttcap%
\pgfsetroundjoin%
\definecolor{currentfill}{rgb}{0.328604,0.439712,0.869587}%
\pgfsetfillcolor{currentfill}%
\pgfsetlinewidth{0.000000pt}%
\definecolor{currentstroke}{rgb}{0.000000,0.000000,0.000000}%
\pgfsetstrokecolor{currentstroke}%
\pgfsetdash{}{0pt}%
\pgfpathmoveto{\pgfqpoint{2.349853in}{0.872712in}}%
\pgfpathlineto{\pgfqpoint{2.789950in}{0.798346in}}%
\pgfpathlineto{\pgfqpoint{3.043326in}{2.141782in}}%
\pgfpathlineto{\pgfqpoint{2.590432in}{2.210709in}}%
\pgfpathlineto{\pgfqpoint{2.349853in}{0.872712in}}%
\pgfpathclose%
\pgfusepath{fill}%
\end{pgfscope}%
\begin{pgfscope}%
\pgfpathrectangle{\pgfqpoint{0.017500in}{0.017500in}}{\pgfqpoint{6.218282in}{3.580372in}} %
\pgfusepath{clip}%
\pgfsetbuttcap%
\pgfsetroundjoin%
\definecolor{currentfill}{rgb}{0.328604,0.439712,0.869587}%
\pgfsetfillcolor{currentfill}%
\pgfsetlinewidth{0.000000pt}%
\definecolor{currentstroke}{rgb}{0.000000,0.000000,0.000000}%
\pgfsetstrokecolor{currentstroke}%
\pgfsetdash{}{0pt}%
\pgfpathmoveto{\pgfqpoint{2.349853in}{0.872712in}}%
\pgfpathlineto{\pgfqpoint{2.789950in}{0.798346in}}%
\pgfpathlineto{\pgfqpoint{3.043326in}{2.141782in}}%
\pgfpathlineto{\pgfqpoint{2.590432in}{2.210709in}}%
\pgfpathlineto{\pgfqpoint{2.349853in}{0.872712in}}%
\pgfpathclose%
\pgfusepath{fill}%
\end{pgfscope}%
\begin{pgfscope}%
\pgfpathrectangle{\pgfqpoint{0.017500in}{0.017500in}}{\pgfqpoint{6.218282in}{3.580372in}} %
\pgfusepath{clip}%
\pgfsetbuttcap%
\pgfsetroundjoin%
\definecolor{currentfill}{rgb}{0.805723,0.259813,0.230562}%
\pgfsetfillcolor{currentfill}%
\pgfsetlinewidth{0.000000pt}%
\definecolor{currentstroke}{rgb}{0.000000,0.000000,0.000000}%
\pgfsetstrokecolor{currentstroke}%
\pgfsetdash{}{0pt}%
\pgfpathmoveto{\pgfqpoint{2.855513in}{2.451418in}}%
\pgfpathlineto{\pgfqpoint{3.116513in}{2.603812in}}%
\pgfpathlineto{\pgfqpoint{2.855513in}{2.451418in}}%
\pgfpathclose%
\pgfusepath{fill}%
\end{pgfscope}%
\begin{pgfscope}%
\pgfpathrectangle{\pgfqpoint{0.017500in}{0.017500in}}{\pgfqpoint{6.218282in}{3.580372in}} %
\pgfusepath{clip}%
\pgfsetbuttcap%
\pgfsetroundjoin%
\definecolor{currentfill}{rgb}{0.805723,0.259813,0.230562}%
\pgfsetfillcolor{currentfill}%
\pgfsetlinewidth{0.000000pt}%
\definecolor{currentstroke}{rgb}{0.000000,0.000000,0.000000}%
\pgfsetstrokecolor{currentstroke}%
\pgfsetdash{}{0pt}%
\pgfpathmoveto{\pgfqpoint{2.855513in}{2.451418in}}%
\pgfpathlineto{\pgfqpoint{3.116513in}{2.603812in}}%
\pgfpathlineto{\pgfqpoint{2.855513in}{2.451418in}}%
\pgfpathclose%
\pgfusepath{fill}%
\end{pgfscope}%
\begin{pgfscope}%
\pgfpathrectangle{\pgfqpoint{0.017500in}{0.017500in}}{\pgfqpoint{6.218282in}{3.580372in}} %
\pgfusepath{clip}%
\pgfsetbuttcap%
\pgfsetroundjoin%
\definecolor{currentfill}{rgb}{0.805723,0.259813,0.230562}%
\pgfsetfillcolor{currentfill}%
\pgfsetlinewidth{0.000000pt}%
\definecolor{currentstroke}{rgb}{0.000000,0.000000,0.000000}%
\pgfsetstrokecolor{currentstroke}%
\pgfsetdash{}{0pt}%
\pgfpathmoveto{\pgfqpoint{2.855513in}{2.451418in}}%
\pgfpathlineto{\pgfqpoint{3.116513in}{2.603812in}}%
\pgfpathlineto{\pgfqpoint{2.855513in}{2.451418in}}%
\pgfpathclose%
\pgfusepath{fill}%
\end{pgfscope}%
\begin{pgfscope}%
\pgfpathrectangle{\pgfqpoint{0.017500in}{0.017500in}}{\pgfqpoint{6.218282in}{3.580372in}} %
\pgfusepath{clip}%
\pgfsetbuttcap%
\pgfsetroundjoin%
\definecolor{currentfill}{rgb}{0.805723,0.259813,0.230562}%
\pgfsetfillcolor{currentfill}%
\pgfsetlinewidth{0.000000pt}%
\definecolor{currentstroke}{rgb}{0.000000,0.000000,0.000000}%
\pgfsetstrokecolor{currentstroke}%
\pgfsetdash{}{0pt}%
\pgfpathmoveto{\pgfqpoint{2.855513in}{2.451418in}}%
\pgfpathlineto{\pgfqpoint{3.116513in}{2.603812in}}%
\pgfpathlineto{\pgfqpoint{2.855513in}{2.451418in}}%
\pgfpathclose%
\pgfusepath{fill}%
\end{pgfscope}%
\begin{pgfscope}%
\pgfpathrectangle{\pgfqpoint{0.017500in}{0.017500in}}{\pgfqpoint{6.218282in}{3.580372in}} %
\pgfusepath{clip}%
\pgfsetbuttcap%
\pgfsetroundjoin%
\definecolor{currentfill}{rgb}{0.805723,0.259813,0.230562}%
\pgfsetfillcolor{currentfill}%
\pgfsetlinewidth{0.000000pt}%
\definecolor{currentstroke}{rgb}{0.000000,0.000000,0.000000}%
\pgfsetstrokecolor{currentstroke}%
\pgfsetdash{}{0pt}%
\pgfpathmoveto{\pgfqpoint{2.855513in}{2.451418in}}%
\pgfpathlineto{\pgfqpoint{3.116513in}{2.603812in}}%
\pgfpathlineto{\pgfqpoint{2.855513in}{2.451418in}}%
\pgfpathclose%
\pgfusepath{fill}%
\end{pgfscope}%
\begin{pgfscope}%
\pgfpathrectangle{\pgfqpoint{0.017500in}{0.017500in}}{\pgfqpoint{6.218282in}{3.580372in}} %
\pgfusepath{clip}%
\pgfsetbuttcap%
\pgfsetroundjoin%
\definecolor{currentfill}{rgb}{0.805723,0.259813,0.230562}%
\pgfsetfillcolor{currentfill}%
\pgfsetlinewidth{0.000000pt}%
\definecolor{currentstroke}{rgb}{0.000000,0.000000,0.000000}%
\pgfsetstrokecolor{currentstroke}%
\pgfsetdash{}{0pt}%
\pgfpathmoveto{\pgfqpoint{2.855513in}{2.451418in}}%
\pgfpathlineto{\pgfqpoint{3.116513in}{2.603812in}}%
\pgfpathlineto{\pgfqpoint{2.855513in}{2.451418in}}%
\pgfpathclose%
\pgfusepath{fill}%
\end{pgfscope}%
\begin{pgfscope}%
\pgfpathrectangle{\pgfqpoint{0.017500in}{0.017500in}}{\pgfqpoint{6.218282in}{3.580372in}} %
\pgfusepath{clip}%
\pgfsetbuttcap%
\pgfsetroundjoin%
\definecolor{currentfill}{rgb}{0.805723,0.259813,0.230562}%
\pgfsetfillcolor{currentfill}%
\pgfsetlinewidth{0.000000pt}%
\definecolor{currentstroke}{rgb}{0.000000,0.000000,0.000000}%
\pgfsetstrokecolor{currentstroke}%
\pgfsetdash{}{0pt}%
\pgfpathmoveto{\pgfqpoint{2.855513in}{2.451418in}}%
\pgfpathlineto{\pgfqpoint{3.116513in}{2.603812in}}%
\pgfpathlineto{\pgfqpoint{2.855513in}{2.451418in}}%
\pgfpathclose%
\pgfusepath{fill}%
\end{pgfscope}%
\begin{pgfscope}%
\pgfpathrectangle{\pgfqpoint{0.017500in}{0.017500in}}{\pgfqpoint{6.218282in}{3.580372in}} %
\pgfusepath{clip}%
\pgfsetbuttcap%
\pgfsetroundjoin%
\definecolor{currentfill}{rgb}{0.805723,0.259813,0.230562}%
\pgfsetfillcolor{currentfill}%
\pgfsetlinewidth{0.000000pt}%
\definecolor{currentstroke}{rgb}{0.000000,0.000000,0.000000}%
\pgfsetstrokecolor{currentstroke}%
\pgfsetdash{}{0pt}%
\pgfpathmoveto{\pgfqpoint{2.855513in}{2.451418in}}%
\pgfpathlineto{\pgfqpoint{3.116513in}{2.603812in}}%
\pgfpathlineto{\pgfqpoint{2.855513in}{2.451418in}}%
\pgfpathclose%
\pgfusepath{fill}%
\end{pgfscope}%
\begin{pgfscope}%
\pgfpathrectangle{\pgfqpoint{0.017500in}{0.017500in}}{\pgfqpoint{6.218282in}{3.580372in}} %
\pgfusepath{clip}%
\pgfsetbuttcap%
\pgfsetroundjoin%
\definecolor{currentfill}{rgb}{0.805723,0.259813,0.230562}%
\pgfsetfillcolor{currentfill}%
\pgfsetlinewidth{0.000000pt}%
\definecolor{currentstroke}{rgb}{0.000000,0.000000,0.000000}%
\pgfsetstrokecolor{currentstroke}%
\pgfsetdash{}{0pt}%
\pgfpathmoveto{\pgfqpoint{2.855513in}{2.451418in}}%
\pgfpathlineto{\pgfqpoint{3.116513in}{2.603812in}}%
\pgfpathlineto{\pgfqpoint{2.855513in}{2.451418in}}%
\pgfpathclose%
\pgfusepath{fill}%
\end{pgfscope}%
\begin{pgfscope}%
\pgfpathrectangle{\pgfqpoint{0.017500in}{0.017500in}}{\pgfqpoint{6.218282in}{3.580372in}} %
\pgfusepath{clip}%
\pgfsetbuttcap%
\pgfsetroundjoin%
\definecolor{currentfill}{rgb}{0.805723,0.259813,0.230562}%
\pgfsetfillcolor{currentfill}%
\pgfsetlinewidth{0.000000pt}%
\definecolor{currentstroke}{rgb}{0.000000,0.000000,0.000000}%
\pgfsetstrokecolor{currentstroke}%
\pgfsetdash{}{0pt}%
\pgfpathmoveto{\pgfqpoint{2.855513in}{2.451418in}}%
\pgfpathlineto{\pgfqpoint{3.116513in}{2.603812in}}%
\pgfpathlineto{\pgfqpoint{2.855513in}{2.451418in}}%
\pgfpathclose%
\pgfusepath{fill}%
\end{pgfscope}%
\begin{pgfscope}%
\pgfpathrectangle{\pgfqpoint{0.017500in}{0.017500in}}{\pgfqpoint{6.218282in}{3.580372in}} %
\pgfusepath{clip}%
\pgfsetbuttcap%
\pgfsetroundjoin%
\definecolor{currentfill}{rgb}{0.805723,0.259813,0.230562}%
\pgfsetfillcolor{currentfill}%
\pgfsetlinewidth{0.000000pt}%
\definecolor{currentstroke}{rgb}{0.000000,0.000000,0.000000}%
\pgfsetstrokecolor{currentstroke}%
\pgfsetdash{}{0pt}%
\pgfpathmoveto{\pgfqpoint{2.855513in}{2.451418in}}%
\pgfpathlineto{\pgfqpoint{3.116513in}{2.603812in}}%
\pgfpathlineto{\pgfqpoint{2.855513in}{2.451418in}}%
\pgfpathclose%
\pgfusepath{fill}%
\end{pgfscope}%
\begin{pgfscope}%
\pgfpathrectangle{\pgfqpoint{0.017500in}{0.017500in}}{\pgfqpoint{6.218282in}{3.580372in}} %
\pgfusepath{clip}%
\pgfsetbuttcap%
\pgfsetroundjoin%
\definecolor{currentfill}{rgb}{0.805723,0.259813,0.230562}%
\pgfsetfillcolor{currentfill}%
\pgfsetlinewidth{0.000000pt}%
\definecolor{currentstroke}{rgb}{0.000000,0.000000,0.000000}%
\pgfsetstrokecolor{currentstroke}%
\pgfsetdash{}{0pt}%
\pgfpathmoveto{\pgfqpoint{2.855513in}{2.451418in}}%
\pgfpathlineto{\pgfqpoint{3.116513in}{2.603812in}}%
\pgfpathlineto{\pgfqpoint{2.855513in}{2.451418in}}%
\pgfpathclose%
\pgfusepath{fill}%
\end{pgfscope}%
\begin{pgfscope}%
\pgfpathrectangle{\pgfqpoint{0.017500in}{0.017500in}}{\pgfqpoint{6.218282in}{3.580372in}} %
\pgfusepath{clip}%
\pgfsetbuttcap%
\pgfsetroundjoin%
\definecolor{currentfill}{rgb}{0.805723,0.259813,0.230562}%
\pgfsetfillcolor{currentfill}%
\pgfsetlinewidth{0.000000pt}%
\definecolor{currentstroke}{rgb}{0.000000,0.000000,0.000000}%
\pgfsetstrokecolor{currentstroke}%
\pgfsetdash{}{0pt}%
\pgfpathmoveto{\pgfqpoint{2.855513in}{2.451418in}}%
\pgfpathlineto{\pgfqpoint{3.116513in}{2.603812in}}%
\pgfpathlineto{\pgfqpoint{2.855513in}{2.451418in}}%
\pgfpathclose%
\pgfusepath{fill}%
\end{pgfscope}%
\begin{pgfscope}%
\pgfpathrectangle{\pgfqpoint{0.017500in}{0.017500in}}{\pgfqpoint{6.218282in}{3.580372in}} %
\pgfusepath{clip}%
\pgfsetbuttcap%
\pgfsetroundjoin%
\definecolor{currentfill}{rgb}{0.805723,0.259813,0.230562}%
\pgfsetfillcolor{currentfill}%
\pgfsetlinewidth{0.000000pt}%
\definecolor{currentstroke}{rgb}{0.000000,0.000000,0.000000}%
\pgfsetstrokecolor{currentstroke}%
\pgfsetdash{}{0pt}%
\pgfpathmoveto{\pgfqpoint{2.855513in}{2.451418in}}%
\pgfpathlineto{\pgfqpoint{3.116513in}{2.603812in}}%
\pgfpathlineto{\pgfqpoint{2.855513in}{2.451418in}}%
\pgfpathclose%
\pgfusepath{fill}%
\end{pgfscope}%
\begin{pgfscope}%
\pgfpathrectangle{\pgfqpoint{0.017500in}{0.017500in}}{\pgfqpoint{6.218282in}{3.580372in}} %
\pgfusepath{clip}%
\pgfsetbuttcap%
\pgfsetroundjoin%
\definecolor{currentfill}{rgb}{0.805723,0.259813,0.230562}%
\pgfsetfillcolor{currentfill}%
\pgfsetlinewidth{0.000000pt}%
\definecolor{currentstroke}{rgb}{0.000000,0.000000,0.000000}%
\pgfsetstrokecolor{currentstroke}%
\pgfsetdash{}{0pt}%
\pgfpathmoveto{\pgfqpoint{2.855513in}{2.451418in}}%
\pgfpathlineto{\pgfqpoint{3.116513in}{2.603812in}}%
\pgfpathlineto{\pgfqpoint{2.855513in}{2.451418in}}%
\pgfpathclose%
\pgfusepath{fill}%
\end{pgfscope}%
\begin{pgfscope}%
\pgfpathrectangle{\pgfqpoint{0.017500in}{0.017500in}}{\pgfqpoint{6.218282in}{3.580372in}} %
\pgfusepath{clip}%
\pgfsetbuttcap%
\pgfsetroundjoin%
\definecolor{currentfill}{rgb}{0.805723,0.259813,0.230562}%
\pgfsetfillcolor{currentfill}%
\pgfsetlinewidth{0.000000pt}%
\definecolor{currentstroke}{rgb}{0.000000,0.000000,0.000000}%
\pgfsetstrokecolor{currentstroke}%
\pgfsetdash{}{0pt}%
\pgfpathmoveto{\pgfqpoint{2.855513in}{2.451418in}}%
\pgfpathlineto{\pgfqpoint{3.116513in}{2.603812in}}%
\pgfpathlineto{\pgfqpoint{2.855513in}{2.451418in}}%
\pgfpathclose%
\pgfusepath{fill}%
\end{pgfscope}%
\begin{pgfscope}%
\pgfpathrectangle{\pgfqpoint{0.017500in}{0.017500in}}{\pgfqpoint{6.218282in}{3.580372in}} %
\pgfusepath{clip}%
\pgfsetbuttcap%
\pgfsetroundjoin%
\definecolor{currentfill}{rgb}{0.805723,0.259813,0.230562}%
\pgfsetfillcolor{currentfill}%
\pgfsetlinewidth{0.000000pt}%
\definecolor{currentstroke}{rgb}{0.000000,0.000000,0.000000}%
\pgfsetstrokecolor{currentstroke}%
\pgfsetdash{}{0pt}%
\pgfpathmoveto{\pgfqpoint{2.855513in}{2.451418in}}%
\pgfpathlineto{\pgfqpoint{3.116513in}{2.603812in}}%
\pgfpathlineto{\pgfqpoint{2.855513in}{2.451418in}}%
\pgfpathclose%
\pgfusepath{fill}%
\end{pgfscope}%
\begin{pgfscope}%
\pgfpathrectangle{\pgfqpoint{0.017500in}{0.017500in}}{\pgfqpoint{6.218282in}{3.580372in}} %
\pgfusepath{clip}%
\pgfsetbuttcap%
\pgfsetroundjoin%
\definecolor{currentfill}{rgb}{0.805723,0.259813,0.230562}%
\pgfsetfillcolor{currentfill}%
\pgfsetlinewidth{0.000000pt}%
\definecolor{currentstroke}{rgb}{0.000000,0.000000,0.000000}%
\pgfsetstrokecolor{currentstroke}%
\pgfsetdash{}{0pt}%
\pgfpathmoveto{\pgfqpoint{2.855513in}{2.451418in}}%
\pgfpathlineto{\pgfqpoint{3.116513in}{2.603812in}}%
\pgfpathlineto{\pgfqpoint{2.855513in}{2.451418in}}%
\pgfpathclose%
\pgfusepath{fill}%
\end{pgfscope}%
\begin{pgfscope}%
\pgfpathrectangle{\pgfqpoint{0.017500in}{0.017500in}}{\pgfqpoint{6.218282in}{3.580372in}} %
\pgfusepath{clip}%
\pgfsetbuttcap%
\pgfsetroundjoin%
\definecolor{currentfill}{rgb}{0.805723,0.259813,0.230562}%
\pgfsetfillcolor{currentfill}%
\pgfsetlinewidth{0.000000pt}%
\definecolor{currentstroke}{rgb}{0.000000,0.000000,0.000000}%
\pgfsetstrokecolor{currentstroke}%
\pgfsetdash{}{0pt}%
\pgfpathmoveto{\pgfqpoint{2.855513in}{2.451418in}}%
\pgfpathlineto{\pgfqpoint{3.116513in}{2.603812in}}%
\pgfpathlineto{\pgfqpoint{2.855513in}{2.451418in}}%
\pgfpathclose%
\pgfusepath{fill}%
\end{pgfscope}%
\begin{pgfscope}%
\pgfpathrectangle{\pgfqpoint{0.017500in}{0.017500in}}{\pgfqpoint{6.218282in}{3.580372in}} %
\pgfusepath{clip}%
\pgfsetbuttcap%
\pgfsetroundjoin%
\definecolor{currentfill}{rgb}{0.805723,0.259813,0.230562}%
\pgfsetfillcolor{currentfill}%
\pgfsetlinewidth{0.000000pt}%
\definecolor{currentstroke}{rgb}{0.000000,0.000000,0.000000}%
\pgfsetstrokecolor{currentstroke}%
\pgfsetdash{}{0pt}%
\pgfpathmoveto{\pgfqpoint{2.855513in}{2.451418in}}%
\pgfpathlineto{\pgfqpoint{3.116513in}{2.603812in}}%
\pgfpathlineto{\pgfqpoint{2.855513in}{2.451418in}}%
\pgfpathclose%
\pgfusepath{fill}%
\end{pgfscope}%
\begin{pgfscope}%
\pgfpathrectangle{\pgfqpoint{0.017500in}{0.017500in}}{\pgfqpoint{6.218282in}{3.580372in}} %
\pgfusepath{clip}%
\pgfsetbuttcap%
\pgfsetroundjoin%
\definecolor{currentfill}{rgb}{0.805723,0.259813,0.230562}%
\pgfsetfillcolor{currentfill}%
\pgfsetlinewidth{0.000000pt}%
\definecolor{currentstroke}{rgb}{0.000000,0.000000,0.000000}%
\pgfsetstrokecolor{currentstroke}%
\pgfsetdash{}{0pt}%
\pgfpathmoveto{\pgfqpoint{2.855513in}{2.451418in}}%
\pgfpathlineto{\pgfqpoint{3.116513in}{2.603812in}}%
\pgfpathlineto{\pgfqpoint{2.855513in}{2.451418in}}%
\pgfpathclose%
\pgfusepath{fill}%
\end{pgfscope}%
\begin{pgfscope}%
\pgfpathrectangle{\pgfqpoint{0.017500in}{0.017500in}}{\pgfqpoint{6.218282in}{3.580372in}} %
\pgfusepath{clip}%
\pgfsetbuttcap%
\pgfsetroundjoin%
\definecolor{currentfill}{rgb}{0.805723,0.259813,0.230562}%
\pgfsetfillcolor{currentfill}%
\pgfsetlinewidth{0.000000pt}%
\definecolor{currentstroke}{rgb}{0.000000,0.000000,0.000000}%
\pgfsetstrokecolor{currentstroke}%
\pgfsetdash{}{0pt}%
\pgfpathmoveto{\pgfqpoint{2.855513in}{2.451418in}}%
\pgfpathlineto{\pgfqpoint{3.116513in}{2.603812in}}%
\pgfpathlineto{\pgfqpoint{2.855513in}{2.451418in}}%
\pgfpathclose%
\pgfusepath{fill}%
\end{pgfscope}%
\begin{pgfscope}%
\pgfpathrectangle{\pgfqpoint{0.017500in}{0.017500in}}{\pgfqpoint{6.218282in}{3.580372in}} %
\pgfusepath{clip}%
\pgfsetbuttcap%
\pgfsetroundjoin%
\definecolor{currentfill}{rgb}{0.805723,0.259813,0.230562}%
\pgfsetfillcolor{currentfill}%
\pgfsetlinewidth{0.000000pt}%
\definecolor{currentstroke}{rgb}{0.000000,0.000000,0.000000}%
\pgfsetstrokecolor{currentstroke}%
\pgfsetdash{}{0pt}%
\pgfpathmoveto{\pgfqpoint{2.855513in}{2.451418in}}%
\pgfpathlineto{\pgfqpoint{3.116513in}{2.603812in}}%
\pgfpathlineto{\pgfqpoint{2.855513in}{2.451418in}}%
\pgfpathclose%
\pgfusepath{fill}%
\end{pgfscope}%
\begin{pgfscope}%
\pgfpathrectangle{\pgfqpoint{0.017500in}{0.017500in}}{\pgfqpoint{6.218282in}{3.580372in}} %
\pgfusepath{clip}%
\pgfsetbuttcap%
\pgfsetroundjoin%
\definecolor{currentfill}{rgb}{0.805723,0.259813,0.230562}%
\pgfsetfillcolor{currentfill}%
\pgfsetlinewidth{0.000000pt}%
\definecolor{currentstroke}{rgb}{0.000000,0.000000,0.000000}%
\pgfsetstrokecolor{currentstroke}%
\pgfsetdash{}{0pt}%
\pgfpathmoveto{\pgfqpoint{2.855513in}{2.451418in}}%
\pgfpathlineto{\pgfqpoint{3.116513in}{2.603812in}}%
\pgfpathlineto{\pgfqpoint{2.855513in}{2.451418in}}%
\pgfpathclose%
\pgfusepath{fill}%
\end{pgfscope}%
\begin{pgfscope}%
\pgfpathrectangle{\pgfqpoint{0.017500in}{0.017500in}}{\pgfqpoint{6.218282in}{3.580372in}} %
\pgfusepath{clip}%
\pgfsetbuttcap%
\pgfsetroundjoin%
\definecolor{currentfill}{rgb}{0.805723,0.259813,0.230562}%
\pgfsetfillcolor{currentfill}%
\pgfsetlinewidth{0.000000pt}%
\definecolor{currentstroke}{rgb}{0.000000,0.000000,0.000000}%
\pgfsetstrokecolor{currentstroke}%
\pgfsetdash{}{0pt}%
\pgfpathmoveto{\pgfqpoint{2.855513in}{2.451418in}}%
\pgfpathlineto{\pgfqpoint{3.116513in}{2.603812in}}%
\pgfpathlineto{\pgfqpoint{2.855513in}{2.451418in}}%
\pgfpathclose%
\pgfusepath{fill}%
\end{pgfscope}%
\begin{pgfscope}%
\pgfpathrectangle{\pgfqpoint{0.017500in}{0.017500in}}{\pgfqpoint{6.218282in}{3.580372in}} %
\pgfusepath{clip}%
\pgfsetbuttcap%
\pgfsetroundjoin%
\definecolor{currentfill}{rgb}{0.805723,0.259813,0.230562}%
\pgfsetfillcolor{currentfill}%
\pgfsetlinewidth{0.000000pt}%
\definecolor{currentstroke}{rgb}{0.000000,0.000000,0.000000}%
\pgfsetstrokecolor{currentstroke}%
\pgfsetdash{}{0pt}%
\pgfpathmoveto{\pgfqpoint{2.855513in}{2.451418in}}%
\pgfpathlineto{\pgfqpoint{3.116513in}{2.603812in}}%
\pgfpathlineto{\pgfqpoint{2.855513in}{2.451418in}}%
\pgfpathclose%
\pgfusepath{fill}%
\end{pgfscope}%
\begin{pgfscope}%
\pgfpathrectangle{\pgfqpoint{0.017500in}{0.017500in}}{\pgfqpoint{6.218282in}{3.580372in}} %
\pgfusepath{clip}%
\pgfsetbuttcap%
\pgfsetroundjoin%
\definecolor{currentfill}{rgb}{0.805723,0.259813,0.230562}%
\pgfsetfillcolor{currentfill}%
\pgfsetlinewidth{0.000000pt}%
\definecolor{currentstroke}{rgb}{0.000000,0.000000,0.000000}%
\pgfsetstrokecolor{currentstroke}%
\pgfsetdash{}{0pt}%
\pgfpathmoveto{\pgfqpoint{2.855513in}{2.451418in}}%
\pgfpathlineto{\pgfqpoint{3.116513in}{2.603812in}}%
\pgfpathlineto{\pgfqpoint{2.855513in}{2.451418in}}%
\pgfpathclose%
\pgfusepath{fill}%
\end{pgfscope}%
\begin{pgfscope}%
\pgfpathrectangle{\pgfqpoint{0.017500in}{0.017500in}}{\pgfqpoint{6.218282in}{3.580372in}} %
\pgfusepath{clip}%
\pgfsetbuttcap%
\pgfsetroundjoin%
\definecolor{currentfill}{rgb}{0.805723,0.259813,0.230562}%
\pgfsetfillcolor{currentfill}%
\pgfsetlinewidth{0.000000pt}%
\definecolor{currentstroke}{rgb}{0.000000,0.000000,0.000000}%
\pgfsetstrokecolor{currentstroke}%
\pgfsetdash{}{0pt}%
\pgfpathmoveto{\pgfqpoint{2.855513in}{2.451418in}}%
\pgfpathlineto{\pgfqpoint{3.116513in}{2.603812in}}%
\pgfpathlineto{\pgfqpoint{2.855513in}{2.451418in}}%
\pgfpathclose%
\pgfusepath{fill}%
\end{pgfscope}%
\begin{pgfscope}%
\pgfpathrectangle{\pgfqpoint{0.017500in}{0.017500in}}{\pgfqpoint{6.218282in}{3.580372in}} %
\pgfusepath{clip}%
\pgfsetbuttcap%
\pgfsetroundjoin%
\definecolor{currentfill}{rgb}{0.805723,0.259813,0.230562}%
\pgfsetfillcolor{currentfill}%
\pgfsetlinewidth{0.000000pt}%
\definecolor{currentstroke}{rgb}{0.000000,0.000000,0.000000}%
\pgfsetstrokecolor{currentstroke}%
\pgfsetdash{}{0pt}%
\pgfpathmoveto{\pgfqpoint{2.855513in}{2.451418in}}%
\pgfpathlineto{\pgfqpoint{3.116513in}{2.603812in}}%
\pgfpathlineto{\pgfqpoint{2.855513in}{2.451418in}}%
\pgfpathclose%
\pgfusepath{fill}%
\end{pgfscope}%
\begin{pgfscope}%
\pgfpathrectangle{\pgfqpoint{0.017500in}{0.017500in}}{\pgfqpoint{6.218282in}{3.580372in}} %
\pgfusepath{clip}%
\pgfsetbuttcap%
\pgfsetroundjoin%
\definecolor{currentfill}{rgb}{0.805723,0.259813,0.230562}%
\pgfsetfillcolor{currentfill}%
\pgfsetlinewidth{0.000000pt}%
\definecolor{currentstroke}{rgb}{0.000000,0.000000,0.000000}%
\pgfsetstrokecolor{currentstroke}%
\pgfsetdash{}{0pt}%
\pgfpathmoveto{\pgfqpoint{2.855513in}{2.451418in}}%
\pgfpathlineto{\pgfqpoint{3.116513in}{2.603812in}}%
\pgfpathlineto{\pgfqpoint{2.855513in}{2.451418in}}%
\pgfpathclose%
\pgfusepath{fill}%
\end{pgfscope}%
\begin{pgfscope}%
\pgfpathrectangle{\pgfqpoint{0.017500in}{0.017500in}}{\pgfqpoint{6.218282in}{3.580372in}} %
\pgfusepath{clip}%
\pgfsetbuttcap%
\pgfsetroundjoin%
\definecolor{currentfill}{rgb}{0.800830,0.250829,0.225696}%
\pgfsetfillcolor{currentfill}%
\pgfsetlinewidth{0.000000pt}%
\definecolor{currentstroke}{rgb}{0.000000,0.000000,0.000000}%
\pgfsetstrokecolor{currentstroke}%
\pgfsetdash{}{0pt}%
\pgfpathmoveto{\pgfqpoint{2.855513in}{2.451418in}}%
\pgfpathlineto{\pgfqpoint{3.306485in}{2.382499in}}%
\pgfpathlineto{\pgfqpoint{3.564492in}{2.541257in}}%
\pgfpathlineto{\pgfqpoint{3.116513in}{2.603812in}}%
\pgfpathlineto{\pgfqpoint{2.855513in}{2.451418in}}%
\pgfpathclose%
\pgfusepath{fill}%
\end{pgfscope}%
\begin{pgfscope}%
\pgfpathrectangle{\pgfqpoint{0.017500in}{0.017500in}}{\pgfqpoint{6.218282in}{3.580372in}} %
\pgfusepath{clip}%
\pgfsetbuttcap%
\pgfsetroundjoin%
\definecolor{currentfill}{rgb}{0.800830,0.250829,0.225696}%
\pgfsetfillcolor{currentfill}%
\pgfsetlinewidth{0.000000pt}%
\definecolor{currentstroke}{rgb}{0.000000,0.000000,0.000000}%
\pgfsetstrokecolor{currentstroke}%
\pgfsetdash{}{0pt}%
\pgfpathmoveto{\pgfqpoint{2.855513in}{2.451418in}}%
\pgfpathlineto{\pgfqpoint{3.306485in}{2.382499in}}%
\pgfpathlineto{\pgfqpoint{3.564492in}{2.541257in}}%
\pgfpathlineto{\pgfqpoint{3.116513in}{2.603812in}}%
\pgfpathlineto{\pgfqpoint{2.855513in}{2.451418in}}%
\pgfpathclose%
\pgfusepath{fill}%
\end{pgfscope}%
\begin{pgfscope}%
\pgfpathrectangle{\pgfqpoint{0.017500in}{0.017500in}}{\pgfqpoint{6.218282in}{3.580372in}} %
\pgfusepath{clip}%
\pgfsetbuttcap%
\pgfsetroundjoin%
\definecolor{currentfill}{rgb}{0.800830,0.250829,0.225696}%
\pgfsetfillcolor{currentfill}%
\pgfsetlinewidth{0.000000pt}%
\definecolor{currentstroke}{rgb}{0.000000,0.000000,0.000000}%
\pgfsetstrokecolor{currentstroke}%
\pgfsetdash{}{0pt}%
\pgfpathmoveto{\pgfqpoint{2.855513in}{2.451418in}}%
\pgfpathlineto{\pgfqpoint{3.306485in}{2.382499in}}%
\pgfpathlineto{\pgfqpoint{3.564492in}{2.541257in}}%
\pgfpathlineto{\pgfqpoint{3.116513in}{2.603812in}}%
\pgfpathlineto{\pgfqpoint{2.855513in}{2.451418in}}%
\pgfpathclose%
\pgfusepath{fill}%
\end{pgfscope}%
\begin{pgfscope}%
\pgfpathrectangle{\pgfqpoint{0.017500in}{0.017500in}}{\pgfqpoint{6.218282in}{3.580372in}} %
\pgfusepath{clip}%
\pgfsetbuttcap%
\pgfsetroundjoin%
\definecolor{currentfill}{rgb}{0.800830,0.250829,0.225696}%
\pgfsetfillcolor{currentfill}%
\pgfsetlinewidth{0.000000pt}%
\definecolor{currentstroke}{rgb}{0.000000,0.000000,0.000000}%
\pgfsetstrokecolor{currentstroke}%
\pgfsetdash{}{0pt}%
\pgfpathmoveto{\pgfqpoint{2.855513in}{2.451418in}}%
\pgfpathlineto{\pgfqpoint{3.306485in}{2.382499in}}%
\pgfpathlineto{\pgfqpoint{3.564492in}{2.541257in}}%
\pgfpathlineto{\pgfqpoint{3.116513in}{2.603812in}}%
\pgfpathlineto{\pgfqpoint{2.855513in}{2.451418in}}%
\pgfpathclose%
\pgfusepath{fill}%
\end{pgfscope}%
\begin{pgfscope}%
\pgfpathrectangle{\pgfqpoint{0.017500in}{0.017500in}}{\pgfqpoint{6.218282in}{3.580372in}} %
\pgfusepath{clip}%
\pgfsetbuttcap%
\pgfsetroundjoin%
\definecolor{currentfill}{rgb}{0.800830,0.250829,0.225696}%
\pgfsetfillcolor{currentfill}%
\pgfsetlinewidth{0.000000pt}%
\definecolor{currentstroke}{rgb}{0.000000,0.000000,0.000000}%
\pgfsetstrokecolor{currentstroke}%
\pgfsetdash{}{0pt}%
\pgfpathmoveto{\pgfqpoint{2.855513in}{2.451418in}}%
\pgfpathlineto{\pgfqpoint{3.306485in}{2.382499in}}%
\pgfpathlineto{\pgfqpoint{3.564492in}{2.541257in}}%
\pgfpathlineto{\pgfqpoint{3.116513in}{2.603812in}}%
\pgfpathlineto{\pgfqpoint{2.855513in}{2.451418in}}%
\pgfpathclose%
\pgfusepath{fill}%
\end{pgfscope}%
\begin{pgfscope}%
\pgfpathrectangle{\pgfqpoint{0.017500in}{0.017500in}}{\pgfqpoint{6.218282in}{3.580372in}} %
\pgfusepath{clip}%
\pgfsetbuttcap%
\pgfsetroundjoin%
\definecolor{currentfill}{rgb}{0.800830,0.250829,0.225696}%
\pgfsetfillcolor{currentfill}%
\pgfsetlinewidth{0.000000pt}%
\definecolor{currentstroke}{rgb}{0.000000,0.000000,0.000000}%
\pgfsetstrokecolor{currentstroke}%
\pgfsetdash{}{0pt}%
\pgfpathmoveto{\pgfqpoint{2.855513in}{2.451418in}}%
\pgfpathlineto{\pgfqpoint{3.306485in}{2.382499in}}%
\pgfpathlineto{\pgfqpoint{3.564492in}{2.541257in}}%
\pgfpathlineto{\pgfqpoint{3.116513in}{2.603812in}}%
\pgfpathlineto{\pgfqpoint{2.855513in}{2.451418in}}%
\pgfpathclose%
\pgfusepath{fill}%
\end{pgfscope}%
\begin{pgfscope}%
\pgfpathrectangle{\pgfqpoint{0.017500in}{0.017500in}}{\pgfqpoint{6.218282in}{3.580372in}} %
\pgfusepath{clip}%
\pgfsetbuttcap%
\pgfsetroundjoin%
\definecolor{currentfill}{rgb}{0.234377,0.305542,0.759680}%
\pgfsetfillcolor{currentfill}%
\pgfsetlinewidth{0.000000pt}%
\definecolor{currentstroke}{rgb}{0.000000,0.000000,0.000000}%
\pgfsetstrokecolor{currentstroke}%
\pgfsetdash{}{0pt}%
\pgfpathmoveto{\pgfqpoint{2.789950in}{0.798346in}}%
\pgfpathlineto{\pgfqpoint{3.043326in}{2.141782in}}%
\pgfpathlineto{\pgfqpoint{2.789950in}{0.798346in}}%
\pgfpathclose%
\pgfusepath{fill}%
\end{pgfscope}%
\begin{pgfscope}%
\pgfpathrectangle{\pgfqpoint{0.017500in}{0.017500in}}{\pgfqpoint{6.218282in}{3.580372in}} %
\pgfusepath{clip}%
\pgfsetbuttcap%
\pgfsetroundjoin%
\definecolor{currentfill}{rgb}{0.234377,0.305542,0.759680}%
\pgfsetfillcolor{currentfill}%
\pgfsetlinewidth{0.000000pt}%
\definecolor{currentstroke}{rgb}{0.000000,0.000000,0.000000}%
\pgfsetstrokecolor{currentstroke}%
\pgfsetdash{}{0pt}%
\pgfpathmoveto{\pgfqpoint{2.789950in}{0.798346in}}%
\pgfpathlineto{\pgfqpoint{3.043326in}{2.141782in}}%
\pgfpathlineto{\pgfqpoint{2.789950in}{0.798346in}}%
\pgfpathclose%
\pgfusepath{fill}%
\end{pgfscope}%
\begin{pgfscope}%
\pgfpathrectangle{\pgfqpoint{0.017500in}{0.017500in}}{\pgfqpoint{6.218282in}{3.580372in}} %
\pgfusepath{clip}%
\pgfsetbuttcap%
\pgfsetroundjoin%
\definecolor{currentfill}{rgb}{0.234377,0.305542,0.759680}%
\pgfsetfillcolor{currentfill}%
\pgfsetlinewidth{0.000000pt}%
\definecolor{currentstroke}{rgb}{0.000000,0.000000,0.000000}%
\pgfsetstrokecolor{currentstroke}%
\pgfsetdash{}{0pt}%
\pgfpathmoveto{\pgfqpoint{2.789950in}{0.798346in}}%
\pgfpathlineto{\pgfqpoint{3.043326in}{2.141782in}}%
\pgfpathlineto{\pgfqpoint{2.789950in}{0.798346in}}%
\pgfpathclose%
\pgfusepath{fill}%
\end{pgfscope}%
\begin{pgfscope}%
\pgfpathrectangle{\pgfqpoint{0.017500in}{0.017500in}}{\pgfqpoint{6.218282in}{3.580372in}} %
\pgfusepath{clip}%
\pgfsetbuttcap%
\pgfsetroundjoin%
\definecolor{currentfill}{rgb}{0.234377,0.305542,0.759680}%
\pgfsetfillcolor{currentfill}%
\pgfsetlinewidth{0.000000pt}%
\definecolor{currentstroke}{rgb}{0.000000,0.000000,0.000000}%
\pgfsetstrokecolor{currentstroke}%
\pgfsetdash{}{0pt}%
\pgfpathmoveto{\pgfqpoint{2.789950in}{0.798346in}}%
\pgfpathlineto{\pgfqpoint{3.043326in}{2.141782in}}%
\pgfpathlineto{\pgfqpoint{2.789950in}{0.798346in}}%
\pgfpathclose%
\pgfusepath{fill}%
\end{pgfscope}%
\begin{pgfscope}%
\pgfpathrectangle{\pgfqpoint{0.017500in}{0.017500in}}{\pgfqpoint{6.218282in}{3.580372in}} %
\pgfusepath{clip}%
\pgfsetbuttcap%
\pgfsetroundjoin%
\definecolor{currentfill}{rgb}{0.234377,0.305542,0.759680}%
\pgfsetfillcolor{currentfill}%
\pgfsetlinewidth{0.000000pt}%
\definecolor{currentstroke}{rgb}{0.000000,0.000000,0.000000}%
\pgfsetstrokecolor{currentstroke}%
\pgfsetdash{}{0pt}%
\pgfpathmoveto{\pgfqpoint{2.789950in}{0.798346in}}%
\pgfpathlineto{\pgfqpoint{3.043326in}{2.141782in}}%
\pgfpathlineto{\pgfqpoint{2.789950in}{0.798346in}}%
\pgfpathclose%
\pgfusepath{fill}%
\end{pgfscope}%
\begin{pgfscope}%
\pgfpathrectangle{\pgfqpoint{0.017500in}{0.017500in}}{\pgfqpoint{6.218282in}{3.580372in}} %
\pgfusepath{clip}%
\pgfsetbuttcap%
\pgfsetroundjoin%
\definecolor{currentfill}{rgb}{0.234377,0.305542,0.759680}%
\pgfsetfillcolor{currentfill}%
\pgfsetlinewidth{0.000000pt}%
\definecolor{currentstroke}{rgb}{0.000000,0.000000,0.000000}%
\pgfsetstrokecolor{currentstroke}%
\pgfsetdash{}{0pt}%
\pgfpathmoveto{\pgfqpoint{2.789950in}{0.798346in}}%
\pgfpathlineto{\pgfqpoint{3.043326in}{2.141782in}}%
\pgfpathlineto{\pgfqpoint{2.789950in}{0.798346in}}%
\pgfpathclose%
\pgfusepath{fill}%
\end{pgfscope}%
\begin{pgfscope}%
\pgfpathrectangle{\pgfqpoint{0.017500in}{0.017500in}}{\pgfqpoint{6.218282in}{3.580372in}} %
\pgfusepath{clip}%
\pgfsetbuttcap%
\pgfsetroundjoin%
\definecolor{currentfill}{rgb}{0.234377,0.305542,0.759680}%
\pgfsetfillcolor{currentfill}%
\pgfsetlinewidth{0.000000pt}%
\definecolor{currentstroke}{rgb}{0.000000,0.000000,0.000000}%
\pgfsetstrokecolor{currentstroke}%
\pgfsetdash{}{0pt}%
\pgfpathmoveto{\pgfqpoint{2.789950in}{0.798346in}}%
\pgfpathlineto{\pgfqpoint{3.043326in}{2.141782in}}%
\pgfpathlineto{\pgfqpoint{2.789950in}{0.798346in}}%
\pgfpathclose%
\pgfusepath{fill}%
\end{pgfscope}%
\begin{pgfscope}%
\pgfpathrectangle{\pgfqpoint{0.017500in}{0.017500in}}{\pgfqpoint{6.218282in}{3.580372in}} %
\pgfusepath{clip}%
\pgfsetbuttcap%
\pgfsetroundjoin%
\definecolor{currentfill}{rgb}{0.234377,0.305542,0.759680}%
\pgfsetfillcolor{currentfill}%
\pgfsetlinewidth{0.000000pt}%
\definecolor{currentstroke}{rgb}{0.000000,0.000000,0.000000}%
\pgfsetstrokecolor{currentstroke}%
\pgfsetdash{}{0pt}%
\pgfpathmoveto{\pgfqpoint{2.789950in}{0.798346in}}%
\pgfpathlineto{\pgfqpoint{3.043326in}{2.141782in}}%
\pgfpathlineto{\pgfqpoint{2.789950in}{0.798346in}}%
\pgfpathclose%
\pgfusepath{fill}%
\end{pgfscope}%
\begin{pgfscope}%
\pgfpathrectangle{\pgfqpoint{0.017500in}{0.017500in}}{\pgfqpoint{6.218282in}{3.580372in}} %
\pgfusepath{clip}%
\pgfsetbuttcap%
\pgfsetroundjoin%
\definecolor{currentfill}{rgb}{0.234377,0.305542,0.759680}%
\pgfsetfillcolor{currentfill}%
\pgfsetlinewidth{0.000000pt}%
\definecolor{currentstroke}{rgb}{0.000000,0.000000,0.000000}%
\pgfsetstrokecolor{currentstroke}%
\pgfsetdash{}{0pt}%
\pgfpathmoveto{\pgfqpoint{2.789950in}{0.798346in}}%
\pgfpathlineto{\pgfqpoint{3.043326in}{2.141782in}}%
\pgfpathlineto{\pgfqpoint{2.789950in}{0.798346in}}%
\pgfpathclose%
\pgfusepath{fill}%
\end{pgfscope}%
\begin{pgfscope}%
\pgfpathrectangle{\pgfqpoint{0.017500in}{0.017500in}}{\pgfqpoint{6.218282in}{3.580372in}} %
\pgfusepath{clip}%
\pgfsetbuttcap%
\pgfsetroundjoin%
\definecolor{currentfill}{rgb}{0.234377,0.305542,0.759680}%
\pgfsetfillcolor{currentfill}%
\pgfsetlinewidth{0.000000pt}%
\definecolor{currentstroke}{rgb}{0.000000,0.000000,0.000000}%
\pgfsetstrokecolor{currentstroke}%
\pgfsetdash{}{0pt}%
\pgfpathmoveto{\pgfqpoint{2.789950in}{0.798346in}}%
\pgfpathlineto{\pgfqpoint{3.043326in}{2.141782in}}%
\pgfpathlineto{\pgfqpoint{2.789950in}{0.798346in}}%
\pgfpathclose%
\pgfusepath{fill}%
\end{pgfscope}%
\begin{pgfscope}%
\pgfpathrectangle{\pgfqpoint{0.017500in}{0.017500in}}{\pgfqpoint{6.218282in}{3.580372in}} %
\pgfusepath{clip}%
\pgfsetbuttcap%
\pgfsetroundjoin%
\definecolor{currentfill}{rgb}{0.234377,0.305542,0.759680}%
\pgfsetfillcolor{currentfill}%
\pgfsetlinewidth{0.000000pt}%
\definecolor{currentstroke}{rgb}{0.000000,0.000000,0.000000}%
\pgfsetstrokecolor{currentstroke}%
\pgfsetdash{}{0pt}%
\pgfpathmoveto{\pgfqpoint{2.789950in}{0.798346in}}%
\pgfpathlineto{\pgfqpoint{3.043326in}{2.141782in}}%
\pgfpathlineto{\pgfqpoint{2.789950in}{0.798346in}}%
\pgfpathclose%
\pgfusepath{fill}%
\end{pgfscope}%
\begin{pgfscope}%
\pgfpathrectangle{\pgfqpoint{0.017500in}{0.017500in}}{\pgfqpoint{6.218282in}{3.580372in}} %
\pgfusepath{clip}%
\pgfsetbuttcap%
\pgfsetroundjoin%
\definecolor{currentfill}{rgb}{0.234377,0.305542,0.759680}%
\pgfsetfillcolor{currentfill}%
\pgfsetlinewidth{0.000000pt}%
\definecolor{currentstroke}{rgb}{0.000000,0.000000,0.000000}%
\pgfsetstrokecolor{currentstroke}%
\pgfsetdash{}{0pt}%
\pgfpathmoveto{\pgfqpoint{2.789950in}{0.798346in}}%
\pgfpathlineto{\pgfqpoint{3.043326in}{2.141782in}}%
\pgfpathlineto{\pgfqpoint{2.789950in}{0.798346in}}%
\pgfpathclose%
\pgfusepath{fill}%
\end{pgfscope}%
\begin{pgfscope}%
\pgfpathrectangle{\pgfqpoint{0.017500in}{0.017500in}}{\pgfqpoint{6.218282in}{3.580372in}} %
\pgfusepath{clip}%
\pgfsetbuttcap%
\pgfsetroundjoin%
\definecolor{currentfill}{rgb}{0.234377,0.305542,0.759680}%
\pgfsetfillcolor{currentfill}%
\pgfsetlinewidth{0.000000pt}%
\definecolor{currentstroke}{rgb}{0.000000,0.000000,0.000000}%
\pgfsetstrokecolor{currentstroke}%
\pgfsetdash{}{0pt}%
\pgfpathmoveto{\pgfqpoint{2.789950in}{0.798346in}}%
\pgfpathlineto{\pgfqpoint{3.043326in}{2.141782in}}%
\pgfpathlineto{\pgfqpoint{2.789950in}{0.798346in}}%
\pgfpathclose%
\pgfusepath{fill}%
\end{pgfscope}%
\begin{pgfscope}%
\pgfpathrectangle{\pgfqpoint{0.017500in}{0.017500in}}{\pgfqpoint{6.218282in}{3.580372in}} %
\pgfusepath{clip}%
\pgfsetbuttcap%
\pgfsetroundjoin%
\definecolor{currentfill}{rgb}{0.234377,0.305542,0.759680}%
\pgfsetfillcolor{currentfill}%
\pgfsetlinewidth{0.000000pt}%
\definecolor{currentstroke}{rgb}{0.000000,0.000000,0.000000}%
\pgfsetstrokecolor{currentstroke}%
\pgfsetdash{}{0pt}%
\pgfpathmoveto{\pgfqpoint{2.789950in}{0.798346in}}%
\pgfpathlineto{\pgfqpoint{3.043326in}{2.141782in}}%
\pgfpathlineto{\pgfqpoint{2.789950in}{0.798346in}}%
\pgfpathclose%
\pgfusepath{fill}%
\end{pgfscope}%
\begin{pgfscope}%
\pgfpathrectangle{\pgfqpoint{0.017500in}{0.017500in}}{\pgfqpoint{6.218282in}{3.580372in}} %
\pgfusepath{clip}%
\pgfsetbuttcap%
\pgfsetroundjoin%
\definecolor{currentfill}{rgb}{0.234377,0.305542,0.759680}%
\pgfsetfillcolor{currentfill}%
\pgfsetlinewidth{0.000000pt}%
\definecolor{currentstroke}{rgb}{0.000000,0.000000,0.000000}%
\pgfsetstrokecolor{currentstroke}%
\pgfsetdash{}{0pt}%
\pgfpathmoveto{\pgfqpoint{2.789950in}{0.798346in}}%
\pgfpathlineto{\pgfqpoint{3.043326in}{2.141782in}}%
\pgfpathlineto{\pgfqpoint{2.789950in}{0.798346in}}%
\pgfpathclose%
\pgfusepath{fill}%
\end{pgfscope}%
\begin{pgfscope}%
\pgfpathrectangle{\pgfqpoint{0.017500in}{0.017500in}}{\pgfqpoint{6.218282in}{3.580372in}} %
\pgfusepath{clip}%
\pgfsetbuttcap%
\pgfsetroundjoin%
\definecolor{currentfill}{rgb}{0.234377,0.305542,0.759680}%
\pgfsetfillcolor{currentfill}%
\pgfsetlinewidth{0.000000pt}%
\definecolor{currentstroke}{rgb}{0.000000,0.000000,0.000000}%
\pgfsetstrokecolor{currentstroke}%
\pgfsetdash{}{0pt}%
\pgfpathmoveto{\pgfqpoint{2.789950in}{0.798346in}}%
\pgfpathlineto{\pgfqpoint{3.043326in}{2.141782in}}%
\pgfpathlineto{\pgfqpoint{2.789950in}{0.798346in}}%
\pgfpathclose%
\pgfusepath{fill}%
\end{pgfscope}%
\begin{pgfscope}%
\pgfpathrectangle{\pgfqpoint{0.017500in}{0.017500in}}{\pgfqpoint{6.218282in}{3.580372in}} %
\pgfusepath{clip}%
\pgfsetbuttcap%
\pgfsetroundjoin%
\definecolor{currentfill}{rgb}{0.234377,0.305542,0.759680}%
\pgfsetfillcolor{currentfill}%
\pgfsetlinewidth{0.000000pt}%
\definecolor{currentstroke}{rgb}{0.000000,0.000000,0.000000}%
\pgfsetstrokecolor{currentstroke}%
\pgfsetdash{}{0pt}%
\pgfpathmoveto{\pgfqpoint{2.789950in}{0.798346in}}%
\pgfpathlineto{\pgfqpoint{3.043326in}{2.141782in}}%
\pgfpathlineto{\pgfqpoint{2.789950in}{0.798346in}}%
\pgfpathclose%
\pgfusepath{fill}%
\end{pgfscope}%
\begin{pgfscope}%
\pgfpathrectangle{\pgfqpoint{0.017500in}{0.017500in}}{\pgfqpoint{6.218282in}{3.580372in}} %
\pgfusepath{clip}%
\pgfsetbuttcap%
\pgfsetroundjoin%
\definecolor{currentfill}{rgb}{0.234377,0.305542,0.759680}%
\pgfsetfillcolor{currentfill}%
\pgfsetlinewidth{0.000000pt}%
\definecolor{currentstroke}{rgb}{0.000000,0.000000,0.000000}%
\pgfsetstrokecolor{currentstroke}%
\pgfsetdash{}{0pt}%
\pgfpathmoveto{\pgfqpoint{2.789950in}{0.798346in}}%
\pgfpathlineto{\pgfqpoint{3.043326in}{2.141782in}}%
\pgfpathlineto{\pgfqpoint{2.789950in}{0.798346in}}%
\pgfpathclose%
\pgfusepath{fill}%
\end{pgfscope}%
\begin{pgfscope}%
\pgfpathrectangle{\pgfqpoint{0.017500in}{0.017500in}}{\pgfqpoint{6.218282in}{3.580372in}} %
\pgfusepath{clip}%
\pgfsetbuttcap%
\pgfsetroundjoin%
\definecolor{currentfill}{rgb}{0.234377,0.305542,0.759680}%
\pgfsetfillcolor{currentfill}%
\pgfsetlinewidth{0.000000pt}%
\definecolor{currentstroke}{rgb}{0.000000,0.000000,0.000000}%
\pgfsetstrokecolor{currentstroke}%
\pgfsetdash{}{0pt}%
\pgfpathmoveto{\pgfqpoint{2.789950in}{0.798346in}}%
\pgfpathlineto{\pgfqpoint{3.043326in}{2.141782in}}%
\pgfpathlineto{\pgfqpoint{2.789950in}{0.798346in}}%
\pgfpathclose%
\pgfusepath{fill}%
\end{pgfscope}%
\begin{pgfscope}%
\pgfpathrectangle{\pgfqpoint{0.017500in}{0.017500in}}{\pgfqpoint{6.218282in}{3.580372in}} %
\pgfusepath{clip}%
\pgfsetbuttcap%
\pgfsetroundjoin%
\definecolor{currentfill}{rgb}{0.234377,0.305542,0.759680}%
\pgfsetfillcolor{currentfill}%
\pgfsetlinewidth{0.000000pt}%
\definecolor{currentstroke}{rgb}{0.000000,0.000000,0.000000}%
\pgfsetstrokecolor{currentstroke}%
\pgfsetdash{}{0pt}%
\pgfpathmoveto{\pgfqpoint{2.789950in}{0.798346in}}%
\pgfpathlineto{\pgfqpoint{3.043326in}{2.141782in}}%
\pgfpathlineto{\pgfqpoint{2.789950in}{0.798346in}}%
\pgfpathclose%
\pgfusepath{fill}%
\end{pgfscope}%
\begin{pgfscope}%
\pgfpathrectangle{\pgfqpoint{0.017500in}{0.017500in}}{\pgfqpoint{6.218282in}{3.580372in}} %
\pgfusepath{clip}%
\pgfsetbuttcap%
\pgfsetroundjoin%
\definecolor{currentfill}{rgb}{0.234377,0.305542,0.759680}%
\pgfsetfillcolor{currentfill}%
\pgfsetlinewidth{0.000000pt}%
\definecolor{currentstroke}{rgb}{0.000000,0.000000,0.000000}%
\pgfsetstrokecolor{currentstroke}%
\pgfsetdash{}{0pt}%
\pgfpathmoveto{\pgfqpoint{2.789950in}{0.798346in}}%
\pgfpathlineto{\pgfqpoint{3.043326in}{2.141782in}}%
\pgfpathlineto{\pgfqpoint{2.789950in}{0.798346in}}%
\pgfpathclose%
\pgfusepath{fill}%
\end{pgfscope}%
\begin{pgfscope}%
\pgfpathrectangle{\pgfqpoint{0.017500in}{0.017500in}}{\pgfqpoint{6.218282in}{3.580372in}} %
\pgfusepath{clip}%
\pgfsetbuttcap%
\pgfsetroundjoin%
\definecolor{currentfill}{rgb}{0.234377,0.305542,0.759680}%
\pgfsetfillcolor{currentfill}%
\pgfsetlinewidth{0.000000pt}%
\definecolor{currentstroke}{rgb}{0.000000,0.000000,0.000000}%
\pgfsetstrokecolor{currentstroke}%
\pgfsetdash{}{0pt}%
\pgfpathmoveto{\pgfqpoint{2.789950in}{0.798346in}}%
\pgfpathlineto{\pgfqpoint{3.043326in}{2.141782in}}%
\pgfpathlineto{\pgfqpoint{2.789950in}{0.798346in}}%
\pgfpathclose%
\pgfusepath{fill}%
\end{pgfscope}%
\begin{pgfscope}%
\pgfpathrectangle{\pgfqpoint{0.017500in}{0.017500in}}{\pgfqpoint{6.218282in}{3.580372in}} %
\pgfusepath{clip}%
\pgfsetbuttcap%
\pgfsetroundjoin%
\definecolor{currentfill}{rgb}{0.234377,0.305542,0.759680}%
\pgfsetfillcolor{currentfill}%
\pgfsetlinewidth{0.000000pt}%
\definecolor{currentstroke}{rgb}{0.000000,0.000000,0.000000}%
\pgfsetstrokecolor{currentstroke}%
\pgfsetdash{}{0pt}%
\pgfpathmoveto{\pgfqpoint{2.789950in}{0.798346in}}%
\pgfpathlineto{\pgfqpoint{3.043326in}{2.141782in}}%
\pgfpathlineto{\pgfqpoint{2.789950in}{0.798346in}}%
\pgfpathclose%
\pgfusepath{fill}%
\end{pgfscope}%
\begin{pgfscope}%
\pgfpathrectangle{\pgfqpoint{0.017500in}{0.017500in}}{\pgfqpoint{6.218282in}{3.580372in}} %
\pgfusepath{clip}%
\pgfsetbuttcap%
\pgfsetroundjoin%
\definecolor{currentfill}{rgb}{0.234377,0.305542,0.759680}%
\pgfsetfillcolor{currentfill}%
\pgfsetlinewidth{0.000000pt}%
\definecolor{currentstroke}{rgb}{0.000000,0.000000,0.000000}%
\pgfsetstrokecolor{currentstroke}%
\pgfsetdash{}{0pt}%
\pgfpathmoveto{\pgfqpoint{2.789950in}{0.798346in}}%
\pgfpathlineto{\pgfqpoint{3.043326in}{2.141782in}}%
\pgfpathlineto{\pgfqpoint{2.789950in}{0.798346in}}%
\pgfpathclose%
\pgfusepath{fill}%
\end{pgfscope}%
\begin{pgfscope}%
\pgfpathrectangle{\pgfqpoint{0.017500in}{0.017500in}}{\pgfqpoint{6.218282in}{3.580372in}} %
\pgfusepath{clip}%
\pgfsetbuttcap%
\pgfsetroundjoin%
\definecolor{currentfill}{rgb}{0.234377,0.305542,0.759680}%
\pgfsetfillcolor{currentfill}%
\pgfsetlinewidth{0.000000pt}%
\definecolor{currentstroke}{rgb}{0.000000,0.000000,0.000000}%
\pgfsetstrokecolor{currentstroke}%
\pgfsetdash{}{0pt}%
\pgfpathmoveto{\pgfqpoint{2.789950in}{0.798346in}}%
\pgfpathlineto{\pgfqpoint{3.043326in}{2.141782in}}%
\pgfpathlineto{\pgfqpoint{2.789950in}{0.798346in}}%
\pgfpathclose%
\pgfusepath{fill}%
\end{pgfscope}%
\begin{pgfscope}%
\pgfpathrectangle{\pgfqpoint{0.017500in}{0.017500in}}{\pgfqpoint{6.218282in}{3.580372in}} %
\pgfusepath{clip}%
\pgfsetbuttcap%
\pgfsetroundjoin%
\definecolor{currentfill}{rgb}{0.234377,0.305542,0.759680}%
\pgfsetfillcolor{currentfill}%
\pgfsetlinewidth{0.000000pt}%
\definecolor{currentstroke}{rgb}{0.000000,0.000000,0.000000}%
\pgfsetstrokecolor{currentstroke}%
\pgfsetdash{}{0pt}%
\pgfpathmoveto{\pgfqpoint{2.789950in}{0.798346in}}%
\pgfpathlineto{\pgfqpoint{3.043326in}{2.141782in}}%
\pgfpathlineto{\pgfqpoint{2.789950in}{0.798346in}}%
\pgfpathclose%
\pgfusepath{fill}%
\end{pgfscope}%
\begin{pgfscope}%
\pgfpathrectangle{\pgfqpoint{0.017500in}{0.017500in}}{\pgfqpoint{6.218282in}{3.580372in}} %
\pgfusepath{clip}%
\pgfsetbuttcap%
\pgfsetroundjoin%
\definecolor{currentfill}{rgb}{0.234377,0.305542,0.759680}%
\pgfsetfillcolor{currentfill}%
\pgfsetlinewidth{0.000000pt}%
\definecolor{currentstroke}{rgb}{0.000000,0.000000,0.000000}%
\pgfsetstrokecolor{currentstroke}%
\pgfsetdash{}{0pt}%
\pgfpathmoveto{\pgfqpoint{2.789950in}{0.798346in}}%
\pgfpathlineto{\pgfqpoint{3.043326in}{2.141782in}}%
\pgfpathlineto{\pgfqpoint{2.789950in}{0.798346in}}%
\pgfpathclose%
\pgfusepath{fill}%
\end{pgfscope}%
\begin{pgfscope}%
\pgfpathrectangle{\pgfqpoint{0.017500in}{0.017500in}}{\pgfqpoint{6.218282in}{3.580372in}} %
\pgfusepath{clip}%
\pgfsetbuttcap%
\pgfsetroundjoin%
\definecolor{currentfill}{rgb}{0.234377,0.305542,0.759680}%
\pgfsetfillcolor{currentfill}%
\pgfsetlinewidth{0.000000pt}%
\definecolor{currentstroke}{rgb}{0.000000,0.000000,0.000000}%
\pgfsetstrokecolor{currentstroke}%
\pgfsetdash{}{0pt}%
\pgfpathmoveto{\pgfqpoint{2.789950in}{0.798346in}}%
\pgfpathlineto{\pgfqpoint{3.043326in}{2.141782in}}%
\pgfpathlineto{\pgfqpoint{2.789950in}{0.798346in}}%
\pgfpathclose%
\pgfusepath{fill}%
\end{pgfscope}%
\begin{pgfscope}%
\pgfpathrectangle{\pgfqpoint{0.017500in}{0.017500in}}{\pgfqpoint{6.218282in}{3.580372in}} %
\pgfusepath{clip}%
\pgfsetbuttcap%
\pgfsetroundjoin%
\definecolor{currentfill}{rgb}{0.234377,0.305542,0.759680}%
\pgfsetfillcolor{currentfill}%
\pgfsetlinewidth{0.000000pt}%
\definecolor{currentstroke}{rgb}{0.000000,0.000000,0.000000}%
\pgfsetstrokecolor{currentstroke}%
\pgfsetdash{}{0pt}%
\pgfpathmoveto{\pgfqpoint{2.789950in}{0.798346in}}%
\pgfpathlineto{\pgfqpoint{3.043326in}{2.141782in}}%
\pgfpathlineto{\pgfqpoint{2.789950in}{0.798346in}}%
\pgfpathclose%
\pgfusepath{fill}%
\end{pgfscope}%
\begin{pgfscope}%
\pgfpathrectangle{\pgfqpoint{0.017500in}{0.017500in}}{\pgfqpoint{6.218282in}{3.580372in}} %
\pgfusepath{clip}%
\pgfsetbuttcap%
\pgfsetroundjoin%
\definecolor{currentfill}{rgb}{0.234377,0.305542,0.759680}%
\pgfsetfillcolor{currentfill}%
\pgfsetlinewidth{0.000000pt}%
\definecolor{currentstroke}{rgb}{0.000000,0.000000,0.000000}%
\pgfsetstrokecolor{currentstroke}%
\pgfsetdash{}{0pt}%
\pgfpathmoveto{\pgfqpoint{2.789950in}{0.798346in}}%
\pgfpathlineto{\pgfqpoint{3.043326in}{2.141782in}}%
\pgfpathlineto{\pgfqpoint{2.789950in}{0.798346in}}%
\pgfpathclose%
\pgfusepath{fill}%
\end{pgfscope}%
\begin{pgfscope}%
\pgfpathrectangle{\pgfqpoint{0.017500in}{0.017500in}}{\pgfqpoint{6.218282in}{3.580372in}} %
\pgfusepath{clip}%
\pgfsetbuttcap%
\pgfsetroundjoin%
\definecolor{currentfill}{rgb}{0.899534,0.440692,0.344107}%
\pgfsetfillcolor{currentfill}%
\pgfsetlinewidth{0.000000pt}%
\definecolor{currentstroke}{rgb}{0.000000,0.000000,0.000000}%
\pgfsetstrokecolor{currentstroke}%
\pgfsetdash{}{0pt}%
\pgfpathmoveto{\pgfqpoint{2.143341in}{2.291288in}}%
\pgfpathlineto{\pgfqpoint{2.410872in}{2.508083in}}%
\pgfpathlineto{\pgfqpoint{2.143341in}{2.291288in}}%
\pgfpathclose%
\pgfusepath{fill}%
\end{pgfscope}%
\begin{pgfscope}%
\pgfpathrectangle{\pgfqpoint{0.017500in}{0.017500in}}{\pgfqpoint{6.218282in}{3.580372in}} %
\pgfusepath{clip}%
\pgfsetbuttcap%
\pgfsetroundjoin%
\definecolor{currentfill}{rgb}{0.899534,0.440692,0.344107}%
\pgfsetfillcolor{currentfill}%
\pgfsetlinewidth{0.000000pt}%
\definecolor{currentstroke}{rgb}{0.000000,0.000000,0.000000}%
\pgfsetstrokecolor{currentstroke}%
\pgfsetdash{}{0pt}%
\pgfpathmoveto{\pgfqpoint{2.143341in}{2.291288in}}%
\pgfpathlineto{\pgfqpoint{2.410872in}{2.508083in}}%
\pgfpathlineto{\pgfqpoint{2.143341in}{2.291288in}}%
\pgfpathclose%
\pgfusepath{fill}%
\end{pgfscope}%
\begin{pgfscope}%
\pgfpathrectangle{\pgfqpoint{0.017500in}{0.017500in}}{\pgfqpoint{6.218282in}{3.580372in}} %
\pgfusepath{clip}%
\pgfsetbuttcap%
\pgfsetroundjoin%
\definecolor{currentfill}{rgb}{0.899534,0.440692,0.344107}%
\pgfsetfillcolor{currentfill}%
\pgfsetlinewidth{0.000000pt}%
\definecolor{currentstroke}{rgb}{0.000000,0.000000,0.000000}%
\pgfsetstrokecolor{currentstroke}%
\pgfsetdash{}{0pt}%
\pgfpathmoveto{\pgfqpoint{2.143341in}{2.291288in}}%
\pgfpathlineto{\pgfqpoint{2.410872in}{2.508083in}}%
\pgfpathlineto{\pgfqpoint{2.143341in}{2.291288in}}%
\pgfpathclose%
\pgfusepath{fill}%
\end{pgfscope}%
\begin{pgfscope}%
\pgfpathrectangle{\pgfqpoint{0.017500in}{0.017500in}}{\pgfqpoint{6.218282in}{3.580372in}} %
\pgfusepath{clip}%
\pgfsetbuttcap%
\pgfsetroundjoin%
\definecolor{currentfill}{rgb}{0.899534,0.440692,0.344107}%
\pgfsetfillcolor{currentfill}%
\pgfsetlinewidth{0.000000pt}%
\definecolor{currentstroke}{rgb}{0.000000,0.000000,0.000000}%
\pgfsetstrokecolor{currentstroke}%
\pgfsetdash{}{0pt}%
\pgfpathmoveto{\pgfqpoint{2.143341in}{2.291288in}}%
\pgfpathlineto{\pgfqpoint{2.410872in}{2.508083in}}%
\pgfpathlineto{\pgfqpoint{2.143341in}{2.291288in}}%
\pgfpathclose%
\pgfusepath{fill}%
\end{pgfscope}%
\begin{pgfscope}%
\pgfpathrectangle{\pgfqpoint{0.017500in}{0.017500in}}{\pgfqpoint{6.218282in}{3.580372in}} %
\pgfusepath{clip}%
\pgfsetbuttcap%
\pgfsetroundjoin%
\definecolor{currentfill}{rgb}{0.899534,0.440692,0.344107}%
\pgfsetfillcolor{currentfill}%
\pgfsetlinewidth{0.000000pt}%
\definecolor{currentstroke}{rgb}{0.000000,0.000000,0.000000}%
\pgfsetstrokecolor{currentstroke}%
\pgfsetdash{}{0pt}%
\pgfpathmoveto{\pgfqpoint{2.143341in}{2.291288in}}%
\pgfpathlineto{\pgfqpoint{2.410872in}{2.508083in}}%
\pgfpathlineto{\pgfqpoint{2.143341in}{2.291288in}}%
\pgfpathclose%
\pgfusepath{fill}%
\end{pgfscope}%
\begin{pgfscope}%
\pgfpathrectangle{\pgfqpoint{0.017500in}{0.017500in}}{\pgfqpoint{6.218282in}{3.580372in}} %
\pgfusepath{clip}%
\pgfsetbuttcap%
\pgfsetroundjoin%
\definecolor{currentfill}{rgb}{0.899534,0.440692,0.344107}%
\pgfsetfillcolor{currentfill}%
\pgfsetlinewidth{0.000000pt}%
\definecolor{currentstroke}{rgb}{0.000000,0.000000,0.000000}%
\pgfsetstrokecolor{currentstroke}%
\pgfsetdash{}{0pt}%
\pgfpathmoveto{\pgfqpoint{2.143341in}{2.291288in}}%
\pgfpathlineto{\pgfqpoint{2.410872in}{2.508083in}}%
\pgfpathlineto{\pgfqpoint{2.143341in}{2.291288in}}%
\pgfpathclose%
\pgfusepath{fill}%
\end{pgfscope}%
\begin{pgfscope}%
\pgfpathrectangle{\pgfqpoint{0.017500in}{0.017500in}}{\pgfqpoint{6.218282in}{3.580372in}} %
\pgfusepath{clip}%
\pgfsetbuttcap%
\pgfsetroundjoin%
\definecolor{currentfill}{rgb}{0.899534,0.440692,0.344107}%
\pgfsetfillcolor{currentfill}%
\pgfsetlinewidth{0.000000pt}%
\definecolor{currentstroke}{rgb}{0.000000,0.000000,0.000000}%
\pgfsetstrokecolor{currentstroke}%
\pgfsetdash{}{0pt}%
\pgfpathmoveto{\pgfqpoint{2.143341in}{2.291288in}}%
\pgfpathlineto{\pgfqpoint{2.410872in}{2.508083in}}%
\pgfpathlineto{\pgfqpoint{2.143341in}{2.291288in}}%
\pgfpathclose%
\pgfusepath{fill}%
\end{pgfscope}%
\begin{pgfscope}%
\pgfpathrectangle{\pgfqpoint{0.017500in}{0.017500in}}{\pgfqpoint{6.218282in}{3.580372in}} %
\pgfusepath{clip}%
\pgfsetbuttcap%
\pgfsetroundjoin%
\definecolor{currentfill}{rgb}{0.899534,0.440692,0.344107}%
\pgfsetfillcolor{currentfill}%
\pgfsetlinewidth{0.000000pt}%
\definecolor{currentstroke}{rgb}{0.000000,0.000000,0.000000}%
\pgfsetstrokecolor{currentstroke}%
\pgfsetdash{}{0pt}%
\pgfpathmoveto{\pgfqpoint{2.143341in}{2.291288in}}%
\pgfpathlineto{\pgfqpoint{2.410872in}{2.508083in}}%
\pgfpathlineto{\pgfqpoint{2.143341in}{2.291288in}}%
\pgfpathclose%
\pgfusepath{fill}%
\end{pgfscope}%
\begin{pgfscope}%
\pgfpathrectangle{\pgfqpoint{0.017500in}{0.017500in}}{\pgfqpoint{6.218282in}{3.580372in}} %
\pgfusepath{clip}%
\pgfsetbuttcap%
\pgfsetroundjoin%
\definecolor{currentfill}{rgb}{0.899534,0.440692,0.344107}%
\pgfsetfillcolor{currentfill}%
\pgfsetlinewidth{0.000000pt}%
\definecolor{currentstroke}{rgb}{0.000000,0.000000,0.000000}%
\pgfsetstrokecolor{currentstroke}%
\pgfsetdash{}{0pt}%
\pgfpathmoveto{\pgfqpoint{2.143341in}{2.291288in}}%
\pgfpathlineto{\pgfqpoint{2.410872in}{2.508083in}}%
\pgfpathlineto{\pgfqpoint{2.143341in}{2.291288in}}%
\pgfpathclose%
\pgfusepath{fill}%
\end{pgfscope}%
\begin{pgfscope}%
\pgfpathrectangle{\pgfqpoint{0.017500in}{0.017500in}}{\pgfqpoint{6.218282in}{3.580372in}} %
\pgfusepath{clip}%
\pgfsetbuttcap%
\pgfsetroundjoin%
\definecolor{currentfill}{rgb}{0.899534,0.440692,0.344107}%
\pgfsetfillcolor{currentfill}%
\pgfsetlinewidth{0.000000pt}%
\definecolor{currentstroke}{rgb}{0.000000,0.000000,0.000000}%
\pgfsetstrokecolor{currentstroke}%
\pgfsetdash{}{0pt}%
\pgfpathmoveto{\pgfqpoint{2.143341in}{2.291288in}}%
\pgfpathlineto{\pgfqpoint{2.410872in}{2.508083in}}%
\pgfpathlineto{\pgfqpoint{2.143341in}{2.291288in}}%
\pgfpathclose%
\pgfusepath{fill}%
\end{pgfscope}%
\begin{pgfscope}%
\pgfpathrectangle{\pgfqpoint{0.017500in}{0.017500in}}{\pgfqpoint{6.218282in}{3.580372in}} %
\pgfusepath{clip}%
\pgfsetbuttcap%
\pgfsetroundjoin%
\definecolor{currentfill}{rgb}{0.899534,0.440692,0.344107}%
\pgfsetfillcolor{currentfill}%
\pgfsetlinewidth{0.000000pt}%
\definecolor{currentstroke}{rgb}{0.000000,0.000000,0.000000}%
\pgfsetstrokecolor{currentstroke}%
\pgfsetdash{}{0pt}%
\pgfpathmoveto{\pgfqpoint{2.143341in}{2.291288in}}%
\pgfpathlineto{\pgfqpoint{2.410872in}{2.508083in}}%
\pgfpathlineto{\pgfqpoint{2.143341in}{2.291288in}}%
\pgfpathclose%
\pgfusepath{fill}%
\end{pgfscope}%
\begin{pgfscope}%
\pgfpathrectangle{\pgfqpoint{0.017500in}{0.017500in}}{\pgfqpoint{6.218282in}{3.580372in}} %
\pgfusepath{clip}%
\pgfsetbuttcap%
\pgfsetroundjoin%
\definecolor{currentfill}{rgb}{0.899534,0.440692,0.344107}%
\pgfsetfillcolor{currentfill}%
\pgfsetlinewidth{0.000000pt}%
\definecolor{currentstroke}{rgb}{0.000000,0.000000,0.000000}%
\pgfsetstrokecolor{currentstroke}%
\pgfsetdash{}{0pt}%
\pgfpathmoveto{\pgfqpoint{2.143341in}{2.291288in}}%
\pgfpathlineto{\pgfqpoint{2.410872in}{2.508083in}}%
\pgfpathlineto{\pgfqpoint{2.143341in}{2.291288in}}%
\pgfpathclose%
\pgfusepath{fill}%
\end{pgfscope}%
\begin{pgfscope}%
\pgfpathrectangle{\pgfqpoint{0.017500in}{0.017500in}}{\pgfqpoint{6.218282in}{3.580372in}} %
\pgfusepath{clip}%
\pgfsetbuttcap%
\pgfsetroundjoin%
\definecolor{currentfill}{rgb}{0.899534,0.440692,0.344107}%
\pgfsetfillcolor{currentfill}%
\pgfsetlinewidth{0.000000pt}%
\definecolor{currentstroke}{rgb}{0.000000,0.000000,0.000000}%
\pgfsetstrokecolor{currentstroke}%
\pgfsetdash{}{0pt}%
\pgfpathmoveto{\pgfqpoint{2.143341in}{2.291288in}}%
\pgfpathlineto{\pgfqpoint{2.410872in}{2.508083in}}%
\pgfpathlineto{\pgfqpoint{2.143341in}{2.291288in}}%
\pgfpathclose%
\pgfusepath{fill}%
\end{pgfscope}%
\begin{pgfscope}%
\pgfpathrectangle{\pgfqpoint{0.017500in}{0.017500in}}{\pgfqpoint{6.218282in}{3.580372in}} %
\pgfusepath{clip}%
\pgfsetbuttcap%
\pgfsetroundjoin%
\definecolor{currentfill}{rgb}{0.899534,0.440692,0.344107}%
\pgfsetfillcolor{currentfill}%
\pgfsetlinewidth{0.000000pt}%
\definecolor{currentstroke}{rgb}{0.000000,0.000000,0.000000}%
\pgfsetstrokecolor{currentstroke}%
\pgfsetdash{}{0pt}%
\pgfpathmoveto{\pgfqpoint{2.143341in}{2.291288in}}%
\pgfpathlineto{\pgfqpoint{2.410872in}{2.508083in}}%
\pgfpathlineto{\pgfqpoint{2.143341in}{2.291288in}}%
\pgfpathclose%
\pgfusepath{fill}%
\end{pgfscope}%
\begin{pgfscope}%
\pgfpathrectangle{\pgfqpoint{0.017500in}{0.017500in}}{\pgfqpoint{6.218282in}{3.580372in}} %
\pgfusepath{clip}%
\pgfsetbuttcap%
\pgfsetroundjoin%
\definecolor{currentfill}{rgb}{0.899534,0.440692,0.344107}%
\pgfsetfillcolor{currentfill}%
\pgfsetlinewidth{0.000000pt}%
\definecolor{currentstroke}{rgb}{0.000000,0.000000,0.000000}%
\pgfsetstrokecolor{currentstroke}%
\pgfsetdash{}{0pt}%
\pgfpathmoveto{\pgfqpoint{2.143341in}{2.291288in}}%
\pgfpathlineto{\pgfqpoint{2.410872in}{2.508083in}}%
\pgfpathlineto{\pgfqpoint{2.143341in}{2.291288in}}%
\pgfpathclose%
\pgfusepath{fill}%
\end{pgfscope}%
\begin{pgfscope}%
\pgfpathrectangle{\pgfqpoint{0.017500in}{0.017500in}}{\pgfqpoint{6.218282in}{3.580372in}} %
\pgfusepath{clip}%
\pgfsetbuttcap%
\pgfsetroundjoin%
\definecolor{currentfill}{rgb}{0.899534,0.440692,0.344107}%
\pgfsetfillcolor{currentfill}%
\pgfsetlinewidth{0.000000pt}%
\definecolor{currentstroke}{rgb}{0.000000,0.000000,0.000000}%
\pgfsetstrokecolor{currentstroke}%
\pgfsetdash{}{0pt}%
\pgfpathmoveto{\pgfqpoint{2.143341in}{2.291288in}}%
\pgfpathlineto{\pgfqpoint{2.410872in}{2.508083in}}%
\pgfpathlineto{\pgfqpoint{2.143341in}{2.291288in}}%
\pgfpathclose%
\pgfusepath{fill}%
\end{pgfscope}%
\begin{pgfscope}%
\pgfpathrectangle{\pgfqpoint{0.017500in}{0.017500in}}{\pgfqpoint{6.218282in}{3.580372in}} %
\pgfusepath{clip}%
\pgfsetbuttcap%
\pgfsetroundjoin%
\definecolor{currentfill}{rgb}{0.899534,0.440692,0.344107}%
\pgfsetfillcolor{currentfill}%
\pgfsetlinewidth{0.000000pt}%
\definecolor{currentstroke}{rgb}{0.000000,0.000000,0.000000}%
\pgfsetstrokecolor{currentstroke}%
\pgfsetdash{}{0pt}%
\pgfpathmoveto{\pgfqpoint{2.143341in}{2.291288in}}%
\pgfpathlineto{\pgfqpoint{2.410872in}{2.508083in}}%
\pgfpathlineto{\pgfqpoint{2.143341in}{2.291288in}}%
\pgfpathclose%
\pgfusepath{fill}%
\end{pgfscope}%
\begin{pgfscope}%
\pgfpathrectangle{\pgfqpoint{0.017500in}{0.017500in}}{\pgfqpoint{6.218282in}{3.580372in}} %
\pgfusepath{clip}%
\pgfsetbuttcap%
\pgfsetroundjoin%
\definecolor{currentfill}{rgb}{0.899534,0.440692,0.344107}%
\pgfsetfillcolor{currentfill}%
\pgfsetlinewidth{0.000000pt}%
\definecolor{currentstroke}{rgb}{0.000000,0.000000,0.000000}%
\pgfsetstrokecolor{currentstroke}%
\pgfsetdash{}{0pt}%
\pgfpathmoveto{\pgfqpoint{2.143341in}{2.291288in}}%
\pgfpathlineto{\pgfqpoint{2.410872in}{2.508083in}}%
\pgfpathlineto{\pgfqpoint{2.143341in}{2.291288in}}%
\pgfpathclose%
\pgfusepath{fill}%
\end{pgfscope}%
\begin{pgfscope}%
\pgfpathrectangle{\pgfqpoint{0.017500in}{0.017500in}}{\pgfqpoint{6.218282in}{3.580372in}} %
\pgfusepath{clip}%
\pgfsetbuttcap%
\pgfsetroundjoin%
\definecolor{currentfill}{rgb}{0.899534,0.440692,0.344107}%
\pgfsetfillcolor{currentfill}%
\pgfsetlinewidth{0.000000pt}%
\definecolor{currentstroke}{rgb}{0.000000,0.000000,0.000000}%
\pgfsetstrokecolor{currentstroke}%
\pgfsetdash{}{0pt}%
\pgfpathmoveto{\pgfqpoint{2.143341in}{2.291288in}}%
\pgfpathlineto{\pgfqpoint{2.410872in}{2.508083in}}%
\pgfpathlineto{\pgfqpoint{2.143341in}{2.291288in}}%
\pgfpathclose%
\pgfusepath{fill}%
\end{pgfscope}%
\begin{pgfscope}%
\pgfpathrectangle{\pgfqpoint{0.017500in}{0.017500in}}{\pgfqpoint{6.218282in}{3.580372in}} %
\pgfusepath{clip}%
\pgfsetbuttcap%
\pgfsetroundjoin%
\definecolor{currentfill}{rgb}{0.899534,0.440692,0.344107}%
\pgfsetfillcolor{currentfill}%
\pgfsetlinewidth{0.000000pt}%
\definecolor{currentstroke}{rgb}{0.000000,0.000000,0.000000}%
\pgfsetstrokecolor{currentstroke}%
\pgfsetdash{}{0pt}%
\pgfpathmoveto{\pgfqpoint{2.143341in}{2.291288in}}%
\pgfpathlineto{\pgfqpoint{2.410872in}{2.508083in}}%
\pgfpathlineto{\pgfqpoint{2.143341in}{2.291288in}}%
\pgfpathclose%
\pgfusepath{fill}%
\end{pgfscope}%
\begin{pgfscope}%
\pgfpathrectangle{\pgfqpoint{0.017500in}{0.017500in}}{\pgfqpoint{6.218282in}{3.580372in}} %
\pgfusepath{clip}%
\pgfsetbuttcap%
\pgfsetroundjoin%
\definecolor{currentfill}{rgb}{0.899534,0.440692,0.344107}%
\pgfsetfillcolor{currentfill}%
\pgfsetlinewidth{0.000000pt}%
\definecolor{currentstroke}{rgb}{0.000000,0.000000,0.000000}%
\pgfsetstrokecolor{currentstroke}%
\pgfsetdash{}{0pt}%
\pgfpathmoveto{\pgfqpoint{2.143341in}{2.291288in}}%
\pgfpathlineto{\pgfqpoint{2.410872in}{2.508083in}}%
\pgfpathlineto{\pgfqpoint{2.143341in}{2.291288in}}%
\pgfpathclose%
\pgfusepath{fill}%
\end{pgfscope}%
\begin{pgfscope}%
\pgfpathrectangle{\pgfqpoint{0.017500in}{0.017500in}}{\pgfqpoint{6.218282in}{3.580372in}} %
\pgfusepath{clip}%
\pgfsetbuttcap%
\pgfsetroundjoin%
\definecolor{currentfill}{rgb}{0.899534,0.440692,0.344107}%
\pgfsetfillcolor{currentfill}%
\pgfsetlinewidth{0.000000pt}%
\definecolor{currentstroke}{rgb}{0.000000,0.000000,0.000000}%
\pgfsetstrokecolor{currentstroke}%
\pgfsetdash{}{0pt}%
\pgfpathmoveto{\pgfqpoint{2.143341in}{2.291288in}}%
\pgfpathlineto{\pgfqpoint{2.410872in}{2.508083in}}%
\pgfpathlineto{\pgfqpoint{2.143341in}{2.291288in}}%
\pgfpathclose%
\pgfusepath{fill}%
\end{pgfscope}%
\begin{pgfscope}%
\pgfpathrectangle{\pgfqpoint{0.017500in}{0.017500in}}{\pgfqpoint{6.218282in}{3.580372in}} %
\pgfusepath{clip}%
\pgfsetbuttcap%
\pgfsetroundjoin%
\definecolor{currentfill}{rgb}{0.899534,0.440692,0.344107}%
\pgfsetfillcolor{currentfill}%
\pgfsetlinewidth{0.000000pt}%
\definecolor{currentstroke}{rgb}{0.000000,0.000000,0.000000}%
\pgfsetstrokecolor{currentstroke}%
\pgfsetdash{}{0pt}%
\pgfpathmoveto{\pgfqpoint{2.143341in}{2.291288in}}%
\pgfpathlineto{\pgfqpoint{2.410872in}{2.508083in}}%
\pgfpathlineto{\pgfqpoint{2.143341in}{2.291288in}}%
\pgfpathclose%
\pgfusepath{fill}%
\end{pgfscope}%
\begin{pgfscope}%
\pgfpathrectangle{\pgfqpoint{0.017500in}{0.017500in}}{\pgfqpoint{6.218282in}{3.580372in}} %
\pgfusepath{clip}%
\pgfsetbuttcap%
\pgfsetroundjoin%
\definecolor{currentfill}{rgb}{0.899534,0.440692,0.344107}%
\pgfsetfillcolor{currentfill}%
\pgfsetlinewidth{0.000000pt}%
\definecolor{currentstroke}{rgb}{0.000000,0.000000,0.000000}%
\pgfsetstrokecolor{currentstroke}%
\pgfsetdash{}{0pt}%
\pgfpathmoveto{\pgfqpoint{2.143341in}{2.291288in}}%
\pgfpathlineto{\pgfqpoint{2.410872in}{2.508083in}}%
\pgfpathlineto{\pgfqpoint{2.143341in}{2.291288in}}%
\pgfpathclose%
\pgfusepath{fill}%
\end{pgfscope}%
\begin{pgfscope}%
\pgfpathrectangle{\pgfqpoint{0.017500in}{0.017500in}}{\pgfqpoint{6.218282in}{3.580372in}} %
\pgfusepath{clip}%
\pgfsetbuttcap%
\pgfsetroundjoin%
\definecolor{currentfill}{rgb}{0.899534,0.440692,0.344107}%
\pgfsetfillcolor{currentfill}%
\pgfsetlinewidth{0.000000pt}%
\definecolor{currentstroke}{rgb}{0.000000,0.000000,0.000000}%
\pgfsetstrokecolor{currentstroke}%
\pgfsetdash{}{0pt}%
\pgfpathmoveto{\pgfqpoint{2.143341in}{2.291288in}}%
\pgfpathlineto{\pgfqpoint{2.410872in}{2.508083in}}%
\pgfpathlineto{\pgfqpoint{2.143341in}{2.291288in}}%
\pgfpathclose%
\pgfusepath{fill}%
\end{pgfscope}%
\begin{pgfscope}%
\pgfpathrectangle{\pgfqpoint{0.017500in}{0.017500in}}{\pgfqpoint{6.218282in}{3.580372in}} %
\pgfusepath{clip}%
\pgfsetbuttcap%
\pgfsetroundjoin%
\definecolor{currentfill}{rgb}{0.899534,0.440692,0.344107}%
\pgfsetfillcolor{currentfill}%
\pgfsetlinewidth{0.000000pt}%
\definecolor{currentstroke}{rgb}{0.000000,0.000000,0.000000}%
\pgfsetstrokecolor{currentstroke}%
\pgfsetdash{}{0pt}%
\pgfpathmoveto{\pgfqpoint{2.143341in}{2.291288in}}%
\pgfpathlineto{\pgfqpoint{2.410872in}{2.508083in}}%
\pgfpathlineto{\pgfqpoint{2.143341in}{2.291288in}}%
\pgfpathclose%
\pgfusepath{fill}%
\end{pgfscope}%
\begin{pgfscope}%
\pgfpathrectangle{\pgfqpoint{0.017500in}{0.017500in}}{\pgfqpoint{6.218282in}{3.580372in}} %
\pgfusepath{clip}%
\pgfsetbuttcap%
\pgfsetroundjoin%
\definecolor{currentfill}{rgb}{0.899534,0.440692,0.344107}%
\pgfsetfillcolor{currentfill}%
\pgfsetlinewidth{0.000000pt}%
\definecolor{currentstroke}{rgb}{0.000000,0.000000,0.000000}%
\pgfsetstrokecolor{currentstroke}%
\pgfsetdash{}{0pt}%
\pgfpathmoveto{\pgfqpoint{2.143341in}{2.291288in}}%
\pgfpathlineto{\pgfqpoint{2.410872in}{2.508083in}}%
\pgfpathlineto{\pgfqpoint{2.143341in}{2.291288in}}%
\pgfpathclose%
\pgfusepath{fill}%
\end{pgfscope}%
\begin{pgfscope}%
\pgfpathrectangle{\pgfqpoint{0.017500in}{0.017500in}}{\pgfqpoint{6.218282in}{3.580372in}} %
\pgfusepath{clip}%
\pgfsetbuttcap%
\pgfsetroundjoin%
\definecolor{currentfill}{rgb}{0.899534,0.440692,0.344107}%
\pgfsetfillcolor{currentfill}%
\pgfsetlinewidth{0.000000pt}%
\definecolor{currentstroke}{rgb}{0.000000,0.000000,0.000000}%
\pgfsetstrokecolor{currentstroke}%
\pgfsetdash{}{0pt}%
\pgfpathmoveto{\pgfqpoint{2.143341in}{2.291288in}}%
\pgfpathlineto{\pgfqpoint{2.410872in}{2.508083in}}%
\pgfpathlineto{\pgfqpoint{2.143341in}{2.291288in}}%
\pgfpathclose%
\pgfusepath{fill}%
\end{pgfscope}%
\begin{pgfscope}%
\pgfpathrectangle{\pgfqpoint{0.017500in}{0.017500in}}{\pgfqpoint{6.218282in}{3.580372in}} %
\pgfusepath{clip}%
\pgfsetbuttcap%
\pgfsetroundjoin%
\definecolor{currentfill}{rgb}{0.899534,0.440692,0.344107}%
\pgfsetfillcolor{currentfill}%
\pgfsetlinewidth{0.000000pt}%
\definecolor{currentstroke}{rgb}{0.000000,0.000000,0.000000}%
\pgfsetstrokecolor{currentstroke}%
\pgfsetdash{}{0pt}%
\pgfpathmoveto{\pgfqpoint{2.143341in}{2.291288in}}%
\pgfpathlineto{\pgfqpoint{2.410872in}{2.508083in}}%
\pgfpathlineto{\pgfqpoint{2.143341in}{2.291288in}}%
\pgfpathclose%
\pgfusepath{fill}%
\end{pgfscope}%
\begin{pgfscope}%
\pgfpathrectangle{\pgfqpoint{0.017500in}{0.017500in}}{\pgfqpoint{6.218282in}{3.580372in}} %
\pgfusepath{clip}%
\pgfsetbuttcap%
\pgfsetroundjoin%
\definecolor{currentfill}{rgb}{0.899534,0.440692,0.344107}%
\pgfsetfillcolor{currentfill}%
\pgfsetlinewidth{0.000000pt}%
\definecolor{currentstroke}{rgb}{0.000000,0.000000,0.000000}%
\pgfsetstrokecolor{currentstroke}%
\pgfsetdash{}{0pt}%
\pgfpathmoveto{\pgfqpoint{2.143341in}{2.291288in}}%
\pgfpathlineto{\pgfqpoint{2.410872in}{2.508083in}}%
\pgfpathlineto{\pgfqpoint{2.143341in}{2.291288in}}%
\pgfpathclose%
\pgfusepath{fill}%
\end{pgfscope}%
\begin{pgfscope}%
\pgfpathrectangle{\pgfqpoint{0.017500in}{0.017500in}}{\pgfqpoint{6.218282in}{3.580372in}} %
\pgfusepath{clip}%
\pgfsetbuttcap%
\pgfsetroundjoin%
\definecolor{currentfill}{rgb}{0.763520,0.178667,0.193396}%
\pgfsetfillcolor{currentfill}%
\pgfsetlinewidth{0.000000pt}%
\definecolor{currentstroke}{rgb}{0.000000,0.000000,0.000000}%
\pgfsetstrokecolor{currentstroke}%
\pgfsetdash{}{0pt}%
\pgfpathmoveto{\pgfqpoint{4.018447in}{2.468167in}}%
\pgfpathlineto{\pgfqpoint{4.267628in}{2.613520in}}%
\pgfpathlineto{\pgfqpoint{4.018447in}{2.468167in}}%
\pgfpathclose%
\pgfusepath{fill}%
\end{pgfscope}%
\begin{pgfscope}%
\pgfpathrectangle{\pgfqpoint{0.017500in}{0.017500in}}{\pgfqpoint{6.218282in}{3.580372in}} %
\pgfusepath{clip}%
\pgfsetbuttcap%
\pgfsetroundjoin%
\definecolor{currentfill}{rgb}{0.763520,0.178667,0.193396}%
\pgfsetfillcolor{currentfill}%
\pgfsetlinewidth{0.000000pt}%
\definecolor{currentstroke}{rgb}{0.000000,0.000000,0.000000}%
\pgfsetstrokecolor{currentstroke}%
\pgfsetdash{}{0pt}%
\pgfpathmoveto{\pgfqpoint{4.018447in}{2.468167in}}%
\pgfpathlineto{\pgfqpoint{4.267628in}{2.613520in}}%
\pgfpathlineto{\pgfqpoint{4.018447in}{2.468167in}}%
\pgfpathclose%
\pgfusepath{fill}%
\end{pgfscope}%
\begin{pgfscope}%
\pgfpathrectangle{\pgfqpoint{0.017500in}{0.017500in}}{\pgfqpoint{6.218282in}{3.580372in}} %
\pgfusepath{clip}%
\pgfsetbuttcap%
\pgfsetroundjoin%
\definecolor{currentfill}{rgb}{0.763520,0.178667,0.193396}%
\pgfsetfillcolor{currentfill}%
\pgfsetlinewidth{0.000000pt}%
\definecolor{currentstroke}{rgb}{0.000000,0.000000,0.000000}%
\pgfsetstrokecolor{currentstroke}%
\pgfsetdash{}{0pt}%
\pgfpathmoveto{\pgfqpoint{4.018447in}{2.468167in}}%
\pgfpathlineto{\pgfqpoint{4.267628in}{2.613520in}}%
\pgfpathlineto{\pgfqpoint{4.018447in}{2.468167in}}%
\pgfpathclose%
\pgfusepath{fill}%
\end{pgfscope}%
\begin{pgfscope}%
\pgfpathrectangle{\pgfqpoint{0.017500in}{0.017500in}}{\pgfqpoint{6.218282in}{3.580372in}} %
\pgfusepath{clip}%
\pgfsetbuttcap%
\pgfsetroundjoin%
\definecolor{currentfill}{rgb}{0.763520,0.178667,0.193396}%
\pgfsetfillcolor{currentfill}%
\pgfsetlinewidth{0.000000pt}%
\definecolor{currentstroke}{rgb}{0.000000,0.000000,0.000000}%
\pgfsetstrokecolor{currentstroke}%
\pgfsetdash{}{0pt}%
\pgfpathmoveto{\pgfqpoint{4.018447in}{2.468167in}}%
\pgfpathlineto{\pgfqpoint{4.267628in}{2.613520in}}%
\pgfpathlineto{\pgfqpoint{4.018447in}{2.468167in}}%
\pgfpathclose%
\pgfusepath{fill}%
\end{pgfscope}%
\begin{pgfscope}%
\pgfpathrectangle{\pgfqpoint{0.017500in}{0.017500in}}{\pgfqpoint{6.218282in}{3.580372in}} %
\pgfusepath{clip}%
\pgfsetbuttcap%
\pgfsetroundjoin%
\definecolor{currentfill}{rgb}{0.763520,0.178667,0.193396}%
\pgfsetfillcolor{currentfill}%
\pgfsetlinewidth{0.000000pt}%
\definecolor{currentstroke}{rgb}{0.000000,0.000000,0.000000}%
\pgfsetstrokecolor{currentstroke}%
\pgfsetdash{}{0pt}%
\pgfpathmoveto{\pgfqpoint{4.018447in}{2.468167in}}%
\pgfpathlineto{\pgfqpoint{4.267628in}{2.613520in}}%
\pgfpathlineto{\pgfqpoint{4.018447in}{2.468167in}}%
\pgfpathclose%
\pgfusepath{fill}%
\end{pgfscope}%
\begin{pgfscope}%
\pgfpathrectangle{\pgfqpoint{0.017500in}{0.017500in}}{\pgfqpoint{6.218282in}{3.580372in}} %
\pgfusepath{clip}%
\pgfsetbuttcap%
\pgfsetroundjoin%
\definecolor{currentfill}{rgb}{0.763520,0.178667,0.193396}%
\pgfsetfillcolor{currentfill}%
\pgfsetlinewidth{0.000000pt}%
\definecolor{currentstroke}{rgb}{0.000000,0.000000,0.000000}%
\pgfsetstrokecolor{currentstroke}%
\pgfsetdash{}{0pt}%
\pgfpathmoveto{\pgfqpoint{4.018447in}{2.468167in}}%
\pgfpathlineto{\pgfqpoint{4.267628in}{2.613520in}}%
\pgfpathlineto{\pgfqpoint{4.018447in}{2.468167in}}%
\pgfpathclose%
\pgfusepath{fill}%
\end{pgfscope}%
\begin{pgfscope}%
\pgfpathrectangle{\pgfqpoint{0.017500in}{0.017500in}}{\pgfqpoint{6.218282in}{3.580372in}} %
\pgfusepath{clip}%
\pgfsetbuttcap%
\pgfsetroundjoin%
\definecolor{currentfill}{rgb}{0.763520,0.178667,0.193396}%
\pgfsetfillcolor{currentfill}%
\pgfsetlinewidth{0.000000pt}%
\definecolor{currentstroke}{rgb}{0.000000,0.000000,0.000000}%
\pgfsetstrokecolor{currentstroke}%
\pgfsetdash{}{0pt}%
\pgfpathmoveto{\pgfqpoint{4.018447in}{2.468167in}}%
\pgfpathlineto{\pgfqpoint{4.267628in}{2.613520in}}%
\pgfpathlineto{\pgfqpoint{4.018447in}{2.468167in}}%
\pgfpathclose%
\pgfusepath{fill}%
\end{pgfscope}%
\begin{pgfscope}%
\pgfpathrectangle{\pgfqpoint{0.017500in}{0.017500in}}{\pgfqpoint{6.218282in}{3.580372in}} %
\pgfusepath{clip}%
\pgfsetbuttcap%
\pgfsetroundjoin%
\definecolor{currentfill}{rgb}{0.763520,0.178667,0.193396}%
\pgfsetfillcolor{currentfill}%
\pgfsetlinewidth{0.000000pt}%
\definecolor{currentstroke}{rgb}{0.000000,0.000000,0.000000}%
\pgfsetstrokecolor{currentstroke}%
\pgfsetdash{}{0pt}%
\pgfpathmoveto{\pgfqpoint{4.018447in}{2.468167in}}%
\pgfpathlineto{\pgfqpoint{4.267628in}{2.613520in}}%
\pgfpathlineto{\pgfqpoint{4.018447in}{2.468167in}}%
\pgfpathclose%
\pgfusepath{fill}%
\end{pgfscope}%
\begin{pgfscope}%
\pgfpathrectangle{\pgfqpoint{0.017500in}{0.017500in}}{\pgfqpoint{6.218282in}{3.580372in}} %
\pgfusepath{clip}%
\pgfsetbuttcap%
\pgfsetroundjoin%
\definecolor{currentfill}{rgb}{0.763520,0.178667,0.193396}%
\pgfsetfillcolor{currentfill}%
\pgfsetlinewidth{0.000000pt}%
\definecolor{currentstroke}{rgb}{0.000000,0.000000,0.000000}%
\pgfsetstrokecolor{currentstroke}%
\pgfsetdash{}{0pt}%
\pgfpathmoveto{\pgfqpoint{4.018447in}{2.468167in}}%
\pgfpathlineto{\pgfqpoint{4.267628in}{2.613520in}}%
\pgfpathlineto{\pgfqpoint{4.018447in}{2.468167in}}%
\pgfpathclose%
\pgfusepath{fill}%
\end{pgfscope}%
\begin{pgfscope}%
\pgfpathrectangle{\pgfqpoint{0.017500in}{0.017500in}}{\pgfqpoint{6.218282in}{3.580372in}} %
\pgfusepath{clip}%
\pgfsetbuttcap%
\pgfsetroundjoin%
\definecolor{currentfill}{rgb}{0.763520,0.178667,0.193396}%
\pgfsetfillcolor{currentfill}%
\pgfsetlinewidth{0.000000pt}%
\definecolor{currentstroke}{rgb}{0.000000,0.000000,0.000000}%
\pgfsetstrokecolor{currentstroke}%
\pgfsetdash{}{0pt}%
\pgfpathmoveto{\pgfqpoint{4.018447in}{2.468167in}}%
\pgfpathlineto{\pgfqpoint{4.267628in}{2.613520in}}%
\pgfpathlineto{\pgfqpoint{4.018447in}{2.468167in}}%
\pgfpathclose%
\pgfusepath{fill}%
\end{pgfscope}%
\begin{pgfscope}%
\pgfpathrectangle{\pgfqpoint{0.017500in}{0.017500in}}{\pgfqpoint{6.218282in}{3.580372in}} %
\pgfusepath{clip}%
\pgfsetbuttcap%
\pgfsetroundjoin%
\definecolor{currentfill}{rgb}{0.763520,0.178667,0.193396}%
\pgfsetfillcolor{currentfill}%
\pgfsetlinewidth{0.000000pt}%
\definecolor{currentstroke}{rgb}{0.000000,0.000000,0.000000}%
\pgfsetstrokecolor{currentstroke}%
\pgfsetdash{}{0pt}%
\pgfpathmoveto{\pgfqpoint{4.018447in}{2.468167in}}%
\pgfpathlineto{\pgfqpoint{4.267628in}{2.613520in}}%
\pgfpathlineto{\pgfqpoint{4.018447in}{2.468167in}}%
\pgfpathclose%
\pgfusepath{fill}%
\end{pgfscope}%
\begin{pgfscope}%
\pgfpathrectangle{\pgfqpoint{0.017500in}{0.017500in}}{\pgfqpoint{6.218282in}{3.580372in}} %
\pgfusepath{clip}%
\pgfsetbuttcap%
\pgfsetroundjoin%
\definecolor{currentfill}{rgb}{0.763520,0.178667,0.193396}%
\pgfsetfillcolor{currentfill}%
\pgfsetlinewidth{0.000000pt}%
\definecolor{currentstroke}{rgb}{0.000000,0.000000,0.000000}%
\pgfsetstrokecolor{currentstroke}%
\pgfsetdash{}{0pt}%
\pgfpathmoveto{\pgfqpoint{4.018447in}{2.468167in}}%
\pgfpathlineto{\pgfqpoint{4.267628in}{2.613520in}}%
\pgfpathlineto{\pgfqpoint{4.018447in}{2.468167in}}%
\pgfpathclose%
\pgfusepath{fill}%
\end{pgfscope}%
\begin{pgfscope}%
\pgfpathrectangle{\pgfqpoint{0.017500in}{0.017500in}}{\pgfqpoint{6.218282in}{3.580372in}} %
\pgfusepath{clip}%
\pgfsetbuttcap%
\pgfsetroundjoin%
\definecolor{currentfill}{rgb}{0.763520,0.178667,0.193396}%
\pgfsetfillcolor{currentfill}%
\pgfsetlinewidth{0.000000pt}%
\definecolor{currentstroke}{rgb}{0.000000,0.000000,0.000000}%
\pgfsetstrokecolor{currentstroke}%
\pgfsetdash{}{0pt}%
\pgfpathmoveto{\pgfqpoint{4.018447in}{2.468167in}}%
\pgfpathlineto{\pgfqpoint{4.267628in}{2.613520in}}%
\pgfpathlineto{\pgfqpoint{4.018447in}{2.468167in}}%
\pgfpathclose%
\pgfusepath{fill}%
\end{pgfscope}%
\begin{pgfscope}%
\pgfpathrectangle{\pgfqpoint{0.017500in}{0.017500in}}{\pgfqpoint{6.218282in}{3.580372in}} %
\pgfusepath{clip}%
\pgfsetbuttcap%
\pgfsetroundjoin%
\definecolor{currentfill}{rgb}{0.763520,0.178667,0.193396}%
\pgfsetfillcolor{currentfill}%
\pgfsetlinewidth{0.000000pt}%
\definecolor{currentstroke}{rgb}{0.000000,0.000000,0.000000}%
\pgfsetstrokecolor{currentstroke}%
\pgfsetdash{}{0pt}%
\pgfpathmoveto{\pgfqpoint{4.018447in}{2.468167in}}%
\pgfpathlineto{\pgfqpoint{4.267628in}{2.613520in}}%
\pgfpathlineto{\pgfqpoint{4.018447in}{2.468167in}}%
\pgfpathclose%
\pgfusepath{fill}%
\end{pgfscope}%
\begin{pgfscope}%
\pgfpathrectangle{\pgfqpoint{0.017500in}{0.017500in}}{\pgfqpoint{6.218282in}{3.580372in}} %
\pgfusepath{clip}%
\pgfsetbuttcap%
\pgfsetroundjoin%
\definecolor{currentfill}{rgb}{0.763520,0.178667,0.193396}%
\pgfsetfillcolor{currentfill}%
\pgfsetlinewidth{0.000000pt}%
\definecolor{currentstroke}{rgb}{0.000000,0.000000,0.000000}%
\pgfsetstrokecolor{currentstroke}%
\pgfsetdash{}{0pt}%
\pgfpathmoveto{\pgfqpoint{4.018447in}{2.468167in}}%
\pgfpathlineto{\pgfqpoint{4.267628in}{2.613520in}}%
\pgfpathlineto{\pgfqpoint{4.018447in}{2.468167in}}%
\pgfpathclose%
\pgfusepath{fill}%
\end{pgfscope}%
\begin{pgfscope}%
\pgfpathrectangle{\pgfqpoint{0.017500in}{0.017500in}}{\pgfqpoint{6.218282in}{3.580372in}} %
\pgfusepath{clip}%
\pgfsetbuttcap%
\pgfsetroundjoin%
\definecolor{currentfill}{rgb}{0.763520,0.178667,0.193396}%
\pgfsetfillcolor{currentfill}%
\pgfsetlinewidth{0.000000pt}%
\definecolor{currentstroke}{rgb}{0.000000,0.000000,0.000000}%
\pgfsetstrokecolor{currentstroke}%
\pgfsetdash{}{0pt}%
\pgfpathmoveto{\pgfqpoint{4.018447in}{2.468167in}}%
\pgfpathlineto{\pgfqpoint{4.267628in}{2.613520in}}%
\pgfpathlineto{\pgfqpoint{4.018447in}{2.468167in}}%
\pgfpathclose%
\pgfusepath{fill}%
\end{pgfscope}%
\begin{pgfscope}%
\pgfpathrectangle{\pgfqpoint{0.017500in}{0.017500in}}{\pgfqpoint{6.218282in}{3.580372in}} %
\pgfusepath{clip}%
\pgfsetbuttcap%
\pgfsetroundjoin%
\definecolor{currentfill}{rgb}{0.763520,0.178667,0.193396}%
\pgfsetfillcolor{currentfill}%
\pgfsetlinewidth{0.000000pt}%
\definecolor{currentstroke}{rgb}{0.000000,0.000000,0.000000}%
\pgfsetstrokecolor{currentstroke}%
\pgfsetdash{}{0pt}%
\pgfpathmoveto{\pgfqpoint{4.018447in}{2.468167in}}%
\pgfpathlineto{\pgfqpoint{4.267628in}{2.613520in}}%
\pgfpathlineto{\pgfqpoint{4.018447in}{2.468167in}}%
\pgfpathclose%
\pgfusepath{fill}%
\end{pgfscope}%
\begin{pgfscope}%
\pgfpathrectangle{\pgfqpoint{0.017500in}{0.017500in}}{\pgfqpoint{6.218282in}{3.580372in}} %
\pgfusepath{clip}%
\pgfsetbuttcap%
\pgfsetroundjoin%
\definecolor{currentfill}{rgb}{0.763520,0.178667,0.193396}%
\pgfsetfillcolor{currentfill}%
\pgfsetlinewidth{0.000000pt}%
\definecolor{currentstroke}{rgb}{0.000000,0.000000,0.000000}%
\pgfsetstrokecolor{currentstroke}%
\pgfsetdash{}{0pt}%
\pgfpathmoveto{\pgfqpoint{4.018447in}{2.468167in}}%
\pgfpathlineto{\pgfqpoint{4.267628in}{2.613520in}}%
\pgfpathlineto{\pgfqpoint{4.018447in}{2.468167in}}%
\pgfpathclose%
\pgfusepath{fill}%
\end{pgfscope}%
\begin{pgfscope}%
\pgfpathrectangle{\pgfqpoint{0.017500in}{0.017500in}}{\pgfqpoint{6.218282in}{3.580372in}} %
\pgfusepath{clip}%
\pgfsetbuttcap%
\pgfsetroundjoin%
\definecolor{currentfill}{rgb}{0.763520,0.178667,0.193396}%
\pgfsetfillcolor{currentfill}%
\pgfsetlinewidth{0.000000pt}%
\definecolor{currentstroke}{rgb}{0.000000,0.000000,0.000000}%
\pgfsetstrokecolor{currentstroke}%
\pgfsetdash{}{0pt}%
\pgfpathmoveto{\pgfqpoint{4.018447in}{2.468167in}}%
\pgfpathlineto{\pgfqpoint{4.267628in}{2.613520in}}%
\pgfpathlineto{\pgfqpoint{4.018447in}{2.468167in}}%
\pgfpathclose%
\pgfusepath{fill}%
\end{pgfscope}%
\begin{pgfscope}%
\pgfpathrectangle{\pgfqpoint{0.017500in}{0.017500in}}{\pgfqpoint{6.218282in}{3.580372in}} %
\pgfusepath{clip}%
\pgfsetbuttcap%
\pgfsetroundjoin%
\definecolor{currentfill}{rgb}{0.763520,0.178667,0.193396}%
\pgfsetfillcolor{currentfill}%
\pgfsetlinewidth{0.000000pt}%
\definecolor{currentstroke}{rgb}{0.000000,0.000000,0.000000}%
\pgfsetstrokecolor{currentstroke}%
\pgfsetdash{}{0pt}%
\pgfpathmoveto{\pgfqpoint{4.018447in}{2.468167in}}%
\pgfpathlineto{\pgfqpoint{4.267628in}{2.613520in}}%
\pgfpathlineto{\pgfqpoint{4.018447in}{2.468167in}}%
\pgfpathclose%
\pgfusepath{fill}%
\end{pgfscope}%
\begin{pgfscope}%
\pgfpathrectangle{\pgfqpoint{0.017500in}{0.017500in}}{\pgfqpoint{6.218282in}{3.580372in}} %
\pgfusepath{clip}%
\pgfsetbuttcap%
\pgfsetroundjoin%
\definecolor{currentfill}{rgb}{0.763520,0.178667,0.193396}%
\pgfsetfillcolor{currentfill}%
\pgfsetlinewidth{0.000000pt}%
\definecolor{currentstroke}{rgb}{0.000000,0.000000,0.000000}%
\pgfsetstrokecolor{currentstroke}%
\pgfsetdash{}{0pt}%
\pgfpathmoveto{\pgfqpoint{4.018447in}{2.468167in}}%
\pgfpathlineto{\pgfqpoint{4.267628in}{2.613520in}}%
\pgfpathlineto{\pgfqpoint{4.018447in}{2.468167in}}%
\pgfpathclose%
\pgfusepath{fill}%
\end{pgfscope}%
\begin{pgfscope}%
\pgfpathrectangle{\pgfqpoint{0.017500in}{0.017500in}}{\pgfqpoint{6.218282in}{3.580372in}} %
\pgfusepath{clip}%
\pgfsetbuttcap%
\pgfsetroundjoin%
\definecolor{currentfill}{rgb}{0.763520,0.178667,0.193396}%
\pgfsetfillcolor{currentfill}%
\pgfsetlinewidth{0.000000pt}%
\definecolor{currentstroke}{rgb}{0.000000,0.000000,0.000000}%
\pgfsetstrokecolor{currentstroke}%
\pgfsetdash{}{0pt}%
\pgfpathmoveto{\pgfqpoint{4.018447in}{2.468167in}}%
\pgfpathlineto{\pgfqpoint{4.267628in}{2.613520in}}%
\pgfpathlineto{\pgfqpoint{4.018447in}{2.468167in}}%
\pgfpathclose%
\pgfusepath{fill}%
\end{pgfscope}%
\begin{pgfscope}%
\pgfpathrectangle{\pgfqpoint{0.017500in}{0.017500in}}{\pgfqpoint{6.218282in}{3.580372in}} %
\pgfusepath{clip}%
\pgfsetbuttcap%
\pgfsetroundjoin%
\definecolor{currentfill}{rgb}{0.763520,0.178667,0.193396}%
\pgfsetfillcolor{currentfill}%
\pgfsetlinewidth{0.000000pt}%
\definecolor{currentstroke}{rgb}{0.000000,0.000000,0.000000}%
\pgfsetstrokecolor{currentstroke}%
\pgfsetdash{}{0pt}%
\pgfpathmoveto{\pgfqpoint{4.018447in}{2.468167in}}%
\pgfpathlineto{\pgfqpoint{4.267628in}{2.613520in}}%
\pgfpathlineto{\pgfqpoint{4.018447in}{2.468167in}}%
\pgfpathclose%
\pgfusepath{fill}%
\end{pgfscope}%
\begin{pgfscope}%
\pgfpathrectangle{\pgfqpoint{0.017500in}{0.017500in}}{\pgfqpoint{6.218282in}{3.580372in}} %
\pgfusepath{clip}%
\pgfsetbuttcap%
\pgfsetroundjoin%
\definecolor{currentfill}{rgb}{0.763520,0.178667,0.193396}%
\pgfsetfillcolor{currentfill}%
\pgfsetlinewidth{0.000000pt}%
\definecolor{currentstroke}{rgb}{0.000000,0.000000,0.000000}%
\pgfsetstrokecolor{currentstroke}%
\pgfsetdash{}{0pt}%
\pgfpathmoveto{\pgfqpoint{4.018447in}{2.468167in}}%
\pgfpathlineto{\pgfqpoint{4.267628in}{2.613520in}}%
\pgfpathlineto{\pgfqpoint{4.018447in}{2.468167in}}%
\pgfpathclose%
\pgfusepath{fill}%
\end{pgfscope}%
\begin{pgfscope}%
\pgfpathrectangle{\pgfqpoint{0.017500in}{0.017500in}}{\pgfqpoint{6.218282in}{3.580372in}} %
\pgfusepath{clip}%
\pgfsetbuttcap%
\pgfsetroundjoin%
\definecolor{currentfill}{rgb}{0.763520,0.178667,0.193396}%
\pgfsetfillcolor{currentfill}%
\pgfsetlinewidth{0.000000pt}%
\definecolor{currentstroke}{rgb}{0.000000,0.000000,0.000000}%
\pgfsetstrokecolor{currentstroke}%
\pgfsetdash{}{0pt}%
\pgfpathmoveto{\pgfqpoint{4.018447in}{2.468167in}}%
\pgfpathlineto{\pgfqpoint{4.267628in}{2.613520in}}%
\pgfpathlineto{\pgfqpoint{4.018447in}{2.468167in}}%
\pgfpathclose%
\pgfusepath{fill}%
\end{pgfscope}%
\begin{pgfscope}%
\pgfpathrectangle{\pgfqpoint{0.017500in}{0.017500in}}{\pgfqpoint{6.218282in}{3.580372in}} %
\pgfusepath{clip}%
\pgfsetbuttcap%
\pgfsetroundjoin%
\definecolor{currentfill}{rgb}{0.763520,0.178667,0.193396}%
\pgfsetfillcolor{currentfill}%
\pgfsetlinewidth{0.000000pt}%
\definecolor{currentstroke}{rgb}{0.000000,0.000000,0.000000}%
\pgfsetstrokecolor{currentstroke}%
\pgfsetdash{}{0pt}%
\pgfpathmoveto{\pgfqpoint{4.018447in}{2.468167in}}%
\pgfpathlineto{\pgfqpoint{4.267628in}{2.613520in}}%
\pgfpathlineto{\pgfqpoint{4.018447in}{2.468167in}}%
\pgfpathclose%
\pgfusepath{fill}%
\end{pgfscope}%
\begin{pgfscope}%
\pgfpathrectangle{\pgfqpoint{0.017500in}{0.017500in}}{\pgfqpoint{6.218282in}{3.580372in}} %
\pgfusepath{clip}%
\pgfsetbuttcap%
\pgfsetroundjoin%
\definecolor{currentfill}{rgb}{0.763520,0.178667,0.193396}%
\pgfsetfillcolor{currentfill}%
\pgfsetlinewidth{0.000000pt}%
\definecolor{currentstroke}{rgb}{0.000000,0.000000,0.000000}%
\pgfsetstrokecolor{currentstroke}%
\pgfsetdash{}{0pt}%
\pgfpathmoveto{\pgfqpoint{4.018447in}{2.468167in}}%
\pgfpathlineto{\pgfqpoint{4.267628in}{2.613520in}}%
\pgfpathlineto{\pgfqpoint{4.018447in}{2.468167in}}%
\pgfpathclose%
\pgfusepath{fill}%
\end{pgfscope}%
\begin{pgfscope}%
\pgfpathrectangle{\pgfqpoint{0.017500in}{0.017500in}}{\pgfqpoint{6.218282in}{3.580372in}} %
\pgfusepath{clip}%
\pgfsetbuttcap%
\pgfsetroundjoin%
\definecolor{currentfill}{rgb}{0.763520,0.178667,0.193396}%
\pgfsetfillcolor{currentfill}%
\pgfsetlinewidth{0.000000pt}%
\definecolor{currentstroke}{rgb}{0.000000,0.000000,0.000000}%
\pgfsetstrokecolor{currentstroke}%
\pgfsetdash{}{0pt}%
\pgfpathmoveto{\pgfqpoint{4.018447in}{2.468167in}}%
\pgfpathlineto{\pgfqpoint{4.267628in}{2.613520in}}%
\pgfpathlineto{\pgfqpoint{4.018447in}{2.468167in}}%
\pgfpathclose%
\pgfusepath{fill}%
\end{pgfscope}%
\begin{pgfscope}%
\pgfpathrectangle{\pgfqpoint{0.017500in}{0.017500in}}{\pgfqpoint{6.218282in}{3.580372in}} %
\pgfusepath{clip}%
\pgfsetbuttcap%
\pgfsetroundjoin%
\definecolor{currentfill}{rgb}{0.763520,0.178667,0.193396}%
\pgfsetfillcolor{currentfill}%
\pgfsetlinewidth{0.000000pt}%
\definecolor{currentstroke}{rgb}{0.000000,0.000000,0.000000}%
\pgfsetstrokecolor{currentstroke}%
\pgfsetdash{}{0pt}%
\pgfpathmoveto{\pgfqpoint{4.018447in}{2.468167in}}%
\pgfpathlineto{\pgfqpoint{4.267628in}{2.613520in}}%
\pgfpathlineto{\pgfqpoint{4.018447in}{2.468167in}}%
\pgfpathclose%
\pgfusepath{fill}%
\end{pgfscope}%
\begin{pgfscope}%
\pgfpathrectangle{\pgfqpoint{0.017500in}{0.017500in}}{\pgfqpoint{6.218282in}{3.580372in}} %
\pgfusepath{clip}%
\pgfsetbuttcap%
\pgfsetroundjoin%
\definecolor{currentfill}{rgb}{0.763520,0.178667,0.193396}%
\pgfsetfillcolor{currentfill}%
\pgfsetlinewidth{0.000000pt}%
\definecolor{currentstroke}{rgb}{0.000000,0.000000,0.000000}%
\pgfsetstrokecolor{currentstroke}%
\pgfsetdash{}{0pt}%
\pgfpathmoveto{\pgfqpoint{4.018447in}{2.468167in}}%
\pgfpathlineto{\pgfqpoint{4.267628in}{2.613520in}}%
\pgfpathlineto{\pgfqpoint{4.018447in}{2.468167in}}%
\pgfpathclose%
\pgfusepath{fill}%
\end{pgfscope}%
\begin{pgfscope}%
\pgfpathrectangle{\pgfqpoint{0.017500in}{0.017500in}}{\pgfqpoint{6.218282in}{3.580372in}} %
\pgfusepath{clip}%
\pgfsetbuttcap%
\pgfsetroundjoin%
\definecolor{currentfill}{rgb}{0.892138,0.425389,0.333289}%
\pgfsetfillcolor{currentfill}%
\pgfsetlinewidth{0.000000pt}%
\definecolor{currentstroke}{rgb}{0.000000,0.000000,0.000000}%
\pgfsetstrokecolor{currentstroke}%
\pgfsetdash{}{0pt}%
\pgfpathmoveto{\pgfqpoint{2.143341in}{2.291288in}}%
\pgfpathlineto{\pgfqpoint{2.590432in}{2.210709in}}%
\pgfpathlineto{\pgfqpoint{2.855513in}{2.451418in}}%
\pgfpathlineto{\pgfqpoint{2.410872in}{2.508083in}}%
\pgfpathlineto{\pgfqpoint{2.143341in}{2.291288in}}%
\pgfpathclose%
\pgfusepath{fill}%
\end{pgfscope}%
\begin{pgfscope}%
\pgfpathrectangle{\pgfqpoint{0.017500in}{0.017500in}}{\pgfqpoint{6.218282in}{3.580372in}} %
\pgfusepath{clip}%
\pgfsetbuttcap%
\pgfsetroundjoin%
\definecolor{currentfill}{rgb}{0.892138,0.425389,0.333289}%
\pgfsetfillcolor{currentfill}%
\pgfsetlinewidth{0.000000pt}%
\definecolor{currentstroke}{rgb}{0.000000,0.000000,0.000000}%
\pgfsetstrokecolor{currentstroke}%
\pgfsetdash{}{0pt}%
\pgfpathmoveto{\pgfqpoint{2.143341in}{2.291288in}}%
\pgfpathlineto{\pgfqpoint{2.590432in}{2.210709in}}%
\pgfpathlineto{\pgfqpoint{2.855513in}{2.451418in}}%
\pgfpathlineto{\pgfqpoint{2.410872in}{2.508083in}}%
\pgfpathlineto{\pgfqpoint{2.143341in}{2.291288in}}%
\pgfpathclose%
\pgfusepath{fill}%
\end{pgfscope}%
\begin{pgfscope}%
\pgfpathrectangle{\pgfqpoint{0.017500in}{0.017500in}}{\pgfqpoint{6.218282in}{3.580372in}} %
\pgfusepath{clip}%
\pgfsetbuttcap%
\pgfsetroundjoin%
\definecolor{currentfill}{rgb}{0.892138,0.425389,0.333289}%
\pgfsetfillcolor{currentfill}%
\pgfsetlinewidth{0.000000pt}%
\definecolor{currentstroke}{rgb}{0.000000,0.000000,0.000000}%
\pgfsetstrokecolor{currentstroke}%
\pgfsetdash{}{0pt}%
\pgfpathmoveto{\pgfqpoint{2.143341in}{2.291288in}}%
\pgfpathlineto{\pgfqpoint{2.590432in}{2.210709in}}%
\pgfpathlineto{\pgfqpoint{2.855513in}{2.451418in}}%
\pgfpathlineto{\pgfqpoint{2.410872in}{2.508083in}}%
\pgfpathlineto{\pgfqpoint{2.143341in}{2.291288in}}%
\pgfpathclose%
\pgfusepath{fill}%
\end{pgfscope}%
\begin{pgfscope}%
\pgfpathrectangle{\pgfqpoint{0.017500in}{0.017500in}}{\pgfqpoint{6.218282in}{3.580372in}} %
\pgfusepath{clip}%
\pgfsetbuttcap%
\pgfsetroundjoin%
\definecolor{currentfill}{rgb}{0.892138,0.425389,0.333289}%
\pgfsetfillcolor{currentfill}%
\pgfsetlinewidth{0.000000pt}%
\definecolor{currentstroke}{rgb}{0.000000,0.000000,0.000000}%
\pgfsetstrokecolor{currentstroke}%
\pgfsetdash{}{0pt}%
\pgfpathmoveto{\pgfqpoint{2.143341in}{2.291288in}}%
\pgfpathlineto{\pgfqpoint{2.590432in}{2.210709in}}%
\pgfpathlineto{\pgfqpoint{2.855513in}{2.451418in}}%
\pgfpathlineto{\pgfqpoint{2.410872in}{2.508083in}}%
\pgfpathlineto{\pgfqpoint{2.143341in}{2.291288in}}%
\pgfpathclose%
\pgfusepath{fill}%
\end{pgfscope}%
\begin{pgfscope}%
\pgfpathrectangle{\pgfqpoint{0.017500in}{0.017500in}}{\pgfqpoint{6.218282in}{3.580372in}} %
\pgfusepath{clip}%
\pgfsetbuttcap%
\pgfsetroundjoin%
\definecolor{currentfill}{rgb}{0.892138,0.425389,0.333289}%
\pgfsetfillcolor{currentfill}%
\pgfsetlinewidth{0.000000pt}%
\definecolor{currentstroke}{rgb}{0.000000,0.000000,0.000000}%
\pgfsetstrokecolor{currentstroke}%
\pgfsetdash{}{0pt}%
\pgfpathmoveto{\pgfqpoint{2.143341in}{2.291288in}}%
\pgfpathlineto{\pgfqpoint{2.590432in}{2.210709in}}%
\pgfpathlineto{\pgfqpoint{2.855513in}{2.451418in}}%
\pgfpathlineto{\pgfqpoint{2.410872in}{2.508083in}}%
\pgfpathlineto{\pgfqpoint{2.143341in}{2.291288in}}%
\pgfpathclose%
\pgfusepath{fill}%
\end{pgfscope}%
\begin{pgfscope}%
\pgfpathrectangle{\pgfqpoint{0.017500in}{0.017500in}}{\pgfqpoint{6.218282in}{3.580372in}} %
\pgfusepath{clip}%
\pgfsetbuttcap%
\pgfsetroundjoin%
\definecolor{currentfill}{rgb}{0.892138,0.425389,0.333289}%
\pgfsetfillcolor{currentfill}%
\pgfsetlinewidth{0.000000pt}%
\definecolor{currentstroke}{rgb}{0.000000,0.000000,0.000000}%
\pgfsetstrokecolor{currentstroke}%
\pgfsetdash{}{0pt}%
\pgfpathmoveto{\pgfqpoint{2.143341in}{2.291288in}}%
\pgfpathlineto{\pgfqpoint{2.590432in}{2.210709in}}%
\pgfpathlineto{\pgfqpoint{2.855513in}{2.451418in}}%
\pgfpathlineto{\pgfqpoint{2.410872in}{2.508083in}}%
\pgfpathlineto{\pgfqpoint{2.143341in}{2.291288in}}%
\pgfpathclose%
\pgfusepath{fill}%
\end{pgfscope}%
\begin{pgfscope}%
\pgfpathrectangle{\pgfqpoint{0.017500in}{0.017500in}}{\pgfqpoint{6.218282in}{3.580372in}} %
\pgfusepath{clip}%
\pgfsetbuttcap%
\pgfsetroundjoin%
\definecolor{currentfill}{rgb}{0.758112,0.168122,0.188827}%
\pgfsetfillcolor{currentfill}%
\pgfsetlinewidth{0.000000pt}%
\definecolor{currentstroke}{rgb}{0.000000,0.000000,0.000000}%
\pgfsetstrokecolor{currentstroke}%
\pgfsetdash{}{0pt}%
\pgfpathmoveto{\pgfqpoint{4.018447in}{2.468167in}}%
\pgfpathlineto{\pgfqpoint{4.478468in}{2.393972in}}%
\pgfpathlineto{\pgfqpoint{4.724651in}{2.549289in}}%
\pgfpathlineto{\pgfqpoint{4.267628in}{2.613520in}}%
\pgfpathlineto{\pgfqpoint{4.018447in}{2.468167in}}%
\pgfpathclose%
\pgfusepath{fill}%
\end{pgfscope}%
\begin{pgfscope}%
\pgfpathrectangle{\pgfqpoint{0.017500in}{0.017500in}}{\pgfqpoint{6.218282in}{3.580372in}} %
\pgfusepath{clip}%
\pgfsetbuttcap%
\pgfsetroundjoin%
\definecolor{currentfill}{rgb}{0.758112,0.168122,0.188827}%
\pgfsetfillcolor{currentfill}%
\pgfsetlinewidth{0.000000pt}%
\definecolor{currentstroke}{rgb}{0.000000,0.000000,0.000000}%
\pgfsetstrokecolor{currentstroke}%
\pgfsetdash{}{0pt}%
\pgfpathmoveto{\pgfqpoint{4.018447in}{2.468167in}}%
\pgfpathlineto{\pgfqpoint{4.478468in}{2.393972in}}%
\pgfpathlineto{\pgfqpoint{4.724651in}{2.549289in}}%
\pgfpathlineto{\pgfqpoint{4.267628in}{2.613520in}}%
\pgfpathlineto{\pgfqpoint{4.018447in}{2.468167in}}%
\pgfpathclose%
\pgfusepath{fill}%
\end{pgfscope}%
\begin{pgfscope}%
\pgfpathrectangle{\pgfqpoint{0.017500in}{0.017500in}}{\pgfqpoint{6.218282in}{3.580372in}} %
\pgfusepath{clip}%
\pgfsetbuttcap%
\pgfsetroundjoin%
\definecolor{currentfill}{rgb}{0.758112,0.168122,0.188827}%
\pgfsetfillcolor{currentfill}%
\pgfsetlinewidth{0.000000pt}%
\definecolor{currentstroke}{rgb}{0.000000,0.000000,0.000000}%
\pgfsetstrokecolor{currentstroke}%
\pgfsetdash{}{0pt}%
\pgfpathmoveto{\pgfqpoint{4.018447in}{2.468167in}}%
\pgfpathlineto{\pgfqpoint{4.478468in}{2.393972in}}%
\pgfpathlineto{\pgfqpoint{4.724651in}{2.549289in}}%
\pgfpathlineto{\pgfqpoint{4.267628in}{2.613520in}}%
\pgfpathlineto{\pgfqpoint{4.018447in}{2.468167in}}%
\pgfpathclose%
\pgfusepath{fill}%
\end{pgfscope}%
\begin{pgfscope}%
\pgfpathrectangle{\pgfqpoint{0.017500in}{0.017500in}}{\pgfqpoint{6.218282in}{3.580372in}} %
\pgfusepath{clip}%
\pgfsetbuttcap%
\pgfsetroundjoin%
\definecolor{currentfill}{rgb}{0.758112,0.168122,0.188827}%
\pgfsetfillcolor{currentfill}%
\pgfsetlinewidth{0.000000pt}%
\definecolor{currentstroke}{rgb}{0.000000,0.000000,0.000000}%
\pgfsetstrokecolor{currentstroke}%
\pgfsetdash{}{0pt}%
\pgfpathmoveto{\pgfqpoint{4.018447in}{2.468167in}}%
\pgfpathlineto{\pgfqpoint{4.478468in}{2.393972in}}%
\pgfpathlineto{\pgfqpoint{4.724651in}{2.549289in}}%
\pgfpathlineto{\pgfqpoint{4.267628in}{2.613520in}}%
\pgfpathlineto{\pgfqpoint{4.018447in}{2.468167in}}%
\pgfpathclose%
\pgfusepath{fill}%
\end{pgfscope}%
\begin{pgfscope}%
\pgfpathrectangle{\pgfqpoint{0.017500in}{0.017500in}}{\pgfqpoint{6.218282in}{3.580372in}} %
\pgfusepath{clip}%
\pgfsetbuttcap%
\pgfsetroundjoin%
\definecolor{currentfill}{rgb}{0.758112,0.168122,0.188827}%
\pgfsetfillcolor{currentfill}%
\pgfsetlinewidth{0.000000pt}%
\definecolor{currentstroke}{rgb}{0.000000,0.000000,0.000000}%
\pgfsetstrokecolor{currentstroke}%
\pgfsetdash{}{0pt}%
\pgfpathmoveto{\pgfqpoint{4.018447in}{2.468167in}}%
\pgfpathlineto{\pgfqpoint{4.478468in}{2.393972in}}%
\pgfpathlineto{\pgfqpoint{4.724651in}{2.549289in}}%
\pgfpathlineto{\pgfqpoint{4.267628in}{2.613520in}}%
\pgfpathlineto{\pgfqpoint{4.018447in}{2.468167in}}%
\pgfpathclose%
\pgfusepath{fill}%
\end{pgfscope}%
\begin{pgfscope}%
\pgfpathrectangle{\pgfqpoint{0.017500in}{0.017500in}}{\pgfqpoint{6.218282in}{3.580372in}} %
\pgfusepath{clip}%
\pgfsetbuttcap%
\pgfsetroundjoin%
\definecolor{currentfill}{rgb}{0.758112,0.168122,0.188827}%
\pgfsetfillcolor{currentfill}%
\pgfsetlinewidth{0.000000pt}%
\definecolor{currentstroke}{rgb}{0.000000,0.000000,0.000000}%
\pgfsetstrokecolor{currentstroke}%
\pgfsetdash{}{0pt}%
\pgfpathmoveto{\pgfqpoint{4.018447in}{2.468167in}}%
\pgfpathlineto{\pgfqpoint{4.478468in}{2.393972in}}%
\pgfpathlineto{\pgfqpoint{4.724651in}{2.549289in}}%
\pgfpathlineto{\pgfqpoint{4.267628in}{2.613520in}}%
\pgfpathlineto{\pgfqpoint{4.018447in}{2.468167in}}%
\pgfpathclose%
\pgfusepath{fill}%
\end{pgfscope}%
\begin{pgfscope}%
\pgfpathrectangle{\pgfqpoint{0.017500in}{0.017500in}}{\pgfqpoint{6.218282in}{3.580372in}} %
\pgfusepath{clip}%
\pgfsetbuttcap%
\pgfsetroundjoin%
\definecolor{currentfill}{rgb}{0.333490,0.446265,0.874452}%
\pgfsetfillcolor{currentfill}%
\pgfsetlinewidth{0.000000pt}%
\definecolor{currentstroke}{rgb}{0.000000,0.000000,0.000000}%
\pgfsetstrokecolor{currentstroke}%
\pgfsetdash{}{0pt}%
\pgfpathmoveto{\pgfqpoint{2.789950in}{0.798346in}}%
\pgfpathlineto{\pgfqpoint{3.236099in}{0.723497in}}%
\pgfpathlineto{\pgfqpoint{3.502561in}{2.073396in}}%
\pgfpathlineto{\pgfqpoint{3.043326in}{2.141782in}}%
\pgfpathlineto{\pgfqpoint{2.789950in}{0.798346in}}%
\pgfpathclose%
\pgfusepath{fill}%
\end{pgfscope}%
\begin{pgfscope}%
\pgfpathrectangle{\pgfqpoint{0.017500in}{0.017500in}}{\pgfqpoint{6.218282in}{3.580372in}} %
\pgfusepath{clip}%
\pgfsetbuttcap%
\pgfsetroundjoin%
\definecolor{currentfill}{rgb}{0.333490,0.446265,0.874452}%
\pgfsetfillcolor{currentfill}%
\pgfsetlinewidth{0.000000pt}%
\definecolor{currentstroke}{rgb}{0.000000,0.000000,0.000000}%
\pgfsetstrokecolor{currentstroke}%
\pgfsetdash{}{0pt}%
\pgfpathmoveto{\pgfqpoint{2.789950in}{0.798346in}}%
\pgfpathlineto{\pgfqpoint{3.236099in}{0.723497in}}%
\pgfpathlineto{\pgfqpoint{3.502561in}{2.073396in}}%
\pgfpathlineto{\pgfqpoint{3.043326in}{2.141782in}}%
\pgfpathlineto{\pgfqpoint{2.789950in}{0.798346in}}%
\pgfpathclose%
\pgfusepath{fill}%
\end{pgfscope}%
\begin{pgfscope}%
\pgfpathrectangle{\pgfqpoint{0.017500in}{0.017500in}}{\pgfqpoint{6.218282in}{3.580372in}} %
\pgfusepath{clip}%
\pgfsetbuttcap%
\pgfsetroundjoin%
\definecolor{currentfill}{rgb}{0.333490,0.446265,0.874452}%
\pgfsetfillcolor{currentfill}%
\pgfsetlinewidth{0.000000pt}%
\definecolor{currentstroke}{rgb}{0.000000,0.000000,0.000000}%
\pgfsetstrokecolor{currentstroke}%
\pgfsetdash{}{0pt}%
\pgfpathmoveto{\pgfqpoint{2.789950in}{0.798346in}}%
\pgfpathlineto{\pgfqpoint{3.236099in}{0.723497in}}%
\pgfpathlineto{\pgfqpoint{3.502561in}{2.073396in}}%
\pgfpathlineto{\pgfqpoint{3.043326in}{2.141782in}}%
\pgfpathlineto{\pgfqpoint{2.789950in}{0.798346in}}%
\pgfpathclose%
\pgfusepath{fill}%
\end{pgfscope}%
\begin{pgfscope}%
\pgfpathrectangle{\pgfqpoint{0.017500in}{0.017500in}}{\pgfqpoint{6.218282in}{3.580372in}} %
\pgfusepath{clip}%
\pgfsetbuttcap%
\pgfsetroundjoin%
\definecolor{currentfill}{rgb}{0.333490,0.446265,0.874452}%
\pgfsetfillcolor{currentfill}%
\pgfsetlinewidth{0.000000pt}%
\definecolor{currentstroke}{rgb}{0.000000,0.000000,0.000000}%
\pgfsetstrokecolor{currentstroke}%
\pgfsetdash{}{0pt}%
\pgfpathmoveto{\pgfqpoint{2.789950in}{0.798346in}}%
\pgfpathlineto{\pgfqpoint{3.236099in}{0.723497in}}%
\pgfpathlineto{\pgfqpoint{3.502561in}{2.073396in}}%
\pgfpathlineto{\pgfqpoint{3.043326in}{2.141782in}}%
\pgfpathlineto{\pgfqpoint{2.789950in}{0.798346in}}%
\pgfpathclose%
\pgfusepath{fill}%
\end{pgfscope}%
\begin{pgfscope}%
\pgfpathrectangle{\pgfqpoint{0.017500in}{0.017500in}}{\pgfqpoint{6.218282in}{3.580372in}} %
\pgfusepath{clip}%
\pgfsetbuttcap%
\pgfsetroundjoin%
\definecolor{currentfill}{rgb}{0.333490,0.446265,0.874452}%
\pgfsetfillcolor{currentfill}%
\pgfsetlinewidth{0.000000pt}%
\definecolor{currentstroke}{rgb}{0.000000,0.000000,0.000000}%
\pgfsetstrokecolor{currentstroke}%
\pgfsetdash{}{0pt}%
\pgfpathmoveto{\pgfqpoint{2.789950in}{0.798346in}}%
\pgfpathlineto{\pgfqpoint{3.236099in}{0.723497in}}%
\pgfpathlineto{\pgfqpoint{3.502561in}{2.073396in}}%
\pgfpathlineto{\pgfqpoint{3.043326in}{2.141782in}}%
\pgfpathlineto{\pgfqpoint{2.789950in}{0.798346in}}%
\pgfpathclose%
\pgfusepath{fill}%
\end{pgfscope}%
\begin{pgfscope}%
\pgfpathrectangle{\pgfqpoint{0.017500in}{0.017500in}}{\pgfqpoint{6.218282in}{3.580372in}} %
\pgfusepath{clip}%
\pgfsetbuttcap%
\pgfsetroundjoin%
\definecolor{currentfill}{rgb}{0.333490,0.446265,0.874452}%
\pgfsetfillcolor{currentfill}%
\pgfsetlinewidth{0.000000pt}%
\definecolor{currentstroke}{rgb}{0.000000,0.000000,0.000000}%
\pgfsetstrokecolor{currentstroke}%
\pgfsetdash{}{0pt}%
\pgfpathmoveto{\pgfqpoint{2.789950in}{0.798346in}}%
\pgfpathlineto{\pgfqpoint{3.236099in}{0.723497in}}%
\pgfpathlineto{\pgfqpoint{3.502561in}{2.073396in}}%
\pgfpathlineto{\pgfqpoint{3.043326in}{2.141782in}}%
\pgfpathlineto{\pgfqpoint{2.789950in}{0.798346in}}%
\pgfpathclose%
\pgfusepath{fill}%
\end{pgfscope}%
\begin{pgfscope}%
\pgfpathrectangle{\pgfqpoint{0.017500in}{0.017500in}}{\pgfqpoint{6.218282in}{3.580372in}} %
\pgfusepath{clip}%
\pgfsetbuttcap%
\pgfsetroundjoin%
\definecolor{currentfill}{rgb}{0.800830,0.250829,0.225696}%
\pgfsetfillcolor{currentfill}%
\pgfsetlinewidth{0.000000pt}%
\definecolor{currentstroke}{rgb}{0.000000,0.000000,0.000000}%
\pgfsetstrokecolor{currentstroke}%
\pgfsetdash{}{0pt}%
\pgfpathmoveto{\pgfqpoint{3.306485in}{2.382499in}}%
\pgfpathlineto{\pgfqpoint{3.564492in}{2.541257in}}%
\pgfpathlineto{\pgfqpoint{3.306485in}{2.382499in}}%
\pgfpathclose%
\pgfusepath{fill}%
\end{pgfscope}%
\begin{pgfscope}%
\pgfpathrectangle{\pgfqpoint{0.017500in}{0.017500in}}{\pgfqpoint{6.218282in}{3.580372in}} %
\pgfusepath{clip}%
\pgfsetbuttcap%
\pgfsetroundjoin%
\definecolor{currentfill}{rgb}{0.800830,0.250829,0.225696}%
\pgfsetfillcolor{currentfill}%
\pgfsetlinewidth{0.000000pt}%
\definecolor{currentstroke}{rgb}{0.000000,0.000000,0.000000}%
\pgfsetstrokecolor{currentstroke}%
\pgfsetdash{}{0pt}%
\pgfpathmoveto{\pgfqpoint{3.306485in}{2.382499in}}%
\pgfpathlineto{\pgfqpoint{3.564492in}{2.541257in}}%
\pgfpathlineto{\pgfqpoint{3.306485in}{2.382499in}}%
\pgfpathclose%
\pgfusepath{fill}%
\end{pgfscope}%
\begin{pgfscope}%
\pgfpathrectangle{\pgfqpoint{0.017500in}{0.017500in}}{\pgfqpoint{6.218282in}{3.580372in}} %
\pgfusepath{clip}%
\pgfsetbuttcap%
\pgfsetroundjoin%
\definecolor{currentfill}{rgb}{0.800830,0.250829,0.225696}%
\pgfsetfillcolor{currentfill}%
\pgfsetlinewidth{0.000000pt}%
\definecolor{currentstroke}{rgb}{0.000000,0.000000,0.000000}%
\pgfsetstrokecolor{currentstroke}%
\pgfsetdash{}{0pt}%
\pgfpathmoveto{\pgfqpoint{3.306485in}{2.382499in}}%
\pgfpathlineto{\pgfqpoint{3.564492in}{2.541257in}}%
\pgfpathlineto{\pgfqpoint{3.306485in}{2.382499in}}%
\pgfpathclose%
\pgfusepath{fill}%
\end{pgfscope}%
\begin{pgfscope}%
\pgfpathrectangle{\pgfqpoint{0.017500in}{0.017500in}}{\pgfqpoint{6.218282in}{3.580372in}} %
\pgfusepath{clip}%
\pgfsetbuttcap%
\pgfsetroundjoin%
\definecolor{currentfill}{rgb}{0.800830,0.250829,0.225696}%
\pgfsetfillcolor{currentfill}%
\pgfsetlinewidth{0.000000pt}%
\definecolor{currentstroke}{rgb}{0.000000,0.000000,0.000000}%
\pgfsetstrokecolor{currentstroke}%
\pgfsetdash{}{0pt}%
\pgfpathmoveto{\pgfqpoint{3.306485in}{2.382499in}}%
\pgfpathlineto{\pgfqpoint{3.564492in}{2.541257in}}%
\pgfpathlineto{\pgfqpoint{3.306485in}{2.382499in}}%
\pgfpathclose%
\pgfusepath{fill}%
\end{pgfscope}%
\begin{pgfscope}%
\pgfpathrectangle{\pgfqpoint{0.017500in}{0.017500in}}{\pgfqpoint{6.218282in}{3.580372in}} %
\pgfusepath{clip}%
\pgfsetbuttcap%
\pgfsetroundjoin%
\definecolor{currentfill}{rgb}{0.800830,0.250829,0.225696}%
\pgfsetfillcolor{currentfill}%
\pgfsetlinewidth{0.000000pt}%
\definecolor{currentstroke}{rgb}{0.000000,0.000000,0.000000}%
\pgfsetstrokecolor{currentstroke}%
\pgfsetdash{}{0pt}%
\pgfpathmoveto{\pgfqpoint{3.306485in}{2.382499in}}%
\pgfpathlineto{\pgfqpoint{3.564492in}{2.541257in}}%
\pgfpathlineto{\pgfqpoint{3.306485in}{2.382499in}}%
\pgfpathclose%
\pgfusepath{fill}%
\end{pgfscope}%
\begin{pgfscope}%
\pgfpathrectangle{\pgfqpoint{0.017500in}{0.017500in}}{\pgfqpoint{6.218282in}{3.580372in}} %
\pgfusepath{clip}%
\pgfsetbuttcap%
\pgfsetroundjoin%
\definecolor{currentfill}{rgb}{0.800830,0.250829,0.225696}%
\pgfsetfillcolor{currentfill}%
\pgfsetlinewidth{0.000000pt}%
\definecolor{currentstroke}{rgb}{0.000000,0.000000,0.000000}%
\pgfsetstrokecolor{currentstroke}%
\pgfsetdash{}{0pt}%
\pgfpathmoveto{\pgfqpoint{3.306485in}{2.382499in}}%
\pgfpathlineto{\pgfqpoint{3.564492in}{2.541257in}}%
\pgfpathlineto{\pgfqpoint{3.306485in}{2.382499in}}%
\pgfpathclose%
\pgfusepath{fill}%
\end{pgfscope}%
\begin{pgfscope}%
\pgfpathrectangle{\pgfqpoint{0.017500in}{0.017500in}}{\pgfqpoint{6.218282in}{3.580372in}} %
\pgfusepath{clip}%
\pgfsetbuttcap%
\pgfsetroundjoin%
\definecolor{currentfill}{rgb}{0.800830,0.250829,0.225696}%
\pgfsetfillcolor{currentfill}%
\pgfsetlinewidth{0.000000pt}%
\definecolor{currentstroke}{rgb}{0.000000,0.000000,0.000000}%
\pgfsetstrokecolor{currentstroke}%
\pgfsetdash{}{0pt}%
\pgfpathmoveto{\pgfqpoint{3.306485in}{2.382499in}}%
\pgfpathlineto{\pgfqpoint{3.564492in}{2.541257in}}%
\pgfpathlineto{\pgfqpoint{3.306485in}{2.382499in}}%
\pgfpathclose%
\pgfusepath{fill}%
\end{pgfscope}%
\begin{pgfscope}%
\pgfpathrectangle{\pgfqpoint{0.017500in}{0.017500in}}{\pgfqpoint{6.218282in}{3.580372in}} %
\pgfusepath{clip}%
\pgfsetbuttcap%
\pgfsetroundjoin%
\definecolor{currentfill}{rgb}{0.800830,0.250829,0.225696}%
\pgfsetfillcolor{currentfill}%
\pgfsetlinewidth{0.000000pt}%
\definecolor{currentstroke}{rgb}{0.000000,0.000000,0.000000}%
\pgfsetstrokecolor{currentstroke}%
\pgfsetdash{}{0pt}%
\pgfpathmoveto{\pgfqpoint{3.306485in}{2.382499in}}%
\pgfpathlineto{\pgfqpoint{3.564492in}{2.541257in}}%
\pgfpathlineto{\pgfqpoint{3.306485in}{2.382499in}}%
\pgfpathclose%
\pgfusepath{fill}%
\end{pgfscope}%
\begin{pgfscope}%
\pgfpathrectangle{\pgfqpoint{0.017500in}{0.017500in}}{\pgfqpoint{6.218282in}{3.580372in}} %
\pgfusepath{clip}%
\pgfsetbuttcap%
\pgfsetroundjoin%
\definecolor{currentfill}{rgb}{0.800830,0.250829,0.225696}%
\pgfsetfillcolor{currentfill}%
\pgfsetlinewidth{0.000000pt}%
\definecolor{currentstroke}{rgb}{0.000000,0.000000,0.000000}%
\pgfsetstrokecolor{currentstroke}%
\pgfsetdash{}{0pt}%
\pgfpathmoveto{\pgfqpoint{3.306485in}{2.382499in}}%
\pgfpathlineto{\pgfqpoint{3.564492in}{2.541257in}}%
\pgfpathlineto{\pgfqpoint{3.306485in}{2.382499in}}%
\pgfpathclose%
\pgfusepath{fill}%
\end{pgfscope}%
\begin{pgfscope}%
\pgfpathrectangle{\pgfqpoint{0.017500in}{0.017500in}}{\pgfqpoint{6.218282in}{3.580372in}} %
\pgfusepath{clip}%
\pgfsetbuttcap%
\pgfsetroundjoin%
\definecolor{currentfill}{rgb}{0.800830,0.250829,0.225696}%
\pgfsetfillcolor{currentfill}%
\pgfsetlinewidth{0.000000pt}%
\definecolor{currentstroke}{rgb}{0.000000,0.000000,0.000000}%
\pgfsetstrokecolor{currentstroke}%
\pgfsetdash{}{0pt}%
\pgfpathmoveto{\pgfqpoint{3.306485in}{2.382499in}}%
\pgfpathlineto{\pgfqpoint{3.564492in}{2.541257in}}%
\pgfpathlineto{\pgfqpoint{3.306485in}{2.382499in}}%
\pgfpathclose%
\pgfusepath{fill}%
\end{pgfscope}%
\begin{pgfscope}%
\pgfpathrectangle{\pgfqpoint{0.017500in}{0.017500in}}{\pgfqpoint{6.218282in}{3.580372in}} %
\pgfusepath{clip}%
\pgfsetbuttcap%
\pgfsetroundjoin%
\definecolor{currentfill}{rgb}{0.800830,0.250829,0.225696}%
\pgfsetfillcolor{currentfill}%
\pgfsetlinewidth{0.000000pt}%
\definecolor{currentstroke}{rgb}{0.000000,0.000000,0.000000}%
\pgfsetstrokecolor{currentstroke}%
\pgfsetdash{}{0pt}%
\pgfpathmoveto{\pgfqpoint{3.306485in}{2.382499in}}%
\pgfpathlineto{\pgfqpoint{3.564492in}{2.541257in}}%
\pgfpathlineto{\pgfqpoint{3.306485in}{2.382499in}}%
\pgfpathclose%
\pgfusepath{fill}%
\end{pgfscope}%
\begin{pgfscope}%
\pgfpathrectangle{\pgfqpoint{0.017500in}{0.017500in}}{\pgfqpoint{6.218282in}{3.580372in}} %
\pgfusepath{clip}%
\pgfsetbuttcap%
\pgfsetroundjoin%
\definecolor{currentfill}{rgb}{0.800830,0.250829,0.225696}%
\pgfsetfillcolor{currentfill}%
\pgfsetlinewidth{0.000000pt}%
\definecolor{currentstroke}{rgb}{0.000000,0.000000,0.000000}%
\pgfsetstrokecolor{currentstroke}%
\pgfsetdash{}{0pt}%
\pgfpathmoveto{\pgfqpoint{3.306485in}{2.382499in}}%
\pgfpathlineto{\pgfqpoint{3.564492in}{2.541257in}}%
\pgfpathlineto{\pgfqpoint{3.306485in}{2.382499in}}%
\pgfpathclose%
\pgfusepath{fill}%
\end{pgfscope}%
\begin{pgfscope}%
\pgfpathrectangle{\pgfqpoint{0.017500in}{0.017500in}}{\pgfqpoint{6.218282in}{3.580372in}} %
\pgfusepath{clip}%
\pgfsetbuttcap%
\pgfsetroundjoin%
\definecolor{currentfill}{rgb}{0.800830,0.250829,0.225696}%
\pgfsetfillcolor{currentfill}%
\pgfsetlinewidth{0.000000pt}%
\definecolor{currentstroke}{rgb}{0.000000,0.000000,0.000000}%
\pgfsetstrokecolor{currentstroke}%
\pgfsetdash{}{0pt}%
\pgfpathmoveto{\pgfqpoint{3.306485in}{2.382499in}}%
\pgfpathlineto{\pgfqpoint{3.564492in}{2.541257in}}%
\pgfpathlineto{\pgfqpoint{3.306485in}{2.382499in}}%
\pgfpathclose%
\pgfusepath{fill}%
\end{pgfscope}%
\begin{pgfscope}%
\pgfpathrectangle{\pgfqpoint{0.017500in}{0.017500in}}{\pgfqpoint{6.218282in}{3.580372in}} %
\pgfusepath{clip}%
\pgfsetbuttcap%
\pgfsetroundjoin%
\definecolor{currentfill}{rgb}{0.800830,0.250829,0.225696}%
\pgfsetfillcolor{currentfill}%
\pgfsetlinewidth{0.000000pt}%
\definecolor{currentstroke}{rgb}{0.000000,0.000000,0.000000}%
\pgfsetstrokecolor{currentstroke}%
\pgfsetdash{}{0pt}%
\pgfpathmoveto{\pgfqpoint{3.306485in}{2.382499in}}%
\pgfpathlineto{\pgfqpoint{3.564492in}{2.541257in}}%
\pgfpathlineto{\pgfqpoint{3.306485in}{2.382499in}}%
\pgfpathclose%
\pgfusepath{fill}%
\end{pgfscope}%
\begin{pgfscope}%
\pgfpathrectangle{\pgfqpoint{0.017500in}{0.017500in}}{\pgfqpoint{6.218282in}{3.580372in}} %
\pgfusepath{clip}%
\pgfsetbuttcap%
\pgfsetroundjoin%
\definecolor{currentfill}{rgb}{0.800830,0.250829,0.225696}%
\pgfsetfillcolor{currentfill}%
\pgfsetlinewidth{0.000000pt}%
\definecolor{currentstroke}{rgb}{0.000000,0.000000,0.000000}%
\pgfsetstrokecolor{currentstroke}%
\pgfsetdash{}{0pt}%
\pgfpathmoveto{\pgfqpoint{3.306485in}{2.382499in}}%
\pgfpathlineto{\pgfqpoint{3.564492in}{2.541257in}}%
\pgfpathlineto{\pgfqpoint{3.306485in}{2.382499in}}%
\pgfpathclose%
\pgfusepath{fill}%
\end{pgfscope}%
\begin{pgfscope}%
\pgfpathrectangle{\pgfqpoint{0.017500in}{0.017500in}}{\pgfqpoint{6.218282in}{3.580372in}} %
\pgfusepath{clip}%
\pgfsetbuttcap%
\pgfsetroundjoin%
\definecolor{currentfill}{rgb}{0.800830,0.250829,0.225696}%
\pgfsetfillcolor{currentfill}%
\pgfsetlinewidth{0.000000pt}%
\definecolor{currentstroke}{rgb}{0.000000,0.000000,0.000000}%
\pgfsetstrokecolor{currentstroke}%
\pgfsetdash{}{0pt}%
\pgfpathmoveto{\pgfqpoint{3.306485in}{2.382499in}}%
\pgfpathlineto{\pgfqpoint{3.564492in}{2.541257in}}%
\pgfpathlineto{\pgfqpoint{3.306485in}{2.382499in}}%
\pgfpathclose%
\pgfusepath{fill}%
\end{pgfscope}%
\begin{pgfscope}%
\pgfpathrectangle{\pgfqpoint{0.017500in}{0.017500in}}{\pgfqpoint{6.218282in}{3.580372in}} %
\pgfusepath{clip}%
\pgfsetbuttcap%
\pgfsetroundjoin%
\definecolor{currentfill}{rgb}{0.800830,0.250829,0.225696}%
\pgfsetfillcolor{currentfill}%
\pgfsetlinewidth{0.000000pt}%
\definecolor{currentstroke}{rgb}{0.000000,0.000000,0.000000}%
\pgfsetstrokecolor{currentstroke}%
\pgfsetdash{}{0pt}%
\pgfpathmoveto{\pgfqpoint{3.306485in}{2.382499in}}%
\pgfpathlineto{\pgfqpoint{3.564492in}{2.541257in}}%
\pgfpathlineto{\pgfqpoint{3.306485in}{2.382499in}}%
\pgfpathclose%
\pgfusepath{fill}%
\end{pgfscope}%
\begin{pgfscope}%
\pgfpathrectangle{\pgfqpoint{0.017500in}{0.017500in}}{\pgfqpoint{6.218282in}{3.580372in}} %
\pgfusepath{clip}%
\pgfsetbuttcap%
\pgfsetroundjoin%
\definecolor{currentfill}{rgb}{0.800830,0.250829,0.225696}%
\pgfsetfillcolor{currentfill}%
\pgfsetlinewidth{0.000000pt}%
\definecolor{currentstroke}{rgb}{0.000000,0.000000,0.000000}%
\pgfsetstrokecolor{currentstroke}%
\pgfsetdash{}{0pt}%
\pgfpathmoveto{\pgfqpoint{3.306485in}{2.382499in}}%
\pgfpathlineto{\pgfqpoint{3.564492in}{2.541257in}}%
\pgfpathlineto{\pgfqpoint{3.306485in}{2.382499in}}%
\pgfpathclose%
\pgfusepath{fill}%
\end{pgfscope}%
\begin{pgfscope}%
\pgfpathrectangle{\pgfqpoint{0.017500in}{0.017500in}}{\pgfqpoint{6.218282in}{3.580372in}} %
\pgfusepath{clip}%
\pgfsetbuttcap%
\pgfsetroundjoin%
\definecolor{currentfill}{rgb}{0.800830,0.250829,0.225696}%
\pgfsetfillcolor{currentfill}%
\pgfsetlinewidth{0.000000pt}%
\definecolor{currentstroke}{rgb}{0.000000,0.000000,0.000000}%
\pgfsetstrokecolor{currentstroke}%
\pgfsetdash{}{0pt}%
\pgfpathmoveto{\pgfqpoint{3.306485in}{2.382499in}}%
\pgfpathlineto{\pgfqpoint{3.564492in}{2.541257in}}%
\pgfpathlineto{\pgfqpoint{3.306485in}{2.382499in}}%
\pgfpathclose%
\pgfusepath{fill}%
\end{pgfscope}%
\begin{pgfscope}%
\pgfpathrectangle{\pgfqpoint{0.017500in}{0.017500in}}{\pgfqpoint{6.218282in}{3.580372in}} %
\pgfusepath{clip}%
\pgfsetbuttcap%
\pgfsetroundjoin%
\definecolor{currentfill}{rgb}{0.800830,0.250829,0.225696}%
\pgfsetfillcolor{currentfill}%
\pgfsetlinewidth{0.000000pt}%
\definecolor{currentstroke}{rgb}{0.000000,0.000000,0.000000}%
\pgfsetstrokecolor{currentstroke}%
\pgfsetdash{}{0pt}%
\pgfpathmoveto{\pgfqpoint{3.306485in}{2.382499in}}%
\pgfpathlineto{\pgfqpoint{3.564492in}{2.541257in}}%
\pgfpathlineto{\pgfqpoint{3.306485in}{2.382499in}}%
\pgfpathclose%
\pgfusepath{fill}%
\end{pgfscope}%
\begin{pgfscope}%
\pgfpathrectangle{\pgfqpoint{0.017500in}{0.017500in}}{\pgfqpoint{6.218282in}{3.580372in}} %
\pgfusepath{clip}%
\pgfsetbuttcap%
\pgfsetroundjoin%
\definecolor{currentfill}{rgb}{0.800830,0.250829,0.225696}%
\pgfsetfillcolor{currentfill}%
\pgfsetlinewidth{0.000000pt}%
\definecolor{currentstroke}{rgb}{0.000000,0.000000,0.000000}%
\pgfsetstrokecolor{currentstroke}%
\pgfsetdash{}{0pt}%
\pgfpathmoveto{\pgfqpoint{3.306485in}{2.382499in}}%
\pgfpathlineto{\pgfqpoint{3.564492in}{2.541257in}}%
\pgfpathlineto{\pgfqpoint{3.306485in}{2.382499in}}%
\pgfpathclose%
\pgfusepath{fill}%
\end{pgfscope}%
\begin{pgfscope}%
\pgfpathrectangle{\pgfqpoint{0.017500in}{0.017500in}}{\pgfqpoint{6.218282in}{3.580372in}} %
\pgfusepath{clip}%
\pgfsetbuttcap%
\pgfsetroundjoin%
\definecolor{currentfill}{rgb}{0.800830,0.250829,0.225696}%
\pgfsetfillcolor{currentfill}%
\pgfsetlinewidth{0.000000pt}%
\definecolor{currentstroke}{rgb}{0.000000,0.000000,0.000000}%
\pgfsetstrokecolor{currentstroke}%
\pgfsetdash{}{0pt}%
\pgfpathmoveto{\pgfqpoint{3.306485in}{2.382499in}}%
\pgfpathlineto{\pgfqpoint{3.564492in}{2.541257in}}%
\pgfpathlineto{\pgfqpoint{3.306485in}{2.382499in}}%
\pgfpathclose%
\pgfusepath{fill}%
\end{pgfscope}%
\begin{pgfscope}%
\pgfpathrectangle{\pgfqpoint{0.017500in}{0.017500in}}{\pgfqpoint{6.218282in}{3.580372in}} %
\pgfusepath{clip}%
\pgfsetbuttcap%
\pgfsetroundjoin%
\definecolor{currentfill}{rgb}{0.800830,0.250829,0.225696}%
\pgfsetfillcolor{currentfill}%
\pgfsetlinewidth{0.000000pt}%
\definecolor{currentstroke}{rgb}{0.000000,0.000000,0.000000}%
\pgfsetstrokecolor{currentstroke}%
\pgfsetdash{}{0pt}%
\pgfpathmoveto{\pgfqpoint{3.306485in}{2.382499in}}%
\pgfpathlineto{\pgfqpoint{3.564492in}{2.541257in}}%
\pgfpathlineto{\pgfqpoint{3.306485in}{2.382499in}}%
\pgfpathclose%
\pgfusepath{fill}%
\end{pgfscope}%
\begin{pgfscope}%
\pgfpathrectangle{\pgfqpoint{0.017500in}{0.017500in}}{\pgfqpoint{6.218282in}{3.580372in}} %
\pgfusepath{clip}%
\pgfsetbuttcap%
\pgfsetroundjoin%
\definecolor{currentfill}{rgb}{0.800830,0.250829,0.225696}%
\pgfsetfillcolor{currentfill}%
\pgfsetlinewidth{0.000000pt}%
\definecolor{currentstroke}{rgb}{0.000000,0.000000,0.000000}%
\pgfsetstrokecolor{currentstroke}%
\pgfsetdash{}{0pt}%
\pgfpathmoveto{\pgfqpoint{3.306485in}{2.382499in}}%
\pgfpathlineto{\pgfqpoint{3.564492in}{2.541257in}}%
\pgfpathlineto{\pgfqpoint{3.306485in}{2.382499in}}%
\pgfpathclose%
\pgfusepath{fill}%
\end{pgfscope}%
\begin{pgfscope}%
\pgfpathrectangle{\pgfqpoint{0.017500in}{0.017500in}}{\pgfqpoint{6.218282in}{3.580372in}} %
\pgfusepath{clip}%
\pgfsetbuttcap%
\pgfsetroundjoin%
\definecolor{currentfill}{rgb}{0.800830,0.250829,0.225696}%
\pgfsetfillcolor{currentfill}%
\pgfsetlinewidth{0.000000pt}%
\definecolor{currentstroke}{rgb}{0.000000,0.000000,0.000000}%
\pgfsetstrokecolor{currentstroke}%
\pgfsetdash{}{0pt}%
\pgfpathmoveto{\pgfqpoint{3.306485in}{2.382499in}}%
\pgfpathlineto{\pgfqpoint{3.564492in}{2.541257in}}%
\pgfpathlineto{\pgfqpoint{3.306485in}{2.382499in}}%
\pgfpathclose%
\pgfusepath{fill}%
\end{pgfscope}%
\begin{pgfscope}%
\pgfpathrectangle{\pgfqpoint{0.017500in}{0.017500in}}{\pgfqpoint{6.218282in}{3.580372in}} %
\pgfusepath{clip}%
\pgfsetbuttcap%
\pgfsetroundjoin%
\definecolor{currentfill}{rgb}{0.800830,0.250829,0.225696}%
\pgfsetfillcolor{currentfill}%
\pgfsetlinewidth{0.000000pt}%
\definecolor{currentstroke}{rgb}{0.000000,0.000000,0.000000}%
\pgfsetstrokecolor{currentstroke}%
\pgfsetdash{}{0pt}%
\pgfpathmoveto{\pgfqpoint{3.306485in}{2.382499in}}%
\pgfpathlineto{\pgfqpoint{3.564492in}{2.541257in}}%
\pgfpathlineto{\pgfqpoint{3.306485in}{2.382499in}}%
\pgfpathclose%
\pgfusepath{fill}%
\end{pgfscope}%
\begin{pgfscope}%
\pgfpathrectangle{\pgfqpoint{0.017500in}{0.017500in}}{\pgfqpoint{6.218282in}{3.580372in}} %
\pgfusepath{clip}%
\pgfsetbuttcap%
\pgfsetroundjoin%
\definecolor{currentfill}{rgb}{0.800830,0.250829,0.225696}%
\pgfsetfillcolor{currentfill}%
\pgfsetlinewidth{0.000000pt}%
\definecolor{currentstroke}{rgb}{0.000000,0.000000,0.000000}%
\pgfsetstrokecolor{currentstroke}%
\pgfsetdash{}{0pt}%
\pgfpathmoveto{\pgfqpoint{3.306485in}{2.382499in}}%
\pgfpathlineto{\pgfqpoint{3.564492in}{2.541257in}}%
\pgfpathlineto{\pgfqpoint{3.306485in}{2.382499in}}%
\pgfpathclose%
\pgfusepath{fill}%
\end{pgfscope}%
\begin{pgfscope}%
\pgfpathrectangle{\pgfqpoint{0.017500in}{0.017500in}}{\pgfqpoint{6.218282in}{3.580372in}} %
\pgfusepath{clip}%
\pgfsetbuttcap%
\pgfsetroundjoin%
\definecolor{currentfill}{rgb}{0.800830,0.250829,0.225696}%
\pgfsetfillcolor{currentfill}%
\pgfsetlinewidth{0.000000pt}%
\definecolor{currentstroke}{rgb}{0.000000,0.000000,0.000000}%
\pgfsetstrokecolor{currentstroke}%
\pgfsetdash{}{0pt}%
\pgfpathmoveto{\pgfqpoint{3.306485in}{2.382499in}}%
\pgfpathlineto{\pgfqpoint{3.564492in}{2.541257in}}%
\pgfpathlineto{\pgfqpoint{3.306485in}{2.382499in}}%
\pgfpathclose%
\pgfusepath{fill}%
\end{pgfscope}%
\begin{pgfscope}%
\pgfpathrectangle{\pgfqpoint{0.017500in}{0.017500in}}{\pgfqpoint{6.218282in}{3.580372in}} %
\pgfusepath{clip}%
\pgfsetbuttcap%
\pgfsetroundjoin%
\definecolor{currentfill}{rgb}{0.800830,0.250829,0.225696}%
\pgfsetfillcolor{currentfill}%
\pgfsetlinewidth{0.000000pt}%
\definecolor{currentstroke}{rgb}{0.000000,0.000000,0.000000}%
\pgfsetstrokecolor{currentstroke}%
\pgfsetdash{}{0pt}%
\pgfpathmoveto{\pgfqpoint{3.306485in}{2.382499in}}%
\pgfpathlineto{\pgfqpoint{3.564492in}{2.541257in}}%
\pgfpathlineto{\pgfqpoint{3.306485in}{2.382499in}}%
\pgfpathclose%
\pgfusepath{fill}%
\end{pgfscope}%
\begin{pgfscope}%
\pgfpathrectangle{\pgfqpoint{0.017500in}{0.017500in}}{\pgfqpoint{6.218282in}{3.580372in}} %
\pgfusepath{clip}%
\pgfsetbuttcap%
\pgfsetroundjoin%
\definecolor{currentfill}{rgb}{0.800830,0.250829,0.225696}%
\pgfsetfillcolor{currentfill}%
\pgfsetlinewidth{0.000000pt}%
\definecolor{currentstroke}{rgb}{0.000000,0.000000,0.000000}%
\pgfsetstrokecolor{currentstroke}%
\pgfsetdash{}{0pt}%
\pgfpathmoveto{\pgfqpoint{3.306485in}{2.382499in}}%
\pgfpathlineto{\pgfqpoint{3.564492in}{2.541257in}}%
\pgfpathlineto{\pgfqpoint{3.306485in}{2.382499in}}%
\pgfpathclose%
\pgfusepath{fill}%
\end{pgfscope}%
\begin{pgfscope}%
\pgfpathrectangle{\pgfqpoint{0.017500in}{0.017500in}}{\pgfqpoint{6.218282in}{3.580372in}} %
\pgfusepath{clip}%
\pgfsetbuttcap%
\pgfsetroundjoin%
\definecolor{currentfill}{rgb}{0.800830,0.250829,0.225696}%
\pgfsetfillcolor{currentfill}%
\pgfsetlinewidth{0.000000pt}%
\definecolor{currentstroke}{rgb}{0.000000,0.000000,0.000000}%
\pgfsetstrokecolor{currentstroke}%
\pgfsetdash{}{0pt}%
\pgfpathmoveto{\pgfqpoint{3.306485in}{2.382499in}}%
\pgfpathlineto{\pgfqpoint{3.763398in}{2.297457in}}%
\pgfpathlineto{\pgfqpoint{4.018447in}{2.468167in}}%
\pgfpathlineto{\pgfqpoint{3.564492in}{2.541257in}}%
\pgfpathlineto{\pgfqpoint{3.306485in}{2.382499in}}%
\pgfpathclose%
\pgfusepath{fill}%
\end{pgfscope}%
\begin{pgfscope}%
\pgfpathrectangle{\pgfqpoint{0.017500in}{0.017500in}}{\pgfqpoint{6.218282in}{3.580372in}} %
\pgfusepath{clip}%
\pgfsetbuttcap%
\pgfsetroundjoin%
\definecolor{currentfill}{rgb}{0.800830,0.250829,0.225696}%
\pgfsetfillcolor{currentfill}%
\pgfsetlinewidth{0.000000pt}%
\definecolor{currentstroke}{rgb}{0.000000,0.000000,0.000000}%
\pgfsetstrokecolor{currentstroke}%
\pgfsetdash{}{0pt}%
\pgfpathmoveto{\pgfqpoint{3.306485in}{2.382499in}}%
\pgfpathlineto{\pgfqpoint{3.763398in}{2.297457in}}%
\pgfpathlineto{\pgfqpoint{4.018447in}{2.468167in}}%
\pgfpathlineto{\pgfqpoint{3.564492in}{2.541257in}}%
\pgfpathlineto{\pgfqpoint{3.306485in}{2.382499in}}%
\pgfpathclose%
\pgfusepath{fill}%
\end{pgfscope}%
\begin{pgfscope}%
\pgfpathrectangle{\pgfqpoint{0.017500in}{0.017500in}}{\pgfqpoint{6.218282in}{3.580372in}} %
\pgfusepath{clip}%
\pgfsetbuttcap%
\pgfsetroundjoin%
\definecolor{currentfill}{rgb}{0.800830,0.250829,0.225696}%
\pgfsetfillcolor{currentfill}%
\pgfsetlinewidth{0.000000pt}%
\definecolor{currentstroke}{rgb}{0.000000,0.000000,0.000000}%
\pgfsetstrokecolor{currentstroke}%
\pgfsetdash{}{0pt}%
\pgfpathmoveto{\pgfqpoint{3.306485in}{2.382499in}}%
\pgfpathlineto{\pgfqpoint{3.763398in}{2.297457in}}%
\pgfpathlineto{\pgfqpoint{4.018447in}{2.468167in}}%
\pgfpathlineto{\pgfqpoint{3.564492in}{2.541257in}}%
\pgfpathlineto{\pgfqpoint{3.306485in}{2.382499in}}%
\pgfpathclose%
\pgfusepath{fill}%
\end{pgfscope}%
\begin{pgfscope}%
\pgfpathrectangle{\pgfqpoint{0.017500in}{0.017500in}}{\pgfqpoint{6.218282in}{3.580372in}} %
\pgfusepath{clip}%
\pgfsetbuttcap%
\pgfsetroundjoin%
\definecolor{currentfill}{rgb}{0.800830,0.250829,0.225696}%
\pgfsetfillcolor{currentfill}%
\pgfsetlinewidth{0.000000pt}%
\definecolor{currentstroke}{rgb}{0.000000,0.000000,0.000000}%
\pgfsetstrokecolor{currentstroke}%
\pgfsetdash{}{0pt}%
\pgfpathmoveto{\pgfqpoint{3.306485in}{2.382499in}}%
\pgfpathlineto{\pgfqpoint{3.763398in}{2.297457in}}%
\pgfpathlineto{\pgfqpoint{4.018447in}{2.468167in}}%
\pgfpathlineto{\pgfqpoint{3.564492in}{2.541257in}}%
\pgfpathlineto{\pgfqpoint{3.306485in}{2.382499in}}%
\pgfpathclose%
\pgfusepath{fill}%
\end{pgfscope}%
\begin{pgfscope}%
\pgfpathrectangle{\pgfqpoint{0.017500in}{0.017500in}}{\pgfqpoint{6.218282in}{3.580372in}} %
\pgfusepath{clip}%
\pgfsetbuttcap%
\pgfsetroundjoin%
\definecolor{currentfill}{rgb}{0.800830,0.250829,0.225696}%
\pgfsetfillcolor{currentfill}%
\pgfsetlinewidth{0.000000pt}%
\definecolor{currentstroke}{rgb}{0.000000,0.000000,0.000000}%
\pgfsetstrokecolor{currentstroke}%
\pgfsetdash{}{0pt}%
\pgfpathmoveto{\pgfqpoint{3.306485in}{2.382499in}}%
\pgfpathlineto{\pgfqpoint{3.763398in}{2.297457in}}%
\pgfpathlineto{\pgfqpoint{4.018447in}{2.468167in}}%
\pgfpathlineto{\pgfqpoint{3.564492in}{2.541257in}}%
\pgfpathlineto{\pgfqpoint{3.306485in}{2.382499in}}%
\pgfpathclose%
\pgfusepath{fill}%
\end{pgfscope}%
\begin{pgfscope}%
\pgfpathrectangle{\pgfqpoint{0.017500in}{0.017500in}}{\pgfqpoint{6.218282in}{3.580372in}} %
\pgfusepath{clip}%
\pgfsetbuttcap%
\pgfsetroundjoin%
\definecolor{currentfill}{rgb}{0.800830,0.250829,0.225696}%
\pgfsetfillcolor{currentfill}%
\pgfsetlinewidth{0.000000pt}%
\definecolor{currentstroke}{rgb}{0.000000,0.000000,0.000000}%
\pgfsetstrokecolor{currentstroke}%
\pgfsetdash{}{0pt}%
\pgfpathmoveto{\pgfqpoint{3.306485in}{2.382499in}}%
\pgfpathlineto{\pgfqpoint{3.763398in}{2.297457in}}%
\pgfpathlineto{\pgfqpoint{4.018447in}{2.468167in}}%
\pgfpathlineto{\pgfqpoint{3.564492in}{2.541257in}}%
\pgfpathlineto{\pgfqpoint{3.306485in}{2.382499in}}%
\pgfpathclose%
\pgfusepath{fill}%
\end{pgfscope}%
\begin{pgfscope}%
\pgfpathrectangle{\pgfqpoint{0.017500in}{0.017500in}}{\pgfqpoint{6.218282in}{3.580372in}} %
\pgfusepath{clip}%
\pgfsetbuttcap%
\pgfsetroundjoin%
\definecolor{currentfill}{rgb}{0.899534,0.440692,0.344107}%
\pgfsetfillcolor{currentfill}%
\pgfsetlinewidth{0.000000pt}%
\definecolor{currentstroke}{rgb}{0.000000,0.000000,0.000000}%
\pgfsetstrokecolor{currentstroke}%
\pgfsetdash{}{0pt}%
\pgfpathmoveto{\pgfqpoint{2.590432in}{2.210709in}}%
\pgfpathlineto{\pgfqpoint{2.855513in}{2.451418in}}%
\pgfpathlineto{\pgfqpoint{2.590432in}{2.210709in}}%
\pgfpathclose%
\pgfusepath{fill}%
\end{pgfscope}%
\begin{pgfscope}%
\pgfpathrectangle{\pgfqpoint{0.017500in}{0.017500in}}{\pgfqpoint{6.218282in}{3.580372in}} %
\pgfusepath{clip}%
\pgfsetbuttcap%
\pgfsetroundjoin%
\definecolor{currentfill}{rgb}{0.899534,0.440692,0.344107}%
\pgfsetfillcolor{currentfill}%
\pgfsetlinewidth{0.000000pt}%
\definecolor{currentstroke}{rgb}{0.000000,0.000000,0.000000}%
\pgfsetstrokecolor{currentstroke}%
\pgfsetdash{}{0pt}%
\pgfpathmoveto{\pgfqpoint{2.590432in}{2.210709in}}%
\pgfpathlineto{\pgfqpoint{2.855513in}{2.451418in}}%
\pgfpathlineto{\pgfqpoint{2.590432in}{2.210709in}}%
\pgfpathclose%
\pgfusepath{fill}%
\end{pgfscope}%
\begin{pgfscope}%
\pgfpathrectangle{\pgfqpoint{0.017500in}{0.017500in}}{\pgfqpoint{6.218282in}{3.580372in}} %
\pgfusepath{clip}%
\pgfsetbuttcap%
\pgfsetroundjoin%
\definecolor{currentfill}{rgb}{0.899534,0.440692,0.344107}%
\pgfsetfillcolor{currentfill}%
\pgfsetlinewidth{0.000000pt}%
\definecolor{currentstroke}{rgb}{0.000000,0.000000,0.000000}%
\pgfsetstrokecolor{currentstroke}%
\pgfsetdash{}{0pt}%
\pgfpathmoveto{\pgfqpoint{2.590432in}{2.210709in}}%
\pgfpathlineto{\pgfqpoint{2.855513in}{2.451418in}}%
\pgfpathlineto{\pgfqpoint{2.590432in}{2.210709in}}%
\pgfpathclose%
\pgfusepath{fill}%
\end{pgfscope}%
\begin{pgfscope}%
\pgfpathrectangle{\pgfqpoint{0.017500in}{0.017500in}}{\pgfqpoint{6.218282in}{3.580372in}} %
\pgfusepath{clip}%
\pgfsetbuttcap%
\pgfsetroundjoin%
\definecolor{currentfill}{rgb}{0.899534,0.440692,0.344107}%
\pgfsetfillcolor{currentfill}%
\pgfsetlinewidth{0.000000pt}%
\definecolor{currentstroke}{rgb}{0.000000,0.000000,0.000000}%
\pgfsetstrokecolor{currentstroke}%
\pgfsetdash{}{0pt}%
\pgfpathmoveto{\pgfqpoint{2.590432in}{2.210709in}}%
\pgfpathlineto{\pgfqpoint{2.855513in}{2.451418in}}%
\pgfpathlineto{\pgfqpoint{2.590432in}{2.210709in}}%
\pgfpathclose%
\pgfusepath{fill}%
\end{pgfscope}%
\begin{pgfscope}%
\pgfpathrectangle{\pgfqpoint{0.017500in}{0.017500in}}{\pgfqpoint{6.218282in}{3.580372in}} %
\pgfusepath{clip}%
\pgfsetbuttcap%
\pgfsetroundjoin%
\definecolor{currentfill}{rgb}{0.899534,0.440692,0.344107}%
\pgfsetfillcolor{currentfill}%
\pgfsetlinewidth{0.000000pt}%
\definecolor{currentstroke}{rgb}{0.000000,0.000000,0.000000}%
\pgfsetstrokecolor{currentstroke}%
\pgfsetdash{}{0pt}%
\pgfpathmoveto{\pgfqpoint{2.590432in}{2.210709in}}%
\pgfpathlineto{\pgfqpoint{2.855513in}{2.451418in}}%
\pgfpathlineto{\pgfqpoint{2.590432in}{2.210709in}}%
\pgfpathclose%
\pgfusepath{fill}%
\end{pgfscope}%
\begin{pgfscope}%
\pgfpathrectangle{\pgfqpoint{0.017500in}{0.017500in}}{\pgfqpoint{6.218282in}{3.580372in}} %
\pgfusepath{clip}%
\pgfsetbuttcap%
\pgfsetroundjoin%
\definecolor{currentfill}{rgb}{0.899534,0.440692,0.344107}%
\pgfsetfillcolor{currentfill}%
\pgfsetlinewidth{0.000000pt}%
\definecolor{currentstroke}{rgb}{0.000000,0.000000,0.000000}%
\pgfsetstrokecolor{currentstroke}%
\pgfsetdash{}{0pt}%
\pgfpathmoveto{\pgfqpoint{2.590432in}{2.210709in}}%
\pgfpathlineto{\pgfqpoint{2.855513in}{2.451418in}}%
\pgfpathlineto{\pgfqpoint{2.590432in}{2.210709in}}%
\pgfpathclose%
\pgfusepath{fill}%
\end{pgfscope}%
\begin{pgfscope}%
\pgfpathrectangle{\pgfqpoint{0.017500in}{0.017500in}}{\pgfqpoint{6.218282in}{3.580372in}} %
\pgfusepath{clip}%
\pgfsetbuttcap%
\pgfsetroundjoin%
\definecolor{currentfill}{rgb}{0.899534,0.440692,0.344107}%
\pgfsetfillcolor{currentfill}%
\pgfsetlinewidth{0.000000pt}%
\definecolor{currentstroke}{rgb}{0.000000,0.000000,0.000000}%
\pgfsetstrokecolor{currentstroke}%
\pgfsetdash{}{0pt}%
\pgfpathmoveto{\pgfqpoint{2.590432in}{2.210709in}}%
\pgfpathlineto{\pgfqpoint{2.855513in}{2.451418in}}%
\pgfpathlineto{\pgfqpoint{2.590432in}{2.210709in}}%
\pgfpathclose%
\pgfusepath{fill}%
\end{pgfscope}%
\begin{pgfscope}%
\pgfpathrectangle{\pgfqpoint{0.017500in}{0.017500in}}{\pgfqpoint{6.218282in}{3.580372in}} %
\pgfusepath{clip}%
\pgfsetbuttcap%
\pgfsetroundjoin%
\definecolor{currentfill}{rgb}{0.899534,0.440692,0.344107}%
\pgfsetfillcolor{currentfill}%
\pgfsetlinewidth{0.000000pt}%
\definecolor{currentstroke}{rgb}{0.000000,0.000000,0.000000}%
\pgfsetstrokecolor{currentstroke}%
\pgfsetdash{}{0pt}%
\pgfpathmoveto{\pgfqpoint{2.590432in}{2.210709in}}%
\pgfpathlineto{\pgfqpoint{2.855513in}{2.451418in}}%
\pgfpathlineto{\pgfqpoint{2.590432in}{2.210709in}}%
\pgfpathclose%
\pgfusepath{fill}%
\end{pgfscope}%
\begin{pgfscope}%
\pgfpathrectangle{\pgfqpoint{0.017500in}{0.017500in}}{\pgfqpoint{6.218282in}{3.580372in}} %
\pgfusepath{clip}%
\pgfsetbuttcap%
\pgfsetroundjoin%
\definecolor{currentfill}{rgb}{0.899534,0.440692,0.344107}%
\pgfsetfillcolor{currentfill}%
\pgfsetlinewidth{0.000000pt}%
\definecolor{currentstroke}{rgb}{0.000000,0.000000,0.000000}%
\pgfsetstrokecolor{currentstroke}%
\pgfsetdash{}{0pt}%
\pgfpathmoveto{\pgfqpoint{2.590432in}{2.210709in}}%
\pgfpathlineto{\pgfqpoint{2.855513in}{2.451418in}}%
\pgfpathlineto{\pgfqpoint{2.590432in}{2.210709in}}%
\pgfpathclose%
\pgfusepath{fill}%
\end{pgfscope}%
\begin{pgfscope}%
\pgfpathrectangle{\pgfqpoint{0.017500in}{0.017500in}}{\pgfqpoint{6.218282in}{3.580372in}} %
\pgfusepath{clip}%
\pgfsetbuttcap%
\pgfsetroundjoin%
\definecolor{currentfill}{rgb}{0.899534,0.440692,0.344107}%
\pgfsetfillcolor{currentfill}%
\pgfsetlinewidth{0.000000pt}%
\definecolor{currentstroke}{rgb}{0.000000,0.000000,0.000000}%
\pgfsetstrokecolor{currentstroke}%
\pgfsetdash{}{0pt}%
\pgfpathmoveto{\pgfqpoint{2.590432in}{2.210709in}}%
\pgfpathlineto{\pgfqpoint{2.855513in}{2.451418in}}%
\pgfpathlineto{\pgfqpoint{2.590432in}{2.210709in}}%
\pgfpathclose%
\pgfusepath{fill}%
\end{pgfscope}%
\begin{pgfscope}%
\pgfpathrectangle{\pgfqpoint{0.017500in}{0.017500in}}{\pgfqpoint{6.218282in}{3.580372in}} %
\pgfusepath{clip}%
\pgfsetbuttcap%
\pgfsetroundjoin%
\definecolor{currentfill}{rgb}{0.899534,0.440692,0.344107}%
\pgfsetfillcolor{currentfill}%
\pgfsetlinewidth{0.000000pt}%
\definecolor{currentstroke}{rgb}{0.000000,0.000000,0.000000}%
\pgfsetstrokecolor{currentstroke}%
\pgfsetdash{}{0pt}%
\pgfpathmoveto{\pgfqpoint{2.590432in}{2.210709in}}%
\pgfpathlineto{\pgfqpoint{2.855513in}{2.451418in}}%
\pgfpathlineto{\pgfqpoint{2.590432in}{2.210709in}}%
\pgfpathclose%
\pgfusepath{fill}%
\end{pgfscope}%
\begin{pgfscope}%
\pgfpathrectangle{\pgfqpoint{0.017500in}{0.017500in}}{\pgfqpoint{6.218282in}{3.580372in}} %
\pgfusepath{clip}%
\pgfsetbuttcap%
\pgfsetroundjoin%
\definecolor{currentfill}{rgb}{0.899534,0.440692,0.344107}%
\pgfsetfillcolor{currentfill}%
\pgfsetlinewidth{0.000000pt}%
\definecolor{currentstroke}{rgb}{0.000000,0.000000,0.000000}%
\pgfsetstrokecolor{currentstroke}%
\pgfsetdash{}{0pt}%
\pgfpathmoveto{\pgfqpoint{2.590432in}{2.210709in}}%
\pgfpathlineto{\pgfqpoint{2.855513in}{2.451418in}}%
\pgfpathlineto{\pgfqpoint{2.590432in}{2.210709in}}%
\pgfpathclose%
\pgfusepath{fill}%
\end{pgfscope}%
\begin{pgfscope}%
\pgfpathrectangle{\pgfqpoint{0.017500in}{0.017500in}}{\pgfqpoint{6.218282in}{3.580372in}} %
\pgfusepath{clip}%
\pgfsetbuttcap%
\pgfsetroundjoin%
\definecolor{currentfill}{rgb}{0.899534,0.440692,0.344107}%
\pgfsetfillcolor{currentfill}%
\pgfsetlinewidth{0.000000pt}%
\definecolor{currentstroke}{rgb}{0.000000,0.000000,0.000000}%
\pgfsetstrokecolor{currentstroke}%
\pgfsetdash{}{0pt}%
\pgfpathmoveto{\pgfqpoint{2.590432in}{2.210709in}}%
\pgfpathlineto{\pgfqpoint{2.855513in}{2.451418in}}%
\pgfpathlineto{\pgfqpoint{2.590432in}{2.210709in}}%
\pgfpathclose%
\pgfusepath{fill}%
\end{pgfscope}%
\begin{pgfscope}%
\pgfpathrectangle{\pgfqpoint{0.017500in}{0.017500in}}{\pgfqpoint{6.218282in}{3.580372in}} %
\pgfusepath{clip}%
\pgfsetbuttcap%
\pgfsetroundjoin%
\definecolor{currentfill}{rgb}{0.899534,0.440692,0.344107}%
\pgfsetfillcolor{currentfill}%
\pgfsetlinewidth{0.000000pt}%
\definecolor{currentstroke}{rgb}{0.000000,0.000000,0.000000}%
\pgfsetstrokecolor{currentstroke}%
\pgfsetdash{}{0pt}%
\pgfpathmoveto{\pgfqpoint{2.590432in}{2.210709in}}%
\pgfpathlineto{\pgfqpoint{2.855513in}{2.451418in}}%
\pgfpathlineto{\pgfqpoint{2.590432in}{2.210709in}}%
\pgfpathclose%
\pgfusepath{fill}%
\end{pgfscope}%
\begin{pgfscope}%
\pgfpathrectangle{\pgfqpoint{0.017500in}{0.017500in}}{\pgfqpoint{6.218282in}{3.580372in}} %
\pgfusepath{clip}%
\pgfsetbuttcap%
\pgfsetroundjoin%
\definecolor{currentfill}{rgb}{0.899534,0.440692,0.344107}%
\pgfsetfillcolor{currentfill}%
\pgfsetlinewidth{0.000000pt}%
\definecolor{currentstroke}{rgb}{0.000000,0.000000,0.000000}%
\pgfsetstrokecolor{currentstroke}%
\pgfsetdash{}{0pt}%
\pgfpathmoveto{\pgfqpoint{2.590432in}{2.210709in}}%
\pgfpathlineto{\pgfqpoint{2.855513in}{2.451418in}}%
\pgfpathlineto{\pgfqpoint{2.590432in}{2.210709in}}%
\pgfpathclose%
\pgfusepath{fill}%
\end{pgfscope}%
\begin{pgfscope}%
\pgfpathrectangle{\pgfqpoint{0.017500in}{0.017500in}}{\pgfqpoint{6.218282in}{3.580372in}} %
\pgfusepath{clip}%
\pgfsetbuttcap%
\pgfsetroundjoin%
\definecolor{currentfill}{rgb}{0.899534,0.440692,0.344107}%
\pgfsetfillcolor{currentfill}%
\pgfsetlinewidth{0.000000pt}%
\definecolor{currentstroke}{rgb}{0.000000,0.000000,0.000000}%
\pgfsetstrokecolor{currentstroke}%
\pgfsetdash{}{0pt}%
\pgfpathmoveto{\pgfqpoint{2.590432in}{2.210709in}}%
\pgfpathlineto{\pgfqpoint{2.855513in}{2.451418in}}%
\pgfpathlineto{\pgfqpoint{2.590432in}{2.210709in}}%
\pgfpathclose%
\pgfusepath{fill}%
\end{pgfscope}%
\begin{pgfscope}%
\pgfpathrectangle{\pgfqpoint{0.017500in}{0.017500in}}{\pgfqpoint{6.218282in}{3.580372in}} %
\pgfusepath{clip}%
\pgfsetbuttcap%
\pgfsetroundjoin%
\definecolor{currentfill}{rgb}{0.899534,0.440692,0.344107}%
\pgfsetfillcolor{currentfill}%
\pgfsetlinewidth{0.000000pt}%
\definecolor{currentstroke}{rgb}{0.000000,0.000000,0.000000}%
\pgfsetstrokecolor{currentstroke}%
\pgfsetdash{}{0pt}%
\pgfpathmoveto{\pgfqpoint{2.590432in}{2.210709in}}%
\pgfpathlineto{\pgfqpoint{2.855513in}{2.451418in}}%
\pgfpathlineto{\pgfqpoint{2.590432in}{2.210709in}}%
\pgfpathclose%
\pgfusepath{fill}%
\end{pgfscope}%
\begin{pgfscope}%
\pgfpathrectangle{\pgfqpoint{0.017500in}{0.017500in}}{\pgfqpoint{6.218282in}{3.580372in}} %
\pgfusepath{clip}%
\pgfsetbuttcap%
\pgfsetroundjoin%
\definecolor{currentfill}{rgb}{0.899534,0.440692,0.344107}%
\pgfsetfillcolor{currentfill}%
\pgfsetlinewidth{0.000000pt}%
\definecolor{currentstroke}{rgb}{0.000000,0.000000,0.000000}%
\pgfsetstrokecolor{currentstroke}%
\pgfsetdash{}{0pt}%
\pgfpathmoveto{\pgfqpoint{2.590432in}{2.210709in}}%
\pgfpathlineto{\pgfqpoint{2.855513in}{2.451418in}}%
\pgfpathlineto{\pgfqpoint{2.590432in}{2.210709in}}%
\pgfpathclose%
\pgfusepath{fill}%
\end{pgfscope}%
\begin{pgfscope}%
\pgfpathrectangle{\pgfqpoint{0.017500in}{0.017500in}}{\pgfqpoint{6.218282in}{3.580372in}} %
\pgfusepath{clip}%
\pgfsetbuttcap%
\pgfsetroundjoin%
\definecolor{currentfill}{rgb}{0.899534,0.440692,0.344107}%
\pgfsetfillcolor{currentfill}%
\pgfsetlinewidth{0.000000pt}%
\definecolor{currentstroke}{rgb}{0.000000,0.000000,0.000000}%
\pgfsetstrokecolor{currentstroke}%
\pgfsetdash{}{0pt}%
\pgfpathmoveto{\pgfqpoint{2.590432in}{2.210709in}}%
\pgfpathlineto{\pgfqpoint{2.855513in}{2.451418in}}%
\pgfpathlineto{\pgfqpoint{2.590432in}{2.210709in}}%
\pgfpathclose%
\pgfusepath{fill}%
\end{pgfscope}%
\begin{pgfscope}%
\pgfpathrectangle{\pgfqpoint{0.017500in}{0.017500in}}{\pgfqpoint{6.218282in}{3.580372in}} %
\pgfusepath{clip}%
\pgfsetbuttcap%
\pgfsetroundjoin%
\definecolor{currentfill}{rgb}{0.899534,0.440692,0.344107}%
\pgfsetfillcolor{currentfill}%
\pgfsetlinewidth{0.000000pt}%
\definecolor{currentstroke}{rgb}{0.000000,0.000000,0.000000}%
\pgfsetstrokecolor{currentstroke}%
\pgfsetdash{}{0pt}%
\pgfpathmoveto{\pgfqpoint{2.590432in}{2.210709in}}%
\pgfpathlineto{\pgfqpoint{2.855513in}{2.451418in}}%
\pgfpathlineto{\pgfqpoint{2.590432in}{2.210709in}}%
\pgfpathclose%
\pgfusepath{fill}%
\end{pgfscope}%
\begin{pgfscope}%
\pgfpathrectangle{\pgfqpoint{0.017500in}{0.017500in}}{\pgfqpoint{6.218282in}{3.580372in}} %
\pgfusepath{clip}%
\pgfsetbuttcap%
\pgfsetroundjoin%
\definecolor{currentfill}{rgb}{0.899534,0.440692,0.344107}%
\pgfsetfillcolor{currentfill}%
\pgfsetlinewidth{0.000000pt}%
\definecolor{currentstroke}{rgb}{0.000000,0.000000,0.000000}%
\pgfsetstrokecolor{currentstroke}%
\pgfsetdash{}{0pt}%
\pgfpathmoveto{\pgfqpoint{2.590432in}{2.210709in}}%
\pgfpathlineto{\pgfqpoint{2.855513in}{2.451418in}}%
\pgfpathlineto{\pgfqpoint{2.590432in}{2.210709in}}%
\pgfpathclose%
\pgfusepath{fill}%
\end{pgfscope}%
\begin{pgfscope}%
\pgfpathrectangle{\pgfqpoint{0.017500in}{0.017500in}}{\pgfqpoint{6.218282in}{3.580372in}} %
\pgfusepath{clip}%
\pgfsetbuttcap%
\pgfsetroundjoin%
\definecolor{currentfill}{rgb}{0.899534,0.440692,0.344107}%
\pgfsetfillcolor{currentfill}%
\pgfsetlinewidth{0.000000pt}%
\definecolor{currentstroke}{rgb}{0.000000,0.000000,0.000000}%
\pgfsetstrokecolor{currentstroke}%
\pgfsetdash{}{0pt}%
\pgfpathmoveto{\pgfqpoint{2.590432in}{2.210709in}}%
\pgfpathlineto{\pgfqpoint{2.855513in}{2.451418in}}%
\pgfpathlineto{\pgfqpoint{2.590432in}{2.210709in}}%
\pgfpathclose%
\pgfusepath{fill}%
\end{pgfscope}%
\begin{pgfscope}%
\pgfpathrectangle{\pgfqpoint{0.017500in}{0.017500in}}{\pgfqpoint{6.218282in}{3.580372in}} %
\pgfusepath{clip}%
\pgfsetbuttcap%
\pgfsetroundjoin%
\definecolor{currentfill}{rgb}{0.899534,0.440692,0.344107}%
\pgfsetfillcolor{currentfill}%
\pgfsetlinewidth{0.000000pt}%
\definecolor{currentstroke}{rgb}{0.000000,0.000000,0.000000}%
\pgfsetstrokecolor{currentstroke}%
\pgfsetdash{}{0pt}%
\pgfpathmoveto{\pgfqpoint{2.590432in}{2.210709in}}%
\pgfpathlineto{\pgfqpoint{2.855513in}{2.451418in}}%
\pgfpathlineto{\pgfqpoint{2.590432in}{2.210709in}}%
\pgfpathclose%
\pgfusepath{fill}%
\end{pgfscope}%
\begin{pgfscope}%
\pgfpathrectangle{\pgfqpoint{0.017500in}{0.017500in}}{\pgfqpoint{6.218282in}{3.580372in}} %
\pgfusepath{clip}%
\pgfsetbuttcap%
\pgfsetroundjoin%
\definecolor{currentfill}{rgb}{0.899534,0.440692,0.344107}%
\pgfsetfillcolor{currentfill}%
\pgfsetlinewidth{0.000000pt}%
\definecolor{currentstroke}{rgb}{0.000000,0.000000,0.000000}%
\pgfsetstrokecolor{currentstroke}%
\pgfsetdash{}{0pt}%
\pgfpathmoveto{\pgfqpoint{2.590432in}{2.210709in}}%
\pgfpathlineto{\pgfqpoint{2.855513in}{2.451418in}}%
\pgfpathlineto{\pgfqpoint{2.590432in}{2.210709in}}%
\pgfpathclose%
\pgfusepath{fill}%
\end{pgfscope}%
\begin{pgfscope}%
\pgfpathrectangle{\pgfqpoint{0.017500in}{0.017500in}}{\pgfqpoint{6.218282in}{3.580372in}} %
\pgfusepath{clip}%
\pgfsetbuttcap%
\pgfsetroundjoin%
\definecolor{currentfill}{rgb}{0.899534,0.440692,0.344107}%
\pgfsetfillcolor{currentfill}%
\pgfsetlinewidth{0.000000pt}%
\definecolor{currentstroke}{rgb}{0.000000,0.000000,0.000000}%
\pgfsetstrokecolor{currentstroke}%
\pgfsetdash{}{0pt}%
\pgfpathmoveto{\pgfqpoint{2.590432in}{2.210709in}}%
\pgfpathlineto{\pgfqpoint{2.855513in}{2.451418in}}%
\pgfpathlineto{\pgfqpoint{2.590432in}{2.210709in}}%
\pgfpathclose%
\pgfusepath{fill}%
\end{pgfscope}%
\begin{pgfscope}%
\pgfpathrectangle{\pgfqpoint{0.017500in}{0.017500in}}{\pgfqpoint{6.218282in}{3.580372in}} %
\pgfusepath{clip}%
\pgfsetbuttcap%
\pgfsetroundjoin%
\definecolor{currentfill}{rgb}{0.899534,0.440692,0.344107}%
\pgfsetfillcolor{currentfill}%
\pgfsetlinewidth{0.000000pt}%
\definecolor{currentstroke}{rgb}{0.000000,0.000000,0.000000}%
\pgfsetstrokecolor{currentstroke}%
\pgfsetdash{}{0pt}%
\pgfpathmoveto{\pgfqpoint{2.590432in}{2.210709in}}%
\pgfpathlineto{\pgfqpoint{2.855513in}{2.451418in}}%
\pgfpathlineto{\pgfqpoint{2.590432in}{2.210709in}}%
\pgfpathclose%
\pgfusepath{fill}%
\end{pgfscope}%
\begin{pgfscope}%
\pgfpathrectangle{\pgfqpoint{0.017500in}{0.017500in}}{\pgfqpoint{6.218282in}{3.580372in}} %
\pgfusepath{clip}%
\pgfsetbuttcap%
\pgfsetroundjoin%
\definecolor{currentfill}{rgb}{0.899534,0.440692,0.344107}%
\pgfsetfillcolor{currentfill}%
\pgfsetlinewidth{0.000000pt}%
\definecolor{currentstroke}{rgb}{0.000000,0.000000,0.000000}%
\pgfsetstrokecolor{currentstroke}%
\pgfsetdash{}{0pt}%
\pgfpathmoveto{\pgfqpoint{2.590432in}{2.210709in}}%
\pgfpathlineto{\pgfqpoint{2.855513in}{2.451418in}}%
\pgfpathlineto{\pgfqpoint{2.590432in}{2.210709in}}%
\pgfpathclose%
\pgfusepath{fill}%
\end{pgfscope}%
\begin{pgfscope}%
\pgfpathrectangle{\pgfqpoint{0.017500in}{0.017500in}}{\pgfqpoint{6.218282in}{3.580372in}} %
\pgfusepath{clip}%
\pgfsetbuttcap%
\pgfsetroundjoin%
\definecolor{currentfill}{rgb}{0.899534,0.440692,0.344107}%
\pgfsetfillcolor{currentfill}%
\pgfsetlinewidth{0.000000pt}%
\definecolor{currentstroke}{rgb}{0.000000,0.000000,0.000000}%
\pgfsetstrokecolor{currentstroke}%
\pgfsetdash{}{0pt}%
\pgfpathmoveto{\pgfqpoint{2.590432in}{2.210709in}}%
\pgfpathlineto{\pgfqpoint{2.855513in}{2.451418in}}%
\pgfpathlineto{\pgfqpoint{2.590432in}{2.210709in}}%
\pgfpathclose%
\pgfusepath{fill}%
\end{pgfscope}%
\begin{pgfscope}%
\pgfpathrectangle{\pgfqpoint{0.017500in}{0.017500in}}{\pgfqpoint{6.218282in}{3.580372in}} %
\pgfusepath{clip}%
\pgfsetbuttcap%
\pgfsetroundjoin%
\definecolor{currentfill}{rgb}{0.899534,0.440692,0.344107}%
\pgfsetfillcolor{currentfill}%
\pgfsetlinewidth{0.000000pt}%
\definecolor{currentstroke}{rgb}{0.000000,0.000000,0.000000}%
\pgfsetstrokecolor{currentstroke}%
\pgfsetdash{}{0pt}%
\pgfpathmoveto{\pgfqpoint{2.590432in}{2.210709in}}%
\pgfpathlineto{\pgfqpoint{2.855513in}{2.451418in}}%
\pgfpathlineto{\pgfqpoint{2.590432in}{2.210709in}}%
\pgfpathclose%
\pgfusepath{fill}%
\end{pgfscope}%
\begin{pgfscope}%
\pgfpathrectangle{\pgfqpoint{0.017500in}{0.017500in}}{\pgfqpoint{6.218282in}{3.580372in}} %
\pgfusepath{clip}%
\pgfsetbuttcap%
\pgfsetroundjoin%
\definecolor{currentfill}{rgb}{0.899534,0.440692,0.344107}%
\pgfsetfillcolor{currentfill}%
\pgfsetlinewidth{0.000000pt}%
\definecolor{currentstroke}{rgb}{0.000000,0.000000,0.000000}%
\pgfsetstrokecolor{currentstroke}%
\pgfsetdash{}{0pt}%
\pgfpathmoveto{\pgfqpoint{2.590432in}{2.210709in}}%
\pgfpathlineto{\pgfqpoint{2.855513in}{2.451418in}}%
\pgfpathlineto{\pgfqpoint{2.590432in}{2.210709in}}%
\pgfpathclose%
\pgfusepath{fill}%
\end{pgfscope}%
\begin{pgfscope}%
\pgfpathrectangle{\pgfqpoint{0.017500in}{0.017500in}}{\pgfqpoint{6.218282in}{3.580372in}} %
\pgfusepath{clip}%
\pgfsetbuttcap%
\pgfsetroundjoin%
\definecolor{currentfill}{rgb}{0.243520,0.319189,0.771672}%
\pgfsetfillcolor{currentfill}%
\pgfsetlinewidth{0.000000pt}%
\definecolor{currentstroke}{rgb}{0.000000,0.000000,0.000000}%
\pgfsetstrokecolor{currentstroke}%
\pgfsetdash{}{0pt}%
\pgfpathmoveto{\pgfqpoint{3.236099in}{0.723497in}}%
\pgfpathlineto{\pgfqpoint{3.502561in}{2.073396in}}%
\pgfpathlineto{\pgfqpoint{3.236099in}{0.723497in}}%
\pgfpathclose%
\pgfusepath{fill}%
\end{pgfscope}%
\begin{pgfscope}%
\pgfpathrectangle{\pgfqpoint{0.017500in}{0.017500in}}{\pgfqpoint{6.218282in}{3.580372in}} %
\pgfusepath{clip}%
\pgfsetbuttcap%
\pgfsetroundjoin%
\definecolor{currentfill}{rgb}{0.243520,0.319189,0.771672}%
\pgfsetfillcolor{currentfill}%
\pgfsetlinewidth{0.000000pt}%
\definecolor{currentstroke}{rgb}{0.000000,0.000000,0.000000}%
\pgfsetstrokecolor{currentstroke}%
\pgfsetdash{}{0pt}%
\pgfpathmoveto{\pgfqpoint{3.236099in}{0.723497in}}%
\pgfpathlineto{\pgfqpoint{3.502561in}{2.073396in}}%
\pgfpathlineto{\pgfqpoint{3.236099in}{0.723497in}}%
\pgfpathclose%
\pgfusepath{fill}%
\end{pgfscope}%
\begin{pgfscope}%
\pgfpathrectangle{\pgfqpoint{0.017500in}{0.017500in}}{\pgfqpoint{6.218282in}{3.580372in}} %
\pgfusepath{clip}%
\pgfsetbuttcap%
\pgfsetroundjoin%
\definecolor{currentfill}{rgb}{0.243520,0.319189,0.771672}%
\pgfsetfillcolor{currentfill}%
\pgfsetlinewidth{0.000000pt}%
\definecolor{currentstroke}{rgb}{0.000000,0.000000,0.000000}%
\pgfsetstrokecolor{currentstroke}%
\pgfsetdash{}{0pt}%
\pgfpathmoveto{\pgfqpoint{3.236099in}{0.723497in}}%
\pgfpathlineto{\pgfqpoint{3.502561in}{2.073396in}}%
\pgfpathlineto{\pgfqpoint{3.236099in}{0.723497in}}%
\pgfpathclose%
\pgfusepath{fill}%
\end{pgfscope}%
\begin{pgfscope}%
\pgfpathrectangle{\pgfqpoint{0.017500in}{0.017500in}}{\pgfqpoint{6.218282in}{3.580372in}} %
\pgfusepath{clip}%
\pgfsetbuttcap%
\pgfsetroundjoin%
\definecolor{currentfill}{rgb}{0.243520,0.319189,0.771672}%
\pgfsetfillcolor{currentfill}%
\pgfsetlinewidth{0.000000pt}%
\definecolor{currentstroke}{rgb}{0.000000,0.000000,0.000000}%
\pgfsetstrokecolor{currentstroke}%
\pgfsetdash{}{0pt}%
\pgfpathmoveto{\pgfqpoint{3.236099in}{0.723497in}}%
\pgfpathlineto{\pgfqpoint{3.502561in}{2.073396in}}%
\pgfpathlineto{\pgfqpoint{3.236099in}{0.723497in}}%
\pgfpathclose%
\pgfusepath{fill}%
\end{pgfscope}%
\begin{pgfscope}%
\pgfpathrectangle{\pgfqpoint{0.017500in}{0.017500in}}{\pgfqpoint{6.218282in}{3.580372in}} %
\pgfusepath{clip}%
\pgfsetbuttcap%
\pgfsetroundjoin%
\definecolor{currentfill}{rgb}{0.243520,0.319189,0.771672}%
\pgfsetfillcolor{currentfill}%
\pgfsetlinewidth{0.000000pt}%
\definecolor{currentstroke}{rgb}{0.000000,0.000000,0.000000}%
\pgfsetstrokecolor{currentstroke}%
\pgfsetdash{}{0pt}%
\pgfpathmoveto{\pgfqpoint{3.236099in}{0.723497in}}%
\pgfpathlineto{\pgfqpoint{3.502561in}{2.073396in}}%
\pgfpathlineto{\pgfqpoint{3.236099in}{0.723497in}}%
\pgfpathclose%
\pgfusepath{fill}%
\end{pgfscope}%
\begin{pgfscope}%
\pgfpathrectangle{\pgfqpoint{0.017500in}{0.017500in}}{\pgfqpoint{6.218282in}{3.580372in}} %
\pgfusepath{clip}%
\pgfsetbuttcap%
\pgfsetroundjoin%
\definecolor{currentfill}{rgb}{0.243520,0.319189,0.771672}%
\pgfsetfillcolor{currentfill}%
\pgfsetlinewidth{0.000000pt}%
\definecolor{currentstroke}{rgb}{0.000000,0.000000,0.000000}%
\pgfsetstrokecolor{currentstroke}%
\pgfsetdash{}{0pt}%
\pgfpathmoveto{\pgfqpoint{3.236099in}{0.723497in}}%
\pgfpathlineto{\pgfqpoint{3.502561in}{2.073396in}}%
\pgfpathlineto{\pgfqpoint{3.236099in}{0.723497in}}%
\pgfpathclose%
\pgfusepath{fill}%
\end{pgfscope}%
\begin{pgfscope}%
\pgfpathrectangle{\pgfqpoint{0.017500in}{0.017500in}}{\pgfqpoint{6.218282in}{3.580372in}} %
\pgfusepath{clip}%
\pgfsetbuttcap%
\pgfsetroundjoin%
\definecolor{currentfill}{rgb}{0.243520,0.319189,0.771672}%
\pgfsetfillcolor{currentfill}%
\pgfsetlinewidth{0.000000pt}%
\definecolor{currentstroke}{rgb}{0.000000,0.000000,0.000000}%
\pgfsetstrokecolor{currentstroke}%
\pgfsetdash{}{0pt}%
\pgfpathmoveto{\pgfqpoint{3.236099in}{0.723497in}}%
\pgfpathlineto{\pgfqpoint{3.502561in}{2.073396in}}%
\pgfpathlineto{\pgfqpoint{3.236099in}{0.723497in}}%
\pgfpathclose%
\pgfusepath{fill}%
\end{pgfscope}%
\begin{pgfscope}%
\pgfpathrectangle{\pgfqpoint{0.017500in}{0.017500in}}{\pgfqpoint{6.218282in}{3.580372in}} %
\pgfusepath{clip}%
\pgfsetbuttcap%
\pgfsetroundjoin%
\definecolor{currentfill}{rgb}{0.243520,0.319189,0.771672}%
\pgfsetfillcolor{currentfill}%
\pgfsetlinewidth{0.000000pt}%
\definecolor{currentstroke}{rgb}{0.000000,0.000000,0.000000}%
\pgfsetstrokecolor{currentstroke}%
\pgfsetdash{}{0pt}%
\pgfpathmoveto{\pgfqpoint{3.236099in}{0.723497in}}%
\pgfpathlineto{\pgfqpoint{3.502561in}{2.073396in}}%
\pgfpathlineto{\pgfqpoint{3.236099in}{0.723497in}}%
\pgfpathclose%
\pgfusepath{fill}%
\end{pgfscope}%
\begin{pgfscope}%
\pgfpathrectangle{\pgfqpoint{0.017500in}{0.017500in}}{\pgfqpoint{6.218282in}{3.580372in}} %
\pgfusepath{clip}%
\pgfsetbuttcap%
\pgfsetroundjoin%
\definecolor{currentfill}{rgb}{0.243520,0.319189,0.771672}%
\pgfsetfillcolor{currentfill}%
\pgfsetlinewidth{0.000000pt}%
\definecolor{currentstroke}{rgb}{0.000000,0.000000,0.000000}%
\pgfsetstrokecolor{currentstroke}%
\pgfsetdash{}{0pt}%
\pgfpathmoveto{\pgfqpoint{3.236099in}{0.723497in}}%
\pgfpathlineto{\pgfqpoint{3.502561in}{2.073396in}}%
\pgfpathlineto{\pgfqpoint{3.236099in}{0.723497in}}%
\pgfpathclose%
\pgfusepath{fill}%
\end{pgfscope}%
\begin{pgfscope}%
\pgfpathrectangle{\pgfqpoint{0.017500in}{0.017500in}}{\pgfqpoint{6.218282in}{3.580372in}} %
\pgfusepath{clip}%
\pgfsetbuttcap%
\pgfsetroundjoin%
\definecolor{currentfill}{rgb}{0.243520,0.319189,0.771672}%
\pgfsetfillcolor{currentfill}%
\pgfsetlinewidth{0.000000pt}%
\definecolor{currentstroke}{rgb}{0.000000,0.000000,0.000000}%
\pgfsetstrokecolor{currentstroke}%
\pgfsetdash{}{0pt}%
\pgfpathmoveto{\pgfqpoint{3.236099in}{0.723497in}}%
\pgfpathlineto{\pgfqpoint{3.502561in}{2.073396in}}%
\pgfpathlineto{\pgfqpoint{3.236099in}{0.723497in}}%
\pgfpathclose%
\pgfusepath{fill}%
\end{pgfscope}%
\begin{pgfscope}%
\pgfpathrectangle{\pgfqpoint{0.017500in}{0.017500in}}{\pgfqpoint{6.218282in}{3.580372in}} %
\pgfusepath{clip}%
\pgfsetbuttcap%
\pgfsetroundjoin%
\definecolor{currentfill}{rgb}{0.243520,0.319189,0.771672}%
\pgfsetfillcolor{currentfill}%
\pgfsetlinewidth{0.000000pt}%
\definecolor{currentstroke}{rgb}{0.000000,0.000000,0.000000}%
\pgfsetstrokecolor{currentstroke}%
\pgfsetdash{}{0pt}%
\pgfpathmoveto{\pgfqpoint{3.236099in}{0.723497in}}%
\pgfpathlineto{\pgfqpoint{3.502561in}{2.073396in}}%
\pgfpathlineto{\pgfqpoint{3.236099in}{0.723497in}}%
\pgfpathclose%
\pgfusepath{fill}%
\end{pgfscope}%
\begin{pgfscope}%
\pgfpathrectangle{\pgfqpoint{0.017500in}{0.017500in}}{\pgfqpoint{6.218282in}{3.580372in}} %
\pgfusepath{clip}%
\pgfsetbuttcap%
\pgfsetroundjoin%
\definecolor{currentfill}{rgb}{0.243520,0.319189,0.771672}%
\pgfsetfillcolor{currentfill}%
\pgfsetlinewidth{0.000000pt}%
\definecolor{currentstroke}{rgb}{0.000000,0.000000,0.000000}%
\pgfsetstrokecolor{currentstroke}%
\pgfsetdash{}{0pt}%
\pgfpathmoveto{\pgfqpoint{3.236099in}{0.723497in}}%
\pgfpathlineto{\pgfqpoint{3.502561in}{2.073396in}}%
\pgfpathlineto{\pgfqpoint{3.236099in}{0.723497in}}%
\pgfpathclose%
\pgfusepath{fill}%
\end{pgfscope}%
\begin{pgfscope}%
\pgfpathrectangle{\pgfqpoint{0.017500in}{0.017500in}}{\pgfqpoint{6.218282in}{3.580372in}} %
\pgfusepath{clip}%
\pgfsetbuttcap%
\pgfsetroundjoin%
\definecolor{currentfill}{rgb}{0.243520,0.319189,0.771672}%
\pgfsetfillcolor{currentfill}%
\pgfsetlinewidth{0.000000pt}%
\definecolor{currentstroke}{rgb}{0.000000,0.000000,0.000000}%
\pgfsetstrokecolor{currentstroke}%
\pgfsetdash{}{0pt}%
\pgfpathmoveto{\pgfqpoint{3.236099in}{0.723497in}}%
\pgfpathlineto{\pgfqpoint{3.502561in}{2.073396in}}%
\pgfpathlineto{\pgfqpoint{3.236099in}{0.723497in}}%
\pgfpathclose%
\pgfusepath{fill}%
\end{pgfscope}%
\begin{pgfscope}%
\pgfpathrectangle{\pgfqpoint{0.017500in}{0.017500in}}{\pgfqpoint{6.218282in}{3.580372in}} %
\pgfusepath{clip}%
\pgfsetbuttcap%
\pgfsetroundjoin%
\definecolor{currentfill}{rgb}{0.243520,0.319189,0.771672}%
\pgfsetfillcolor{currentfill}%
\pgfsetlinewidth{0.000000pt}%
\definecolor{currentstroke}{rgb}{0.000000,0.000000,0.000000}%
\pgfsetstrokecolor{currentstroke}%
\pgfsetdash{}{0pt}%
\pgfpathmoveto{\pgfqpoint{3.236099in}{0.723497in}}%
\pgfpathlineto{\pgfqpoint{3.502561in}{2.073396in}}%
\pgfpathlineto{\pgfqpoint{3.236099in}{0.723497in}}%
\pgfpathclose%
\pgfusepath{fill}%
\end{pgfscope}%
\begin{pgfscope}%
\pgfpathrectangle{\pgfqpoint{0.017500in}{0.017500in}}{\pgfqpoint{6.218282in}{3.580372in}} %
\pgfusepath{clip}%
\pgfsetbuttcap%
\pgfsetroundjoin%
\definecolor{currentfill}{rgb}{0.243520,0.319189,0.771672}%
\pgfsetfillcolor{currentfill}%
\pgfsetlinewidth{0.000000pt}%
\definecolor{currentstroke}{rgb}{0.000000,0.000000,0.000000}%
\pgfsetstrokecolor{currentstroke}%
\pgfsetdash{}{0pt}%
\pgfpathmoveto{\pgfqpoint{3.236099in}{0.723497in}}%
\pgfpathlineto{\pgfqpoint{3.502561in}{2.073396in}}%
\pgfpathlineto{\pgfqpoint{3.236099in}{0.723497in}}%
\pgfpathclose%
\pgfusepath{fill}%
\end{pgfscope}%
\begin{pgfscope}%
\pgfpathrectangle{\pgfqpoint{0.017500in}{0.017500in}}{\pgfqpoint{6.218282in}{3.580372in}} %
\pgfusepath{clip}%
\pgfsetbuttcap%
\pgfsetroundjoin%
\definecolor{currentfill}{rgb}{0.243520,0.319189,0.771672}%
\pgfsetfillcolor{currentfill}%
\pgfsetlinewidth{0.000000pt}%
\definecolor{currentstroke}{rgb}{0.000000,0.000000,0.000000}%
\pgfsetstrokecolor{currentstroke}%
\pgfsetdash{}{0pt}%
\pgfpathmoveto{\pgfqpoint{3.236099in}{0.723497in}}%
\pgfpathlineto{\pgfqpoint{3.502561in}{2.073396in}}%
\pgfpathlineto{\pgfqpoint{3.236099in}{0.723497in}}%
\pgfpathclose%
\pgfusepath{fill}%
\end{pgfscope}%
\begin{pgfscope}%
\pgfpathrectangle{\pgfqpoint{0.017500in}{0.017500in}}{\pgfqpoint{6.218282in}{3.580372in}} %
\pgfusepath{clip}%
\pgfsetbuttcap%
\pgfsetroundjoin%
\definecolor{currentfill}{rgb}{0.243520,0.319189,0.771672}%
\pgfsetfillcolor{currentfill}%
\pgfsetlinewidth{0.000000pt}%
\definecolor{currentstroke}{rgb}{0.000000,0.000000,0.000000}%
\pgfsetstrokecolor{currentstroke}%
\pgfsetdash{}{0pt}%
\pgfpathmoveto{\pgfqpoint{3.236099in}{0.723497in}}%
\pgfpathlineto{\pgfqpoint{3.502561in}{2.073396in}}%
\pgfpathlineto{\pgfqpoint{3.236099in}{0.723497in}}%
\pgfpathclose%
\pgfusepath{fill}%
\end{pgfscope}%
\begin{pgfscope}%
\pgfpathrectangle{\pgfqpoint{0.017500in}{0.017500in}}{\pgfqpoint{6.218282in}{3.580372in}} %
\pgfusepath{clip}%
\pgfsetbuttcap%
\pgfsetroundjoin%
\definecolor{currentfill}{rgb}{0.243520,0.319189,0.771672}%
\pgfsetfillcolor{currentfill}%
\pgfsetlinewidth{0.000000pt}%
\definecolor{currentstroke}{rgb}{0.000000,0.000000,0.000000}%
\pgfsetstrokecolor{currentstroke}%
\pgfsetdash{}{0pt}%
\pgfpathmoveto{\pgfqpoint{3.236099in}{0.723497in}}%
\pgfpathlineto{\pgfqpoint{3.502561in}{2.073396in}}%
\pgfpathlineto{\pgfqpoint{3.236099in}{0.723497in}}%
\pgfpathclose%
\pgfusepath{fill}%
\end{pgfscope}%
\begin{pgfscope}%
\pgfpathrectangle{\pgfqpoint{0.017500in}{0.017500in}}{\pgfqpoint{6.218282in}{3.580372in}} %
\pgfusepath{clip}%
\pgfsetbuttcap%
\pgfsetroundjoin%
\definecolor{currentfill}{rgb}{0.243520,0.319189,0.771672}%
\pgfsetfillcolor{currentfill}%
\pgfsetlinewidth{0.000000pt}%
\definecolor{currentstroke}{rgb}{0.000000,0.000000,0.000000}%
\pgfsetstrokecolor{currentstroke}%
\pgfsetdash{}{0pt}%
\pgfpathmoveto{\pgfqpoint{3.236099in}{0.723497in}}%
\pgfpathlineto{\pgfqpoint{3.502561in}{2.073396in}}%
\pgfpathlineto{\pgfqpoint{3.236099in}{0.723497in}}%
\pgfpathclose%
\pgfusepath{fill}%
\end{pgfscope}%
\begin{pgfscope}%
\pgfpathrectangle{\pgfqpoint{0.017500in}{0.017500in}}{\pgfqpoint{6.218282in}{3.580372in}} %
\pgfusepath{clip}%
\pgfsetbuttcap%
\pgfsetroundjoin%
\definecolor{currentfill}{rgb}{0.243520,0.319189,0.771672}%
\pgfsetfillcolor{currentfill}%
\pgfsetlinewidth{0.000000pt}%
\definecolor{currentstroke}{rgb}{0.000000,0.000000,0.000000}%
\pgfsetstrokecolor{currentstroke}%
\pgfsetdash{}{0pt}%
\pgfpathmoveto{\pgfqpoint{3.236099in}{0.723497in}}%
\pgfpathlineto{\pgfqpoint{3.502561in}{2.073396in}}%
\pgfpathlineto{\pgfqpoint{3.236099in}{0.723497in}}%
\pgfpathclose%
\pgfusepath{fill}%
\end{pgfscope}%
\begin{pgfscope}%
\pgfpathrectangle{\pgfqpoint{0.017500in}{0.017500in}}{\pgfqpoint{6.218282in}{3.580372in}} %
\pgfusepath{clip}%
\pgfsetbuttcap%
\pgfsetroundjoin%
\definecolor{currentfill}{rgb}{0.243520,0.319189,0.771672}%
\pgfsetfillcolor{currentfill}%
\pgfsetlinewidth{0.000000pt}%
\definecolor{currentstroke}{rgb}{0.000000,0.000000,0.000000}%
\pgfsetstrokecolor{currentstroke}%
\pgfsetdash{}{0pt}%
\pgfpathmoveto{\pgfqpoint{3.236099in}{0.723497in}}%
\pgfpathlineto{\pgfqpoint{3.502561in}{2.073396in}}%
\pgfpathlineto{\pgfqpoint{3.236099in}{0.723497in}}%
\pgfpathclose%
\pgfusepath{fill}%
\end{pgfscope}%
\begin{pgfscope}%
\pgfpathrectangle{\pgfqpoint{0.017500in}{0.017500in}}{\pgfqpoint{6.218282in}{3.580372in}} %
\pgfusepath{clip}%
\pgfsetbuttcap%
\pgfsetroundjoin%
\definecolor{currentfill}{rgb}{0.243520,0.319189,0.771672}%
\pgfsetfillcolor{currentfill}%
\pgfsetlinewidth{0.000000pt}%
\definecolor{currentstroke}{rgb}{0.000000,0.000000,0.000000}%
\pgfsetstrokecolor{currentstroke}%
\pgfsetdash{}{0pt}%
\pgfpathmoveto{\pgfqpoint{3.236099in}{0.723497in}}%
\pgfpathlineto{\pgfqpoint{3.502561in}{2.073396in}}%
\pgfpathlineto{\pgfqpoint{3.236099in}{0.723497in}}%
\pgfpathclose%
\pgfusepath{fill}%
\end{pgfscope}%
\begin{pgfscope}%
\pgfpathrectangle{\pgfqpoint{0.017500in}{0.017500in}}{\pgfqpoint{6.218282in}{3.580372in}} %
\pgfusepath{clip}%
\pgfsetbuttcap%
\pgfsetroundjoin%
\definecolor{currentfill}{rgb}{0.243520,0.319189,0.771672}%
\pgfsetfillcolor{currentfill}%
\pgfsetlinewidth{0.000000pt}%
\definecolor{currentstroke}{rgb}{0.000000,0.000000,0.000000}%
\pgfsetstrokecolor{currentstroke}%
\pgfsetdash{}{0pt}%
\pgfpathmoveto{\pgfqpoint{3.236099in}{0.723497in}}%
\pgfpathlineto{\pgfqpoint{3.502561in}{2.073396in}}%
\pgfpathlineto{\pgfqpoint{3.236099in}{0.723497in}}%
\pgfpathclose%
\pgfusepath{fill}%
\end{pgfscope}%
\begin{pgfscope}%
\pgfpathrectangle{\pgfqpoint{0.017500in}{0.017500in}}{\pgfqpoint{6.218282in}{3.580372in}} %
\pgfusepath{clip}%
\pgfsetbuttcap%
\pgfsetroundjoin%
\definecolor{currentfill}{rgb}{0.243520,0.319189,0.771672}%
\pgfsetfillcolor{currentfill}%
\pgfsetlinewidth{0.000000pt}%
\definecolor{currentstroke}{rgb}{0.000000,0.000000,0.000000}%
\pgfsetstrokecolor{currentstroke}%
\pgfsetdash{}{0pt}%
\pgfpathmoveto{\pgfqpoint{3.236099in}{0.723497in}}%
\pgfpathlineto{\pgfqpoint{3.502561in}{2.073396in}}%
\pgfpathlineto{\pgfqpoint{3.236099in}{0.723497in}}%
\pgfpathclose%
\pgfusepath{fill}%
\end{pgfscope}%
\begin{pgfscope}%
\pgfpathrectangle{\pgfqpoint{0.017500in}{0.017500in}}{\pgfqpoint{6.218282in}{3.580372in}} %
\pgfusepath{clip}%
\pgfsetbuttcap%
\pgfsetroundjoin%
\definecolor{currentfill}{rgb}{0.243520,0.319189,0.771672}%
\pgfsetfillcolor{currentfill}%
\pgfsetlinewidth{0.000000pt}%
\definecolor{currentstroke}{rgb}{0.000000,0.000000,0.000000}%
\pgfsetstrokecolor{currentstroke}%
\pgfsetdash{}{0pt}%
\pgfpathmoveto{\pgfqpoint{3.236099in}{0.723497in}}%
\pgfpathlineto{\pgfqpoint{3.502561in}{2.073396in}}%
\pgfpathlineto{\pgfqpoint{3.236099in}{0.723497in}}%
\pgfpathclose%
\pgfusepath{fill}%
\end{pgfscope}%
\begin{pgfscope}%
\pgfpathrectangle{\pgfqpoint{0.017500in}{0.017500in}}{\pgfqpoint{6.218282in}{3.580372in}} %
\pgfusepath{clip}%
\pgfsetbuttcap%
\pgfsetroundjoin%
\definecolor{currentfill}{rgb}{0.243520,0.319189,0.771672}%
\pgfsetfillcolor{currentfill}%
\pgfsetlinewidth{0.000000pt}%
\definecolor{currentstroke}{rgb}{0.000000,0.000000,0.000000}%
\pgfsetstrokecolor{currentstroke}%
\pgfsetdash{}{0pt}%
\pgfpathmoveto{\pgfqpoint{3.236099in}{0.723497in}}%
\pgfpathlineto{\pgfqpoint{3.502561in}{2.073396in}}%
\pgfpathlineto{\pgfqpoint{3.236099in}{0.723497in}}%
\pgfpathclose%
\pgfusepath{fill}%
\end{pgfscope}%
\begin{pgfscope}%
\pgfpathrectangle{\pgfqpoint{0.017500in}{0.017500in}}{\pgfqpoint{6.218282in}{3.580372in}} %
\pgfusepath{clip}%
\pgfsetbuttcap%
\pgfsetroundjoin%
\definecolor{currentfill}{rgb}{0.243520,0.319189,0.771672}%
\pgfsetfillcolor{currentfill}%
\pgfsetlinewidth{0.000000pt}%
\definecolor{currentstroke}{rgb}{0.000000,0.000000,0.000000}%
\pgfsetstrokecolor{currentstroke}%
\pgfsetdash{}{0pt}%
\pgfpathmoveto{\pgfqpoint{3.236099in}{0.723497in}}%
\pgfpathlineto{\pgfqpoint{3.502561in}{2.073396in}}%
\pgfpathlineto{\pgfqpoint{3.236099in}{0.723497in}}%
\pgfpathclose%
\pgfusepath{fill}%
\end{pgfscope}%
\begin{pgfscope}%
\pgfpathrectangle{\pgfqpoint{0.017500in}{0.017500in}}{\pgfqpoint{6.218282in}{3.580372in}} %
\pgfusepath{clip}%
\pgfsetbuttcap%
\pgfsetroundjoin%
\definecolor{currentfill}{rgb}{0.243520,0.319189,0.771672}%
\pgfsetfillcolor{currentfill}%
\pgfsetlinewidth{0.000000pt}%
\definecolor{currentstroke}{rgb}{0.000000,0.000000,0.000000}%
\pgfsetstrokecolor{currentstroke}%
\pgfsetdash{}{0pt}%
\pgfpathmoveto{\pgfqpoint{3.236099in}{0.723497in}}%
\pgfpathlineto{\pgfqpoint{3.502561in}{2.073396in}}%
\pgfpathlineto{\pgfqpoint{3.236099in}{0.723497in}}%
\pgfpathclose%
\pgfusepath{fill}%
\end{pgfscope}%
\begin{pgfscope}%
\pgfpathrectangle{\pgfqpoint{0.017500in}{0.017500in}}{\pgfqpoint{6.218282in}{3.580372in}} %
\pgfusepath{clip}%
\pgfsetbuttcap%
\pgfsetroundjoin%
\definecolor{currentfill}{rgb}{0.243520,0.319189,0.771672}%
\pgfsetfillcolor{currentfill}%
\pgfsetlinewidth{0.000000pt}%
\definecolor{currentstroke}{rgb}{0.000000,0.000000,0.000000}%
\pgfsetstrokecolor{currentstroke}%
\pgfsetdash{}{0pt}%
\pgfpathmoveto{\pgfqpoint{3.236099in}{0.723497in}}%
\pgfpathlineto{\pgfqpoint{3.502561in}{2.073396in}}%
\pgfpathlineto{\pgfqpoint{3.236099in}{0.723497in}}%
\pgfpathclose%
\pgfusepath{fill}%
\end{pgfscope}%
\begin{pgfscope}%
\pgfpathrectangle{\pgfqpoint{0.017500in}{0.017500in}}{\pgfqpoint{6.218282in}{3.580372in}} %
\pgfusepath{clip}%
\pgfsetbuttcap%
\pgfsetroundjoin%
\definecolor{currentfill}{rgb}{0.243520,0.319189,0.771672}%
\pgfsetfillcolor{currentfill}%
\pgfsetlinewidth{0.000000pt}%
\definecolor{currentstroke}{rgb}{0.000000,0.000000,0.000000}%
\pgfsetstrokecolor{currentstroke}%
\pgfsetdash{}{0pt}%
\pgfpathmoveto{\pgfqpoint{3.236099in}{0.723497in}}%
\pgfpathlineto{\pgfqpoint{3.502561in}{2.073396in}}%
\pgfpathlineto{\pgfqpoint{3.236099in}{0.723497in}}%
\pgfpathclose%
\pgfusepath{fill}%
\end{pgfscope}%
\begin{pgfscope}%
\pgfpathrectangle{\pgfqpoint{0.017500in}{0.017500in}}{\pgfqpoint{6.218282in}{3.580372in}} %
\pgfusepath{clip}%
\pgfsetbuttcap%
\pgfsetroundjoin%
\definecolor{currentfill}{rgb}{0.763520,0.178667,0.193396}%
\pgfsetfillcolor{currentfill}%
\pgfsetlinewidth{0.000000pt}%
\definecolor{currentstroke}{rgb}{0.000000,0.000000,0.000000}%
\pgfsetstrokecolor{currentstroke}%
\pgfsetdash{}{0pt}%
\pgfpathmoveto{\pgfqpoint{4.478468in}{2.393972in}}%
\pgfpathlineto{\pgfqpoint{4.724651in}{2.549289in}}%
\pgfpathlineto{\pgfqpoint{4.478468in}{2.393972in}}%
\pgfpathclose%
\pgfusepath{fill}%
\end{pgfscope}%
\begin{pgfscope}%
\pgfpathrectangle{\pgfqpoint{0.017500in}{0.017500in}}{\pgfqpoint{6.218282in}{3.580372in}} %
\pgfusepath{clip}%
\pgfsetbuttcap%
\pgfsetroundjoin%
\definecolor{currentfill}{rgb}{0.763520,0.178667,0.193396}%
\pgfsetfillcolor{currentfill}%
\pgfsetlinewidth{0.000000pt}%
\definecolor{currentstroke}{rgb}{0.000000,0.000000,0.000000}%
\pgfsetstrokecolor{currentstroke}%
\pgfsetdash{}{0pt}%
\pgfpathmoveto{\pgfqpoint{4.478468in}{2.393972in}}%
\pgfpathlineto{\pgfqpoint{4.724651in}{2.549289in}}%
\pgfpathlineto{\pgfqpoint{4.478468in}{2.393972in}}%
\pgfpathclose%
\pgfusepath{fill}%
\end{pgfscope}%
\begin{pgfscope}%
\pgfpathrectangle{\pgfqpoint{0.017500in}{0.017500in}}{\pgfqpoint{6.218282in}{3.580372in}} %
\pgfusepath{clip}%
\pgfsetbuttcap%
\pgfsetroundjoin%
\definecolor{currentfill}{rgb}{0.763520,0.178667,0.193396}%
\pgfsetfillcolor{currentfill}%
\pgfsetlinewidth{0.000000pt}%
\definecolor{currentstroke}{rgb}{0.000000,0.000000,0.000000}%
\pgfsetstrokecolor{currentstroke}%
\pgfsetdash{}{0pt}%
\pgfpathmoveto{\pgfqpoint{4.478468in}{2.393972in}}%
\pgfpathlineto{\pgfqpoint{4.724651in}{2.549289in}}%
\pgfpathlineto{\pgfqpoint{4.478468in}{2.393972in}}%
\pgfpathclose%
\pgfusepath{fill}%
\end{pgfscope}%
\begin{pgfscope}%
\pgfpathrectangle{\pgfqpoint{0.017500in}{0.017500in}}{\pgfqpoint{6.218282in}{3.580372in}} %
\pgfusepath{clip}%
\pgfsetbuttcap%
\pgfsetroundjoin%
\definecolor{currentfill}{rgb}{0.763520,0.178667,0.193396}%
\pgfsetfillcolor{currentfill}%
\pgfsetlinewidth{0.000000pt}%
\definecolor{currentstroke}{rgb}{0.000000,0.000000,0.000000}%
\pgfsetstrokecolor{currentstroke}%
\pgfsetdash{}{0pt}%
\pgfpathmoveto{\pgfqpoint{4.478468in}{2.393972in}}%
\pgfpathlineto{\pgfqpoint{4.724651in}{2.549289in}}%
\pgfpathlineto{\pgfqpoint{4.478468in}{2.393972in}}%
\pgfpathclose%
\pgfusepath{fill}%
\end{pgfscope}%
\begin{pgfscope}%
\pgfpathrectangle{\pgfqpoint{0.017500in}{0.017500in}}{\pgfqpoint{6.218282in}{3.580372in}} %
\pgfusepath{clip}%
\pgfsetbuttcap%
\pgfsetroundjoin%
\definecolor{currentfill}{rgb}{0.763520,0.178667,0.193396}%
\pgfsetfillcolor{currentfill}%
\pgfsetlinewidth{0.000000pt}%
\definecolor{currentstroke}{rgb}{0.000000,0.000000,0.000000}%
\pgfsetstrokecolor{currentstroke}%
\pgfsetdash{}{0pt}%
\pgfpathmoveto{\pgfqpoint{4.478468in}{2.393972in}}%
\pgfpathlineto{\pgfqpoint{4.724651in}{2.549289in}}%
\pgfpathlineto{\pgfqpoint{4.478468in}{2.393972in}}%
\pgfpathclose%
\pgfusepath{fill}%
\end{pgfscope}%
\begin{pgfscope}%
\pgfpathrectangle{\pgfqpoint{0.017500in}{0.017500in}}{\pgfqpoint{6.218282in}{3.580372in}} %
\pgfusepath{clip}%
\pgfsetbuttcap%
\pgfsetroundjoin%
\definecolor{currentfill}{rgb}{0.763520,0.178667,0.193396}%
\pgfsetfillcolor{currentfill}%
\pgfsetlinewidth{0.000000pt}%
\definecolor{currentstroke}{rgb}{0.000000,0.000000,0.000000}%
\pgfsetstrokecolor{currentstroke}%
\pgfsetdash{}{0pt}%
\pgfpathmoveto{\pgfqpoint{4.478468in}{2.393972in}}%
\pgfpathlineto{\pgfqpoint{4.724651in}{2.549289in}}%
\pgfpathlineto{\pgfqpoint{4.478468in}{2.393972in}}%
\pgfpathclose%
\pgfusepath{fill}%
\end{pgfscope}%
\begin{pgfscope}%
\pgfpathrectangle{\pgfqpoint{0.017500in}{0.017500in}}{\pgfqpoint{6.218282in}{3.580372in}} %
\pgfusepath{clip}%
\pgfsetbuttcap%
\pgfsetroundjoin%
\definecolor{currentfill}{rgb}{0.763520,0.178667,0.193396}%
\pgfsetfillcolor{currentfill}%
\pgfsetlinewidth{0.000000pt}%
\definecolor{currentstroke}{rgb}{0.000000,0.000000,0.000000}%
\pgfsetstrokecolor{currentstroke}%
\pgfsetdash{}{0pt}%
\pgfpathmoveto{\pgfqpoint{4.478468in}{2.393972in}}%
\pgfpathlineto{\pgfqpoint{4.724651in}{2.549289in}}%
\pgfpathlineto{\pgfqpoint{4.478468in}{2.393972in}}%
\pgfpathclose%
\pgfusepath{fill}%
\end{pgfscope}%
\begin{pgfscope}%
\pgfpathrectangle{\pgfqpoint{0.017500in}{0.017500in}}{\pgfqpoint{6.218282in}{3.580372in}} %
\pgfusepath{clip}%
\pgfsetbuttcap%
\pgfsetroundjoin%
\definecolor{currentfill}{rgb}{0.763520,0.178667,0.193396}%
\pgfsetfillcolor{currentfill}%
\pgfsetlinewidth{0.000000pt}%
\definecolor{currentstroke}{rgb}{0.000000,0.000000,0.000000}%
\pgfsetstrokecolor{currentstroke}%
\pgfsetdash{}{0pt}%
\pgfpathmoveto{\pgfqpoint{4.478468in}{2.393972in}}%
\pgfpathlineto{\pgfqpoint{4.724651in}{2.549289in}}%
\pgfpathlineto{\pgfqpoint{4.478468in}{2.393972in}}%
\pgfpathclose%
\pgfusepath{fill}%
\end{pgfscope}%
\begin{pgfscope}%
\pgfpathrectangle{\pgfqpoint{0.017500in}{0.017500in}}{\pgfqpoint{6.218282in}{3.580372in}} %
\pgfusepath{clip}%
\pgfsetbuttcap%
\pgfsetroundjoin%
\definecolor{currentfill}{rgb}{0.763520,0.178667,0.193396}%
\pgfsetfillcolor{currentfill}%
\pgfsetlinewidth{0.000000pt}%
\definecolor{currentstroke}{rgb}{0.000000,0.000000,0.000000}%
\pgfsetstrokecolor{currentstroke}%
\pgfsetdash{}{0pt}%
\pgfpathmoveto{\pgfqpoint{4.478468in}{2.393972in}}%
\pgfpathlineto{\pgfqpoint{4.724651in}{2.549289in}}%
\pgfpathlineto{\pgfqpoint{4.478468in}{2.393972in}}%
\pgfpathclose%
\pgfusepath{fill}%
\end{pgfscope}%
\begin{pgfscope}%
\pgfpathrectangle{\pgfqpoint{0.017500in}{0.017500in}}{\pgfqpoint{6.218282in}{3.580372in}} %
\pgfusepath{clip}%
\pgfsetbuttcap%
\pgfsetroundjoin%
\definecolor{currentfill}{rgb}{0.763520,0.178667,0.193396}%
\pgfsetfillcolor{currentfill}%
\pgfsetlinewidth{0.000000pt}%
\definecolor{currentstroke}{rgb}{0.000000,0.000000,0.000000}%
\pgfsetstrokecolor{currentstroke}%
\pgfsetdash{}{0pt}%
\pgfpathmoveto{\pgfqpoint{4.478468in}{2.393972in}}%
\pgfpathlineto{\pgfqpoint{4.724651in}{2.549289in}}%
\pgfpathlineto{\pgfqpoint{4.478468in}{2.393972in}}%
\pgfpathclose%
\pgfusepath{fill}%
\end{pgfscope}%
\begin{pgfscope}%
\pgfpathrectangle{\pgfqpoint{0.017500in}{0.017500in}}{\pgfqpoint{6.218282in}{3.580372in}} %
\pgfusepath{clip}%
\pgfsetbuttcap%
\pgfsetroundjoin%
\definecolor{currentfill}{rgb}{0.763520,0.178667,0.193396}%
\pgfsetfillcolor{currentfill}%
\pgfsetlinewidth{0.000000pt}%
\definecolor{currentstroke}{rgb}{0.000000,0.000000,0.000000}%
\pgfsetstrokecolor{currentstroke}%
\pgfsetdash{}{0pt}%
\pgfpathmoveto{\pgfqpoint{4.478468in}{2.393972in}}%
\pgfpathlineto{\pgfqpoint{4.724651in}{2.549289in}}%
\pgfpathlineto{\pgfqpoint{4.478468in}{2.393972in}}%
\pgfpathclose%
\pgfusepath{fill}%
\end{pgfscope}%
\begin{pgfscope}%
\pgfpathrectangle{\pgfqpoint{0.017500in}{0.017500in}}{\pgfqpoint{6.218282in}{3.580372in}} %
\pgfusepath{clip}%
\pgfsetbuttcap%
\pgfsetroundjoin%
\definecolor{currentfill}{rgb}{0.763520,0.178667,0.193396}%
\pgfsetfillcolor{currentfill}%
\pgfsetlinewidth{0.000000pt}%
\definecolor{currentstroke}{rgb}{0.000000,0.000000,0.000000}%
\pgfsetstrokecolor{currentstroke}%
\pgfsetdash{}{0pt}%
\pgfpathmoveto{\pgfqpoint{4.478468in}{2.393972in}}%
\pgfpathlineto{\pgfqpoint{4.724651in}{2.549289in}}%
\pgfpathlineto{\pgfqpoint{4.478468in}{2.393972in}}%
\pgfpathclose%
\pgfusepath{fill}%
\end{pgfscope}%
\begin{pgfscope}%
\pgfpathrectangle{\pgfqpoint{0.017500in}{0.017500in}}{\pgfqpoint{6.218282in}{3.580372in}} %
\pgfusepath{clip}%
\pgfsetbuttcap%
\pgfsetroundjoin%
\definecolor{currentfill}{rgb}{0.763520,0.178667,0.193396}%
\pgfsetfillcolor{currentfill}%
\pgfsetlinewidth{0.000000pt}%
\definecolor{currentstroke}{rgb}{0.000000,0.000000,0.000000}%
\pgfsetstrokecolor{currentstroke}%
\pgfsetdash{}{0pt}%
\pgfpathmoveto{\pgfqpoint{4.478468in}{2.393972in}}%
\pgfpathlineto{\pgfqpoint{4.724651in}{2.549289in}}%
\pgfpathlineto{\pgfqpoint{4.478468in}{2.393972in}}%
\pgfpathclose%
\pgfusepath{fill}%
\end{pgfscope}%
\begin{pgfscope}%
\pgfpathrectangle{\pgfqpoint{0.017500in}{0.017500in}}{\pgfqpoint{6.218282in}{3.580372in}} %
\pgfusepath{clip}%
\pgfsetbuttcap%
\pgfsetroundjoin%
\definecolor{currentfill}{rgb}{0.763520,0.178667,0.193396}%
\pgfsetfillcolor{currentfill}%
\pgfsetlinewidth{0.000000pt}%
\definecolor{currentstroke}{rgb}{0.000000,0.000000,0.000000}%
\pgfsetstrokecolor{currentstroke}%
\pgfsetdash{}{0pt}%
\pgfpathmoveto{\pgfqpoint{4.478468in}{2.393972in}}%
\pgfpathlineto{\pgfqpoint{4.724651in}{2.549289in}}%
\pgfpathlineto{\pgfqpoint{4.478468in}{2.393972in}}%
\pgfpathclose%
\pgfusepath{fill}%
\end{pgfscope}%
\begin{pgfscope}%
\pgfpathrectangle{\pgfqpoint{0.017500in}{0.017500in}}{\pgfqpoint{6.218282in}{3.580372in}} %
\pgfusepath{clip}%
\pgfsetbuttcap%
\pgfsetroundjoin%
\definecolor{currentfill}{rgb}{0.763520,0.178667,0.193396}%
\pgfsetfillcolor{currentfill}%
\pgfsetlinewidth{0.000000pt}%
\definecolor{currentstroke}{rgb}{0.000000,0.000000,0.000000}%
\pgfsetstrokecolor{currentstroke}%
\pgfsetdash{}{0pt}%
\pgfpathmoveto{\pgfqpoint{4.478468in}{2.393972in}}%
\pgfpathlineto{\pgfqpoint{4.724651in}{2.549289in}}%
\pgfpathlineto{\pgfqpoint{4.478468in}{2.393972in}}%
\pgfpathclose%
\pgfusepath{fill}%
\end{pgfscope}%
\begin{pgfscope}%
\pgfpathrectangle{\pgfqpoint{0.017500in}{0.017500in}}{\pgfqpoint{6.218282in}{3.580372in}} %
\pgfusepath{clip}%
\pgfsetbuttcap%
\pgfsetroundjoin%
\definecolor{currentfill}{rgb}{0.763520,0.178667,0.193396}%
\pgfsetfillcolor{currentfill}%
\pgfsetlinewidth{0.000000pt}%
\definecolor{currentstroke}{rgb}{0.000000,0.000000,0.000000}%
\pgfsetstrokecolor{currentstroke}%
\pgfsetdash{}{0pt}%
\pgfpathmoveto{\pgfqpoint{4.478468in}{2.393972in}}%
\pgfpathlineto{\pgfqpoint{4.724651in}{2.549289in}}%
\pgfpathlineto{\pgfqpoint{4.478468in}{2.393972in}}%
\pgfpathclose%
\pgfusepath{fill}%
\end{pgfscope}%
\begin{pgfscope}%
\pgfpathrectangle{\pgfqpoint{0.017500in}{0.017500in}}{\pgfqpoint{6.218282in}{3.580372in}} %
\pgfusepath{clip}%
\pgfsetbuttcap%
\pgfsetroundjoin%
\definecolor{currentfill}{rgb}{0.763520,0.178667,0.193396}%
\pgfsetfillcolor{currentfill}%
\pgfsetlinewidth{0.000000pt}%
\definecolor{currentstroke}{rgb}{0.000000,0.000000,0.000000}%
\pgfsetstrokecolor{currentstroke}%
\pgfsetdash{}{0pt}%
\pgfpathmoveto{\pgfqpoint{4.478468in}{2.393972in}}%
\pgfpathlineto{\pgfqpoint{4.724651in}{2.549289in}}%
\pgfpathlineto{\pgfqpoint{4.478468in}{2.393972in}}%
\pgfpathclose%
\pgfusepath{fill}%
\end{pgfscope}%
\begin{pgfscope}%
\pgfpathrectangle{\pgfqpoint{0.017500in}{0.017500in}}{\pgfqpoint{6.218282in}{3.580372in}} %
\pgfusepath{clip}%
\pgfsetbuttcap%
\pgfsetroundjoin%
\definecolor{currentfill}{rgb}{0.763520,0.178667,0.193396}%
\pgfsetfillcolor{currentfill}%
\pgfsetlinewidth{0.000000pt}%
\definecolor{currentstroke}{rgb}{0.000000,0.000000,0.000000}%
\pgfsetstrokecolor{currentstroke}%
\pgfsetdash{}{0pt}%
\pgfpathmoveto{\pgfqpoint{4.478468in}{2.393972in}}%
\pgfpathlineto{\pgfqpoint{4.724651in}{2.549289in}}%
\pgfpathlineto{\pgfqpoint{4.478468in}{2.393972in}}%
\pgfpathclose%
\pgfusepath{fill}%
\end{pgfscope}%
\begin{pgfscope}%
\pgfpathrectangle{\pgfqpoint{0.017500in}{0.017500in}}{\pgfqpoint{6.218282in}{3.580372in}} %
\pgfusepath{clip}%
\pgfsetbuttcap%
\pgfsetroundjoin%
\definecolor{currentfill}{rgb}{0.763520,0.178667,0.193396}%
\pgfsetfillcolor{currentfill}%
\pgfsetlinewidth{0.000000pt}%
\definecolor{currentstroke}{rgb}{0.000000,0.000000,0.000000}%
\pgfsetstrokecolor{currentstroke}%
\pgfsetdash{}{0pt}%
\pgfpathmoveto{\pgfqpoint{4.478468in}{2.393972in}}%
\pgfpathlineto{\pgfqpoint{4.724651in}{2.549289in}}%
\pgfpathlineto{\pgfqpoint{4.478468in}{2.393972in}}%
\pgfpathclose%
\pgfusepath{fill}%
\end{pgfscope}%
\begin{pgfscope}%
\pgfpathrectangle{\pgfqpoint{0.017500in}{0.017500in}}{\pgfqpoint{6.218282in}{3.580372in}} %
\pgfusepath{clip}%
\pgfsetbuttcap%
\pgfsetroundjoin%
\definecolor{currentfill}{rgb}{0.763520,0.178667,0.193396}%
\pgfsetfillcolor{currentfill}%
\pgfsetlinewidth{0.000000pt}%
\definecolor{currentstroke}{rgb}{0.000000,0.000000,0.000000}%
\pgfsetstrokecolor{currentstroke}%
\pgfsetdash{}{0pt}%
\pgfpathmoveto{\pgfqpoint{4.478468in}{2.393972in}}%
\pgfpathlineto{\pgfqpoint{4.724651in}{2.549289in}}%
\pgfpathlineto{\pgfqpoint{4.478468in}{2.393972in}}%
\pgfpathclose%
\pgfusepath{fill}%
\end{pgfscope}%
\begin{pgfscope}%
\pgfpathrectangle{\pgfqpoint{0.017500in}{0.017500in}}{\pgfqpoint{6.218282in}{3.580372in}} %
\pgfusepath{clip}%
\pgfsetbuttcap%
\pgfsetroundjoin%
\definecolor{currentfill}{rgb}{0.763520,0.178667,0.193396}%
\pgfsetfillcolor{currentfill}%
\pgfsetlinewidth{0.000000pt}%
\definecolor{currentstroke}{rgb}{0.000000,0.000000,0.000000}%
\pgfsetstrokecolor{currentstroke}%
\pgfsetdash{}{0pt}%
\pgfpathmoveto{\pgfqpoint{4.478468in}{2.393972in}}%
\pgfpathlineto{\pgfqpoint{4.724651in}{2.549289in}}%
\pgfpathlineto{\pgfqpoint{4.478468in}{2.393972in}}%
\pgfpathclose%
\pgfusepath{fill}%
\end{pgfscope}%
\begin{pgfscope}%
\pgfpathrectangle{\pgfqpoint{0.017500in}{0.017500in}}{\pgfqpoint{6.218282in}{3.580372in}} %
\pgfusepath{clip}%
\pgfsetbuttcap%
\pgfsetroundjoin%
\definecolor{currentfill}{rgb}{0.763520,0.178667,0.193396}%
\pgfsetfillcolor{currentfill}%
\pgfsetlinewidth{0.000000pt}%
\definecolor{currentstroke}{rgb}{0.000000,0.000000,0.000000}%
\pgfsetstrokecolor{currentstroke}%
\pgfsetdash{}{0pt}%
\pgfpathmoveto{\pgfqpoint{4.478468in}{2.393972in}}%
\pgfpathlineto{\pgfqpoint{4.724651in}{2.549289in}}%
\pgfpathlineto{\pgfqpoint{4.478468in}{2.393972in}}%
\pgfpathclose%
\pgfusepath{fill}%
\end{pgfscope}%
\begin{pgfscope}%
\pgfpathrectangle{\pgfqpoint{0.017500in}{0.017500in}}{\pgfqpoint{6.218282in}{3.580372in}} %
\pgfusepath{clip}%
\pgfsetbuttcap%
\pgfsetroundjoin%
\definecolor{currentfill}{rgb}{0.763520,0.178667,0.193396}%
\pgfsetfillcolor{currentfill}%
\pgfsetlinewidth{0.000000pt}%
\definecolor{currentstroke}{rgb}{0.000000,0.000000,0.000000}%
\pgfsetstrokecolor{currentstroke}%
\pgfsetdash{}{0pt}%
\pgfpathmoveto{\pgfqpoint{4.478468in}{2.393972in}}%
\pgfpathlineto{\pgfqpoint{4.724651in}{2.549289in}}%
\pgfpathlineto{\pgfqpoint{4.478468in}{2.393972in}}%
\pgfpathclose%
\pgfusepath{fill}%
\end{pgfscope}%
\begin{pgfscope}%
\pgfpathrectangle{\pgfqpoint{0.017500in}{0.017500in}}{\pgfqpoint{6.218282in}{3.580372in}} %
\pgfusepath{clip}%
\pgfsetbuttcap%
\pgfsetroundjoin%
\definecolor{currentfill}{rgb}{0.763520,0.178667,0.193396}%
\pgfsetfillcolor{currentfill}%
\pgfsetlinewidth{0.000000pt}%
\definecolor{currentstroke}{rgb}{0.000000,0.000000,0.000000}%
\pgfsetstrokecolor{currentstroke}%
\pgfsetdash{}{0pt}%
\pgfpathmoveto{\pgfqpoint{4.478468in}{2.393972in}}%
\pgfpathlineto{\pgfqpoint{4.724651in}{2.549289in}}%
\pgfpathlineto{\pgfqpoint{4.478468in}{2.393972in}}%
\pgfpathclose%
\pgfusepath{fill}%
\end{pgfscope}%
\begin{pgfscope}%
\pgfpathrectangle{\pgfqpoint{0.017500in}{0.017500in}}{\pgfqpoint{6.218282in}{3.580372in}} %
\pgfusepath{clip}%
\pgfsetbuttcap%
\pgfsetroundjoin%
\definecolor{currentfill}{rgb}{0.763520,0.178667,0.193396}%
\pgfsetfillcolor{currentfill}%
\pgfsetlinewidth{0.000000pt}%
\definecolor{currentstroke}{rgb}{0.000000,0.000000,0.000000}%
\pgfsetstrokecolor{currentstroke}%
\pgfsetdash{}{0pt}%
\pgfpathmoveto{\pgfqpoint{4.478468in}{2.393972in}}%
\pgfpathlineto{\pgfqpoint{4.724651in}{2.549289in}}%
\pgfpathlineto{\pgfqpoint{4.478468in}{2.393972in}}%
\pgfpathclose%
\pgfusepath{fill}%
\end{pgfscope}%
\begin{pgfscope}%
\pgfpathrectangle{\pgfqpoint{0.017500in}{0.017500in}}{\pgfqpoint{6.218282in}{3.580372in}} %
\pgfusepath{clip}%
\pgfsetbuttcap%
\pgfsetroundjoin%
\definecolor{currentfill}{rgb}{0.763520,0.178667,0.193396}%
\pgfsetfillcolor{currentfill}%
\pgfsetlinewidth{0.000000pt}%
\definecolor{currentstroke}{rgb}{0.000000,0.000000,0.000000}%
\pgfsetstrokecolor{currentstroke}%
\pgfsetdash{}{0pt}%
\pgfpathmoveto{\pgfqpoint{4.478468in}{2.393972in}}%
\pgfpathlineto{\pgfqpoint{4.724651in}{2.549289in}}%
\pgfpathlineto{\pgfqpoint{4.478468in}{2.393972in}}%
\pgfpathclose%
\pgfusepath{fill}%
\end{pgfscope}%
\begin{pgfscope}%
\pgfpathrectangle{\pgfqpoint{0.017500in}{0.017500in}}{\pgfqpoint{6.218282in}{3.580372in}} %
\pgfusepath{clip}%
\pgfsetbuttcap%
\pgfsetroundjoin%
\definecolor{currentfill}{rgb}{0.763520,0.178667,0.193396}%
\pgfsetfillcolor{currentfill}%
\pgfsetlinewidth{0.000000pt}%
\definecolor{currentstroke}{rgb}{0.000000,0.000000,0.000000}%
\pgfsetstrokecolor{currentstroke}%
\pgfsetdash{}{0pt}%
\pgfpathmoveto{\pgfqpoint{4.478468in}{2.393972in}}%
\pgfpathlineto{\pgfqpoint{4.724651in}{2.549289in}}%
\pgfpathlineto{\pgfqpoint{4.478468in}{2.393972in}}%
\pgfpathclose%
\pgfusepath{fill}%
\end{pgfscope}%
\begin{pgfscope}%
\pgfpathrectangle{\pgfqpoint{0.017500in}{0.017500in}}{\pgfqpoint{6.218282in}{3.580372in}} %
\pgfusepath{clip}%
\pgfsetbuttcap%
\pgfsetroundjoin%
\definecolor{currentfill}{rgb}{0.763520,0.178667,0.193396}%
\pgfsetfillcolor{currentfill}%
\pgfsetlinewidth{0.000000pt}%
\definecolor{currentstroke}{rgb}{0.000000,0.000000,0.000000}%
\pgfsetstrokecolor{currentstroke}%
\pgfsetdash{}{0pt}%
\pgfpathmoveto{\pgfqpoint{4.478468in}{2.393972in}}%
\pgfpathlineto{\pgfqpoint{4.724651in}{2.549289in}}%
\pgfpathlineto{\pgfqpoint{4.478468in}{2.393972in}}%
\pgfpathclose%
\pgfusepath{fill}%
\end{pgfscope}%
\begin{pgfscope}%
\pgfpathrectangle{\pgfqpoint{0.017500in}{0.017500in}}{\pgfqpoint{6.218282in}{3.580372in}} %
\pgfusepath{clip}%
\pgfsetbuttcap%
\pgfsetroundjoin%
\definecolor{currentfill}{rgb}{0.763520,0.178667,0.193396}%
\pgfsetfillcolor{currentfill}%
\pgfsetlinewidth{0.000000pt}%
\definecolor{currentstroke}{rgb}{0.000000,0.000000,0.000000}%
\pgfsetstrokecolor{currentstroke}%
\pgfsetdash{}{0pt}%
\pgfpathmoveto{\pgfqpoint{4.478468in}{2.393972in}}%
\pgfpathlineto{\pgfqpoint{4.724651in}{2.549289in}}%
\pgfpathlineto{\pgfqpoint{4.478468in}{2.393972in}}%
\pgfpathclose%
\pgfusepath{fill}%
\end{pgfscope}%
\begin{pgfscope}%
\pgfpathrectangle{\pgfqpoint{0.017500in}{0.017500in}}{\pgfqpoint{6.218282in}{3.580372in}} %
\pgfusepath{clip}%
\pgfsetbuttcap%
\pgfsetroundjoin%
\definecolor{currentfill}{rgb}{0.763520,0.178667,0.193396}%
\pgfsetfillcolor{currentfill}%
\pgfsetlinewidth{0.000000pt}%
\definecolor{currentstroke}{rgb}{0.000000,0.000000,0.000000}%
\pgfsetstrokecolor{currentstroke}%
\pgfsetdash{}{0pt}%
\pgfpathmoveto{\pgfqpoint{4.478468in}{2.393972in}}%
\pgfpathlineto{\pgfqpoint{4.724651in}{2.549289in}}%
\pgfpathlineto{\pgfqpoint{4.478468in}{2.393972in}}%
\pgfpathclose%
\pgfusepath{fill}%
\end{pgfscope}%
\begin{pgfscope}%
\pgfpathrectangle{\pgfqpoint{0.017500in}{0.017500in}}{\pgfqpoint{6.218282in}{3.580372in}} %
\pgfusepath{clip}%
\pgfsetbuttcap%
\pgfsetroundjoin%
\definecolor{currentfill}{rgb}{0.888390,0.417703,0.327898}%
\pgfsetfillcolor{currentfill}%
\pgfsetlinewidth{0.000000pt}%
\definecolor{currentstroke}{rgb}{0.000000,0.000000,0.000000}%
\pgfsetstrokecolor{currentstroke}%
\pgfsetdash{}{0pt}%
\pgfpathmoveto{\pgfqpoint{2.590432in}{2.210709in}}%
\pgfpathlineto{\pgfqpoint{3.043326in}{2.141782in}}%
\pgfpathlineto{\pgfqpoint{3.306485in}{2.382499in}}%
\pgfpathlineto{\pgfqpoint{2.855513in}{2.451418in}}%
\pgfpathlineto{\pgfqpoint{2.590432in}{2.210709in}}%
\pgfpathclose%
\pgfusepath{fill}%
\end{pgfscope}%
\begin{pgfscope}%
\pgfpathrectangle{\pgfqpoint{0.017500in}{0.017500in}}{\pgfqpoint{6.218282in}{3.580372in}} %
\pgfusepath{clip}%
\pgfsetbuttcap%
\pgfsetroundjoin%
\definecolor{currentfill}{rgb}{0.888390,0.417703,0.327898}%
\pgfsetfillcolor{currentfill}%
\pgfsetlinewidth{0.000000pt}%
\definecolor{currentstroke}{rgb}{0.000000,0.000000,0.000000}%
\pgfsetstrokecolor{currentstroke}%
\pgfsetdash{}{0pt}%
\pgfpathmoveto{\pgfqpoint{2.590432in}{2.210709in}}%
\pgfpathlineto{\pgfqpoint{3.043326in}{2.141782in}}%
\pgfpathlineto{\pgfqpoint{3.306485in}{2.382499in}}%
\pgfpathlineto{\pgfqpoint{2.855513in}{2.451418in}}%
\pgfpathlineto{\pgfqpoint{2.590432in}{2.210709in}}%
\pgfpathclose%
\pgfusepath{fill}%
\end{pgfscope}%
\begin{pgfscope}%
\pgfpathrectangle{\pgfqpoint{0.017500in}{0.017500in}}{\pgfqpoint{6.218282in}{3.580372in}} %
\pgfusepath{clip}%
\pgfsetbuttcap%
\pgfsetroundjoin%
\definecolor{currentfill}{rgb}{0.888390,0.417703,0.327898}%
\pgfsetfillcolor{currentfill}%
\pgfsetlinewidth{0.000000pt}%
\definecolor{currentstroke}{rgb}{0.000000,0.000000,0.000000}%
\pgfsetstrokecolor{currentstroke}%
\pgfsetdash{}{0pt}%
\pgfpathmoveto{\pgfqpoint{2.590432in}{2.210709in}}%
\pgfpathlineto{\pgfqpoint{3.043326in}{2.141782in}}%
\pgfpathlineto{\pgfqpoint{3.306485in}{2.382499in}}%
\pgfpathlineto{\pgfqpoint{2.855513in}{2.451418in}}%
\pgfpathlineto{\pgfqpoint{2.590432in}{2.210709in}}%
\pgfpathclose%
\pgfusepath{fill}%
\end{pgfscope}%
\begin{pgfscope}%
\pgfpathrectangle{\pgfqpoint{0.017500in}{0.017500in}}{\pgfqpoint{6.218282in}{3.580372in}} %
\pgfusepath{clip}%
\pgfsetbuttcap%
\pgfsetroundjoin%
\definecolor{currentfill}{rgb}{0.888390,0.417703,0.327898}%
\pgfsetfillcolor{currentfill}%
\pgfsetlinewidth{0.000000pt}%
\definecolor{currentstroke}{rgb}{0.000000,0.000000,0.000000}%
\pgfsetstrokecolor{currentstroke}%
\pgfsetdash{}{0pt}%
\pgfpathmoveto{\pgfqpoint{2.590432in}{2.210709in}}%
\pgfpathlineto{\pgfqpoint{3.043326in}{2.141782in}}%
\pgfpathlineto{\pgfqpoint{3.306485in}{2.382499in}}%
\pgfpathlineto{\pgfqpoint{2.855513in}{2.451418in}}%
\pgfpathlineto{\pgfqpoint{2.590432in}{2.210709in}}%
\pgfpathclose%
\pgfusepath{fill}%
\end{pgfscope}%
\begin{pgfscope}%
\pgfpathrectangle{\pgfqpoint{0.017500in}{0.017500in}}{\pgfqpoint{6.218282in}{3.580372in}} %
\pgfusepath{clip}%
\pgfsetbuttcap%
\pgfsetroundjoin%
\definecolor{currentfill}{rgb}{0.888390,0.417703,0.327898}%
\pgfsetfillcolor{currentfill}%
\pgfsetlinewidth{0.000000pt}%
\definecolor{currentstroke}{rgb}{0.000000,0.000000,0.000000}%
\pgfsetstrokecolor{currentstroke}%
\pgfsetdash{}{0pt}%
\pgfpathmoveto{\pgfqpoint{2.590432in}{2.210709in}}%
\pgfpathlineto{\pgfqpoint{3.043326in}{2.141782in}}%
\pgfpathlineto{\pgfqpoint{3.306485in}{2.382499in}}%
\pgfpathlineto{\pgfqpoint{2.855513in}{2.451418in}}%
\pgfpathlineto{\pgfqpoint{2.590432in}{2.210709in}}%
\pgfpathclose%
\pgfusepath{fill}%
\end{pgfscope}%
\begin{pgfscope}%
\pgfpathrectangle{\pgfqpoint{0.017500in}{0.017500in}}{\pgfqpoint{6.218282in}{3.580372in}} %
\pgfusepath{clip}%
\pgfsetbuttcap%
\pgfsetroundjoin%
\definecolor{currentfill}{rgb}{0.888390,0.417703,0.327898}%
\pgfsetfillcolor{currentfill}%
\pgfsetlinewidth{0.000000pt}%
\definecolor{currentstroke}{rgb}{0.000000,0.000000,0.000000}%
\pgfsetstrokecolor{currentstroke}%
\pgfsetdash{}{0pt}%
\pgfpathmoveto{\pgfqpoint{2.590432in}{2.210709in}}%
\pgfpathlineto{\pgfqpoint{3.043326in}{2.141782in}}%
\pgfpathlineto{\pgfqpoint{3.306485in}{2.382499in}}%
\pgfpathlineto{\pgfqpoint{2.855513in}{2.451418in}}%
\pgfpathlineto{\pgfqpoint{2.590432in}{2.210709in}}%
\pgfpathclose%
\pgfusepath{fill}%
\end{pgfscope}%
\begin{pgfscope}%
\pgfpathrectangle{\pgfqpoint{0.017500in}{0.017500in}}{\pgfqpoint{6.218282in}{3.580372in}} %
\pgfusepath{clip}%
\pgfsetbuttcap%
\pgfsetroundjoin%
\definecolor{currentfill}{rgb}{0.810616,0.268797,0.235428}%
\pgfsetfillcolor{currentfill}%
\pgfsetlinewidth{0.000000pt}%
\definecolor{currentstroke}{rgb}{0.000000,0.000000,0.000000}%
\pgfsetstrokecolor{currentstroke}%
\pgfsetdash{}{0pt}%
\pgfpathmoveto{\pgfqpoint{3.763398in}{2.297457in}}%
\pgfpathlineto{\pgfqpoint{4.018447in}{2.468167in}}%
\pgfpathlineto{\pgfqpoint{3.763398in}{2.297457in}}%
\pgfpathclose%
\pgfusepath{fill}%
\end{pgfscope}%
\begin{pgfscope}%
\pgfpathrectangle{\pgfqpoint{0.017500in}{0.017500in}}{\pgfqpoint{6.218282in}{3.580372in}} %
\pgfusepath{clip}%
\pgfsetbuttcap%
\pgfsetroundjoin%
\definecolor{currentfill}{rgb}{0.810616,0.268797,0.235428}%
\pgfsetfillcolor{currentfill}%
\pgfsetlinewidth{0.000000pt}%
\definecolor{currentstroke}{rgb}{0.000000,0.000000,0.000000}%
\pgfsetstrokecolor{currentstroke}%
\pgfsetdash{}{0pt}%
\pgfpathmoveto{\pgfqpoint{3.763398in}{2.297457in}}%
\pgfpathlineto{\pgfqpoint{4.018447in}{2.468167in}}%
\pgfpathlineto{\pgfqpoint{3.763398in}{2.297457in}}%
\pgfpathclose%
\pgfusepath{fill}%
\end{pgfscope}%
\begin{pgfscope}%
\pgfpathrectangle{\pgfqpoint{0.017500in}{0.017500in}}{\pgfqpoint{6.218282in}{3.580372in}} %
\pgfusepath{clip}%
\pgfsetbuttcap%
\pgfsetroundjoin%
\definecolor{currentfill}{rgb}{0.810616,0.268797,0.235428}%
\pgfsetfillcolor{currentfill}%
\pgfsetlinewidth{0.000000pt}%
\definecolor{currentstroke}{rgb}{0.000000,0.000000,0.000000}%
\pgfsetstrokecolor{currentstroke}%
\pgfsetdash{}{0pt}%
\pgfpathmoveto{\pgfqpoint{3.763398in}{2.297457in}}%
\pgfpathlineto{\pgfqpoint{4.018447in}{2.468167in}}%
\pgfpathlineto{\pgfqpoint{3.763398in}{2.297457in}}%
\pgfpathclose%
\pgfusepath{fill}%
\end{pgfscope}%
\begin{pgfscope}%
\pgfpathrectangle{\pgfqpoint{0.017500in}{0.017500in}}{\pgfqpoint{6.218282in}{3.580372in}} %
\pgfusepath{clip}%
\pgfsetbuttcap%
\pgfsetroundjoin%
\definecolor{currentfill}{rgb}{0.810616,0.268797,0.235428}%
\pgfsetfillcolor{currentfill}%
\pgfsetlinewidth{0.000000pt}%
\definecolor{currentstroke}{rgb}{0.000000,0.000000,0.000000}%
\pgfsetstrokecolor{currentstroke}%
\pgfsetdash{}{0pt}%
\pgfpathmoveto{\pgfqpoint{3.763398in}{2.297457in}}%
\pgfpathlineto{\pgfqpoint{4.018447in}{2.468167in}}%
\pgfpathlineto{\pgfqpoint{3.763398in}{2.297457in}}%
\pgfpathclose%
\pgfusepath{fill}%
\end{pgfscope}%
\begin{pgfscope}%
\pgfpathrectangle{\pgfqpoint{0.017500in}{0.017500in}}{\pgfqpoint{6.218282in}{3.580372in}} %
\pgfusepath{clip}%
\pgfsetbuttcap%
\pgfsetroundjoin%
\definecolor{currentfill}{rgb}{0.810616,0.268797,0.235428}%
\pgfsetfillcolor{currentfill}%
\pgfsetlinewidth{0.000000pt}%
\definecolor{currentstroke}{rgb}{0.000000,0.000000,0.000000}%
\pgfsetstrokecolor{currentstroke}%
\pgfsetdash{}{0pt}%
\pgfpathmoveto{\pgfqpoint{3.763398in}{2.297457in}}%
\pgfpathlineto{\pgfqpoint{4.018447in}{2.468167in}}%
\pgfpathlineto{\pgfqpoint{3.763398in}{2.297457in}}%
\pgfpathclose%
\pgfusepath{fill}%
\end{pgfscope}%
\begin{pgfscope}%
\pgfpathrectangle{\pgfqpoint{0.017500in}{0.017500in}}{\pgfqpoint{6.218282in}{3.580372in}} %
\pgfusepath{clip}%
\pgfsetbuttcap%
\pgfsetroundjoin%
\definecolor{currentfill}{rgb}{0.810616,0.268797,0.235428}%
\pgfsetfillcolor{currentfill}%
\pgfsetlinewidth{0.000000pt}%
\definecolor{currentstroke}{rgb}{0.000000,0.000000,0.000000}%
\pgfsetstrokecolor{currentstroke}%
\pgfsetdash{}{0pt}%
\pgfpathmoveto{\pgfqpoint{3.763398in}{2.297457in}}%
\pgfpathlineto{\pgfqpoint{4.018447in}{2.468167in}}%
\pgfpathlineto{\pgfqpoint{3.763398in}{2.297457in}}%
\pgfpathclose%
\pgfusepath{fill}%
\end{pgfscope}%
\begin{pgfscope}%
\pgfpathrectangle{\pgfqpoint{0.017500in}{0.017500in}}{\pgfqpoint{6.218282in}{3.580372in}} %
\pgfusepath{clip}%
\pgfsetbuttcap%
\pgfsetroundjoin%
\definecolor{currentfill}{rgb}{0.810616,0.268797,0.235428}%
\pgfsetfillcolor{currentfill}%
\pgfsetlinewidth{0.000000pt}%
\definecolor{currentstroke}{rgb}{0.000000,0.000000,0.000000}%
\pgfsetstrokecolor{currentstroke}%
\pgfsetdash{}{0pt}%
\pgfpathmoveto{\pgfqpoint{3.763398in}{2.297457in}}%
\pgfpathlineto{\pgfqpoint{4.018447in}{2.468167in}}%
\pgfpathlineto{\pgfqpoint{3.763398in}{2.297457in}}%
\pgfpathclose%
\pgfusepath{fill}%
\end{pgfscope}%
\begin{pgfscope}%
\pgfpathrectangle{\pgfqpoint{0.017500in}{0.017500in}}{\pgfqpoint{6.218282in}{3.580372in}} %
\pgfusepath{clip}%
\pgfsetbuttcap%
\pgfsetroundjoin%
\definecolor{currentfill}{rgb}{0.810616,0.268797,0.235428}%
\pgfsetfillcolor{currentfill}%
\pgfsetlinewidth{0.000000pt}%
\definecolor{currentstroke}{rgb}{0.000000,0.000000,0.000000}%
\pgfsetstrokecolor{currentstroke}%
\pgfsetdash{}{0pt}%
\pgfpathmoveto{\pgfqpoint{3.763398in}{2.297457in}}%
\pgfpathlineto{\pgfqpoint{4.018447in}{2.468167in}}%
\pgfpathlineto{\pgfqpoint{3.763398in}{2.297457in}}%
\pgfpathclose%
\pgfusepath{fill}%
\end{pgfscope}%
\begin{pgfscope}%
\pgfpathrectangle{\pgfqpoint{0.017500in}{0.017500in}}{\pgfqpoint{6.218282in}{3.580372in}} %
\pgfusepath{clip}%
\pgfsetbuttcap%
\pgfsetroundjoin%
\definecolor{currentfill}{rgb}{0.810616,0.268797,0.235428}%
\pgfsetfillcolor{currentfill}%
\pgfsetlinewidth{0.000000pt}%
\definecolor{currentstroke}{rgb}{0.000000,0.000000,0.000000}%
\pgfsetstrokecolor{currentstroke}%
\pgfsetdash{}{0pt}%
\pgfpathmoveto{\pgfqpoint{3.763398in}{2.297457in}}%
\pgfpathlineto{\pgfqpoint{4.018447in}{2.468167in}}%
\pgfpathlineto{\pgfqpoint{3.763398in}{2.297457in}}%
\pgfpathclose%
\pgfusepath{fill}%
\end{pgfscope}%
\begin{pgfscope}%
\pgfpathrectangle{\pgfqpoint{0.017500in}{0.017500in}}{\pgfqpoint{6.218282in}{3.580372in}} %
\pgfusepath{clip}%
\pgfsetbuttcap%
\pgfsetroundjoin%
\definecolor{currentfill}{rgb}{0.810616,0.268797,0.235428}%
\pgfsetfillcolor{currentfill}%
\pgfsetlinewidth{0.000000pt}%
\definecolor{currentstroke}{rgb}{0.000000,0.000000,0.000000}%
\pgfsetstrokecolor{currentstroke}%
\pgfsetdash{}{0pt}%
\pgfpathmoveto{\pgfqpoint{3.763398in}{2.297457in}}%
\pgfpathlineto{\pgfqpoint{4.018447in}{2.468167in}}%
\pgfpathlineto{\pgfqpoint{3.763398in}{2.297457in}}%
\pgfpathclose%
\pgfusepath{fill}%
\end{pgfscope}%
\begin{pgfscope}%
\pgfpathrectangle{\pgfqpoint{0.017500in}{0.017500in}}{\pgfqpoint{6.218282in}{3.580372in}} %
\pgfusepath{clip}%
\pgfsetbuttcap%
\pgfsetroundjoin%
\definecolor{currentfill}{rgb}{0.810616,0.268797,0.235428}%
\pgfsetfillcolor{currentfill}%
\pgfsetlinewidth{0.000000pt}%
\definecolor{currentstroke}{rgb}{0.000000,0.000000,0.000000}%
\pgfsetstrokecolor{currentstroke}%
\pgfsetdash{}{0pt}%
\pgfpathmoveto{\pgfqpoint{3.763398in}{2.297457in}}%
\pgfpathlineto{\pgfqpoint{4.018447in}{2.468167in}}%
\pgfpathlineto{\pgfqpoint{3.763398in}{2.297457in}}%
\pgfpathclose%
\pgfusepath{fill}%
\end{pgfscope}%
\begin{pgfscope}%
\pgfpathrectangle{\pgfqpoint{0.017500in}{0.017500in}}{\pgfqpoint{6.218282in}{3.580372in}} %
\pgfusepath{clip}%
\pgfsetbuttcap%
\pgfsetroundjoin%
\definecolor{currentfill}{rgb}{0.810616,0.268797,0.235428}%
\pgfsetfillcolor{currentfill}%
\pgfsetlinewidth{0.000000pt}%
\definecolor{currentstroke}{rgb}{0.000000,0.000000,0.000000}%
\pgfsetstrokecolor{currentstroke}%
\pgfsetdash{}{0pt}%
\pgfpathmoveto{\pgfqpoint{3.763398in}{2.297457in}}%
\pgfpathlineto{\pgfqpoint{4.018447in}{2.468167in}}%
\pgfpathlineto{\pgfqpoint{3.763398in}{2.297457in}}%
\pgfpathclose%
\pgfusepath{fill}%
\end{pgfscope}%
\begin{pgfscope}%
\pgfpathrectangle{\pgfqpoint{0.017500in}{0.017500in}}{\pgfqpoint{6.218282in}{3.580372in}} %
\pgfusepath{clip}%
\pgfsetbuttcap%
\pgfsetroundjoin%
\definecolor{currentfill}{rgb}{0.810616,0.268797,0.235428}%
\pgfsetfillcolor{currentfill}%
\pgfsetlinewidth{0.000000pt}%
\definecolor{currentstroke}{rgb}{0.000000,0.000000,0.000000}%
\pgfsetstrokecolor{currentstroke}%
\pgfsetdash{}{0pt}%
\pgfpathmoveto{\pgfqpoint{3.763398in}{2.297457in}}%
\pgfpathlineto{\pgfqpoint{4.018447in}{2.468167in}}%
\pgfpathlineto{\pgfqpoint{3.763398in}{2.297457in}}%
\pgfpathclose%
\pgfusepath{fill}%
\end{pgfscope}%
\begin{pgfscope}%
\pgfpathrectangle{\pgfqpoint{0.017500in}{0.017500in}}{\pgfqpoint{6.218282in}{3.580372in}} %
\pgfusepath{clip}%
\pgfsetbuttcap%
\pgfsetroundjoin%
\definecolor{currentfill}{rgb}{0.810616,0.268797,0.235428}%
\pgfsetfillcolor{currentfill}%
\pgfsetlinewidth{0.000000pt}%
\definecolor{currentstroke}{rgb}{0.000000,0.000000,0.000000}%
\pgfsetstrokecolor{currentstroke}%
\pgfsetdash{}{0pt}%
\pgfpathmoveto{\pgfqpoint{3.763398in}{2.297457in}}%
\pgfpathlineto{\pgfqpoint{4.018447in}{2.468167in}}%
\pgfpathlineto{\pgfqpoint{3.763398in}{2.297457in}}%
\pgfpathclose%
\pgfusepath{fill}%
\end{pgfscope}%
\begin{pgfscope}%
\pgfpathrectangle{\pgfqpoint{0.017500in}{0.017500in}}{\pgfqpoint{6.218282in}{3.580372in}} %
\pgfusepath{clip}%
\pgfsetbuttcap%
\pgfsetroundjoin%
\definecolor{currentfill}{rgb}{0.810616,0.268797,0.235428}%
\pgfsetfillcolor{currentfill}%
\pgfsetlinewidth{0.000000pt}%
\definecolor{currentstroke}{rgb}{0.000000,0.000000,0.000000}%
\pgfsetstrokecolor{currentstroke}%
\pgfsetdash{}{0pt}%
\pgfpathmoveto{\pgfqpoint{3.763398in}{2.297457in}}%
\pgfpathlineto{\pgfqpoint{4.018447in}{2.468167in}}%
\pgfpathlineto{\pgfqpoint{3.763398in}{2.297457in}}%
\pgfpathclose%
\pgfusepath{fill}%
\end{pgfscope}%
\begin{pgfscope}%
\pgfpathrectangle{\pgfqpoint{0.017500in}{0.017500in}}{\pgfqpoint{6.218282in}{3.580372in}} %
\pgfusepath{clip}%
\pgfsetbuttcap%
\pgfsetroundjoin%
\definecolor{currentfill}{rgb}{0.810616,0.268797,0.235428}%
\pgfsetfillcolor{currentfill}%
\pgfsetlinewidth{0.000000pt}%
\definecolor{currentstroke}{rgb}{0.000000,0.000000,0.000000}%
\pgfsetstrokecolor{currentstroke}%
\pgfsetdash{}{0pt}%
\pgfpathmoveto{\pgfqpoint{3.763398in}{2.297457in}}%
\pgfpathlineto{\pgfqpoint{4.018447in}{2.468167in}}%
\pgfpathlineto{\pgfqpoint{3.763398in}{2.297457in}}%
\pgfpathclose%
\pgfusepath{fill}%
\end{pgfscope}%
\begin{pgfscope}%
\pgfpathrectangle{\pgfqpoint{0.017500in}{0.017500in}}{\pgfqpoint{6.218282in}{3.580372in}} %
\pgfusepath{clip}%
\pgfsetbuttcap%
\pgfsetroundjoin%
\definecolor{currentfill}{rgb}{0.810616,0.268797,0.235428}%
\pgfsetfillcolor{currentfill}%
\pgfsetlinewidth{0.000000pt}%
\definecolor{currentstroke}{rgb}{0.000000,0.000000,0.000000}%
\pgfsetstrokecolor{currentstroke}%
\pgfsetdash{}{0pt}%
\pgfpathmoveto{\pgfqpoint{3.763398in}{2.297457in}}%
\pgfpathlineto{\pgfqpoint{4.018447in}{2.468167in}}%
\pgfpathlineto{\pgfqpoint{3.763398in}{2.297457in}}%
\pgfpathclose%
\pgfusepath{fill}%
\end{pgfscope}%
\begin{pgfscope}%
\pgfpathrectangle{\pgfqpoint{0.017500in}{0.017500in}}{\pgfqpoint{6.218282in}{3.580372in}} %
\pgfusepath{clip}%
\pgfsetbuttcap%
\pgfsetroundjoin%
\definecolor{currentfill}{rgb}{0.810616,0.268797,0.235428}%
\pgfsetfillcolor{currentfill}%
\pgfsetlinewidth{0.000000pt}%
\definecolor{currentstroke}{rgb}{0.000000,0.000000,0.000000}%
\pgfsetstrokecolor{currentstroke}%
\pgfsetdash{}{0pt}%
\pgfpathmoveto{\pgfqpoint{3.763398in}{2.297457in}}%
\pgfpathlineto{\pgfqpoint{4.018447in}{2.468167in}}%
\pgfpathlineto{\pgfqpoint{3.763398in}{2.297457in}}%
\pgfpathclose%
\pgfusepath{fill}%
\end{pgfscope}%
\begin{pgfscope}%
\pgfpathrectangle{\pgfqpoint{0.017500in}{0.017500in}}{\pgfqpoint{6.218282in}{3.580372in}} %
\pgfusepath{clip}%
\pgfsetbuttcap%
\pgfsetroundjoin%
\definecolor{currentfill}{rgb}{0.810616,0.268797,0.235428}%
\pgfsetfillcolor{currentfill}%
\pgfsetlinewidth{0.000000pt}%
\definecolor{currentstroke}{rgb}{0.000000,0.000000,0.000000}%
\pgfsetstrokecolor{currentstroke}%
\pgfsetdash{}{0pt}%
\pgfpathmoveto{\pgfqpoint{3.763398in}{2.297457in}}%
\pgfpathlineto{\pgfqpoint{4.018447in}{2.468167in}}%
\pgfpathlineto{\pgfqpoint{3.763398in}{2.297457in}}%
\pgfpathclose%
\pgfusepath{fill}%
\end{pgfscope}%
\begin{pgfscope}%
\pgfpathrectangle{\pgfqpoint{0.017500in}{0.017500in}}{\pgfqpoint{6.218282in}{3.580372in}} %
\pgfusepath{clip}%
\pgfsetbuttcap%
\pgfsetroundjoin%
\definecolor{currentfill}{rgb}{0.810616,0.268797,0.235428}%
\pgfsetfillcolor{currentfill}%
\pgfsetlinewidth{0.000000pt}%
\definecolor{currentstroke}{rgb}{0.000000,0.000000,0.000000}%
\pgfsetstrokecolor{currentstroke}%
\pgfsetdash{}{0pt}%
\pgfpathmoveto{\pgfqpoint{3.763398in}{2.297457in}}%
\pgfpathlineto{\pgfqpoint{4.018447in}{2.468167in}}%
\pgfpathlineto{\pgfqpoint{3.763398in}{2.297457in}}%
\pgfpathclose%
\pgfusepath{fill}%
\end{pgfscope}%
\begin{pgfscope}%
\pgfpathrectangle{\pgfqpoint{0.017500in}{0.017500in}}{\pgfqpoint{6.218282in}{3.580372in}} %
\pgfusepath{clip}%
\pgfsetbuttcap%
\pgfsetroundjoin%
\definecolor{currentfill}{rgb}{0.810616,0.268797,0.235428}%
\pgfsetfillcolor{currentfill}%
\pgfsetlinewidth{0.000000pt}%
\definecolor{currentstroke}{rgb}{0.000000,0.000000,0.000000}%
\pgfsetstrokecolor{currentstroke}%
\pgfsetdash{}{0pt}%
\pgfpathmoveto{\pgfqpoint{3.763398in}{2.297457in}}%
\pgfpathlineto{\pgfqpoint{4.018447in}{2.468167in}}%
\pgfpathlineto{\pgfqpoint{3.763398in}{2.297457in}}%
\pgfpathclose%
\pgfusepath{fill}%
\end{pgfscope}%
\begin{pgfscope}%
\pgfpathrectangle{\pgfqpoint{0.017500in}{0.017500in}}{\pgfqpoint{6.218282in}{3.580372in}} %
\pgfusepath{clip}%
\pgfsetbuttcap%
\pgfsetroundjoin%
\definecolor{currentfill}{rgb}{0.810616,0.268797,0.235428}%
\pgfsetfillcolor{currentfill}%
\pgfsetlinewidth{0.000000pt}%
\definecolor{currentstroke}{rgb}{0.000000,0.000000,0.000000}%
\pgfsetstrokecolor{currentstroke}%
\pgfsetdash{}{0pt}%
\pgfpathmoveto{\pgfqpoint{3.763398in}{2.297457in}}%
\pgfpathlineto{\pgfqpoint{4.018447in}{2.468167in}}%
\pgfpathlineto{\pgfqpoint{3.763398in}{2.297457in}}%
\pgfpathclose%
\pgfusepath{fill}%
\end{pgfscope}%
\begin{pgfscope}%
\pgfpathrectangle{\pgfqpoint{0.017500in}{0.017500in}}{\pgfqpoint{6.218282in}{3.580372in}} %
\pgfusepath{clip}%
\pgfsetbuttcap%
\pgfsetroundjoin%
\definecolor{currentfill}{rgb}{0.810616,0.268797,0.235428}%
\pgfsetfillcolor{currentfill}%
\pgfsetlinewidth{0.000000pt}%
\definecolor{currentstroke}{rgb}{0.000000,0.000000,0.000000}%
\pgfsetstrokecolor{currentstroke}%
\pgfsetdash{}{0pt}%
\pgfpathmoveto{\pgfqpoint{3.763398in}{2.297457in}}%
\pgfpathlineto{\pgfqpoint{4.018447in}{2.468167in}}%
\pgfpathlineto{\pgfqpoint{3.763398in}{2.297457in}}%
\pgfpathclose%
\pgfusepath{fill}%
\end{pgfscope}%
\begin{pgfscope}%
\pgfpathrectangle{\pgfqpoint{0.017500in}{0.017500in}}{\pgfqpoint{6.218282in}{3.580372in}} %
\pgfusepath{clip}%
\pgfsetbuttcap%
\pgfsetroundjoin%
\definecolor{currentfill}{rgb}{0.810616,0.268797,0.235428}%
\pgfsetfillcolor{currentfill}%
\pgfsetlinewidth{0.000000pt}%
\definecolor{currentstroke}{rgb}{0.000000,0.000000,0.000000}%
\pgfsetstrokecolor{currentstroke}%
\pgfsetdash{}{0pt}%
\pgfpathmoveto{\pgfqpoint{3.763398in}{2.297457in}}%
\pgfpathlineto{\pgfqpoint{4.018447in}{2.468167in}}%
\pgfpathlineto{\pgfqpoint{3.763398in}{2.297457in}}%
\pgfpathclose%
\pgfusepath{fill}%
\end{pgfscope}%
\begin{pgfscope}%
\pgfpathrectangle{\pgfqpoint{0.017500in}{0.017500in}}{\pgfqpoint{6.218282in}{3.580372in}} %
\pgfusepath{clip}%
\pgfsetbuttcap%
\pgfsetroundjoin%
\definecolor{currentfill}{rgb}{0.810616,0.268797,0.235428}%
\pgfsetfillcolor{currentfill}%
\pgfsetlinewidth{0.000000pt}%
\definecolor{currentstroke}{rgb}{0.000000,0.000000,0.000000}%
\pgfsetstrokecolor{currentstroke}%
\pgfsetdash{}{0pt}%
\pgfpathmoveto{\pgfqpoint{3.763398in}{2.297457in}}%
\pgfpathlineto{\pgfqpoint{4.018447in}{2.468167in}}%
\pgfpathlineto{\pgfqpoint{3.763398in}{2.297457in}}%
\pgfpathclose%
\pgfusepath{fill}%
\end{pgfscope}%
\begin{pgfscope}%
\pgfpathrectangle{\pgfqpoint{0.017500in}{0.017500in}}{\pgfqpoint{6.218282in}{3.580372in}} %
\pgfusepath{clip}%
\pgfsetbuttcap%
\pgfsetroundjoin%
\definecolor{currentfill}{rgb}{0.810616,0.268797,0.235428}%
\pgfsetfillcolor{currentfill}%
\pgfsetlinewidth{0.000000pt}%
\definecolor{currentstroke}{rgb}{0.000000,0.000000,0.000000}%
\pgfsetstrokecolor{currentstroke}%
\pgfsetdash{}{0pt}%
\pgfpathmoveto{\pgfqpoint{3.763398in}{2.297457in}}%
\pgfpathlineto{\pgfqpoint{4.018447in}{2.468167in}}%
\pgfpathlineto{\pgfqpoint{3.763398in}{2.297457in}}%
\pgfpathclose%
\pgfusepath{fill}%
\end{pgfscope}%
\begin{pgfscope}%
\pgfpathrectangle{\pgfqpoint{0.017500in}{0.017500in}}{\pgfqpoint{6.218282in}{3.580372in}} %
\pgfusepath{clip}%
\pgfsetbuttcap%
\pgfsetroundjoin%
\definecolor{currentfill}{rgb}{0.810616,0.268797,0.235428}%
\pgfsetfillcolor{currentfill}%
\pgfsetlinewidth{0.000000pt}%
\definecolor{currentstroke}{rgb}{0.000000,0.000000,0.000000}%
\pgfsetstrokecolor{currentstroke}%
\pgfsetdash{}{0pt}%
\pgfpathmoveto{\pgfqpoint{3.763398in}{2.297457in}}%
\pgfpathlineto{\pgfqpoint{4.018447in}{2.468167in}}%
\pgfpathlineto{\pgfqpoint{3.763398in}{2.297457in}}%
\pgfpathclose%
\pgfusepath{fill}%
\end{pgfscope}%
\begin{pgfscope}%
\pgfpathrectangle{\pgfqpoint{0.017500in}{0.017500in}}{\pgfqpoint{6.218282in}{3.580372in}} %
\pgfusepath{clip}%
\pgfsetbuttcap%
\pgfsetroundjoin%
\definecolor{currentfill}{rgb}{0.810616,0.268797,0.235428}%
\pgfsetfillcolor{currentfill}%
\pgfsetlinewidth{0.000000pt}%
\definecolor{currentstroke}{rgb}{0.000000,0.000000,0.000000}%
\pgfsetstrokecolor{currentstroke}%
\pgfsetdash{}{0pt}%
\pgfpathmoveto{\pgfqpoint{3.763398in}{2.297457in}}%
\pgfpathlineto{\pgfqpoint{4.018447in}{2.468167in}}%
\pgfpathlineto{\pgfqpoint{3.763398in}{2.297457in}}%
\pgfpathclose%
\pgfusepath{fill}%
\end{pgfscope}%
\begin{pgfscope}%
\pgfpathrectangle{\pgfqpoint{0.017500in}{0.017500in}}{\pgfqpoint{6.218282in}{3.580372in}} %
\pgfusepath{clip}%
\pgfsetbuttcap%
\pgfsetroundjoin%
\definecolor{currentfill}{rgb}{0.810616,0.268797,0.235428}%
\pgfsetfillcolor{currentfill}%
\pgfsetlinewidth{0.000000pt}%
\definecolor{currentstroke}{rgb}{0.000000,0.000000,0.000000}%
\pgfsetstrokecolor{currentstroke}%
\pgfsetdash{}{0pt}%
\pgfpathmoveto{\pgfqpoint{3.763398in}{2.297457in}}%
\pgfpathlineto{\pgfqpoint{4.018447in}{2.468167in}}%
\pgfpathlineto{\pgfqpoint{3.763398in}{2.297457in}}%
\pgfpathclose%
\pgfusepath{fill}%
\end{pgfscope}%
\begin{pgfscope}%
\pgfpathrectangle{\pgfqpoint{0.017500in}{0.017500in}}{\pgfqpoint{6.218282in}{3.580372in}} %
\pgfusepath{clip}%
\pgfsetbuttcap%
\pgfsetroundjoin%
\definecolor{currentfill}{rgb}{0.810616,0.268797,0.235428}%
\pgfsetfillcolor{currentfill}%
\pgfsetlinewidth{0.000000pt}%
\definecolor{currentstroke}{rgb}{0.000000,0.000000,0.000000}%
\pgfsetstrokecolor{currentstroke}%
\pgfsetdash{}{0pt}%
\pgfpathmoveto{\pgfqpoint{3.763398in}{2.297457in}}%
\pgfpathlineto{\pgfqpoint{4.018447in}{2.468167in}}%
\pgfpathlineto{\pgfqpoint{3.763398in}{2.297457in}}%
\pgfpathclose%
\pgfusepath{fill}%
\end{pgfscope}%
\begin{pgfscope}%
\pgfpathrectangle{\pgfqpoint{0.017500in}{0.017500in}}{\pgfqpoint{6.218282in}{3.580372in}} %
\pgfusepath{clip}%
\pgfsetbuttcap%
\pgfsetroundjoin%
\definecolor{currentfill}{rgb}{0.338377,0.452819,0.879317}%
\pgfsetfillcolor{currentfill}%
\pgfsetlinewidth{0.000000pt}%
\definecolor{currentstroke}{rgb}{0.000000,0.000000,0.000000}%
\pgfsetstrokecolor{currentstroke}%
\pgfsetdash{}{0pt}%
\pgfpathmoveto{\pgfqpoint{3.236099in}{0.723497in}}%
\pgfpathlineto{\pgfqpoint{3.688351in}{0.641693in}}%
\pgfpathlineto{\pgfqpoint{3.968076in}{1.994042in}}%
\pgfpathlineto{\pgfqpoint{3.502561in}{2.073396in}}%
\pgfpathlineto{\pgfqpoint{3.236099in}{0.723497in}}%
\pgfpathclose%
\pgfusepath{fill}%
\end{pgfscope}%
\begin{pgfscope}%
\pgfpathrectangle{\pgfqpoint{0.017500in}{0.017500in}}{\pgfqpoint{6.218282in}{3.580372in}} %
\pgfusepath{clip}%
\pgfsetbuttcap%
\pgfsetroundjoin%
\definecolor{currentfill}{rgb}{0.338377,0.452819,0.879317}%
\pgfsetfillcolor{currentfill}%
\pgfsetlinewidth{0.000000pt}%
\definecolor{currentstroke}{rgb}{0.000000,0.000000,0.000000}%
\pgfsetstrokecolor{currentstroke}%
\pgfsetdash{}{0pt}%
\pgfpathmoveto{\pgfqpoint{3.236099in}{0.723497in}}%
\pgfpathlineto{\pgfqpoint{3.688351in}{0.641693in}}%
\pgfpathlineto{\pgfqpoint{3.968076in}{1.994042in}}%
\pgfpathlineto{\pgfqpoint{3.502561in}{2.073396in}}%
\pgfpathlineto{\pgfqpoint{3.236099in}{0.723497in}}%
\pgfpathclose%
\pgfusepath{fill}%
\end{pgfscope}%
\begin{pgfscope}%
\pgfpathrectangle{\pgfqpoint{0.017500in}{0.017500in}}{\pgfqpoint{6.218282in}{3.580372in}} %
\pgfusepath{clip}%
\pgfsetbuttcap%
\pgfsetroundjoin%
\definecolor{currentfill}{rgb}{0.338377,0.452819,0.879317}%
\pgfsetfillcolor{currentfill}%
\pgfsetlinewidth{0.000000pt}%
\definecolor{currentstroke}{rgb}{0.000000,0.000000,0.000000}%
\pgfsetstrokecolor{currentstroke}%
\pgfsetdash{}{0pt}%
\pgfpathmoveto{\pgfqpoint{3.236099in}{0.723497in}}%
\pgfpathlineto{\pgfqpoint{3.688351in}{0.641693in}}%
\pgfpathlineto{\pgfqpoint{3.968076in}{1.994042in}}%
\pgfpathlineto{\pgfqpoint{3.502561in}{2.073396in}}%
\pgfpathlineto{\pgfqpoint{3.236099in}{0.723497in}}%
\pgfpathclose%
\pgfusepath{fill}%
\end{pgfscope}%
\begin{pgfscope}%
\pgfpathrectangle{\pgfqpoint{0.017500in}{0.017500in}}{\pgfqpoint{6.218282in}{3.580372in}} %
\pgfusepath{clip}%
\pgfsetbuttcap%
\pgfsetroundjoin%
\definecolor{currentfill}{rgb}{0.338377,0.452819,0.879317}%
\pgfsetfillcolor{currentfill}%
\pgfsetlinewidth{0.000000pt}%
\definecolor{currentstroke}{rgb}{0.000000,0.000000,0.000000}%
\pgfsetstrokecolor{currentstroke}%
\pgfsetdash{}{0pt}%
\pgfpathmoveto{\pgfqpoint{3.236099in}{0.723497in}}%
\pgfpathlineto{\pgfqpoint{3.688351in}{0.641693in}}%
\pgfpathlineto{\pgfqpoint{3.968076in}{1.994042in}}%
\pgfpathlineto{\pgfqpoint{3.502561in}{2.073396in}}%
\pgfpathlineto{\pgfqpoint{3.236099in}{0.723497in}}%
\pgfpathclose%
\pgfusepath{fill}%
\end{pgfscope}%
\begin{pgfscope}%
\pgfpathrectangle{\pgfqpoint{0.017500in}{0.017500in}}{\pgfqpoint{6.218282in}{3.580372in}} %
\pgfusepath{clip}%
\pgfsetbuttcap%
\pgfsetroundjoin%
\definecolor{currentfill}{rgb}{0.338377,0.452819,0.879317}%
\pgfsetfillcolor{currentfill}%
\pgfsetlinewidth{0.000000pt}%
\definecolor{currentstroke}{rgb}{0.000000,0.000000,0.000000}%
\pgfsetstrokecolor{currentstroke}%
\pgfsetdash{}{0pt}%
\pgfpathmoveto{\pgfqpoint{3.236099in}{0.723497in}}%
\pgfpathlineto{\pgfqpoint{3.688351in}{0.641693in}}%
\pgfpathlineto{\pgfqpoint{3.968076in}{1.994042in}}%
\pgfpathlineto{\pgfqpoint{3.502561in}{2.073396in}}%
\pgfpathlineto{\pgfqpoint{3.236099in}{0.723497in}}%
\pgfpathclose%
\pgfusepath{fill}%
\end{pgfscope}%
\begin{pgfscope}%
\pgfpathrectangle{\pgfqpoint{0.017500in}{0.017500in}}{\pgfqpoint{6.218282in}{3.580372in}} %
\pgfusepath{clip}%
\pgfsetbuttcap%
\pgfsetroundjoin%
\definecolor{currentfill}{rgb}{0.338377,0.452819,0.879317}%
\pgfsetfillcolor{currentfill}%
\pgfsetlinewidth{0.000000pt}%
\definecolor{currentstroke}{rgb}{0.000000,0.000000,0.000000}%
\pgfsetstrokecolor{currentstroke}%
\pgfsetdash{}{0pt}%
\pgfpathmoveto{\pgfqpoint{3.236099in}{0.723497in}}%
\pgfpathlineto{\pgfqpoint{3.688351in}{0.641693in}}%
\pgfpathlineto{\pgfqpoint{3.968076in}{1.994042in}}%
\pgfpathlineto{\pgfqpoint{3.502561in}{2.073396in}}%
\pgfpathlineto{\pgfqpoint{3.236099in}{0.723497in}}%
\pgfpathclose%
\pgfusepath{fill}%
\end{pgfscope}%
\begin{pgfscope}%
\pgfpathrectangle{\pgfqpoint{0.017500in}{0.017500in}}{\pgfqpoint{6.218282in}{3.580372in}} %
\pgfusepath{clip}%
\pgfsetbuttcap%
\pgfsetroundjoin%
\definecolor{currentfill}{rgb}{0.805723,0.259813,0.230562}%
\pgfsetfillcolor{currentfill}%
\pgfsetlinewidth{0.000000pt}%
\definecolor{currentstroke}{rgb}{0.000000,0.000000,0.000000}%
\pgfsetstrokecolor{currentstroke}%
\pgfsetdash{}{0pt}%
\pgfpathmoveto{\pgfqpoint{3.763398in}{2.297457in}}%
\pgfpathlineto{\pgfqpoint{4.226975in}{2.236848in}}%
\pgfpathlineto{\pgfqpoint{4.478468in}{2.393972in}}%
\pgfpathlineto{\pgfqpoint{4.018447in}{2.468167in}}%
\pgfpathlineto{\pgfqpoint{3.763398in}{2.297457in}}%
\pgfpathclose%
\pgfusepath{fill}%
\end{pgfscope}%
\begin{pgfscope}%
\pgfpathrectangle{\pgfqpoint{0.017500in}{0.017500in}}{\pgfqpoint{6.218282in}{3.580372in}} %
\pgfusepath{clip}%
\pgfsetbuttcap%
\pgfsetroundjoin%
\definecolor{currentfill}{rgb}{0.805723,0.259813,0.230562}%
\pgfsetfillcolor{currentfill}%
\pgfsetlinewidth{0.000000pt}%
\definecolor{currentstroke}{rgb}{0.000000,0.000000,0.000000}%
\pgfsetstrokecolor{currentstroke}%
\pgfsetdash{}{0pt}%
\pgfpathmoveto{\pgfqpoint{3.763398in}{2.297457in}}%
\pgfpathlineto{\pgfqpoint{4.226975in}{2.236848in}}%
\pgfpathlineto{\pgfqpoint{4.478468in}{2.393972in}}%
\pgfpathlineto{\pgfqpoint{4.018447in}{2.468167in}}%
\pgfpathlineto{\pgfqpoint{3.763398in}{2.297457in}}%
\pgfpathclose%
\pgfusepath{fill}%
\end{pgfscope}%
\begin{pgfscope}%
\pgfpathrectangle{\pgfqpoint{0.017500in}{0.017500in}}{\pgfqpoint{6.218282in}{3.580372in}} %
\pgfusepath{clip}%
\pgfsetbuttcap%
\pgfsetroundjoin%
\definecolor{currentfill}{rgb}{0.805723,0.259813,0.230562}%
\pgfsetfillcolor{currentfill}%
\pgfsetlinewidth{0.000000pt}%
\definecolor{currentstroke}{rgb}{0.000000,0.000000,0.000000}%
\pgfsetstrokecolor{currentstroke}%
\pgfsetdash{}{0pt}%
\pgfpathmoveto{\pgfqpoint{3.763398in}{2.297457in}}%
\pgfpathlineto{\pgfqpoint{4.226975in}{2.236848in}}%
\pgfpathlineto{\pgfqpoint{4.478468in}{2.393972in}}%
\pgfpathlineto{\pgfqpoint{4.018447in}{2.468167in}}%
\pgfpathlineto{\pgfqpoint{3.763398in}{2.297457in}}%
\pgfpathclose%
\pgfusepath{fill}%
\end{pgfscope}%
\begin{pgfscope}%
\pgfpathrectangle{\pgfqpoint{0.017500in}{0.017500in}}{\pgfqpoint{6.218282in}{3.580372in}} %
\pgfusepath{clip}%
\pgfsetbuttcap%
\pgfsetroundjoin%
\definecolor{currentfill}{rgb}{0.805723,0.259813,0.230562}%
\pgfsetfillcolor{currentfill}%
\pgfsetlinewidth{0.000000pt}%
\definecolor{currentstroke}{rgb}{0.000000,0.000000,0.000000}%
\pgfsetstrokecolor{currentstroke}%
\pgfsetdash{}{0pt}%
\pgfpathmoveto{\pgfqpoint{3.763398in}{2.297457in}}%
\pgfpathlineto{\pgfqpoint{4.226975in}{2.236848in}}%
\pgfpathlineto{\pgfqpoint{4.478468in}{2.393972in}}%
\pgfpathlineto{\pgfqpoint{4.018447in}{2.468167in}}%
\pgfpathlineto{\pgfqpoint{3.763398in}{2.297457in}}%
\pgfpathclose%
\pgfusepath{fill}%
\end{pgfscope}%
\begin{pgfscope}%
\pgfpathrectangle{\pgfqpoint{0.017500in}{0.017500in}}{\pgfqpoint{6.218282in}{3.580372in}} %
\pgfusepath{clip}%
\pgfsetbuttcap%
\pgfsetroundjoin%
\definecolor{currentfill}{rgb}{0.805723,0.259813,0.230562}%
\pgfsetfillcolor{currentfill}%
\pgfsetlinewidth{0.000000pt}%
\definecolor{currentstroke}{rgb}{0.000000,0.000000,0.000000}%
\pgfsetstrokecolor{currentstroke}%
\pgfsetdash{}{0pt}%
\pgfpathmoveto{\pgfqpoint{3.763398in}{2.297457in}}%
\pgfpathlineto{\pgfqpoint{4.226975in}{2.236848in}}%
\pgfpathlineto{\pgfqpoint{4.478468in}{2.393972in}}%
\pgfpathlineto{\pgfqpoint{4.018447in}{2.468167in}}%
\pgfpathlineto{\pgfqpoint{3.763398in}{2.297457in}}%
\pgfpathclose%
\pgfusepath{fill}%
\end{pgfscope}%
\begin{pgfscope}%
\pgfpathrectangle{\pgfqpoint{0.017500in}{0.017500in}}{\pgfqpoint{6.218282in}{3.580372in}} %
\pgfusepath{clip}%
\pgfsetbuttcap%
\pgfsetroundjoin%
\definecolor{currentfill}{rgb}{0.805723,0.259813,0.230562}%
\pgfsetfillcolor{currentfill}%
\pgfsetlinewidth{0.000000pt}%
\definecolor{currentstroke}{rgb}{0.000000,0.000000,0.000000}%
\pgfsetstrokecolor{currentstroke}%
\pgfsetdash{}{0pt}%
\pgfpathmoveto{\pgfqpoint{3.763398in}{2.297457in}}%
\pgfpathlineto{\pgfqpoint{4.226975in}{2.236848in}}%
\pgfpathlineto{\pgfqpoint{4.478468in}{2.393972in}}%
\pgfpathlineto{\pgfqpoint{4.018447in}{2.468167in}}%
\pgfpathlineto{\pgfqpoint{3.763398in}{2.297457in}}%
\pgfpathclose%
\pgfusepath{fill}%
\end{pgfscope}%
\begin{pgfscope}%
\pgfpathrectangle{\pgfqpoint{0.017500in}{0.017500in}}{\pgfqpoint{6.218282in}{3.580372in}} %
\pgfusepath{clip}%
\pgfsetbuttcap%
\pgfsetroundjoin%
\definecolor{currentfill}{rgb}{0.243520,0.319189,0.771672}%
\pgfsetfillcolor{currentfill}%
\pgfsetlinewidth{0.000000pt}%
\definecolor{currentstroke}{rgb}{0.000000,0.000000,0.000000}%
\pgfsetstrokecolor{currentstroke}%
\pgfsetdash{}{0pt}%
\pgfpathmoveto{\pgfqpoint{3.688351in}{0.641693in}}%
\pgfpathlineto{\pgfqpoint{3.968076in}{1.994042in}}%
\pgfpathlineto{\pgfqpoint{3.688351in}{0.641693in}}%
\pgfpathclose%
\pgfusepath{fill}%
\end{pgfscope}%
\begin{pgfscope}%
\pgfpathrectangle{\pgfqpoint{0.017500in}{0.017500in}}{\pgfqpoint{6.218282in}{3.580372in}} %
\pgfusepath{clip}%
\pgfsetbuttcap%
\pgfsetroundjoin%
\definecolor{currentfill}{rgb}{0.243520,0.319189,0.771672}%
\pgfsetfillcolor{currentfill}%
\pgfsetlinewidth{0.000000pt}%
\definecolor{currentstroke}{rgb}{0.000000,0.000000,0.000000}%
\pgfsetstrokecolor{currentstroke}%
\pgfsetdash{}{0pt}%
\pgfpathmoveto{\pgfqpoint{3.688351in}{0.641693in}}%
\pgfpathlineto{\pgfqpoint{3.968076in}{1.994042in}}%
\pgfpathlineto{\pgfqpoint{3.688351in}{0.641693in}}%
\pgfpathclose%
\pgfusepath{fill}%
\end{pgfscope}%
\begin{pgfscope}%
\pgfpathrectangle{\pgfqpoint{0.017500in}{0.017500in}}{\pgfqpoint{6.218282in}{3.580372in}} %
\pgfusepath{clip}%
\pgfsetbuttcap%
\pgfsetroundjoin%
\definecolor{currentfill}{rgb}{0.243520,0.319189,0.771672}%
\pgfsetfillcolor{currentfill}%
\pgfsetlinewidth{0.000000pt}%
\definecolor{currentstroke}{rgb}{0.000000,0.000000,0.000000}%
\pgfsetstrokecolor{currentstroke}%
\pgfsetdash{}{0pt}%
\pgfpathmoveto{\pgfqpoint{3.688351in}{0.641693in}}%
\pgfpathlineto{\pgfqpoint{3.968076in}{1.994042in}}%
\pgfpathlineto{\pgfqpoint{3.688351in}{0.641693in}}%
\pgfpathclose%
\pgfusepath{fill}%
\end{pgfscope}%
\begin{pgfscope}%
\pgfpathrectangle{\pgfqpoint{0.017500in}{0.017500in}}{\pgfqpoint{6.218282in}{3.580372in}} %
\pgfusepath{clip}%
\pgfsetbuttcap%
\pgfsetroundjoin%
\definecolor{currentfill}{rgb}{0.243520,0.319189,0.771672}%
\pgfsetfillcolor{currentfill}%
\pgfsetlinewidth{0.000000pt}%
\definecolor{currentstroke}{rgb}{0.000000,0.000000,0.000000}%
\pgfsetstrokecolor{currentstroke}%
\pgfsetdash{}{0pt}%
\pgfpathmoveto{\pgfqpoint{3.688351in}{0.641693in}}%
\pgfpathlineto{\pgfqpoint{3.968076in}{1.994042in}}%
\pgfpathlineto{\pgfqpoint{3.688351in}{0.641693in}}%
\pgfpathclose%
\pgfusepath{fill}%
\end{pgfscope}%
\begin{pgfscope}%
\pgfpathrectangle{\pgfqpoint{0.017500in}{0.017500in}}{\pgfqpoint{6.218282in}{3.580372in}} %
\pgfusepath{clip}%
\pgfsetbuttcap%
\pgfsetroundjoin%
\definecolor{currentfill}{rgb}{0.243520,0.319189,0.771672}%
\pgfsetfillcolor{currentfill}%
\pgfsetlinewidth{0.000000pt}%
\definecolor{currentstroke}{rgb}{0.000000,0.000000,0.000000}%
\pgfsetstrokecolor{currentstroke}%
\pgfsetdash{}{0pt}%
\pgfpathmoveto{\pgfqpoint{3.688351in}{0.641693in}}%
\pgfpathlineto{\pgfqpoint{3.968076in}{1.994042in}}%
\pgfpathlineto{\pgfqpoint{3.688351in}{0.641693in}}%
\pgfpathclose%
\pgfusepath{fill}%
\end{pgfscope}%
\begin{pgfscope}%
\pgfpathrectangle{\pgfqpoint{0.017500in}{0.017500in}}{\pgfqpoint{6.218282in}{3.580372in}} %
\pgfusepath{clip}%
\pgfsetbuttcap%
\pgfsetroundjoin%
\definecolor{currentfill}{rgb}{0.243520,0.319189,0.771672}%
\pgfsetfillcolor{currentfill}%
\pgfsetlinewidth{0.000000pt}%
\definecolor{currentstroke}{rgb}{0.000000,0.000000,0.000000}%
\pgfsetstrokecolor{currentstroke}%
\pgfsetdash{}{0pt}%
\pgfpathmoveto{\pgfqpoint{3.688351in}{0.641693in}}%
\pgfpathlineto{\pgfqpoint{3.968076in}{1.994042in}}%
\pgfpathlineto{\pgfqpoint{3.688351in}{0.641693in}}%
\pgfpathclose%
\pgfusepath{fill}%
\end{pgfscope}%
\begin{pgfscope}%
\pgfpathrectangle{\pgfqpoint{0.017500in}{0.017500in}}{\pgfqpoint{6.218282in}{3.580372in}} %
\pgfusepath{clip}%
\pgfsetbuttcap%
\pgfsetroundjoin%
\definecolor{currentfill}{rgb}{0.243520,0.319189,0.771672}%
\pgfsetfillcolor{currentfill}%
\pgfsetlinewidth{0.000000pt}%
\definecolor{currentstroke}{rgb}{0.000000,0.000000,0.000000}%
\pgfsetstrokecolor{currentstroke}%
\pgfsetdash{}{0pt}%
\pgfpathmoveto{\pgfqpoint{3.688351in}{0.641693in}}%
\pgfpathlineto{\pgfqpoint{3.968076in}{1.994042in}}%
\pgfpathlineto{\pgfqpoint{3.688351in}{0.641693in}}%
\pgfpathclose%
\pgfusepath{fill}%
\end{pgfscope}%
\begin{pgfscope}%
\pgfpathrectangle{\pgfqpoint{0.017500in}{0.017500in}}{\pgfqpoint{6.218282in}{3.580372in}} %
\pgfusepath{clip}%
\pgfsetbuttcap%
\pgfsetroundjoin%
\definecolor{currentfill}{rgb}{0.243520,0.319189,0.771672}%
\pgfsetfillcolor{currentfill}%
\pgfsetlinewidth{0.000000pt}%
\definecolor{currentstroke}{rgb}{0.000000,0.000000,0.000000}%
\pgfsetstrokecolor{currentstroke}%
\pgfsetdash{}{0pt}%
\pgfpathmoveto{\pgfqpoint{3.688351in}{0.641693in}}%
\pgfpathlineto{\pgfqpoint{3.968076in}{1.994042in}}%
\pgfpathlineto{\pgfqpoint{3.688351in}{0.641693in}}%
\pgfpathclose%
\pgfusepath{fill}%
\end{pgfscope}%
\begin{pgfscope}%
\pgfpathrectangle{\pgfqpoint{0.017500in}{0.017500in}}{\pgfqpoint{6.218282in}{3.580372in}} %
\pgfusepath{clip}%
\pgfsetbuttcap%
\pgfsetroundjoin%
\definecolor{currentfill}{rgb}{0.243520,0.319189,0.771672}%
\pgfsetfillcolor{currentfill}%
\pgfsetlinewidth{0.000000pt}%
\definecolor{currentstroke}{rgb}{0.000000,0.000000,0.000000}%
\pgfsetstrokecolor{currentstroke}%
\pgfsetdash{}{0pt}%
\pgfpathmoveto{\pgfqpoint{3.688351in}{0.641693in}}%
\pgfpathlineto{\pgfqpoint{3.968076in}{1.994042in}}%
\pgfpathlineto{\pgfqpoint{3.688351in}{0.641693in}}%
\pgfpathclose%
\pgfusepath{fill}%
\end{pgfscope}%
\begin{pgfscope}%
\pgfpathrectangle{\pgfqpoint{0.017500in}{0.017500in}}{\pgfqpoint{6.218282in}{3.580372in}} %
\pgfusepath{clip}%
\pgfsetbuttcap%
\pgfsetroundjoin%
\definecolor{currentfill}{rgb}{0.243520,0.319189,0.771672}%
\pgfsetfillcolor{currentfill}%
\pgfsetlinewidth{0.000000pt}%
\definecolor{currentstroke}{rgb}{0.000000,0.000000,0.000000}%
\pgfsetstrokecolor{currentstroke}%
\pgfsetdash{}{0pt}%
\pgfpathmoveto{\pgfqpoint{3.688351in}{0.641693in}}%
\pgfpathlineto{\pgfqpoint{3.968076in}{1.994042in}}%
\pgfpathlineto{\pgfqpoint{3.688351in}{0.641693in}}%
\pgfpathclose%
\pgfusepath{fill}%
\end{pgfscope}%
\begin{pgfscope}%
\pgfpathrectangle{\pgfqpoint{0.017500in}{0.017500in}}{\pgfqpoint{6.218282in}{3.580372in}} %
\pgfusepath{clip}%
\pgfsetbuttcap%
\pgfsetroundjoin%
\definecolor{currentfill}{rgb}{0.243520,0.319189,0.771672}%
\pgfsetfillcolor{currentfill}%
\pgfsetlinewidth{0.000000pt}%
\definecolor{currentstroke}{rgb}{0.000000,0.000000,0.000000}%
\pgfsetstrokecolor{currentstroke}%
\pgfsetdash{}{0pt}%
\pgfpathmoveto{\pgfqpoint{3.688351in}{0.641693in}}%
\pgfpathlineto{\pgfqpoint{3.968076in}{1.994042in}}%
\pgfpathlineto{\pgfqpoint{3.688351in}{0.641693in}}%
\pgfpathclose%
\pgfusepath{fill}%
\end{pgfscope}%
\begin{pgfscope}%
\pgfpathrectangle{\pgfqpoint{0.017500in}{0.017500in}}{\pgfqpoint{6.218282in}{3.580372in}} %
\pgfusepath{clip}%
\pgfsetbuttcap%
\pgfsetroundjoin%
\definecolor{currentfill}{rgb}{0.243520,0.319189,0.771672}%
\pgfsetfillcolor{currentfill}%
\pgfsetlinewidth{0.000000pt}%
\definecolor{currentstroke}{rgb}{0.000000,0.000000,0.000000}%
\pgfsetstrokecolor{currentstroke}%
\pgfsetdash{}{0pt}%
\pgfpathmoveto{\pgfqpoint{3.688351in}{0.641693in}}%
\pgfpathlineto{\pgfqpoint{3.968076in}{1.994042in}}%
\pgfpathlineto{\pgfqpoint{3.688351in}{0.641693in}}%
\pgfpathclose%
\pgfusepath{fill}%
\end{pgfscope}%
\begin{pgfscope}%
\pgfpathrectangle{\pgfqpoint{0.017500in}{0.017500in}}{\pgfqpoint{6.218282in}{3.580372in}} %
\pgfusepath{clip}%
\pgfsetbuttcap%
\pgfsetroundjoin%
\definecolor{currentfill}{rgb}{0.243520,0.319189,0.771672}%
\pgfsetfillcolor{currentfill}%
\pgfsetlinewidth{0.000000pt}%
\definecolor{currentstroke}{rgb}{0.000000,0.000000,0.000000}%
\pgfsetstrokecolor{currentstroke}%
\pgfsetdash{}{0pt}%
\pgfpathmoveto{\pgfqpoint{3.688351in}{0.641693in}}%
\pgfpathlineto{\pgfqpoint{3.968076in}{1.994042in}}%
\pgfpathlineto{\pgfqpoint{3.688351in}{0.641693in}}%
\pgfpathclose%
\pgfusepath{fill}%
\end{pgfscope}%
\begin{pgfscope}%
\pgfpathrectangle{\pgfqpoint{0.017500in}{0.017500in}}{\pgfqpoint{6.218282in}{3.580372in}} %
\pgfusepath{clip}%
\pgfsetbuttcap%
\pgfsetroundjoin%
\definecolor{currentfill}{rgb}{0.243520,0.319189,0.771672}%
\pgfsetfillcolor{currentfill}%
\pgfsetlinewidth{0.000000pt}%
\definecolor{currentstroke}{rgb}{0.000000,0.000000,0.000000}%
\pgfsetstrokecolor{currentstroke}%
\pgfsetdash{}{0pt}%
\pgfpathmoveto{\pgfqpoint{3.688351in}{0.641693in}}%
\pgfpathlineto{\pgfqpoint{3.968076in}{1.994042in}}%
\pgfpathlineto{\pgfqpoint{3.688351in}{0.641693in}}%
\pgfpathclose%
\pgfusepath{fill}%
\end{pgfscope}%
\begin{pgfscope}%
\pgfpathrectangle{\pgfqpoint{0.017500in}{0.017500in}}{\pgfqpoint{6.218282in}{3.580372in}} %
\pgfusepath{clip}%
\pgfsetbuttcap%
\pgfsetroundjoin%
\definecolor{currentfill}{rgb}{0.243520,0.319189,0.771672}%
\pgfsetfillcolor{currentfill}%
\pgfsetlinewidth{0.000000pt}%
\definecolor{currentstroke}{rgb}{0.000000,0.000000,0.000000}%
\pgfsetstrokecolor{currentstroke}%
\pgfsetdash{}{0pt}%
\pgfpathmoveto{\pgfqpoint{3.688351in}{0.641693in}}%
\pgfpathlineto{\pgfqpoint{3.968076in}{1.994042in}}%
\pgfpathlineto{\pgfqpoint{3.688351in}{0.641693in}}%
\pgfpathclose%
\pgfusepath{fill}%
\end{pgfscope}%
\begin{pgfscope}%
\pgfpathrectangle{\pgfqpoint{0.017500in}{0.017500in}}{\pgfqpoint{6.218282in}{3.580372in}} %
\pgfusepath{clip}%
\pgfsetbuttcap%
\pgfsetroundjoin%
\definecolor{currentfill}{rgb}{0.243520,0.319189,0.771672}%
\pgfsetfillcolor{currentfill}%
\pgfsetlinewidth{0.000000pt}%
\definecolor{currentstroke}{rgb}{0.000000,0.000000,0.000000}%
\pgfsetstrokecolor{currentstroke}%
\pgfsetdash{}{0pt}%
\pgfpathmoveto{\pgfqpoint{3.688351in}{0.641693in}}%
\pgfpathlineto{\pgfqpoint{3.968076in}{1.994042in}}%
\pgfpathlineto{\pgfqpoint{3.688351in}{0.641693in}}%
\pgfpathclose%
\pgfusepath{fill}%
\end{pgfscope}%
\begin{pgfscope}%
\pgfpathrectangle{\pgfqpoint{0.017500in}{0.017500in}}{\pgfqpoint{6.218282in}{3.580372in}} %
\pgfusepath{clip}%
\pgfsetbuttcap%
\pgfsetroundjoin%
\definecolor{currentfill}{rgb}{0.243520,0.319189,0.771672}%
\pgfsetfillcolor{currentfill}%
\pgfsetlinewidth{0.000000pt}%
\definecolor{currentstroke}{rgb}{0.000000,0.000000,0.000000}%
\pgfsetstrokecolor{currentstroke}%
\pgfsetdash{}{0pt}%
\pgfpathmoveto{\pgfqpoint{3.688351in}{0.641693in}}%
\pgfpathlineto{\pgfqpoint{3.968076in}{1.994042in}}%
\pgfpathlineto{\pgfqpoint{3.688351in}{0.641693in}}%
\pgfpathclose%
\pgfusepath{fill}%
\end{pgfscope}%
\begin{pgfscope}%
\pgfpathrectangle{\pgfqpoint{0.017500in}{0.017500in}}{\pgfqpoint{6.218282in}{3.580372in}} %
\pgfusepath{clip}%
\pgfsetbuttcap%
\pgfsetroundjoin%
\definecolor{currentfill}{rgb}{0.243520,0.319189,0.771672}%
\pgfsetfillcolor{currentfill}%
\pgfsetlinewidth{0.000000pt}%
\definecolor{currentstroke}{rgb}{0.000000,0.000000,0.000000}%
\pgfsetstrokecolor{currentstroke}%
\pgfsetdash{}{0pt}%
\pgfpathmoveto{\pgfqpoint{3.688351in}{0.641693in}}%
\pgfpathlineto{\pgfqpoint{3.968076in}{1.994042in}}%
\pgfpathlineto{\pgfqpoint{3.688351in}{0.641693in}}%
\pgfpathclose%
\pgfusepath{fill}%
\end{pgfscope}%
\begin{pgfscope}%
\pgfpathrectangle{\pgfqpoint{0.017500in}{0.017500in}}{\pgfqpoint{6.218282in}{3.580372in}} %
\pgfusepath{clip}%
\pgfsetbuttcap%
\pgfsetroundjoin%
\definecolor{currentfill}{rgb}{0.243520,0.319189,0.771672}%
\pgfsetfillcolor{currentfill}%
\pgfsetlinewidth{0.000000pt}%
\definecolor{currentstroke}{rgb}{0.000000,0.000000,0.000000}%
\pgfsetstrokecolor{currentstroke}%
\pgfsetdash{}{0pt}%
\pgfpathmoveto{\pgfqpoint{3.688351in}{0.641693in}}%
\pgfpathlineto{\pgfqpoint{3.968076in}{1.994042in}}%
\pgfpathlineto{\pgfqpoint{3.688351in}{0.641693in}}%
\pgfpathclose%
\pgfusepath{fill}%
\end{pgfscope}%
\begin{pgfscope}%
\pgfpathrectangle{\pgfqpoint{0.017500in}{0.017500in}}{\pgfqpoint{6.218282in}{3.580372in}} %
\pgfusepath{clip}%
\pgfsetbuttcap%
\pgfsetroundjoin%
\definecolor{currentfill}{rgb}{0.243520,0.319189,0.771672}%
\pgfsetfillcolor{currentfill}%
\pgfsetlinewidth{0.000000pt}%
\definecolor{currentstroke}{rgb}{0.000000,0.000000,0.000000}%
\pgfsetstrokecolor{currentstroke}%
\pgfsetdash{}{0pt}%
\pgfpathmoveto{\pgfqpoint{3.688351in}{0.641693in}}%
\pgfpathlineto{\pgfqpoint{3.968076in}{1.994042in}}%
\pgfpathlineto{\pgfqpoint{3.688351in}{0.641693in}}%
\pgfpathclose%
\pgfusepath{fill}%
\end{pgfscope}%
\begin{pgfscope}%
\pgfpathrectangle{\pgfqpoint{0.017500in}{0.017500in}}{\pgfqpoint{6.218282in}{3.580372in}} %
\pgfusepath{clip}%
\pgfsetbuttcap%
\pgfsetroundjoin%
\definecolor{currentfill}{rgb}{0.243520,0.319189,0.771672}%
\pgfsetfillcolor{currentfill}%
\pgfsetlinewidth{0.000000pt}%
\definecolor{currentstroke}{rgb}{0.000000,0.000000,0.000000}%
\pgfsetstrokecolor{currentstroke}%
\pgfsetdash{}{0pt}%
\pgfpathmoveto{\pgfqpoint{3.688351in}{0.641693in}}%
\pgfpathlineto{\pgfqpoint{3.968076in}{1.994042in}}%
\pgfpathlineto{\pgfqpoint{3.688351in}{0.641693in}}%
\pgfpathclose%
\pgfusepath{fill}%
\end{pgfscope}%
\begin{pgfscope}%
\pgfpathrectangle{\pgfqpoint{0.017500in}{0.017500in}}{\pgfqpoint{6.218282in}{3.580372in}} %
\pgfusepath{clip}%
\pgfsetbuttcap%
\pgfsetroundjoin%
\definecolor{currentfill}{rgb}{0.243520,0.319189,0.771672}%
\pgfsetfillcolor{currentfill}%
\pgfsetlinewidth{0.000000pt}%
\definecolor{currentstroke}{rgb}{0.000000,0.000000,0.000000}%
\pgfsetstrokecolor{currentstroke}%
\pgfsetdash{}{0pt}%
\pgfpathmoveto{\pgfqpoint{3.688351in}{0.641693in}}%
\pgfpathlineto{\pgfqpoint{3.968076in}{1.994042in}}%
\pgfpathlineto{\pgfqpoint{3.688351in}{0.641693in}}%
\pgfpathclose%
\pgfusepath{fill}%
\end{pgfscope}%
\begin{pgfscope}%
\pgfpathrectangle{\pgfqpoint{0.017500in}{0.017500in}}{\pgfqpoint{6.218282in}{3.580372in}} %
\pgfusepath{clip}%
\pgfsetbuttcap%
\pgfsetroundjoin%
\definecolor{currentfill}{rgb}{0.243520,0.319189,0.771672}%
\pgfsetfillcolor{currentfill}%
\pgfsetlinewidth{0.000000pt}%
\definecolor{currentstroke}{rgb}{0.000000,0.000000,0.000000}%
\pgfsetstrokecolor{currentstroke}%
\pgfsetdash{}{0pt}%
\pgfpathmoveto{\pgfqpoint{3.688351in}{0.641693in}}%
\pgfpathlineto{\pgfqpoint{3.968076in}{1.994042in}}%
\pgfpathlineto{\pgfqpoint{3.688351in}{0.641693in}}%
\pgfpathclose%
\pgfusepath{fill}%
\end{pgfscope}%
\begin{pgfscope}%
\pgfpathrectangle{\pgfqpoint{0.017500in}{0.017500in}}{\pgfqpoint{6.218282in}{3.580372in}} %
\pgfusepath{clip}%
\pgfsetbuttcap%
\pgfsetroundjoin%
\definecolor{currentfill}{rgb}{0.243520,0.319189,0.771672}%
\pgfsetfillcolor{currentfill}%
\pgfsetlinewidth{0.000000pt}%
\definecolor{currentstroke}{rgb}{0.000000,0.000000,0.000000}%
\pgfsetstrokecolor{currentstroke}%
\pgfsetdash{}{0pt}%
\pgfpathmoveto{\pgfqpoint{3.688351in}{0.641693in}}%
\pgfpathlineto{\pgfqpoint{3.968076in}{1.994042in}}%
\pgfpathlineto{\pgfqpoint{3.688351in}{0.641693in}}%
\pgfpathclose%
\pgfusepath{fill}%
\end{pgfscope}%
\begin{pgfscope}%
\pgfpathrectangle{\pgfqpoint{0.017500in}{0.017500in}}{\pgfqpoint{6.218282in}{3.580372in}} %
\pgfusepath{clip}%
\pgfsetbuttcap%
\pgfsetroundjoin%
\definecolor{currentfill}{rgb}{0.243520,0.319189,0.771672}%
\pgfsetfillcolor{currentfill}%
\pgfsetlinewidth{0.000000pt}%
\definecolor{currentstroke}{rgb}{0.000000,0.000000,0.000000}%
\pgfsetstrokecolor{currentstroke}%
\pgfsetdash{}{0pt}%
\pgfpathmoveto{\pgfqpoint{3.688351in}{0.641693in}}%
\pgfpathlineto{\pgfqpoint{3.968076in}{1.994042in}}%
\pgfpathlineto{\pgfqpoint{3.688351in}{0.641693in}}%
\pgfpathclose%
\pgfusepath{fill}%
\end{pgfscope}%
\begin{pgfscope}%
\pgfpathrectangle{\pgfqpoint{0.017500in}{0.017500in}}{\pgfqpoint{6.218282in}{3.580372in}} %
\pgfusepath{clip}%
\pgfsetbuttcap%
\pgfsetroundjoin%
\definecolor{currentfill}{rgb}{0.243520,0.319189,0.771672}%
\pgfsetfillcolor{currentfill}%
\pgfsetlinewidth{0.000000pt}%
\definecolor{currentstroke}{rgb}{0.000000,0.000000,0.000000}%
\pgfsetstrokecolor{currentstroke}%
\pgfsetdash{}{0pt}%
\pgfpathmoveto{\pgfqpoint{3.688351in}{0.641693in}}%
\pgfpathlineto{\pgfqpoint{3.968076in}{1.994042in}}%
\pgfpathlineto{\pgfqpoint{3.688351in}{0.641693in}}%
\pgfpathclose%
\pgfusepath{fill}%
\end{pgfscope}%
\begin{pgfscope}%
\pgfpathrectangle{\pgfqpoint{0.017500in}{0.017500in}}{\pgfqpoint{6.218282in}{3.580372in}} %
\pgfusepath{clip}%
\pgfsetbuttcap%
\pgfsetroundjoin%
\definecolor{currentfill}{rgb}{0.243520,0.319189,0.771672}%
\pgfsetfillcolor{currentfill}%
\pgfsetlinewidth{0.000000pt}%
\definecolor{currentstroke}{rgb}{0.000000,0.000000,0.000000}%
\pgfsetstrokecolor{currentstroke}%
\pgfsetdash{}{0pt}%
\pgfpathmoveto{\pgfqpoint{3.688351in}{0.641693in}}%
\pgfpathlineto{\pgfqpoint{3.968076in}{1.994042in}}%
\pgfpathlineto{\pgfqpoint{3.688351in}{0.641693in}}%
\pgfpathclose%
\pgfusepath{fill}%
\end{pgfscope}%
\begin{pgfscope}%
\pgfpathrectangle{\pgfqpoint{0.017500in}{0.017500in}}{\pgfqpoint{6.218282in}{3.580372in}} %
\pgfusepath{clip}%
\pgfsetbuttcap%
\pgfsetroundjoin%
\definecolor{currentfill}{rgb}{0.243520,0.319189,0.771672}%
\pgfsetfillcolor{currentfill}%
\pgfsetlinewidth{0.000000pt}%
\definecolor{currentstroke}{rgb}{0.000000,0.000000,0.000000}%
\pgfsetstrokecolor{currentstroke}%
\pgfsetdash{}{0pt}%
\pgfpathmoveto{\pgfqpoint{3.688351in}{0.641693in}}%
\pgfpathlineto{\pgfqpoint{3.968076in}{1.994042in}}%
\pgfpathlineto{\pgfqpoint{3.688351in}{0.641693in}}%
\pgfpathclose%
\pgfusepath{fill}%
\end{pgfscope}%
\begin{pgfscope}%
\pgfpathrectangle{\pgfqpoint{0.017500in}{0.017500in}}{\pgfqpoint{6.218282in}{3.580372in}} %
\pgfusepath{clip}%
\pgfsetbuttcap%
\pgfsetroundjoin%
\definecolor{currentfill}{rgb}{0.243520,0.319189,0.771672}%
\pgfsetfillcolor{currentfill}%
\pgfsetlinewidth{0.000000pt}%
\definecolor{currentstroke}{rgb}{0.000000,0.000000,0.000000}%
\pgfsetstrokecolor{currentstroke}%
\pgfsetdash{}{0pt}%
\pgfpathmoveto{\pgfqpoint{3.688351in}{0.641693in}}%
\pgfpathlineto{\pgfqpoint{3.968076in}{1.994042in}}%
\pgfpathlineto{\pgfqpoint{3.688351in}{0.641693in}}%
\pgfpathclose%
\pgfusepath{fill}%
\end{pgfscope}%
\begin{pgfscope}%
\pgfpathrectangle{\pgfqpoint{0.017500in}{0.017500in}}{\pgfqpoint{6.218282in}{3.580372in}} %
\pgfusepath{clip}%
\pgfsetbuttcap%
\pgfsetroundjoin%
\definecolor{currentfill}{rgb}{0.243520,0.319189,0.771672}%
\pgfsetfillcolor{currentfill}%
\pgfsetlinewidth{0.000000pt}%
\definecolor{currentstroke}{rgb}{0.000000,0.000000,0.000000}%
\pgfsetstrokecolor{currentstroke}%
\pgfsetdash{}{0pt}%
\pgfpathmoveto{\pgfqpoint{3.688351in}{0.641693in}}%
\pgfpathlineto{\pgfqpoint{3.968076in}{1.994042in}}%
\pgfpathlineto{\pgfqpoint{3.688351in}{0.641693in}}%
\pgfpathclose%
\pgfusepath{fill}%
\end{pgfscope}%
\begin{pgfscope}%
\pgfpathrectangle{\pgfqpoint{0.017500in}{0.017500in}}{\pgfqpoint{6.218282in}{3.580372in}} %
\pgfusepath{clip}%
\pgfsetbuttcap%
\pgfsetroundjoin%
\definecolor{currentfill}{rgb}{0.895885,0.433075,0.338681}%
\pgfsetfillcolor{currentfill}%
\pgfsetlinewidth{0.000000pt}%
\definecolor{currentstroke}{rgb}{0.000000,0.000000,0.000000}%
\pgfsetstrokecolor{currentstroke}%
\pgfsetdash{}{0pt}%
\pgfpathmoveto{\pgfqpoint{3.043326in}{2.141782in}}%
\pgfpathlineto{\pgfqpoint{3.306485in}{2.382499in}}%
\pgfpathlineto{\pgfqpoint{3.043326in}{2.141782in}}%
\pgfpathclose%
\pgfusepath{fill}%
\end{pgfscope}%
\begin{pgfscope}%
\pgfpathrectangle{\pgfqpoint{0.017500in}{0.017500in}}{\pgfqpoint{6.218282in}{3.580372in}} %
\pgfusepath{clip}%
\pgfsetbuttcap%
\pgfsetroundjoin%
\definecolor{currentfill}{rgb}{0.895885,0.433075,0.338681}%
\pgfsetfillcolor{currentfill}%
\pgfsetlinewidth{0.000000pt}%
\definecolor{currentstroke}{rgb}{0.000000,0.000000,0.000000}%
\pgfsetstrokecolor{currentstroke}%
\pgfsetdash{}{0pt}%
\pgfpathmoveto{\pgfqpoint{3.043326in}{2.141782in}}%
\pgfpathlineto{\pgfqpoint{3.306485in}{2.382499in}}%
\pgfpathlineto{\pgfqpoint{3.043326in}{2.141782in}}%
\pgfpathclose%
\pgfusepath{fill}%
\end{pgfscope}%
\begin{pgfscope}%
\pgfpathrectangle{\pgfqpoint{0.017500in}{0.017500in}}{\pgfqpoint{6.218282in}{3.580372in}} %
\pgfusepath{clip}%
\pgfsetbuttcap%
\pgfsetroundjoin%
\definecolor{currentfill}{rgb}{0.895885,0.433075,0.338681}%
\pgfsetfillcolor{currentfill}%
\pgfsetlinewidth{0.000000pt}%
\definecolor{currentstroke}{rgb}{0.000000,0.000000,0.000000}%
\pgfsetstrokecolor{currentstroke}%
\pgfsetdash{}{0pt}%
\pgfpathmoveto{\pgfqpoint{3.043326in}{2.141782in}}%
\pgfpathlineto{\pgfqpoint{3.306485in}{2.382499in}}%
\pgfpathlineto{\pgfqpoint{3.043326in}{2.141782in}}%
\pgfpathclose%
\pgfusepath{fill}%
\end{pgfscope}%
\begin{pgfscope}%
\pgfpathrectangle{\pgfqpoint{0.017500in}{0.017500in}}{\pgfqpoint{6.218282in}{3.580372in}} %
\pgfusepath{clip}%
\pgfsetbuttcap%
\pgfsetroundjoin%
\definecolor{currentfill}{rgb}{0.895885,0.433075,0.338681}%
\pgfsetfillcolor{currentfill}%
\pgfsetlinewidth{0.000000pt}%
\definecolor{currentstroke}{rgb}{0.000000,0.000000,0.000000}%
\pgfsetstrokecolor{currentstroke}%
\pgfsetdash{}{0pt}%
\pgfpathmoveto{\pgfqpoint{3.043326in}{2.141782in}}%
\pgfpathlineto{\pgfqpoint{3.306485in}{2.382499in}}%
\pgfpathlineto{\pgfqpoint{3.043326in}{2.141782in}}%
\pgfpathclose%
\pgfusepath{fill}%
\end{pgfscope}%
\begin{pgfscope}%
\pgfpathrectangle{\pgfqpoint{0.017500in}{0.017500in}}{\pgfqpoint{6.218282in}{3.580372in}} %
\pgfusepath{clip}%
\pgfsetbuttcap%
\pgfsetroundjoin%
\definecolor{currentfill}{rgb}{0.895885,0.433075,0.338681}%
\pgfsetfillcolor{currentfill}%
\pgfsetlinewidth{0.000000pt}%
\definecolor{currentstroke}{rgb}{0.000000,0.000000,0.000000}%
\pgfsetstrokecolor{currentstroke}%
\pgfsetdash{}{0pt}%
\pgfpathmoveto{\pgfqpoint{3.043326in}{2.141782in}}%
\pgfpathlineto{\pgfqpoint{3.306485in}{2.382499in}}%
\pgfpathlineto{\pgfqpoint{3.043326in}{2.141782in}}%
\pgfpathclose%
\pgfusepath{fill}%
\end{pgfscope}%
\begin{pgfscope}%
\pgfpathrectangle{\pgfqpoint{0.017500in}{0.017500in}}{\pgfqpoint{6.218282in}{3.580372in}} %
\pgfusepath{clip}%
\pgfsetbuttcap%
\pgfsetroundjoin%
\definecolor{currentfill}{rgb}{0.895885,0.433075,0.338681}%
\pgfsetfillcolor{currentfill}%
\pgfsetlinewidth{0.000000pt}%
\definecolor{currentstroke}{rgb}{0.000000,0.000000,0.000000}%
\pgfsetstrokecolor{currentstroke}%
\pgfsetdash{}{0pt}%
\pgfpathmoveto{\pgfqpoint{3.043326in}{2.141782in}}%
\pgfpathlineto{\pgfqpoint{3.306485in}{2.382499in}}%
\pgfpathlineto{\pgfqpoint{3.043326in}{2.141782in}}%
\pgfpathclose%
\pgfusepath{fill}%
\end{pgfscope}%
\begin{pgfscope}%
\pgfpathrectangle{\pgfqpoint{0.017500in}{0.017500in}}{\pgfqpoint{6.218282in}{3.580372in}} %
\pgfusepath{clip}%
\pgfsetbuttcap%
\pgfsetroundjoin%
\definecolor{currentfill}{rgb}{0.895885,0.433075,0.338681}%
\pgfsetfillcolor{currentfill}%
\pgfsetlinewidth{0.000000pt}%
\definecolor{currentstroke}{rgb}{0.000000,0.000000,0.000000}%
\pgfsetstrokecolor{currentstroke}%
\pgfsetdash{}{0pt}%
\pgfpathmoveto{\pgfqpoint{3.043326in}{2.141782in}}%
\pgfpathlineto{\pgfqpoint{3.306485in}{2.382499in}}%
\pgfpathlineto{\pgfqpoint{3.043326in}{2.141782in}}%
\pgfpathclose%
\pgfusepath{fill}%
\end{pgfscope}%
\begin{pgfscope}%
\pgfpathrectangle{\pgfqpoint{0.017500in}{0.017500in}}{\pgfqpoint{6.218282in}{3.580372in}} %
\pgfusepath{clip}%
\pgfsetbuttcap%
\pgfsetroundjoin%
\definecolor{currentfill}{rgb}{0.895885,0.433075,0.338681}%
\pgfsetfillcolor{currentfill}%
\pgfsetlinewidth{0.000000pt}%
\definecolor{currentstroke}{rgb}{0.000000,0.000000,0.000000}%
\pgfsetstrokecolor{currentstroke}%
\pgfsetdash{}{0pt}%
\pgfpathmoveto{\pgfqpoint{3.043326in}{2.141782in}}%
\pgfpathlineto{\pgfqpoint{3.306485in}{2.382499in}}%
\pgfpathlineto{\pgfqpoint{3.043326in}{2.141782in}}%
\pgfpathclose%
\pgfusepath{fill}%
\end{pgfscope}%
\begin{pgfscope}%
\pgfpathrectangle{\pgfqpoint{0.017500in}{0.017500in}}{\pgfqpoint{6.218282in}{3.580372in}} %
\pgfusepath{clip}%
\pgfsetbuttcap%
\pgfsetroundjoin%
\definecolor{currentfill}{rgb}{0.895885,0.433075,0.338681}%
\pgfsetfillcolor{currentfill}%
\pgfsetlinewidth{0.000000pt}%
\definecolor{currentstroke}{rgb}{0.000000,0.000000,0.000000}%
\pgfsetstrokecolor{currentstroke}%
\pgfsetdash{}{0pt}%
\pgfpathmoveto{\pgfqpoint{3.043326in}{2.141782in}}%
\pgfpathlineto{\pgfqpoint{3.306485in}{2.382499in}}%
\pgfpathlineto{\pgfqpoint{3.043326in}{2.141782in}}%
\pgfpathclose%
\pgfusepath{fill}%
\end{pgfscope}%
\begin{pgfscope}%
\pgfpathrectangle{\pgfqpoint{0.017500in}{0.017500in}}{\pgfqpoint{6.218282in}{3.580372in}} %
\pgfusepath{clip}%
\pgfsetbuttcap%
\pgfsetroundjoin%
\definecolor{currentfill}{rgb}{0.895885,0.433075,0.338681}%
\pgfsetfillcolor{currentfill}%
\pgfsetlinewidth{0.000000pt}%
\definecolor{currentstroke}{rgb}{0.000000,0.000000,0.000000}%
\pgfsetstrokecolor{currentstroke}%
\pgfsetdash{}{0pt}%
\pgfpathmoveto{\pgfqpoint{3.043326in}{2.141782in}}%
\pgfpathlineto{\pgfqpoint{3.306485in}{2.382499in}}%
\pgfpathlineto{\pgfqpoint{3.043326in}{2.141782in}}%
\pgfpathclose%
\pgfusepath{fill}%
\end{pgfscope}%
\begin{pgfscope}%
\pgfpathrectangle{\pgfqpoint{0.017500in}{0.017500in}}{\pgfqpoint{6.218282in}{3.580372in}} %
\pgfusepath{clip}%
\pgfsetbuttcap%
\pgfsetroundjoin%
\definecolor{currentfill}{rgb}{0.895885,0.433075,0.338681}%
\pgfsetfillcolor{currentfill}%
\pgfsetlinewidth{0.000000pt}%
\definecolor{currentstroke}{rgb}{0.000000,0.000000,0.000000}%
\pgfsetstrokecolor{currentstroke}%
\pgfsetdash{}{0pt}%
\pgfpathmoveto{\pgfqpoint{3.043326in}{2.141782in}}%
\pgfpathlineto{\pgfqpoint{3.306485in}{2.382499in}}%
\pgfpathlineto{\pgfqpoint{3.043326in}{2.141782in}}%
\pgfpathclose%
\pgfusepath{fill}%
\end{pgfscope}%
\begin{pgfscope}%
\pgfpathrectangle{\pgfqpoint{0.017500in}{0.017500in}}{\pgfqpoint{6.218282in}{3.580372in}} %
\pgfusepath{clip}%
\pgfsetbuttcap%
\pgfsetroundjoin%
\definecolor{currentfill}{rgb}{0.895885,0.433075,0.338681}%
\pgfsetfillcolor{currentfill}%
\pgfsetlinewidth{0.000000pt}%
\definecolor{currentstroke}{rgb}{0.000000,0.000000,0.000000}%
\pgfsetstrokecolor{currentstroke}%
\pgfsetdash{}{0pt}%
\pgfpathmoveto{\pgfqpoint{3.043326in}{2.141782in}}%
\pgfpathlineto{\pgfqpoint{3.306485in}{2.382499in}}%
\pgfpathlineto{\pgfqpoint{3.043326in}{2.141782in}}%
\pgfpathclose%
\pgfusepath{fill}%
\end{pgfscope}%
\begin{pgfscope}%
\pgfpathrectangle{\pgfqpoint{0.017500in}{0.017500in}}{\pgfqpoint{6.218282in}{3.580372in}} %
\pgfusepath{clip}%
\pgfsetbuttcap%
\pgfsetroundjoin%
\definecolor{currentfill}{rgb}{0.895885,0.433075,0.338681}%
\pgfsetfillcolor{currentfill}%
\pgfsetlinewidth{0.000000pt}%
\definecolor{currentstroke}{rgb}{0.000000,0.000000,0.000000}%
\pgfsetstrokecolor{currentstroke}%
\pgfsetdash{}{0pt}%
\pgfpathmoveto{\pgfqpoint{3.043326in}{2.141782in}}%
\pgfpathlineto{\pgfqpoint{3.306485in}{2.382499in}}%
\pgfpathlineto{\pgfqpoint{3.043326in}{2.141782in}}%
\pgfpathclose%
\pgfusepath{fill}%
\end{pgfscope}%
\begin{pgfscope}%
\pgfpathrectangle{\pgfqpoint{0.017500in}{0.017500in}}{\pgfqpoint{6.218282in}{3.580372in}} %
\pgfusepath{clip}%
\pgfsetbuttcap%
\pgfsetroundjoin%
\definecolor{currentfill}{rgb}{0.895885,0.433075,0.338681}%
\pgfsetfillcolor{currentfill}%
\pgfsetlinewidth{0.000000pt}%
\definecolor{currentstroke}{rgb}{0.000000,0.000000,0.000000}%
\pgfsetstrokecolor{currentstroke}%
\pgfsetdash{}{0pt}%
\pgfpathmoveto{\pgfqpoint{3.043326in}{2.141782in}}%
\pgfpathlineto{\pgfqpoint{3.306485in}{2.382499in}}%
\pgfpathlineto{\pgfqpoint{3.043326in}{2.141782in}}%
\pgfpathclose%
\pgfusepath{fill}%
\end{pgfscope}%
\begin{pgfscope}%
\pgfpathrectangle{\pgfqpoint{0.017500in}{0.017500in}}{\pgfqpoint{6.218282in}{3.580372in}} %
\pgfusepath{clip}%
\pgfsetbuttcap%
\pgfsetroundjoin%
\definecolor{currentfill}{rgb}{0.895885,0.433075,0.338681}%
\pgfsetfillcolor{currentfill}%
\pgfsetlinewidth{0.000000pt}%
\definecolor{currentstroke}{rgb}{0.000000,0.000000,0.000000}%
\pgfsetstrokecolor{currentstroke}%
\pgfsetdash{}{0pt}%
\pgfpathmoveto{\pgfqpoint{3.043326in}{2.141782in}}%
\pgfpathlineto{\pgfqpoint{3.306485in}{2.382499in}}%
\pgfpathlineto{\pgfqpoint{3.043326in}{2.141782in}}%
\pgfpathclose%
\pgfusepath{fill}%
\end{pgfscope}%
\begin{pgfscope}%
\pgfpathrectangle{\pgfqpoint{0.017500in}{0.017500in}}{\pgfqpoint{6.218282in}{3.580372in}} %
\pgfusepath{clip}%
\pgfsetbuttcap%
\pgfsetroundjoin%
\definecolor{currentfill}{rgb}{0.895885,0.433075,0.338681}%
\pgfsetfillcolor{currentfill}%
\pgfsetlinewidth{0.000000pt}%
\definecolor{currentstroke}{rgb}{0.000000,0.000000,0.000000}%
\pgfsetstrokecolor{currentstroke}%
\pgfsetdash{}{0pt}%
\pgfpathmoveto{\pgfqpoint{3.043326in}{2.141782in}}%
\pgfpathlineto{\pgfqpoint{3.306485in}{2.382499in}}%
\pgfpathlineto{\pgfqpoint{3.043326in}{2.141782in}}%
\pgfpathclose%
\pgfusepath{fill}%
\end{pgfscope}%
\begin{pgfscope}%
\pgfpathrectangle{\pgfqpoint{0.017500in}{0.017500in}}{\pgfqpoint{6.218282in}{3.580372in}} %
\pgfusepath{clip}%
\pgfsetbuttcap%
\pgfsetroundjoin%
\definecolor{currentfill}{rgb}{0.895885,0.433075,0.338681}%
\pgfsetfillcolor{currentfill}%
\pgfsetlinewidth{0.000000pt}%
\definecolor{currentstroke}{rgb}{0.000000,0.000000,0.000000}%
\pgfsetstrokecolor{currentstroke}%
\pgfsetdash{}{0pt}%
\pgfpathmoveto{\pgfqpoint{3.043326in}{2.141782in}}%
\pgfpathlineto{\pgfqpoint{3.306485in}{2.382499in}}%
\pgfpathlineto{\pgfqpoint{3.043326in}{2.141782in}}%
\pgfpathclose%
\pgfusepath{fill}%
\end{pgfscope}%
\begin{pgfscope}%
\pgfpathrectangle{\pgfqpoint{0.017500in}{0.017500in}}{\pgfqpoint{6.218282in}{3.580372in}} %
\pgfusepath{clip}%
\pgfsetbuttcap%
\pgfsetroundjoin%
\definecolor{currentfill}{rgb}{0.895885,0.433075,0.338681}%
\pgfsetfillcolor{currentfill}%
\pgfsetlinewidth{0.000000pt}%
\definecolor{currentstroke}{rgb}{0.000000,0.000000,0.000000}%
\pgfsetstrokecolor{currentstroke}%
\pgfsetdash{}{0pt}%
\pgfpathmoveto{\pgfqpoint{3.043326in}{2.141782in}}%
\pgfpathlineto{\pgfqpoint{3.306485in}{2.382499in}}%
\pgfpathlineto{\pgfqpoint{3.043326in}{2.141782in}}%
\pgfpathclose%
\pgfusepath{fill}%
\end{pgfscope}%
\begin{pgfscope}%
\pgfpathrectangle{\pgfqpoint{0.017500in}{0.017500in}}{\pgfqpoint{6.218282in}{3.580372in}} %
\pgfusepath{clip}%
\pgfsetbuttcap%
\pgfsetroundjoin%
\definecolor{currentfill}{rgb}{0.895885,0.433075,0.338681}%
\pgfsetfillcolor{currentfill}%
\pgfsetlinewidth{0.000000pt}%
\definecolor{currentstroke}{rgb}{0.000000,0.000000,0.000000}%
\pgfsetstrokecolor{currentstroke}%
\pgfsetdash{}{0pt}%
\pgfpathmoveto{\pgfqpoint{3.043326in}{2.141782in}}%
\pgfpathlineto{\pgfqpoint{3.306485in}{2.382499in}}%
\pgfpathlineto{\pgfqpoint{3.043326in}{2.141782in}}%
\pgfpathclose%
\pgfusepath{fill}%
\end{pgfscope}%
\begin{pgfscope}%
\pgfpathrectangle{\pgfqpoint{0.017500in}{0.017500in}}{\pgfqpoint{6.218282in}{3.580372in}} %
\pgfusepath{clip}%
\pgfsetbuttcap%
\pgfsetroundjoin%
\definecolor{currentfill}{rgb}{0.895885,0.433075,0.338681}%
\pgfsetfillcolor{currentfill}%
\pgfsetlinewidth{0.000000pt}%
\definecolor{currentstroke}{rgb}{0.000000,0.000000,0.000000}%
\pgfsetstrokecolor{currentstroke}%
\pgfsetdash{}{0pt}%
\pgfpathmoveto{\pgfqpoint{3.043326in}{2.141782in}}%
\pgfpathlineto{\pgfqpoint{3.306485in}{2.382499in}}%
\pgfpathlineto{\pgfqpoint{3.043326in}{2.141782in}}%
\pgfpathclose%
\pgfusepath{fill}%
\end{pgfscope}%
\begin{pgfscope}%
\pgfpathrectangle{\pgfqpoint{0.017500in}{0.017500in}}{\pgfqpoint{6.218282in}{3.580372in}} %
\pgfusepath{clip}%
\pgfsetbuttcap%
\pgfsetroundjoin%
\definecolor{currentfill}{rgb}{0.895885,0.433075,0.338681}%
\pgfsetfillcolor{currentfill}%
\pgfsetlinewidth{0.000000pt}%
\definecolor{currentstroke}{rgb}{0.000000,0.000000,0.000000}%
\pgfsetstrokecolor{currentstroke}%
\pgfsetdash{}{0pt}%
\pgfpathmoveto{\pgfqpoint{3.043326in}{2.141782in}}%
\pgfpathlineto{\pgfqpoint{3.306485in}{2.382499in}}%
\pgfpathlineto{\pgfqpoint{3.043326in}{2.141782in}}%
\pgfpathclose%
\pgfusepath{fill}%
\end{pgfscope}%
\begin{pgfscope}%
\pgfpathrectangle{\pgfqpoint{0.017500in}{0.017500in}}{\pgfqpoint{6.218282in}{3.580372in}} %
\pgfusepath{clip}%
\pgfsetbuttcap%
\pgfsetroundjoin%
\definecolor{currentfill}{rgb}{0.895885,0.433075,0.338681}%
\pgfsetfillcolor{currentfill}%
\pgfsetlinewidth{0.000000pt}%
\definecolor{currentstroke}{rgb}{0.000000,0.000000,0.000000}%
\pgfsetstrokecolor{currentstroke}%
\pgfsetdash{}{0pt}%
\pgfpathmoveto{\pgfqpoint{3.043326in}{2.141782in}}%
\pgfpathlineto{\pgfqpoint{3.306485in}{2.382499in}}%
\pgfpathlineto{\pgfqpoint{3.043326in}{2.141782in}}%
\pgfpathclose%
\pgfusepath{fill}%
\end{pgfscope}%
\begin{pgfscope}%
\pgfpathrectangle{\pgfqpoint{0.017500in}{0.017500in}}{\pgfqpoint{6.218282in}{3.580372in}} %
\pgfusepath{clip}%
\pgfsetbuttcap%
\pgfsetroundjoin%
\definecolor{currentfill}{rgb}{0.895885,0.433075,0.338681}%
\pgfsetfillcolor{currentfill}%
\pgfsetlinewidth{0.000000pt}%
\definecolor{currentstroke}{rgb}{0.000000,0.000000,0.000000}%
\pgfsetstrokecolor{currentstroke}%
\pgfsetdash{}{0pt}%
\pgfpathmoveto{\pgfqpoint{3.043326in}{2.141782in}}%
\pgfpathlineto{\pgfqpoint{3.306485in}{2.382499in}}%
\pgfpathlineto{\pgfqpoint{3.043326in}{2.141782in}}%
\pgfpathclose%
\pgfusepath{fill}%
\end{pgfscope}%
\begin{pgfscope}%
\pgfpathrectangle{\pgfqpoint{0.017500in}{0.017500in}}{\pgfqpoint{6.218282in}{3.580372in}} %
\pgfusepath{clip}%
\pgfsetbuttcap%
\pgfsetroundjoin%
\definecolor{currentfill}{rgb}{0.895885,0.433075,0.338681}%
\pgfsetfillcolor{currentfill}%
\pgfsetlinewidth{0.000000pt}%
\definecolor{currentstroke}{rgb}{0.000000,0.000000,0.000000}%
\pgfsetstrokecolor{currentstroke}%
\pgfsetdash{}{0pt}%
\pgfpathmoveto{\pgfqpoint{3.043326in}{2.141782in}}%
\pgfpathlineto{\pgfqpoint{3.306485in}{2.382499in}}%
\pgfpathlineto{\pgfqpoint{3.043326in}{2.141782in}}%
\pgfpathclose%
\pgfusepath{fill}%
\end{pgfscope}%
\begin{pgfscope}%
\pgfpathrectangle{\pgfqpoint{0.017500in}{0.017500in}}{\pgfqpoint{6.218282in}{3.580372in}} %
\pgfusepath{clip}%
\pgfsetbuttcap%
\pgfsetroundjoin%
\definecolor{currentfill}{rgb}{0.895885,0.433075,0.338681}%
\pgfsetfillcolor{currentfill}%
\pgfsetlinewidth{0.000000pt}%
\definecolor{currentstroke}{rgb}{0.000000,0.000000,0.000000}%
\pgfsetstrokecolor{currentstroke}%
\pgfsetdash{}{0pt}%
\pgfpathmoveto{\pgfqpoint{3.043326in}{2.141782in}}%
\pgfpathlineto{\pgfqpoint{3.306485in}{2.382499in}}%
\pgfpathlineto{\pgfqpoint{3.043326in}{2.141782in}}%
\pgfpathclose%
\pgfusepath{fill}%
\end{pgfscope}%
\begin{pgfscope}%
\pgfpathrectangle{\pgfqpoint{0.017500in}{0.017500in}}{\pgfqpoint{6.218282in}{3.580372in}} %
\pgfusepath{clip}%
\pgfsetbuttcap%
\pgfsetroundjoin%
\definecolor{currentfill}{rgb}{0.895885,0.433075,0.338681}%
\pgfsetfillcolor{currentfill}%
\pgfsetlinewidth{0.000000pt}%
\definecolor{currentstroke}{rgb}{0.000000,0.000000,0.000000}%
\pgfsetstrokecolor{currentstroke}%
\pgfsetdash{}{0pt}%
\pgfpathmoveto{\pgfqpoint{3.043326in}{2.141782in}}%
\pgfpathlineto{\pgfqpoint{3.306485in}{2.382499in}}%
\pgfpathlineto{\pgfqpoint{3.043326in}{2.141782in}}%
\pgfpathclose%
\pgfusepath{fill}%
\end{pgfscope}%
\begin{pgfscope}%
\pgfpathrectangle{\pgfqpoint{0.017500in}{0.017500in}}{\pgfqpoint{6.218282in}{3.580372in}} %
\pgfusepath{clip}%
\pgfsetbuttcap%
\pgfsetroundjoin%
\definecolor{currentfill}{rgb}{0.895885,0.433075,0.338681}%
\pgfsetfillcolor{currentfill}%
\pgfsetlinewidth{0.000000pt}%
\definecolor{currentstroke}{rgb}{0.000000,0.000000,0.000000}%
\pgfsetstrokecolor{currentstroke}%
\pgfsetdash{}{0pt}%
\pgfpathmoveto{\pgfqpoint{3.043326in}{2.141782in}}%
\pgfpathlineto{\pgfqpoint{3.306485in}{2.382499in}}%
\pgfpathlineto{\pgfqpoint{3.043326in}{2.141782in}}%
\pgfpathclose%
\pgfusepath{fill}%
\end{pgfscope}%
\begin{pgfscope}%
\pgfpathrectangle{\pgfqpoint{0.017500in}{0.017500in}}{\pgfqpoint{6.218282in}{3.580372in}} %
\pgfusepath{clip}%
\pgfsetbuttcap%
\pgfsetroundjoin%
\definecolor{currentfill}{rgb}{0.895885,0.433075,0.338681}%
\pgfsetfillcolor{currentfill}%
\pgfsetlinewidth{0.000000pt}%
\definecolor{currentstroke}{rgb}{0.000000,0.000000,0.000000}%
\pgfsetstrokecolor{currentstroke}%
\pgfsetdash{}{0pt}%
\pgfpathmoveto{\pgfqpoint{3.043326in}{2.141782in}}%
\pgfpathlineto{\pgfqpoint{3.306485in}{2.382499in}}%
\pgfpathlineto{\pgfqpoint{3.043326in}{2.141782in}}%
\pgfpathclose%
\pgfusepath{fill}%
\end{pgfscope}%
\begin{pgfscope}%
\pgfpathrectangle{\pgfqpoint{0.017500in}{0.017500in}}{\pgfqpoint{6.218282in}{3.580372in}} %
\pgfusepath{clip}%
\pgfsetbuttcap%
\pgfsetroundjoin%
\definecolor{currentfill}{rgb}{0.895885,0.433075,0.338681}%
\pgfsetfillcolor{currentfill}%
\pgfsetlinewidth{0.000000pt}%
\definecolor{currentstroke}{rgb}{0.000000,0.000000,0.000000}%
\pgfsetstrokecolor{currentstroke}%
\pgfsetdash{}{0pt}%
\pgfpathmoveto{\pgfqpoint{3.043326in}{2.141782in}}%
\pgfpathlineto{\pgfqpoint{3.306485in}{2.382499in}}%
\pgfpathlineto{\pgfqpoint{3.043326in}{2.141782in}}%
\pgfpathclose%
\pgfusepath{fill}%
\end{pgfscope}%
\begin{pgfscope}%
\pgfpathrectangle{\pgfqpoint{0.017500in}{0.017500in}}{\pgfqpoint{6.218282in}{3.580372in}} %
\pgfusepath{clip}%
\pgfsetbuttcap%
\pgfsetroundjoin%
\definecolor{currentfill}{rgb}{0.895885,0.433075,0.338681}%
\pgfsetfillcolor{currentfill}%
\pgfsetlinewidth{0.000000pt}%
\definecolor{currentstroke}{rgb}{0.000000,0.000000,0.000000}%
\pgfsetstrokecolor{currentstroke}%
\pgfsetdash{}{0pt}%
\pgfpathmoveto{\pgfqpoint{3.043326in}{2.141782in}}%
\pgfpathlineto{\pgfqpoint{3.306485in}{2.382499in}}%
\pgfpathlineto{\pgfqpoint{3.043326in}{2.141782in}}%
\pgfpathclose%
\pgfusepath{fill}%
\end{pgfscope}%
\begin{pgfscope}%
\pgfpathrectangle{\pgfqpoint{0.017500in}{0.017500in}}{\pgfqpoint{6.218282in}{3.580372in}} %
\pgfusepath{clip}%
\pgfsetbuttcap%
\pgfsetroundjoin%
\definecolor{currentfill}{rgb}{0.888390,0.417703,0.327898}%
\pgfsetfillcolor{currentfill}%
\pgfsetlinewidth{0.000000pt}%
\definecolor{currentstroke}{rgb}{0.000000,0.000000,0.000000}%
\pgfsetstrokecolor{currentstroke}%
\pgfsetdash{}{0pt}%
\pgfpathmoveto{\pgfqpoint{3.043326in}{2.141782in}}%
\pgfpathlineto{\pgfqpoint{3.502561in}{2.073396in}}%
\pgfpathlineto{\pgfqpoint{3.763398in}{2.297457in}}%
\pgfpathlineto{\pgfqpoint{3.306485in}{2.382499in}}%
\pgfpathlineto{\pgfqpoint{3.043326in}{2.141782in}}%
\pgfpathclose%
\pgfusepath{fill}%
\end{pgfscope}%
\begin{pgfscope}%
\pgfpathrectangle{\pgfqpoint{0.017500in}{0.017500in}}{\pgfqpoint{6.218282in}{3.580372in}} %
\pgfusepath{clip}%
\pgfsetbuttcap%
\pgfsetroundjoin%
\definecolor{currentfill}{rgb}{0.888390,0.417703,0.327898}%
\pgfsetfillcolor{currentfill}%
\pgfsetlinewidth{0.000000pt}%
\definecolor{currentstroke}{rgb}{0.000000,0.000000,0.000000}%
\pgfsetstrokecolor{currentstroke}%
\pgfsetdash{}{0pt}%
\pgfpathmoveto{\pgfqpoint{3.043326in}{2.141782in}}%
\pgfpathlineto{\pgfqpoint{3.502561in}{2.073396in}}%
\pgfpathlineto{\pgfqpoint{3.763398in}{2.297457in}}%
\pgfpathlineto{\pgfqpoint{3.306485in}{2.382499in}}%
\pgfpathlineto{\pgfqpoint{3.043326in}{2.141782in}}%
\pgfpathclose%
\pgfusepath{fill}%
\end{pgfscope}%
\begin{pgfscope}%
\pgfpathrectangle{\pgfqpoint{0.017500in}{0.017500in}}{\pgfqpoint{6.218282in}{3.580372in}} %
\pgfusepath{clip}%
\pgfsetbuttcap%
\pgfsetroundjoin%
\definecolor{currentfill}{rgb}{0.888390,0.417703,0.327898}%
\pgfsetfillcolor{currentfill}%
\pgfsetlinewidth{0.000000pt}%
\definecolor{currentstroke}{rgb}{0.000000,0.000000,0.000000}%
\pgfsetstrokecolor{currentstroke}%
\pgfsetdash{}{0pt}%
\pgfpathmoveto{\pgfqpoint{3.043326in}{2.141782in}}%
\pgfpathlineto{\pgfqpoint{3.502561in}{2.073396in}}%
\pgfpathlineto{\pgfqpoint{3.763398in}{2.297457in}}%
\pgfpathlineto{\pgfqpoint{3.306485in}{2.382499in}}%
\pgfpathlineto{\pgfqpoint{3.043326in}{2.141782in}}%
\pgfpathclose%
\pgfusepath{fill}%
\end{pgfscope}%
\begin{pgfscope}%
\pgfpathrectangle{\pgfqpoint{0.017500in}{0.017500in}}{\pgfqpoint{6.218282in}{3.580372in}} %
\pgfusepath{clip}%
\pgfsetbuttcap%
\pgfsetroundjoin%
\definecolor{currentfill}{rgb}{0.888390,0.417703,0.327898}%
\pgfsetfillcolor{currentfill}%
\pgfsetlinewidth{0.000000pt}%
\definecolor{currentstroke}{rgb}{0.000000,0.000000,0.000000}%
\pgfsetstrokecolor{currentstroke}%
\pgfsetdash{}{0pt}%
\pgfpathmoveto{\pgfqpoint{3.043326in}{2.141782in}}%
\pgfpathlineto{\pgfqpoint{3.502561in}{2.073396in}}%
\pgfpathlineto{\pgfqpoint{3.763398in}{2.297457in}}%
\pgfpathlineto{\pgfqpoint{3.306485in}{2.382499in}}%
\pgfpathlineto{\pgfqpoint{3.043326in}{2.141782in}}%
\pgfpathclose%
\pgfusepath{fill}%
\end{pgfscope}%
\begin{pgfscope}%
\pgfpathrectangle{\pgfqpoint{0.017500in}{0.017500in}}{\pgfqpoint{6.218282in}{3.580372in}} %
\pgfusepath{clip}%
\pgfsetbuttcap%
\pgfsetroundjoin%
\definecolor{currentfill}{rgb}{0.888390,0.417703,0.327898}%
\pgfsetfillcolor{currentfill}%
\pgfsetlinewidth{0.000000pt}%
\definecolor{currentstroke}{rgb}{0.000000,0.000000,0.000000}%
\pgfsetstrokecolor{currentstroke}%
\pgfsetdash{}{0pt}%
\pgfpathmoveto{\pgfqpoint{3.043326in}{2.141782in}}%
\pgfpathlineto{\pgfqpoint{3.502561in}{2.073396in}}%
\pgfpathlineto{\pgfqpoint{3.763398in}{2.297457in}}%
\pgfpathlineto{\pgfqpoint{3.306485in}{2.382499in}}%
\pgfpathlineto{\pgfqpoint{3.043326in}{2.141782in}}%
\pgfpathclose%
\pgfusepath{fill}%
\end{pgfscope}%
\begin{pgfscope}%
\pgfpathrectangle{\pgfqpoint{0.017500in}{0.017500in}}{\pgfqpoint{6.218282in}{3.580372in}} %
\pgfusepath{clip}%
\pgfsetbuttcap%
\pgfsetroundjoin%
\definecolor{currentfill}{rgb}{0.888390,0.417703,0.327898}%
\pgfsetfillcolor{currentfill}%
\pgfsetlinewidth{0.000000pt}%
\definecolor{currentstroke}{rgb}{0.000000,0.000000,0.000000}%
\pgfsetstrokecolor{currentstroke}%
\pgfsetdash{}{0pt}%
\pgfpathmoveto{\pgfqpoint{3.043326in}{2.141782in}}%
\pgfpathlineto{\pgfqpoint{3.502561in}{2.073396in}}%
\pgfpathlineto{\pgfqpoint{3.763398in}{2.297457in}}%
\pgfpathlineto{\pgfqpoint{3.306485in}{2.382499in}}%
\pgfpathlineto{\pgfqpoint{3.043326in}{2.141782in}}%
\pgfpathclose%
\pgfusepath{fill}%
\end{pgfscope}%
\begin{pgfscope}%
\pgfpathrectangle{\pgfqpoint{0.017500in}{0.017500in}}{\pgfqpoint{6.218282in}{3.580372in}} %
\pgfusepath{clip}%
\pgfsetbuttcap%
\pgfsetroundjoin%
\definecolor{currentfill}{rgb}{0.800830,0.250829,0.225696}%
\pgfsetfillcolor{currentfill}%
\pgfsetlinewidth{0.000000pt}%
\definecolor{currentstroke}{rgb}{0.000000,0.000000,0.000000}%
\pgfsetstrokecolor{currentstroke}%
\pgfsetdash{}{0pt}%
\pgfpathmoveto{\pgfqpoint{4.226975in}{2.236848in}}%
\pgfpathlineto{\pgfqpoint{4.478468in}{2.393972in}}%
\pgfpathlineto{\pgfqpoint{4.226975in}{2.236848in}}%
\pgfpathclose%
\pgfusepath{fill}%
\end{pgfscope}%
\begin{pgfscope}%
\pgfpathrectangle{\pgfqpoint{0.017500in}{0.017500in}}{\pgfqpoint{6.218282in}{3.580372in}} %
\pgfusepath{clip}%
\pgfsetbuttcap%
\pgfsetroundjoin%
\definecolor{currentfill}{rgb}{0.800830,0.250829,0.225696}%
\pgfsetfillcolor{currentfill}%
\pgfsetlinewidth{0.000000pt}%
\definecolor{currentstroke}{rgb}{0.000000,0.000000,0.000000}%
\pgfsetstrokecolor{currentstroke}%
\pgfsetdash{}{0pt}%
\pgfpathmoveto{\pgfqpoint{4.226975in}{2.236848in}}%
\pgfpathlineto{\pgfqpoint{4.478468in}{2.393972in}}%
\pgfpathlineto{\pgfqpoint{4.226975in}{2.236848in}}%
\pgfpathclose%
\pgfusepath{fill}%
\end{pgfscope}%
\begin{pgfscope}%
\pgfpathrectangle{\pgfqpoint{0.017500in}{0.017500in}}{\pgfqpoint{6.218282in}{3.580372in}} %
\pgfusepath{clip}%
\pgfsetbuttcap%
\pgfsetroundjoin%
\definecolor{currentfill}{rgb}{0.800830,0.250829,0.225696}%
\pgfsetfillcolor{currentfill}%
\pgfsetlinewidth{0.000000pt}%
\definecolor{currentstroke}{rgb}{0.000000,0.000000,0.000000}%
\pgfsetstrokecolor{currentstroke}%
\pgfsetdash{}{0pt}%
\pgfpathmoveto{\pgfqpoint{4.226975in}{2.236848in}}%
\pgfpathlineto{\pgfqpoint{4.478468in}{2.393972in}}%
\pgfpathlineto{\pgfqpoint{4.226975in}{2.236848in}}%
\pgfpathclose%
\pgfusepath{fill}%
\end{pgfscope}%
\begin{pgfscope}%
\pgfpathrectangle{\pgfqpoint{0.017500in}{0.017500in}}{\pgfqpoint{6.218282in}{3.580372in}} %
\pgfusepath{clip}%
\pgfsetbuttcap%
\pgfsetroundjoin%
\definecolor{currentfill}{rgb}{0.800830,0.250829,0.225696}%
\pgfsetfillcolor{currentfill}%
\pgfsetlinewidth{0.000000pt}%
\definecolor{currentstroke}{rgb}{0.000000,0.000000,0.000000}%
\pgfsetstrokecolor{currentstroke}%
\pgfsetdash{}{0pt}%
\pgfpathmoveto{\pgfqpoint{4.226975in}{2.236848in}}%
\pgfpathlineto{\pgfqpoint{4.478468in}{2.393972in}}%
\pgfpathlineto{\pgfqpoint{4.226975in}{2.236848in}}%
\pgfpathclose%
\pgfusepath{fill}%
\end{pgfscope}%
\begin{pgfscope}%
\pgfpathrectangle{\pgfqpoint{0.017500in}{0.017500in}}{\pgfqpoint{6.218282in}{3.580372in}} %
\pgfusepath{clip}%
\pgfsetbuttcap%
\pgfsetroundjoin%
\definecolor{currentfill}{rgb}{0.800830,0.250829,0.225696}%
\pgfsetfillcolor{currentfill}%
\pgfsetlinewidth{0.000000pt}%
\definecolor{currentstroke}{rgb}{0.000000,0.000000,0.000000}%
\pgfsetstrokecolor{currentstroke}%
\pgfsetdash{}{0pt}%
\pgfpathmoveto{\pgfqpoint{4.226975in}{2.236848in}}%
\pgfpathlineto{\pgfqpoint{4.478468in}{2.393972in}}%
\pgfpathlineto{\pgfqpoint{4.226975in}{2.236848in}}%
\pgfpathclose%
\pgfusepath{fill}%
\end{pgfscope}%
\begin{pgfscope}%
\pgfpathrectangle{\pgfqpoint{0.017500in}{0.017500in}}{\pgfqpoint{6.218282in}{3.580372in}} %
\pgfusepath{clip}%
\pgfsetbuttcap%
\pgfsetroundjoin%
\definecolor{currentfill}{rgb}{0.800830,0.250829,0.225696}%
\pgfsetfillcolor{currentfill}%
\pgfsetlinewidth{0.000000pt}%
\definecolor{currentstroke}{rgb}{0.000000,0.000000,0.000000}%
\pgfsetstrokecolor{currentstroke}%
\pgfsetdash{}{0pt}%
\pgfpathmoveto{\pgfqpoint{4.226975in}{2.236848in}}%
\pgfpathlineto{\pgfqpoint{4.478468in}{2.393972in}}%
\pgfpathlineto{\pgfqpoint{4.226975in}{2.236848in}}%
\pgfpathclose%
\pgfusepath{fill}%
\end{pgfscope}%
\begin{pgfscope}%
\pgfpathrectangle{\pgfqpoint{0.017500in}{0.017500in}}{\pgfqpoint{6.218282in}{3.580372in}} %
\pgfusepath{clip}%
\pgfsetbuttcap%
\pgfsetroundjoin%
\definecolor{currentfill}{rgb}{0.800830,0.250829,0.225696}%
\pgfsetfillcolor{currentfill}%
\pgfsetlinewidth{0.000000pt}%
\definecolor{currentstroke}{rgb}{0.000000,0.000000,0.000000}%
\pgfsetstrokecolor{currentstroke}%
\pgfsetdash{}{0pt}%
\pgfpathmoveto{\pgfqpoint{4.226975in}{2.236848in}}%
\pgfpathlineto{\pgfqpoint{4.478468in}{2.393972in}}%
\pgfpathlineto{\pgfqpoint{4.226975in}{2.236848in}}%
\pgfpathclose%
\pgfusepath{fill}%
\end{pgfscope}%
\begin{pgfscope}%
\pgfpathrectangle{\pgfqpoint{0.017500in}{0.017500in}}{\pgfqpoint{6.218282in}{3.580372in}} %
\pgfusepath{clip}%
\pgfsetbuttcap%
\pgfsetroundjoin%
\definecolor{currentfill}{rgb}{0.800830,0.250829,0.225696}%
\pgfsetfillcolor{currentfill}%
\pgfsetlinewidth{0.000000pt}%
\definecolor{currentstroke}{rgb}{0.000000,0.000000,0.000000}%
\pgfsetstrokecolor{currentstroke}%
\pgfsetdash{}{0pt}%
\pgfpathmoveto{\pgfqpoint{4.226975in}{2.236848in}}%
\pgfpathlineto{\pgfqpoint{4.478468in}{2.393972in}}%
\pgfpathlineto{\pgfqpoint{4.226975in}{2.236848in}}%
\pgfpathclose%
\pgfusepath{fill}%
\end{pgfscope}%
\begin{pgfscope}%
\pgfpathrectangle{\pgfqpoint{0.017500in}{0.017500in}}{\pgfqpoint{6.218282in}{3.580372in}} %
\pgfusepath{clip}%
\pgfsetbuttcap%
\pgfsetroundjoin%
\definecolor{currentfill}{rgb}{0.800830,0.250829,0.225696}%
\pgfsetfillcolor{currentfill}%
\pgfsetlinewidth{0.000000pt}%
\definecolor{currentstroke}{rgb}{0.000000,0.000000,0.000000}%
\pgfsetstrokecolor{currentstroke}%
\pgfsetdash{}{0pt}%
\pgfpathmoveto{\pgfqpoint{4.226975in}{2.236848in}}%
\pgfpathlineto{\pgfqpoint{4.478468in}{2.393972in}}%
\pgfpathlineto{\pgfqpoint{4.226975in}{2.236848in}}%
\pgfpathclose%
\pgfusepath{fill}%
\end{pgfscope}%
\begin{pgfscope}%
\pgfpathrectangle{\pgfqpoint{0.017500in}{0.017500in}}{\pgfqpoint{6.218282in}{3.580372in}} %
\pgfusepath{clip}%
\pgfsetbuttcap%
\pgfsetroundjoin%
\definecolor{currentfill}{rgb}{0.800830,0.250829,0.225696}%
\pgfsetfillcolor{currentfill}%
\pgfsetlinewidth{0.000000pt}%
\definecolor{currentstroke}{rgb}{0.000000,0.000000,0.000000}%
\pgfsetstrokecolor{currentstroke}%
\pgfsetdash{}{0pt}%
\pgfpathmoveto{\pgfqpoint{4.226975in}{2.236848in}}%
\pgfpathlineto{\pgfqpoint{4.478468in}{2.393972in}}%
\pgfpathlineto{\pgfqpoint{4.226975in}{2.236848in}}%
\pgfpathclose%
\pgfusepath{fill}%
\end{pgfscope}%
\begin{pgfscope}%
\pgfpathrectangle{\pgfqpoint{0.017500in}{0.017500in}}{\pgfqpoint{6.218282in}{3.580372in}} %
\pgfusepath{clip}%
\pgfsetbuttcap%
\pgfsetroundjoin%
\definecolor{currentfill}{rgb}{0.800830,0.250829,0.225696}%
\pgfsetfillcolor{currentfill}%
\pgfsetlinewidth{0.000000pt}%
\definecolor{currentstroke}{rgb}{0.000000,0.000000,0.000000}%
\pgfsetstrokecolor{currentstroke}%
\pgfsetdash{}{0pt}%
\pgfpathmoveto{\pgfqpoint{4.226975in}{2.236848in}}%
\pgfpathlineto{\pgfqpoint{4.478468in}{2.393972in}}%
\pgfpathlineto{\pgfqpoint{4.226975in}{2.236848in}}%
\pgfpathclose%
\pgfusepath{fill}%
\end{pgfscope}%
\begin{pgfscope}%
\pgfpathrectangle{\pgfqpoint{0.017500in}{0.017500in}}{\pgfqpoint{6.218282in}{3.580372in}} %
\pgfusepath{clip}%
\pgfsetbuttcap%
\pgfsetroundjoin%
\definecolor{currentfill}{rgb}{0.800830,0.250829,0.225696}%
\pgfsetfillcolor{currentfill}%
\pgfsetlinewidth{0.000000pt}%
\definecolor{currentstroke}{rgb}{0.000000,0.000000,0.000000}%
\pgfsetstrokecolor{currentstroke}%
\pgfsetdash{}{0pt}%
\pgfpathmoveto{\pgfqpoint{4.226975in}{2.236848in}}%
\pgfpathlineto{\pgfqpoint{4.478468in}{2.393972in}}%
\pgfpathlineto{\pgfqpoint{4.226975in}{2.236848in}}%
\pgfpathclose%
\pgfusepath{fill}%
\end{pgfscope}%
\begin{pgfscope}%
\pgfpathrectangle{\pgfqpoint{0.017500in}{0.017500in}}{\pgfqpoint{6.218282in}{3.580372in}} %
\pgfusepath{clip}%
\pgfsetbuttcap%
\pgfsetroundjoin%
\definecolor{currentfill}{rgb}{0.800830,0.250829,0.225696}%
\pgfsetfillcolor{currentfill}%
\pgfsetlinewidth{0.000000pt}%
\definecolor{currentstroke}{rgb}{0.000000,0.000000,0.000000}%
\pgfsetstrokecolor{currentstroke}%
\pgfsetdash{}{0pt}%
\pgfpathmoveto{\pgfqpoint{4.226975in}{2.236848in}}%
\pgfpathlineto{\pgfqpoint{4.478468in}{2.393972in}}%
\pgfpathlineto{\pgfqpoint{4.226975in}{2.236848in}}%
\pgfpathclose%
\pgfusepath{fill}%
\end{pgfscope}%
\begin{pgfscope}%
\pgfpathrectangle{\pgfqpoint{0.017500in}{0.017500in}}{\pgfqpoint{6.218282in}{3.580372in}} %
\pgfusepath{clip}%
\pgfsetbuttcap%
\pgfsetroundjoin%
\definecolor{currentfill}{rgb}{0.800830,0.250829,0.225696}%
\pgfsetfillcolor{currentfill}%
\pgfsetlinewidth{0.000000pt}%
\definecolor{currentstroke}{rgb}{0.000000,0.000000,0.000000}%
\pgfsetstrokecolor{currentstroke}%
\pgfsetdash{}{0pt}%
\pgfpathmoveto{\pgfqpoint{4.226975in}{2.236848in}}%
\pgfpathlineto{\pgfqpoint{4.478468in}{2.393972in}}%
\pgfpathlineto{\pgfqpoint{4.226975in}{2.236848in}}%
\pgfpathclose%
\pgfusepath{fill}%
\end{pgfscope}%
\begin{pgfscope}%
\pgfpathrectangle{\pgfqpoint{0.017500in}{0.017500in}}{\pgfqpoint{6.218282in}{3.580372in}} %
\pgfusepath{clip}%
\pgfsetbuttcap%
\pgfsetroundjoin%
\definecolor{currentfill}{rgb}{0.800830,0.250829,0.225696}%
\pgfsetfillcolor{currentfill}%
\pgfsetlinewidth{0.000000pt}%
\definecolor{currentstroke}{rgb}{0.000000,0.000000,0.000000}%
\pgfsetstrokecolor{currentstroke}%
\pgfsetdash{}{0pt}%
\pgfpathmoveto{\pgfqpoint{4.226975in}{2.236848in}}%
\pgfpathlineto{\pgfqpoint{4.478468in}{2.393972in}}%
\pgfpathlineto{\pgfqpoint{4.226975in}{2.236848in}}%
\pgfpathclose%
\pgfusepath{fill}%
\end{pgfscope}%
\begin{pgfscope}%
\pgfpathrectangle{\pgfqpoint{0.017500in}{0.017500in}}{\pgfqpoint{6.218282in}{3.580372in}} %
\pgfusepath{clip}%
\pgfsetbuttcap%
\pgfsetroundjoin%
\definecolor{currentfill}{rgb}{0.800830,0.250829,0.225696}%
\pgfsetfillcolor{currentfill}%
\pgfsetlinewidth{0.000000pt}%
\definecolor{currentstroke}{rgb}{0.000000,0.000000,0.000000}%
\pgfsetstrokecolor{currentstroke}%
\pgfsetdash{}{0pt}%
\pgfpathmoveto{\pgfqpoint{4.226975in}{2.236848in}}%
\pgfpathlineto{\pgfqpoint{4.478468in}{2.393972in}}%
\pgfpathlineto{\pgfqpoint{4.226975in}{2.236848in}}%
\pgfpathclose%
\pgfusepath{fill}%
\end{pgfscope}%
\begin{pgfscope}%
\pgfpathrectangle{\pgfqpoint{0.017500in}{0.017500in}}{\pgfqpoint{6.218282in}{3.580372in}} %
\pgfusepath{clip}%
\pgfsetbuttcap%
\pgfsetroundjoin%
\definecolor{currentfill}{rgb}{0.800830,0.250829,0.225696}%
\pgfsetfillcolor{currentfill}%
\pgfsetlinewidth{0.000000pt}%
\definecolor{currentstroke}{rgb}{0.000000,0.000000,0.000000}%
\pgfsetstrokecolor{currentstroke}%
\pgfsetdash{}{0pt}%
\pgfpathmoveto{\pgfqpoint{4.226975in}{2.236848in}}%
\pgfpathlineto{\pgfqpoint{4.478468in}{2.393972in}}%
\pgfpathlineto{\pgfqpoint{4.226975in}{2.236848in}}%
\pgfpathclose%
\pgfusepath{fill}%
\end{pgfscope}%
\begin{pgfscope}%
\pgfpathrectangle{\pgfqpoint{0.017500in}{0.017500in}}{\pgfqpoint{6.218282in}{3.580372in}} %
\pgfusepath{clip}%
\pgfsetbuttcap%
\pgfsetroundjoin%
\definecolor{currentfill}{rgb}{0.800830,0.250829,0.225696}%
\pgfsetfillcolor{currentfill}%
\pgfsetlinewidth{0.000000pt}%
\definecolor{currentstroke}{rgb}{0.000000,0.000000,0.000000}%
\pgfsetstrokecolor{currentstroke}%
\pgfsetdash{}{0pt}%
\pgfpathmoveto{\pgfqpoint{4.226975in}{2.236848in}}%
\pgfpathlineto{\pgfqpoint{4.478468in}{2.393972in}}%
\pgfpathlineto{\pgfqpoint{4.226975in}{2.236848in}}%
\pgfpathclose%
\pgfusepath{fill}%
\end{pgfscope}%
\begin{pgfscope}%
\pgfpathrectangle{\pgfqpoint{0.017500in}{0.017500in}}{\pgfqpoint{6.218282in}{3.580372in}} %
\pgfusepath{clip}%
\pgfsetbuttcap%
\pgfsetroundjoin%
\definecolor{currentfill}{rgb}{0.800830,0.250829,0.225696}%
\pgfsetfillcolor{currentfill}%
\pgfsetlinewidth{0.000000pt}%
\definecolor{currentstroke}{rgb}{0.000000,0.000000,0.000000}%
\pgfsetstrokecolor{currentstroke}%
\pgfsetdash{}{0pt}%
\pgfpathmoveto{\pgfqpoint{4.226975in}{2.236848in}}%
\pgfpathlineto{\pgfqpoint{4.478468in}{2.393972in}}%
\pgfpathlineto{\pgfqpoint{4.226975in}{2.236848in}}%
\pgfpathclose%
\pgfusepath{fill}%
\end{pgfscope}%
\begin{pgfscope}%
\pgfpathrectangle{\pgfqpoint{0.017500in}{0.017500in}}{\pgfqpoint{6.218282in}{3.580372in}} %
\pgfusepath{clip}%
\pgfsetbuttcap%
\pgfsetroundjoin%
\definecolor{currentfill}{rgb}{0.800830,0.250829,0.225696}%
\pgfsetfillcolor{currentfill}%
\pgfsetlinewidth{0.000000pt}%
\definecolor{currentstroke}{rgb}{0.000000,0.000000,0.000000}%
\pgfsetstrokecolor{currentstroke}%
\pgfsetdash{}{0pt}%
\pgfpathmoveto{\pgfqpoint{4.226975in}{2.236848in}}%
\pgfpathlineto{\pgfqpoint{4.478468in}{2.393972in}}%
\pgfpathlineto{\pgfqpoint{4.226975in}{2.236848in}}%
\pgfpathclose%
\pgfusepath{fill}%
\end{pgfscope}%
\begin{pgfscope}%
\pgfpathrectangle{\pgfqpoint{0.017500in}{0.017500in}}{\pgfqpoint{6.218282in}{3.580372in}} %
\pgfusepath{clip}%
\pgfsetbuttcap%
\pgfsetroundjoin%
\definecolor{currentfill}{rgb}{0.800830,0.250829,0.225696}%
\pgfsetfillcolor{currentfill}%
\pgfsetlinewidth{0.000000pt}%
\definecolor{currentstroke}{rgb}{0.000000,0.000000,0.000000}%
\pgfsetstrokecolor{currentstroke}%
\pgfsetdash{}{0pt}%
\pgfpathmoveto{\pgfqpoint{4.226975in}{2.236848in}}%
\pgfpathlineto{\pgfqpoint{4.478468in}{2.393972in}}%
\pgfpathlineto{\pgfqpoint{4.226975in}{2.236848in}}%
\pgfpathclose%
\pgfusepath{fill}%
\end{pgfscope}%
\begin{pgfscope}%
\pgfpathrectangle{\pgfqpoint{0.017500in}{0.017500in}}{\pgfqpoint{6.218282in}{3.580372in}} %
\pgfusepath{clip}%
\pgfsetbuttcap%
\pgfsetroundjoin%
\definecolor{currentfill}{rgb}{0.800830,0.250829,0.225696}%
\pgfsetfillcolor{currentfill}%
\pgfsetlinewidth{0.000000pt}%
\definecolor{currentstroke}{rgb}{0.000000,0.000000,0.000000}%
\pgfsetstrokecolor{currentstroke}%
\pgfsetdash{}{0pt}%
\pgfpathmoveto{\pgfqpoint{4.226975in}{2.236848in}}%
\pgfpathlineto{\pgfqpoint{4.478468in}{2.393972in}}%
\pgfpathlineto{\pgfqpoint{4.226975in}{2.236848in}}%
\pgfpathclose%
\pgfusepath{fill}%
\end{pgfscope}%
\begin{pgfscope}%
\pgfpathrectangle{\pgfqpoint{0.017500in}{0.017500in}}{\pgfqpoint{6.218282in}{3.580372in}} %
\pgfusepath{clip}%
\pgfsetbuttcap%
\pgfsetroundjoin%
\definecolor{currentfill}{rgb}{0.800830,0.250829,0.225696}%
\pgfsetfillcolor{currentfill}%
\pgfsetlinewidth{0.000000pt}%
\definecolor{currentstroke}{rgb}{0.000000,0.000000,0.000000}%
\pgfsetstrokecolor{currentstroke}%
\pgfsetdash{}{0pt}%
\pgfpathmoveto{\pgfqpoint{4.226975in}{2.236848in}}%
\pgfpathlineto{\pgfqpoint{4.478468in}{2.393972in}}%
\pgfpathlineto{\pgfqpoint{4.226975in}{2.236848in}}%
\pgfpathclose%
\pgfusepath{fill}%
\end{pgfscope}%
\begin{pgfscope}%
\pgfpathrectangle{\pgfqpoint{0.017500in}{0.017500in}}{\pgfqpoint{6.218282in}{3.580372in}} %
\pgfusepath{clip}%
\pgfsetbuttcap%
\pgfsetroundjoin%
\definecolor{currentfill}{rgb}{0.800830,0.250829,0.225696}%
\pgfsetfillcolor{currentfill}%
\pgfsetlinewidth{0.000000pt}%
\definecolor{currentstroke}{rgb}{0.000000,0.000000,0.000000}%
\pgfsetstrokecolor{currentstroke}%
\pgfsetdash{}{0pt}%
\pgfpathmoveto{\pgfqpoint{4.226975in}{2.236848in}}%
\pgfpathlineto{\pgfqpoint{4.478468in}{2.393972in}}%
\pgfpathlineto{\pgfqpoint{4.226975in}{2.236848in}}%
\pgfpathclose%
\pgfusepath{fill}%
\end{pgfscope}%
\begin{pgfscope}%
\pgfpathrectangle{\pgfqpoint{0.017500in}{0.017500in}}{\pgfqpoint{6.218282in}{3.580372in}} %
\pgfusepath{clip}%
\pgfsetbuttcap%
\pgfsetroundjoin%
\definecolor{currentfill}{rgb}{0.800830,0.250829,0.225696}%
\pgfsetfillcolor{currentfill}%
\pgfsetlinewidth{0.000000pt}%
\definecolor{currentstroke}{rgb}{0.000000,0.000000,0.000000}%
\pgfsetstrokecolor{currentstroke}%
\pgfsetdash{}{0pt}%
\pgfpathmoveto{\pgfqpoint{4.226975in}{2.236848in}}%
\pgfpathlineto{\pgfqpoint{4.478468in}{2.393972in}}%
\pgfpathlineto{\pgfqpoint{4.226975in}{2.236848in}}%
\pgfpathclose%
\pgfusepath{fill}%
\end{pgfscope}%
\begin{pgfscope}%
\pgfpathrectangle{\pgfqpoint{0.017500in}{0.017500in}}{\pgfqpoint{6.218282in}{3.580372in}} %
\pgfusepath{clip}%
\pgfsetbuttcap%
\pgfsetroundjoin%
\definecolor{currentfill}{rgb}{0.800830,0.250829,0.225696}%
\pgfsetfillcolor{currentfill}%
\pgfsetlinewidth{0.000000pt}%
\definecolor{currentstroke}{rgb}{0.000000,0.000000,0.000000}%
\pgfsetstrokecolor{currentstroke}%
\pgfsetdash{}{0pt}%
\pgfpathmoveto{\pgfqpoint{4.226975in}{2.236848in}}%
\pgfpathlineto{\pgfqpoint{4.478468in}{2.393972in}}%
\pgfpathlineto{\pgfqpoint{4.226975in}{2.236848in}}%
\pgfpathclose%
\pgfusepath{fill}%
\end{pgfscope}%
\begin{pgfscope}%
\pgfpathrectangle{\pgfqpoint{0.017500in}{0.017500in}}{\pgfqpoint{6.218282in}{3.580372in}} %
\pgfusepath{clip}%
\pgfsetbuttcap%
\pgfsetroundjoin%
\definecolor{currentfill}{rgb}{0.800830,0.250829,0.225696}%
\pgfsetfillcolor{currentfill}%
\pgfsetlinewidth{0.000000pt}%
\definecolor{currentstroke}{rgb}{0.000000,0.000000,0.000000}%
\pgfsetstrokecolor{currentstroke}%
\pgfsetdash{}{0pt}%
\pgfpathmoveto{\pgfqpoint{4.226975in}{2.236848in}}%
\pgfpathlineto{\pgfqpoint{4.478468in}{2.393972in}}%
\pgfpathlineto{\pgfqpoint{4.226975in}{2.236848in}}%
\pgfpathclose%
\pgfusepath{fill}%
\end{pgfscope}%
\begin{pgfscope}%
\pgfpathrectangle{\pgfqpoint{0.017500in}{0.017500in}}{\pgfqpoint{6.218282in}{3.580372in}} %
\pgfusepath{clip}%
\pgfsetbuttcap%
\pgfsetroundjoin%
\definecolor{currentfill}{rgb}{0.800830,0.250829,0.225696}%
\pgfsetfillcolor{currentfill}%
\pgfsetlinewidth{0.000000pt}%
\definecolor{currentstroke}{rgb}{0.000000,0.000000,0.000000}%
\pgfsetstrokecolor{currentstroke}%
\pgfsetdash{}{0pt}%
\pgfpathmoveto{\pgfqpoint{4.226975in}{2.236848in}}%
\pgfpathlineto{\pgfqpoint{4.478468in}{2.393972in}}%
\pgfpathlineto{\pgfqpoint{4.226975in}{2.236848in}}%
\pgfpathclose%
\pgfusepath{fill}%
\end{pgfscope}%
\begin{pgfscope}%
\pgfpathrectangle{\pgfqpoint{0.017500in}{0.017500in}}{\pgfqpoint{6.218282in}{3.580372in}} %
\pgfusepath{clip}%
\pgfsetbuttcap%
\pgfsetroundjoin%
\definecolor{currentfill}{rgb}{0.800830,0.250829,0.225696}%
\pgfsetfillcolor{currentfill}%
\pgfsetlinewidth{0.000000pt}%
\definecolor{currentstroke}{rgb}{0.000000,0.000000,0.000000}%
\pgfsetstrokecolor{currentstroke}%
\pgfsetdash{}{0pt}%
\pgfpathmoveto{\pgfqpoint{4.226975in}{2.236848in}}%
\pgfpathlineto{\pgfqpoint{4.478468in}{2.393972in}}%
\pgfpathlineto{\pgfqpoint{4.226975in}{2.236848in}}%
\pgfpathclose%
\pgfusepath{fill}%
\end{pgfscope}%
\begin{pgfscope}%
\pgfpathrectangle{\pgfqpoint{0.017500in}{0.017500in}}{\pgfqpoint{6.218282in}{3.580372in}} %
\pgfusepath{clip}%
\pgfsetbuttcap%
\pgfsetroundjoin%
\definecolor{currentfill}{rgb}{0.800830,0.250829,0.225696}%
\pgfsetfillcolor{currentfill}%
\pgfsetlinewidth{0.000000pt}%
\definecolor{currentstroke}{rgb}{0.000000,0.000000,0.000000}%
\pgfsetstrokecolor{currentstroke}%
\pgfsetdash{}{0pt}%
\pgfpathmoveto{\pgfqpoint{4.226975in}{2.236848in}}%
\pgfpathlineto{\pgfqpoint{4.478468in}{2.393972in}}%
\pgfpathlineto{\pgfqpoint{4.226975in}{2.236848in}}%
\pgfpathclose%
\pgfusepath{fill}%
\end{pgfscope}%
\begin{pgfscope}%
\pgfpathrectangle{\pgfqpoint{0.017500in}{0.017500in}}{\pgfqpoint{6.218282in}{3.580372in}} %
\pgfusepath{clip}%
\pgfsetbuttcap%
\pgfsetroundjoin%
\definecolor{currentfill}{rgb}{0.895885,0.433075,0.338681}%
\pgfsetfillcolor{currentfill}%
\pgfsetlinewidth{0.000000pt}%
\definecolor{currentstroke}{rgb}{0.000000,0.000000,0.000000}%
\pgfsetstrokecolor{currentstroke}%
\pgfsetdash{}{0pt}%
\pgfpathmoveto{\pgfqpoint{3.502561in}{2.073396in}}%
\pgfpathlineto{\pgfqpoint{3.763398in}{2.297457in}}%
\pgfpathlineto{\pgfqpoint{3.502561in}{2.073396in}}%
\pgfpathclose%
\pgfusepath{fill}%
\end{pgfscope}%
\begin{pgfscope}%
\pgfpathrectangle{\pgfqpoint{0.017500in}{0.017500in}}{\pgfqpoint{6.218282in}{3.580372in}} %
\pgfusepath{clip}%
\pgfsetbuttcap%
\pgfsetroundjoin%
\definecolor{currentfill}{rgb}{0.895885,0.433075,0.338681}%
\pgfsetfillcolor{currentfill}%
\pgfsetlinewidth{0.000000pt}%
\definecolor{currentstroke}{rgb}{0.000000,0.000000,0.000000}%
\pgfsetstrokecolor{currentstroke}%
\pgfsetdash{}{0pt}%
\pgfpathmoveto{\pgfqpoint{3.502561in}{2.073396in}}%
\pgfpathlineto{\pgfqpoint{3.763398in}{2.297457in}}%
\pgfpathlineto{\pgfqpoint{3.502561in}{2.073396in}}%
\pgfpathclose%
\pgfusepath{fill}%
\end{pgfscope}%
\begin{pgfscope}%
\pgfpathrectangle{\pgfqpoint{0.017500in}{0.017500in}}{\pgfqpoint{6.218282in}{3.580372in}} %
\pgfusepath{clip}%
\pgfsetbuttcap%
\pgfsetroundjoin%
\definecolor{currentfill}{rgb}{0.895885,0.433075,0.338681}%
\pgfsetfillcolor{currentfill}%
\pgfsetlinewidth{0.000000pt}%
\definecolor{currentstroke}{rgb}{0.000000,0.000000,0.000000}%
\pgfsetstrokecolor{currentstroke}%
\pgfsetdash{}{0pt}%
\pgfpathmoveto{\pgfqpoint{3.502561in}{2.073396in}}%
\pgfpathlineto{\pgfqpoint{3.763398in}{2.297457in}}%
\pgfpathlineto{\pgfqpoint{3.502561in}{2.073396in}}%
\pgfpathclose%
\pgfusepath{fill}%
\end{pgfscope}%
\begin{pgfscope}%
\pgfpathrectangle{\pgfqpoint{0.017500in}{0.017500in}}{\pgfqpoint{6.218282in}{3.580372in}} %
\pgfusepath{clip}%
\pgfsetbuttcap%
\pgfsetroundjoin%
\definecolor{currentfill}{rgb}{0.895885,0.433075,0.338681}%
\pgfsetfillcolor{currentfill}%
\pgfsetlinewidth{0.000000pt}%
\definecolor{currentstroke}{rgb}{0.000000,0.000000,0.000000}%
\pgfsetstrokecolor{currentstroke}%
\pgfsetdash{}{0pt}%
\pgfpathmoveto{\pgfqpoint{3.502561in}{2.073396in}}%
\pgfpathlineto{\pgfqpoint{3.763398in}{2.297457in}}%
\pgfpathlineto{\pgfqpoint{3.502561in}{2.073396in}}%
\pgfpathclose%
\pgfusepath{fill}%
\end{pgfscope}%
\begin{pgfscope}%
\pgfpathrectangle{\pgfqpoint{0.017500in}{0.017500in}}{\pgfqpoint{6.218282in}{3.580372in}} %
\pgfusepath{clip}%
\pgfsetbuttcap%
\pgfsetroundjoin%
\definecolor{currentfill}{rgb}{0.895885,0.433075,0.338681}%
\pgfsetfillcolor{currentfill}%
\pgfsetlinewidth{0.000000pt}%
\definecolor{currentstroke}{rgb}{0.000000,0.000000,0.000000}%
\pgfsetstrokecolor{currentstroke}%
\pgfsetdash{}{0pt}%
\pgfpathmoveto{\pgfqpoint{3.502561in}{2.073396in}}%
\pgfpathlineto{\pgfqpoint{3.763398in}{2.297457in}}%
\pgfpathlineto{\pgfqpoint{3.502561in}{2.073396in}}%
\pgfpathclose%
\pgfusepath{fill}%
\end{pgfscope}%
\begin{pgfscope}%
\pgfpathrectangle{\pgfqpoint{0.017500in}{0.017500in}}{\pgfqpoint{6.218282in}{3.580372in}} %
\pgfusepath{clip}%
\pgfsetbuttcap%
\pgfsetroundjoin%
\definecolor{currentfill}{rgb}{0.895885,0.433075,0.338681}%
\pgfsetfillcolor{currentfill}%
\pgfsetlinewidth{0.000000pt}%
\definecolor{currentstroke}{rgb}{0.000000,0.000000,0.000000}%
\pgfsetstrokecolor{currentstroke}%
\pgfsetdash{}{0pt}%
\pgfpathmoveto{\pgfqpoint{3.502561in}{2.073396in}}%
\pgfpathlineto{\pgfqpoint{3.763398in}{2.297457in}}%
\pgfpathlineto{\pgfqpoint{3.502561in}{2.073396in}}%
\pgfpathclose%
\pgfusepath{fill}%
\end{pgfscope}%
\begin{pgfscope}%
\pgfpathrectangle{\pgfqpoint{0.017500in}{0.017500in}}{\pgfqpoint{6.218282in}{3.580372in}} %
\pgfusepath{clip}%
\pgfsetbuttcap%
\pgfsetroundjoin%
\definecolor{currentfill}{rgb}{0.895885,0.433075,0.338681}%
\pgfsetfillcolor{currentfill}%
\pgfsetlinewidth{0.000000pt}%
\definecolor{currentstroke}{rgb}{0.000000,0.000000,0.000000}%
\pgfsetstrokecolor{currentstroke}%
\pgfsetdash{}{0pt}%
\pgfpathmoveto{\pgfqpoint{3.502561in}{2.073396in}}%
\pgfpathlineto{\pgfqpoint{3.763398in}{2.297457in}}%
\pgfpathlineto{\pgfqpoint{3.502561in}{2.073396in}}%
\pgfpathclose%
\pgfusepath{fill}%
\end{pgfscope}%
\begin{pgfscope}%
\pgfpathrectangle{\pgfqpoint{0.017500in}{0.017500in}}{\pgfqpoint{6.218282in}{3.580372in}} %
\pgfusepath{clip}%
\pgfsetbuttcap%
\pgfsetroundjoin%
\definecolor{currentfill}{rgb}{0.895885,0.433075,0.338681}%
\pgfsetfillcolor{currentfill}%
\pgfsetlinewidth{0.000000pt}%
\definecolor{currentstroke}{rgb}{0.000000,0.000000,0.000000}%
\pgfsetstrokecolor{currentstroke}%
\pgfsetdash{}{0pt}%
\pgfpathmoveto{\pgfqpoint{3.502561in}{2.073396in}}%
\pgfpathlineto{\pgfqpoint{3.763398in}{2.297457in}}%
\pgfpathlineto{\pgfqpoint{3.502561in}{2.073396in}}%
\pgfpathclose%
\pgfusepath{fill}%
\end{pgfscope}%
\begin{pgfscope}%
\pgfpathrectangle{\pgfqpoint{0.017500in}{0.017500in}}{\pgfqpoint{6.218282in}{3.580372in}} %
\pgfusepath{clip}%
\pgfsetbuttcap%
\pgfsetroundjoin%
\definecolor{currentfill}{rgb}{0.895885,0.433075,0.338681}%
\pgfsetfillcolor{currentfill}%
\pgfsetlinewidth{0.000000pt}%
\definecolor{currentstroke}{rgb}{0.000000,0.000000,0.000000}%
\pgfsetstrokecolor{currentstroke}%
\pgfsetdash{}{0pt}%
\pgfpathmoveto{\pgfqpoint{3.502561in}{2.073396in}}%
\pgfpathlineto{\pgfqpoint{3.763398in}{2.297457in}}%
\pgfpathlineto{\pgfqpoint{3.502561in}{2.073396in}}%
\pgfpathclose%
\pgfusepath{fill}%
\end{pgfscope}%
\begin{pgfscope}%
\pgfpathrectangle{\pgfqpoint{0.017500in}{0.017500in}}{\pgfqpoint{6.218282in}{3.580372in}} %
\pgfusepath{clip}%
\pgfsetbuttcap%
\pgfsetroundjoin%
\definecolor{currentfill}{rgb}{0.895885,0.433075,0.338681}%
\pgfsetfillcolor{currentfill}%
\pgfsetlinewidth{0.000000pt}%
\definecolor{currentstroke}{rgb}{0.000000,0.000000,0.000000}%
\pgfsetstrokecolor{currentstroke}%
\pgfsetdash{}{0pt}%
\pgfpathmoveto{\pgfqpoint{3.502561in}{2.073396in}}%
\pgfpathlineto{\pgfqpoint{3.763398in}{2.297457in}}%
\pgfpathlineto{\pgfqpoint{3.502561in}{2.073396in}}%
\pgfpathclose%
\pgfusepath{fill}%
\end{pgfscope}%
\begin{pgfscope}%
\pgfpathrectangle{\pgfqpoint{0.017500in}{0.017500in}}{\pgfqpoint{6.218282in}{3.580372in}} %
\pgfusepath{clip}%
\pgfsetbuttcap%
\pgfsetroundjoin%
\definecolor{currentfill}{rgb}{0.895885,0.433075,0.338681}%
\pgfsetfillcolor{currentfill}%
\pgfsetlinewidth{0.000000pt}%
\definecolor{currentstroke}{rgb}{0.000000,0.000000,0.000000}%
\pgfsetstrokecolor{currentstroke}%
\pgfsetdash{}{0pt}%
\pgfpathmoveto{\pgfqpoint{3.502561in}{2.073396in}}%
\pgfpathlineto{\pgfqpoint{3.763398in}{2.297457in}}%
\pgfpathlineto{\pgfqpoint{3.502561in}{2.073396in}}%
\pgfpathclose%
\pgfusepath{fill}%
\end{pgfscope}%
\begin{pgfscope}%
\pgfpathrectangle{\pgfqpoint{0.017500in}{0.017500in}}{\pgfqpoint{6.218282in}{3.580372in}} %
\pgfusepath{clip}%
\pgfsetbuttcap%
\pgfsetroundjoin%
\definecolor{currentfill}{rgb}{0.895885,0.433075,0.338681}%
\pgfsetfillcolor{currentfill}%
\pgfsetlinewidth{0.000000pt}%
\definecolor{currentstroke}{rgb}{0.000000,0.000000,0.000000}%
\pgfsetstrokecolor{currentstroke}%
\pgfsetdash{}{0pt}%
\pgfpathmoveto{\pgfqpoint{3.502561in}{2.073396in}}%
\pgfpathlineto{\pgfqpoint{3.763398in}{2.297457in}}%
\pgfpathlineto{\pgfqpoint{3.502561in}{2.073396in}}%
\pgfpathclose%
\pgfusepath{fill}%
\end{pgfscope}%
\begin{pgfscope}%
\pgfpathrectangle{\pgfqpoint{0.017500in}{0.017500in}}{\pgfqpoint{6.218282in}{3.580372in}} %
\pgfusepath{clip}%
\pgfsetbuttcap%
\pgfsetroundjoin%
\definecolor{currentfill}{rgb}{0.895885,0.433075,0.338681}%
\pgfsetfillcolor{currentfill}%
\pgfsetlinewidth{0.000000pt}%
\definecolor{currentstroke}{rgb}{0.000000,0.000000,0.000000}%
\pgfsetstrokecolor{currentstroke}%
\pgfsetdash{}{0pt}%
\pgfpathmoveto{\pgfqpoint{3.502561in}{2.073396in}}%
\pgfpathlineto{\pgfqpoint{3.763398in}{2.297457in}}%
\pgfpathlineto{\pgfqpoint{3.502561in}{2.073396in}}%
\pgfpathclose%
\pgfusepath{fill}%
\end{pgfscope}%
\begin{pgfscope}%
\pgfpathrectangle{\pgfqpoint{0.017500in}{0.017500in}}{\pgfqpoint{6.218282in}{3.580372in}} %
\pgfusepath{clip}%
\pgfsetbuttcap%
\pgfsetroundjoin%
\definecolor{currentfill}{rgb}{0.895885,0.433075,0.338681}%
\pgfsetfillcolor{currentfill}%
\pgfsetlinewidth{0.000000pt}%
\definecolor{currentstroke}{rgb}{0.000000,0.000000,0.000000}%
\pgfsetstrokecolor{currentstroke}%
\pgfsetdash{}{0pt}%
\pgfpathmoveto{\pgfqpoint{3.502561in}{2.073396in}}%
\pgfpathlineto{\pgfqpoint{3.763398in}{2.297457in}}%
\pgfpathlineto{\pgfqpoint{3.502561in}{2.073396in}}%
\pgfpathclose%
\pgfusepath{fill}%
\end{pgfscope}%
\begin{pgfscope}%
\pgfpathrectangle{\pgfqpoint{0.017500in}{0.017500in}}{\pgfqpoint{6.218282in}{3.580372in}} %
\pgfusepath{clip}%
\pgfsetbuttcap%
\pgfsetroundjoin%
\definecolor{currentfill}{rgb}{0.895885,0.433075,0.338681}%
\pgfsetfillcolor{currentfill}%
\pgfsetlinewidth{0.000000pt}%
\definecolor{currentstroke}{rgb}{0.000000,0.000000,0.000000}%
\pgfsetstrokecolor{currentstroke}%
\pgfsetdash{}{0pt}%
\pgfpathmoveto{\pgfqpoint{3.502561in}{2.073396in}}%
\pgfpathlineto{\pgfqpoint{3.763398in}{2.297457in}}%
\pgfpathlineto{\pgfqpoint{3.502561in}{2.073396in}}%
\pgfpathclose%
\pgfusepath{fill}%
\end{pgfscope}%
\begin{pgfscope}%
\pgfpathrectangle{\pgfqpoint{0.017500in}{0.017500in}}{\pgfqpoint{6.218282in}{3.580372in}} %
\pgfusepath{clip}%
\pgfsetbuttcap%
\pgfsetroundjoin%
\definecolor{currentfill}{rgb}{0.895885,0.433075,0.338681}%
\pgfsetfillcolor{currentfill}%
\pgfsetlinewidth{0.000000pt}%
\definecolor{currentstroke}{rgb}{0.000000,0.000000,0.000000}%
\pgfsetstrokecolor{currentstroke}%
\pgfsetdash{}{0pt}%
\pgfpathmoveto{\pgfqpoint{3.502561in}{2.073396in}}%
\pgfpathlineto{\pgfqpoint{3.763398in}{2.297457in}}%
\pgfpathlineto{\pgfqpoint{3.502561in}{2.073396in}}%
\pgfpathclose%
\pgfusepath{fill}%
\end{pgfscope}%
\begin{pgfscope}%
\pgfpathrectangle{\pgfqpoint{0.017500in}{0.017500in}}{\pgfqpoint{6.218282in}{3.580372in}} %
\pgfusepath{clip}%
\pgfsetbuttcap%
\pgfsetroundjoin%
\definecolor{currentfill}{rgb}{0.895885,0.433075,0.338681}%
\pgfsetfillcolor{currentfill}%
\pgfsetlinewidth{0.000000pt}%
\definecolor{currentstroke}{rgb}{0.000000,0.000000,0.000000}%
\pgfsetstrokecolor{currentstroke}%
\pgfsetdash{}{0pt}%
\pgfpathmoveto{\pgfqpoint{3.502561in}{2.073396in}}%
\pgfpathlineto{\pgfqpoint{3.763398in}{2.297457in}}%
\pgfpathlineto{\pgfqpoint{3.502561in}{2.073396in}}%
\pgfpathclose%
\pgfusepath{fill}%
\end{pgfscope}%
\begin{pgfscope}%
\pgfpathrectangle{\pgfqpoint{0.017500in}{0.017500in}}{\pgfqpoint{6.218282in}{3.580372in}} %
\pgfusepath{clip}%
\pgfsetbuttcap%
\pgfsetroundjoin%
\definecolor{currentfill}{rgb}{0.895885,0.433075,0.338681}%
\pgfsetfillcolor{currentfill}%
\pgfsetlinewidth{0.000000pt}%
\definecolor{currentstroke}{rgb}{0.000000,0.000000,0.000000}%
\pgfsetstrokecolor{currentstroke}%
\pgfsetdash{}{0pt}%
\pgfpathmoveto{\pgfqpoint{3.502561in}{2.073396in}}%
\pgfpathlineto{\pgfqpoint{3.763398in}{2.297457in}}%
\pgfpathlineto{\pgfqpoint{3.502561in}{2.073396in}}%
\pgfpathclose%
\pgfusepath{fill}%
\end{pgfscope}%
\begin{pgfscope}%
\pgfpathrectangle{\pgfqpoint{0.017500in}{0.017500in}}{\pgfqpoint{6.218282in}{3.580372in}} %
\pgfusepath{clip}%
\pgfsetbuttcap%
\pgfsetroundjoin%
\definecolor{currentfill}{rgb}{0.895885,0.433075,0.338681}%
\pgfsetfillcolor{currentfill}%
\pgfsetlinewidth{0.000000pt}%
\definecolor{currentstroke}{rgb}{0.000000,0.000000,0.000000}%
\pgfsetstrokecolor{currentstroke}%
\pgfsetdash{}{0pt}%
\pgfpathmoveto{\pgfqpoint{3.502561in}{2.073396in}}%
\pgfpathlineto{\pgfqpoint{3.763398in}{2.297457in}}%
\pgfpathlineto{\pgfqpoint{3.502561in}{2.073396in}}%
\pgfpathclose%
\pgfusepath{fill}%
\end{pgfscope}%
\begin{pgfscope}%
\pgfpathrectangle{\pgfqpoint{0.017500in}{0.017500in}}{\pgfqpoint{6.218282in}{3.580372in}} %
\pgfusepath{clip}%
\pgfsetbuttcap%
\pgfsetroundjoin%
\definecolor{currentfill}{rgb}{0.895885,0.433075,0.338681}%
\pgfsetfillcolor{currentfill}%
\pgfsetlinewidth{0.000000pt}%
\definecolor{currentstroke}{rgb}{0.000000,0.000000,0.000000}%
\pgfsetstrokecolor{currentstroke}%
\pgfsetdash{}{0pt}%
\pgfpathmoveto{\pgfqpoint{3.502561in}{2.073396in}}%
\pgfpathlineto{\pgfqpoint{3.763398in}{2.297457in}}%
\pgfpathlineto{\pgfqpoint{3.502561in}{2.073396in}}%
\pgfpathclose%
\pgfusepath{fill}%
\end{pgfscope}%
\begin{pgfscope}%
\pgfpathrectangle{\pgfqpoint{0.017500in}{0.017500in}}{\pgfqpoint{6.218282in}{3.580372in}} %
\pgfusepath{clip}%
\pgfsetbuttcap%
\pgfsetroundjoin%
\definecolor{currentfill}{rgb}{0.895885,0.433075,0.338681}%
\pgfsetfillcolor{currentfill}%
\pgfsetlinewidth{0.000000pt}%
\definecolor{currentstroke}{rgb}{0.000000,0.000000,0.000000}%
\pgfsetstrokecolor{currentstroke}%
\pgfsetdash{}{0pt}%
\pgfpathmoveto{\pgfqpoint{3.502561in}{2.073396in}}%
\pgfpathlineto{\pgfqpoint{3.763398in}{2.297457in}}%
\pgfpathlineto{\pgfqpoint{3.502561in}{2.073396in}}%
\pgfpathclose%
\pgfusepath{fill}%
\end{pgfscope}%
\begin{pgfscope}%
\pgfpathrectangle{\pgfqpoint{0.017500in}{0.017500in}}{\pgfqpoint{6.218282in}{3.580372in}} %
\pgfusepath{clip}%
\pgfsetbuttcap%
\pgfsetroundjoin%
\definecolor{currentfill}{rgb}{0.895885,0.433075,0.338681}%
\pgfsetfillcolor{currentfill}%
\pgfsetlinewidth{0.000000pt}%
\definecolor{currentstroke}{rgb}{0.000000,0.000000,0.000000}%
\pgfsetstrokecolor{currentstroke}%
\pgfsetdash{}{0pt}%
\pgfpathmoveto{\pgfqpoint{3.502561in}{2.073396in}}%
\pgfpathlineto{\pgfqpoint{3.763398in}{2.297457in}}%
\pgfpathlineto{\pgfqpoint{3.502561in}{2.073396in}}%
\pgfpathclose%
\pgfusepath{fill}%
\end{pgfscope}%
\begin{pgfscope}%
\pgfpathrectangle{\pgfqpoint{0.017500in}{0.017500in}}{\pgfqpoint{6.218282in}{3.580372in}} %
\pgfusepath{clip}%
\pgfsetbuttcap%
\pgfsetroundjoin%
\definecolor{currentfill}{rgb}{0.895885,0.433075,0.338681}%
\pgfsetfillcolor{currentfill}%
\pgfsetlinewidth{0.000000pt}%
\definecolor{currentstroke}{rgb}{0.000000,0.000000,0.000000}%
\pgfsetstrokecolor{currentstroke}%
\pgfsetdash{}{0pt}%
\pgfpathmoveto{\pgfqpoint{3.502561in}{2.073396in}}%
\pgfpathlineto{\pgfqpoint{3.763398in}{2.297457in}}%
\pgfpathlineto{\pgfqpoint{3.502561in}{2.073396in}}%
\pgfpathclose%
\pgfusepath{fill}%
\end{pgfscope}%
\begin{pgfscope}%
\pgfpathrectangle{\pgfqpoint{0.017500in}{0.017500in}}{\pgfqpoint{6.218282in}{3.580372in}} %
\pgfusepath{clip}%
\pgfsetbuttcap%
\pgfsetroundjoin%
\definecolor{currentfill}{rgb}{0.895885,0.433075,0.338681}%
\pgfsetfillcolor{currentfill}%
\pgfsetlinewidth{0.000000pt}%
\definecolor{currentstroke}{rgb}{0.000000,0.000000,0.000000}%
\pgfsetstrokecolor{currentstroke}%
\pgfsetdash{}{0pt}%
\pgfpathmoveto{\pgfqpoint{3.502561in}{2.073396in}}%
\pgfpathlineto{\pgfqpoint{3.763398in}{2.297457in}}%
\pgfpathlineto{\pgfqpoint{3.502561in}{2.073396in}}%
\pgfpathclose%
\pgfusepath{fill}%
\end{pgfscope}%
\begin{pgfscope}%
\pgfpathrectangle{\pgfqpoint{0.017500in}{0.017500in}}{\pgfqpoint{6.218282in}{3.580372in}} %
\pgfusepath{clip}%
\pgfsetbuttcap%
\pgfsetroundjoin%
\definecolor{currentfill}{rgb}{0.895885,0.433075,0.338681}%
\pgfsetfillcolor{currentfill}%
\pgfsetlinewidth{0.000000pt}%
\definecolor{currentstroke}{rgb}{0.000000,0.000000,0.000000}%
\pgfsetstrokecolor{currentstroke}%
\pgfsetdash{}{0pt}%
\pgfpathmoveto{\pgfqpoint{3.502561in}{2.073396in}}%
\pgfpathlineto{\pgfqpoint{3.763398in}{2.297457in}}%
\pgfpathlineto{\pgfqpoint{3.502561in}{2.073396in}}%
\pgfpathclose%
\pgfusepath{fill}%
\end{pgfscope}%
\begin{pgfscope}%
\pgfpathrectangle{\pgfqpoint{0.017500in}{0.017500in}}{\pgfqpoint{6.218282in}{3.580372in}} %
\pgfusepath{clip}%
\pgfsetbuttcap%
\pgfsetroundjoin%
\definecolor{currentfill}{rgb}{0.895885,0.433075,0.338681}%
\pgfsetfillcolor{currentfill}%
\pgfsetlinewidth{0.000000pt}%
\definecolor{currentstroke}{rgb}{0.000000,0.000000,0.000000}%
\pgfsetstrokecolor{currentstroke}%
\pgfsetdash{}{0pt}%
\pgfpathmoveto{\pgfqpoint{3.502561in}{2.073396in}}%
\pgfpathlineto{\pgfqpoint{3.763398in}{2.297457in}}%
\pgfpathlineto{\pgfqpoint{3.502561in}{2.073396in}}%
\pgfpathclose%
\pgfusepath{fill}%
\end{pgfscope}%
\begin{pgfscope}%
\pgfpathrectangle{\pgfqpoint{0.017500in}{0.017500in}}{\pgfqpoint{6.218282in}{3.580372in}} %
\pgfusepath{clip}%
\pgfsetbuttcap%
\pgfsetroundjoin%
\definecolor{currentfill}{rgb}{0.895885,0.433075,0.338681}%
\pgfsetfillcolor{currentfill}%
\pgfsetlinewidth{0.000000pt}%
\definecolor{currentstroke}{rgb}{0.000000,0.000000,0.000000}%
\pgfsetstrokecolor{currentstroke}%
\pgfsetdash{}{0pt}%
\pgfpathmoveto{\pgfqpoint{3.502561in}{2.073396in}}%
\pgfpathlineto{\pgfqpoint{3.763398in}{2.297457in}}%
\pgfpathlineto{\pgfqpoint{3.502561in}{2.073396in}}%
\pgfpathclose%
\pgfusepath{fill}%
\end{pgfscope}%
\begin{pgfscope}%
\pgfpathrectangle{\pgfqpoint{0.017500in}{0.017500in}}{\pgfqpoint{6.218282in}{3.580372in}} %
\pgfusepath{clip}%
\pgfsetbuttcap%
\pgfsetroundjoin%
\definecolor{currentfill}{rgb}{0.895885,0.433075,0.338681}%
\pgfsetfillcolor{currentfill}%
\pgfsetlinewidth{0.000000pt}%
\definecolor{currentstroke}{rgb}{0.000000,0.000000,0.000000}%
\pgfsetstrokecolor{currentstroke}%
\pgfsetdash{}{0pt}%
\pgfpathmoveto{\pgfqpoint{3.502561in}{2.073396in}}%
\pgfpathlineto{\pgfqpoint{3.763398in}{2.297457in}}%
\pgfpathlineto{\pgfqpoint{3.502561in}{2.073396in}}%
\pgfpathclose%
\pgfusepath{fill}%
\end{pgfscope}%
\begin{pgfscope}%
\pgfpathrectangle{\pgfqpoint{0.017500in}{0.017500in}}{\pgfqpoint{6.218282in}{3.580372in}} %
\pgfusepath{clip}%
\pgfsetbuttcap%
\pgfsetroundjoin%
\definecolor{currentfill}{rgb}{0.895885,0.433075,0.338681}%
\pgfsetfillcolor{currentfill}%
\pgfsetlinewidth{0.000000pt}%
\definecolor{currentstroke}{rgb}{0.000000,0.000000,0.000000}%
\pgfsetstrokecolor{currentstroke}%
\pgfsetdash{}{0pt}%
\pgfpathmoveto{\pgfqpoint{3.502561in}{2.073396in}}%
\pgfpathlineto{\pgfqpoint{3.763398in}{2.297457in}}%
\pgfpathlineto{\pgfqpoint{3.502561in}{2.073396in}}%
\pgfpathclose%
\pgfusepath{fill}%
\end{pgfscope}%
\begin{pgfscope}%
\pgfpathrectangle{\pgfqpoint{0.017500in}{0.017500in}}{\pgfqpoint{6.218282in}{3.580372in}} %
\pgfusepath{clip}%
\pgfsetbuttcap%
\pgfsetroundjoin%
\definecolor{currentfill}{rgb}{0.895885,0.433075,0.338681}%
\pgfsetfillcolor{currentfill}%
\pgfsetlinewidth{0.000000pt}%
\definecolor{currentstroke}{rgb}{0.000000,0.000000,0.000000}%
\pgfsetstrokecolor{currentstroke}%
\pgfsetdash{}{0pt}%
\pgfpathmoveto{\pgfqpoint{3.502561in}{2.073396in}}%
\pgfpathlineto{\pgfqpoint{3.763398in}{2.297457in}}%
\pgfpathlineto{\pgfqpoint{3.502561in}{2.073396in}}%
\pgfpathclose%
\pgfusepath{fill}%
\end{pgfscope}%
\begin{pgfscope}%
\pgfpathrectangle{\pgfqpoint{0.017500in}{0.017500in}}{\pgfqpoint{6.218282in}{3.580372in}} %
\pgfusepath{clip}%
\pgfsetbuttcap%
\pgfsetroundjoin%
\definecolor{currentfill}{rgb}{0.888390,0.417703,0.327898}%
\pgfsetfillcolor{currentfill}%
\pgfsetlinewidth{0.000000pt}%
\definecolor{currentstroke}{rgb}{0.000000,0.000000,0.000000}%
\pgfsetstrokecolor{currentstroke}%
\pgfsetdash{}{0pt}%
\pgfpathmoveto{\pgfqpoint{3.502561in}{2.073396in}}%
\pgfpathlineto{\pgfqpoint{3.968076in}{1.994042in}}%
\pgfpathlineto{\pgfqpoint{4.226975in}{2.236848in}}%
\pgfpathlineto{\pgfqpoint{3.763398in}{2.297457in}}%
\pgfpathlineto{\pgfqpoint{3.502561in}{2.073396in}}%
\pgfpathclose%
\pgfusepath{fill}%
\end{pgfscope}%
\begin{pgfscope}%
\pgfpathrectangle{\pgfqpoint{0.017500in}{0.017500in}}{\pgfqpoint{6.218282in}{3.580372in}} %
\pgfusepath{clip}%
\pgfsetbuttcap%
\pgfsetroundjoin%
\definecolor{currentfill}{rgb}{0.888390,0.417703,0.327898}%
\pgfsetfillcolor{currentfill}%
\pgfsetlinewidth{0.000000pt}%
\definecolor{currentstroke}{rgb}{0.000000,0.000000,0.000000}%
\pgfsetstrokecolor{currentstroke}%
\pgfsetdash{}{0pt}%
\pgfpathmoveto{\pgfqpoint{3.502561in}{2.073396in}}%
\pgfpathlineto{\pgfqpoint{3.968076in}{1.994042in}}%
\pgfpathlineto{\pgfqpoint{4.226975in}{2.236848in}}%
\pgfpathlineto{\pgfqpoint{3.763398in}{2.297457in}}%
\pgfpathlineto{\pgfqpoint{3.502561in}{2.073396in}}%
\pgfpathclose%
\pgfusepath{fill}%
\end{pgfscope}%
\begin{pgfscope}%
\pgfpathrectangle{\pgfqpoint{0.017500in}{0.017500in}}{\pgfqpoint{6.218282in}{3.580372in}} %
\pgfusepath{clip}%
\pgfsetbuttcap%
\pgfsetroundjoin%
\definecolor{currentfill}{rgb}{0.888390,0.417703,0.327898}%
\pgfsetfillcolor{currentfill}%
\pgfsetlinewidth{0.000000pt}%
\definecolor{currentstroke}{rgb}{0.000000,0.000000,0.000000}%
\pgfsetstrokecolor{currentstroke}%
\pgfsetdash{}{0pt}%
\pgfpathmoveto{\pgfqpoint{3.502561in}{2.073396in}}%
\pgfpathlineto{\pgfqpoint{3.968076in}{1.994042in}}%
\pgfpathlineto{\pgfqpoint{4.226975in}{2.236848in}}%
\pgfpathlineto{\pgfqpoint{3.763398in}{2.297457in}}%
\pgfpathlineto{\pgfqpoint{3.502561in}{2.073396in}}%
\pgfpathclose%
\pgfusepath{fill}%
\end{pgfscope}%
\begin{pgfscope}%
\pgfpathrectangle{\pgfqpoint{0.017500in}{0.017500in}}{\pgfqpoint{6.218282in}{3.580372in}} %
\pgfusepath{clip}%
\pgfsetbuttcap%
\pgfsetroundjoin%
\definecolor{currentfill}{rgb}{0.888390,0.417703,0.327898}%
\pgfsetfillcolor{currentfill}%
\pgfsetlinewidth{0.000000pt}%
\definecolor{currentstroke}{rgb}{0.000000,0.000000,0.000000}%
\pgfsetstrokecolor{currentstroke}%
\pgfsetdash{}{0pt}%
\pgfpathmoveto{\pgfqpoint{3.502561in}{2.073396in}}%
\pgfpathlineto{\pgfqpoint{3.968076in}{1.994042in}}%
\pgfpathlineto{\pgfqpoint{4.226975in}{2.236848in}}%
\pgfpathlineto{\pgfqpoint{3.763398in}{2.297457in}}%
\pgfpathlineto{\pgfqpoint{3.502561in}{2.073396in}}%
\pgfpathclose%
\pgfusepath{fill}%
\end{pgfscope}%
\begin{pgfscope}%
\pgfpathrectangle{\pgfqpoint{0.017500in}{0.017500in}}{\pgfqpoint{6.218282in}{3.580372in}} %
\pgfusepath{clip}%
\pgfsetbuttcap%
\pgfsetroundjoin%
\definecolor{currentfill}{rgb}{0.888390,0.417703,0.327898}%
\pgfsetfillcolor{currentfill}%
\pgfsetlinewidth{0.000000pt}%
\definecolor{currentstroke}{rgb}{0.000000,0.000000,0.000000}%
\pgfsetstrokecolor{currentstroke}%
\pgfsetdash{}{0pt}%
\pgfpathmoveto{\pgfqpoint{3.502561in}{2.073396in}}%
\pgfpathlineto{\pgfqpoint{3.968076in}{1.994042in}}%
\pgfpathlineto{\pgfqpoint{4.226975in}{2.236848in}}%
\pgfpathlineto{\pgfqpoint{3.763398in}{2.297457in}}%
\pgfpathlineto{\pgfqpoint{3.502561in}{2.073396in}}%
\pgfpathclose%
\pgfusepath{fill}%
\end{pgfscope}%
\begin{pgfscope}%
\pgfpathrectangle{\pgfqpoint{0.017500in}{0.017500in}}{\pgfqpoint{6.218282in}{3.580372in}} %
\pgfusepath{clip}%
\pgfsetbuttcap%
\pgfsetroundjoin%
\definecolor{currentfill}{rgb}{0.888390,0.417703,0.327898}%
\pgfsetfillcolor{currentfill}%
\pgfsetlinewidth{0.000000pt}%
\definecolor{currentstroke}{rgb}{0.000000,0.000000,0.000000}%
\pgfsetstrokecolor{currentstroke}%
\pgfsetdash{}{0pt}%
\pgfpathmoveto{\pgfqpoint{3.502561in}{2.073396in}}%
\pgfpathlineto{\pgfqpoint{3.968076in}{1.994042in}}%
\pgfpathlineto{\pgfqpoint{4.226975in}{2.236848in}}%
\pgfpathlineto{\pgfqpoint{3.763398in}{2.297457in}}%
\pgfpathlineto{\pgfqpoint{3.502561in}{2.073396in}}%
\pgfpathclose%
\pgfusepath{fill}%
\end{pgfscope}%
\begin{pgfscope}%
\pgfpathrectangle{\pgfqpoint{0.017500in}{0.017500in}}{\pgfqpoint{6.218282in}{3.580372in}} %
\pgfusepath{clip}%
\pgfsetbuttcap%
\pgfsetroundjoin%
\definecolor{currentfill}{rgb}{0.892138,0.425389,0.333289}%
\pgfsetfillcolor{currentfill}%
\pgfsetlinewidth{0.000000pt}%
\definecolor{currentstroke}{rgb}{0.000000,0.000000,0.000000}%
\pgfsetstrokecolor{currentstroke}%
\pgfsetdash{}{0pt}%
\pgfpathmoveto{\pgfqpoint{3.968076in}{1.994042in}}%
\pgfpathlineto{\pgfqpoint{4.226975in}{2.236848in}}%
\pgfpathlineto{\pgfqpoint{3.968076in}{1.994042in}}%
\pgfpathclose%
\pgfusepath{fill}%
\end{pgfscope}%
\begin{pgfscope}%
\pgfpathrectangle{\pgfqpoint{0.017500in}{0.017500in}}{\pgfqpoint{6.218282in}{3.580372in}} %
\pgfusepath{clip}%
\pgfsetbuttcap%
\pgfsetroundjoin%
\definecolor{currentfill}{rgb}{0.892138,0.425389,0.333289}%
\pgfsetfillcolor{currentfill}%
\pgfsetlinewidth{0.000000pt}%
\definecolor{currentstroke}{rgb}{0.000000,0.000000,0.000000}%
\pgfsetstrokecolor{currentstroke}%
\pgfsetdash{}{0pt}%
\pgfpathmoveto{\pgfqpoint{3.968076in}{1.994042in}}%
\pgfpathlineto{\pgfqpoint{4.226975in}{2.236848in}}%
\pgfpathlineto{\pgfqpoint{3.968076in}{1.994042in}}%
\pgfpathclose%
\pgfusepath{fill}%
\end{pgfscope}%
\begin{pgfscope}%
\pgfpathrectangle{\pgfqpoint{0.017500in}{0.017500in}}{\pgfqpoint{6.218282in}{3.580372in}} %
\pgfusepath{clip}%
\pgfsetbuttcap%
\pgfsetroundjoin%
\definecolor{currentfill}{rgb}{0.892138,0.425389,0.333289}%
\pgfsetfillcolor{currentfill}%
\pgfsetlinewidth{0.000000pt}%
\definecolor{currentstroke}{rgb}{0.000000,0.000000,0.000000}%
\pgfsetstrokecolor{currentstroke}%
\pgfsetdash{}{0pt}%
\pgfpathmoveto{\pgfqpoint{3.968076in}{1.994042in}}%
\pgfpathlineto{\pgfqpoint{4.226975in}{2.236848in}}%
\pgfpathlineto{\pgfqpoint{3.968076in}{1.994042in}}%
\pgfpathclose%
\pgfusepath{fill}%
\end{pgfscope}%
\begin{pgfscope}%
\pgfpathrectangle{\pgfqpoint{0.017500in}{0.017500in}}{\pgfqpoint{6.218282in}{3.580372in}} %
\pgfusepath{clip}%
\pgfsetbuttcap%
\pgfsetroundjoin%
\definecolor{currentfill}{rgb}{0.892138,0.425389,0.333289}%
\pgfsetfillcolor{currentfill}%
\pgfsetlinewidth{0.000000pt}%
\definecolor{currentstroke}{rgb}{0.000000,0.000000,0.000000}%
\pgfsetstrokecolor{currentstroke}%
\pgfsetdash{}{0pt}%
\pgfpathmoveto{\pgfqpoint{3.968076in}{1.994042in}}%
\pgfpathlineto{\pgfqpoint{4.226975in}{2.236848in}}%
\pgfpathlineto{\pgfqpoint{3.968076in}{1.994042in}}%
\pgfpathclose%
\pgfusepath{fill}%
\end{pgfscope}%
\begin{pgfscope}%
\pgfpathrectangle{\pgfqpoint{0.017500in}{0.017500in}}{\pgfqpoint{6.218282in}{3.580372in}} %
\pgfusepath{clip}%
\pgfsetbuttcap%
\pgfsetroundjoin%
\definecolor{currentfill}{rgb}{0.892138,0.425389,0.333289}%
\pgfsetfillcolor{currentfill}%
\pgfsetlinewidth{0.000000pt}%
\definecolor{currentstroke}{rgb}{0.000000,0.000000,0.000000}%
\pgfsetstrokecolor{currentstroke}%
\pgfsetdash{}{0pt}%
\pgfpathmoveto{\pgfqpoint{3.968076in}{1.994042in}}%
\pgfpathlineto{\pgfqpoint{4.226975in}{2.236848in}}%
\pgfpathlineto{\pgfqpoint{3.968076in}{1.994042in}}%
\pgfpathclose%
\pgfusepath{fill}%
\end{pgfscope}%
\begin{pgfscope}%
\pgfpathrectangle{\pgfqpoint{0.017500in}{0.017500in}}{\pgfqpoint{6.218282in}{3.580372in}} %
\pgfusepath{clip}%
\pgfsetbuttcap%
\pgfsetroundjoin%
\definecolor{currentfill}{rgb}{0.892138,0.425389,0.333289}%
\pgfsetfillcolor{currentfill}%
\pgfsetlinewidth{0.000000pt}%
\definecolor{currentstroke}{rgb}{0.000000,0.000000,0.000000}%
\pgfsetstrokecolor{currentstroke}%
\pgfsetdash{}{0pt}%
\pgfpathmoveto{\pgfqpoint{3.968076in}{1.994042in}}%
\pgfpathlineto{\pgfqpoint{4.226975in}{2.236848in}}%
\pgfpathlineto{\pgfqpoint{3.968076in}{1.994042in}}%
\pgfpathclose%
\pgfusepath{fill}%
\end{pgfscope}%
\begin{pgfscope}%
\pgfpathrectangle{\pgfqpoint{0.017500in}{0.017500in}}{\pgfqpoint{6.218282in}{3.580372in}} %
\pgfusepath{clip}%
\pgfsetbuttcap%
\pgfsetroundjoin%
\definecolor{currentfill}{rgb}{0.892138,0.425389,0.333289}%
\pgfsetfillcolor{currentfill}%
\pgfsetlinewidth{0.000000pt}%
\definecolor{currentstroke}{rgb}{0.000000,0.000000,0.000000}%
\pgfsetstrokecolor{currentstroke}%
\pgfsetdash{}{0pt}%
\pgfpathmoveto{\pgfqpoint{3.968076in}{1.994042in}}%
\pgfpathlineto{\pgfqpoint{4.226975in}{2.236848in}}%
\pgfpathlineto{\pgfqpoint{3.968076in}{1.994042in}}%
\pgfpathclose%
\pgfusepath{fill}%
\end{pgfscope}%
\begin{pgfscope}%
\pgfpathrectangle{\pgfqpoint{0.017500in}{0.017500in}}{\pgfqpoint{6.218282in}{3.580372in}} %
\pgfusepath{clip}%
\pgfsetbuttcap%
\pgfsetroundjoin%
\definecolor{currentfill}{rgb}{0.892138,0.425389,0.333289}%
\pgfsetfillcolor{currentfill}%
\pgfsetlinewidth{0.000000pt}%
\definecolor{currentstroke}{rgb}{0.000000,0.000000,0.000000}%
\pgfsetstrokecolor{currentstroke}%
\pgfsetdash{}{0pt}%
\pgfpathmoveto{\pgfqpoint{3.968076in}{1.994042in}}%
\pgfpathlineto{\pgfqpoint{4.226975in}{2.236848in}}%
\pgfpathlineto{\pgfqpoint{3.968076in}{1.994042in}}%
\pgfpathclose%
\pgfusepath{fill}%
\end{pgfscope}%
\begin{pgfscope}%
\pgfpathrectangle{\pgfqpoint{0.017500in}{0.017500in}}{\pgfqpoint{6.218282in}{3.580372in}} %
\pgfusepath{clip}%
\pgfsetbuttcap%
\pgfsetroundjoin%
\definecolor{currentfill}{rgb}{0.892138,0.425389,0.333289}%
\pgfsetfillcolor{currentfill}%
\pgfsetlinewidth{0.000000pt}%
\definecolor{currentstroke}{rgb}{0.000000,0.000000,0.000000}%
\pgfsetstrokecolor{currentstroke}%
\pgfsetdash{}{0pt}%
\pgfpathmoveto{\pgfqpoint{3.968076in}{1.994042in}}%
\pgfpathlineto{\pgfqpoint{4.226975in}{2.236848in}}%
\pgfpathlineto{\pgfqpoint{3.968076in}{1.994042in}}%
\pgfpathclose%
\pgfusepath{fill}%
\end{pgfscope}%
\begin{pgfscope}%
\pgfpathrectangle{\pgfqpoint{0.017500in}{0.017500in}}{\pgfqpoint{6.218282in}{3.580372in}} %
\pgfusepath{clip}%
\pgfsetbuttcap%
\pgfsetroundjoin%
\definecolor{currentfill}{rgb}{0.892138,0.425389,0.333289}%
\pgfsetfillcolor{currentfill}%
\pgfsetlinewidth{0.000000pt}%
\definecolor{currentstroke}{rgb}{0.000000,0.000000,0.000000}%
\pgfsetstrokecolor{currentstroke}%
\pgfsetdash{}{0pt}%
\pgfpathmoveto{\pgfqpoint{3.968076in}{1.994042in}}%
\pgfpathlineto{\pgfqpoint{4.226975in}{2.236848in}}%
\pgfpathlineto{\pgfqpoint{3.968076in}{1.994042in}}%
\pgfpathclose%
\pgfusepath{fill}%
\end{pgfscope}%
\begin{pgfscope}%
\pgfpathrectangle{\pgfqpoint{0.017500in}{0.017500in}}{\pgfqpoint{6.218282in}{3.580372in}} %
\pgfusepath{clip}%
\pgfsetbuttcap%
\pgfsetroundjoin%
\definecolor{currentfill}{rgb}{0.892138,0.425389,0.333289}%
\pgfsetfillcolor{currentfill}%
\pgfsetlinewidth{0.000000pt}%
\definecolor{currentstroke}{rgb}{0.000000,0.000000,0.000000}%
\pgfsetstrokecolor{currentstroke}%
\pgfsetdash{}{0pt}%
\pgfpathmoveto{\pgfqpoint{3.968076in}{1.994042in}}%
\pgfpathlineto{\pgfqpoint{4.226975in}{2.236848in}}%
\pgfpathlineto{\pgfqpoint{3.968076in}{1.994042in}}%
\pgfpathclose%
\pgfusepath{fill}%
\end{pgfscope}%
\begin{pgfscope}%
\pgfpathrectangle{\pgfqpoint{0.017500in}{0.017500in}}{\pgfqpoint{6.218282in}{3.580372in}} %
\pgfusepath{clip}%
\pgfsetbuttcap%
\pgfsetroundjoin%
\definecolor{currentfill}{rgb}{0.892138,0.425389,0.333289}%
\pgfsetfillcolor{currentfill}%
\pgfsetlinewidth{0.000000pt}%
\definecolor{currentstroke}{rgb}{0.000000,0.000000,0.000000}%
\pgfsetstrokecolor{currentstroke}%
\pgfsetdash{}{0pt}%
\pgfpathmoveto{\pgfqpoint{3.968076in}{1.994042in}}%
\pgfpathlineto{\pgfqpoint{4.226975in}{2.236848in}}%
\pgfpathlineto{\pgfqpoint{3.968076in}{1.994042in}}%
\pgfpathclose%
\pgfusepath{fill}%
\end{pgfscope}%
\begin{pgfscope}%
\pgfpathrectangle{\pgfqpoint{0.017500in}{0.017500in}}{\pgfqpoint{6.218282in}{3.580372in}} %
\pgfusepath{clip}%
\pgfsetbuttcap%
\pgfsetroundjoin%
\definecolor{currentfill}{rgb}{0.892138,0.425389,0.333289}%
\pgfsetfillcolor{currentfill}%
\pgfsetlinewidth{0.000000pt}%
\definecolor{currentstroke}{rgb}{0.000000,0.000000,0.000000}%
\pgfsetstrokecolor{currentstroke}%
\pgfsetdash{}{0pt}%
\pgfpathmoveto{\pgfqpoint{3.968076in}{1.994042in}}%
\pgfpathlineto{\pgfqpoint{4.226975in}{2.236848in}}%
\pgfpathlineto{\pgfqpoint{3.968076in}{1.994042in}}%
\pgfpathclose%
\pgfusepath{fill}%
\end{pgfscope}%
\begin{pgfscope}%
\pgfpathrectangle{\pgfqpoint{0.017500in}{0.017500in}}{\pgfqpoint{6.218282in}{3.580372in}} %
\pgfusepath{clip}%
\pgfsetbuttcap%
\pgfsetroundjoin%
\definecolor{currentfill}{rgb}{0.892138,0.425389,0.333289}%
\pgfsetfillcolor{currentfill}%
\pgfsetlinewidth{0.000000pt}%
\definecolor{currentstroke}{rgb}{0.000000,0.000000,0.000000}%
\pgfsetstrokecolor{currentstroke}%
\pgfsetdash{}{0pt}%
\pgfpathmoveto{\pgfqpoint{3.968076in}{1.994042in}}%
\pgfpathlineto{\pgfqpoint{4.226975in}{2.236848in}}%
\pgfpathlineto{\pgfqpoint{3.968076in}{1.994042in}}%
\pgfpathclose%
\pgfusepath{fill}%
\end{pgfscope}%
\begin{pgfscope}%
\pgfpathrectangle{\pgfqpoint{0.017500in}{0.017500in}}{\pgfqpoint{6.218282in}{3.580372in}} %
\pgfusepath{clip}%
\pgfsetbuttcap%
\pgfsetroundjoin%
\definecolor{currentfill}{rgb}{0.892138,0.425389,0.333289}%
\pgfsetfillcolor{currentfill}%
\pgfsetlinewidth{0.000000pt}%
\definecolor{currentstroke}{rgb}{0.000000,0.000000,0.000000}%
\pgfsetstrokecolor{currentstroke}%
\pgfsetdash{}{0pt}%
\pgfpathmoveto{\pgfqpoint{3.968076in}{1.994042in}}%
\pgfpathlineto{\pgfqpoint{4.226975in}{2.236848in}}%
\pgfpathlineto{\pgfqpoint{3.968076in}{1.994042in}}%
\pgfpathclose%
\pgfusepath{fill}%
\end{pgfscope}%
\begin{pgfscope}%
\pgfpathrectangle{\pgfqpoint{0.017500in}{0.017500in}}{\pgfqpoint{6.218282in}{3.580372in}} %
\pgfusepath{clip}%
\pgfsetbuttcap%
\pgfsetroundjoin%
\definecolor{currentfill}{rgb}{0.892138,0.425389,0.333289}%
\pgfsetfillcolor{currentfill}%
\pgfsetlinewidth{0.000000pt}%
\definecolor{currentstroke}{rgb}{0.000000,0.000000,0.000000}%
\pgfsetstrokecolor{currentstroke}%
\pgfsetdash{}{0pt}%
\pgfpathmoveto{\pgfqpoint{3.968076in}{1.994042in}}%
\pgfpathlineto{\pgfqpoint{4.226975in}{2.236848in}}%
\pgfpathlineto{\pgfqpoint{3.968076in}{1.994042in}}%
\pgfpathclose%
\pgfusepath{fill}%
\end{pgfscope}%
\begin{pgfscope}%
\pgfpathrectangle{\pgfqpoint{0.017500in}{0.017500in}}{\pgfqpoint{6.218282in}{3.580372in}} %
\pgfusepath{clip}%
\pgfsetbuttcap%
\pgfsetroundjoin%
\definecolor{currentfill}{rgb}{0.892138,0.425389,0.333289}%
\pgfsetfillcolor{currentfill}%
\pgfsetlinewidth{0.000000pt}%
\definecolor{currentstroke}{rgb}{0.000000,0.000000,0.000000}%
\pgfsetstrokecolor{currentstroke}%
\pgfsetdash{}{0pt}%
\pgfpathmoveto{\pgfqpoint{3.968076in}{1.994042in}}%
\pgfpathlineto{\pgfqpoint{4.226975in}{2.236848in}}%
\pgfpathlineto{\pgfqpoint{3.968076in}{1.994042in}}%
\pgfpathclose%
\pgfusepath{fill}%
\end{pgfscope}%
\begin{pgfscope}%
\pgfpathrectangle{\pgfqpoint{0.017500in}{0.017500in}}{\pgfqpoint{6.218282in}{3.580372in}} %
\pgfusepath{clip}%
\pgfsetbuttcap%
\pgfsetroundjoin%
\definecolor{currentfill}{rgb}{0.892138,0.425389,0.333289}%
\pgfsetfillcolor{currentfill}%
\pgfsetlinewidth{0.000000pt}%
\definecolor{currentstroke}{rgb}{0.000000,0.000000,0.000000}%
\pgfsetstrokecolor{currentstroke}%
\pgfsetdash{}{0pt}%
\pgfpathmoveto{\pgfqpoint{3.968076in}{1.994042in}}%
\pgfpathlineto{\pgfqpoint{4.226975in}{2.236848in}}%
\pgfpathlineto{\pgfqpoint{3.968076in}{1.994042in}}%
\pgfpathclose%
\pgfusepath{fill}%
\end{pgfscope}%
\begin{pgfscope}%
\pgfpathrectangle{\pgfqpoint{0.017500in}{0.017500in}}{\pgfqpoint{6.218282in}{3.580372in}} %
\pgfusepath{clip}%
\pgfsetbuttcap%
\pgfsetroundjoin%
\definecolor{currentfill}{rgb}{0.892138,0.425389,0.333289}%
\pgfsetfillcolor{currentfill}%
\pgfsetlinewidth{0.000000pt}%
\definecolor{currentstroke}{rgb}{0.000000,0.000000,0.000000}%
\pgfsetstrokecolor{currentstroke}%
\pgfsetdash{}{0pt}%
\pgfpathmoveto{\pgfqpoint{3.968076in}{1.994042in}}%
\pgfpathlineto{\pgfqpoint{4.226975in}{2.236848in}}%
\pgfpathlineto{\pgfqpoint{3.968076in}{1.994042in}}%
\pgfpathclose%
\pgfusepath{fill}%
\end{pgfscope}%
\begin{pgfscope}%
\pgfpathrectangle{\pgfqpoint{0.017500in}{0.017500in}}{\pgfqpoint{6.218282in}{3.580372in}} %
\pgfusepath{clip}%
\pgfsetbuttcap%
\pgfsetroundjoin%
\definecolor{currentfill}{rgb}{0.892138,0.425389,0.333289}%
\pgfsetfillcolor{currentfill}%
\pgfsetlinewidth{0.000000pt}%
\definecolor{currentstroke}{rgb}{0.000000,0.000000,0.000000}%
\pgfsetstrokecolor{currentstroke}%
\pgfsetdash{}{0pt}%
\pgfpathmoveto{\pgfqpoint{3.968076in}{1.994042in}}%
\pgfpathlineto{\pgfqpoint{4.226975in}{2.236848in}}%
\pgfpathlineto{\pgfqpoint{3.968076in}{1.994042in}}%
\pgfpathclose%
\pgfusepath{fill}%
\end{pgfscope}%
\begin{pgfscope}%
\pgfpathrectangle{\pgfqpoint{0.017500in}{0.017500in}}{\pgfqpoint{6.218282in}{3.580372in}} %
\pgfusepath{clip}%
\pgfsetbuttcap%
\pgfsetroundjoin%
\definecolor{currentfill}{rgb}{0.892138,0.425389,0.333289}%
\pgfsetfillcolor{currentfill}%
\pgfsetlinewidth{0.000000pt}%
\definecolor{currentstroke}{rgb}{0.000000,0.000000,0.000000}%
\pgfsetstrokecolor{currentstroke}%
\pgfsetdash{}{0pt}%
\pgfpathmoveto{\pgfqpoint{3.968076in}{1.994042in}}%
\pgfpathlineto{\pgfqpoint{4.226975in}{2.236848in}}%
\pgfpathlineto{\pgfqpoint{3.968076in}{1.994042in}}%
\pgfpathclose%
\pgfusepath{fill}%
\end{pgfscope}%
\begin{pgfscope}%
\pgfpathrectangle{\pgfqpoint{0.017500in}{0.017500in}}{\pgfqpoint{6.218282in}{3.580372in}} %
\pgfusepath{clip}%
\pgfsetbuttcap%
\pgfsetroundjoin%
\definecolor{currentfill}{rgb}{0.892138,0.425389,0.333289}%
\pgfsetfillcolor{currentfill}%
\pgfsetlinewidth{0.000000pt}%
\definecolor{currentstroke}{rgb}{0.000000,0.000000,0.000000}%
\pgfsetstrokecolor{currentstroke}%
\pgfsetdash{}{0pt}%
\pgfpathmoveto{\pgfqpoint{3.968076in}{1.994042in}}%
\pgfpathlineto{\pgfqpoint{4.226975in}{2.236848in}}%
\pgfpathlineto{\pgfqpoint{3.968076in}{1.994042in}}%
\pgfpathclose%
\pgfusepath{fill}%
\end{pgfscope}%
\begin{pgfscope}%
\pgfpathrectangle{\pgfqpoint{0.017500in}{0.017500in}}{\pgfqpoint{6.218282in}{3.580372in}} %
\pgfusepath{clip}%
\pgfsetbuttcap%
\pgfsetroundjoin%
\definecolor{currentfill}{rgb}{0.892138,0.425389,0.333289}%
\pgfsetfillcolor{currentfill}%
\pgfsetlinewidth{0.000000pt}%
\definecolor{currentstroke}{rgb}{0.000000,0.000000,0.000000}%
\pgfsetstrokecolor{currentstroke}%
\pgfsetdash{}{0pt}%
\pgfpathmoveto{\pgfqpoint{3.968076in}{1.994042in}}%
\pgfpathlineto{\pgfqpoint{4.226975in}{2.236848in}}%
\pgfpathlineto{\pgfqpoint{3.968076in}{1.994042in}}%
\pgfpathclose%
\pgfusepath{fill}%
\end{pgfscope}%
\begin{pgfscope}%
\pgfpathrectangle{\pgfqpoint{0.017500in}{0.017500in}}{\pgfqpoint{6.218282in}{3.580372in}} %
\pgfusepath{clip}%
\pgfsetbuttcap%
\pgfsetroundjoin%
\definecolor{currentfill}{rgb}{0.892138,0.425389,0.333289}%
\pgfsetfillcolor{currentfill}%
\pgfsetlinewidth{0.000000pt}%
\definecolor{currentstroke}{rgb}{0.000000,0.000000,0.000000}%
\pgfsetstrokecolor{currentstroke}%
\pgfsetdash{}{0pt}%
\pgfpathmoveto{\pgfqpoint{3.968076in}{1.994042in}}%
\pgfpathlineto{\pgfqpoint{4.226975in}{2.236848in}}%
\pgfpathlineto{\pgfqpoint{3.968076in}{1.994042in}}%
\pgfpathclose%
\pgfusepath{fill}%
\end{pgfscope}%
\begin{pgfscope}%
\pgfpathrectangle{\pgfqpoint{0.017500in}{0.017500in}}{\pgfqpoint{6.218282in}{3.580372in}} %
\pgfusepath{clip}%
\pgfsetbuttcap%
\pgfsetroundjoin%
\definecolor{currentfill}{rgb}{0.892138,0.425389,0.333289}%
\pgfsetfillcolor{currentfill}%
\pgfsetlinewidth{0.000000pt}%
\definecolor{currentstroke}{rgb}{0.000000,0.000000,0.000000}%
\pgfsetstrokecolor{currentstroke}%
\pgfsetdash{}{0pt}%
\pgfpathmoveto{\pgfqpoint{3.968076in}{1.994042in}}%
\pgfpathlineto{\pgfqpoint{4.226975in}{2.236848in}}%
\pgfpathlineto{\pgfqpoint{3.968076in}{1.994042in}}%
\pgfpathclose%
\pgfusepath{fill}%
\end{pgfscope}%
\begin{pgfscope}%
\pgfpathrectangle{\pgfqpoint{0.017500in}{0.017500in}}{\pgfqpoint{6.218282in}{3.580372in}} %
\pgfusepath{clip}%
\pgfsetbuttcap%
\pgfsetroundjoin%
\definecolor{currentfill}{rgb}{0.892138,0.425389,0.333289}%
\pgfsetfillcolor{currentfill}%
\pgfsetlinewidth{0.000000pt}%
\definecolor{currentstroke}{rgb}{0.000000,0.000000,0.000000}%
\pgfsetstrokecolor{currentstroke}%
\pgfsetdash{}{0pt}%
\pgfpathmoveto{\pgfqpoint{3.968076in}{1.994042in}}%
\pgfpathlineto{\pgfqpoint{4.226975in}{2.236848in}}%
\pgfpathlineto{\pgfqpoint{3.968076in}{1.994042in}}%
\pgfpathclose%
\pgfusepath{fill}%
\end{pgfscope}%
\begin{pgfscope}%
\pgfpathrectangle{\pgfqpoint{0.017500in}{0.017500in}}{\pgfqpoint{6.218282in}{3.580372in}} %
\pgfusepath{clip}%
\pgfsetbuttcap%
\pgfsetroundjoin%
\definecolor{currentfill}{rgb}{0.892138,0.425389,0.333289}%
\pgfsetfillcolor{currentfill}%
\pgfsetlinewidth{0.000000pt}%
\definecolor{currentstroke}{rgb}{0.000000,0.000000,0.000000}%
\pgfsetstrokecolor{currentstroke}%
\pgfsetdash{}{0pt}%
\pgfpathmoveto{\pgfqpoint{3.968076in}{1.994042in}}%
\pgfpathlineto{\pgfqpoint{4.226975in}{2.236848in}}%
\pgfpathlineto{\pgfqpoint{3.968076in}{1.994042in}}%
\pgfpathclose%
\pgfusepath{fill}%
\end{pgfscope}%
\begin{pgfscope}%
\pgfpathrectangle{\pgfqpoint{0.017500in}{0.017500in}}{\pgfqpoint{6.218282in}{3.580372in}} %
\pgfusepath{clip}%
\pgfsetbuttcap%
\pgfsetroundjoin%
\definecolor{currentfill}{rgb}{0.892138,0.425389,0.333289}%
\pgfsetfillcolor{currentfill}%
\pgfsetlinewidth{0.000000pt}%
\definecolor{currentstroke}{rgb}{0.000000,0.000000,0.000000}%
\pgfsetstrokecolor{currentstroke}%
\pgfsetdash{}{0pt}%
\pgfpathmoveto{\pgfqpoint{3.968076in}{1.994042in}}%
\pgfpathlineto{\pgfqpoint{4.226975in}{2.236848in}}%
\pgfpathlineto{\pgfqpoint{3.968076in}{1.994042in}}%
\pgfpathclose%
\pgfusepath{fill}%
\end{pgfscope}%
\begin{pgfscope}%
\pgfpathrectangle{\pgfqpoint{0.017500in}{0.017500in}}{\pgfqpoint{6.218282in}{3.580372in}} %
\pgfusepath{clip}%
\pgfsetbuttcap%
\pgfsetroundjoin%
\definecolor{currentfill}{rgb}{0.892138,0.425389,0.333289}%
\pgfsetfillcolor{currentfill}%
\pgfsetlinewidth{0.000000pt}%
\definecolor{currentstroke}{rgb}{0.000000,0.000000,0.000000}%
\pgfsetstrokecolor{currentstroke}%
\pgfsetdash{}{0pt}%
\pgfpathmoveto{\pgfqpoint{3.968076in}{1.994042in}}%
\pgfpathlineto{\pgfqpoint{4.226975in}{2.236848in}}%
\pgfpathlineto{\pgfqpoint{3.968076in}{1.994042in}}%
\pgfpathclose%
\pgfusepath{fill}%
\end{pgfscope}%
\begin{pgfscope}%
\pgfpathrectangle{\pgfqpoint{0.017500in}{0.017500in}}{\pgfqpoint{6.218282in}{3.580372in}} %
\pgfusepath{clip}%
\pgfsetbuttcap%
\pgfsetroundjoin%
\definecolor{currentfill}{rgb}{0.892138,0.425389,0.333289}%
\pgfsetfillcolor{currentfill}%
\pgfsetlinewidth{0.000000pt}%
\definecolor{currentstroke}{rgb}{0.000000,0.000000,0.000000}%
\pgfsetstrokecolor{currentstroke}%
\pgfsetdash{}{0pt}%
\pgfpathmoveto{\pgfqpoint{3.968076in}{1.994042in}}%
\pgfpathlineto{\pgfqpoint{4.226975in}{2.236848in}}%
\pgfpathlineto{\pgfqpoint{3.968076in}{1.994042in}}%
\pgfpathclose%
\pgfusepath{fill}%
\end{pgfscope}%
\end{pgfpicture}%
\makeatother%
\endgroup%

	\caption{Learning surface for SemEval and CONLL2000 Chunk.}
\end{figure}
\newpage
\pagestyle{plain}
\newgeometry{twoside, inner=4.8cm, outer=3.2cm}
