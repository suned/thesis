\section{Information Extraction}
\label{information_extraction}
In natural language processing, information extraction is the problem of extracting structured information from unstructured text. Many practical information extraction problems fall in one of two categories: \textbf{named entity recognition}, or \textbf{relation extraction} \citep{jurafsky09}.

\subsection{Named Entity Recognition}
\label{named_entity_recognition}
A named entity is roughly anything that has a proper name. The task in named entity recognition is to label mentions of entities such as people, organisations or places occurring in natural language. As an example, consider the sentence: 
$$
\text{Jim bought 300 shares of Acme Corp. in 2006.}
$$ 
A named entity recognition system designed to extract the entities \textit{person} and \textit{organisation} should ideally assign the labels:
$$
	[\text{Jim}]_{person} \text{ bought 300 shares of } [\text{Acme Corp.}]_{organisation} \text{ in 2006.}
$$
This is a difficult problem because of two types of ambiguity. Firstly, two distinct entities may share the same name and category, such as \textit{Francis Bacon} the painter and \textit{Francis Bacon} the philosopher. Secondly, two distinct entities can have the same name, but belong to different categories such as \textit{JFK} the former American president and \textit{JFK} the airport near New York.
\\\\
Named entity recognition can be framed as a sequence labelling problem. A common approach is to apply so called tokenisation to the text, i.e finding boundaries between words and punctuation, and associate each token with a label indicating which entity it belongs to. BIO (figure \ref{bio}) is a widely used labelling scheme in which token labels indicate whether the token is at the \textbf{B}eginning, \textbf{I}nside, or \textbf{O}utside an entity mention.
\begin{figure}
	\centering
	\begin{tabular}{c c c c c c c c c c c}
	Jim & bought & 300 & shares & of & Acme & Corp & . & in & 2006 & . \\
	B-PER & O & O & O & O & B-ORG & I-ORG & I-ORG & O & O & O
	\end{tabular}
	\caption{A sentence labelled with the BIO labels for named entity recognition.}
	\label{bio}
\end{figure}

\subsection{Relation Extraction}
\label{relation_extract}
Relation extraction refers to the problem of identifying relationships between entities. As an example, consider the sentence 
$$
\text{Yesterday, New York based Foo Inc. announced their acquisition of Bar Corp.}
$$ 
Imagine we have designed a relation extraction system that recognises the relation \textit{MergerBetween(organisation, organisation)} between two mentions of organisations. Ideally, we would like that system to extract the relation \textit{MergerBetween(Foo Inc., Bar Corp.)} from the above sentence.