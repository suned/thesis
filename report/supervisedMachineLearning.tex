\section{Supervised Machine Learning}
\label{supervised_machine_learning}

Most modern solutions to the information extraction problems in \ref{information_extraction} are based on supervised machine learning techniques. In this setting a system learns to recognize the named entities or relations between them from examples provided by a human annotator. In this section we formally describe this approach and introduce important theoretical tools for understanding it's limitations.

\subsection{The Supervised Learning Problem}
\label{the_supervised_learning_problem}
A training set $\mathcal{D}$ of $N$ examples $(\mathbf{x}_i, \mathbf{y}_i)$ of inputs $\mathbf{x}_i$ and corresponding labels $\mathbf{y}_i$ is created by a human annotator. Each $\mathbf{x}_i$ belongs to an input space $\mathcal{X}$, for example the set of all english sentences. Each $\mathbf{y}_i$ belongs to an output space $\mathcal{Y}$ of labels, for example the set of all sequences of BIO tags. As designers of the learning system we specify a set of functions $h: \mathcal{X} \mapsto \mathcal{Y}$, the so called \textbf{hypothesis space} $\mathcal{H}$. We want to find a function $h \in \mathcal{H}$, sometimes called a \textbf{model} or \textbf{hypothesis}, that can automatically assign labels to a new set of un-labeled inputs $\mathcal{D}_{test} = \{ \mathbf{x}_i \mid \mathbf{x}_i \in \mathcal{X}\}$ at some point in the future. 

Supervised machine learning is the science of how to use an algorithm to find a function $h$ using $\mathcal{D}$ that performs well on $\mathcal{D}_{test}$ as measured by some performance measure $e$. In theoretical analyses of classification problems such as named entity recognition or relation extraction where $\mathcal{Y}$ is discrete, we typically use binary error $e(\mathbf{y}_1, \mathbf{y_2}) = \mathbb{I}[\mathbf{y}_1 \neq \mathbf{y}_2]$. Importantly, we are only interested in the performance of $h$ on $\mathcal{D}$ to the extent that it informs us how the system will perform on future data \citep{yaser12}.
\\\\
We can formalize the preference for functions $h$ that perform well on examples outside of the training set with a quantity known as \textbf{generalization error}.

\begin{definition}[generalization error] \label{generalisation_error}
	Let $P(\mathbf{x}, \mathbf{y})$ be a joint probability distribution over inputs $\mathbf{x} \in \mathcal{X}$ and labels $\mathbf{y} \in \mathcal{Y}$. Let $e(\mathbf{y}_1, \mathbf{y_2})$ be an error function that measures agreement between labels $\mathbf{y}_1$ and $\mathbf{y}_2$. Then the generalization error $E$ of a function $h: \mathcal{X} \mapsto \mathcal{Y}$ is defined as:
	$$
		E(h) = \mathbb{E}_{\mathbf{x},\mathbf{y}\sim P(\mathbf{x}, \mathbf{y})}[e(h(\mathbf{x}), \mathbf{y})]
	$$
\end{definition}
\noindent
Formally, the objective of supervised machine learning is to find a function $h^*$ in a space of functions $\mathcal{H}$ that minimizes $E(h)$. We see the process generating the data as random, but with a behavior describable by a distribution $P(\mathbf{x}, \mathbf{y})$. Unfortunately, this distribution is unknown which makes $E$ unknown. However, we can use sampled data $\mathcal{S} = \{(\mathbf{x}, \mathbf{y}) \mid \mathbf{x}, \mathbf{y} \sim P(\mathbf{x}, \mathbf{y})\}$ to estimate $E(h)$ with a quantity known as \textbf{empirical error}:

\begin{definition}[empirical error] \label{empirical_error}
	Let $\mathcal{S}$ be a set of $N$ examples $\{(\mathbf{x}_i, \mathbf{y}_i) \mid \mathbf{x}_i, \mathbf{y}_i \sim P(\mathbf{x}, \mathbf{y})\}$. Then the empirical error $\hat{E}$ is defined as:
	$$
		\hat{E}(h, \mathcal{S}) = \frac{1}{N}\sum\limits_{i=1}^N e(h(\mathbf{x}_i), \mathbf{y}_i)
	$$
\end{definition}

Because $\mathcal{S}$ is a random quantity, it's dangerous to use $\hat{E}$ to estimate $E$. We risk that the samples are not representative of $P(\mathbf{x}, \mathbf{y})$, leading us to believe that $h$ is great, when in fact it's terrible. We can bound the probability that $\hat{E}$ is a bad estimate of $E$ if we make two assumptions:
\begin{enumerate}
	\item The samples in $\mathcal{S}$ are drawn independently from $P(\mathbf{x}, \mathbf{y})$. In other words, observing any one sample did not change the probability of observing any other sample.
	\item $h$ is independent of $\mathcal{S}$. In other words, $h$ was not specifically chosen based on the sample.
\end{enumerate}
\noindent
These assumptions enable us to apply \textbf{Hoeffding's inequality} to bound the probability that $\hat{E}$ is far away from $E$:

\begin{theorem}[Hoeffding's inequality]
	let $E(h)$ be defined as in definition \ref{generalisation_error}, and let $E(h, \mathcal{S})$ be defined as in definition \ref{empirical_error}. Then:
	$$
	\mathbb{P}\left( |E(h) - \hat{E}(h, \mathcal{S})| \geq \epsilon \right) \leq 2e^{-2N\epsilon^2}
	$$
\end{theorem}

The inequality tells us that the probability that $E$ is more than $\epsilon$ away from $\hat{E}$ decreases exponentially in $\epsilon$ and $N$. In other words, the more samples in $\mathcal{S}$, the less likely it is that $\hat{E}$ will be misleading. Estimating $E$ with a sample that's independent of $h$ is a technique called \textbf{validation} and will be discussed in section \ref{validation}
\\\\
Because $\mathcal{D}$ is used to select $h$, Hoeffding's inequality doesn't hold and we need more sophisticated techniques to understand the relationship between $\hat{E}$ and $E$. The central question in supervised machine learning is: \textit{how can we best define $\mathcal{H}$ and use $\mathcal{D}$ to make $E$ small?} Answering this question is the objective of a field of research known as \textbf{statistical learning theory}.

\subsection{Statistical Learning Theory}
\label{statistical_learning_theory}
$\mathcal{D}$ is the only information we have about $P(\mathbf{x}, \mathbf{y})$, and therefore also the only information we have about $E$. If our goal is to minimize $E$, a straight-forward idea would be to find a function $g \in \mathcal{H}$ that minimizes the \textbf{training error} $\hat{E}(h, \mathcal{D})$ in the hope that $g$ will also minimize $E$. 

As we argued in section \ref{supervised_machine_learning}, using $\hat{E}$ to estimate $E$ can be misleading. Moreover, because $D$ is used to specifically choose $g$ that makes $\hat{E}$ small, the guarantees provided by Hoeffding's inequality no longer holds, and therefore it may be possible to select $g$ such that $\hat{E}(g, \mathcal{D})$ is small and $E(g)$ is large, even when we have a large number of training examples.

The phenomena where training error is small but generalization error is large is known as \textbf{overfitting} \citep{yaser12}. As the name implies, it's caused by harmful idiosyncrasies of $\mathcal{D}$ that causes us to select a $g$ with a larger $E$ than other functions in $\mathcal{H}$. These idiosyncrasies of $\mathcal{D}$ are ultimately the product of \textbf{noise}.
\\\\
In general, noise comes in two forms. The first form is known as \textbf{stochastic noise}. This type of noise is introduced by variation in the relationship between $\mathbf{x}$ and $\mathbf{y}$ that is irrelevant to the problem we are trying to solve. For example, human error where a human annotator incorrectly labels a piece of text is a common source of stochastic noise in information extraction. Selecting a $g$ that repeats this error is a case of overfitting because $g$ will have lower training error but larger generalization error than another $h$ that doesn't predict the incorrect annotation.
\\\\
The second type of noise is called \textbf{deterministic noise}. This type of noise may be introduced when the relationship between $\mathbf{x}$ and $\mathbf{y}$ is deterministic, but $\mathcal{H}$ doesn't have the capacity to represent this relationship exactly.

To understand deterministic noise, imagine that the training data is generated by a deterministic function $f$ such that $\vector{y}_i = f(\vector{x}_i)$. Deterministic noise is present when even $h^*$ can't represent the deterministic relationship exactly. Suppose that we get a $\mathcal{D}$ that contains a sample $(\mathbf{x}_i, \mathbf{y}_i)$ that falls outside the capacity of $h^*$, that is, $h^*(\mathbf{x}_i) \neq \mathbf{y}_i$. Now further imagine that, in order to minimize $\hat{E}$, we select a $g$ that predicts this sample, such that $g(\mathbf{x}_i) = \mathbf{y}_i$. This is a case of overfitting since we know that there is at least one function in $\mathcal{H}$ with lower generalization error than $g$, namely $h^*$.
\\\\
The risk of overfitting is linked to the diversity of $\mathcal{H}$. When we say that $\mathcal{H}$ is diverse, we roughly mean that the functions $h \in \mathcal{H}$ are very different from each other. The more diverse $\mathcal{H}$ is, the greater the risk that there exists a $h \in \mathcal{H}$ that will overfit $\mathcal{D}$.

A \textbf{dichotomy} is a central concept in measuring the diversity of $\mathcal{H}$. A dichotomy is a specific sequence of $N$ labels. For simplicity, most theoretical analyses of $\mathcal{H}$ assume a binary output space $\mathcal{Y} = \{0, 1\}$ and we will too. In that case, if $N = 3$ then (0 1 0) is a dichotomy and so is (1 0 0). We have listed all dichotomies for $N = 3$ in figure \ref{dichotomies}.

\begin{figure}[h]
	\begin{center}
			\begin{tabular}{c}
		(0 0 0) \\
		(1 0 0) \\
		(0 1 0) \\
		(0 0 1) \\
		(1 1 0) \\
		(0 1 1) \\
		(1 0 1) \\
		(1 1 1) \\
	\end{tabular}
	\end{center}
	\caption{All dichotomies for $\mathcal{Y} = \{0, 1\}$ and $N = 3$. There are $2^3 = 8$ ways to choose a sequence of 3 labels from 2 possibilities.}
	\label{dichotomies}
\end{figure}

Dichotomies allow us to group similar functions. By simple combinatorics the number of dichotomies for $N$ must be smaller than or equal to $2^N$ if $\mathcal{Y}$ is binary. There may be infinitely many functions in $\mathcal{H}$, but on a specific $\mathcal{D}$, many of them will produce the same dichotomy since the number of training examples in $\mathcal{D}$ is finite.
This allows us to quantify the diversity of $\mathcal{H}$ in terms of the number of dichotomies it's able to realize on a set of $N$ points. This is achieved by a measure known as the \textbf{growth function}.
\begin{definition}[growth function]
	\label{growth_function}
	Let $\mathcal{H}(N) = \{(h(\mathbf{x}_1), \dots, h(\mathbf{x}_N))\ \mid h \in \mathcal{H}, \mathbf{x}_i \in \mathcal{X}\}$ be the set of all dichotomies generated by $\mathcal{H}$ on $N$ points, and let $|\cdot|$ be the set cardinality function. Then the growth function $m$ is:
	$$
		m(N, \mathcal{H}) = \max |\mathcal{H}(N)|
	$$
\end{definition}
In words, the growth function measures the maximum number of dichotomies that are realizable by $\mathcal{H}$ on $N$ points. To compute $m(N, \mathcal{H})$, we consider any choice of $N$ points from the whole input space $\mathcal{X}$, select the set that realizes the most dichotomies and count them.
\\\\
The growth function allows us to account for redundancy in $\mathcal{H}$. If two functions $h_i \in \mathcal{H}$ and $h_j \in \mathcal{H}$ realise the same dichotomy on $\mathcal{D}$, then any statement based only on $\mathcal{D}$ will be either true or false for for both $h_i$ and $h_j$. This makes it possible to group the events \textit{$\hat{E}(h_i, \mathcal{D})$ is far away from $E(h_i)$} and \textit{$\hat{E}(h_j, \mathcal{D})$ is far away from $E(h_j)$}, and thereby avoiding to overestimate the probability of the union of both events occurring.

If $\mathcal{H}$ is infinite, the number of redundant functions in $\mathcal{H}$ will also be infinite since the number of dichotomies on $N$ points is finite. If $m(N, \mathcal{H})$ is much smaller than $2^N$, the number of redundant functions in $\mathcal{H}$ will be so large as to make the probability that $\hat{E}$ is far away from $E$ very small.
\\\\
 This line of reasoning is the basis of the Vapnik-Chervonenkis bound which bounds $E(h)$ in terms of $\hat{E}(h, \mathcal{D})$ \citep{vapnik1971}:

\begin{theorem}[Vapnik-Chervonenkis bound]
	\label{vc_bound}
	Let $m(N, \mathcal{H})$ be defined as in definition \ref{growth_function}, $E(h)$ as in \ref{generalisation_error}, and $\hat{E}(h, \mathcal{D})$ as in \ref{empirical_error}. Then, with probability $1 - \delta$:
	$$
	E(h) \leq \hat{E}(h, \mathcal{D}) + \sqrt{\frac{8}{N}\ln \frac{4m(2N, \mathcal{H})}{\delta}}
	$$
\end{theorem}
\noindent
The bound tells us that $E(h)$ will be close to $\hat{E}(h, \mathcal{D})$ if $m(N, \mathcal{H})$ is small and $N$ is large. Intuitively, this tells us that a set $\mathcal{H}$ that contains "simple" functions will make it easier to choose $g$ such that generalization error will be close to training error, where simple means: functions that realize a small number of dichotomies. Using a small hypothesis space means we require fewer training examples in order to guarantee that $\hat{E}$ is close to $E$.

On the other hand, having a set $\mathcal{H}$ that can realize a large number of dichotomies on $N$ points, will make it easier to find a function that will make $\hat{E}(h, \mathcal{D})$ small. Using a $\mathcal{H}$ with functions that are too simple is called \textbf{underfitting}. It occurs when we search for a function in the set of functions $\mathcal{H}$, when there is another, more diverse set of functions $\mathcal{G}$ which contain a function with lower generalization error.
\\\\
This analysis tells us that an optimally diverse $\mathcal{H}$ balances the tradeoff between the risk of overfitting, represented in the bound by $m$, and the risk of underfitting, represented by $\hat{E}$. In practice, underfitting is less of a problem than overfitting since modern supervised machine learning algorithms search in extremely diverse spaces of functions $\mathcal{H}$. In fact, most $\mathcal{H}$ are so diverse that steps must be taken to avoid using all of $\mathcal{H}$ when learning from it. These techniques are known as \textbf{regularization}, which we will see an instance of in section \ref{regularisation}.
\\\\
A simple rewrite of theorem \ref{vc_bound} leads to a very popular equivalent formulation: a \textbf{sample complexity} bound. Sample complexity denotes the number of training examples required for a certain level of generalization performance. Exercising a bit of algebra on the Vapnik-Chervonenkis bound leads to the insight that in order for $E$ to be no more than $\epsilon$ away from $\hat{E}$ with probability $1 - \delta$, it requires:
$$
N \geq \frac{1}{\epsilon^2}\ln \frac{4m(2N, \hypspace)}{\delta}
$$
As we will see in part \ref{multi-task_learning}, sample complexity bounds will prove to be valuable tools for understanding how multi-task learning can help reduce the annotation burden for new supervised machine learning problems.

\subsection{Validation}
\label{validation}
Statistical learning theory tells us how to design $\mathcal{H}$ given a dataset by revealing the relationship between $\trerror{h}$ and $E(h)$. While Vapnik-Chervonenkis analysis gives us a theoretical bound on $E(h)$, we may be interested in getting a concrete empirical estimate of $E$, for example in order to decide whether a system is good enough to be put in to production.
\\\\
In general, $\mathcal{D}$ is unsuited for this estimation because of \textbf{bias}: we use $\mathcal{D}$ to specifically select $g$ to minimize $\trerror{h}$, and so the performance of $g$ on $\mathcal{D}$ is likely an optimistic estimate of $E$.

In order to get an unbiased empirical estimate of $E$ we can split $\mathcal{D}$ into two datasets: $\dval$ containing $V$ samples and $\dtrain$ containing $N - V$ samples. $\dtrain$ is used by the learning system to find a function $g^- \in \mathcal{H}$. The minus superscript indicates that the function was selected using only a subset of $\mathcal{D}$. $\dval$ is used to compute $\valerror{g^-}$ as an unbiased estimate of $E$.
\\\\
We can use Vapnik-Chervonenkis analysis to bound the error of $\valerror{g^-}$ as an estimate of $E(g^-)$ \citep{yaser12}. We can view $\dval$ as a training set, which we use to search a hypothesis space containing just $g^-$. This leads to:
$$
E(g^-) \leq \valerror{g^-} + O(\frac{1}{\sqrt{V}})
$$
The inequality tells us that $V$ should be large in order for $\valerror{g^-}$ to be close to $E(g^-)$. This presents a problem since increasing the size of $V$ decreases the number of examples available for training. Though hard to prove theoretically, it's empirically well documented that more training data lead to lower generalization error \citep{yaser12}. In other words, making $V$ large will lead to a very accurate estimate of a very poor hypothesis.
\\\\
\textbf{Cross validation} is a technique that may be used to overcome this dilemma. In this setting, $\mathcal{D}$ is split into $K$ parts called \textbf{folds} containing $\frac{N}{K}$ samples each. $\dtrain$ is then composed of $K - 1$ folds, and the remaining fold is used as $\dval$. This leads to $K$ iterations of the learning procedure yielding $K$ hypotheses $g^-_k$ and $K$ estimates of generalization error $e_k = \valerror{g^-_k}$. We define the cross-validation error as the average of these estimates:
$$
E_{cv} = \frac{1}{K}\sum\limits_{k=1}^K e_k
$$
We want to know $E$, but it would be almost as useful to know the expected $E$ of our learning system when trained on any dataset $\mathcal{D}$ drawn from $P(\vector{x}, \vector{y})$ of size $N$. For this purpose, we can define 
$$
\tilde{E}(N) = \mathbb{E}_{\mathcal{D}}[E(g)]
$$
In words, the expected generalization error with respect to datasets of size $N$. The expected value of $E_{cv}$ is $\tilde{E}(N - \frac{N}{K})$. To see why, consider the expected value of a single estimate $e_k$:
$$
\mathbb{E}[e_k] = \mathbb{E}_{\mathcal{D}_{train}}\mathbb{E}_{\mathcal{D}_{val}}[\valerror{g^-_k}] = \mathbb{E}_{\mathcal{D}_{train}}[E(g^-_k)] = \tilde{E}(N - \frac{N}{K})
$$
Since the equality holds for a single estimate, it also holds for the average $E_{cv}$ \citep{yaser12}.

In words, the cross validation error estimates the expected generalization error of the learning system when trained on $N - \frac{N}{K}$ samples. We can control the number of samples available for finding each function $g^-_k$ by increasing $K$, at no cost of estimation accuracy. However, increasing $K$ increases computation time since we have to search the hypothesis space $K$ times.