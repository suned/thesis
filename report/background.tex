\chapter{Background}
\label{background}

In this part, we exemplify the information extraction problem and formally describe the supervised machine learning and multi-task learning setting. In addition, we describe neural networks, the concept of representation learning and its relevance to multi-task learning.

\section{Information Extraction}
\label{information_extraction}
In natural language processing, information extraction is the problem of extracting structured information from unstructured text. Many practical information extraction problems fall in one of two categories: \textbf{named entity recognition} or \textbf{relation extraction} \citep{jurafsky09}. We introduce each of them in this section, and explain the challenges they pose.

\subsection{Named Entity Recognition}
\label{named_entity_recognition}
A named entity is roughly anything that has a proper name. The goal of named entity recognition is to label mentions of entities such as people, organizations or places occurring in natural language. The list of things these systems are tasked with recognizing is often extended to include things that aren't technically named entities such as amounts of money or calendar dates.

As an example, consider the sentence: 
$$
\text{Jim bought 300 shares of Acme Corp. in 2006.}
$$ 
A named entity recognition system designed to extract the entities \textit{person} and \textit{organization} should ideally assign the labels:
$$
	[\text{Jim}]_{person} \text{ bought 300 shares of } [\text{Acme Corp.}]_{organization} \text{ in 2006.}
$$
This is a difficult problem because of two types of ambiguity. Firstly, two distinct entities may share the same name and category, such as \textit{Francis Bacon} the painter and \textit{Francis Bacon} the philosopher. Secondly, two distinct entities can have the same name, but belong to different categories such as \textit{JFK} the former American president and \textit{JFK} the airport near New York. This means that named entity recognition systems need to have some model of the context in which these entities appear in order to produce correct output.
\\\\
Named entity recognition can be framed as a sequence labeling problem. A common approach is to apply so called tokenization to the text, i.e finding boundaries between words and punctuation, and associate each token with a label indicating which entity it belongs to. BIO-labeling (figure \ref{bio}) is a widely used labeling scheme in which token labels indicate whether the token is at the \textbf{B}eginning, \textbf{I}nside, or \textbf{O}utside an entity mention.
\begin{figure}
	\begin{center}
		\begin{tabular}{c c c c c c c c c c c}
	Jim & bought & 300 & shares & of & Acme & Corp & . & in & 2006 & . \\
	\texttt{B-PER} & \texttt{O} & \texttt{O} & \texttt{O} & \texttt{O} & \texttt{B-ORG} & \texttt{I-ORG} & \texttt{I-ORG} & \texttt{O} & \texttt{O} & \texttt{O}
	\end{tabular}
	\end{center}
	\caption{A sentence labeled with BIO labels for named entity recognition.}
	\label{bio}
\end{figure}

\subsection{Relation Extraction}
\label{relation_extract}
The goal of relation extraction is to identify relationships such as \textit{Family} or \textit{Employment} in natural language. The set of relations we would like a relation extraction system to recognize is commonly referred to as the \textbf{inventory}. Most often, the inventory is limited to relations between named entities. In some relation extraction tasks however, the goal is to more generally recognize relations between nouns \citep{hendrickx2009}. In both cases, the words between which a relation exists are referred to as the \textbf{arguments} of the relation.
\\\\
As an example, consider the sentence: 
$$
\text{Yesterday, New York based Foo Inc. announced their acquisition of Bar Corp.}
$$ 
Imagine we have designed a relation extraction system that recognizes the relation \textit{MergerBetween(organization, organization)} between two mentions of organizations. Ideally, we would like that system to extract the relation \textit{MergerBetween(Foo Inc., Bar Corp.)} from the above sentence.
\\\\
To simplify the relation extraction problem, it's often solved in three steps:
\begin{enumerate}
	\item \textbf{Named entity recognition} \enspace Identify the named entities in the input text.
	\item \textbf{Relation detection} \enspace For each pair of named entities in the input text, determine if a relation exists between them. This is a binary classification problem where the input is the text and the named entities detected in step 1, and the output is yes/no.
	\item \textbf{Relation classification} \enspace Classify each of the detected relations in the previous step. This a multi-label classification problem where the input is the input text and the named entities for which a relation was detected in step 2, and the output is a relation label.
\end{enumerate}
In this thesis we focus on step 3: assigning labels to detected relations. This is a difficult problem because of ambiguity. As an example, consider the sentence \textit{Susan left JFK}. Imagine that we want to design a relation extraction system that can detect the relations \textit{Physical(person, location): a person has a physical relation to a location} and \textit{Personal-Social(person, person): two persons have a social relation}. Both can reasonably be assigned the previous sentence, depending on whether \textit{JFK} refers to the airport near New York, or the former American president. Just as in named entity recognition, providing the correct label in this situation depends on context information.
\\\\
Early relation extraction systems relied on hand-crafted lexical and syntactic rules for detecting relations. \citet{hearst1992} is perhaps the earliest example of this approach. She considers the following sentence:
\begin{quote}
	\textit{Agar is a substance prepared from a mixture of red algae such as Gelidium for laboratory or industrial use.}
\end{quote}
Most people won't know what \textit{Gelidium} is. From the context we can infer that it's a type of algae however. She suggest that the following lexico-syntactic pattern between two noun phrases $NP_1$ and $NP_2$:
$$
NP_1\text{ such as }NP_2
$$
implies the relation $Hyponym(NP_1, NP_2)$. By performing a syntactic parse of the input sentence we can try to extract hyponym relations between noun phrases using such manually created rules. Because of the huge amount of variation found in natural languages, this is of course a cumbersome yet brittle approach.
\\\\
More recent solutions rely on supervised machine learning techniques to solve relation extraction problems. In this setting, a system learns to recognize relations in the inventory from annotated examples. The earliest examples of such systems relied on hand-crafted features of words in the neighborhood of the relation arguments, for example: \textit{the words between the arguments are "such as"} \citep{jurafsky09}. As we will see in part \ref{neural_networks}, the promise of avoiding complicated hand-crafted features of the sentence, but having the system learn useful lexico-syntactic features on its own is the major attraction of solutions based on neural networks.

\subsection{Accuracy Measures}
Information extraction systems are often evaluated empirically by applying them to collections of text, so called corpora, in which $N$ mentions of named entities or relations are known. In these tests, accuracy measures for each class $c$ of information we wish to extract are usually defined in terms of how many times the system correctly predicted class $c$. Most metrics use the following terminology:

\begin{center}
	\begin{tabular}{r | c c}
	 & \textbf{predicted as $c$} & \textbf{predicted as not $c$}  \\ \hline
	$c$ & True positives ($tp$) & False negatives ($fn$) \\
	\textbf{not} $c$ & False positives ($fp$) & True negatives ($tn$)
\end{tabular}
\end{center}
Where for example $tp$ is the number of true positives produced for class $c$.
\\\\
The distribution of labels used in both named entity recognition and relation extraction is often highly imbalanced. Consider for example the BIO labelling scheme for named entity recognition in figure \ref{bio}. Most words will be outside a mention of a named entity, and will have the label \texttt{O}. Using simple accuracy $\frac{tp + tn}{tp + tn + fn + fp}$ as a performance metric for a system that outputs bio labels for each token in the text is therefore not very informative, since a useless system which labels all tokens with \texttt{O} would achieve high performance.
\\\\
\textbf{Precision} and \textbf{recall} are more appropriate performance metrics for this reason. Precision $\frac{tp}{tp + fp}$ is the fraction of information items for which the system predicted class $c$ that actually belonged to class $c$.
Recall $\frac{tp}{tp + fn}$ on the other hand is the fraction of information items in the corpora of class $c$ that the system correctly extracted.
\\\\
In a multi-class classification problem we are forced to decide how to average these metrics across classes. Specifically, there are two ways of averaging an accuracy measure across $C$ different classes: micro and macro averaging \citep{sokolova2009}. In macro averaging, an accuracy measure is computed for each class $c$ separately, and then averaged across all $C$ classes. For example macro-precision $p_{M}$:
$$
p_{M} = \frac{1}{C}\sum_{c=1}^C p_c
$$
Where $p_c$ is the precision of the system for class $c$. Micro averaging on the other hand, averages an accuracy by accumulating $tp$, $tn$, $fp$ and $fn$ across all $C$ classes. For example micro-precision $p_{\mu}$:
$$
p_\mu = \frac{\sum\limits_{c=1}^C tp_c}{\sum\limits_{c=1}^C tp_c + fp_c}
$$
Where for example $tp_c$ is the true positives a system produces for class $c$.
\\\\
The main difference between macro and micro averages of accuracy measures is that micro averaging gives more weight to more frequent classes. In other words, micro averaging encodes the bias that infrequent classes are unimportant, and a misclassification of an example of such a class should not penalise the accuracy measure as much as a misclassification of a more frequent class. Whether on not this a reasonable bias depends on the problem. In order to be agnostic about the frequency of semantic relations we use macro averaging for all our reporting in this thesis.
\\\\
To get a single number that summarizes the performance, precision $p$ and recall $r$ are often combined into a single metric: the $F1$ measure. $F1$ is defined as the harmonic mean of precision and recall $\frac{2pr}{p + r}$. Variations that use the micro and macro versions of precision and recall can naturally be computed as the harmonic mean of the micro or macro precision and recall respectively.
\section{Supervised Machine Learning}
\label{supervised_machine_learning}

Most modern solutions to the information extraction problems in \ref{information_extraction} are based on supervised machine learning techniques. In this setting, a system learns to recognise the named entities or relations between them from examples provided by a human annotator. In this section we formally describe this approach and introduce important theoretical tools for understanding supervised machine learning.

\subsection{The Supervised Learning Problem}
\label{the_supervised_learning_problem}
A set $\mathcal{D}_{train}$ of $N$ training examples $(\mathbf{x}_i, \mathbf{y}_i)$ of inputs $\mathbf{x}_i$ and corresponding labels $\mathbf{y}_i$ is created by a human annotator. Each $\mathbf{x}_i$ belongs to an input space $\mathcal{X}$, for example the set of all english sentences. Each $\mathbf{y}_i$ belongs to a space $\mathcal{Y}$ of labels, for example the set of all sequences of BIO tags. As designers of the learning system, we specify the so called \textbf{hypothesis space} $\mathcal{H}$, a set of functions $h: \mathcal{X} \mapsto \mathcal{Y}$. We want to find a function $h \in \mathcal{H}$, sometimes called a \textbf{model} or \textbf{hypothesis}, that can automatically assign labels to a new set of un-labeled inputs $\mathcal{D}_{test} = \{ \mathbf{x}_i \mid \mathbf{x}_i \in \mathcal{X}\}$ at some point in the future. 

Supervised machine learning is the science of how to use an algorithm to find a function $h$ using $\mathcal{D}_{train}$ that performs well on $\mathcal{D}_{test}$, as measured by some performance measure $e$. In classification problems such as named entity recognition or relation extraction where $\mathcal{Y}$ is discrete, we typically use binary error $e(\mathbf{y}_1, \mathbf{y_2}) = \mathbb{I}[\mathbf{y}_1 \neq \mathbf{y}_2]$. Importantly, we are not explicitly interested in the performance of $h$ on $\mathcal{D}_{train}$ \citep{yaser12}.
\\\\
We can formalise the preference for functions $h$ that perform well on examples outside of the training set with a quantity known as \textbf{generalisation error}.

\begin{definition}[generalisation error] \label{generalisation_error}
	Let $P(\mathbf{x}, \mathbf{y})$ be a joint probability distribution over inputs $\mathbf{x} \in \mathcal{X}$ and labels $\mathbf{y} \in \mathcal{Y}$. Let $e(\mathbf{y}_1, \mathbf{y_2}) = \mathbb{I}[\mathbf{y}_1 \neq \mathbf{y}_2]$ be the binary error function that measures agreement between labels $\mathbf{y}_1$ and $\mathbf{y}_2$. Then the generalisation error $E$ of a function $h: \mathcal{X} \mapsto \mathcal{Y}$ is defined as:
	$$
		E(h) = \mathbb{E}_{\mathbf{x},\mathbf{y}\sim P(\mathbf{x}, \mathbf{y})}[e(h(\mathbf{x}), \mathbf{y})]
	$$
\end{definition}
Now, formally, the objective of supervised machine learning is to find a function $h^*$ in a space of functions $\mathcal{H}$ that minimises $E(h)$. We see the process generating the data as random, but with a behaviour describable by a distribution $P(\mathbf{x}, \mathbf{y})$. Unfortunately, this distribution is unknown, which makes $E$ unknown. However, we can use sampled data $\mathcal{D} = \{(\mathbf{x}, \mathbf{y}) \mid \mathbf{x}, \mathbf{y} \sim P(\mathbf{x}, \mathbf{y})\}$ to estimate $E(h)$ with a quantity known as \textbf{empirical error}:

\begin{definition}[empirical error] \label{empirical_error}
	Let $\mathcal{D}$ be a set of $N$ examples $\{(\mathbf{x}_i, \mathbf{y}_i) \mid \mathbf{x}_i, \mathbf{y}_i \sim P(\mathbf{x}, \mathbf{y})\}$. Then the empirical error $\hat{E}$ is defined as:
	$$
		\hat{E}(h, \mathcal{D}) = \frac{1}{N}\sum\limits_{i=1}^N e(h(\mathbf{x}_i), \mathbf{y}_i)
	$$
\end{definition}

Because $\mathcal{D}$ is a random quantity, it's dangerous to use $\hat{E}$ to estimate $E$. We risk that the samples are not representative of $P(\mathbf{x}, \mathbf{y})$, leading us to believe that $h$ is great, when in fact it's terrible. We can bound the probability that $\hat{E}$ is a bad estimate of $E$ if we make two assumptions:

Firstly, we assume that the samples in $\mathcal{D}$ are drawn independently from $P(\mathbf{x}, \mathbf{y})$, that is observing any one sample did not change the probability of observing any other sample.

Secondly, we assume that $h$ is independent of $\mathcal{D}$, in other words, that $h$ was not specifically chosen based on the sample. These assumptions enable us to apply \textbf{Hoeffding's inequality} to bound the probability that $\hat{E}$ is far away from $E$:

\begin{theorem}[Hoeffding's inequality]
	let $E(h)$ be defined as in definition \ref{generalisation_error}, and let $E(h, \mathcal{D})$ be defined as in definition \ref{empirical_error}. Then:
	$$
	\mathbb{P}\left( |E(h) - \hat{E}(h, \mathcal{D})| \geq \epsilon \right) \leq 2e^{-2N\epsilon^2}
	$$
\end{theorem}

The inequality tells us that the probability that $E$ is more than $\epsilon$ away from $\hat{E}$ decreases exponentially in $\epsilon$ and $N$. In other words, the more samples in $D$, the less likely it is that $E$ will be misleading.
\\\\
Estimating $E$ with a sample that is independent of $h$ is a technique called \textbf{validation}. In validation, the sample provided by a human annotator is split into two datasets, $\mathcal{D}_{train}$, which we intend to use to search for $h^*$, and $\mathcal{D}_{validate}$, which saved until we are done searching. Since $\mathcal{D}_{validate}$ is independent of whichever $h$ we selected, Hoeffding's inequality applies and $\mathcal{D}_{validate}$ can be used to estimate $E$.
\\\\
Because $\mathcal{D}_{train}$ is used to select $h$, it cannot be used to estimate $E$ by Hoeffding's inequality, and we need more sophisticated techniques to understand the relationship between $\mathcal{D}_{train}$ and $E$. The central question in supervised machine learning is \textit{how can we best define $\mathcal{H}$ and use $\mathcal{D}_{train}$ to make $E$ small?} Answering this question is the objective of a field of research known as \textbf{statistical learning theory}.

\subsection{Validation}

\subsection{Statistical Learning Theory}
\label{statistical_learning_theory}
We would like to know how best to define $\mathcal{H}$ and use $\mathcal{D}_{train}$ in order to make $E$ small. $\mathcal{D}_{train}$ is the only information we have about $P(\mathbf{x}, \mathbf{y})$, and therefore also the only information we have about $E$. A straight-forward idea would be to find a function $g \in \mathcal{H}$ that minimises the \textbf{training error} $\hat{E}(g, \mathcal{D}_{train})$ in the hope that $g$ will also minimise $E$. 

As we argued in section \ref{supervised_machine_learning}, using $\hat{E}$ to estimate $E$ can be misleading. Moreover, because $D_{train}$ is used to specifically choose $g$ that makes $\hat{E}$ small, the guarantees provided by Hoeffding's inequality no longer holds, and therefore it may be possible to select $g$ such that $\hat{E}(g, \mathcal{D}_{train})$ is small and $E(g)$ is large, even when we have a large number of training examples.

The phenomena where training error is small but generalisation error is large is known as \textbf{overfitting}. As the name implies, it's caused by harmful idiosyncrasies of $\mathcal{D}_{train}$ that, when used to minimise $\hat{E}(h, \mathcal{D}_{train})$, leads us to a $g$ with a larger $E$ than other functions in $\mathcal{H}$. These idiosyncrasies of $\mathcal{D}_{train}$ are ultimately the product of \textbf{noise}.
\\\\
In general, noise comes in two forms. The first form is known as \textbf{stochastic noise}. This type of noise is introduced by variation in the relationship between $\mathbf{x}$ and $\mathbf{y}$ that is inherently unpredictable. For example, human error is a common source of stochastic noise in information extraction, where an annotator incorrectly labels a piece of text. Selecting a $g$ that repeats this error is a case of overfitting, because $g$ will have lower training error but larger generalisation error than another $h$ that doesn't predict the incorrect annotation, since presumably the error is the exception to the rule.
\\\\
The second type of noise is called \textbf{deterministic noise}. This type of noise may be introduced when the relationship between $\mathbf{x}$ and $\mathbf{y}$ is deterministic, but $\mathcal{H}$ doesn't have the capacity to represent this relationship exactly.

To understand deterministic noise, imagine that even $h^*$ can't represent the deterministic relationship $\mathbf{y} = f(\mathbf{x})$ exactly. Suppose that we get a $\mathcal{D}_{train}$ that contains a sample $(\mathbf{x}_i, \mathbf{y}_i)$ that falls outside the capacity of $h^*$, that is, $h^*(\mathbf{x}_i) \neq \mathbf{y}_i$. Now further imagine that in order to minimise $\hat{E}$, we select a $g$ that predicts this sample, such that $h(\mathbf{x}_i) = \mathbf{y}_i$. This is a case of overfitting since we know that there is at least one function in $\mathcal{H}$ with lower generalisation error than $g$, namely $h^*$.
\\\\
The risk of overfitting is linked to the diversity of $\mathcal{H}$. By diversity of $\mathcal{H}$, we roughly mean how different any function in $\mathcal{H}$ is from any other function in $\mathcal{H}$. The more diverse $\mathcal{H}$ is, the greater the risk that there exists a $h \in \mathcal{H}$ that will overfit $\mathcal{D}_{train}$.

A \textbf{dichotomy} is a central concept in measuring the diversity of $\mathcal{H}$. A dichotomy is a specific sequence of $N$ labels. For example, if $\mathcal{Y} = \{0, 1\}$, and $N = 3$, then (0 1 0) is a dichotomy, and so is (1 0 0). We have listed all dichotomies for $N = 3$ in figure \ref{dichotomies}.

\begin{figure}[h]
	\begin{center}
			\begin{tabular}{c}
		(0 0 0) \\
		(1 0 0) \\
		(0 1 0) \\
		(0 0 1) \\
		(1 1 0) \\
		(0 1 1) \\
		(1 0 1) \\
		(1 1 1) \\
	\end{tabular}
	\end{center}
	\caption{All dichotomies for $\mathcal{Y} = \{0, 1\}$ and $N = 3$. There are $2^3 = 8$ ways to choose a sequence of 3 labels from 2 possibilities.}
	\label{dichotomies}
\end{figure}

Dichotomies allow us to group similar functions. In the rest of this section, let's assume that $\mathcal{Y} = \{0, 1\}$. By simple combinatorics the number of dichotomies for $N$ must be smaller than or equal to $2^N$. There may be infinitely many functions in $\mathcal{H}$, but on a specific $\mathcal{D}_{train}$, many of them will produce the same dichotomy since the number of training examples in $\mathcal{D}_{train}$ is finite.
This allows us to quantify the diversity of $\mathcal{H}$ in terms of the number of dichotomies it's able to realise on a set of $N$ points. This is achieved by a measure known as the \textbf{growth function}.
\begin{definition}[growth function]
	\label{growth_function}
	Let $\mathcal{H}(N) = \{(h(\mathbf{x}_1), \dots, h(\mathbf{x}_N))\ \mid h \in \mathcal{H}, \mathbf{x}_i \in \mathcal{X}\}$ be the set of all dichotomies generated by $\mathcal{H}$ on $N$ points, and let $|\cdot|$ be the set cardinality function. Then the growth function $m$ is:
	$$
		m(N, \mathcal{H}) = \max |\mathcal{H}(N)|
	$$
\end{definition}
In words, the growth function measures the maximum number of dichotomies that are realisable by $\mathcal{H}$ on $N$ points. To compute $m(N, \mathcal{H})$, we consider any choice of $N$ points from the whole input space $\mathcal{X}$, select the set that realises the most dichotomies and count them.
\\\\
The growth function allows us to account for redundancy in $\mathcal{H}$. If two functions $h_i \in \mathcal{H}$ and $h_j \in \mathcal{H}$ realise the same dichotomy on $\mathcal{D}$, then any statement based only on $\mathcal{D}$ will be either true or false for for both $h_i$ and $h_j$. This makes it possible to group the events \textit{$\hat{E}(h_i, \mathcal{D})$ is far away from $E(h_i)$} and \textit{$\hat{E}(h_j, \mathcal{D})$ is far away from $E(h_j)$}, and thereby avoiding to overestimate the probability of the union of both events occurring.

If $\mathcal{H}$ is infinite, the number of redundant functions in $\mathcal{H}$ will also be infinite, since the number of dichotomies on $N$ points is finite. If $m(N, \mathcal{H})$ is much smaller than $2^N$, the number of redundant functions in $\mathcal{H}$ will be so large as to make the probability that $\hat{E}$ is far away from $E$ very small.
\\\\
 This line of reasoning is the basis of the Vapnik-Chervonenkis bound, which bounds $E(h)$ in terms of $\hat{E}(h, \mathcal{D}_{train})$:

\begin{theorem}[Vapnik-Chervonenkis bound]
	\label{vc_bound}
	Let $m(N, \mathcal{H})$ be defined as in definition \ref{growth_function}, $E(h)$ as \ref{generalisation_error}, and $\hat{E}(h, \mathcal{D})$ as in \ref{empirical_error}. Then, with probability $1 - \delta$:
	$$
	E(h) \leq \hat{E}(h, \mathcal{D}_{train}) + \sqrt{\frac{8}{N}\ln \frac{4m(2N, \mathcal{H})}{\delta}}
	$$
\end{theorem}
The bound tells us that $E(h)$ will be close to $\hat{E}(h, \mathcal{D}_{train})$ if $m(N, \mathcal{H})$ is small and $N$ is large. Intuitively, this tells us that a set $\mathcal{H}$ that contains "simple" functions will make it easier to choose $g$ such that generalisation error will be close to training error, where simple means: functions that realise a small number of dichotomies.

On the other hand, having a set $\mathcal{H}$ that can realise a large number of dichotomies on $N$ points, will make it easier to find a function that will make $\hat{E}(h, \mathcal{D}_{train})$ small. Using a $\mathcal{H}$ with functions that are too simple is called \textbf{underfitting}. It occurs when we search for a function in the set of functions $\mathcal{H}$, when there is another, more diverse set of functions $\mathcal{G}$ which contain a function with lower generalisation error.
\\\\
This analysis tells us that an optimally diverse $\mathcal{H}$ balances the tradeoff between the risk of overfitting, represented in the bound by $m$, and the risk of underfitting, represented by $\hat{E}$. In practice, underfitting is less of a problem than overfitting, since modern supervised machine learning algorithms search in extremely diverse spaces of functions $\mathcal{H}$. In fact, most $\mathcal{H}$ are so diverse that steps must be taken to avoid minimising $\hat{E}$ as much as is actually possible. These techniques are known as \textbf{regularisation}, which we will see an instance of in section \ref{early_stopping}.

\section{Summary}
In this section we have seen that the purpose of named entity recognition is to identify mentions of entities such as people, organisations and places in natural language. The purpose of relation extraction systems is to identify relationships between them. 

We have seen that simple accuracy is uninformative as an evaluation measure in information extraction, and described the alternative precision and recall.

We have described the formal setting of on supervised machine learning. We have discussed concepts such as overfitting and noise, diversity of the set of functions $\mathcal{H}$ from which to choose $h$, and its impact on training and generalisation error.
