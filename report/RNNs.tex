\section{Word Vectors}
The way text is represented in a computer doesn't in general encode any information about semantic similarities between words or sentences. Instead, text is most often represented as sequences of discrete symbols. Learning a $h$ that maps from discrete input space where distances between points don't encode similarity, such as words, to a prediction, such as the presence of a named entity, may be more difficult than learning a mapping from continuous input space to a prediction, since a continuous function can be expected to have some smoothness properties, i.e similar inputs should have similar outputs \citep{bengio2003}.
\\\\
For this reason some effort has been devoted to designing real-valued vector representation of words, so called \textbf{word vectors}, that encode semantic similarities such that words with similar meaning are close to each other in word-vector space. The notion of "meaning" of a word is a philosophically challenging one. A simple definition which leads to simple but useful algorithms is that words have similar meaning if they are used in similar contexts.

This leads to the idea of representing words as vectors of co-occurrence counts. Two words $w_i$ and $w_j$ co-occur in a context of $c$ words if $w_j$ appears somewhere in a window of $c$ words from $w_i$ in some piece of text. By representing $w_i$ as a vector $\vector{w}_i \in \mathbb{R}^V$ of co-occurrence counts for the $V$ words in some vocabulary, words that occur in similar contexts will be close to each other in co-occurrence vector space.
\\\\
The main problems with this representation is that $V$ may be very large, and $\vector{w}_i$ may be very sparse, that is, most of its components are 0 since most words never co-occur together. Recent solutions to this problem learn lower dimensional word vectors using co-occurrence statistics. \textbf{GloVe} is a recent and successful technique for learning word vectors that encode much useful syntactic and semantic information \citep{pennington2014}. In GloVe, each word $w_i$ is represented by a word vectors $\vector{w}_i$, and a context word vector $\tilde{\vector{w}}_i$. The vectors are initialized randomly

 Glove vectors are learned by minimizing the objective function:
$$
\sum\limits_{i=1}^V\sum\limits_{i=1}^V f(\matrix{X}_{ij})(\vector{w}_i^T\tilde{\vector{w}}_j + b_i + \tilde{b}_j - \ln \matrix{X}_{ij})^2
$$
Where $\matrix{X}_{ij}$ is the co-occurrence count for word $w_i$ and $w_j$, and $b_i$ and $\tilde{b}_j$ are bias terms. $f$ is a weighting function that gives low weight to infrequent terms and caps extremely frequent terms, defined as:
$$
f(x) = \begin{cases}
	(x / x_{max})^{3/4} & \text{when $x < x_{max}$} \\
	1 & \text{otherwise}
\end{cases}
$$
Minimizing this objective leads to word vectors whose dot products are close to log-co-occurrence counts for the words they represent. It can be shown that this has the effect that word vector differences encode information about ratios of log-co-occurrence probabilities which are highly informative of semantic similarity.
\\\\
It is now common practice to incorporate word vectors in neural network models for natural language processing tasks in a so called embedding layer. In particular, words are mapped to word vectors through a word-embedding matrix such that column $i$ in this matrix corresponds to word $i$ in the vocabulary. The weights of the word embedding matrix is then optimized through gradient descent just as any other weight of the network. 

In this scheme, the components of the word vectors are parameters that can be trained by back propagation to yield word vector representations that are informative for a given task. These word embedding vectors, or simply word embeddings, can be initialized with small random components as any other neural network parameter, or they can be initialized with pre-learned word vectors, for example GloVe vectors which often leads to great improvements on a host of language processing tasks \citep{collobert2008, collobert2011, nguyen2015, kim2014, zhang2015}