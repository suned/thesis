\chapter{Neural Networks}
\label{neural_networks}

In this part we describe how to define $\mathcal{H}$ using functions called \textbf{neural networks}. One advantage of these functions is that they're easy to adapt to multi-task learning. We begin by describing how to design $\mathcal{H}$ using neural networks. We then turn to the issue of how to use $\mathcal{D}$ to search this hypothesis space. We then cover regularization techniques the purpose of which is to prevent overfitting. Lastly, we introduce convolutional neural networks which are specialized functions often used for text classification problems such as relation extraction.

\section{Feed-Forward Neural Networks}
\section{Learning Algorithm}
\label{learning_algorithm}
\subsection{Gradient Descent}
\subsection{Backpropagation}
\subsection{Regularisation}
\label{early_stopping}
\subsection{Adam}
\section{Convolutional Neural Networks}
A convolution $f * k$ is a mathematical operation that takes as input two functions $f$ and $k$.

\begin{definition}[convolution] \label{convolution}
	Let $f(x) \in \mathbb{R}$ and $k(x) \in \mathbb{R}$ be two real-valued functions defined for the entire real number line. Then the convolution $f * k$ is defined as
	$$
		(f * k)(x) = \int f(y)k(x - y)dy
	$$
\end{definition}

In practical applications involving computers, $f$ and $k$ are discrete, and the integral turns into a sum:
$$
(f * k)(x) = \sum\limits_{y=-\infty}^\infty f(y)k(x - y)
$$
Most functions in practical applications of convolutions represent signals such as images, sound or text, which are only defined over a limited range of indices $x$. In these cases, it's assumed that whenever $x$ is beyond the domain of $f$ or $k$ the output of either function is 0.
\\\\
We can think of a convolution as a weighted sum of the output of $f$ where the output of $k$ acts as the weights. This view of convolution is used heavily in signal processing applications where $k$ is chosen to produce certain properties in the convolution output such as reducing noise in $f$. In this setting $k$ is often referred to as a \textbf{kernel}. As an example, consider the noisy signal convolved with a gaussian kernel in figure \ref{gaussian_convolution}.

\begin{figure}
	\centering
	%% Creator: Matplotlib, PGF backend
%%
%% To include the figure in your LaTeX document, write
%%   \input{<filename>.pgf}
%%
%% Make sure the required packages are loaded in your preamble
%%   \usepackage{pgf}
%%
%% Figures using additional raster images can only be included by \input if
%% they are in the same directory as the main LaTeX file. For loading figures
%% from other directories you can use the `import` package
%%   \usepackage{import}
%% and then include the figures with
%%   \import{<path to file>}{<filename>.pgf}
%%
%% Matplotlib used the following preamble
%%   \usepackage{fontspec}
%%   \setmainfont{Palatino}
%%   \setsansfont{Lucida Grande}
%%   \setmonofont{Andale Mono}
%%
\begingroup%
\makeatletter%
\begin{pgfpicture}%
\pgfpathrectangle{\pgfpointorigin}{\pgfqpoint{4.940612in}{2.730000in}}%
\pgfusepath{use as bounding box, clip}%
\begin{pgfscope}%
\pgfsetbuttcap%
\pgfsetmiterjoin%
\definecolor{currentfill}{rgb}{1.000000,1.000000,1.000000}%
\pgfsetfillcolor{currentfill}%
\pgfsetlinewidth{0.000000pt}%
\definecolor{currentstroke}{rgb}{1.000000,1.000000,1.000000}%
\pgfsetstrokecolor{currentstroke}%
\pgfsetdash{}{0pt}%
\pgfpathmoveto{\pgfqpoint{0.000000in}{0.000000in}}%
\pgfpathlineto{\pgfqpoint{4.940612in}{0.000000in}}%
\pgfpathlineto{\pgfqpoint{4.940612in}{2.730000in}}%
\pgfpathlineto{\pgfqpoint{0.000000in}{2.730000in}}%
\pgfpathclose%
\pgfusepath{fill}%
\end{pgfscope}%
\begin{pgfscope}%
\pgfsetbuttcap%
\pgfsetmiterjoin%
\definecolor{currentfill}{rgb}{1.000000,1.000000,1.000000}%
\pgfsetfillcolor{currentfill}%
\pgfsetlinewidth{0.000000pt}%
\definecolor{currentstroke}{rgb}{0.000000,0.000000,0.000000}%
\pgfsetstrokecolor{currentstroke}%
\pgfsetstrokeopacity{0.000000}%
\pgfsetdash{}{0pt}%
\pgfpathmoveto{\pgfqpoint{0.273112in}{0.017500in}}%
\pgfpathlineto{\pgfqpoint{4.923112in}{0.017500in}}%
\pgfpathlineto{\pgfqpoint{4.923112in}{2.712500in}}%
\pgfpathlineto{\pgfqpoint{0.273112in}{2.712500in}}%
\pgfpathclose%
\pgfusepath{fill}%
\end{pgfscope}%
\begin{pgfscope}%
\pgfsetbuttcap%
\pgfsetroundjoin%
\definecolor{currentfill}{rgb}{0.000000,0.000000,0.000000}%
\pgfsetfillcolor{currentfill}%
\pgfsetlinewidth{0.803000pt}%
\definecolor{currentstroke}{rgb}{0.000000,0.000000,0.000000}%
\pgfsetstrokecolor{currentstroke}%
\pgfsetdash{}{0pt}%
\pgfsys@defobject{currentmarker}{\pgfqpoint{0.000000in}{-0.048611in}}{\pgfqpoint{0.000000in}{0.000000in}}{%
\pgfpathmoveto{\pgfqpoint{0.000000in}{0.000000in}}%
\pgfpathlineto{\pgfqpoint{0.000000in}{-0.048611in}}%
\pgfusepath{stroke,fill}%
}%
\begin{pgfscope}%
\pgfsys@transformshift{0.484476in}{1.339482in}%
\pgfsys@useobject{currentmarker}{}%
\end{pgfscope}%
\end{pgfscope}%
\begin{pgfscope}%
\pgftext[x=0.484476in,y=1.242260in,,top]{\rmfamily\fontsize{8.000000}{9.600000}\selectfont 0}%
\end{pgfscope}%
\begin{pgfscope}%
\pgfsetbuttcap%
\pgfsetroundjoin%
\definecolor{currentfill}{rgb}{0.000000,0.000000,0.000000}%
\pgfsetfillcolor{currentfill}%
\pgfsetlinewidth{0.803000pt}%
\definecolor{currentstroke}{rgb}{0.000000,0.000000,0.000000}%
\pgfsetstrokecolor{currentstroke}%
\pgfsetdash{}{0pt}%
\pgfsys@defobject{currentmarker}{\pgfqpoint{0.000000in}{-0.048611in}}{\pgfqpoint{0.000000in}{0.000000in}}{%
\pgfpathmoveto{\pgfqpoint{0.000000in}{0.000000in}}%
\pgfpathlineto{\pgfqpoint{0.000000in}{-0.048611in}}%
\pgfusepath{stroke,fill}%
}%
\begin{pgfscope}%
\pgfsys@transformshift{1.495785in}{1.339482in}%
\pgfsys@useobject{currentmarker}{}%
\end{pgfscope}%
\end{pgfscope}%
\begin{pgfscope}%
\pgftext[x=1.495785in,y=1.242260in,,top]{\rmfamily\fontsize{8.000000}{9.600000}\selectfont 5}%
\end{pgfscope}%
\begin{pgfscope}%
\pgfsetbuttcap%
\pgfsetroundjoin%
\definecolor{currentfill}{rgb}{0.000000,0.000000,0.000000}%
\pgfsetfillcolor{currentfill}%
\pgfsetlinewidth{0.803000pt}%
\definecolor{currentstroke}{rgb}{0.000000,0.000000,0.000000}%
\pgfsetstrokecolor{currentstroke}%
\pgfsetdash{}{0pt}%
\pgfsys@defobject{currentmarker}{\pgfqpoint{0.000000in}{-0.048611in}}{\pgfqpoint{0.000000in}{0.000000in}}{%
\pgfpathmoveto{\pgfqpoint{0.000000in}{0.000000in}}%
\pgfpathlineto{\pgfqpoint{0.000000in}{-0.048611in}}%
\pgfusepath{stroke,fill}%
}%
\begin{pgfscope}%
\pgfsys@transformshift{2.507094in}{1.339482in}%
\pgfsys@useobject{currentmarker}{}%
\end{pgfscope}%
\end{pgfscope}%
\begin{pgfscope}%
\pgftext[x=2.507094in,y=1.242260in,,top]{\rmfamily\fontsize{8.000000}{9.600000}\selectfont 10}%
\end{pgfscope}%
\begin{pgfscope}%
\pgfsetbuttcap%
\pgfsetroundjoin%
\definecolor{currentfill}{rgb}{0.000000,0.000000,0.000000}%
\pgfsetfillcolor{currentfill}%
\pgfsetlinewidth{0.803000pt}%
\definecolor{currentstroke}{rgb}{0.000000,0.000000,0.000000}%
\pgfsetstrokecolor{currentstroke}%
\pgfsetdash{}{0pt}%
\pgfsys@defobject{currentmarker}{\pgfqpoint{0.000000in}{-0.048611in}}{\pgfqpoint{0.000000in}{0.000000in}}{%
\pgfpathmoveto{\pgfqpoint{0.000000in}{0.000000in}}%
\pgfpathlineto{\pgfqpoint{0.000000in}{-0.048611in}}%
\pgfusepath{stroke,fill}%
}%
\begin{pgfscope}%
\pgfsys@transformshift{3.518403in}{1.339482in}%
\pgfsys@useobject{currentmarker}{}%
\end{pgfscope}%
\end{pgfscope}%
\begin{pgfscope}%
\pgftext[x=3.518403in,y=1.242260in,,top]{\rmfamily\fontsize{8.000000}{9.600000}\selectfont 15}%
\end{pgfscope}%
\begin{pgfscope}%
\pgfsetbuttcap%
\pgfsetroundjoin%
\definecolor{currentfill}{rgb}{0.000000,0.000000,0.000000}%
\pgfsetfillcolor{currentfill}%
\pgfsetlinewidth{0.803000pt}%
\definecolor{currentstroke}{rgb}{0.000000,0.000000,0.000000}%
\pgfsetstrokecolor{currentstroke}%
\pgfsetdash{}{0pt}%
\pgfsys@defobject{currentmarker}{\pgfqpoint{0.000000in}{-0.048611in}}{\pgfqpoint{0.000000in}{0.000000in}}{%
\pgfpathmoveto{\pgfqpoint{0.000000in}{0.000000in}}%
\pgfpathlineto{\pgfqpoint{0.000000in}{-0.048611in}}%
\pgfusepath{stroke,fill}%
}%
\begin{pgfscope}%
\pgfsys@transformshift{4.529713in}{1.339482in}%
\pgfsys@useobject{currentmarker}{}%
\end{pgfscope}%
\end{pgfscope}%
\begin{pgfscope}%
\pgftext[x=4.529713in,y=1.242260in,,top]{\rmfamily\fontsize{8.000000}{9.600000}\selectfont 20}%
\end{pgfscope}%
\begin{pgfscope}%
\pgfsetbuttcap%
\pgfsetroundjoin%
\definecolor{currentfill}{rgb}{0.000000,0.000000,0.000000}%
\pgfsetfillcolor{currentfill}%
\pgfsetlinewidth{0.803000pt}%
\definecolor{currentstroke}{rgb}{0.000000,0.000000,0.000000}%
\pgfsetstrokecolor{currentstroke}%
\pgfsetdash{}{0pt}%
\pgfsys@defobject{currentmarker}{\pgfqpoint{-0.048611in}{0.000000in}}{\pgfqpoint{0.000000in}{0.000000in}}{%
\pgfpathmoveto{\pgfqpoint{0.000000in}{0.000000in}}%
\pgfpathlineto{\pgfqpoint{-0.048611in}{0.000000in}}%
\pgfusepath{stroke,fill}%
}%
\begin{pgfscope}%
\pgfsys@transformshift{0.273112in}{0.298820in}%
\pgfsys@useobject{currentmarker}{}%
\end{pgfscope}%
\end{pgfscope}%
\begin{pgfscope}%
\pgftext[x=0.000000in,y=0.258401in,left,base]{\rmfamily\fontsize{8.000000}{9.600000}\selectfont -1.0}%
\end{pgfscope}%
\begin{pgfscope}%
\pgfsetbuttcap%
\pgfsetroundjoin%
\definecolor{currentfill}{rgb}{0.000000,0.000000,0.000000}%
\pgfsetfillcolor{currentfill}%
\pgfsetlinewidth{0.803000pt}%
\definecolor{currentstroke}{rgb}{0.000000,0.000000,0.000000}%
\pgfsetstrokecolor{currentstroke}%
\pgfsetdash{}{0pt}%
\pgfsys@defobject{currentmarker}{\pgfqpoint{-0.048611in}{0.000000in}}{\pgfqpoint{0.000000in}{0.000000in}}{%
\pgfpathmoveto{\pgfqpoint{0.000000in}{0.000000in}}%
\pgfpathlineto{\pgfqpoint{-0.048611in}{0.000000in}}%
\pgfusepath{stroke,fill}%
}%
\begin{pgfscope}%
\pgfsys@transformshift{0.273112in}{0.819151in}%
\pgfsys@useobject{currentmarker}{}%
\end{pgfscope}%
\end{pgfscope}%
\begin{pgfscope}%
\pgftext[x=0.000000in,y=0.778732in,left,base]{\rmfamily\fontsize{8.000000}{9.600000}\selectfont -0.5}%
\end{pgfscope}%
\begin{pgfscope}%
\pgfsetbuttcap%
\pgfsetroundjoin%
\definecolor{currentfill}{rgb}{0.000000,0.000000,0.000000}%
\pgfsetfillcolor{currentfill}%
\pgfsetlinewidth{0.803000pt}%
\definecolor{currentstroke}{rgb}{0.000000,0.000000,0.000000}%
\pgfsetstrokecolor{currentstroke}%
\pgfsetdash{}{0pt}%
\pgfsys@defobject{currentmarker}{\pgfqpoint{-0.048611in}{0.000000in}}{\pgfqpoint{0.000000in}{0.000000in}}{%
\pgfpathmoveto{\pgfqpoint{0.000000in}{0.000000in}}%
\pgfpathlineto{\pgfqpoint{-0.048611in}{0.000000in}}%
\pgfusepath{stroke,fill}%
}%
\begin{pgfscope}%
\pgfsys@transformshift{0.273112in}{1.339482in}%
\pgfsys@useobject{currentmarker}{}%
\end{pgfscope}%
\end{pgfscope}%
\begin{pgfscope}%
\pgftext[x=0.037001in,y=1.299063in,left,base]{\rmfamily\fontsize{8.000000}{9.600000}\selectfont 0.0}%
\end{pgfscope}%
\begin{pgfscope}%
\pgfsetbuttcap%
\pgfsetroundjoin%
\definecolor{currentfill}{rgb}{0.000000,0.000000,0.000000}%
\pgfsetfillcolor{currentfill}%
\pgfsetlinewidth{0.803000pt}%
\definecolor{currentstroke}{rgb}{0.000000,0.000000,0.000000}%
\pgfsetstrokecolor{currentstroke}%
\pgfsetdash{}{0pt}%
\pgfsys@defobject{currentmarker}{\pgfqpoint{-0.048611in}{0.000000in}}{\pgfqpoint{0.000000in}{0.000000in}}{%
\pgfpathmoveto{\pgfqpoint{0.000000in}{0.000000in}}%
\pgfpathlineto{\pgfqpoint{-0.048611in}{0.000000in}}%
\pgfusepath{stroke,fill}%
}%
\begin{pgfscope}%
\pgfsys@transformshift{0.273112in}{1.859813in}%
\pgfsys@useobject{currentmarker}{}%
\end{pgfscope}%
\end{pgfscope}%
\begin{pgfscope}%
\pgftext[x=0.037001in,y=1.819394in,left,base]{\rmfamily\fontsize{8.000000}{9.600000}\selectfont 0.5}%
\end{pgfscope}%
\begin{pgfscope}%
\pgfsetbuttcap%
\pgfsetroundjoin%
\definecolor{currentfill}{rgb}{0.000000,0.000000,0.000000}%
\pgfsetfillcolor{currentfill}%
\pgfsetlinewidth{0.803000pt}%
\definecolor{currentstroke}{rgb}{0.000000,0.000000,0.000000}%
\pgfsetstrokecolor{currentstroke}%
\pgfsetdash{}{0pt}%
\pgfsys@defobject{currentmarker}{\pgfqpoint{-0.048611in}{0.000000in}}{\pgfqpoint{0.000000in}{0.000000in}}{%
\pgfpathmoveto{\pgfqpoint{0.000000in}{0.000000in}}%
\pgfpathlineto{\pgfqpoint{-0.048611in}{0.000000in}}%
\pgfusepath{stroke,fill}%
}%
\begin{pgfscope}%
\pgfsys@transformshift{0.273112in}{2.380144in}%
\pgfsys@useobject{currentmarker}{}%
\end{pgfscope}%
\end{pgfscope}%
\begin{pgfscope}%
\pgftext[x=0.037001in,y=2.339725in,left,base]{\rmfamily\fontsize{8.000000}{9.600000}\selectfont 1.0}%
\end{pgfscope}%
\begin{pgfscope}%
\pgfpathrectangle{\pgfqpoint{0.273112in}{0.017500in}}{\pgfqpoint{4.650000in}{2.695000in}} %
\pgfusepath{clip}%
\pgfsetrectcap%
\pgfsetroundjoin%
\pgfsetlinewidth{1.505625pt}%
\definecolor{currentstroke}{rgb}{0.121569,0.466667,0.705882}%
\pgfsetstrokecolor{currentstroke}%
\pgfsetdash{}{0pt}%
\pgfpathmoveto{\pgfqpoint{0.484476in}{1.342494in}}%
\pgfpathlineto{\pgfqpoint{0.504702in}{1.563380in}}%
\pgfpathlineto{\pgfqpoint{0.524928in}{1.672009in}}%
\pgfpathlineto{\pgfqpoint{0.545154in}{1.577821in}}%
\pgfpathlineto{\pgfqpoint{0.565380in}{1.866027in}}%
\pgfpathlineto{\pgfqpoint{0.585607in}{2.135902in}}%
\pgfpathlineto{\pgfqpoint{0.605833in}{2.015192in}}%
\pgfpathlineto{\pgfqpoint{0.626059in}{1.862277in}}%
\pgfpathlineto{\pgfqpoint{0.646285in}{2.151491in}}%
\pgfpathlineto{\pgfqpoint{0.666511in}{2.256866in}}%
\pgfpathlineto{\pgfqpoint{0.686737in}{2.426698in}}%
\pgfpathlineto{\pgfqpoint{0.706964in}{2.272827in}}%
\pgfpathlineto{\pgfqpoint{0.727190in}{2.360620in}}%
\pgfpathlineto{\pgfqpoint{0.747416in}{2.423446in}}%
\pgfpathlineto{\pgfqpoint{0.767642in}{2.355340in}}%
\pgfpathlineto{\pgfqpoint{0.787868in}{2.314282in}}%
\pgfpathlineto{\pgfqpoint{0.808095in}{2.305891in}}%
\pgfpathlineto{\pgfqpoint{0.848547in}{2.335951in}}%
\pgfpathlineto{\pgfqpoint{0.868773in}{2.385512in}}%
\pgfpathlineto{\pgfqpoint{0.888999in}{2.197513in}}%
\pgfpathlineto{\pgfqpoint{0.909225in}{2.203119in}}%
\pgfpathlineto{\pgfqpoint{0.929452in}{2.181757in}}%
\pgfpathlineto{\pgfqpoint{0.949678in}{2.386380in}}%
\pgfpathlineto{\pgfqpoint{0.969904in}{1.972695in}}%
\pgfpathlineto{\pgfqpoint{0.990130in}{1.874828in}}%
\pgfpathlineto{\pgfqpoint{1.010356in}{1.976399in}}%
\pgfpathlineto{\pgfqpoint{1.030583in}{1.822837in}}%
\pgfpathlineto{\pgfqpoint{1.050809in}{1.697954in}}%
\pgfpathlineto{\pgfqpoint{1.071035in}{1.606750in}}%
\pgfpathlineto{\pgfqpoint{1.091261in}{1.504783in}}%
\pgfpathlineto{\pgfqpoint{1.111487in}{1.311026in}}%
\pgfpathlineto{\pgfqpoint{1.131714in}{1.062012in}}%
\pgfpathlineto{\pgfqpoint{1.151940in}{1.200992in}}%
\pgfpathlineto{\pgfqpoint{1.172166in}{0.993618in}}%
\pgfpathlineto{\pgfqpoint{1.192392in}{1.051367in}}%
\pgfpathlineto{\pgfqpoint{1.212618in}{0.874360in}}%
\pgfpathlineto{\pgfqpoint{1.232844in}{0.853500in}}%
\pgfpathlineto{\pgfqpoint{1.253071in}{0.790833in}}%
\pgfpathlineto{\pgfqpoint{1.273297in}{0.583068in}}%
\pgfpathlineto{\pgfqpoint{1.293523in}{0.494741in}}%
\pgfpathlineto{\pgfqpoint{1.313749in}{0.351154in}}%
\pgfpathlineto{\pgfqpoint{1.333975in}{0.404441in}}%
\pgfpathlineto{\pgfqpoint{1.354202in}{0.508554in}}%
\pgfpathlineto{\pgfqpoint{1.374428in}{0.335281in}}%
\pgfpathlineto{\pgfqpoint{1.394654in}{0.353123in}}%
\pgfpathlineto{\pgfqpoint{1.414880in}{0.241914in}}%
\pgfpathlineto{\pgfqpoint{1.435106in}{0.287999in}}%
\pgfpathlineto{\pgfqpoint{1.455332in}{0.213156in}}%
\pgfpathlineto{\pgfqpoint{1.475559in}{0.433382in}}%
\pgfpathlineto{\pgfqpoint{1.495785in}{0.302171in}}%
\pgfpathlineto{\pgfqpoint{1.516011in}{0.406628in}}%
\pgfpathlineto{\pgfqpoint{1.536237in}{0.435672in}}%
\pgfpathlineto{\pgfqpoint{1.556463in}{0.433679in}}%
\pgfpathlineto{\pgfqpoint{1.576690in}{0.311704in}}%
\pgfpathlineto{\pgfqpoint{1.596916in}{0.561284in}}%
\pgfpathlineto{\pgfqpoint{1.617142in}{0.686262in}}%
\pgfpathlineto{\pgfqpoint{1.637368in}{0.695184in}}%
\pgfpathlineto{\pgfqpoint{1.657594in}{0.846040in}}%
\pgfpathlineto{\pgfqpoint{1.677821in}{0.938923in}}%
\pgfpathlineto{\pgfqpoint{1.698047in}{1.099422in}}%
\pgfpathlineto{\pgfqpoint{1.718273in}{1.274146in}}%
\pgfpathlineto{\pgfqpoint{1.738499in}{1.355037in}}%
\pgfpathlineto{\pgfqpoint{1.758725in}{1.430275in}}%
\pgfpathlineto{\pgfqpoint{1.778951in}{1.483734in}}%
\pgfpathlineto{\pgfqpoint{1.799178in}{1.241312in}}%
\pgfpathlineto{\pgfqpoint{1.819404in}{1.663793in}}%
\pgfpathlineto{\pgfqpoint{1.839630in}{1.843370in}}%
\pgfpathlineto{\pgfqpoint{1.859856in}{1.872066in}}%
\pgfpathlineto{\pgfqpoint{1.880082in}{2.017885in}}%
\pgfpathlineto{\pgfqpoint{1.900309in}{1.923459in}}%
\pgfpathlineto{\pgfqpoint{1.920535in}{2.099815in}}%
\pgfpathlineto{\pgfqpoint{1.940761in}{2.288232in}}%
\pgfpathlineto{\pgfqpoint{1.960987in}{2.313282in}}%
\pgfpathlineto{\pgfqpoint{1.981213in}{2.447756in}}%
\pgfpathlineto{\pgfqpoint{2.001439in}{2.319778in}}%
\pgfpathlineto{\pgfqpoint{2.021666in}{2.468400in}}%
\pgfpathlineto{\pgfqpoint{2.041892in}{2.333721in}}%
\pgfpathlineto{\pgfqpoint{2.082344in}{2.590000in}}%
\pgfpathlineto{\pgfqpoint{2.102570in}{2.266738in}}%
\pgfpathlineto{\pgfqpoint{2.122797in}{2.343581in}}%
\pgfpathlineto{\pgfqpoint{2.143023in}{2.337600in}}%
\pgfpathlineto{\pgfqpoint{2.163249in}{2.432308in}}%
\pgfpathlineto{\pgfqpoint{2.183475in}{2.109139in}}%
\pgfpathlineto{\pgfqpoint{2.203701in}{2.119898in}}%
\pgfpathlineto{\pgfqpoint{2.223928in}{2.170949in}}%
\pgfpathlineto{\pgfqpoint{2.244154in}{2.102113in}}%
\pgfpathlineto{\pgfqpoint{2.264380in}{1.904063in}}%
\pgfpathlineto{\pgfqpoint{2.284606in}{1.993987in}}%
\pgfpathlineto{\pgfqpoint{2.304832in}{1.966802in}}%
\pgfpathlineto{\pgfqpoint{2.325058in}{1.552661in}}%
\pgfpathlineto{\pgfqpoint{2.345285in}{1.536507in}}%
\pgfpathlineto{\pgfqpoint{2.365511in}{1.303714in}}%
\pgfpathlineto{\pgfqpoint{2.385737in}{1.413558in}}%
\pgfpathlineto{\pgfqpoint{2.405963in}{1.325914in}}%
\pgfpathlineto{\pgfqpoint{2.426189in}{0.931695in}}%
\pgfpathlineto{\pgfqpoint{2.446416in}{1.043449in}}%
\pgfpathlineto{\pgfqpoint{2.466642in}{1.042744in}}%
\pgfpathlineto{\pgfqpoint{2.486868in}{0.780236in}}%
\pgfpathlineto{\pgfqpoint{2.507094in}{0.735991in}}%
\pgfpathlineto{\pgfqpoint{2.527320in}{0.619364in}}%
\pgfpathlineto{\pgfqpoint{2.547546in}{0.510970in}}%
\pgfpathlineto{\pgfqpoint{2.567773in}{0.564385in}}%
\pgfpathlineto{\pgfqpoint{2.587999in}{0.354859in}}%
\pgfpathlineto{\pgfqpoint{2.608225in}{0.233638in}}%
\pgfpathlineto{\pgfqpoint{2.628451in}{0.406487in}}%
\pgfpathlineto{\pgfqpoint{2.648677in}{0.328320in}}%
\pgfpathlineto{\pgfqpoint{2.668904in}{0.269420in}}%
\pgfpathlineto{\pgfqpoint{2.689130in}{0.232275in}}%
\pgfpathlineto{\pgfqpoint{2.709356in}{0.293929in}}%
\pgfpathlineto{\pgfqpoint{2.729582in}{0.179730in}}%
\pgfpathlineto{\pgfqpoint{2.749808in}{0.405980in}}%
\pgfpathlineto{\pgfqpoint{2.770035in}{0.277302in}}%
\pgfpathlineto{\pgfqpoint{2.790261in}{0.446832in}}%
\pgfpathlineto{\pgfqpoint{2.810487in}{0.471473in}}%
\pgfpathlineto{\pgfqpoint{2.830713in}{0.452264in}}%
\pgfpathlineto{\pgfqpoint{2.850939in}{0.585376in}}%
\pgfpathlineto{\pgfqpoint{2.871165in}{0.603508in}}%
\pgfpathlineto{\pgfqpoint{2.891392in}{0.851430in}}%
\pgfpathlineto{\pgfqpoint{2.911618in}{0.933995in}}%
\pgfpathlineto{\pgfqpoint{2.931844in}{0.900181in}}%
\pgfpathlineto{\pgfqpoint{2.952070in}{1.161663in}}%
\pgfpathlineto{\pgfqpoint{2.972296in}{1.158935in}}%
\pgfpathlineto{\pgfqpoint{2.992523in}{1.326610in}}%
\pgfpathlineto{\pgfqpoint{3.012749in}{1.320229in}}%
\pgfpathlineto{\pgfqpoint{3.032975in}{1.475868in}}%
\pgfpathlineto{\pgfqpoint{3.053201in}{1.508283in}}%
\pgfpathlineto{\pgfqpoint{3.073427in}{1.702779in}}%
\pgfpathlineto{\pgfqpoint{3.093653in}{1.593989in}}%
\pgfpathlineto{\pgfqpoint{3.113880in}{1.883792in}}%
\pgfpathlineto{\pgfqpoint{3.134106in}{1.897127in}}%
\pgfpathlineto{\pgfqpoint{3.154332in}{1.940100in}}%
\pgfpathlineto{\pgfqpoint{3.174558in}{2.006459in}}%
\pgfpathlineto{\pgfqpoint{3.194784in}{2.058391in}}%
\pgfpathlineto{\pgfqpoint{3.215011in}{2.456733in}}%
\pgfpathlineto{\pgfqpoint{3.235237in}{2.322482in}}%
\pgfpathlineto{\pgfqpoint{3.255463in}{2.252628in}}%
\pgfpathlineto{\pgfqpoint{3.275689in}{2.218237in}}%
\pgfpathlineto{\pgfqpoint{3.295915in}{2.334972in}}%
\pgfpathlineto{\pgfqpoint{3.316142in}{2.192402in}}%
\pgfpathlineto{\pgfqpoint{3.336368in}{2.268027in}}%
\pgfpathlineto{\pgfqpoint{3.356594in}{2.384195in}}%
\pgfpathlineto{\pgfqpoint{3.376820in}{2.299501in}}%
\pgfpathlineto{\pgfqpoint{3.397046in}{2.262746in}}%
\pgfpathlineto{\pgfqpoint{3.417272in}{2.342916in}}%
\pgfpathlineto{\pgfqpoint{3.437499in}{2.275572in}}%
\pgfpathlineto{\pgfqpoint{3.457725in}{2.122694in}}%
\pgfpathlineto{\pgfqpoint{3.477951in}{2.247142in}}%
\pgfpathlineto{\pgfqpoint{3.498177in}{2.105530in}}%
\pgfpathlineto{\pgfqpoint{3.518403in}{2.046919in}}%
\pgfpathlineto{\pgfqpoint{3.538630in}{1.913832in}}%
\pgfpathlineto{\pgfqpoint{3.558856in}{2.011710in}}%
\pgfpathlineto{\pgfqpoint{3.579082in}{1.842392in}}%
\pgfpathlineto{\pgfqpoint{3.599308in}{1.714908in}}%
\pgfpathlineto{\pgfqpoint{3.619534in}{1.621781in}}%
\pgfpathlineto{\pgfqpoint{3.639760in}{1.557089in}}%
\pgfpathlineto{\pgfqpoint{3.659987in}{1.188636in}}%
\pgfpathlineto{\pgfqpoint{3.680213in}{1.361905in}}%
\pgfpathlineto{\pgfqpoint{3.720665in}{1.047300in}}%
\pgfpathlineto{\pgfqpoint{3.740891in}{0.991873in}}%
\pgfpathlineto{\pgfqpoint{3.761118in}{0.689041in}}%
\pgfpathlineto{\pgfqpoint{3.781344in}{0.686652in}}%
\pgfpathlineto{\pgfqpoint{3.801570in}{0.755776in}}%
\pgfpathlineto{\pgfqpoint{3.821796in}{0.413662in}}%
\pgfpathlineto{\pgfqpoint{3.842022in}{0.641259in}}%
\pgfpathlineto{\pgfqpoint{3.862249in}{0.531600in}}%
\pgfpathlineto{\pgfqpoint{3.882475in}{0.395746in}}%
\pgfpathlineto{\pgfqpoint{3.922927in}{0.298346in}}%
\pgfpathlineto{\pgfqpoint{3.943153in}{0.236679in}}%
\pgfpathlineto{\pgfqpoint{3.963379in}{0.140000in}}%
\pgfpathlineto{\pgfqpoint{3.983606in}{0.302881in}}%
\pgfpathlineto{\pgfqpoint{4.003832in}{0.375860in}}%
\pgfpathlineto{\pgfqpoint{4.024058in}{0.417671in}}%
\pgfpathlineto{\pgfqpoint{4.044284in}{0.278610in}}%
\pgfpathlineto{\pgfqpoint{4.064510in}{0.387513in}}%
\pgfpathlineto{\pgfqpoint{4.084737in}{0.508133in}}%
\pgfpathlineto{\pgfqpoint{4.104963in}{0.519643in}}%
\pgfpathlineto{\pgfqpoint{4.125189in}{0.393775in}}%
\pgfpathlineto{\pgfqpoint{4.145415in}{0.607086in}}%
\pgfpathlineto{\pgfqpoint{4.165641in}{0.584939in}}%
\pgfpathlineto{\pgfqpoint{4.185867in}{0.870329in}}%
\pgfpathlineto{\pgfqpoint{4.206094in}{0.971070in}}%
\pgfpathlineto{\pgfqpoint{4.226320in}{0.992057in}}%
\pgfpathlineto{\pgfqpoint{4.246546in}{1.071416in}}%
\pgfpathlineto{\pgfqpoint{4.266772in}{1.158036in}}%
\pgfpathlineto{\pgfqpoint{4.286998in}{1.142903in}}%
\pgfpathlineto{\pgfqpoint{4.307225in}{1.340004in}}%
\pgfpathlineto{\pgfqpoint{4.327451in}{1.467626in}}%
\pgfpathlineto{\pgfqpoint{4.347677in}{1.647833in}}%
\pgfpathlineto{\pgfqpoint{4.367903in}{1.856265in}}%
\pgfpathlineto{\pgfqpoint{4.388129in}{1.975271in}}%
\pgfpathlineto{\pgfqpoint{4.408356in}{1.941777in}}%
\pgfpathlineto{\pgfqpoint{4.428582in}{1.901101in}}%
\pgfpathlineto{\pgfqpoint{4.448808in}{2.105610in}}%
\pgfpathlineto{\pgfqpoint{4.469034in}{1.996311in}}%
\pgfpathlineto{\pgfqpoint{4.489260in}{2.235889in}}%
\pgfpathlineto{\pgfqpoint{4.509486in}{2.280050in}}%
\pgfpathlineto{\pgfqpoint{4.529713in}{2.226883in}}%
\pgfpathlineto{\pgfqpoint{4.549939in}{2.418028in}}%
\pgfpathlineto{\pgfqpoint{4.570165in}{2.317160in}}%
\pgfpathlineto{\pgfqpoint{4.590391in}{2.355634in}}%
\pgfpathlineto{\pgfqpoint{4.610617in}{2.262542in}}%
\pgfpathlineto{\pgfqpoint{4.630844in}{2.429857in}}%
\pgfpathlineto{\pgfqpoint{4.651070in}{2.248228in}}%
\pgfpathlineto{\pgfqpoint{4.671296in}{2.392245in}}%
\pgfpathlineto{\pgfqpoint{4.691522in}{2.311186in}}%
\pgfpathlineto{\pgfqpoint{4.711748in}{2.223940in}}%
\pgfpathlineto{\pgfqpoint{4.711748in}{2.223940in}}%
\pgfusepath{stroke}%
\end{pgfscope}%
\begin{pgfscope}%
\pgfpathrectangle{\pgfqpoint{0.273112in}{0.017500in}}{\pgfqpoint{4.650000in}{2.695000in}} %
\pgfusepath{clip}%
\pgfsetrectcap%
\pgfsetroundjoin%
\pgfsetlinewidth{1.505625pt}%
\definecolor{currentstroke}{rgb}{1.000000,0.498039,0.054902}%
\pgfsetstrokecolor{currentstroke}%
\pgfsetdash{}{0pt}%
\pgfpathmoveto{\pgfqpoint{0.484476in}{1.339482in}}%
\pgfpathlineto{\pgfqpoint{2.446416in}{1.339665in}}%
\pgfpathlineto{\pgfqpoint{2.466642in}{1.340538in}}%
\pgfpathlineto{\pgfqpoint{2.486868in}{1.344214in}}%
\pgfpathlineto{\pgfqpoint{2.507094in}{1.355997in}}%
\pgfpathlineto{\pgfqpoint{2.527320in}{1.384375in}}%
\pgfpathlineto{\pgfqpoint{2.547546in}{1.434520in}}%
\pgfpathlineto{\pgfqpoint{2.567773in}{1.496173in}}%
\pgfpathlineto{\pgfqpoint{2.587999in}{1.540677in}}%
\pgfpathlineto{\pgfqpoint{2.608225in}{1.540677in}}%
\pgfpathlineto{\pgfqpoint{2.628451in}{1.496173in}}%
\pgfpathlineto{\pgfqpoint{2.648677in}{1.434520in}}%
\pgfpathlineto{\pgfqpoint{2.668904in}{1.384375in}}%
\pgfpathlineto{\pgfqpoint{2.689130in}{1.355997in}}%
\pgfpathlineto{\pgfqpoint{2.709356in}{1.344214in}}%
\pgfpathlineto{\pgfqpoint{2.729582in}{1.340538in}}%
\pgfpathlineto{\pgfqpoint{2.749808in}{1.339665in}}%
\pgfpathlineto{\pgfqpoint{2.830713in}{1.339482in}}%
\pgfpathlineto{\pgfqpoint{4.711748in}{1.339482in}}%
\pgfpathlineto{\pgfqpoint{4.711748in}{1.339482in}}%
\pgfusepath{stroke}%
\end{pgfscope}%
\begin{pgfscope}%
\pgfpathrectangle{\pgfqpoint{0.273112in}{0.017500in}}{\pgfqpoint{4.650000in}{2.695000in}} %
\pgfusepath{clip}%
\pgfsetrectcap%
\pgfsetroundjoin%
\pgfsetlinewidth{1.505625pt}%
\definecolor{currentstroke}{rgb}{0.172549,0.627451,0.172549}%
\pgfsetstrokecolor{currentstroke}%
\pgfsetdash{}{0pt}%
\pgfpathmoveto{\pgfqpoint{0.484476in}{1.427203in}}%
\pgfpathlineto{\pgfqpoint{0.504702in}{1.494310in}}%
\pgfpathlineto{\pgfqpoint{0.524928in}{1.579955in}}%
\pgfpathlineto{\pgfqpoint{0.545154in}{1.678136in}}%
\pgfpathlineto{\pgfqpoint{0.565380in}{1.780536in}}%
\pgfpathlineto{\pgfqpoint{0.585607in}{1.876388in}}%
\pgfpathlineto{\pgfqpoint{0.605833in}{1.958253in}}%
\pgfpathlineto{\pgfqpoint{0.646285in}{2.100299in}}%
\pgfpathlineto{\pgfqpoint{0.666511in}{2.174426in}}%
\pgfpathlineto{\pgfqpoint{0.686737in}{2.243817in}}%
\pgfpathlineto{\pgfqpoint{0.706964in}{2.297522in}}%
\pgfpathlineto{\pgfqpoint{0.727190in}{2.331189in}}%
\pgfpathlineto{\pgfqpoint{0.747416in}{2.346982in}}%
\pgfpathlineto{\pgfqpoint{0.767642in}{2.349250in}}%
\pgfpathlineto{\pgfqpoint{0.787868in}{2.343124in}}%
\pgfpathlineto{\pgfqpoint{0.828321in}{2.325438in}}%
\pgfpathlineto{\pgfqpoint{0.848547in}{2.315264in}}%
\pgfpathlineto{\pgfqpoint{0.868773in}{2.299307in}}%
\pgfpathlineto{\pgfqpoint{0.888999in}{2.275706in}}%
\pgfpathlineto{\pgfqpoint{0.909225in}{2.245898in}}%
\pgfpathlineto{\pgfqpoint{0.929452in}{2.209440in}}%
\pgfpathlineto{\pgfqpoint{0.949678in}{2.161008in}}%
\pgfpathlineto{\pgfqpoint{0.969904in}{2.096017in}}%
\pgfpathlineto{\pgfqpoint{0.990130in}{2.017210in}}%
\pgfpathlineto{\pgfqpoint{1.010356in}{1.931498in}}%
\pgfpathlineto{\pgfqpoint{1.030583in}{1.841630in}}%
\pgfpathlineto{\pgfqpoint{1.050809in}{1.744504in}}%
\pgfpathlineto{\pgfqpoint{1.071035in}{1.636499in}}%
\pgfpathlineto{\pgfqpoint{1.091261in}{1.518343in}}%
\pgfpathlineto{\pgfqpoint{1.111487in}{1.396518in}}%
\pgfpathlineto{\pgfqpoint{1.131714in}{1.281175in}}%
\pgfpathlineto{\pgfqpoint{1.151940in}{1.180447in}}%
\pgfpathlineto{\pgfqpoint{1.172166in}{1.095058in}}%
\pgfpathlineto{\pgfqpoint{1.212618in}{0.942879in}}%
\pgfpathlineto{\pgfqpoint{1.232844in}{0.861240in}}%
\pgfpathlineto{\pgfqpoint{1.253071in}{0.771224in}}%
\pgfpathlineto{\pgfqpoint{1.273297in}{0.675927in}}%
\pgfpathlineto{\pgfqpoint{1.293523in}{0.584735in}}%
\pgfpathlineto{\pgfqpoint{1.313749in}{0.509126in}}%
\pgfpathlineto{\pgfqpoint{1.333975in}{0.454607in}}%
\pgfpathlineto{\pgfqpoint{1.354202in}{0.416462in}}%
\pgfpathlineto{\pgfqpoint{1.394654in}{0.353900in}}%
\pgfpathlineto{\pgfqpoint{1.414880in}{0.326687in}}%
\pgfpathlineto{\pgfqpoint{1.435106in}{0.310471in}}%
\pgfpathlineto{\pgfqpoint{1.455332in}{0.310107in}}%
\pgfpathlineto{\pgfqpoint{1.475559in}{0.324242in}}%
\pgfpathlineto{\pgfqpoint{1.495785in}{0.346826in}}%
\pgfpathlineto{\pgfqpoint{1.536237in}{0.396300in}}%
\pgfpathlineto{\pgfqpoint{1.556463in}{0.424457in}}%
\pgfpathlineto{\pgfqpoint{1.576690in}{0.464313in}}%
\pgfpathlineto{\pgfqpoint{1.596916in}{0.523228in}}%
\pgfpathlineto{\pgfqpoint{1.617142in}{0.602185in}}%
\pgfpathlineto{\pgfqpoint{1.637368in}{0.696608in}}%
\pgfpathlineto{\pgfqpoint{1.657594in}{0.801864in}}%
\pgfpathlineto{\pgfqpoint{1.677821in}{0.915493in}}%
\pgfpathlineto{\pgfqpoint{1.718273in}{1.149534in}}%
\pgfpathlineto{\pgfqpoint{1.738499in}{1.252378in}}%
\pgfpathlineto{\pgfqpoint{1.758725in}{1.337390in}}%
\pgfpathlineto{\pgfqpoint{1.778951in}{1.410415in}}%
\pgfpathlineto{\pgfqpoint{1.799178in}{1.486611in}}%
\pgfpathlineto{\pgfqpoint{1.819404in}{1.578907in}}%
\pgfpathlineto{\pgfqpoint{1.859856in}{1.796436in}}%
\pgfpathlineto{\pgfqpoint{1.880082in}{1.897566in}}%
\pgfpathlineto{\pgfqpoint{1.900309in}{1.989463in}}%
\pgfpathlineto{\pgfqpoint{1.940761in}{2.166109in}}%
\pgfpathlineto{\pgfqpoint{1.960987in}{2.246752in}}%
\pgfpathlineto{\pgfqpoint{1.981213in}{2.311490in}}%
\pgfpathlineto{\pgfqpoint{2.001439in}{2.356825in}}%
\pgfpathlineto{\pgfqpoint{2.021666in}{2.386196in}}%
\pgfpathlineto{\pgfqpoint{2.041892in}{2.405009in}}%
\pgfpathlineto{\pgfqpoint{2.062118in}{2.414767in}}%
\pgfpathlineto{\pgfqpoint{2.082344in}{2.412695in}}%
\pgfpathlineto{\pgfqpoint{2.102570in}{2.397071in}}%
\pgfpathlineto{\pgfqpoint{2.122797in}{2.370627in}}%
\pgfpathlineto{\pgfqpoint{2.143023in}{2.336823in}}%
\pgfpathlineto{\pgfqpoint{2.163249in}{2.295706in}}%
\pgfpathlineto{\pgfqpoint{2.183475in}{2.246685in}}%
\pgfpathlineto{\pgfqpoint{2.223928in}{2.137247in}}%
\pgfpathlineto{\pgfqpoint{2.244154in}{2.080306in}}%
\pgfpathlineto{\pgfqpoint{2.264380in}{2.016900in}}%
\pgfpathlineto{\pgfqpoint{2.284606in}{1.940097in}}%
\pgfpathlineto{\pgfqpoint{2.304832in}{1.844209in}}%
\pgfpathlineto{\pgfqpoint{2.325058in}{1.729887in}}%
\pgfpathlineto{\pgfqpoint{2.345285in}{1.606771in}}%
\pgfpathlineto{\pgfqpoint{2.365511in}{1.487293in}}%
\pgfpathlineto{\pgfqpoint{2.385737in}{1.376912in}}%
\pgfpathlineto{\pgfqpoint{2.426189in}{1.171773in}}%
\pgfpathlineto{\pgfqpoint{2.466642in}{0.977833in}}%
\pgfpathlineto{\pgfqpoint{2.527320in}{0.692005in}}%
\pgfpathlineto{\pgfqpoint{2.547546in}{0.603742in}}%
\pgfpathlineto{\pgfqpoint{2.567773in}{0.523572in}}%
\pgfpathlineto{\pgfqpoint{2.587999in}{0.453488in}}%
\pgfpathlineto{\pgfqpoint{2.608225in}{0.396728in}}%
\pgfpathlineto{\pgfqpoint{2.628451in}{0.354966in}}%
\pgfpathlineto{\pgfqpoint{2.648677in}{0.325362in}}%
\pgfpathlineto{\pgfqpoint{2.668904in}{0.303294in}}%
\pgfpathlineto{\pgfqpoint{2.689130in}{0.287665in}}%
\pgfpathlineto{\pgfqpoint{2.709356in}{0.281733in}}%
\pgfpathlineto{\pgfqpoint{2.729582in}{0.289292in}}%
\pgfpathlineto{\pgfqpoint{2.749808in}{0.311227in}}%
\pgfpathlineto{\pgfqpoint{2.770035in}{0.345192in}}%
\pgfpathlineto{\pgfqpoint{2.790261in}{0.387674in}}%
\pgfpathlineto{\pgfqpoint{2.810487in}{0.436683in}}%
\pgfpathlineto{\pgfqpoint{2.830713in}{0.493505in}}%
\pgfpathlineto{\pgfqpoint{2.850939in}{0.561906in}}%
\pgfpathlineto{\pgfqpoint{2.871165in}{0.644419in}}%
\pgfpathlineto{\pgfqpoint{2.891392in}{0.738889in}}%
\pgfpathlineto{\pgfqpoint{2.952070in}{1.039900in}}%
\pgfpathlineto{\pgfqpoint{2.972296in}{1.136669in}}%
\pgfpathlineto{\pgfqpoint{2.992523in}{1.230094in}}%
\pgfpathlineto{\pgfqpoint{3.032975in}{1.408064in}}%
\pgfpathlineto{\pgfqpoint{3.113880in}{1.754589in}}%
\pgfpathlineto{\pgfqpoint{3.174558in}{2.008226in}}%
\pgfpathlineto{\pgfqpoint{3.194784in}{2.095064in}}%
\pgfpathlineto{\pgfqpoint{3.215011in}{2.174142in}}%
\pgfpathlineto{\pgfqpoint{3.235237in}{2.231948in}}%
\pgfpathlineto{\pgfqpoint{3.255463in}{2.262063in}}%
\pgfpathlineto{\pgfqpoint{3.275689in}{2.270877in}}%
\pgfpathlineto{\pgfqpoint{3.295915in}{2.271387in}}%
\pgfpathlineto{\pgfqpoint{3.316142in}{2.273708in}}%
\pgfpathlineto{\pgfqpoint{3.336368in}{2.280934in}}%
\pgfpathlineto{\pgfqpoint{3.356594in}{2.290106in}}%
\pgfpathlineto{\pgfqpoint{3.376820in}{2.295530in}}%
\pgfpathlineto{\pgfqpoint{3.397046in}{2.292485in}}%
\pgfpathlineto{\pgfqpoint{3.417272in}{2.278848in}}%
\pgfpathlineto{\pgfqpoint{3.437499in}{2.254501in}}%
\pgfpathlineto{\pgfqpoint{3.457725in}{2.220461in}}%
\pgfpathlineto{\pgfqpoint{3.477951in}{2.178017in}}%
\pgfpathlineto{\pgfqpoint{3.498177in}{2.128011in}}%
\pgfpathlineto{\pgfqpoint{3.518403in}{2.071236in}}%
\pgfpathlineto{\pgfqpoint{3.538630in}{2.008524in}}%
\pgfpathlineto{\pgfqpoint{3.558856in}{1.938956in}}%
\pgfpathlineto{\pgfqpoint{3.579082in}{1.859128in}}%
\pgfpathlineto{\pgfqpoint{3.599308in}{1.765982in}}%
\pgfpathlineto{\pgfqpoint{3.619534in}{1.660302in}}%
\pgfpathlineto{\pgfqpoint{3.659987in}{1.434065in}}%
\pgfpathlineto{\pgfqpoint{3.740891in}{0.994369in}}%
\pgfpathlineto{\pgfqpoint{3.761118in}{0.882530in}}%
\pgfpathlineto{\pgfqpoint{3.781344in}{0.781661in}}%
\pgfpathlineto{\pgfqpoint{3.801570in}{0.697175in}}%
\pgfpathlineto{\pgfqpoint{3.821796in}{0.628236in}}%
\pgfpathlineto{\pgfqpoint{3.842022in}{0.569370in}}%
\pgfpathlineto{\pgfqpoint{3.862249in}{0.513410in}}%
\pgfpathlineto{\pgfqpoint{3.902701in}{0.395749in}}%
\pgfpathlineto{\pgfqpoint{3.922927in}{0.341235in}}%
\pgfpathlineto{\pgfqpoint{3.943153in}{0.301154in}}%
\pgfpathlineto{\pgfqpoint{3.963379in}{0.283231in}}%
\pgfpathlineto{\pgfqpoint{3.983606in}{0.288769in}}%
\pgfpathlineto{\pgfqpoint{4.003832in}{0.310766in}}%
\pgfpathlineto{\pgfqpoint{4.044284in}{0.367408in}}%
\pgfpathlineto{\pgfqpoint{4.064510in}{0.397063in}}%
\pgfpathlineto{\pgfqpoint{4.084737in}{0.430033in}}%
\pgfpathlineto{\pgfqpoint{4.104963in}{0.467466in}}%
\pgfpathlineto{\pgfqpoint{4.125189in}{0.512804in}}%
\pgfpathlineto{\pgfqpoint{4.145415in}{0.573098in}}%
\pgfpathlineto{\pgfqpoint{4.165641in}{0.653267in}}%
\pgfpathlineto{\pgfqpoint{4.185867in}{0.749600in}}%
\pgfpathlineto{\pgfqpoint{4.206094in}{0.850753in}}%
\pgfpathlineto{\pgfqpoint{4.226320in}{0.945768in}}%
\pgfpathlineto{\pgfqpoint{4.246546in}{1.031531in}}%
\pgfpathlineto{\pgfqpoint{4.266772in}{1.113937in}}%
\pgfpathlineto{\pgfqpoint{4.286998in}{1.203511in}}%
\pgfpathlineto{\pgfqpoint{4.307225in}{1.309189in}}%
\pgfpathlineto{\pgfqpoint{4.327451in}{1.432886in}}%
\pgfpathlineto{\pgfqpoint{4.347677in}{1.566962in}}%
\pgfpathlineto{\pgfqpoint{4.367903in}{1.696547in}}%
\pgfpathlineto{\pgfqpoint{4.388129in}{1.806843in}}%
\pgfpathlineto{\pgfqpoint{4.408356in}{1.891701in}}%
\pgfpathlineto{\pgfqpoint{4.428582in}{1.956794in}}%
\pgfpathlineto{\pgfqpoint{4.448808in}{2.014180in}}%
\pgfpathlineto{\pgfqpoint{4.469034in}{2.073243in}}%
\pgfpathlineto{\pgfqpoint{4.509486in}{2.196660in}}%
\pgfpathlineto{\pgfqpoint{4.529713in}{2.249547in}}%
\pgfpathlineto{\pgfqpoint{4.549939in}{2.289357in}}%
\pgfpathlineto{\pgfqpoint{4.570165in}{2.314488in}}%
\pgfpathlineto{\pgfqpoint{4.590391in}{2.326860in}}%
\pgfpathlineto{\pgfqpoint{4.610617in}{2.330228in}}%
\pgfpathlineto{\pgfqpoint{4.630844in}{2.325892in}}%
\pgfpathlineto{\pgfqpoint{4.651070in}{2.307629in}}%
\pgfpathlineto{\pgfqpoint{4.671296in}{2.259608in}}%
\pgfpathlineto{\pgfqpoint{4.691522in}{2.162126in}}%
\pgfpathlineto{\pgfqpoint{4.711748in}{2.007358in}}%
\pgfpathlineto{\pgfqpoint{4.711748in}{2.007358in}}%
\pgfusepath{stroke}%
\end{pgfscope}%
\begin{pgfscope}%
\pgfsetrectcap%
\pgfsetmiterjoin%
\pgfsetlinewidth{0.501875pt}%
\definecolor{currentstroke}{rgb}{0.000000,0.000000,0.000000}%
\pgfsetstrokecolor{currentstroke}%
\pgfsetdash{}{0pt}%
\pgfpathmoveto{\pgfqpoint{0.273112in}{0.017500in}}%
\pgfpathlineto{\pgfqpoint{0.273112in}{2.712500in}}%
\pgfusepath{stroke}%
\end{pgfscope}%
\begin{pgfscope}%
\pgfsetrectcap%
\pgfsetmiterjoin%
\pgfsetlinewidth{0.501875pt}%
\definecolor{currentstroke}{rgb}{0.000000,0.000000,0.000000}%
\pgfsetstrokecolor{currentstroke}%
\pgfsetdash{}{0pt}%
\pgfpathmoveto{\pgfqpoint{0.273112in}{1.339482in}}%
\pgfpathlineto{\pgfqpoint{4.923112in}{1.339482in}}%
\pgfusepath{stroke}%
\end{pgfscope}%
\begin{pgfscope}%
\pgfsetrectcap%
\pgfsetroundjoin%
\pgfsetlinewidth{1.505625pt}%
\definecolor{currentstroke}{rgb}{0.121569,0.466667,0.705882}%
\pgfsetstrokecolor{currentstroke}%
\pgfsetdash{}{0pt}%
\pgfpathmoveto{\pgfqpoint{0.385612in}{0.583906in}}%
\pgfpathlineto{\pgfqpoint{0.635612in}{0.583906in}}%
\pgfusepath{stroke}%
\end{pgfscope}%
\begin{pgfscope}%
\pgftext[x=0.735612in,y=0.540156in,left,base]{\rmfamily\fontsize{9.000000}{10.800000}\selectfont \(\displaystyle f(x)\)}%
\end{pgfscope}%
\begin{pgfscope}%
\pgfsetrectcap%
\pgfsetroundjoin%
\pgfsetlinewidth{1.505625pt}%
\definecolor{currentstroke}{rgb}{1.000000,0.498039,0.054902}%
\pgfsetstrokecolor{currentstroke}%
\pgfsetdash{}{0pt}%
\pgfpathmoveto{\pgfqpoint{0.385612in}{0.396040in}}%
\pgfpathlineto{\pgfqpoint{0.635612in}{0.396040in}}%
\pgfusepath{stroke}%
\end{pgfscope}%
\begin{pgfscope}%
\pgftext[x=0.735612in,y=0.352290in,left,base]{\rmfamily\fontsize{9.000000}{10.800000}\selectfont \(\displaystyle k(x)\)}%
\end{pgfscope}%
\begin{pgfscope}%
\pgfsetrectcap%
\pgfsetroundjoin%
\pgfsetlinewidth{1.505625pt}%
\definecolor{currentstroke}{rgb}{0.172549,0.627451,0.172549}%
\pgfsetstrokecolor{currentstroke}%
\pgfsetdash{}{0pt}%
\pgfpathmoveto{\pgfqpoint{0.385612in}{0.208174in}}%
\pgfpathlineto{\pgfqpoint{0.635612in}{0.208174in}}%
\pgfusepath{stroke}%
\end{pgfscope}%
\begin{pgfscope}%
\pgftext[x=0.735612in,y=0.164424in,left,base]{\rmfamily\fontsize{9.000000}{10.800000}\selectfont \(\displaystyle (f * k)(x)\)}%
\end{pgfscope}%
\end{pgfpicture}%
\makeatother%
\endgroup%

	\caption{Visualisation of a noisy signal $f$ convolved with a small Gaussian kernel $k$. The output of the convolution $f * k$ captures the general trend of $f$ by averaging the outputs of $f$ at every $x$, such values of $f$ of inputs close to $x$ contribute more to the output of the convolution, than inputs far away from $x$ thanks to the weights of the Gaussian kernel.}
	\label{gaussian_convolution}
\end{figure}

\begin{figure}
	\centering
	%% Creator: Matplotlib, PGF backend
%%
%% To include the figure in your LaTeX document, write
%%   \input{<filename>.pgf}
%%
%% Make sure the required packages are loaded in your preamble
%%   \usepackage{pgf}
%%
%% Figures using additional raster images can only be included by \input if
%% they are in the same directory as the main LaTeX file. For loading figures
%% from other directories you can use the `import` package
%%   \usepackage{import}
%% and then include the figures with
%%   \import{<path to file>}{<filename>.pgf}
%%
%% Matplotlib used the following preamble
%%   \usepackage{fontspec}
%%   \setmainfont{Palatino}
%%   \setsansfont{Lucida Grande}
%%   \setmonofont{Andale Mono}
%%
\begingroup%
\makeatletter%
\begin{pgfpicture}%
\pgfpathrectangle{\pgfpointorigin}{\pgfqpoint{4.940612in}{2.730000in}}%
\pgfusepath{use as bounding box, clip}%
\begin{pgfscope}%
\pgfsetbuttcap%
\pgfsetmiterjoin%
\definecolor{currentfill}{rgb}{1.000000,1.000000,1.000000}%
\pgfsetfillcolor{currentfill}%
\pgfsetlinewidth{0.000000pt}%
\definecolor{currentstroke}{rgb}{1.000000,1.000000,1.000000}%
\pgfsetstrokecolor{currentstroke}%
\pgfsetdash{}{0pt}%
\pgfpathmoveto{\pgfqpoint{0.000000in}{0.000000in}}%
\pgfpathlineto{\pgfqpoint{4.940612in}{0.000000in}}%
\pgfpathlineto{\pgfqpoint{4.940612in}{2.730000in}}%
\pgfpathlineto{\pgfqpoint{0.000000in}{2.730000in}}%
\pgfpathclose%
\pgfusepath{fill}%
\end{pgfscope}%
\begin{pgfscope}%
\pgfsetbuttcap%
\pgfsetmiterjoin%
\definecolor{currentfill}{rgb}{1.000000,1.000000,1.000000}%
\pgfsetfillcolor{currentfill}%
\pgfsetlinewidth{0.000000pt}%
\definecolor{currentstroke}{rgb}{0.000000,0.000000,0.000000}%
\pgfsetstrokecolor{currentstroke}%
\pgfsetstrokeopacity{0.000000}%
\pgfsetdash{}{0pt}%
\pgfpathmoveto{\pgfqpoint{0.273112in}{0.017500in}}%
\pgfpathlineto{\pgfqpoint{4.923112in}{0.017500in}}%
\pgfpathlineto{\pgfqpoint{4.923112in}{2.712500in}}%
\pgfpathlineto{\pgfqpoint{0.273112in}{2.712500in}}%
\pgfpathclose%
\pgfusepath{fill}%
\end{pgfscope}%
\begin{pgfscope}%
\pgfsetbuttcap%
\pgfsetroundjoin%
\definecolor{currentfill}{rgb}{0.000000,0.000000,0.000000}%
\pgfsetfillcolor{currentfill}%
\pgfsetlinewidth{0.803000pt}%
\definecolor{currentstroke}{rgb}{0.000000,0.000000,0.000000}%
\pgfsetstrokecolor{currentstroke}%
\pgfsetdash{}{0pt}%
\pgfsys@defobject{currentmarker}{\pgfqpoint{0.000000in}{-0.048611in}}{\pgfqpoint{0.000000in}{0.000000in}}{%
\pgfpathmoveto{\pgfqpoint{0.000000in}{0.000000in}}%
\pgfpathlineto{\pgfqpoint{0.000000in}{-0.048611in}}%
\pgfusepath{stroke,fill}%
}%
\begin{pgfscope}%
\pgfsys@transformshift{0.484476in}{1.363400in}%
\pgfsys@useobject{currentmarker}{}%
\end{pgfscope}%
\end{pgfscope}%
\begin{pgfscope}%
\pgftext[x=0.484476in,y=1.266178in,,top]{\rmfamily\fontsize{8.000000}{9.600000}\selectfont 0}%
\end{pgfscope}%
\begin{pgfscope}%
\pgfsetbuttcap%
\pgfsetroundjoin%
\definecolor{currentfill}{rgb}{0.000000,0.000000,0.000000}%
\pgfsetfillcolor{currentfill}%
\pgfsetlinewidth{0.803000pt}%
\definecolor{currentstroke}{rgb}{0.000000,0.000000,0.000000}%
\pgfsetstrokecolor{currentstroke}%
\pgfsetdash{}{0pt}%
\pgfsys@defobject{currentmarker}{\pgfqpoint{0.000000in}{-0.048611in}}{\pgfqpoint{0.000000in}{0.000000in}}{%
\pgfpathmoveto{\pgfqpoint{0.000000in}{0.000000in}}%
\pgfpathlineto{\pgfqpoint{0.000000in}{-0.048611in}}%
\pgfusepath{stroke,fill}%
}%
\begin{pgfscope}%
\pgfsys@transformshift{1.515518in}{1.363400in}%
\pgfsys@useobject{currentmarker}{}%
\end{pgfscope}%
\end{pgfscope}%
\begin{pgfscope}%
\pgftext[x=1.515518in,y=1.266178in,,top]{\rmfamily\fontsize{8.000000}{9.600000}\selectfont 5}%
\end{pgfscope}%
\begin{pgfscope}%
\pgfsetbuttcap%
\pgfsetroundjoin%
\definecolor{currentfill}{rgb}{0.000000,0.000000,0.000000}%
\pgfsetfillcolor{currentfill}%
\pgfsetlinewidth{0.803000pt}%
\definecolor{currentstroke}{rgb}{0.000000,0.000000,0.000000}%
\pgfsetstrokecolor{currentstroke}%
\pgfsetdash{}{0pt}%
\pgfsys@defobject{currentmarker}{\pgfqpoint{0.000000in}{-0.048611in}}{\pgfqpoint{0.000000in}{0.000000in}}{%
\pgfpathmoveto{\pgfqpoint{0.000000in}{0.000000in}}%
\pgfpathlineto{\pgfqpoint{0.000000in}{-0.048611in}}%
\pgfusepath{stroke,fill}%
}%
\begin{pgfscope}%
\pgfsys@transformshift{2.546560in}{1.363400in}%
\pgfsys@useobject{currentmarker}{}%
\end{pgfscope}%
\end{pgfscope}%
\begin{pgfscope}%
\pgftext[x=2.546560in,y=1.266178in,,top]{\rmfamily\fontsize{8.000000}{9.600000}\selectfont 10}%
\end{pgfscope}%
\begin{pgfscope}%
\pgfsetbuttcap%
\pgfsetroundjoin%
\definecolor{currentfill}{rgb}{0.000000,0.000000,0.000000}%
\pgfsetfillcolor{currentfill}%
\pgfsetlinewidth{0.803000pt}%
\definecolor{currentstroke}{rgb}{0.000000,0.000000,0.000000}%
\pgfsetstrokecolor{currentstroke}%
\pgfsetdash{}{0pt}%
\pgfsys@defobject{currentmarker}{\pgfqpoint{0.000000in}{-0.048611in}}{\pgfqpoint{0.000000in}{0.000000in}}{%
\pgfpathmoveto{\pgfqpoint{0.000000in}{0.000000in}}%
\pgfpathlineto{\pgfqpoint{0.000000in}{-0.048611in}}%
\pgfusepath{stroke,fill}%
}%
\begin{pgfscope}%
\pgfsys@transformshift{3.577602in}{1.363400in}%
\pgfsys@useobject{currentmarker}{}%
\end{pgfscope}%
\end{pgfscope}%
\begin{pgfscope}%
\pgftext[x=3.577602in,y=1.266178in,,top]{\rmfamily\fontsize{8.000000}{9.600000}\selectfont 15}%
\end{pgfscope}%
\begin{pgfscope}%
\pgfsetbuttcap%
\pgfsetroundjoin%
\definecolor{currentfill}{rgb}{0.000000,0.000000,0.000000}%
\pgfsetfillcolor{currentfill}%
\pgfsetlinewidth{0.803000pt}%
\definecolor{currentstroke}{rgb}{0.000000,0.000000,0.000000}%
\pgfsetstrokecolor{currentstroke}%
\pgfsetdash{}{0pt}%
\pgfsys@defobject{currentmarker}{\pgfqpoint{0.000000in}{-0.048611in}}{\pgfqpoint{0.000000in}{0.000000in}}{%
\pgfpathmoveto{\pgfqpoint{0.000000in}{0.000000in}}%
\pgfpathlineto{\pgfqpoint{0.000000in}{-0.048611in}}%
\pgfusepath{stroke,fill}%
}%
\begin{pgfscope}%
\pgfsys@transformshift{4.608644in}{1.363400in}%
\pgfsys@useobject{currentmarker}{}%
\end{pgfscope}%
\end{pgfscope}%
\begin{pgfscope}%
\pgftext[x=4.608644in,y=1.266178in,,top]{\rmfamily\fontsize{8.000000}{9.600000}\selectfont 20}%
\end{pgfscope}%
\begin{pgfscope}%
\pgfsetbuttcap%
\pgfsetroundjoin%
\definecolor{currentfill}{rgb}{0.000000,0.000000,0.000000}%
\pgfsetfillcolor{currentfill}%
\pgfsetlinewidth{0.803000pt}%
\definecolor{currentstroke}{rgb}{0.000000,0.000000,0.000000}%
\pgfsetstrokecolor{currentstroke}%
\pgfsetdash{}{0pt}%
\pgfsys@defobject{currentmarker}{\pgfqpoint{-0.048611in}{0.000000in}}{\pgfqpoint{0.000000in}{0.000000in}}{%
\pgfpathmoveto{\pgfqpoint{0.000000in}{0.000000in}}%
\pgfpathlineto{\pgfqpoint{-0.048611in}{0.000000in}}%
\pgfusepath{stroke,fill}%
}%
\begin{pgfscope}%
\pgfsys@transformshift{0.273112in}{0.247149in}%
\pgfsys@useobject{currentmarker}{}%
\end{pgfscope}%
\end{pgfscope}%
\begin{pgfscope}%
\pgftext[x=0.000000in,y=0.206731in,left,base]{\rmfamily\fontsize{8.000000}{9.600000}\selectfont -1.0}%
\end{pgfscope}%
\begin{pgfscope}%
\pgfsetbuttcap%
\pgfsetroundjoin%
\definecolor{currentfill}{rgb}{0.000000,0.000000,0.000000}%
\pgfsetfillcolor{currentfill}%
\pgfsetlinewidth{0.803000pt}%
\definecolor{currentstroke}{rgb}{0.000000,0.000000,0.000000}%
\pgfsetstrokecolor{currentstroke}%
\pgfsetdash{}{0pt}%
\pgfsys@defobject{currentmarker}{\pgfqpoint{-0.048611in}{0.000000in}}{\pgfqpoint{0.000000in}{0.000000in}}{%
\pgfpathmoveto{\pgfqpoint{0.000000in}{0.000000in}}%
\pgfpathlineto{\pgfqpoint{-0.048611in}{0.000000in}}%
\pgfusepath{stroke,fill}%
}%
\begin{pgfscope}%
\pgfsys@transformshift{0.273112in}{0.805275in}%
\pgfsys@useobject{currentmarker}{}%
\end{pgfscope}%
\end{pgfscope}%
\begin{pgfscope}%
\pgftext[x=0.000000in,y=0.764856in,left,base]{\rmfamily\fontsize{8.000000}{9.600000}\selectfont -0.5}%
\end{pgfscope}%
\begin{pgfscope}%
\pgfsetbuttcap%
\pgfsetroundjoin%
\definecolor{currentfill}{rgb}{0.000000,0.000000,0.000000}%
\pgfsetfillcolor{currentfill}%
\pgfsetlinewidth{0.803000pt}%
\definecolor{currentstroke}{rgb}{0.000000,0.000000,0.000000}%
\pgfsetstrokecolor{currentstroke}%
\pgfsetdash{}{0pt}%
\pgfsys@defobject{currentmarker}{\pgfqpoint{-0.048611in}{0.000000in}}{\pgfqpoint{0.000000in}{0.000000in}}{%
\pgfpathmoveto{\pgfqpoint{0.000000in}{0.000000in}}%
\pgfpathlineto{\pgfqpoint{-0.048611in}{0.000000in}}%
\pgfusepath{stroke,fill}%
}%
\begin{pgfscope}%
\pgfsys@transformshift{0.273112in}{1.363400in}%
\pgfsys@useobject{currentmarker}{}%
\end{pgfscope}%
\end{pgfscope}%
\begin{pgfscope}%
\pgftext[x=0.037001in,y=1.322982in,left,base]{\rmfamily\fontsize{8.000000}{9.600000}\selectfont 0.0}%
\end{pgfscope}%
\begin{pgfscope}%
\pgfsetbuttcap%
\pgfsetroundjoin%
\definecolor{currentfill}{rgb}{0.000000,0.000000,0.000000}%
\pgfsetfillcolor{currentfill}%
\pgfsetlinewidth{0.803000pt}%
\definecolor{currentstroke}{rgb}{0.000000,0.000000,0.000000}%
\pgfsetstrokecolor{currentstroke}%
\pgfsetdash{}{0pt}%
\pgfsys@defobject{currentmarker}{\pgfqpoint{-0.048611in}{0.000000in}}{\pgfqpoint{0.000000in}{0.000000in}}{%
\pgfpathmoveto{\pgfqpoint{0.000000in}{0.000000in}}%
\pgfpathlineto{\pgfqpoint{-0.048611in}{0.000000in}}%
\pgfusepath{stroke,fill}%
}%
\begin{pgfscope}%
\pgfsys@transformshift{0.273112in}{1.921526in}%
\pgfsys@useobject{currentmarker}{}%
\end{pgfscope}%
\end{pgfscope}%
\begin{pgfscope}%
\pgftext[x=0.037001in,y=1.881107in,left,base]{\rmfamily\fontsize{8.000000}{9.600000}\selectfont 0.5}%
\end{pgfscope}%
\begin{pgfscope}%
\pgfsetbuttcap%
\pgfsetroundjoin%
\definecolor{currentfill}{rgb}{0.000000,0.000000,0.000000}%
\pgfsetfillcolor{currentfill}%
\pgfsetlinewidth{0.803000pt}%
\definecolor{currentstroke}{rgb}{0.000000,0.000000,0.000000}%
\pgfsetstrokecolor{currentstroke}%
\pgfsetdash{}{0pt}%
\pgfsys@defobject{currentmarker}{\pgfqpoint{-0.048611in}{0.000000in}}{\pgfqpoint{0.000000in}{0.000000in}}{%
\pgfpathmoveto{\pgfqpoint{0.000000in}{0.000000in}}%
\pgfpathlineto{\pgfqpoint{-0.048611in}{0.000000in}}%
\pgfusepath{stroke,fill}%
}%
\begin{pgfscope}%
\pgfsys@transformshift{0.273112in}{2.479651in}%
\pgfsys@useobject{currentmarker}{}%
\end{pgfscope}%
\end{pgfscope}%
\begin{pgfscope}%
\pgftext[x=0.037001in,y=2.439233in,left,base]{\rmfamily\fontsize{8.000000}{9.600000}\selectfont 1.0}%
\end{pgfscope}%
\begin{pgfscope}%
\pgfpathrectangle{\pgfqpoint{0.273112in}{0.017500in}}{\pgfqpoint{4.650000in}{2.695000in}} %
\pgfusepath{clip}%
\pgfsetrectcap%
\pgfsetroundjoin%
\pgfsetlinewidth{1.505625pt}%
\definecolor{currentstroke}{rgb}{0.121569,0.466667,0.705882}%
\pgfsetstrokecolor{currentstroke}%
\pgfsetdash{}{0pt}%
\pgfpathmoveto{\pgfqpoint{0.484476in}{2.496926in}}%
\pgfpathlineto{\pgfqpoint{0.587580in}{2.494929in}}%
\pgfpathlineto{\pgfqpoint{0.690684in}{2.522853in}}%
\pgfpathlineto{\pgfqpoint{0.793788in}{2.465674in}}%
\pgfpathlineto{\pgfqpoint{0.896892in}{2.527517in}}%
\pgfpathlineto{\pgfqpoint{0.999997in}{2.519146in}}%
\pgfpathlineto{\pgfqpoint{1.103101in}{2.463161in}}%
\pgfpathlineto{\pgfqpoint{1.206205in}{0.232021in}}%
\pgfpathlineto{\pgfqpoint{1.309309in}{0.334294in}}%
\pgfpathlineto{\pgfqpoint{1.412413in}{0.259927in}}%
\pgfpathlineto{\pgfqpoint{1.515518in}{0.305857in}}%
\pgfpathlineto{\pgfqpoint{1.618622in}{0.181358in}}%
\pgfpathlineto{\pgfqpoint{1.721726in}{0.419503in}}%
\pgfpathlineto{\pgfqpoint{1.824830in}{2.538113in}}%
\pgfpathlineto{\pgfqpoint{1.927935in}{2.472396in}}%
\pgfpathlineto{\pgfqpoint{2.031039in}{2.489780in}}%
\pgfpathlineto{\pgfqpoint{2.134143in}{2.381770in}}%
\pgfpathlineto{\pgfqpoint{2.237247in}{2.517509in}}%
\pgfpathlineto{\pgfqpoint{2.340351in}{2.423620in}}%
\pgfpathlineto{\pgfqpoint{2.443456in}{0.140000in}}%
\pgfpathlineto{\pgfqpoint{2.546560in}{0.238764in}}%
\pgfpathlineto{\pgfqpoint{2.649664in}{0.248770in}}%
\pgfpathlineto{\pgfqpoint{2.752768in}{0.316290in}}%
\pgfpathlineto{\pgfqpoint{2.855872in}{0.148266in}}%
\pgfpathlineto{\pgfqpoint{2.958977in}{0.155742in}}%
\pgfpathlineto{\pgfqpoint{3.062081in}{0.275348in}}%
\pgfpathlineto{\pgfqpoint{3.165185in}{2.580516in}}%
\pgfpathlineto{\pgfqpoint{3.268289in}{2.417650in}}%
\pgfpathlineto{\pgfqpoint{3.371394in}{2.419745in}}%
\pgfpathlineto{\pgfqpoint{3.474498in}{2.590000in}}%
\pgfpathlineto{\pgfqpoint{3.577602in}{2.386289in}}%
\pgfpathlineto{\pgfqpoint{3.680706in}{2.451887in}}%
\pgfpathlineto{\pgfqpoint{3.783810in}{0.141115in}}%
\pgfpathlineto{\pgfqpoint{3.886915in}{0.286349in}}%
\pgfpathlineto{\pgfqpoint{3.990019in}{0.300951in}}%
\pgfpathlineto{\pgfqpoint{4.093123in}{0.299599in}}%
\pgfpathlineto{\pgfqpoint{4.196227in}{0.294299in}}%
\pgfpathlineto{\pgfqpoint{4.299331in}{0.209721in}}%
\pgfpathlineto{\pgfqpoint{4.402436in}{2.433698in}}%
\pgfpathlineto{\pgfqpoint{4.505540in}{2.401005in}}%
\pgfpathlineto{\pgfqpoint{4.608644in}{2.475492in}}%
\pgfpathlineto{\pgfqpoint{4.711748in}{2.505983in}}%
\pgfusepath{stroke}%
\end{pgfscope}%
\begin{pgfscope}%
\pgfpathrectangle{\pgfqpoint{0.273112in}{0.017500in}}{\pgfqpoint{4.650000in}{2.695000in}} %
\pgfusepath{clip}%
\pgfsetrectcap%
\pgfsetroundjoin%
\pgfsetlinewidth{1.505625pt}%
\definecolor{currentstroke}{rgb}{1.000000,0.498039,0.054902}%
\pgfsetstrokecolor{currentstroke}%
\pgfsetdash{}{0pt}%
\pgfpathmoveto{\pgfqpoint{0.484476in}{1.363400in}}%
\pgfpathlineto{\pgfqpoint{0.587580in}{1.363400in}}%
\pgfpathlineto{\pgfqpoint{0.690684in}{1.363400in}}%
\pgfpathlineto{\pgfqpoint{0.793788in}{1.363400in}}%
\pgfpathlineto{\pgfqpoint{0.896892in}{1.363400in}}%
\pgfpathlineto{\pgfqpoint{0.999997in}{1.363400in}}%
\pgfpathlineto{\pgfqpoint{1.103101in}{1.363400in}}%
\pgfpathlineto{\pgfqpoint{1.206205in}{1.363400in}}%
\pgfpathlineto{\pgfqpoint{1.309309in}{1.363400in}}%
\pgfpathlineto{\pgfqpoint{1.412413in}{1.363400in}}%
\pgfpathlineto{\pgfqpoint{1.515518in}{1.363400in}}%
\pgfpathlineto{\pgfqpoint{1.618622in}{1.363400in}}%
\pgfpathlineto{\pgfqpoint{1.721726in}{1.363400in}}%
\pgfpathlineto{\pgfqpoint{1.824830in}{1.363400in}}%
\pgfpathlineto{\pgfqpoint{1.927935in}{1.363400in}}%
\pgfpathlineto{\pgfqpoint{2.031039in}{1.363400in}}%
\pgfpathlineto{\pgfqpoint{2.134143in}{1.363400in}}%
\pgfpathlineto{\pgfqpoint{2.237247in}{1.363400in}}%
\pgfpathlineto{\pgfqpoint{2.340351in}{1.549442in}}%
\pgfpathlineto{\pgfqpoint{2.443456in}{1.549442in}}%
\pgfpathlineto{\pgfqpoint{2.546560in}{1.549442in}}%
\pgfpathlineto{\pgfqpoint{2.649664in}{1.549442in}}%
\pgfpathlineto{\pgfqpoint{2.752768in}{1.549442in}}%
\pgfpathlineto{\pgfqpoint{2.855872in}{1.549442in}}%
\pgfpathlineto{\pgfqpoint{2.958977in}{1.363400in}}%
\pgfpathlineto{\pgfqpoint{3.062081in}{1.363400in}}%
\pgfpathlineto{\pgfqpoint{3.165185in}{1.363400in}}%
\pgfpathlineto{\pgfqpoint{3.268289in}{1.363400in}}%
\pgfpathlineto{\pgfqpoint{3.371394in}{1.363400in}}%
\pgfpathlineto{\pgfqpoint{3.474498in}{1.363400in}}%
\pgfpathlineto{\pgfqpoint{3.577602in}{1.363400in}}%
\pgfpathlineto{\pgfqpoint{3.680706in}{1.363400in}}%
\pgfpathlineto{\pgfqpoint{3.783810in}{1.363400in}}%
\pgfpathlineto{\pgfqpoint{3.886915in}{1.363400in}}%
\pgfpathlineto{\pgfqpoint{3.990019in}{1.363400in}}%
\pgfpathlineto{\pgfqpoint{4.093123in}{1.363400in}}%
\pgfpathlineto{\pgfqpoint{4.196227in}{1.363400in}}%
\pgfpathlineto{\pgfqpoint{4.299331in}{1.363400in}}%
\pgfpathlineto{\pgfqpoint{4.402436in}{1.363400in}}%
\pgfpathlineto{\pgfqpoint{4.505540in}{1.363400in}}%
\pgfpathlineto{\pgfqpoint{4.608644in}{1.363400in}}%
\pgfpathlineto{\pgfqpoint{4.711748in}{1.363400in}}%
\pgfusepath{stroke}%
\end{pgfscope}%
\begin{pgfscope}%
\pgfpathrectangle{\pgfqpoint{0.273112in}{0.017500in}}{\pgfqpoint{4.650000in}{2.695000in}} %
\pgfusepath{clip}%
\pgfsetrectcap%
\pgfsetroundjoin%
\pgfsetlinewidth{1.505625pt}%
\definecolor{currentstroke}{rgb}{0.172549,0.627451,0.172549}%
\pgfsetstrokecolor{currentstroke}%
\pgfsetdash{}{0pt}%
\pgfpathmoveto{\pgfqpoint{0.484476in}{1.934152in}}%
\pgfpathlineto{\pgfqpoint{0.587580in}{2.117864in}}%
\pgfpathlineto{\pgfqpoint{0.690684in}{2.311883in}}%
\pgfpathlineto{\pgfqpoint{0.793788in}{2.504508in}}%
\pgfpathlineto{\pgfqpoint{0.896892in}{2.498880in}}%
\pgfpathlineto{\pgfqpoint{0.999997in}{2.121729in}}%
\pgfpathlineto{\pgfqpoint{1.103101in}{1.756969in}}%
\pgfpathlineto{\pgfqpoint{1.206205in}{1.389345in}}%
\pgfpathlineto{\pgfqpoint{1.309309in}{1.019068in}}%
\pgfpathlineto{\pgfqpoint{1.412413in}{0.629436in}}%
\pgfpathlineto{\pgfqpoint{1.515518in}{0.288827in}}%
\pgfpathlineto{\pgfqpoint{1.618622in}{0.673175in}}%
\pgfpathlineto{\pgfqpoint{1.721726in}{1.029526in}}%
\pgfpathlineto{\pgfqpoint{1.824830in}{1.401168in}}%
\pgfpathlineto{\pgfqpoint{1.927935in}{1.747153in}}%
\pgfpathlineto{\pgfqpoint{2.031039in}{2.136512in}}%
\pgfpathlineto{\pgfqpoint{2.134143in}{2.470532in}}%
\pgfpathlineto{\pgfqpoint{2.237247in}{2.070846in}}%
\pgfpathlineto{\pgfqpoint{2.340351in}{1.698574in}}%
\pgfpathlineto{\pgfqpoint{2.443456in}{1.325072in}}%
\pgfpathlineto{\pgfqpoint{2.546560in}{0.980826in}}%
\pgfpathlineto{\pgfqpoint{2.649664in}{0.585952in}}%
\pgfpathlineto{\pgfqpoint{2.752768in}{0.207972in}}%
\pgfpathlineto{\pgfqpoint{2.855872in}{0.230530in}}%
\pgfpathlineto{\pgfqpoint{2.958977in}{0.620822in}}%
\pgfpathlineto{\pgfqpoint{3.062081in}{0.982302in}}%
\pgfpathlineto{\pgfqpoint{3.165185in}{1.332878in}}%
\pgfpathlineto{\pgfqpoint{3.268289in}{1.739834in}}%
\pgfpathlineto{\pgfqpoint{3.371394in}{2.111591in}}%
\pgfpathlineto{\pgfqpoint{3.474498in}{2.474348in}}%
\pgfpathlineto{\pgfqpoint{3.577602in}{2.067781in}}%
\pgfpathlineto{\pgfqpoint{3.680706in}{1.712564in}}%
\pgfpathlineto{\pgfqpoint{3.783810in}{1.359432in}}%
\pgfpathlineto{\pgfqpoint{3.886915in}{0.977699in}}%
\pgfpathlineto{\pgfqpoint{3.990019in}{0.629033in}}%
\pgfpathlineto{\pgfqpoint{4.093123in}{0.255339in}}%
\pgfpathlineto{\pgfqpoint{4.196227in}{0.637436in}}%
\pgfpathlineto{\pgfqpoint{4.299331in}{0.989879in}}%
\pgfpathlineto{\pgfqpoint{4.402436in}{1.352302in}}%
\pgfpathlineto{\pgfqpoint{4.505540in}{1.720033in}}%
\pgfpathlineto{\pgfqpoint{4.608644in}{1.898217in}}%
\pgfpathlineto{\pgfqpoint{4.711748in}{2.090497in}}%
\pgfusepath{stroke}%
\end{pgfscope}%
\begin{pgfscope}%
\pgfsetrectcap%
\pgfsetmiterjoin%
\pgfsetlinewidth{0.501875pt}%
\definecolor{currentstroke}{rgb}{0.000000,0.000000,0.000000}%
\pgfsetstrokecolor{currentstroke}%
\pgfsetdash{}{0pt}%
\pgfpathmoveto{\pgfqpoint{0.273112in}{0.017500in}}%
\pgfpathlineto{\pgfqpoint{0.273112in}{2.712500in}}%
\pgfusepath{stroke}%
\end{pgfscope}%
\begin{pgfscope}%
\pgfsetrectcap%
\pgfsetmiterjoin%
\pgfsetlinewidth{0.501875pt}%
\definecolor{currentstroke}{rgb}{0.000000,0.000000,0.000000}%
\pgfsetstrokecolor{currentstroke}%
\pgfsetdash{}{0pt}%
\pgfpathmoveto{\pgfqpoint{0.273112in}{1.363400in}}%
\pgfpathlineto{\pgfqpoint{4.923112in}{1.363400in}}%
\pgfusepath{stroke}%
\end{pgfscope}%
\begin{pgfscope}%
\pgfsetrectcap%
\pgfsetroundjoin%
\pgfsetlinewidth{1.505625pt}%
\definecolor{currentstroke}{rgb}{0.121569,0.466667,0.705882}%
\pgfsetstrokecolor{currentstroke}%
\pgfsetdash{}{0pt}%
\pgfpathmoveto{\pgfqpoint{0.385612in}{0.583906in}}%
\pgfpathlineto{\pgfqpoint{0.635612in}{0.583906in}}%
\pgfusepath{stroke}%
\end{pgfscope}%
\begin{pgfscope}%
\pgftext[x=0.735612in,y=0.540156in,left,base]{\rmfamily\fontsize{9.000000}{10.800000}\selectfont \(\displaystyle f(x)\)}%
\end{pgfscope}%
\begin{pgfscope}%
\pgfsetrectcap%
\pgfsetroundjoin%
\pgfsetlinewidth{1.505625pt}%
\definecolor{currentstroke}{rgb}{1.000000,0.498039,0.054902}%
\pgfsetstrokecolor{currentstroke}%
\pgfsetdash{}{0pt}%
\pgfpathmoveto{\pgfqpoint{0.385612in}{0.396040in}}%
\pgfpathlineto{\pgfqpoint{0.635612in}{0.396040in}}%
\pgfusepath{stroke}%
\end{pgfscope}%
\begin{pgfscope}%
\pgftext[x=0.735612in,y=0.352290in,left,base]{\rmfamily\fontsize{9.000000}{10.800000}\selectfont \(\displaystyle k(x)\)}%
\end{pgfscope}%
\begin{pgfscope}%
\pgfsetrectcap%
\pgfsetroundjoin%
\pgfsetlinewidth{1.505625pt}%
\definecolor{currentstroke}{rgb}{0.172549,0.627451,0.172549}%
\pgfsetstrokecolor{currentstroke}%
\pgfsetdash{}{0pt}%
\pgfpathmoveto{\pgfqpoint{0.385612in}{0.208174in}}%
\pgfpathlineto{\pgfqpoint{0.635612in}{0.208174in}}%
\pgfusepath{stroke}%
\end{pgfscope}%
\begin{pgfscope}%
\pgftext[x=0.735612in,y=0.164424in,left,base]{\rmfamily\fontsize{9.000000}{10.800000}\selectfont \(\displaystyle (f * k)(x)\)}%
\end{pgfscope}%
\end{pgfpicture}%
\makeatother%
\endgroup%

	\caption{Visualisation of convolutional kernel as feature detector. When the signal $f$ is similar to the kernel, the output of the convolution is maximally positive.}
	\label{feature_detector}
\end{figure}

The kernel $k$ can also act as a \textbf{feature detector}. When the output of $f$ is closely correlated with the output of $k$, the output of the convolution spikes. See for example figure \ref{feature_detector}.
\\\\
\textbf{Convolutional neural networks} are neural networks that take advantage of convolutions as feature detectors \citep{lecun1989}. By arranging the layers and weights in the network in specific ways, we can construct a network such that the output of each layer $l$ is the output of layer $l - 1$ convolved with a kernel $k$, where the weights of $k$ are exactly the neural network weights connecting the units in layer $l$ and $l - 1$.

Specifically, the weights connecting layers $l$ and $l - 1$ in a convolutional neural network should be arranged such that they are:

\begin{labeling}{shared}
	\item [\textbf{sparse}] each unit in layer $l$ receives input from a small number of units layer $l - 1$.
	\item [\textbf{shared}] the weights connecting units in layer $l$ and $l - 1$ are shared across the layer, in the same way that the same kernel weights are re-used around every index of $f$. See figure \ref{convolutional_network}.
\end{labeling}

These restrictions on the network architecture reduces the number of unique weights of the model. This improves the statistical efficiency of these types of models, i.e reduces the complexity of the induced hypothesis space \citep{goodfellow16}. Moreover, the reduction in the effective number of parameters has the effect of reducing both the memory requirements of storing the network, but also limits the number of operations required to compute the output of the network for a given input.
\\\\
Intuitively, the output at each unit $u$ in $l$ in a convolutional layer indicates how strongly the feature detected by the kernel given by its connecting weights is present in the output of units that $u$ connects to in layer $l - 1$. Since the weights are learned by gradient descent, the feature detected by units in layer $l$ is learnt as well.

Often, the simple presence or absence of a feature in the output of layer $l - 1$ is very informative for the classification task the convolutional network was built to solve. The exact position of a detected feature in layer $l - 1$ is often less informative however. For this reason, convolutional layers are often interleaved with so called \textbf{pooling layers}. The output of a pooling layer can be thought of as a summary how strongly a feature is detected in layer $l$, that discards information about the exact position at which the features was detected. Very commonly, max-pooling is used which simply outputs the maximum value over all outputs of units in layer $l$.

\begin{figure}[h]
	\centering
	\begin{tikzpicture}[->, >=stealth, swap]
			\node [neuron] (x1) at (0,0) {$x_1$};
			\node [neuron] (x2) at (1,0) {$x_2$};
			\node [neuron] (x3) at (2,0) {$x_3$};
			\node [neuron] (x4) at (3,0) {$x_4$};
			\node [neuron] (x5) at (4,0) {$x_5$};
			\node [neuron] (x6) at (5,0) {$x_6$};
			\node [neuron] (sigma1) at (1,2) {$\sigma$};
			\node [neuron] (sigma2) at (2,2) {$\sigma$};
			\node [neuron] (sigma3) at (3,2) {$\sigma$};
			\node [neuron] (sigma4) at (4,2) {$\sigma$};
			
			\draw (x1) edge[red] (sigma1);
			\draw (x2) edge[black] (sigma1);
			\draw (x3) edge[blue] (sigma1);
			
			\draw (x2) edge[red] (sigma2);
			\draw (x3) edge[black] (sigma2);
			\draw (x4) edge[blue] (sigma2);
			
			\draw (x3) edge[red] (sigma3);
			\draw (x4) edge[black] (sigma3);
			\draw (x5) edge[blue] (sigma3);
			
			\draw (x4) edge[red] (sigma4);
			\draw (x5) edge[black] (sigma4);
			\draw (x6) edge[blue] (sigma4);
	\end{tikzpicture}
	\caption{Visual representation of a one-dimensional convolution implemented as the first layer of a convolutional neural network. The connections between the input layer and the convolutional layer are sparse in that each unit is connected only to three of six inputs. The colors of the connections indicate how the weights are shared.}
	\label{convolutional_network}
\end{figure}



\section{Word Vectors}
The way text is represented in a computer doesn't in general encode any information about semantic similarities between words or sentences. Instead, text is most often represented as sequences of discrete symbols. Learning a $h$ that maps from discrete input space where distances between points don't encode similarity, such as words, to a prediction, such as the presence of a named entity, may be more difficult than learning a mapping from continuous input space to a prediction, since a continuous function can be expected to have some smoothness properties, i.e similar inputs should have similar outputs \citep{bengio2003}.
\\\\
For this reason some effort has been devoted to designing real-valued vector representation of words, so called \textbf{word vectors}, that encode semantic similarities such that words with similar meaning are close to each other in word-vector space. The notion of "meaning" of a word is a philosophically challenging one. A simple definition which leads to simple but useful algorithms is that words have similar meaning if they are used in similar contexts.

This leads to the idea of representing words as vectors of co-occurrence counts. Two words $w_i$ and $w_j$ co-occur in a context of $c$ words if $w_j$ appears somewhere in a window of $c$ words from $w_i$ in some piece of text. By representing $w_i$ as a vector $\vector{w}_i \in \mathbb{R}^V$ of co-occurrence counts for the $V$ words in some vocabulary, words that occur in similar contexts will be close to each other in co-occurrence vector space.
\\\\
The main problems with this representation is that $V$ may be very large, and $\vector{w}_i$ may be very sparse, that is, most of its components are 0 since most words never co-occur together. Recent solutions to this problem learn lower dimensional word vectors using co-occurrence statistics. \textbf{GloVe} is a recent and successful technique for learning word vectors that encode much useful syntactic and semantic information \citep{pennington2014}. In GloVe, each word $w_i$ is represented by a word vectors $\vector{w}_i$, and a context word vector $\tilde{\vector{w}}_i$. The vectors are initialized randomly

 Glove vectors are learned by minimizing the objective function:
$$
\sum\limits_{i=1}^V\sum\limits_{i=1}^V f(\matrix{X}_{ij})(\vector{w}_i^T\tilde{\vector{w}}_j + b_i + \tilde{b}_j - \ln \matrix{X}_{ij})^2
$$
Where $\matrix{X}_{ij}$ is the co-occurrence count for word $w_i$ and $w_j$, and $b_i$ and $\tilde{b}_j$ are bias terms. $f$ is a weighting function that gives low weight to infrequent terms and caps extremely frequent terms, defined as:
$$
f(x) = \begin{cases}
	(x / x_{max})^{3/4} & \text{when $x < x_{max}$} \\
	1 & \text{otherwise}
\end{cases}
$$
Minimizing this objective leads to word vectors whose dot products are close to log-co-occurrence counts for the words they represent. It can be shown that this has the effect that word vector differences encode information about ratios of log-co-occurrence probabilities which are highly informative of semantic similarity.
\\\\
It is now common practice to incorporate word vectors in neural network models for natural language processing tasks in a so called embedding layer. In this scheme, the components of the word vectors are parameters that can be trained by back propagation to yield word vector representations that are informative for a given task. These word embedding vectors, or simply word embeddings, can be initialized with small random components as any other neural network parameter, or they can be initialized with pre-learned word vectors, for example GloVe vectors.

\section{Summary}
In this section we have seen how to define $\mathcal{H}$ with neural networks, and we have seen how to search this space using backpropagation and gradient descent. We have presented the Adam algorithm as a useful extension to stochastic gradient descent that incorporates ideas of learning rate scaling and momentum.
\\\\
 Moreover, we have discussed regularization techniques that restrict the learning algorithm to a limited region of $\mathcal{H}$ in order to reduce the risk of overfitting. In particular, we have presented early stopping as a simple yet effective regularization technique. Finally, we have introduced convolutional neural networks that take advantage of convolutions as feature detectors. As we will discuss in part \ref{experiment}, convolutional neural networks are often used to solve sentence classification problems such as relation classification.
 \\\\
 In the next part, we introduce the multi-task learning framework as an extension to Vapnik-Chervonenkis analysis presented in section \ref{statistical_learning_theory}.